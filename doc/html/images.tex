\batchmode


\documentclass[12pt]{book}
\RequirePackage{ifthen}


\usepackage{color}
\usepackage{html}
\usepackage{graphicx}
\usepackage{textcomp}


\usepackage[letterpaper,margin=1.25in]{geometry}
\usepackage{bibnames}
\usepackage{longtable}
\usepackage{multirow}


\definecolor{Blue}{rgb}{0,0,1}
\definecolor{BLUE}{rgb}{0,0,1}
\definecolor{green}{rgb}{0,0.6,0}
\definecolor{Green}{rgb}{0,0.6,0}
\definecolor{GREEN}{rgb}{0,0.6,0}



%
\newenvironment{codeblock}%
	{\begin{quotation}\begin{minipage}[t]{0.9\textwidth}}%
	{\end{minipage}\end{quotation}} 
	%{\begin{flushleft}}%%
\providecommand{\codeblockfont}[1]{\textcolor{blue}{\texttt{#1}}} 

%
\providecommand{\emphpara}[1]{\textbf{#1}}%
\providecommand{\booktitle}[1]{\textit{#1}}%
\providecommand{\paratitle}[1]{\textit{#1}}%
\providecommand{\emphdate}[1]{\textbf{#1}} 

%
\providecommand{\code}[1]{\textcolor{blue}{\texttt{#1}}}%
\providecommand{\cmd}[1]{\textcolor{blue}{\texttt{#1}}}%
\providecommand{\dumarg}[1]{\textit{#1}}%
\providecommand{\codearg}[1]{\textit{#1}}%
\providecommand{\fn}[1]{\textsf{\textit{#1}}}%
\providecommand{\screen}[1]{\textcolor{green}{\texttt{#1}}}%
\providecommand{\vars}[1]{\textcolor{blue}{\texttt{#1}}}%
\providecommand{\class}[1]{\textcolor{blue}{\texttt{#1}}}%
\providecommand{\mods}[1]{\textcolor{blue}{\texttt{#1}}} 



\newlength{\tabletitlewidth} 

\settowidth{\tabletitlewidth}{file and directory names} 



\newlength{\tablebodywidth} 

\setlength{\tablebodywidth}{0.9\textwidth} 

\addtolength{\tablebodywidth}{-4ex} 

\addtolength{\tablebodywidth}{-\tabletitlewidth} 

%
\providecommand{\tabletabulardims}%
	{p{\tabletitlewidth}@{\hspace{4ex}}p{\tablebodywidth}} 

%
\providecommand{\tableitemlinespace}{\baselineskip} 

\newlength{\infotitle} 

\newlength{\infotext} 



\newlength{\remainder}         % length to describe the residual of the
                              
\newlength{\enumlabel}         % length to describe figure sub-label width
                              
%
\providecommand{\htmlfigcaption}[1]{\parbox[c]{70ex}{\footnotesize{#1}}} 

%
\providecommand{\rute}{\textit{Rute User's}}%
\providecommand{\linuxnut}{\textit{Linux in a Nutshell}}%
\providecommand{\pynut}{\textit{Python in a Nutshell}} 

%
\providecommand{\aonehat}{\ensuremath{\widehat{a_1}}}%
\providecommand{\bonehat}{\ensuremath{\widehat{b_1}}}%
\providecommand{\D}{\ensuremath{\mathcal{D}}} 






\pagecolor[gray]{.7}

\usepackage[latin1]{inputenc}



\makeatletter
\AtBeginDocument{\makeatletter
\input /Users/jlin/work/prog/python/qtcm/doc/latex/manual.aux
\makeatother
}

\makeatletter
\count@=\the\catcode`\_ \catcode`\_=8 
\newenvironment{tex2html_wrap}{}{}%
\catcode`\<=12\catcode`\_=\count@
\newcommand{\providedcommand}[1]{\expandafter\providecommand\csname #1\endcsname}%
\newcommand{\renewedcommand}[1]{\expandafter\providecommand\csname #1\endcsname{}%
  \expandafter\renewcommand\csname #1\endcsname}%
\newcommand{\newedenvironment}[1]{\newenvironment{#1}{}{}\renewenvironment{#1}}%
\let\newedcommand\renewedcommand
\let\renewedenvironment\newedenvironment
\makeatother
\let\mathon=$
\let\mathoff=$
\ifx\AtBeginDocument\undefined \newcommand{\AtBeginDocument}[1]{}\fi
\newbox\sizebox
\setlength{\hoffset}{0pt}\setlength{\voffset}{0pt}
\addtolength{\textheight}{\footskip}\setlength{\footskip}{0pt}
\addtolength{\textheight}{\topmargin}\setlength{\topmargin}{0pt}
\addtolength{\textheight}{\headheight}\setlength{\headheight}{0pt}
\addtolength{\textheight}{\headsep}\setlength{\headsep}{0pt}
\setlength{\textwidth}{349pt}
\newwrite\lthtmlwrite
\makeatletter
\let\realnormalsize=\normalsize
\global\topskip=2sp
\def\preveqno{}\let\real@float=\@float \let\realend@float=\end@float
\def\@float{\let\@savefreelist\@freelist\real@float}
\def\liih@math{\ifmmode$\else\bad@math\fi}
\def\end@float{\realend@float\global\let\@freelist\@savefreelist}
\let\real@dbflt=\@dbflt \let\end@dblfloat=\end@float
\let\@largefloatcheck=\relax
\let\if@boxedmulticols=\iftrue
\def\@dbflt{\let\@savefreelist\@freelist\real@dbflt}
\def\adjustnormalsize{\def\normalsize{\mathsurround=0pt \realnormalsize
 \parindent=0pt\abovedisplayskip=0pt\belowdisplayskip=0pt}%
 \def\phantompar{\csname par\endcsname}\normalsize}%
\def\lthtmltypeout#1{{\let\protect\string \immediate\write\lthtmlwrite{#1}}}%
\newcommand\lthtmlhboxmathA{\adjustnormalsize\setbox\sizebox=\hbox\bgroup\kern.05em }%
\newcommand\lthtmlhboxmathB{\adjustnormalsize\setbox\sizebox=\hbox to\hsize\bgroup\hfill }%
\newcommand\lthtmlvboxmathA{\adjustnormalsize\setbox\sizebox=\vbox\bgroup %
 \let\ifinner=\iffalse \let\)\liih@math }%
\newcommand\lthtmlboxmathZ{\@next\next\@currlist{}{\def\next{\voidb@x}}%
 \expandafter\box\next\egroup}%
\newcommand\lthtmlmathtype[1]{\gdef\lthtmlmathenv{#1}}%
\newcommand\lthtmllogmath{\dimen0\ht\sizebox \advance\dimen0\dp\sizebox
  \ifdim\dimen0>.95\vsize
   \lthtmltypeout{%
*** image for \lthtmlmathenv\space is too tall at \the\dimen0, reducing to .95 vsize ***}%
   \ht\sizebox.95\vsize \dp\sizebox\z@ \fi
  \lthtmltypeout{l2hSize %
:\lthtmlmathenv:\the\ht\sizebox::\the\dp\sizebox::\the\wd\sizebox.\preveqno}}%
\newcommand\lthtmlfigureA[1]{\let\@savefreelist\@freelist
       \lthtmlmathtype{#1}\lthtmlvboxmathA}%
\newcommand\lthtmlpictureA{\bgroup\catcode`\_=8 \lthtmlpictureB}%
\newcommand\lthtmlpictureB[1]{\lthtmlmathtype{#1}\egroup
       \let\@savefreelist\@freelist \lthtmlhboxmathB}%
\newcommand\lthtmlpictureZ[1]{\hfill\lthtmlfigureZ}%
\newcommand\lthtmlfigureZ{\lthtmlboxmathZ\lthtmllogmath\copy\sizebox
       \global\let\@freelist\@savefreelist}%
\newcommand\lthtmldisplayA{\bgroup\catcode`\_=8 \lthtmldisplayAi}%
\newcommand\lthtmldisplayAi[1]{\lthtmlmathtype{#1}\egroup\lthtmlvboxmathA}%
\newcommand\lthtmldisplayB[1]{\edef\preveqno{(\theequation)}%
  \lthtmldisplayA{#1}\let\@eqnnum\relax}%
\newcommand\lthtmldisplayZ{\lthtmlboxmathZ\lthtmllogmath\lthtmlsetmath}%
\newcommand\lthtmlinlinemathA{\bgroup\catcode`\_=8 \lthtmlinlinemathB}
\newcommand\lthtmlinlinemathB[1]{\lthtmlmathtype{#1}\egroup\lthtmlhboxmathA
  \vrule height1.5ex width0pt }%
\newcommand\lthtmlinlineA{\bgroup\catcode`\_=8 \lthtmlinlineB}%
\newcommand\lthtmlinlineB[1]{\lthtmlmathtype{#1}\egroup\lthtmlhboxmathA}%
\newcommand\lthtmlinlineZ{\egroup\expandafter\ifdim\dp\sizebox>0pt %
  \expandafter\centerinlinemath\fi\lthtmllogmath\lthtmlsetinline}
\newcommand\lthtmlinlinemathZ{\egroup\expandafter\ifdim\dp\sizebox>0pt %
  \expandafter\centerinlinemath\fi\lthtmllogmath\lthtmlsetmath}
\newcommand\lthtmlindisplaymathZ{\egroup %
  \centerinlinemath\lthtmllogmath\lthtmlsetmath}
\def\lthtmlsetinline{\hbox{\vrule width.1em \vtop{\vbox{%
  \kern.1em\copy\sizebox}\ifdim\dp\sizebox>0pt\kern.1em\else\kern.3pt\fi
  \ifdim\hsize>\wd\sizebox \hrule depth1pt\fi}}}
\def\lthtmlsetmath{\hbox{\vrule width.1em\kern-.05em\vtop{\vbox{%
  \kern.1em\kern0.8 pt\hbox{\hglue.17em\copy\sizebox\hglue0.8 pt}}\kern.3pt%
  \ifdim\dp\sizebox>0pt\kern.1em\fi \kern0.8 pt%
  \ifdim\hsize>\wd\sizebox \hrule depth1pt\fi}}}
\def\centerinlinemath{%
  \dimen1=\ifdim\ht\sizebox<\dp\sizebox \dp\sizebox\else\ht\sizebox\fi
  \advance\dimen1by.5pt \vrule width0pt height\dimen1 depth\dimen1 
 \dp\sizebox=\dimen1\ht\sizebox=\dimen1\relax}

\def\lthtmlcheckvsize{\ifdim\ht\sizebox<\vsize 
  \ifdim\wd\sizebox<\hsize\expandafter\hfill\fi \expandafter\vfill
  \else\expandafter\vss\fi}%
\providecommand{\selectlanguage}[1]{}%
\makeatletter \tracingstats = 1 


\begin{document}
\pagestyle{empty}\thispagestyle{empty}\lthtmltypeout{}%
\lthtmltypeout{latex2htmlLength hsize=\the\hsize}\lthtmltypeout{}%
\lthtmltypeout{latex2htmlLength vsize=\the\vsize}\lthtmltypeout{}%
\lthtmltypeout{latex2htmlLength hoffset=\the\hoffset}\lthtmltypeout{}%
\lthtmltypeout{latex2htmlLength voffset=\the\voffset}\lthtmltypeout{}%
\lthtmltypeout{latex2htmlLength topmargin=\the\topmargin}\lthtmltypeout{}%
\lthtmltypeout{latex2htmlLength topskip=\the\topskip}\lthtmltypeout{}%
\lthtmltypeout{latex2htmlLength headheight=\the\headheight}\lthtmltypeout{}%
\lthtmltypeout{latex2htmlLength headsep=\the\headsep}\lthtmltypeout{}%
\lthtmltypeout{latex2htmlLength parskip=\the\parskip}\lthtmltypeout{}%
\lthtmltypeout{latex2htmlLength oddsidemargin=\the\oddsidemargin}\lthtmltypeout{}%
\makeatletter
\if@twoside\lthtmltypeout{latex2htmlLength evensidemargin=\the\evensidemargin}%
\else\lthtmltypeout{latex2htmlLength evensidemargin=\the\oddsidemargin}\fi%
\lthtmltypeout{}%
\makeatother
\setcounter{page}{1}
\onecolumn

% !!! IMAGES START HERE !!!

{\newpage\clearpage
\lthtmlinlinemathA{tex2html_wrap_inline16007}%
$a_1$%
\lthtmlinlinemathZ
\lthtmlcheckvsize\clearpage}

{\newpage\clearpage
\lthtmlinlinemathA{tex2html_wrap_inline16009}%
$b_1$%
\lthtmlinlinemathZ
\lthtmlcheckvsize\clearpage}

{\newpage\clearpage
\lthtmlinlinemathA{tex2html_wrap_inline16011}%
$V_1$%
\lthtmlinlinemathZ
\lthtmlcheckvsize\clearpage}

\stepcounter{chapter}
\stepcounter{section}
\stepcounter{section}
{\newpage\clearpage
\lthtmlinlinemathA{tex2html_wrap_inline435}%
$\sim$%
\lthtmlinlinemathZ
\lthtmlcheckvsize\clearpage}

\stepcounter{section}
\stepcounter{subsection}
\stepcounter{subsection}
\stepcounter{subsection}
\stepcounter{section}
{\newpage\clearpage
\lthtmlinlineA{tex2html_nomath_inline916}%
\textregistered%
\lthtmlinlineZ
\lthtmlcheckvsize\clearpage}

\stepcounter{section}
\stepcounter{section}
{\newpage\clearpage
\lthtmlinlinemathA{tex2html_wrap_inline824}%
$V_0$%
\lthtmlinlinemathZ
\lthtmlcheckvsize\clearpage}

{\newpage\clearpage
\lthtmlinlinemathA{tex2html_wrap_inline832}%
$5.625^{\circ}$%
\lthtmlinlinemathZ
\lthtmlcheckvsize\clearpage}

{\newpage\clearpage
\lthtmlinlinemathA{tex2html_wrap_inline834}%
$3.75^{\circ}$%
\lthtmlinlinemathZ
\lthtmlcheckvsize\clearpage}

{\newpage\clearpage
\lthtmlinlinemathA{tex2html_wrap_inline16170}%

% latex2html id marker 16170
$\textstyle \parbox{70ex}{\footnotesize{Figure \ref{fig:qtcm.basis}:  
	Vertical profiles of basis functions for
   	(a) temperature $a_1$\  (solid) and humidity $b_1$\  (dashed) and
   	(b) baroclinic component of
   	horizontal velocity $V_1$.}}$%
\lthtmlinlinemathZ
\lthtmlcheckvsize\clearpage}

{\newpage\clearpage
\lthtmldisplayA{displaymath622}%
\begin{displaymath}
   \partial_t \mathbf{v}_1 
      + \ensuremath{\mathcal{D}}_{V1} (\mathbf{v}_0 , \mathbf{v}_1)
      + f \mathbf{k} \times \mathbf{v}_1
      =
   - \kappa \nabla T_1 
      - \epsilon_1 \mathbf{v}_1 
      - \epsilon_{01} \mathbf{v}_0
   
\end{displaymath}%
\lthtmldisplayZ
\lthtmlcheckvsize\clearpage}

{\newpage\clearpage
\lthtmldisplayA{displaymath634}%
\begin{displaymath}
   \partial_t \zeta_0 
      + \mathrm{curl}_z (\ensuremath{\mathcal{D}}_{V0} (\mathbf{v}_0 , \mathbf{v}_1))
      + \beta v_0
      =
   - \mathrm{curl}_z (\epsilon_0 \mathbf{v}_0)
      - \mathrm{curl}_z (\epsilon_{10} \mathbf{v}_1)
   
\end{displaymath}%
\lthtmldisplayZ
\lthtmlcheckvsize\clearpage}

{\newpage\clearpage
\lthtmldisplayA{displaymath646}%
\begin{displaymath}
   \ensuremath{\widehat{a_1}}(\partial_t + \ensuremath{\mathcal{D}}_{T1}) T_1 
      + M_{S1} \nabla \cdot {\bf v}_1 
      =
   \langle Q_c \rangle
      + (g/p_T) (-R^\uparrow_t -R^\downarrow_s + R^\uparrow_s + S_t - S_s + H)
   
\end{displaymath}%
\lthtmldisplayZ
\lthtmlcheckvsize\clearpage}

{\newpage\clearpage
\lthtmldisplayA{displaymath652}%
\begin{displaymath}
   \ensuremath{\widehat{b_1}}(\partial_t + \ensuremath{\mathcal{D}}_{q1}) q_1 
      - M_{q1} \nabla \cdot {\bf v}_1 
      =
   \langle Q_q \rangle
      + (g/p_T) E
   
\end{displaymath}%
\lthtmldisplayZ
\lthtmlcheckvsize\clearpage}

{\newpage\clearpage
\lthtmldisplayA{displaymath662}%
\begin{displaymath}
  -\langle Q_q \rangle = \langle Q_c \rangle 
                              = \epsilon^\ast_c (q_1 - T_1)
\end{displaymath}%
\lthtmldisplayZ
\lthtmlcheckvsize\clearpage}

{\newpage\clearpage
\lthtmlinlinemathA{tex2html_wrap_inline842}%
$Q_c$%
\lthtmlinlinemathZ
\lthtmlcheckvsize\clearpage}

{\newpage\clearpage
\lthtmlinlinemathA{tex2html_wrap_inline844}%
$\epsilon^\ast_c$%
\lthtmlinlinemathZ
\lthtmlcheckvsize\clearpage}

{\newpage\clearpage
\lthtmldisplayA{displaymath665}%
\begin{displaymath}
   \epsilon^\ast_c 
      \equiv 
   \ensuremath{\widehat{a_1}}\ensuremath{\widehat{b_1}}(\ensuremath{\widehat{a_1}}+ \ensuremath{\widehat{b_1}})^{-1} \tau_c^{-1} 
      \mathcal{H}( \mathit{C}_{\mathrm{1}} )
\end{displaymath}%
\lthtmldisplayZ
\lthtmlcheckvsize\clearpage}

{\newpage\clearpage
\lthtmlinlinemathA{tex2html_wrap_inline846}%
$\mathcal{H}( \mathit{C}_{\mathrm{1}} )$%
\lthtmlinlinemathZ
\lthtmlcheckvsize\clearpage}

{\newpage\clearpage
\lthtmlinlinemathA{tex2html_wrap_inline848}%
$C_{1} \leq 0$%
\lthtmlinlinemathZ
\lthtmlcheckvsize\clearpage}

{\newpage\clearpage
\lthtmlinlinemathA{tex2html_wrap_inline850}%
$C_{1} > 0$%
\lthtmlinlinemathZ
\lthtmlcheckvsize\clearpage}

{\newpage\clearpage
\lthtmlinlinemathA{tex2html_wrap_inline852}%
$C_{1}$%
\lthtmlinlinemathZ
\lthtmlcheckvsize\clearpage}

{\newpage\clearpage
\lthtmlinlinemathA{tex2html_wrap_inline854}%
$H$%
\lthtmlinlinemathZ
\lthtmlcheckvsize\clearpage}

{\newpage\clearpage
\lthtmlinlinemathA{tex2html_wrap_inline856}%
$E$%
\lthtmlinlinemathZ
\lthtmlcheckvsize\clearpage}

{\newpage\clearpage
\lthtmldisplayA{displaymath678}%
\begin{displaymath}
   H
      =
   \rho_a C_D \mathrm{V}_s (T_s - (T_{rs} + a_{1s} T_1))
\end{displaymath}%
\lthtmldisplayZ
\lthtmlcheckvsize\clearpage}

{\newpage\clearpage
\lthtmldisplayA{displaymath683}%
\begin{displaymath}
   E
      =
   \rho_a C_D \mathrm{V}_s (q_\mathit{sat} (T_s) 
      - (q_{rs} + b_{1s} q_1))
\end{displaymath}%
\lthtmldisplayZ
\lthtmlcheckvsize\clearpage}

{\newpage\clearpage
\lthtmlinlinemathA{tex2html_wrap_inline858}%
$R$%
\lthtmlinlinemathZ
\lthtmlcheckvsize\clearpage}

{\newpage\clearpage
\lthtmlinlinemathA{tex2html_wrap_inline860}%
$S$%
\lthtmlinlinemathZ
\lthtmlcheckvsize\clearpage}

{\newpage\clearpage
\lthtmlinlinemathA{tex2html_wrap_inline862}%
$\ensuremath{\mathcal{D}}_{V1}$%
\lthtmlinlinemathZ
\lthtmlcheckvsize\clearpage}

{\newpage\clearpage
\lthtmlinlinemathA{tex2html_wrap_inline864}%
$\ensuremath{\mathcal{D}}_{V0}$%
\lthtmlinlinemathZ
\lthtmlcheckvsize\clearpage}

{\newpage\clearpage
\lthtmlinlinemathA{tex2html_wrap_inline868}%
$V_1 (p)$%
\lthtmlinlinemathZ
\lthtmlcheckvsize\clearpage}

{\newpage\clearpage
\lthtmlinlinemathA{tex2html_wrap_inline870}%
$\ensuremath{\mathcal{D}}_{T1}$%
\lthtmlinlinemathZ
\lthtmlcheckvsize\clearpage}

{\newpage\clearpage
\lthtmlinlinemathA{tex2html_wrap_inline872}%
$\ensuremath{\mathcal{D}}_{q1}$%
\lthtmlinlinemathZ
\lthtmlcheckvsize\clearpage}

{\newpage\clearpage
\lthtmlinlinemathA{tex2html_wrap_inline874}%
$\langle X \rangle$%
\lthtmlinlinemathZ
\lthtmlcheckvsize\clearpage}

{\newpage\clearpage
\lthtmlinlinemathA{tex2html_wrap_inline876}%
$\widehat{X}$%
\lthtmlinlinemathZ
\lthtmlcheckvsize\clearpage}

\stepcounter{chapter}
\stepcounter{section}
\stepcounter{section}
\stepcounter{section}
\stepcounter{section}
\stepcounter{section}
\stepcounter{section}
\stepcounter{section}
\stepcounter{subsection}
\stepcounter{subsection}
\stepcounter{subsection}
\stepcounter{subsection}
\stepcounter{subsection}
\stepcounter{subsubsection}
\stepcounter{subsubsection}
\stepcounter{subsection}
{\newpage\clearpage
\lthtmlinlinemathA{tex2html_wrap_inline2561}%
$\,\!$%
\lthtmlinlinemathZ
\lthtmlcheckvsize\clearpage}

\stepcounter{section}
\stepcounter{subsection}
\stepcounter{subsection}
\stepcounter{subsection}
\stepcounter{subsection}
\stepcounter{subsubsection}
\stepcounter{subsubsection}
\stepcounter{subsubsection}
\stepcounter{subsection}
{\newpage\clearpage
\lthtmlinlinemathA{tex2html_wrap_inline3258}%
$\backslash$%
\lthtmlinlinemathZ
\lthtmlcheckvsize\clearpage}

\stepcounter{chapter}
\stepcounter{section}
{\newpage\clearpage
\lthtmlinlinemathA{tex2html_wrap_inline16320}%
$\textstyle \parbox{70ex}{\footnotesize{%
	\textcolor{blue}{\texttt{%
from qtcm import Qtcm \\
inputs = \{\} \\
inputs['runname'] = 'test' \\
inputs['landon'] = 0 \\
inputs['year0'] = 1 \\
inputs['month0'] = 11 \\
inputs['day0'] = 1 \\
inputs['lastday'] = 30 \\
inputs['mrestart'] = 0 \\
inputs['compiled\_form'] = 'parts' \\
model = Qtcm(**inputs) \\
model.run\_session()}}
	}}$%
\lthtmlinlinemathZ
\lthtmlcheckvsize\clearpage}

{\newpage\clearpage
\lthtmlinlinemathA{tex2html_wrap_inline16321}%

% latex2html id marker 16321
$\textstyle \parbox{70ex}{\footnotesize{Figure~\ref{fig:my.first.run}:
	An example of a simple \textcolor{blue}{\texttt{qtcm}} run.}}$%
\lthtmlinlinemathZ
\lthtmlcheckvsize\clearpage}

\stepcounter{section}
{\newpage\clearpage
\lthtmlinlinemathA{tex2html_wrap_inline16323}%
$\textstyle \parbox{70ex}{\footnotesize{%
	\textcolor{blue}{\texttt{%
from qtcm import Qtcm \\
rundirname = 'test' \\
dirbasepath = os.path.join(os.getcwd(), rundirname) \\
inputs = \{\} \\
inputs['bnddir'] = os.path.join( os.getcwd(), 'bnddir', \\
\hspace*{40ex}'r64x42' ) \\
inputs['SSTdir'] = os.path.join( os.getcwd(), 'bnddir', \\
\hspace*{40ex}'r64x42', 'SST\_Reynolds' ) \\
inputs['outdir'] = dirbasepath \\
inputs['runname'] = rundirname \\
inputs['landon'] = 0 \\
inputs['year0'] = 1 \\
inputs['month0'] = 11 \\
inputs['day0'] = 1 \\
inputs['lastday'] = 30 \\
inputs['mrestart'] = 0 \\
inputs['compiled\_form'] = 'parts' \\
model = Qtcm(**inputs) \\
model.run\_session()}}
	}}$%
\lthtmlinlinemathZ
\lthtmlcheckvsize\clearpage}

{\newpage\clearpage
\lthtmlinlinemathA{tex2html_wrap_inline16324}%

% latex2html id marker 16324
$\textstyle \parbox{70ex}{\footnotesize{Figure~\ref{fig:manage.dir.example}:
	An example \textcolor{blue}{\texttt{qtcm}} run showing detailed description of
        input and output directories.}}$%
\lthtmlinlinemathZ
\lthtmlcheckvsize\clearpage}

\stepcounter{section}
\stepcounter{section}
\stepcounter{subsection}
\stepcounter{subsection}
\stepcounter{subsection}
{\newpage\clearpage
\lthtmlinlinemathA{tex2html_wrap_inline16339}%
$\textstyle \parbox{70ex}{\footnotesize{%
	\textcolor{blue}{\texttt{%
inputs['year0'] = 1 \\
inputs['month0'] = 11 \\
inputs['day0'] = 1 \\
inputs['lastday'] = 10 \\
inputs['mrestart'] = 0 \\
inputs['compiled\_form'] = 'parts' \\\\
model = Qtcm(**inputs) \\
model.run\_session() \\
model.u1.value = model.u1.value * 2.0 \\
model.init\_with\_instance\_state = True \\
model.run\_session(cont=30)}}
	}}$%
\lthtmlinlinemathZ
\lthtmlcheckvsize\clearpage}

{\newpage\clearpage
\lthtmlinlinemathA{tex2html_wrap_inline16340}%

% latex2html id marker 16340
$\textstyle \parbox{70ex}{\footnotesize{Figure~\ref{fig:continuation.example}:
	An example of two \textcolor{blue}{\texttt{qtcm}} run sessions where the second
	run session is a continuation of the first.  Assume 
	\textcolor{blue}{\texttt{inputs}} is a dictionary, and that earlier in the
	script the run name and
	all input and output directory names were added
	to the dictionary.}}$%
\lthtmlinlinemathZ
\lthtmlcheckvsize\clearpage}

\stepcounter{subsection}
\stepcounter{section}
\stepcounter{subsection}
\stepcounter{subsection}
\stepcounter{section}
\stepcounter{section}
\stepcounter{subsection}
{\newpage\clearpage
\lthtmlinlinemathA{tex2html_wrap_inline16357}%
$\textstyle \parbox{70ex}{\footnotesize{%
	\textcolor{blue}{\texttt{%
import numpy as N \\
import Scientific as S \\\\
fileobj = S.NetCDFFile(datafn, mode='r') \\\\
data = N.array(fileobj.variables[id].getValue()) \\
data\_name = fileobj.variables[id].long\_name \\
data\_units = fileobj.variables[id].units \\\\
lat = N.array(fileobj.variables['lat'].getValue()) \\
lat\_name = fileobj.variables['lat'].long\_name \\
lat\_units = fileobj.variables['lat'].units \\\\
lon = N.array(fileobj.variables['lon'].getValue()) \\
lon\_name = fileobj.variables['lon'].long\_name \\
lon\_units = fileobj.variables['lon'].units \\\\
time = N.array(fileobj.variables['time'].getValue()) \\
time\_name = fileobj.variables['time'].long\_name \\
time\_units = fileobj.variables['time'].units \\\\
fileobj.close()}}
	}}$%
\lthtmlinlinemathZ
\lthtmlcheckvsize\clearpage}

{\newpage\clearpage
\lthtmlinlinemathA{tex2html_wrap_inline16358}%

% latex2html id marker 16358
$\textstyle \parbox{70ex}{\footnotesize{Figure~\ref{fig:netcdf.read}:
	Example of Python code to read netCDF output.
	See text for description.}}$%
\lthtmlinlinemathZ
\lthtmlcheckvsize\clearpage}

\stepcounter{subsection}
\stepcounter{section}
\stepcounter{chapter}
\stepcounter{section}
\stepcounter{section}
\stepcounter{section}
\stepcounter{section}
\stepcounter{subsection}
\stepcounter{subsection}
\stepcounter{subsection}
\stepcounter{subsubsection}
\stepcounter{subsubsection}
\stepcounter{section}
\stepcounter{subsection}
\stepcounter{subsection}
\stepcounter{subsection}
\stepcounter{subsection}
\stepcounter{subsubsection}
\stepcounter{subsubsection}
\stepcounter{subsubsection}
\stepcounter{subsection}
\stepcounter{section}
{\newpage\clearpage
\lthtmlfigureA{table7217}%
\begin{table}\begin{center}
\fbox{Empty placeholder block for table that would have gone here.}
\end{center}

\end{table}%
\lthtmlfigureZ
\lthtmlcheckvsize\clearpage}

\stepcounter{section}
\stepcounter{subsection}
\stepcounter{subsection}
\stepcounter{subsubsection}
{\newpage\clearpage
\lthtmlinlinemathA{tex2html_wrap_inline7978}%
$<$%
\lthtmlinlinemathZ
\lthtmlcheckvsize\clearpage}

{\newpage\clearpage
\lthtmlinlinemathA{tex2html_wrap_inline7982}%
$^2$%
\lthtmlinlinemathZ
\lthtmlcheckvsize\clearpage}

\stepcounter{subsubsection}
\stepcounter{subsection}
\stepcounter{subsection}
\stepcounter{section}
\stepcounter{section}
\stepcounter{section}
{\newpage\clearpage
\lthtmlinlinemathA{tex2html_wrap_inline16907}%
$\textstyle \parbox{70ex}{\footnotesize{%
	\textcolor{blue}{\texttt{%
inputs = \{\} \\
inputs['runname'] = 'test' \\
inputs['landon'] = 0 \\
inputs['year0'] = 1 \\
inputs['month0'] = 11 \\
inputs['day0'] = 1 \\
inputs['lastday'] = 30 \\
inputs['mrestart'] = 0 \\
inputs['init\_with\_instance\_state'] = True \\
inputs['compiled\_form'] = 'parts'}}
	}}$%
\lthtmlinlinemathZ
\lthtmlcheckvsize\clearpage}

{\newpage\clearpage
\lthtmlinlinemathA{tex2html_wrap_inline16908}%

% latex2html id marker 16908
$\textstyle \parbox{70ex}{\footnotesize{Figure~\ref{fig:defn.of.inputs}:
	The initial definition of the \textcolor{blue}{\texttt{inputs}} dictionary for 
	examples given in Section~\ref{sec:cookbook}.  These settings
	imply that a run session will start on November 1, Year 1,
	last for 30 days, and will be an aquaplanet run.}}$%
\lthtmlinlinemathZ
\lthtmlcheckvsize\clearpage}

{\newpage\clearpage
\lthtmlinlinemathA{tex2html_wrap_inline16915}%
$\textstyle \parbox{70ex}{\footnotesize{%
	\textcolor{blue}{\texttt{%
import os \\
inputs['init\_with\_instance\_state'] = False \\
for i in xrange(0,1002,10): \\
\hspace*{5ex}iname = 'ziml-' + str(i) + 'm' \\
\hspace*{5ex}ipath = os.path.join('proc', iname) \\
\hspace*{5ex}os.makedirs(ipath) \\
\hspace*{5ex}model = Qtcm(**inputs) \\
\hspace*{5ex}model.ziml.value = float(i)  \\
\hspace*{5ex}model.runname.value = iname \\
\hspace*{5ex}model.outdir.value = ipath \\
\hspace*{5ex}model.run\_session() \\
\hspace*{5ex}del model}}
	}}$%
\lthtmlinlinemathZ
\lthtmlcheckvsize\clearpage}

{\newpage\clearpage
\lthtmlinlinemathA{tex2html_wrap_inline16916}%
$\textstyle \parbox{70ex}{\footnotesize{%
	\textcolor{blue}{\texttt{%
import os \\
import numpy as N \\
maxu1 = 0.0 \\
while maxu1 < 10.0: \\
\hspace*{5ex}iziml = 0.1 * maxu1 \\
\hspace*{5ex}iname = 'ziml-' + str(iziml) + 'm' \\
\hspace*{5ex}ipath = os.path.join('proc', iname) \\
\hspace*{5ex}os.makedirs(ipath) \\
\hspace*{5ex}model = Qtcm(**inputs) \\
\hspace*{5ex}try: \\
\hspace*{10ex}model.sync\_set\_py\_values\_to\_snapshot(snapshot=mysnapshot) \\
\hspace*{10ex}model.init\_with\_instance\_state = True \\
\hspace*{5ex}except: \\
\hspace*{10ex}model.init\_with\_instance\_state = False \\
\hspace*{5ex}model.ziml.value = iziml  \\
\hspace*{5ex}model.runname.value = iname \\
\hspace*{5ex}model.outdir.value = ipath \\
\hspace*{5ex}model.run\_session() \\
\hspace*{5ex}maxu1 = N.max(N.abs(model.u1.value)) \\
\hspace*{5ex}mysnapshot = model.snapshot \\
\hspace*{5ex}del model}}
	}}$%
\lthtmlinlinemathZ
\lthtmlcheckvsize\clearpage}

{\newpage\clearpage
\lthtmlinlinemathA{tex2html_wrap_inline16917}%

% latex2html id marker 16917
$\textstyle \parbox{70ex}{\footnotesize{Figure \ref{fig:conditional.test.eg}:
	This code explores different values of
	mixed-layer depth \textcolor{blue}{\texttt{ziml}} for 30~day runs,
	as a function of maximum \textcolor{blue}{\texttt{u1}} magnitude,
	until it finds a case where the maximum \textcolor{blue}{\texttt{u1}} is
	greater than 10~m/s.  (The relationship between
	\textcolor{blue}{\texttt{ziml}} and the maximum of the speed of
	\textcolor{blue}{\texttt{u1}}, where 
	\textcolor{blue}{\texttt{ziml\thinspace=\thinspace0.1\thinspace*\thinspace{maxu1}}}, 
	is made up.)
	With each iteration, the new run uses the snapshot from
	a previous run to initialize (as well as the new value
	of \textcolor{blue}{\texttt{ziml}}); the \textcolor{blue}{\texttt{try}} statement is used to
	ensure the model works even if \textcolor{blue}{\texttt{mysnapshot}} is not
	defined (which is the case the first time around).
	The \textcolor{blue}{\texttt{inputs}} dictionary is initialized with the code in
	Figure~\ref{fig:defn.of.inputs}.}}$%
\lthtmlinlinemathZ
\lthtmlcheckvsize\clearpage}

{\newpage\clearpage
\lthtmlinlinemathA{tex2html_wrap_inline16918}%
$\textstyle \parbox{70ex}{\footnotesize{%
	\textcolor{blue}{\texttt{%
import os \\
\\
class NewQtcm(Qtcm): \\
\hspace*{5ex}def cloud0(self):\\
\hspace*{10ex}[\ldots] \\
\hspace*{5ex}def cloud1(self):\\
\hspace*{10ex}[\ldots] \\
\hspace*{5ex}def cloud2(self):\\
\hspace*{10ex}[\ldots] \\
\hspace*{5ex}[\ldots] \\
\\
inputs['init\_with\_instance\_state'] = False \\
for i in xrange(10): \\
\hspace*{5ex}iname = 'cloudroutine-' + str(i)  \\
\hspace*{5ex}ipath = os.path.join('proc', iname) \\
\hspace*{5ex}os.makedirs(ipath) \\
\hspace*{5ex}model = NewQtcm(**inputs) \\
\hspace*{5ex}model.runlists['atm\_physics1'][1] = 'cloud' + str(i) \\
\hspace*{5ex}model.runname.value = iname \\
\hspace*{5ex}model.outdir.value = ipath \\
\hspace*{5ex}model.run\_session() \\
\hspace*{5ex}del model}}
	}}$%
\lthtmlinlinemathZ
\lthtmlcheckvsize\clearpage}

{\newpage\clearpage
\lthtmlinlinemathA{tex2html_wrap_inline16919}%

% latex2html id marker 16919
$\textstyle \parbox{70ex}{\footnotesize{Figure \ref{fig:alt.param.inherit.eg}:
	Let's say we have 9 different cloud physics schemes we wish
	to try out in 9 different runs.  The easiest way to do this
	is to create a new class \textcolor{blue}{\texttt{NewQtcm}} that
	inherits everything from \textcolor{blue}{\texttt{Qtcm}}, and to which we'll
	add the additional cloud schemes (\textcolor{blue}{\texttt{cloud0}}, \textcolor{blue}{\texttt{cloud1}},
	etc.).
	In the \textcolor{blue}{\texttt{for}} loop, I change the cloud model
	run list entry in the run list that governs
	atmospheric physics at one instant to whatever the cloud
	model is at this point in the loop.
	The \textcolor{blue}{\texttt{inputs}} dictionary is initialized with the code in
	Figure~\ref{fig:defn.of.inputs}.
	Of course, we could do the same thing by running the 9
	models separately, but this set-up makes it easy to do
	hypothesis testing with these 9 models.  For instance, we
	can create a test by which we will choose which of the 9
	models to use:  Within this framework, the selection of
	those models can be altered by changing a string.}}$%
\lthtmlinlinemathZ
\lthtmlcheckvsize\clearpage}

\stepcounter{chapter}
\stepcounter{section}
\stepcounter{section}
\stepcounter{chapter}
\stepcounter{section}
\stepcounter{section}
\stepcounter{section}
\stepcounter{subsection}
\stepcounter{subsubsection}
\stepcounter{subsubsection}
\stepcounter{subsection}
\stepcounter{section}
\stepcounter{subsection}
\stepcounter{subsection}
\stepcounter{subsection}
\stepcounter{subsection}
\stepcounter{section}
\stepcounter{section}
\stepcounter{subsection}
\stepcounter{subsection}
\stepcounter{subsection}
\stepcounter{section}
\stepcounter{subsection}
\stepcounter{subsection}
\stepcounter{subsection}
\stepcounter{section}
\stepcounter{chapter}
{\newpage\clearpage
\lthtmlfigureA{center17036}%
\begin{center}\vbox{%/u/sy/beebe/tex/bib/bibnames.sty, Tue Jul 10 14:19:57 1990
%Edit by Nelson H.F. Beebe <beebe@plot79.math.utah.edu>
%%
%%  @texfile{
%%      author          = "Nelson H. F. Beebe",
%%      version         = "1.03",
%%      date            = "01 August 1991",
%%      filename        = "bibnames.sty",
%%      address         = "Center for Scientific Computing
%%                         and Department of Mathematics
%%                         South Physics Building
%%                         University of Utah
%%                         Salt Lake City, UT 84112
%%                         USA
%%                         Tel: (801) 581-5254",
%%      checksum        = "143     393    4284",
%%      email           = "beebe@science.utah.edu (Internet)",
%%      codetable       = "ISO/ASCII",
%%      supported       = "yes",
%%      docstring       = "This file provides standard definitions of
%%                         proper names needed in TeX-related
%%                         bibliographies.  The definitions are a
%%                         significant extension of those in Rick
%%                         Futura's texnames.sty; the latter is not
%%                         used directly here, because not all sites
%%                         have it.
%%                         The checksum field above contains the
%%                         standard UNIX wc (word count) utility
%%                         output of lines, words, and characters;
%%                         eventually, a better checksum scheme should
%%                         be developed."
%%      }
%
%=======================================================================
% The TeX-related program names that are not defined already in LaTeX
%
% Richard Furuta
% Department of Computer Science
% University of Maryland
% College Park, MD  20742
%
% furuta@mimsy.umd.edu
% seismo!umcp-cs!furuta
%
% October 22, 1986, first release
%
% April 1, 1987:  modified by William LeFebvre, Rice University
%   Now includes definitions for BibTeX and SLiTeX, as they appear in the
%   LaTeX Local User's Guide.
%
% April 10, 1989: modified by NHFB: added use of \ifundefined so as to
% preserve any existing definitions.
%=======================================================================
% TeXbook, p. 308:
\def\ifundefined#1{\expandafter\ifx\csname#1\endcsname\relax}
%
% The following comes from Mike Spivak
\ifundefined{AMSTEX}
    \newcommand{\AMSTEX}{$\cal A$\kern-.1667em\lower.5ex\hbox{$\cal M$}
                \kern-.125em$\cal S$-\TeX}
\fi
% Letter case variants:
\ifundefined{AmSTeX}
    \newcommand{\AmSTeX}{\AMSTEX{}}
\fi
%
\ifundefined{AMSTeX}
    \newcommand{\AMSTeX}{\AMSTEX{}}
\fi
%
% \BibTeX and \SLiTeX, taken from the top of the local user's guide  (--wnl)
\ifundefined{BibTeX}
    \newcommand{\BibTeX}{{\rm B\kern-.05em{\sc i\kern-.025em b}\kern-.08em
                T\kern-.1667em\lower.7ex\hbox{E}\kern-.125emX}}
\fi
%
\ifundefined{CMR}
    \newcommand{\CMR}{{Computer Modern}}
\fi
%
\ifundefined{CWEB}
    \newcommand{\CWEB}{{\tt CWEB}}
\fi
%
\ifundefined{emdash}
    \newcommand{\emdash}{\penalty\exhyphenpenalty---\penalty\exhyphenpenalty}
\fi
%
\ifundefined{LAMSTeX}
\newcommand{\LAMSTeX}{L\raise.42ex\hbox{\kern-.3em\the\scriptfont2 A}%
    \kern-.2em\lower.376ex\hbox{\the\textfont2 M}\kern-.125em
    {\the\textfont2 S}-\\Tex{}}
\fi
%
\ifundefined{FWEB}
    \newcommand{\FWEB}{{\tt FWEB}}
\fi
%
\ifundefined{METAFONT}
    \font\mf=logo10
%    \hyphenchar\mf=-1
    \newcommand{\METAFONT}{{\mf META{\rm{}\-{}}FONT}}
\fi
%
\ifundefined{MF}
    \newcommand{\MF}{\METAFONT}
\fi
%
\ifundefined{ndash}
    \newcommand{\ndash}{\penalty\exhyphenpenalty--\penalty\exhyphenpenalty}
\fi
%
\ifundefined{noopsort}
    \newcommand{\noopsort}[1]{}
\fi
%
\ifundefined{PLOT}
    \newcommand{\PLOT}{{\mbox{\raise.2ex\hbox{$<$}\kern-.06em\hbox{PLOT79}
               \kern-.3em\hbox{\raise.2ex\hbox{$>$}}}}}
\fi
%
\ifundefined{POSTSCRIPT}
    \newcommand{\POSTSCRIPT}{{\sc Post\-Script}}
\fi
%
\ifundefined{PS}
    \newcommand{\PS}{\POSTSCRIPT}
\fi
%
\ifundefined{singleletter}
    \newcommand{\singleletter}[1]{#1}
\fi
%
\ifundefined{SLiTeX}
    \newcommand{\SLiTeX}{{\rm S\kern-.06em{\sc l\kern-.035emi}\kern-.06em
                T\kern -.1667em\lower.7ex\hbox{E}\kern-.125emX}}
\fi
%
\ifundefined{tubissue}
    \newcommand{\tubissue}[2]{\TUB~#1, no.~#2}
\fi
%
\ifundefined{TUB}
    \newcommand{\TUB}{{\em TUG\-boat\/}}
\fi
%
\ifundefined{WEB}
    \newcommand{\WEB}{{\tt WEB}}
\fi

}\end{center}%
\lthtmlfigureZ
\lthtmlcheckvsize\clearpage}

\appendix
\stepcounter{chapter}
\stepcounter{section}
\stepcounter{section}

\end{document}
