% ==========================================================================
% Troubleshooting
%
% By Johnny Lin
% ==========================================================================


% ------ BODY -----
%
\section{Error Messages Produced by \mods{qtcm}}

\begin{description}
\item[\screen{Error-Value too long in SetbyPy module getitem\_str for}
	\dumarg{key}:]
	This message is produced by the Fortran
	subroutine \mods{getitem\_str}
	in the module \mods{SetbyPy} in the compiled QTCM1 Fortran code.
	The code is in the file \fn{setbypy.F90}.  This error occurs when
	the Fortran variable whose name is given by the string \dumarg{key}
	has a value that is greater than the local parameter
	\vars{maxitemlen} in \mods{getitem\_str}.  To fix this, you have
	to go into \fn{setbypy.F90} and change the value of
	\vars{maxitemlen}.

\item[\screen{Error-real\_rank1\_array should be deallocated}:]
	Fortran module \mods{SetByPy}'s subroutine
	\mods{getitem\_real\_array} generates this message
	(or a similar message for other ranks) if the Fortran
	variable for the input \dumarg{key} are allocated on entry
	to the routine.  This may indicate the user has written another
	Fortran routine to access the \mods{real\_rank1\_array} variable
	outside of the standard interfaces..

\item[\screen{Error-Bad call to SetbyPy module \ldots}:]
	Often times, this error occurs because a get or set routine
	in \mods{SetByPy} tried to act on a variable for which the
	corresponding input \dumarg{key} is not defined.  The solution
	is to add that case in the if/then construct for the get and set
	routines in \mods{SetByPy} and rebuild the extension modules.
\end{description}


\section{Other Errors}

\begin{description}
\item[Python cannot find some packages:]
	This error often happens when the version of Python in which
	you have installed all your packages is not the version that
	is called at the Unix command line by typing in \cmd{python}.
	To get around this, 
        define a Unix alias
        that maps \cmd{python2.4} (or whichever version of Python
	has all your packages installed) to \cmd{python}.  If you
	have multiple Python's installed on your system, you might
	have to use a more specific name for the Python executable.
	As a result, you may have to change the test scripts in
	\fn{test} in the \mods{qtcm} distribution directory.

\item[\mods{get\_qtcm1\_item} and compiled QTCM1 model pointer
	variables:]
	If you try to use the \mods{get\_qtcm1\_item} method on a compiled
	QTCM1 model pointer variable 
	(i.e., \vars{u1}, \vars{v1}, \vars{q1}, \vars{T1}),
	 before the compiled
	model \mods{varinit} subroutine is run, you'll get a bus error
	with no additional message.

\item[Mismatch between Python and Fortran array field variables:]
	You change an array field variable on the Python side, but
	it seems like the wrong elements are changed on the Fortran
	side.  Or you type in the same index address for accessing a
	\mods{qtcm} netCDF output array as well as its \class{Qtcm}
	instance attribute counterpart, and find you get different
	answers.  Some possible reasons and fixes:

	\begin{itemize}
	\item This will occur if you haven't accounted for the
		difference in how field variables are saved at the
		Python-level, Fortran-level, and in a netCDF file.
		All netCDF array output is dimensioned (time,
		latitude, longitude) when read into Python using
		the \mods{Scientific} package.  This differs from
		the way \class{Qtcm} saves field variables, \emph{both}
		at the Python- and Fortran-levels, which follows
		Fortran convention (longitude, latitude).

		Note that the way \class{Qtcm} saves field variables
		at the Python- and Fortran-levels is different than
		the default way Python and Fortran save arrays.
		Section~\ref{sec:field.var.shape} for more information.

	\item You may have forgotten that array indices in Python start at
		0, while indices in Fortran (generally) start at 1.
		Also, ranges in Python are exclusive at the upper-bound,
		while ranges in Fortran are inclusive at the upper-bound.
		(Both Python and Fortran array indice ranges are inclusive
		at the lower-bound.)

	\item You may have forgotten some field variables have
		ghost latitudes, and thus there are extra latitude bands
		when the array is stored as a Python or Fortran field
		variable, but there are \emph{no} extra latitude bands
		when the array is stored as netCDF output (the QTCM1
		output routines strip off the ghost latitudes when
		writing those field variables out).
	        See the
        \latexhtml{%
\htmladdnormallinkfoot{QTCM1 manual}%
        {http://www.atmos.ucla.edu/$\sim$csi/qtcm\_man/v2.3/qtcm\_manv2.3.pdf}}%
{\htmladdnormallink{QTCM1 manual}%
        {http://www.atmos.ucla.edu/~csi/qtcm_man/v2.3/qtcm_manv2.3.pdf}}
        \cite{Neelin/etal:2002}
        for details about ghost latitudes.

		The safest and easiest way to tell whether the variable has a
		ghost latitudes is to look at its shape.
		A call to the \class{Qtcm} instance
		method \mods{get\_qtcm1\_item} will give you the array,
		and the use of NumPy's \mods{shape} function will give you
		the shape.
	\end{itemize}
\end{description}




% ===== end of file =====
