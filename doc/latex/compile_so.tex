% ==========================================================================
% Compiling extension modules
%
% By Johnny Lin
% ==========================================================================


% ------ BODY -----
%

The extension modules (\fn{.so} files) are imported and used by
\mods{qtcm} objects, and contain the Fortran QTCM1 model that is
called by the \mods{qtcm} Python wrappers.  These extension modules
are located in the \fn{lib} directory of the \mods{qtcm} distribution,
and, in general, need to be created only when the \mods{qtcm} package
is installed.

Two extension modules are created:  \fn{\_qtcm\_full\_365.so} and
\fn{\_qtcm\_parts\_365.so}.  Both modules define QTCM1 models where:

\begin{itemize}
\item A year is 365 days long 
	(makefile macro \vars{YEAR360} is off).
\item Model output is written to netCDF files
	(makefile macro \vars{NETCDFOUT} is on).
\item The atmospheric boundary layer model is used
	(makefile macro \vars{NO\_ABL} is off).
\item A global domain is used
	(makefile macro \vars{SPONGES} is off).
\item Topography effects due to induced divergence are not included
	(makefile macro \vars{TOPO} is off).
\item Coupling between atmosphere and ocean is through mean fluxes
	(makefile macro \vars{CPLMEAN} is off).
\item The mixed layer ocean model is not used
	(makefile macros \vars{MXL\_OCEAN} and \vars{BLEND\_SST} are both off).
\end{itemize}

(All other makefile macros not listed are also turned off.)
The only difference between these two extension modules is that the
``full'' module is used by \class{Qtcm} instances where
\vars{compiled\_form} is set to \vars{'full'}, and the ``parts''
module is used by \class{Qtcm} instances where \vars{compiled\_form}
is set to \vars{'parts'}.  See Section~\ref{sec:compiledform} for
details about the \vars{compiled\_form} attribute.

The extension modules are created through the following steps:
\begin{enumerate}
\item Go to the \mods{qtcm} distribution directory
	\fn{qtcm-0.1.2}located in
	your build path \fn{/buildpath}.  Go to the \fn{src}
	sub-directory.  This is where all the building of the
	extension modules will take place.

\item Copy the makefile that corresponds to your platform to
	the \fn{src} directory, and rename it \fn{makefile}.
	The \fn{Makefiles} sub-directory of \fn{src} contains
	makefiles for various platforms.

\item In \fn{makefile}, make the following changes:
	\begin{enumerate}
	\item Change the \vars{FC} environment variable as needed, 
		if your Fortran compiler is different.
	\item Change the \vars{FFLAGSM} environment variable, if the
		compiler flags listed are not supported by your
		compiler.
	\item Change the \vars{-I} and \vars{-L} parts of the
		\vars{NCINC} and \vars{NCLIB} environment
		variables so that the paths for the netCDF library and
		include files match your system's installation:
		\begin{codeblock}
		\codeblockfont{%
NCINC=-I/yourpath/netcdf/include \\
NCLIB=-L/yourpath/netcdf/lib -lnetcdf}
		\end{codeblock}
		Set \dumarg{yourpath} to the full path to the
		\fn{netcdf} directory where the \fn{include} and
		\fn{lib} sub-directories are that hold the netCDF
		libraries and include files.
		(You shouldn't have to change the \vars{-l} part of
		\vars{NCLIB}, since it is standard to name the netCDF
		library \fn{libnetcdf.a}.  But if you have a non-standard
		installation, change the \vars{-l} part too.)
	\end{enumerate}

\item At the Unix prompt, type:
\begin{codeblock}
\codeblockfont{%
\small
make clean \&\& make \_qtcm\_full\_365.so \&\& make \_qtcm\_parts\_365.so}
\end{codeblock}
	to clean up leftover files from previous compilations, and to
	compile the extension module shared object files
	\fn{\_qtcm\_full\_365.so} and \fn{\_qtcm\_parts\_365.so}.
\end{enumerate}

The makefile will automatically move the shared object files into
\fn{../lib}, overwriting any pre-existing files of the same name.
A detailed description of the makefile and using \mods{f2py} is
given in Section~\ref{sec:create.new.so}, if you wish to create a
different extension module.




% ===== end of file =====
