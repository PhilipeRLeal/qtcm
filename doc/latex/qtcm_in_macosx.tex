% ==========================================================================
% Description of installing in Mac OS X
%
% By Johnny Lin
% ==========================================================================


% ------ BODY -----
%
%------------------------------------------------------------------------
\subsection{Introduction}

This section describes issues and a summary of the installation steps
I followed to install \mods{qtcm} on a Mac running OS X.
It is a specific realization of the general installation
instructions found in Sections~\ref{sec:install.sum}--\ref{sec:test.qtcm}.
I first worked through these installation steps during June--July 2007,
with updates during July 2008.
The best way to go through this section is to go through
the summary of the installation steps in 
Section~\ref{sec:osx.install.summary},
and looking back to other sections as needed.




%------------------------------------------------------------------------
\subsection{Platform and Unix Dependencies}

This work was done on a MacBook 1.83 GHz Intel Core Duo running Mac OS X
10.4.11.  My machine has 1 GB RAM and 64 GB of disk in its main partition.

I recommend you turn-off your antivirus software before you
do the installs.  
Problems have been
\latexhtml{reported by Fink users\footnote%
		{http://finkproject.org/faq/usage-fink.php?phpLang=en\#kernel-panics}}%
	{\htmladdnormallink{reported by Fink users}%
		{http://finkproject.org/faq/usage-fink.php?phpLang=en#kernel-panics}}
using the Fink package manager with antivirus software enabled.

There are a variety of dependencies that are required to get your Mac
up-and-running as a scientific computing platform.  The most basic is
installing Apple's 
\htmladdnormallinkfoot{XCode}{http://developer.apple.com/tools/xcode/}
developer tools.\footnote%
	{The package should work in Mac OS X 10.4 with XCode 2.4.1 and higher;
	I've tried it with both 2.4.1 and XCode 2.5.  Note that
	XCode 3.1 only works on Mac OS X 10.5.}
This set of tools contains compilers and libraries
needed to do anything further.  You have to be a member of Apple's
Developer Connection, but registration is free.

Besides XCode, there are a variety of Unix libraries and utilities that you
need.  I first tried installing them by myself, from scratch, into
\fn{/usr/local}, but it was hard to keep track of all the dependencies.
A few that did work, and that I installed from their disk images, are:
\htmladdnormallinkfoot{MacTeX}{http://www.tug.org/mactex/}, 
\htmladdnormallinkfoot{MAMP}{http://www.mamp.info/}, and 
\htmladdnormallinkfoot{Tcl/Tk Aqua BI (Batteries Included)}%
	{http://tcltkaqua.sourceforge.net/}.\footnote%
		{Theoretically you can use Fink to install the equivalent
		of these packages, but I like the specific collection 
		found in these packages.  For instance, Tcl/Tk Aqua BI
		runs natively on the Mac.}

For everything else, thankfully, there's the
\htmladdnormallinkfoot{Fink Project}{http://www.finkproject.org/} which
uses a package manager built upon Debian tools to install ports of
Unix programs onto a Mac.  I just 
\htmladdnormallinkfoot{downloaded}%
	{http://www.finkproject.org/download/index.php?phpLang=en}
a binary version of the Fink 0.8.1 installer for Intel Macs,
installed Fink, and used its package management tools to install
(almost) everything else I needed.\footnote%
	{The one drawback of Fink is that it sometimes
	has stability problems.  In those cases, Fink provides
	command line suggestions to fix the problems, which sometimes
	will work.  If not, sometimes
\latexhtml{%
	deleting Fink and everything it installed,\footnote%
	{http://www.finkproject.org/faq/usage-fink.php?phpLang=en\#removing}}{%
\htmladdnormallink{deleting Fink and everything it installed}
	{http://www.finkproject.org/faq/usage-fink.php?phpLang=en#removing},}
	and starting afresh, will do the trick.
	It also appeared to me that sometimes when I installed 
	multiple packages
	via one \cmd{fink install} call, the installation did not work
	as well as when I installed only one package per call.}

Although you do not need anything besides a Fortran compiler and
the netCDF libraries to run QTCM1 in its pure-Fortran form, in order to
manipulate the model and use this Python version \mods{qtcm}, you
need to have Python installed.  The default Python that comes
with the Mac is a little old, so I used Fink to also install
Python 2.5 and related packages, including
\htmladdnormallinkfoot{matplotlib}{http://matplotlib.sourceforge.net/},
\htmladdnormallinkfoot{ScientificPython}{http://dirac.cnrs-orleans.fr/plone/software/scientificpython/},
and
\htmladdnormallinkfoot{SciPy}{http://www.scipy.org}
(see Section~\ref{sec:osx.summary} for details).




%------------------------------------------------------------------------
\subsection{Fortran Compiler}

There are a variety of high-quality, commercial Fortran compilers.
Unfortunately, because I do not have a research budget, I am not able
to use those compilers.  The 
\htmladdnormallinkfoot{GNU Compiler Collection}{http://gcc.gnu.org/}
(GCC) provides a suite of open-source compilers, some of which are the
standards of their language.  Most of the GCC compilers are installed
on your Mac when you install XCode.

\htmladdnormallinkfoot{GNU Fortran}{http://gcc.gnu.org/fortran/}
(\mods{gfortran}), is the Fortran 95 compiler included with the more
recent versions of GCC.
Unfortunately, I was not able to get it to compile QTCM1.
There is a second open-source Fortran compiler,
\htmladdnormallinkfoot{G95}{http://www.g95.org/} (\mods{g95}),
which some feel is farther along in its development than \mods{gfortran}.
I was able to successfully compile QTCM1 with \mods{g95} on my Mac.
I used Fink to install G95
(see Section~\ref{sec:osx.summary} for details).




%------------------------------------------------------------------------
\subsection{NetCDF Libraries}   \label{sec:netcdf}

For some reason, the netCDF libraries and include files
installed by Fink didn't correspond to the files needed
by the calling routines in \mods{qtcm}.  To solve this, I compiled
my own set of 
\htmladdnormallinkfoot{netCDF 3.6.2 libraries}%
	{http://www.unidata.ucar.edu/software/netcdf/}
using the tarball 
\latexhtml{downloaded from UCAR\footnote{http://www.unidata.ucar.edu/downloads/netcdf/netcdf-3\_6\_2/}}%
        {\htmladdnormallink{downloaded from UCAR}{http://www.unidata.ucar.edu/downloads/netcdf/netcdf-3_6_2/}}.

Once I uncompressed and untarred the package, and went into 
the top-level directory of the package, I built the package by typing
the following at the Unix prompt:

\begin{codeblock}
\codeblockfont{%
./configure --prefix=/Users/jlin/extra/netcdf \\
make check \\
make install}
\end{codeblock}

This installed the netCDF binaries, libraries, and include files into
sub-directories \fn{bin}, \fn{lib}, and \fn{include} in 
the directory specified by \vars{--prefix}.
If you want to install the netCDF libraries in the default
(usually \fn{/usr/local}), just leave out the \vars{--prefix}
option.

Note:  When you build netCDF, make sure the build directory
is not in the directory tree of \vars{--prefix}
(or the default directory \fn{/usr/local}).




%------------------------------------------------------------------------
\subsection{Makefile Configuration}  \label{sec:osx.makefile}

	\subsubsection{NetCDF}

In the \fn{src} directory in the \mods{qtcm} distribution, there is a
sub-directory \fn{Makefiles} that contains the makefiles for a
variety of platforms.  Edit the file \fn{makefile.osx\_g95}
so that the lines specifying the environment variables for the
netCDF libraries and include files:

\begin{codeblock}
\codeblockfont{%
NCINC=-I/Users/jlin/extra/netcdf/include \\
NCLIB=-L/Users/jlin/extra/netcdf/lib -lnetcdf}
\end{codeblock}

are changed to the path where your \emph{manually compiled} 
netCDF libraries and include files are.

Copy \fn{makefile.osx\_g95} from the \fn{Makefiles} sub-directory
in \fn{src} into \fn{src}.  
In other words, from the \mods{qtcm} distribution directory
(i.e., \fn{/buildpath}), at the Unix prompt execute:

\begin{codeblock}
\codeblockfont{%
cp src/Makefiles/makefile.osx\_g95 src/makefile}
\end{codeblock}


	\subsubsection{Linking Order}

Compilers in the GNU Compiler Collection (GCC) search libraries
and object files in the order they are listed in the command-line, 
\latexhtml{from left-to-right\footnote%
        {http://gcc.gnu.org/onlinedocs/gcc-4.1.2/gcc/Link-Options.html\#index-l-670}}%
        {\htmladdnormallink{from left-to-right}{http://gcc.gnu.org/onlinedocs/gcc-4.1.2/gcc/Link-Options.html#index-l-670}}.
Thus, if routines in \fn{b.o} call routines in \fn{a.o}, 
you must list the files in the order \fn{a.o b.o}.

For some reason, that isn't the case for \mods{g95}.  Thus, you will
find \mods{g95} makefile rules structured like the following
(below is part of the rule to create an executable (\fn{qtcm}) for
benchmark runs):

% --- Two versions of this rule, one for display in PDF and the other
%     for display in HTML:
%
\begin{latexonly}
\begin{codeblock}
\codeblockfont{%
qtcm: main.o \\
\hspace*{8ex}\$(FC)~-O~\$(NCINC)~-o~\$@ main.o~\$(QTCMLIB)~\$(NCLIB)}
\end{codeblock}
\end{latexonly}

\begin{htmlonly}
\begin{rawhtml}
<p><code><font color="blue">qtcm: main.o<br>
&nbsp;&nbsp;&nbsp;&nbsp;&nbsp;&nbsp;&nbsp;$(FC) -O $(NCINC) -o 
$@ main.o $(QTCMLIB) $(NCLIB)</font></code></p>
\end{rawhtml}
\end{htmlonly}

even though \fn{main.o} depends on the QTCM library 
(specified in macro setting \vars{\$(QTCMLIB)}), which in turn
depends on the netCDF library (specified in macro setting \vars{\$(NCLIB)}).


%------------------------------------------------------------------------
\subsection{Summary of Steps}   \label{sec:osx.install.summary}

The following summarizes all the steps I took to install
\mods{qtcm} in Mac OS X:

\begin{enumerate}
\item Install
	\htmladdnormallinkfoot{XCode 2.5}%
		{http://developer.apple.com/tools/xcode/}.

\item Install 
	\htmladdnormallinkfoot{MacTeX}{http://www.tug.org/mactex/}, 
	\htmladdnormallinkfoot{MAMP}{http://www.mamp.info/}, and 
	\htmladdnormallinkfoot{TCL/Tk Aqua BI (Batteries Included)}%
		{http://tcltkaqua.sourceforge.net/}.

\item Install
	\htmladdnormallinkfoot{Fink 0.8.1}%
		{http://www.finkproject.org/download/index.php?phpLang=en}.
	Make sure you
	\htmladdnormallink{set up your environment}%
		{http://www.finkproject.org/doc/users-guide/install.php\#setup}
	to enable you to use the packages you install with Fink
	(e.g. \vars{PATH} settings, etc.).
	Most of the time, that just means adding the line
	\cmd{source /sw/bin/init.csh} to your \fn{.cshrc} file (or the
	equivalent in your \fn{.bashrc}).

	Note that for many of the packages needed to run \mods{qtcm},
	you need to 
	\htmladdnormallink{configure Fink to download packages 
		from the unstable trees}%
	{http://www.finkproject.org/faq/usage-fink.php?phpLang=en\#unstable}.
	To do that, add \vars{unstable/main} and \vars{unstable/crypto}
	to the \vars{Trees:} line in \fn{/sw/etc/fink.conf}, and run:

	\begin{codeblock}
	\codeblockfont{fink selfupdate} \\
	\codeblockfont{fink index} \\
	\codeblockfont{fink scanpackages} \\
	\codeblockfont{fink update-all}
	\end{codeblock}

	When \cmd{selfupdate} runs, choose \cmd{rsync} for the
	self update method.  If you do not, Fink will not look in the
	unstable trees for packages.

\item Use Fink to install the \mods{g95} Fortran compiler.
	From a Unix prompt, type:

	\begin{codeblock}
	\codeblockfont{fink -$\,\!$-use-binary-dist install g95}
	\end{codeblock}

\item Use Fink to install Python 
	and the NumPy package (which \mods{f2py} is a part of).
	From a Unix prompt, type:

	\begin{codeblock}
	\codeblockfont{%
	fink -$\,\!$-use-binary-dist install python25 \\
	fink -$\,\!$-use-binary-dist install scipy-core-py25}
	\end{codeblock}

	(Numpy used to be called SciPy Core.)  If you want to
	install Python 2.4 instead, just change the ``25'' and ``py25'' above
	(and in later occurrences) to ``24'' and ``py24'', respectively.
	Note that Fink does not have a version of epydoc for Python 2.4,
	so if you wish to create documentation using epydoc, you will
	need to install Python 2.5.

\item Install teTeX and \LaTeX{2HTML} using Fink.
	From a Unix prompt, type:

	\begin{codeblock}
	\codeblockfont{fink -$\,\!$-use-binary-dist install tetex} \\
	\codeblockfont{fink -$\,\!$-use-binary-dist install latex2html}
	\end{codeblock}

	When prompted, choose ghostscript and ghostscript-fonts to
	satistfy the dependency (which should be the default options).
	I tried choosing system-ghostscript8, but Fink looks for
	ghostscript 8.51 and didn't recognize ghostscript 8.57 that
	was already installed in \fn{/usr/local} (via my MacTeX
	install).  \LaTeX{2HTML} has a package required by the
	\mods{qtcm} manual \LaTeX\ file.

\item Install additional programming and
	scientific packages and libraries using Fink.
	From a Unix prompt, type:

	\begin{codeblock}
	\codeblockfont{%
	fink -$\,\!$-use-binary-dist install scientificpython-py25 \\
	fink -$\,\!$-use-binary-dist install matplotlib-py25 \\
	fink -$\,\!$-use-binary-dist install matplotlib-basemap-py25 \\
	fink -$\,\!$-use-binary-dist install matplotlib-basemap-data-py25 \\
	fink -$\,\!$-use-binary-dist install xaw3d \\
	fink -$\,\!$-use-binary-dist install fftw fftw3 \\
	fink -$\,\!$-use-binary-dist install epydoc-py25 \\
	fink -$\,\!$-use-binary-dist install graphviz \\
	fink -$\,\!$-use-binary-dist install scipy-py25}
	\end{codeblock}

\item Manually install netCDF 3.6.2
	(see Section \ref{sec:netcdf}).

\item From this point on, you can follow the
	general instructions given in Section~\ref{sec:install.sum},
	starting with step~\ref{list:download.qtcm.sum}.
	Please do not ignore, however, Section~\ref{sec:install.macosx}'s
	Mac-specific details.

\end{enumerate}



% ===== end of file =====
