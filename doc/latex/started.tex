% ==========================================================================
% Getting Started With qtcm
%
% By Johnny Lin
% ==========================================================================


% ------ BODY -----
%
%---------------------------------------------------------------------
\section{Your First Model Run}

Figure~\ref{fig:my.first.run} shows an example of a script to make
a 30 day seasonal, aquaplanet model run, with run name ``test'',
starting from November 1, Year 1.


%--- Two versions, one for PDF, one for HTML:
\begin{latexonly}
\begin{figure}[htp]
\begin{center}
\begin{codeblock}
\codeblockfont{%
from qtcm import Qtcm \\
inputs = \{\} \\
inputs['runname'] = 'test' \\
inputs['landon'] = 0 \\
inputs['year0'] = 1 \\
inputs['month0'] = 11 \\
inputs['day0'] = 1 \\
inputs['lastday'] = 30 \\
inputs['mrestart'] = 0 \\
inputs['compiled\_form'] = 'parts' \\
model = Qtcm(**inputs) \\
model.run\_session()}
\end{codeblock}
\end{center}
\caption{An example of a simple \mods{qtcm} run.}
\label{fig:my.first.run}
\end{figure}
\end{latexonly}

\begin{htmlonly}
\label{fig:my.first.run}
\begin{center}
\htmlfigcaption{%
	\codeblockfont{%
from qtcm import Qtcm \\
inputs = \{\} \\
inputs['runname'] = 'test' \\
inputs['landon'] = 0 \\
inputs['year0'] = 1 \\
inputs['month0'] = 11 \\
inputs['day0'] = 1 \\
inputs['lastday'] = 30 \\
inputs['mrestart'] = 0 \\
inputs['compiled\_form'] = 'parts' \\
model = Qtcm(**inputs) \\
model.run\_session()}
	}

\htmlfigcaption{Figure~\ref{fig:my.first.run}:
	An example of a simple \mods{qtcm} run.}
\end{center}
\end{htmlonly}



The class describing the QTCM1 model is \class{Qtcm}.  An instance
of \class{Qtcm}, in this example \vars{model}, is created the same
way you create an instance of any class.  When instantiating an
instance of \class{Qtcm}, keyword parameters can be used to override
any default settings.  In the example above, the dictionary
\vars{inputs} specifying all keyword parameters is passed in on the
instantiation of \vars{model}.

The keyword parameter settings in
Figure~\ref{fig:my.first.run} have the following meanings:
\begin{itemize}
\item \vars{runname}:  This string (``test'') is used in the
	output filename.  QTCM1 writes mean and instantaneous
	output files to the directory given in \vars{model.outdir.value},
	with filenames 
	\fn{qm\_}\dumarg{runname}\fn{.nc} for mean output and
	\fn{qi\_}\dumarg{runname}\fn{.nc} for instantaneous output.

\item \vars{landon}: When set to ``0'', the land is turned off and
	the run is an aquaplanet run.  When set to ``1'', the land
	model is turned on.

\item \vars{year0}:  The year the run starts on.

\item \vars{month0}:  The month the run starts on (11 = November).

\item \vars{day0}: The day of the month the run starts on.

\item \vars{lastday}:  The model runs from day 1 to \vars{lastday}.

\item \vars{mrestart}:  When set to ``0'', the run starts from
	default initial conditions
	(see Section~\ref{sec:initial.variables} for a table of
	those values).
	When set to ``1'', the run starts from a restart file.

\item \vars{compiled\_form}:  This keyword sets what form the
	compiled QTCM1 model has, and its value is saved to
	the instance's \vars{compiled\_form} attribute.
	It is a string and can be set either to
	``parts'' or ``full''.  Most of the time, you will want
	to set it to \vars{'parts'}.
	This keyword is the only one
	that must be specified on instantiation; the model instance
	will at least instantiate
	using only the default settings for all the other keyword
	parameters (given in Appendix~\ref{app:defaults.values}).
	See Section~\ref{sec:compiledform} for details about
	what the \vars{compiled\_form} attribute controls.
\end{itemize}

By default, the \vars{SSTmode} attribute, which controls whether the
model will use climatological sea-surface temperatures (SST) 
or real SSTs, is set to the \vars{value} ``seasonal'', thus giving a
run with seasonal forcing at the lower-boundary over the ocean.

This example assumes that the boundary condition files, sea surface
temperature files, and the model output directories are as specified
in submodule \mods{defaults}.  Those values are described in
Section~\ref{sec:defaults.scalar}.




%---------------------------------------------------------------------
\section{Managing Directories}

Most of the time, your boundary condition files and output files
will not be in the locations specified in
Section~\ref{sec:defaults.scalar}, or in the directory your
Python script resides.  The easiest way to tell your \class{Qtcm} 
instance where your input/output files are is to pass them in
as keyword parameters on instantiation.


%--- Two versions, one for PDF, one for HTML:
\begin{latexonly}
\begin{figure}[htp]
\begin{codeblock}
\codeblockfont{%
\small
from qtcm import Qtcm \\
rundirname = 'test' \\
dirbasepath = os.path.join(os.getcwd(), rundirname) \\
inputs = \{\} \\
inputs['bnddir'] = os.path.join( os.getcwd(), 'bnddir', \\
\hspace*{40ex}'r64x42' ) \\
inputs['SSTdir'] = os.path.join( os.getcwd(), 'bnddir', \\
\hspace*{40ex}'r64x42', 'SST\_Reynolds' ) \\
inputs['outdir'] = dirbasepath \\
inputs['runname'] = rundirname \\
inputs['landon'] = 0 \\
inputs['year0'] = 1 \\
inputs['month0'] = 11 \\
inputs['day0'] = 1 \\
inputs['lastday'] = 30 \\
inputs['mrestart'] = 0 \\
inputs['compiled\_form'] = 'parts' \\
model = Qtcm(**inputs) \\
model.run\_session()}
\end{codeblock}

\caption{An example \mods{qtcm} run showing detailed description of
	input and output directories.}
\label{fig:manage.dir.example}
\end{figure}
\end{latexonly}

\begin{htmlonly}
\label{fig:manage.dir.example}
\begin{center}
\htmlfigcaption{%
	\codeblockfont{%
from qtcm import Qtcm \\
rundirname = 'test' \\
dirbasepath = os.path.join(os.getcwd(), rundirname) \\
inputs = \{\} \\
inputs['bnddir'] = os.path.join( os.getcwd(), 'bnddir', \\
\hspace*{40ex}'r64x42' ) \\
inputs['SSTdir'] = os.path.join( os.getcwd(), 'bnddir', \\
\hspace*{40ex}'r64x42', 'SST\_Reynolds' ) \\
inputs['outdir'] = dirbasepath \\
inputs['runname'] = rundirname \\
inputs['landon'] = 0 \\
inputs['year0'] = 1 \\
inputs['month0'] = 11 \\
inputs['day0'] = 1 \\
inputs['lastday'] = 30 \\
inputs['mrestart'] = 0 \\
inputs['compiled\_form'] = 'parts' \\
model = Qtcm(**inputs) \\
model.run\_session()}
	}

\htmlfigcaption{Figure~\ref{fig:manage.dir.example}:
	An example \mods{qtcm} run showing detailed description of
        input and output directories.}
\end{center}
\end{htmlonly}


Figure~\ref{fig:manage.dir.example} shows an example run where those
directories are explicitly specified; in all other aspects, the run
is identical to the one in Figure~\ref{fig:my.first.run}.
In Figure~\ref{fig:manage.dir.example}, output from the model is
directed to the directory described by string variable
\vars{dirbasepath}.  \vars{dirbasepath} is created by joining the
current working directory with the run name given in string variable
\vars{rundirname}.\footnote%
	{The Python \mods{os} module enables platform-independent
	handling of files and directories.  The \mods{os.path.join}
	function resolves paths without the programmer needing to know
	all the possible directory separation characters; the function
	chooses the correct separation character at runtime.  The
	\mods{os.getcwd} function returns the current working directory.}
Setting keyword parameter \vars{outdir} to \vars{dirbasepath} sends
output to \vars{dirbasepath}.  
Keywords \vars{bnddir} and \vars{SSTdir} specify the directories
where non-SST and SST boundary condition files, respectively, are
found.

Interestingly, the default version of QTCM1 does \emph{not} send
all output from the model to \vars{outdir}.  The restart file
\fn{qtcm\_}\dumarg{yyyymmdd}\fn{.restart} (where \dumarg{yyyymmdd}
is the year, month, and day of the model date when the restart
file was written) is written into the current working directory,
not the output directory.  Thus, if you do multiple runs, you'll
have to manually deal with the restart files that will proliferate.

Neither the QTCM1 model nor the \class{Qtcm} object
create the directories specified in \mods{bnddir}, \mods{SSTdir},
and \mods{outdir}.  Failure to do so will create an error.  I use
Python's file management tools to make sure the output directory
is created, and any old output files are deleted.  Here's an example
that does that, using the \vars{dirbasepath} and \vars{rundirname}
variables from Figure~\ref{fig:manage.dir.example}:

\begin{codeblock}
\codeblockfont{%
\small
if not os.path.exists(dirbasepath):  os.makedirs(dirbasepath) \\
qi\_file = os.path.join( dirbasepath, 'qi\_'+rundirname+'.nc' ) \\
qm\_file = os.path.join( dirbasepath, 'qm\_'+rundirname+'.nc' ) \\
if os.path.exists(qi\_file):   os.remove(qi\_file) \\
if os.path.exists(qm\_file):   os.remove(qm\_file)}
\end{codeblock}




%---------------------------------------------------------------------
\section{Model Field Variables}   \label{sec:field.variables.intro}

The term ``field'' variable refers to QTCM1 model variables that 
are accessible at both the compiled Fortran QTCM1 model-level as
well as the Python \class{Qtcm} instance-level.
Field variables are all instances of the \class{Field} class,
and are stored as attributes of the \class{Qtcm} instance.\footnote%
	{Note non-field variables can also be instances of \class{Field},
	and that \class{Qtcm} instances have other attributes that are
	not equal to \class{Field} instances.}

\class{Field} class instances have the following attributes:
\begin{itemize}
\item \vars{id}:  A string naming the field (e.g., ``Qc'', ``mrestart'').
	This string should contain no whitespace.
\item \vars{value}:  The value of the field.  Can be of any type, though
	typically is either a string or numeric scalar or a numeric array.
\item \vars{units}:  A string giving the units of the field.
\item \vars{long\_name}:  A string giving a description of the field.
\end{itemize}

\class{Field} instances also have methods to return the rank 
and typecode of \vars{value}.

Remember, if you want to access the value of a \class{Field} object,
make sure you access that object's \vars{value} attribute.  
Thus, for example,
to assign a variable \vars{foo} to the
\vars{lastday} value for a given
\class{Qtcm} instance \vars{model}, type the following:
\begin{codeblock}
\codeblockfont{%
foo = model.lastday.value}
\end{codeblock}

For scalars, this assignment sets \vars{foo} by value (i.e., a copy
of the value of attribute \vars{model.lastday} is set to \vars{foo}).
In general, however, Python assigns variables by reference.  Use
the \mods{copy} module if you truly want a copy of a field variable's
value (such as an array), rather than an alias.  For more details
about field variables, see Section~\ref{sec:field.variables}.




%---------------------------------------------------------------------
\section{Run Sessions}

	\subsection{What is a Run Session?}

A run session is a unit of simulation where the model is run from
day 1 of simulation to the day specified by the \vars{lastday}
attribute of a \class{Qtcm} instance.  A run session is a
``complete'' model run, at the beginning of which all compiled QTCM1
model variables are set to the values given at the Python-level,
and at the end of which restart files are written, the values
at the Python-level are overwritten by the values in the Fortran
model, and a Python-accessible snapshot is taken of the 
model variables that were written to the restart file.


	\subsection{Changing Variables}

Between run sessions, changing any field variable is as easy
as a Python assignment.  For instance, to change the atmosphere
mixed layer depth to 100~m, just type:
\begin{codeblock}
\codeblockfont{%
model.ziml.value = 100.0}
\end{codeblock}

When changing arrays, be careful to try to match the shape of the 
array.\footnote%
	{At the very least, match the rank of the array, which is required
	for the routines in \mods{setbypy} to properly choose which
	Fortran subroutine to use in reading the Python value.
	I haven't tested if only the rank is needed, however,
	for the passing to work, for a continuation run (my hunch is
	it won't).}
You can use the NumPy \mods{shape} function on a NumPy array to
check its shape.


	\subsection{Continuing a Model Run}  \label{sec:continuation.intro}

Figure~\ref{fig:continuation.example} shows an example of two run
sessions, where the second run session is a continuation of the
first.


%--- Two versions, one for PDF, one for HTML:
\begin{latexonly}
\begin{figure}[htp]
\begin{codeblock}
\codeblockfont{%
\small
inputs['year0'] = 1 \\
inputs['month0'] = 11 \\
inputs['day0'] = 1 \\
inputs['lastday'] = 10 \\
inputs['mrestart'] = 0 \\
inputs['compiled\_form'] = 'parts' \\ \\
model = Qtcm(**inputs) \\
model.run\_session() \\
model.u1.value = model.u1.value * 2.0 \\
model.init\_with\_instance\_state = True \\
model.run\_session(cont=30)}
\end{codeblock}

\caption{An example of two \mods{qtcm} run sessions where the second
	run session is a continuation of the first.  Assume 
	\vars{inputs} is a dictionary, and that earlier in the
	script the run name and
	all input and output directory names were added
	to the dictionary.}
\label{fig:continuation.example}
\end{figure}
\end{latexonly}

\begin{htmlonly}
\label{fig:continuation.example}
\begin{center}
\htmlfigcaption{%
	\codeblockfont{%
inputs['year0'] = 1 \\
inputs['month0'] = 11 \\
inputs['day0'] = 1 \\
inputs['lastday'] = 10 \\
inputs['mrestart'] = 0 \\
inputs['compiled\_form'] = 'parts' \\ \\
model = Qtcm(**inputs) \\
model.run\_session() \\
model.u1.value = model.u1.value * 2.0 \\
model.init\_with\_instance\_state = True \\
model.run\_session(cont=30)}
	}

\htmlfigcaption{Figure~\ref{fig:continuation.example}:
	An example of two \mods{qtcm} run sessions where the second
	run session is a continuation of the first.  Assume 
	\vars{inputs} is a dictionary, and that earlier in the
	script the run name and
	all input and output directory names were added
	to the dictionary.}
\end{center}
\end{htmlonly}


The first run session runs from day 1 to day 10.  The second
run session runs the model for another 30 days.  
Setting the \vars{init\_with\_instance\_state} of
\vars{model} to \vars{True} tells the model to use the
the values of the instance attributes 
(for prognostic variables, right-hand sides, and start date) 
are currently stored \vars{model}
as the initial values for the run\_session.\footnote%
	{Unless overridden, by default, 
	\vars{init\_with\_instance\_state} is set
	to True on \class{Qtcm} instance instantiation.}
The \vars{cont}
keyword in the second \mods{run\_session} call specifies a
continuation run, and the value gives the number of additional
days to run the model.

The set of runs described above would produce the exact same
results as if you had gone into the Fortran model after 10 days,
doubled the first baroclinic mode zonal velocity, and continued
the run for another 30 days.  With the Python example above, however,
you didn't need to know you were going to do that ahead of starting
the model run (which is what a compiled model requires you to do).
Section~\ref{sec:contination.run.sessions} describes continuation
runs in detail.


	\subsection{Passing Restart Snapshots Between Run Sessions}
					\label{sec:snapshot.intro}

The pure-Fortran QTCM1 uses a restart file to enable continuation
runs.  A \class{Qtcm} instance can also make use of that option,
through setting the \vars{mrestart} attribute value
(see Section~\ref{sec:contination.run.sessions} and
Neelin et al.\ \cite{Neelin/etal:2002} for details).  
It's easier, however, instead of using a restart file, to pass 
along a ``snapshot'' dictionary.

The \class{Qtcm} instance method \mods{make\_snapshot} copies the
variables that would be written out to a restart file into a
dictionary that is saves as the instance attribute \vars{snapshot}.
This snapshot can be saved separately, for later recall.  Note that
snapshots are automatically made at the end of a run session.

The following example shows a model \mods{run\_session} call,
following which the snapshot is saved to the variable
\vars{snapshot}:\footnote%
	{Remember Python assignment defaults to assignment by
	reference, so in this example the variable \vars{mysnapshot}
	is a pointer to the \vars{model.snapshot} attribute.
	(However, note that \vars{model.snapshot} itself is not a
	reference, but a distinct copy of those variables; to do
	otherwise would result in a non-static snapshot.)
	If the \vars{model.snapshot} attribute is dereferenced,
	then \vars{mysnapshot} will become the sole pointer to the
	dictionary.}

\begin{codeblock}
\codeblockfont{%
model.run\_session() \\
mysnapshot = model.snapshot}
\end{codeblock}

After taking the snapshot, you might continue the run a while, and
then decide to return to the snapshot you saved.  To do so, use
the \mods{sync\_set\_py\_values\_to\_snapshot}
method to reset the model instance values to
\vars{mysnapshot} before your next run session:
\begin{codeblock}
\codeblockfont{%
model.sync\_set\_py\_values\_to\_snapshot(snapshot=mysnapshot) \\
model.init\_with\_instance\_state = True \\
model.run\_session()}
\end{codeblock}

See Section~\ref{sec:snapshots} for details regarding the use of
snapshots, as well as for a list of what variables are saved in
a snapshot.




%---------------------------------------------------------------------
\section{Creating Multiple Models}

	\subsection{Model Instances}

Creating a new QTCM1 model is as simple as creating another
\class{Qtcm} instance.
For instance, to instantiate two QTCM1
models, \vars{model1} and \vars{model2}, type the following:

\begin{codeblock}
\codeblockfont{%
from qtcm import Qtcm \\
model1 = Qtcm(compiled\_form='parts') \\
model2 = Qtcm(compiled\_form='parts')}
\end{codeblock}

\vars{model1} and \vars{model2} do \emph{not} share any variables
in common, including the extension modules holding the Fortran
code.  In creating the instances, a copy of the extension modules
are saved in temporary directories.


	\subsection{Passing Snapshots To Other Models}

The snapshots described in Section~\ref{sec:snapshot.intro}
can also be passed around to other model instances,
enabling you to easily branch a model run:

\begin{codeblock}
\codeblockfont{%
model.run\_session() \\
mysnapshot = model.snapshot \\
model1.sync\_set\_py\_values\_to\_snapshot(snapshot=mysnapshot) \\
model2.sync\_set\_py\_values\_to\_snapshot(snapshot=mysnapshot) \\
model1.run\_session() \\
model2.run\_session()}
\end{codeblock}

The state of \vars{model} after its run session is used to start
\vars{model1} and \vars{model2}.  This is an easy way to save time
in spinning-up multiple models.




%---------------------------------------------------------------------
\section{Run Lists}		\label{sec:runlist.intro}

This feature of \class{Qtcm} objects is what really gives 
\class{Qtcm} model instances their flexibility.
A run list is a list of strings and dictionaries that specify
what routines to run in order to execute a particular part of
the model.  Each element of the run list specifies the method
or subroutine to execute, and the order of the elements specifies
their execution order.

For instance, the standard run list for initializing the the
atmospheric portion of the model is named ``qtcminit'', and
equals the following list:

\begin{latexonly}
\begin{codeblock}
\codeblockfont{%
\parbox{46ex}{% This file is automatically generated by
% code_to_latex.py.  It lists the code of the qtcminit
% runlist.

['\_\_qtcm.wrapcall.wparinit', \\
 '\_\_qtcm.wrapcall.wbndinit', \\
 'varinit', \\
 \{'\_\_qtcm.wrapcall.wtimemanager': [1]\}, \\
 'atm\_physics1']}}
\end{codeblock}
\end{latexonly}

\begin{htmlonly}
\begin{quotation}
% This file is automatically generated by
% code_to_latex.py.  It lists the code of the qtcminit
% runlist.

['\_\_qtcm.wrapcall.wparinit', \\
 '\_\_qtcm.wrapcall.wbndinit', \\
 'varinit', \\
 \{'\_\_qtcm.wrapcall.wtimemanager': [1]\}, \\
 'atm\_physics1']
\end{quotation}
\end{htmlonly}

This list is stored as an entry in the \vars{runlists} dictionary
(with key \vars{'qtcminit'}).
\vars{runlists} is an attribute of a \class{Qtcm} instance.
Table~\ref{tab:stnd.runlists} lists all standard run lists.

When the run list element in the list is a string, the string gives the
name of the routine to execute.  The routine has no parameter
list.  The routine can be a
compiled QTCM1 model subroutine for which an interface has been
written (e.g., \mods{\_\_qtcm.wrapcall.wparinit}), 
a method of the of the Python model instance 
(e.g., \mods{varinit}), or another run list
(e.g., \vars{atm\_physics1}).

When the run list element is a 1-element dictionary, the key of
the dictionary element is the name of the routine, and the value
of the dictionary element is a list specifying input parameters
to be passed to the routine on call.  Thus, the element:
\begin{codeblock}
\codeblockfont{%
{\{'\_\_qtcm.wrapcall.wtimemanager': [1]\}}}
\end{codeblock}
calls the \mods{\_\_qtcm.wrapcall.wtimemanager} routine, passing in
one input parameter, which in this case is the value 1.

If you want to change the order of the run list, just change the
order of the list.  To add or remove routines to be executed, just
add and remove their names from the run list.
Python provides a number of methods to manipulate
lists (e.g., \mods{append}).  Since lists are dynamic data types
in Python, you do not have to do any recompiling to implement
the change.

The \vars{compiled\_form} attribute must be set to \vars{'parts'}
in the \class{Qtcm} instance in order to take advantage of the run
lists feature of the class.  Run lists are not available for
\vars{compiled\_form\thinspace=\thinspace'full'}, because subroutine
calls are hardwired in the compiled QTCM1 model Fortran code in
that case.




%---------------------------------------------------------------------
\section{Model Output}			\label{sec:output.intro}

	\subsection{NetCDF Output}

Model output is written to netCDF files in the directory
specified by the \class{Qtcm} instance attribute \vars{outdir}.
Mean values are written to an output file beginning with
\fn{qm\_}, and instantaneous values are written to an output
file beginning with \fn{qi\_}.

The frequency of mean output is controlled by \vars{ntout}, and the
frequency of instantaneous output is controlled by \vars{ntouti}.
\vars{ntout.value} gives the number of days over which to average
(and if equals \vars{-30}, monthly means are calculated).
\vars{ntouti.value} gives the frequency in days that instantaneous
values are output (monthly if it equals \vars{-30}).  (See
Section~\ref{sec:initial.variables} for a description of other
output-control variables, and see the QTCM1 manual \cite{Neelin/etal:2002}
for a detailed description of how these variables control output.)

Figure~\ref{fig:netcdf.read} gives an example of a block of code
to read netCDF output, where \vars{datafn} is the netCDF filename, and
\vars{id} is the string name of the field variable (e.g.,
\vars{'u1'}, \vars{'T1'}, etc.).
(Note that the netCDF identifier for field variables is the same as
the name in \class{Qtcm}, except for the variables given in
Table~\ref{tab:qtcm.netcdf.ids}.)

In the code in Figure~\ref{fig:netcdf.read},
the array value is read into \vars{data}, and the longitude values, 
latitude values, and time values are read into variables
\vars{lon}, \vars{lat}, and \vars{time}, respectively.
As netCDF files also hold metadata, a description and the units
of the variable given by \vars{id}, and each dimension, are read
into variables ending in \vars{\_name} and \vars{\_units},
respectively.


%--- Two versions, one for PDF, one for HTML:
\begin{latexonly}
\begin{figure}[htp]
\begin{codeblock}
\codeblockfont{%
import numpy as N \\
import Scientific as S \\ \\
fileobj = S.NetCDFFile(datafn, mode='r') \\ \\
data = N.array(fileobj.variables[id].getValue()) \\
data\_name = fileobj.variables[id].long\_name \\
data\_units = fileobj.variables[id].units \\ \\
lat = N.array(fileobj.variables['lat'].getValue()) \\
lat\_name = fileobj.variables['lat'].long\_name \\
lat\_units = fileobj.variables['lat'].units \\ \\
lon = N.array(fileobj.variables['lon'].getValue()) \\
lon\_name = fileobj.variables['lon'].long\_name \\
lon\_units = fileobj.variables['lon'].units \\ \\
time = N.array(fileobj.variables['time'].getValue()) \\
time\_name = fileobj.variables['time'].long\_name \\
time\_units = fileobj.variables['time'].units \\ \\
fileobj.close()}
\end{codeblock}

\caption{Example of Python code to read netCDF output.
	See text for description.}
\label{fig:netcdf.read}
\end{figure}
\end{latexonly}

\begin{htmlonly}
\label{fig:netcdf.read}
\begin{center}
\htmlfigcaption{%
	\codeblockfont{%
import numpy as N \\
import Scientific as S \\ \\
fileobj = S.NetCDFFile(datafn, mode='r') \\ \\
data = N.array(fileobj.variables[id].getValue()) \\
data\_name = fileobj.variables[id].long\_name \\
data\_units = fileobj.variables[id].units \\ \\
lat = N.array(fileobj.variables['lat'].getValue()) \\
lat\_name = fileobj.variables['lat'].long\_name \\
lat\_units = fileobj.variables['lat'].units \\ \\
lon = N.array(fileobj.variables['lon'].getValue()) \\
lon\_name = fileobj.variables['lon'].long\_name \\
lon\_units = fileobj.variables['lon'].units \\ \\
time = N.array(fileobj.variables['time'].getValue()) \\
time\_name = fileobj.variables['time'].long\_name \\
time\_units = fileobj.variables['time'].units \\ \\
fileobj.close()}
	}

\htmlfigcaption{Figure~\ref{fig:netcdf.read}:
	Example of Python code to read netCDF output.
	See text for description.}
\end{center}
\end{htmlonly}





\begin{table}[tp]
\begin{center}
\begin{tabular}{l|l}
\textbf{\class{Qtcm} Attribute Name} & \textbf{NetCDF Output Name} \\
\hline
\vars{'Qc'}                & \vars{'Prec'} \\
\vars{'FLWut'}             & \vars{'OLR'} \\
\vars{'STYPE'}             & \vars{'stype'}
\end{tabular}
\end{center}
\caption{NetCDF output names for \class{Qtcm} field variables that
	are different from the \class{Qtcm} and compiled QTCM1 model
	variable names.  The netCDF names are case-sensitive.}
\label{tab:qtcm.netcdf.ids}
\end{table}


\emphpara{NB:}  All netCDF array output is dimensioned (time, latitude,
longitude) when read into Python using the \mods{Scientific} package.
This differs from the way \class{Qtcm} saves field variables, which
follows Fortran convention (longitude, latitude).  Please be careful
when relating the two types of arrays.
Section~\ref{sec:field.var.shape} for a discussion of why there is
this discrepancy.


	\subsection{Visualization}	\label{sec:viz.intro}

The \mods{plotm} method of \class{Qtcm} instances creates line
plots or contour plots, as appropriate, of model output of
average fields of run session(s) associated with the instance.
Some examples, assuming \vars{model} is an instance of \class{Qtcm}
and has already executed a run session:
\begin{itemize}
\item \cmd{model.plotm('Qc', lat=1.875)}:
	A time vs.\ longitude contour
          plot is made for the full range of time and longitude,
          at the latitude 1.875 deg N, for mean precipitation.

\item \cmd{model.plotm('Qc', time=10)}:
	A latitude vs.\ longitude contour plot of precipitation
	is made for the full spatial domain at day 10 of the model run.

\item \cmd{model.plotm('Evap', lat=1.875, lon=[100,200])}:  A contour
	plot of time vs.\ longitude of evaporation is made for the
          longitude points between 100 and 200 degrees E, at the
          latitude 1.875 deg N.  

\item \cmd{model.plotm('cl1', lat=1.875, lon=[100,200], time=20)}:
          A deep cloud amount vs.\ longitude line plot is made for
          the longitude points between 100 and 200 degrees east,
          at the latitude 1.875 deg N, at day 20 of the model run.
\end{itemize}

In these examples, the number of days over which the mean is taken
equals \vars{model.ntout.value}.
Also, the \mods{plotm} method automatically takes into account the
\class{Qtcm}/netCDF variable differences described in
Table~\ref{tab:qtcm.netcdf.ids}.



%---------------------------------------------------------------------
\section{Documentation}

Section~\ref{sec:ver} gives the online locations of the
transparent copies of this manual.  
Model formulation is fully described in
Neelin \& Zeng \cite{Neelin/Zeng:2000} and model
results are described in Zeng et~al.\ \cite{Zeng/etal:2000}
(\cite{Neelin/Zeng:2000} is based upon v2.0 of QTCM1
and \cite{Zeng/etal:2000} is based on QTCM1 v2.1).
Additional documentation you'll find useful include:

\begin{itemize}
\item \latexhtml{%
\htmladdnormallinkfoot{The \mods{qtcm} Package API Documentation}%
        {http://www.johnny-lin.com/py\_pkgs/qtcm/doc/html-api/}}%
{\htmladdnormallink{The \mods{qtcm} Package API Documentation}%
        {http://www.johnny-lin.com/py_pkgs/qtcm/doc/html-api/}}

\item \latexhtml{%
\htmladdnormallinkfoot{The Pure-Fortran QTCM1 Manual}%
        {http://www.atmos.ucla.edu/$\sim$csi/qtcm\_man/v2.3/qtcm\_manv2.3.pdf}}%
{\htmladdnormallink{The Pure-Fortran QTCM1 Manual}%
        {http://www.atmos.ucla.edu/~csi/qtcm_man/v2.3/qtcm_manv2.3.pdf}}
\cite{Neelin/etal:2002}

\end{itemize}



% ===== end of file =====
