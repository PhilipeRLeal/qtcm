% ==========================================================================
% Using QTCM
%
% By Johnny Lin
% ==========================================================================


% ------ BODY -----
%

%---------------------------------------------------------------------
\section{Introduction}

This section describes programming practices and issues related to
the \mods{qtcm} package that might be of interest to users wishing
to add/change code in the package.
Please see the package
\latexhtml{API documentation,%
		\footnote{http://www.johnny-lin.com/py\_pkgs/qtcm/doc/html-api/}
		which includes the source code}%
        {\htmladdnormallink{API documentation}%
		{http://www.johnny-lin.com/py\_pkgs/qtcm/doc/html-api/},
		which includes the source code},
for details.




%---------------------------------------------------------------------
\section{Changes to QTCM1 Fortran Files}  \label{sec:f90changes}

The source code used to generate the shared object files used
in this Python package is unchanged
from the pure-Fortran QTCM1 model source code, except in the
following ways:

\begin{itemize}
\item The suffix of all source code files 
	has been changed from \fn{.f90} to \fn{.F90}, 
	in order to ensure the compiler preprocesses 
	the source code.  Some compilers use the capitalization to
	tell whether or not to run the code through a preprocessor.

\item In all \fn{.F90} files, occurrences of:
	\begin{codeblock}
	\codeblockfont{%
	Character(len=130)}
	\end{codeblock}
	are changed to:
	\begin{codeblock}
	\codeblockfont{%
	Character(len=305)}
	\end{codeblock}
	This enables the model to properly deal with longer filenames.
	The number ``305'' is chosen to make search and replace easier.

\item In \fn{qtcmpar.F90}, the 
	\vars{eps\_c} variable is changed from an unchangable
	parameter to a changeable real, 
	so that it can be changed in the model at runtime.

\item All occurrences of an underscore (``\_'') character in a
	subroutine or function name are removed.  The
	presence of the underscore messes up the dynamic lookup
	mechanism for the \mods{f2py} generated extension module.
	The following names are changed, both in subroutine definitions
	and calls:
	\begin{itemize}
	\item \mods{out\_restart} to \mods{outrestart},
	\item \mods{save\_bartr} to \mods{savebartr},
	\item \mods{grad\_phis} to \mods{gradphis}.
	\end{itemize}

\item \fn{driver.F90} is changed so that program
	\mods{driver} becomes a subroutine, and 
	subroutine \mods{driverinit} is deleted (along with
	all calls to it) because basic model initialization is
	handled at the Python level.

\item In \fn{clrad.F90}, subroutine \mods{cloud}, the first
	\vars{COUNTCAP} preprocessor macro, a comment line for
	that ifdef is moved to prevent a warning message during
	building with \mods{f2py}.

\item The order of subroutine \mods{qtcminit} is changed.  The original
	pure-Fortran QTCM1 \mods{qtcminit} code has the following
	calling sequence:

	\begin{codeblock}
        \codeblockfont{%
Call parinit            !Initialize model parameters \\
Call varinit            !Initialize variables \\
Call TimeManager(1)     !mm set model time \\
Call bndinit            !input boundary datasets \\
Call physics1           !diagnostic fields for initial condition}
	\end{codeblock}

	For the \mods{qtcm} package, I've altered this order so
	\mods{bndinit} comes after \mods{parinit} but before \mods{varinit}:
	\begin{codeblock}
        \codeblockfont{%
Call parinit            !Initialize model parameters \\
Call bndinit            !input boundary datasets \\
Call varinit            !Initialize variables \\
Call TimeManager(1)     !mm set model time  \\
Call physics1           !diagnostic fields for initial condition}
	\end{codeblock}

	This is done because \vars{STYPE} is not read in for the
	\vars{landon} \vars{True} case until \mods{bndinit}, but
	in \mods{varinit} \vars{STYPE} is used to calculate the
	original values of \vars{WD} for the non-restart case.  This
	also corrects the conflicting order found in the pure-Fortran
	QTCM1 manual (compare pp.\ 29 and 32).  As far as I can
	tell, \mods{bndinit} has no dependencies that require it
	to come after \mods{timemanager} or \mods{varinit}.

\end{itemize}

In addition, the Fortran files \fn{setbypy.F90}, \fn{wrapcall.F90},
and \fn{varptrinit.F90} are added.  The routines in these files, 
however, just add more flexibility and functionality to the model;
they do not automatically affect any model computations.  See
Section~\ref{sec:newf90} for details.




%---------------------------------------------------------------------
\section{New Interfaces and Fortran Functionality}  \label{sec:newf90}

As described in Section~\ref{sec:f90changes}, the Fortran files
\fn{setbypy.F90}, \fn{wrapcall.F90}, and \fn{varptrinit.F90} are
added to the QTCM1 source directory.  The first two files define the Fortran
90 modules (\mods{SetbyPy} and \mods{WrapCall}) needed to interface
the Python and Fortran levels.  The last file defines a new Fortran
subroutine \mods{varptrinit} that associates QTCM1 model pointer
variables at the Fortran level.  In a pure-Fortran run of QTCM1,
this occurs in subroutine \mods{varinit}; for a
\vars{compiled\_form\thinspace=\thinspace'parts'} run, since the
functionality of the Fortran \mods{varinit} is now in the Python
\mods{varinit} method, a separate Fortran pointer association
subroutine needed to be defined.  The Fortran subroutine \mods{varptrinit}
is called as the \mods{varptrinit} function of the 
\vars{compiled\_form\thinspace=\thinspace'parts'}
\fn{.so} extension module.


	\subsection{Fortran Module \mods{SetbyPy}}   \label{sec:setbypy}

		\subsubsection{Design Description}

This module defines functions and subroutines used to read variables
from the Fortran-level to the Python-level, and in setting Fortran-level
variables using the Python-level values.  These routines are used
by \class{Qtcm} methods \mods{get\_qtcm1\_item} and \mods{set\_qtcm1\_item}
(and dependencies thereof) to ``get'' and ``set'' the Fortran-level
variables.  Note that the Fortran module \mods{SetbyPy} is referred
to in lowercase at the Python level, i.e., as the
attribute \vars{\_\_.qtcm.setbypy} of a \class{Qtcm} instance.

Because Fortran variables are not dynamically typed, separate Fortran
functions and subroutines need to be defined to get and set variables
of different types.\footnote%
	{The \mods{interface} construct in Fortran 90 is supposed to
	provide a way to create a single interface to multiple
	routines, e.g.:
	\begin{codeblock}
	\codeblockfont{%
Interface setitem \\
\hspace*{3ex}Module Procedure setitem\_real, setitem\_int, setitem\_str \\
End Interface}
	\end{codeblock}
	This construct, however, causes a bus error
	(Mac OS X 10.4, Intel).  Thus, I put the
	same functionality in the Python code.}
The \class{Qtcm} methods \mods{get\_qtcm1\_item}
and \mods{set\_qtcm1\_item} know which one of the Fortran routines
to call on the basis of the type and rank of the value for the field
variable in the \mods{defaults} submodule.  This is why all field
variables need to have defaults defined in \mods{defaults}.  For
array variables, the field variable defaults also provide the rank
of the Fortran-level variable being gotten or set.  However, the
array default values do \emph{not} have to have the same shape as
the Fortran-level variables; on the Python-side, variable shape
adjusts to what is declared on the Fortran-side.  
Thus, if you change the resolution of
the compiled QTCM1 model, you do not have to change the dimensions
of the field variable values in \mods{defaults}.

The \class{Qtcm} method \mods{get\_qtcm1\_item} directly calls
the \mods{SetByPy} routines.
The \class{Qtcm} method \mods{set\_qtcm1\_item} makes use of
private instance methods that make the calls to the \mods{SetByPy} routines.

For scalar field variables, \mods{SetByPy} provides functions and
subroutines that provide the value of the variable on output.
For array field variables, \mods{SetByPy}
dynamic \emph{module} arrays are used to pass array
variables in and out; I could not get the 
\mods{SetByPy} Fortran routines to set
locally defined dynamic arrays (that is, locally within a function or
subroutine).\footnote%
	{I tried to implement Fortran subroutine
	\mods{getitem\_real\_array} using traditional array 
	dimension passing 
	(e.g., \code{subroutine foo(nx, ny, a)}) as well
	as declaring the allocatable array inside the subroutine, 
	but neither option worked on my \mods{f2py} (version 2\_3816) 
	and Python (version 2.4.3).}
In the \mods{SetByPy} module, these dynamic arrays
are defined as follows:

\begin{codeblock}
\codeblockfont{%
Real, allocatable, dimension(:) :: real\_rank1\_array \\
Real, allocatable, dimension(:,:) :: real\_rank2\_array \\
Real, allocatable, dimension(:,:,:) :: real\_rank3\_array}
\end{codeblock}

For all field variables, scalar or array, the \mods{SetByPy} module
has a fourth module variable defined, \vars{is\_readable}, that the
Fortran get and set routines will set to \vars{.TRUE.} if the
variable is readable and \vars{.FALSE.} if not (it's declared as a
logical variable).  This Fortran variable can be used to prevent
Python from accessing pointer variables that aren't yet associated
to targets.

In general, \mods{SetByPy} routines make use of Fortran constructs
to enable them to accomodate all possible
variables of a given type and shape.  However, 
for string scalars, the \mods{SetByPy} function \mods{getitem\_str}
has to have a return value of a predefined length, in order to
work properly.  That length is given by the parameter
\vars{maxitemlen} and is set to 505 (the value is chosen to
be larger than all filename variables described in
Section~\ref{sec:f90changes} and to be easily found in
the \fn{.F90} files).


		\subsubsection{Module Structure}

If you're a Fortran programmer, you can probably get all the information
in this section from just reading the \fn{setbypy.F90} file directly.
This description of the module structure, however, permits me to highlight
what you need to do if you want to make additional compiled QTCM1 variables
accessible to Python \class{Qtcm} objects.

\begin{itemize}
\item All \mods{Use} statements are given in the beginning of 
	the \mods{SetByPy} module.  These statements cover
	nearly all of the QTCM1 Fortran
	modules that contain variables of interest.  If the
	QTCM1 variable you're interested in isn't in a module
	listed here, you'll have to add your own
	\mods{Use} statement of that module here.

\item Next comes the definitions for the
	\vars{real\_rank1\_array},
	\vars{real\_rank2\_array}, and
	\vars{real\_rank3\_array} dynamic array variables, and
	the \vars{is\_readable} boolean variable.

\item The \mods{Contains} block of the module defines the module
	routines called by the \class{Qtcm} instance methods to
	set and get the compiled QTCM1 model variables.  The
	routines are:
	\begin{itemize}
	\item Function \mods{getitem\_real}
	\item Subroutine \mods{getitem\_real\_array}
	\item Function \mods{getitem\_int}
	\item Function \mods{getitem\_str}
	\item Subroutine \mods{setitem\_real}
	\item Subroutine \mods{setitem\_real\_array}
	\item Subroutine \mods{setitem\_int}
	\item Subroutine \mods{setitem\_str}
	\end{itemize}

\end{itemize}

Each of the routines in the module \mods{Contains} block is essentially
a list of \mods{if}/\mods{elseif} statements.  The list tests for the
name of the variable of interest (a string), and gets or sets the
compiled QTCM1 model variable corresponding to that name.  For pointer
array variables, a test is also made as to whether or not the variable
has been associated.  If not, the variable is not readable
and \vars{is\_readable} is set to \vars{.FALSE.}\ accordingly.

If you wish to add another compiled QTCM1 model variable to be
accessible to \class{Qtcm} instance methods \mods{get\_qtcm1\_item}
and \mods{set\_qtcm1\_item}, just add another \mods{if}/\mods{else\-if},
like the other \mods{if}/\mods{elseif} blocks, in the Fortran set
and get routines corresponding to the QTCM1 variable type (scalar
vs.\ array, and real, integer, or string).  On the Python side, add
an entry in \mods{defaults} corresponding to the new field variable
you've created access to.  I would strongly recommend making the
Python name of your new field variable
(given in \mods{defaults}) be the same as the compiled
QTCM1 model variable name.



	\subsection{Fortran Module \mods{WrapCall}}   \label{sec:wrapcall}

Most of the time, if you want to call a compiled QTCM1 model subroutine
from the Python level, you will use the version of the subroutine that
is found in this Fortran module.  
Note that the Fortran module \mods{WrapCall} is referred
to in lowercase at the Python level, i.e., as the
attribute \vars{\_\_.qtcm.wrapcall} of a \class{Qtcm} instance.

All the routines in this module do is wrap one of the compiled QTCM1
model routines.  For instance, \mods{WrapCall} subroutine
\mods{wadvcttq} is defined as just:

% --- Two versions of this block, one for display in PDF and the other
%     for display in HTML:
%
\begin{latexonly}
\begin{codeblock}
\codeblockfont{%
Subroutine wadvcttq \\
\hspace*{3ex}Call advcttq \\
End Subroutine wadvcttq}
\end{codeblock}
\end{latexonly}

\begin{htmlonly}
\begin{rawhtml}
<p><code><font color="blue">Subroutine wadvcttq<br>
&nbsp;&nbsp;&nbsp;Call advcttq<br>
End Subroutine wadvcttq</font></code></p>
\end{rawhtml}
\end{htmlonly}

All subroutines in this module begin with ``w'', with the rest of
the name being the Fortran QTCM1 subroutine name.  The calling
interface for the ``w'' version is the same as the Fortran QTCM1
original version.  There are no subroutines in this module that do
not have an exact counterpart in the Fortran QTCM1 code, and thus
this module's subroutines sole purpose is to call other subroutines
in the compiled QTCM1 model.

These wrapper routines are needed because \mods{f2py}, for some
reason I can't figure out, will not properly wrap Fortran routines
(that are then callable at the Python level) that create local
arrays using parameters obtained through a Fortran \mods{use}
statment.  Thus, as an example, a Fortran subroutine \mods{foo}
with the following definition:

% --- Two versions of this block, one for display in PDF and the other
%     for display in HTML:
%
\begin{latexonly}
\begin{codeblock}
\codeblockfont{%
subroutine foo \\
\hspace*{3ex}use dimensions \\
\hspace*{3ex}real a(nx,ny) \\
\hspace*{3ex}[\ldots] \\
end subroutine foo}
\end{codeblock}
\end{latexonly}

\begin{htmlonly}
\begin{rawhtml}
<p><code><font color="blue">
subroutine foo<br>
&nbsp;&nbsp;&nbsp;use dimensions<br>
&nbsp;&nbsp;&nbsp;real a(nx,ny)<br>
&nbsp;&nbsp;&nbsp;[\ldots]<br>
end subroutine foo
</font></code></p>
\end{rawhtml}
\end{htmlonly}


where \vars{nx} and \vars{ny} are defined in the module vars{dimensions},
will return an error, with the result that the extension module
will not be created, or an extension modules that yields output
that is different from running the pure-Fortran version of QTCM1.

By wrapping these calls into this file, I also avoid having to
separate out the Fortran QTCM1 subroutines into separate \fn{.F90}
files.  For Fortran subroutines that you want callable from the
Python level, \mods{f2py} seems to require each Fortran subroutine
to be in its own file of the same name (e.g., the version of
\fn{driver.F90} for this package). If several Fortran subroutines
are all found in a single \fn{.F90} files, \mods{f2py} seems unable
to create wrappers for those subroutines.




%---------------------------------------------------------------------
\section{Python \mods{qtcm} and Pure-Fortran QTCM1 Differences}

This section describes differences between how the \mods{qtcm}
package and the pure-Fortran QTCM1 assign some varables.  A list
of changes to the QTCM1 Fortran Files for use in the \mods{qtcm}
package is found in Section~\ref{sec:f90changes}.


	\subsection{QTCM1 \mods{driverinit}}   \label{sec:driverinit.diffs}

In the pure-Fortran version of QTCM1, by default, the following variables are
set by reference (as given below), not by value, in the \mods{driverinit}
routine:\footnote%
	{In the pure-Fortran version of QTCM1, this routine is found
	in \fn{driver.F90}.}
\begin{codeblock}
\codeblockfont{%
lastday\thinspace=\thinspace{daysperyear} \\
viscxu0\thinspace=\thinspace{viscU} \\
viscyu0\thinspace=\thinspace{viscU} \\
visc4x\thinspace=\thinspace{viscU} \\
visc4y\thinspace=\thinspace{viscU} \\
viscxu1\thinspace=\thinspace{viscU} \\
viscyu1\thinspace=\thinspace{viscU} \\
viscxT\thinspace=\thinspace{viscT} \\
viscyT\thinspace=\thinspace{viscT} \\
viscxq\thinspace=\thinspace{viscQ} \\
viscyq\thinspace=\thinspace{viscQ}}
\end{codeblock}

Thus, in pure-Fortran QTCM1, if you change \vars{daysperyear},
\vars{viscU}, etc.
and recompile (as needed), you will automatically change 
\vars{lastday}, \vars{viscxu0}, etc.
(Though, in the pure-Fortran QTCM1, the default values may be overwritten by
namelist input values.)

The \mods{driverinit} routine is eliminated
in the Python \code{qtcm} package.  Instead, inital values 
of field variables are specified in the \mods{defaults} submodule
and set by value to attributes of the \code{Qtcm} instance.
Thus, for instance, in a \class{Qtcm} instance, \code{lastday} 
is set to \code{365} by default, not to some variable
\vars{daysperyear}.  For the diffusion and viscosity terms,
the \class{Qtcm} instance attributes corresponding to those
terms are set to literals.\footnote%
	{Those literals are defined by \mods{defaults} private
	module variables \vars{\_\_viscT}, \vars{\_\_viscQ},
	and \vars{\_\_viscU}.}

In contrast, in the pure-Fortran QTCM1,
\mods{driverinit} declares local
variables \code{viscU}, \code{viscT}, and \code{viscQ},
and reads values into those variables via the input namelist.
Those values are then used to set
\vars{viscxu0}, \vars{viscyu0}, etc., as described above.
In pure-Fortran QTCM1, \code{viscU}, \code{viscT}, and \code{viscQ}
are not directly accessed anywhere else in the model.
Thus, \code{viscU}, \code{viscT}, and \code{viscQ} are not
defined as field variables in the \code{qtcm} package, and
\class{Qtcm} instances do not have attributes corresponding
to those names.
Additionally, if you wish to change a viscosity parameter
\vars{visc*} (given above), the parameter for each direction
must be set one-by-one even if the flow is isotropic.


	\subsection{The \mods{varinit} Routine}

One of the functions of the pure-Fortran QTCM1 \mods{varinit}
subroutine is to associate the pointer variables \vars{u1}, \vars{v1},
\vars{q1}, and \vars{T1}.  For the extension modules in the \mods{qtcm}
package, a Fortran subroutine \mods{varptrinit} is added that can
also do this association.  This subroutine is called in the
\class{Qtcm} instance method
\latexhtml{\mods{varinit}%
		\footnote{http://www.johnny-lin.com/py\_docs/qtcm/doc/html-api/qtcm.qtcm.Qtcm-class.html\#varinit}}%
	{\htmladdnormallink{\mods{varinit}}{http://www.johnny-lin.com/py_docs/qtcm/doc/html-api/qtcm.qtcm.Qtcm-class.html#varinit}}
(which duplicates and
extends the function of its pure-Fortran counterpart, enabling
alternative ways of handling restart).

The \mods{varptrinit} is not accessed via \mods{wrapcall}.  Remember
that \mods{wrapcall} contains only those routines that were in the
original pure-Fortran QTCM1 code, and that we want to have access
to at the Python level.


	\subsection{The \mods{qtcm} Method of \class{Qtcm}}

The \class{Qtcm} method \mods{qtcm} duplicates the functionality
of the \mods{qtcm} subroutine in the pure-Fortran QTCM1 model.
There are a few differences, however.  First, the \mods{qtcm} method
for \class{Qtcm} instances does not include a call to \mods{cplmean},
which uses mean surface flux for air-sea coupling.  This state is
consistent with the pure-Fortran QTCM1 pre-processor macro
\vars{CPLMEAN} being off.  Thus, if you wish to use mean surface
flux for air-sea coupling, you will have to revise the \mods{qtcm}
method of \class{Qtcm} to call \mods{cplmean}.  You'll also have to
check for any other code additions needed that are associated with
the \vars{CPLMEAN} macro.

Second, the \mods{qtcm} method for \class{Qtcm} instances does not
include the option of not using the atmospheric boundary layer
model.  This is consistent with macro \vars{NO\_ABL} being off.  If
you wish to have no atmospheric boundary layer model, change the
run list \vars{atm\_bartr\_mode} so that the \mods{wsavebartr} and
\mods{wgradphis} routines are not called.  You'll also have to check
for any other code additions needed that are associated with the
\vars{NO\_ABL} macro.



	\subsection{Miscellaneous Differences}

\begin{itemize}
\item In Python \class{Qtcm} instances,
	\vars{dateofmodel} is set to 0 by default.  
	In contrast, in the compiled QTCM1 model,
	the default (i.e., initial value) is calculated from 
	\vars{day0}, \vars{month0}, and \vars{year0}.
	See Section~\ref{sec:init.compiledform.full} for details.

\item The \class{Qtcm} instance attribute
	\vars{\_\_qtcm} is not copyable using \mods{copy.deepcopy}.

\item In general, when executing a \class{Qtcm} instance method, 
	if you change a \class{Qtcm} instance attribute 
	that has a counterpart in the compiled QTCM1 model,
	the compiled QTCM1 counterpart is not changed until the
	end of the method.  Likewise, if you call a compiled QTCM1 model
	subroutine and change a compiled QTCM1 model variable with
	a \class{Qtcm} instance counterpart, the \class{Qtcm}
	instance counterpart is not changed until the end of the
	subroutine.

\item In general, even though some of the compiled QTCM1 model
	Fortran subroutines/functions have counterparts in \class{Qtcm}
	that duplicate the former's functionality, the Fortran
	versions are kept intact so that the
	\vars{compiled\_form\thinspace=\thinspace'full'} case will work.
\end{itemize}




%---------------------------------------------------------------------
\section{Considerations When Adding Fortran Code}

In this section I describe issues to consider if you wish to add
your own compiled code to the package as separate extension modules.
(This is different from creating new standard extension modules,
which is described in Section~\ref{sec:create.new.so}.):

\begin{itemize}
\item The \class{Qtcm} class assumes that the directory path 
	to the original shared object file is the same as for the 
	\mods{package\_version} module.

\item If you want to be able to pass other Fortran variables 
	in and out to/from Python, please see the 
	Section~\ref{sec:setbypy}
	discussion of the Fotran \mods{SetByPy} module.

\item Fortran and Python routines to get and set compiled QTCM1 model
	arrays are currently written only for floating point array.

\item If you ever change 
	\class{Qtcm} instance method
	\mods{\_set\_qtcm\_array\_item\_in\_model}
	to work with non-floating point values, you will also
	have to change the array handling section in 
	\mods{set\_qtcm1\_item}.

\item The restart mechanism in the pure-Fortran QTCM1 model is 
	\emph{not} bit-for-bit correct.  Thus, if you compare the final
	output from a 40 day run with a 30 day run restarted from
	a 10 day run, the output will not be the same.
	This behavior has been duplicated in \class{Qtcm} 
	instances when the \vars{mrestart} flag is used
	and applicable.

\item When creating new extension modules using the \fn{src} makefile,
	be sure you first use the \cmd{make clean} command to clean-up
	any old files.

\end{itemize}




%---------------------------------------------------------------------
\section{Creating New Standard Extension Modules}   \label{sec:create.new.so}

The steps involved in creating the standard extension modules (e.g.,
\fn{\_qtcm\_full\_365.so}, etc.) on installation are given in
Section~\ref{sec:create.so}.  The makefile provided in \fn{/buildpath/src}
uses a Fortran compiler to create the object code, runs \mods{f2py}
to create the shared object file in \fn{src}, and moves the shared
object files into \fn{../lib}, overwriting any pre-existing files
of the same name.  In this section, I describe the makefile and
\mods{f2py} in a little more detail, in case you wish to create
standard extension modules with additions from the ones the default
makefile creates.


	\subsection{Makefile Rules}    \label{sec:makefile.rules}

This section describes the rules of the
makefile found in the \fn{src} directory
of the \mods{qtcm} distribution.  
This makefile is used by the Python package to create the extension
module (\fn{.so} files) imported and used by \mods{qtcm} objects
(as described in Section~\ref{sec:create.so}).
The makefile will, in general, be used only during \mods{qtcm}
installation, but if you wish to recompile the QTCM1 libraries
and make changes in the Python extension module,
you'll want to use/change this makefile.

\begin{description}
\item[clean] Removes old files in preparation for compiling new
	extension modules.

\item[libqtcm.a] Creates library \fn{libqtcm.a} that contains all
	QTCM1 object files in the directory \fn{src},, except
	\fn{setbypy.o}, \fn{wrapcall.o}, \fn{varptrinit.o}, and
	\fn{driver.o}.  This archive is compiled with the netCDF
	libraries.  Previous versions of \fn{libqtcm.a} are overwritten.

\item[\_qtcm\_full\_365.so] Creates the extension module
	\fn{\_qtcm\_full\_365.so}.  \mods{f2py} is run on applicable code
	in \fn{src}, and the extension module is moved to \fn{../lib}.
	Any previous versions of \fn{../lib/\_qtcm\_full\_365.so}
	are overwritten.

\item[\_qtcm\_parts\_365.so] Creates the extension module
	\fn{\_qtcm\_parts\_365.so}.  \mods{f2py} is run on applicable code
	in \fn{src}, and the extension module is moved to \fn{../lib}.
	Any previous versions of \fn{../lib/\_qtcm\_parts\_365.so}
	are overwritten.

\end{description}



	\subsection{Using \mods{f2py}}      \label{sec:using.f2py}

This section briefly describes how \mods{f2py} is used in the
makefile during the creation of the extension modules.
\htmladdnormallink{\mods{F2py}}{http://cens.ioc.ee/projects/f2py2e/} is a
program that generates shared object libraries that allow you to call
Fortran routines in Python.  \mods{F2py} comes with Python's
\htmladdnormallink{NumPy}{http://numpy.scipy.org/}
array handling package, so you do not need to install anything
extra if you have NumPy already installed.

To create the extension modules in \mods{qtcm} using
the makefile described in Section~\ref{sec:makefile.rules},
I use a method similar to the
\latexhtml{``Quick and Smart Way,''\footnote%
{http://cens.ioc.ee/projects/f2py2e/usersguide/index.html\#the-quick-and-smart-way}}%
{\htmladdnormallink{``Quick and Smart Way''}%
{http://cens.ioc.ee/projects/f2py2e/usersguide/index.html#the-quick-and-smart-way}}
described in the \mods{f2py} manual.
For the \fn{\_qtcm\_full\_365.so} extension module, the 
\mods{f2py} call is:

\begin{codeblock}
\codeblockfont{%
f2py --fcompiler=\$(FC) -c -m \_qtcm\_full\_365 driver.F90 $\backslash$ \\
\hspace*{10ex}setbypy.F90 libqtcm.a \$(NCLIB)}
\end{codeblock}

and for the \fn{\_qtcm\_parts\_365.so} extension module, the call is:

\begin{codeblock}
\codeblockfont{%
f2py --fcompiler=\$(FC) -c -m \_qtcm\_parts\_365 $\backslash$ \\
\hspace*{10ex}varptrinit.F90 wrapcall.F90 setbypy.F90 $\backslash$ \\
\hspace*{10ex}libqtcm.a \$(NCLIB)}
\end{codeblock}

For both calls, \vars{FC} and \vars{NCLIB} are the environment
variables in the makefile specifying the Fortran compiler and netCDF
libraries, respectively.  The \vars{-m} flag specifies the extension
module name (without the \fn{.so} suffix).  The \fn{.F90} files
specify the files that have modules and routines that will be
accessible at the extension module level, and the rest of the Fortran
files in QTCM1 are compiled and archived in a library \fn{libqtcm.a}.
For \mods{f2py} to work properly,
the \fn{.F90} files may define \emph{only one} module or routine.

If you add Fortran files containing new modules, and you wish those
modules to be accessible at the Python level, compile your new code
with \mods{f2py}.  If we have a file of such new code, \fn{newcode.F90},
the \mods{f2py} call to create the \fn{\_qtcm\_parts\_365.so}
extension module will become:

\begin{codeblock}
\codeblockfont{%
f2py --fcompiler=\$(FC) -c -m \_qtcm\_parts\_365 $\backslash$ \\
\hspace*{10ex}varptrinit.F90 wrapcall.F90 setbypy.F90 $\backslash$ \\
\hspace*{10ex}newcode.F90 $\backslash$ \\
\hspace*{10ex}libqtcm.a \$(NCLIB)}
\end{codeblock}

If you write new Fortran code for the compiled QTCM1 model that
will \emph{not} be accessed from the Python-level, just add the
object code filename to the variable \vars{QTCMOBJS} in the
makefile; you don't have to do anything else.  If you are adding
Fortran code to existing Fortran modules, it's even easier:  You
don't need change the makefile.  Note that for 64 bit processor
machines, you may have to use \mods{f2py} with the \cmd{-fPIC} flag;
see Section~\ref{sec:sopic} for details on how the lines above will
change.


	\subsection{Two Examples}

\emphpara{A Function:}
Let's say you have written a piece of Fortran code called
\fn{myfunction.F90} that contains one function called
\mods{myfunction}, and you want to have this function
callable from the Python level through the \class{Qtcm} 
instance method \mods{\_\_qtcm.myfunction}.  Do the following:

\begin{enumerate}
\item Move \fn{myfunction.F90} to \fn{src} in the \mods{qtcm}
	distribution directory \fn{/buildpath}.

\item Add \cmd{myfunction.o} to the end of the object file list lines
	after the target names
	\vars{\_qtcm\_full\_365.so} and
	\vars{\_qtcm\_parts\_365.so}.

\item In the
	\vars{\_qtcm\_full\_365.so} and
	\vars{\_qtcm\_parts\_365.so} target descriptions,
	add \cmd{myfunction.F90} to the 
	beginning of the list of \fn{.F90} names 
	in the \mods{f2py} lines.
\end{enumerate}


\emphpara{A Module:} 
Let's say you have written a piece of Fortran code called
\fn{mymodule.F90} that contains the Fortran module \mods{MyModule}
containing multiple routines and variables.  You want to have those
routines and variables callable from the Python level through the
\class{Qtcm} instance attribute \mods{\_\_qtcm.mymodule}.  The steps
to add \mods{MyModule} to the extension modules are exactly the
same as for a single function, with \cmd{mymodule} being
substituted in the makefile everywhere you have \cmd{myfunction}.




%---------------------------------------------------------------------
\section{Attributes and Methods in \class{Qtcm} Instances}

In this section I describe some attributes, particularly private ones,
that may be of interest to developers.
As is the convention in Python, private
attributes and methods are prepended by one or two underscores,
with two underscores being the ``more'' private attribute.
Please see the package
\latexhtml{API documentation%
		\footnote{http://www.johnny-lin.com/py\_pkgs/qtcm/doc/html-api/}}
        {\htmladdnormallink{API documentation}%
		{http://www.johnny-lin.com/py\_pkgs/qtcm/doc/html-api/}}
for details about all variables, including private variables.


	\subsection{Public \mods{num\_settings} Submodule Attributes/Methods}

\begin{itemize}
\item \vars{typecode}:  This module function returns the
	type code of the data array passed in as its argument.

\item \vars{typecodes}:  This dictionary is the same as the
	NumPy (or Numeric and \mods{numarray})
	dictionary \vars{typecodes}, except that the character
	\vars{'S'} and \vars{'c'} are added to the
	\vars{typecodes['Character']} entry, if absent.  This
	functionality is added because I found 
	\vars{typecodes['Character']} had different values in
	Mac OS X and Ubuntu GNU/Linux.
\end{itemize}


	\subsection{Private \mods{qtcm} Submodule Attributes}

This submodule of the package \mods{qtcm} is the module that defines
the \class{Qtcm} class.

\begin{itemize}
\item \vars{\_init\_prog\_dict}:  This dictionary contains
	the default values of all prognostic variables and 
	right-hand sides that can be initialized.  In the
	submodule \mods{qtcm}, it is set to
	the \vars{init\_prognostic\_dict} module variable in
	submodule \mods{defaults}.

\item \vars{\_init\_vars\_keys}:  List of all keys in
	\vars{\_init\_prog\_dict}, plus \vars{'dateofmodel'}
	and \vars{'title'}.  These names correspond to the
	field variables that are usually written out into a
	restart file.

\item \vars{\_test\_field}:  \class{Field} object instance used 
	in type tests.
\end{itemize}



	\subsection{Private \class{Qtcm} Attributes}  
					\label{sec:Qtcm.private.attrib}

\begin{itemize}
\item \vars{\_cont}:  A boolean attribute that is \vars{True}
	if the run session is a continuation run session and
	\vars{False} if not.  Set the value passed in by
	the keyword \vars{cont} when the \mods{run\_session}
	method is executed.

\item \vars{\_monlen}:  Integer array of the number of days in 
	each month, assuming a 365~day year.

\item \vars{\_\_qtcm}:  The extension module that is the
	compiled QTCM1 Fortran model for this instance.
	This attribute is unique for every instance:  The
	extension module \fn{.so} file is first copied to
	a temporary directory (given by the \vars{sodir}
	instance attribute) and then imported to the
	\class{Qtcm} instance.
	This private attribute is set on instantiation.

\item \vars{\_qtcm\_fields\_ids}:  Field ids for all default 
	field variables, set on instantiation.

\item \vars{\_runlists\_long\_names}:  Dictionary holding the
	descriptions of the standard run lists.  The keys of
	the dictionary are the names of the standard run lists.
\end{itemize}




%---------------------------------------------------------------------
\section{Creating Documentation}

The distribution of \mods{qtcm} comes with the full set of
documentation in readable form (PDF and HTML).  The documentation
consists of two kinds:  this User's Guide and the API documentation.
The User's Guide is written in \LaTeX.  The PDF version is generated
directly from \LaTeX, and the HTML version is created by
\LaTeX{2}HTML.

I use the \fn{make\_docs} shell script in \fn{doc} creates all these
documents.  Briefly, that script does the following:

\begin{itemize}
\item In the \fn{doc/latex} directory, uses \cmd{python} to
	run \fn{code\_to\_latex.py}, which generates the
	\LaTeX\ files describing the current \mods{qtcm} 
	package settings, including text in the manual which gives
	all uses of the current version number.

\item \LaTeX\ is run on the \LaTeX\ files in the \fn{doc/latex} directory.
	The PDF generated by the run is moved from \fn{doc/latex} to
	\fn{doc}.

\item \LaTeX{2}HTML is run on the \LaTeX\ files in \fn{doc/latex}.
	The HTML files generated by the run are moved to \fn{doc/html}.

\item \mods{epydoc} is run on the \mods{qtcm} package libraries.
	This is run in \fn{doc}, to make use of the \fn{epydoc}
	configuration file present there.  The syntax from the
	command line is:

\begin{codeblock}
\codeblockfont{%
epydoc -v --config epydocrc [name]}
\end{codeblock}
\vars{[name]} is either \cmd{qtcm}, if the \mods{qtcm} package is
installed in a directory listed in \vars{sys.path}, or 
\vars{[name]} is the name of the directory the \mods{qtcm} package is
located in (e.g., \fn{/usr/lib/python2.4/site-packages/qtcm}).

\end{itemize}

The \fn{make\_docs} script cannot be used without customizing it
to your system, so please \emphpara{DO NOT USE IT} if you do
not know what you are doing.  You could easily wipe out all your
documentation by mistake.





% ===== end of file =====
