% This file is automatically generated by the script
% code_to_latex.py in the doc/latex directory.  It is based upon
% the values found in the qtcm.defaults submodule, and should
% not be hand-edited if you want the values to correspond to
% the values in the qtcm.defaults submodule.
        

\begin{longtable}{l|c|c|c|c|p{0.37\linewidth}}
\textbf{Field} & \textbf{Shape} & \textbf{Max} & \textbf{Min} &
                                \textbf{Units} & \textbf{Description} \\
\hline
\endhead
\vars{Evap} & (1, 1) & 0 & 0 &  &  \\
\vars{FLW} & (1, 1) & 0 & 0 &  &  \\
\vars{FLWds} & (1, 1) & 0 & 0 &  &  \\
\vars{FLWus} & (1, 1) & 0 & 0 &  &  \\
\vars{FLWut} & (1, 1) & 0 & 0 &  &  \\
\vars{FSW} & (1, 1) & 0 & 0 &  &  \\
\vars{FSWds} & (1, 1) & 0 & 0 &  &  \\
\vars{FSWus} & (1, 1) & 0 & 0 &  &  \\
\vars{FSWut} & (1, 1) & 0 & 0 &  &  \\
\vars{FTs} & (1, 1) & 0 & 0 &  &  \\
\vars{Qc} & (1, 1) & 0 & 0 & K & Precipitation \\
\vars{S0} & (1, 1) & 0 & 0 &  &  \\
\vars{STYPE} & (1, 1) & 0 & 0 &  & Surface type; ocean or vegetation type over land \\
\vars{T1} & (1, 1) & 0 & 0 & K &  \\
\vars{Ts} & (1, 1) & 0 & 0 & K & Surface temperature \\
\vars{WD} & (1, 1) & 0 & 0 &  &  \\
\vars{WD0} & (1,) & 0 & 0 &  & Field capacity SIB2/CSU (approximately) \\
\vars{arr1} & (1, 1) & 0 & 0 &  & Auxiliary optional output array 1 \\
\vars{arr2} & (1, 1) & 0 & 0 &  & Auxiliary optional output array 2 \\
\vars{arr3} & (1, 1) & 0 & 0 &  & Auxiliary optional output array 3 \\
\vars{arr4} & (1, 1) & 0 & 0 &  & Auxiliary optional output array 4 \\
\vars{arr5} & (1, 1) & 0 & 0 &  & Auxiliary optional output array 5 \\
\vars{arr6} & (1, 1) & 0 & 0 &  & Auxiliary optional output array 6 \\
\vars{arr7} & (1, 1) & 0 & 0 &  & Auxiliary optional output array 7 \\
\vars{arr8} & (1, 1) & 0 & 0 &  & Auxiliary optional output array 8 \\
\vars{psi0} & (1, 1) & 0 & 0 &  &  \\
\vars{q1} & (1, 1) & 0 & 0 & K &  \\
\vars{rhsu0bar} & (1,) & 0 & 0 &  &  \\
\vars{rhsvort0} & (1, 1, 1) & 0 & 0 &  &  \\
\vars{taux} & (1, 1) & 0 & 0 &  &  \\
\vars{tauy} & (1, 1) & 0 & 0 &  &  \\
\vars{u0} & (1, 1) & 0 & 0 & m/s & Barotropic zonal wind \\
\vars{u1} & (1, 1) & 0 & 0 & m/s & Current time step baroclinic zonal wind \\
\vars{v0} & (1, 1) & 0 & 0 & m/s & Barotropic meridional wind \\
\vars{v1} & (1, 1) & 0 & 0 & m/s &  \\
\vars{vort0} & (1, 1) & 0 & 0 &  &  \\
\end{longtable}
