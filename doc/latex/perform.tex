%=====================================================================
% Model Performance
%=====================================================================


% ----- BEGIN TEXT -----
%
%---------------------------------------------------------------------

The wall-clock time values below give the mean over three
separate 365 day aquaplanet runs,
using climatological sea surface temperature for lower boundary forcing.
NetCDF output is written daily, for both instantaneous and mean values.
The time step is 1200~sec, and the version of \mods{qtcm} used
is 0.1.1.
The horizontal grid spacing of all model versions is
$5.625^{\circ}$ longitude by $3.75^{\circ}$ latitude.
Values are in seconds:
\begin{center}
\begin{tabular}{p{0.5\linewidth}|c|c|c}
\textbf{System} & \textbf{Pure} & \textbf{Full} & \textbf{Parts} \\
\hline
Mac OS X:  MacBook 1.83 GHz Intel Core Duo running Mac OS X
	10.4.10 with 1 GB RAM
	(Python 2.4.3, NumPy 1.0.3, \mods{f2py} 2\_3816).
    & 152.59 & 153.63 & 158.94 \\
\hline
Ubuntu GNU/Linux:  Dell PowerEdge 860 with 2.66 GHz Quad Core Intel
	Xeon processors (64 bit) running Ubuntu 8.04.1 LTS
	(Python 2.5.2, NumPy 1.1.0, \mods{f2py} 2\_5237).
    & 43.73 & 44.79 & 47.45
\end{tabular}
\end{center}

``Pure'' refers to the pure-Fortran version of QTCM1.
``Full'' refers to a \mods{qtcm} run session with \vars{compiled\_form}
set to \vars{'full'}.  ``Parts'' refers to a \mods{qtcm} run session
with \vars{compiled\_form} set to \vars{'parts'}.
(Section~\ref{sec:compiledform} has details about the difference
between compiled forms.)

The \vars{'parts'} version of \mods{qtcm} gives Python the maximum
flexibility in accessing compiled QTCM1 model subroutines and
variables.  The price of that flexibility is an increase in
run time of approximately 4--9\% over the pure-Fortran version.
The difference in performance between the
\vars{'full'} version of \mods{qtcm} and the pure-Fortran version of
QTCM1 is between negligible and 3\% longer.

To make a timing for the pure-Fortran model, go to
\fn{test/benchmarks/timing/work} in \fn{/buildpath} and run the
\fn{timing\_365.sh} script in that directory.  That script runs the
QTCM1 model using \cmd{/usr/bin/time}, which at the end of the
script will output the amount of time it took to make the model
run.  Run the timing script three times and average the values to
obtain a time comparable to the above.

To make a timing for the \mods{qtcm} model, type \cmd{python
timing\_365.py} while in the \fn{test} directory in \fn{/buildpath}.
Three run sessions will be made for \vars{compiled\_form} equal to
\vars{'full'} and \vars{'parts'}, the times are averaged, and the
value are output at the end of the script.




% ====== end file ======
