% ==========================================================================
% Manual for QTCM Python Package
%
% Usage:
% - If you are running this on your own system, you will not have a copy of
%   my master.bib BibTeX database.  To run this, you'll have to comment out:
%
%      \bibliographystyle{chicago-jl}
%      \bibliography{/Users/jlin/work/res/bib/master}
%
%   and comment back in:
%
%      % ==========================================================================
% Manual for QTCM Python Package
%
% Usage:
% - If you are running this on your own system, you will not have a copy of
%   my master.bib BibTeX database.  To run this, you'll have to comment out:
%
%      \bibliographystyle{chicago-jl}
%      \bibliography{/Users/jlin/work/res/bib/master}
%
%   and comment back in:
%
%      % ==========================================================================
% Manual for QTCM Python Package
%
% Usage:
% - If you are running this on your own system, you will not have a copy of
%   my master.bib BibTeX database.  To run this, you'll have to comment out:
%
%      \bibliographystyle{chicago-jl}
%      \bibliography{/Users/jlin/work/res/bib/master}
%
%   and comment back in:
%
%      % ==========================================================================
% Manual for QTCM Python Package
%
% Usage:
% - If you are running this on your own system, you will not have a copy of
%   my master.bib BibTeX database.  To run this, you'll have to comment out:
%
%      \bibliographystyle{chicago-jl}
%      \bibliography{/Users/jlin/work/res/bib/master}
%
%   and comment back in:
%
%      \input{manual.bbl}
%
%   in this file.  Then you can use pdflatex on this file to get the PDF of
%   the manual.  These 3 lines are in the back matter of the document.
%
% Revision Notes:
% - By Johnny Lin, North Park University, http://www.johnny-lin.com/
% - The chicago BibTeX style is unrecognized by latex2html, so I use
%   the plain style.
% ==========================================================================


% ------ DOCUMENT DEFINITIONS ------
%
\documentclass[12pt]{book}
\usepackage{color}
\usepackage{html}
\usepackage{graphicx}
\usepackage{textcomp}
%\usepackage{comment}    %- Unrecognized by latex2html; its use causes errors
%\usepackage{fancyvrb}   %- Unrecognized by latex2html; its use causes errors


%- Packages unrecognized by latex2html, but causes no error:
%
%\usepackage[letterpaper,margin=1in,includefoot]{geometry}
\usepackage[letterpaper,margin=1.25in]{geometry}
\usepackage{bibnames}
\usepackage{longtable}
\usepackage{multirow}


%+ Comment out explicity margin settings since use package geometry:
%\setlength{\topmargin}{0in}
%\setlength{\headheight}{0in}
%\setlength{\headsep}{0in}
%\setlength{\oddsidemargin}{0in}
%\setlength{\evensidemargin}{0in}
%\setlength{\textheight}{8.5in}
%\setlength{\textwidth}{6.5in}




% ------ COMMANDS AND LENGTHS ------
%
% --- Define colors:  Have to do this because for some reason LaTeX
%     sometimes looks for "BLUE" instead of "blue" and complains when
%     "BLUE" isn't found.
%
\definecolor{Blue}{rgb}{0,0,1}
\definecolor{BLUE}{rgb}{0,0,1}
\definecolor{green}{rgb}{0,0.6,0}
\definecolor{Green}{rgb}{0,0.6,0}
\definecolor{GREEN}{rgb}{0,0.6,0}


% --- Format code blocks.  Currently set to print out the code in just 
%     typewriter font with no box.  Will work the same for pdflatex 
%     and latex2html:
%
%     codeblock:  Environment for blocks of computer code or internet 
%       addresses.
%     codeblockfont:  Sets font for codeblocks.
%
\newenvironment{codeblock}%
	{\begin{quotation}\begin{minipage}[t]{0.9\textwidth}}%
	{\end{minipage}\end{quotation}}
	%{\begin{flushleft}}%
	%{\end{flushleft}}
\newcommand{\codeblockfont}[1]{\textcolor{blue}{\texttt{#1}}}
%     *** Version that only works for pdflatex that puts a box around 
%         the block and centers it (commented out).  Note that using
%         fancyvrb is the better way of creating such a boxed section
%         of code, but fancyvrb isn't recognized by latex2html:
%\newenvironment{codeblock}%
%	{\begin{center}\begin{tabular}{|c|} \hline \\ }%
%	{\\ \\ \hline \end{tabular}\end{center}}
%\newcommand{\codeblockfont}[1]{\parbox{0.8\textwidth}{\texttt{#1}}}


% --- Text titling/emphasis settings:
%
%     emphpara:  Emphasis for the first phrase or sentence of a 
%         paragraph.
%     booktitle:  Formats book titles.
%     tabletitle:  Title for an item block in the information table.
%     paratitle:  Title for a paragraph in an item block in the
%         information table.
%     emphdate:  Emphasize date in paragraph text.
%
%     cmd:  Commands
%     dumarg:  Dummy arguments
%     codearg:  Same as dumarg.
%     fn:  File and directory names
%     screen:  Screen display
%     vars:  Variable and attribute names
%     mods:  Module, subroutine, and method names
%     class:  Class names
%     code:  Generic code (avoid using this)
%
\newcommand{\emphpara}[1]{\textbf{#1}}
\newcommand{\booktitle}[1]{\textit{#1}}
%\newcommand{\tabletitle}[1]{\textsf{\textbf{#1}}}
\newcommand{\paratitle}[1]{\textit{#1}}
\newcommand{\emphdate}[1]{\textbf{#1}}

\newcommand{\code}[1]{\textcolor{blue}{\texttt{#1}}}
\newcommand{\cmd}[1]{\textcolor{blue}{\texttt{#1}}}
\newcommand{\dumarg}[1]{\textit{#1}}
\newcommand{\codearg}[1]{\textit{#1}}
\newcommand{\fn}[1]{\textsf{\textit{#1}}}
\newcommand{\screen}[1]{\textcolor{green}{\texttt{#1}}}
\newcommand{\vars}[1]{\textcolor{blue}{\texttt{#1}}}
\newcommand{\class}[1]{\textcolor{blue}{\texttt{#1}}}
\newcommand{\mods}[1]{\textcolor{blue}{\texttt{#1}}}


% --- Special table formatting:
%
%     tabletitlewidth:  Width for title field of an item block in the 
%         information table.
%     tablebodywidth:  Width for body field of an item block in the 
%         information table.
%     tabletabulardims:  Dimensions for the information table, used in
%         the tabular command.
%     tableitemlinespace:  Vertical spacing between item blocks in the
%         information table.
%     infotitle and infotext:  Used for two-column sub-information 
%         tables found in the body field of the information table.  
%         These are not global lengths but have values specific to the 
%         local context in which they're used.
%
\newlength{\tabletitlewidth}
\settowidth{\tabletitlewidth}{file and directory names}

\newlength{\tablebodywidth}
\setlength{\tablebodywidth}{0.9\textwidth}
\addtolength{\tablebodywidth}{-4ex}
\addtolength{\tablebodywidth}{-\tabletitlewidth}

\newcommand{\tabletabulardims}%
	{p{\tabletitlewidth}@{\hspace{4ex}}p{\tablebodywidth}}

\newcommand{\tableitemlinespace}{\baselineskip}
\newlength{\infotitle}
\newlength{\infotext}


% --- Lengths for formatting:
%
\newlength{\remainder}        % length to describe the residual of the
                              %   linewidth minus \enumlabel
\newlength{\enumlabel}        % length to describe figure sub-label width
                              %   (e.g. "(a)")


% --- TtH stuff:
%
%\def\tthdump#1{#1}


% --- LaTeX2HTML stuff:
%
%     htmlfigcaption:  Formatting for HTML replacement figure captions.
%
\newcommand{\htmlfigcaption}[1]{\parbox[c]{70ex}{\footnotesize{#1}}}


% --- Some book title abbreviations:
%
%     rute:  Booktitle for Rute User's.
%     linuxnut:  Booktitle for Linux in a Nutshell.
%     pynut:  Booktitle for Python in a Nutshell.
%
\newcommand{\rute}{\booktitle{Rute User's}}
\newcommand{\linuxnut}{\booktitle{Linux in a Nutshell}}
\newcommand{\pynut}{\booktitle{Python in a Nutshell}}


% --- Define special characters ---
%
\newcommand{\aonehat}{\ensuremath{\widehat{a_1}}}
\newcommand{\bonehat}{\ensuremath{\widehat{b_1}}}
\newcommand{\D}{\ensuremath{\mathcal{D}}}
\def\BibTeX{B\kern-.03em i\kern-.03em b\kern-.15em\TeX}




% ------ BEGINNING OF DOCUMENT TEXT ------
%
\begin{document}

    

    
% ------ TITLE AND TOC ------
%
\title{\mods{qtcm} User's Guide}
\author{Johnny Wei-Bing Lin\thanks{Physics Department, North Park University,
	3225 W.\ Foster Ave., Chicago, IL  60625, USA}}
\date{\today}
\maketitle
\tableofcontents




% ------ BODY ------
%
\chapter{Introduction}
\input{intro}

\chapter{Installation and Configuration}    \label{ch:install}
	\section{Summary and Conventions}      \label{sec:install.sum}
	\input{install_sum}
	\section{Fortran Compiler}             \label{sec:fort.compilers}
	\input{install_fort}
	\section{Required Packages}            \label{sec:py.etc.pkgs}
	\input{install_pkgs}
	\section{Compiling Extension Modules}  \label{sec:create.so}
	\input{compile_so}
	\section{Testing the Installation}     \label{sec:test.qtcm}
	\input{test_qtcm}
	\section{Model Performance}
	\input{perform}
	\section{Installing in Mac OS X}       \label{sec:install.macosx}
	\input{qtcm_in_macosx}
	\section{Installing in Ubuntu}         \label{sec:install.ubuntu}
	\input{qtcm_in_ubuntu}

\chapter{Getting Started With \mods{qtcm}}  \label{ch:getting.started}
\input{started}

\chapter{Using \mods{qtcm}}                 \label{ch:using}
\input{using}

%@@@\chapter{Combining \code{qtcm} with \code{CliMT}}
%@@@\input{climt}

\chapter{Troubleshooting}                   \label{ch:trouble}
\input{trouble}

\chapter{Developer Notes}                   \label{ch:devnotes}
\input{devnotes}

\chapter{Future Work}                       \label{ch:future}
\input{future}




% ----- BACK MATTER OF THE DOCUMENT -----
%
\normalsize
\pagebreak
\bibliographystyle{plain}
\bibliography{/Users/jlin/work/res/bib/master}

%- Uncomment the input line below and comment out the \bibliographystyle
%  and \bibliography lines if you're running this without the master.bib 
%  BibTeX database
%\input{manual.bbl}        

\appendix
\chapter{Field Settings in \mods{defaults}}  \label{app:defaults.values}
\input{defaults}




% ------ END OF DOCUMENT TEXT ------
%
\end{document}


% ===== end of file =====

%
%   in this file.  Then you can use pdflatex on this file to get the PDF of
%   the manual.  These 3 lines are in the back matter of the document.
%
% Revision Notes:
% - By Johnny Lin, North Park University, http://www.johnny-lin.com/
% - The chicago BibTeX style is unrecognized by latex2html, so I use
%   the plain style.
% ==========================================================================


% ------ DOCUMENT DEFINITIONS ------
%
\documentclass[12pt]{book}
\usepackage{color}
\usepackage{html}
\usepackage{graphicx}
\usepackage{textcomp}
%\usepackage{comment}    %- Unrecognized by latex2html; its use causes errors
%\usepackage{fancyvrb}   %- Unrecognized by latex2html; its use causes errors


%- Packages unrecognized by latex2html, but causes no error:
%
%\usepackage[letterpaper,margin=1in,includefoot]{geometry}
\usepackage[letterpaper,margin=1.25in]{geometry}
\usepackage{bibnames}
\usepackage{longtable}
\usepackage{multirow}


%+ Comment out explicity margin settings since use package geometry:
%\setlength{\topmargin}{0in}
%\setlength{\headheight}{0in}
%\setlength{\headsep}{0in}
%\setlength{\oddsidemargin}{0in}
%\setlength{\evensidemargin}{0in}
%\setlength{\textheight}{8.5in}
%\setlength{\textwidth}{6.5in}




% ------ COMMANDS AND LENGTHS ------
%
% --- Define colors:  Have to do this because for some reason LaTeX
%     sometimes looks for "BLUE" instead of "blue" and complains when
%     "BLUE" isn't found.
%
\definecolor{Blue}{rgb}{0,0,1}
\definecolor{BLUE}{rgb}{0,0,1}
\definecolor{green}{rgb}{0,0.6,0}
\definecolor{Green}{rgb}{0,0.6,0}
\definecolor{GREEN}{rgb}{0,0.6,0}


% --- Format code blocks.  Currently set to print out the code in just 
%     typewriter font with no box.  Will work the same for pdflatex 
%     and latex2html:
%
%     codeblock:  Environment for blocks of computer code or internet 
%       addresses.
%     codeblockfont:  Sets font for codeblocks.
%
\newenvironment{codeblock}%
	{\begin{quotation}\begin{minipage}[t]{0.9\textwidth}}%
	{\end{minipage}\end{quotation}}
	%{\begin{flushleft}}%
	%{\end{flushleft}}
\newcommand{\codeblockfont}[1]{\textcolor{blue}{\texttt{#1}}}
%     *** Version that only works for pdflatex that puts a box around 
%         the block and centers it (commented out).  Note that using
%         fancyvrb is the better way of creating such a boxed section
%         of code, but fancyvrb isn't recognized by latex2html:
%\newenvironment{codeblock}%
%	{\begin{center}\begin{tabular}{|c|} \hline \\ }%
%	{\\ \\ \hline \end{tabular}\end{center}}
%\newcommand{\codeblockfont}[1]{\parbox{0.8\textwidth}{\texttt{#1}}}


% --- Text titling/emphasis settings:
%
%     emphpara:  Emphasis for the first phrase or sentence of a 
%         paragraph.
%     booktitle:  Formats book titles.
%     tabletitle:  Title for an item block in the information table.
%     paratitle:  Title for a paragraph in an item block in the
%         information table.
%     emphdate:  Emphasize date in paragraph text.
%
%     cmd:  Commands
%     dumarg:  Dummy arguments
%     codearg:  Same as dumarg.
%     fn:  File and directory names
%     screen:  Screen display
%     vars:  Variable and attribute names
%     mods:  Module, subroutine, and method names
%     class:  Class names
%     code:  Generic code (avoid using this)
%
\newcommand{\emphpara}[1]{\textbf{#1}}
\newcommand{\booktitle}[1]{\textit{#1}}
%\newcommand{\tabletitle}[1]{\textsf{\textbf{#1}}}
\newcommand{\paratitle}[1]{\textit{#1}}
\newcommand{\emphdate}[1]{\textbf{#1}}

\newcommand{\code}[1]{\textcolor{blue}{\texttt{#1}}}
\newcommand{\cmd}[1]{\textcolor{blue}{\texttt{#1}}}
\newcommand{\dumarg}[1]{\textit{#1}}
\newcommand{\codearg}[1]{\textit{#1}}
\newcommand{\fn}[1]{\textsf{\textit{#1}}}
\newcommand{\screen}[1]{\textcolor{green}{\texttt{#1}}}
\newcommand{\vars}[1]{\textcolor{blue}{\texttt{#1}}}
\newcommand{\class}[1]{\textcolor{blue}{\texttt{#1}}}
\newcommand{\mods}[1]{\textcolor{blue}{\texttt{#1}}}


% --- Special table formatting:
%
%     tabletitlewidth:  Width for title field of an item block in the 
%         information table.
%     tablebodywidth:  Width for body field of an item block in the 
%         information table.
%     tabletabulardims:  Dimensions for the information table, used in
%         the tabular command.
%     tableitemlinespace:  Vertical spacing between item blocks in the
%         information table.
%     infotitle and infotext:  Used for two-column sub-information 
%         tables found in the body field of the information table.  
%         These are not global lengths but have values specific to the 
%         local context in which they're used.
%
\newlength{\tabletitlewidth}
\settowidth{\tabletitlewidth}{file and directory names}

\newlength{\tablebodywidth}
\setlength{\tablebodywidth}{0.9\textwidth}
\addtolength{\tablebodywidth}{-4ex}
\addtolength{\tablebodywidth}{-\tabletitlewidth}

\newcommand{\tabletabulardims}%
	{p{\tabletitlewidth}@{\hspace{4ex}}p{\tablebodywidth}}

\newcommand{\tableitemlinespace}{\baselineskip}
\newlength{\infotitle}
\newlength{\infotext}


% --- Lengths for formatting:
%
\newlength{\remainder}        % length to describe the residual of the
                              %   linewidth minus \enumlabel
\newlength{\enumlabel}        % length to describe figure sub-label width
                              %   (e.g. "(a)")


% --- TtH stuff:
%
%\def\tthdump#1{#1}


% --- LaTeX2HTML stuff:
%
%     htmlfigcaption:  Formatting for HTML replacement figure captions.
%
\newcommand{\htmlfigcaption}[1]{\parbox[c]{70ex}{\footnotesize{#1}}}


% --- Some book title abbreviations:
%
%     rute:  Booktitle for Rute User's.
%     linuxnut:  Booktitle for Linux in a Nutshell.
%     pynut:  Booktitle for Python in a Nutshell.
%
\newcommand{\rute}{\booktitle{Rute User's}}
\newcommand{\linuxnut}{\booktitle{Linux in a Nutshell}}
\newcommand{\pynut}{\booktitle{Python in a Nutshell}}


% --- Define special characters ---
%
\newcommand{\aonehat}{\ensuremath{\widehat{a_1}}}
\newcommand{\bonehat}{\ensuremath{\widehat{b_1}}}
\newcommand{\D}{\ensuremath{\mathcal{D}}}
\def\BibTeX{B\kern-.03em i\kern-.03em b\kern-.15em\TeX}




% ------ BEGINNING OF DOCUMENT TEXT ------
%
\begin{document}

    

    
% ------ TITLE AND TOC ------
%
\title{\mods{qtcm} User's Guide}
\author{Johnny Wei-Bing Lin\thanks{Physics Department, North Park University,
	3225 W.\ Foster Ave., Chicago, IL  60625, USA}}
\date{\today}
\maketitle
\tableofcontents




% ------ BODY ------
%
\chapter{Introduction}
%=====================================================================
% Introduction
%=====================================================================


% ----- BEGIN TEXT -----
%
%---------------------------------------------------------------------
\section{How to Read This Manual}

\emphpara{Most users:} 
Just read 
(1) the installation instructions in Chapter~\ref{ch:install},
(2) Chapter~\ref{ch:getting.started},
which tells you all you need to get started using \mods{qtcm}, and
(3) examples in Section~\ref{sec:cookbook} that give a feel
for how you can use the model.

\emphpara{Users having problems:}
Chapter~\ref{ch:trouble} provides troubleshooting tips for
a few problems.
The detailed description of how the package functions, 
in Chapter~\ref{ch:using}, will probably be more useful.

\emphpara{Developers:}
If you want to change the source code, please read
Chapter~\ref{ch:devnotes}.  Chapter~\ref{ch:future} describes
all the things I'd like to do to improve the package, but haven't
gotten to yet.




%---------------------------------------------------------------------
\section{About the Package}

The single-baroclinic mode
Neelin-Zeng Quasi-Equilibrium Tropical Circulation Model
\latexhtml{(QTCM1)\footnote{http://www.atmos.ucla.edu/$\sim$csi}}%
	{\htmladdnormallink{(QTCM1)}{http://www.atmos.ucla.edu/~csi}}
is a primitive equation-based intermediate-level atmospheric model
that focuses on simulating the tropical atmosphere.  Being more
complicated than a simple model, the model has full non-linearity
with a basic representation of baroclinic instability,
includes a radiative-convective feedback package, and includes a
simple land soil moisture routine (but does not include topography).
A brief, but more detailed, description of QTCM1 is given in
Section~\ref{sec:brief_qtcm}.

\htmladdnormallinkfoot{Python}{http://www.python.org}
is an interpreted, object-oriented, multi-platform,
open-source language that is used in a variety of software applications,
ranging from game development to bioinformatics.
In climate studies, Python has been used as the core language for
data analysis
(e.g., \htmladdnormallinkfoot{Climate Data Analysis Tools}{http://cdat.sf.net}),
visualization
(e.g., \htmladdnormallinkfoot{Matplotlib}{http://matplotlib.sf.net}),
and 
modeling
(e.g., \htmladdnormallinkfoot{PyCCSM}{http://code.google.com/p/pyccsm/}).

In comparison to traditional compiled languages like Fortran,
Python's lack of a separate compile step greatly simplifies the
debugging and testing phases of development, because code snippets
can be testing as code is written.
Python's extensive suite of higher-level tools (e.g., statistics,
visualization, string and file manipulation) accessible at runtime 
enables modeling and analysis to occur concurrently.  

The \mods{qtcm} package is an implementation of the Neelin-Zeng
QTCM1 in a Python object-oriented environment.  The conversion
package
\htmladdnormallinkfoot{\mods{f2py}}{http://cens.ioc.ee/projects/f2py2e/} is
used to wrap the QTCM1 Fortran model routines and manage model
execution using Python datatypes and utilities.  The result is a
modeling package where order and choice of subroutine execution can
be altered at runtime.  Model analysis and visualization can also
be integrated with model execution at runtime.




%---------------------------------------------------------------------
\section{Conventions In This Manual}

	\subsection{Audience}

In this manual I assume you have a rudimentary knowledge of Python.
Thus, I do not describe basic Python data types (e.g., dictionaries,
lists), object framework and syntax (e.g., classes, methods,
attributes, instantiation), module and package importing.  If you
need to brush up (or learn) Python, I'd recommend the following
resources:

\begin{itemize}
\item \htmladdnormallinkfoot{Python Tutorial:}{http://docs.python.org/tut/}
	This tutorial was written by Guido van Rossum, Python's original
	author.

\item \htmladdnormallinkfoot{Instant Hacking:}%
	{http://www.hetland.org/python/instant-hacking.php}
	Learn how to program with Python.

\item \htmladdnormallinkfoot{Dive Into Python:}%
	{http://diveintopython.org/index.html}
	Reasonably complete and cohesive. The entire book is available for 
	free online.

\item \htmladdnormallinkfoot{Handbook of the Physics Computing Course:}%
	{http://www.pentangle.net/python/handbook/}
	Written for a science audience.  Recommended.

\item \latexhtml{CDAT/Python Tips for Earth Scientists:\footnote%
	{http://www.johnny-lin.com/cdat\_tips/}}%
	{\htmladdnormallink{CDAT/Python Tips for Earth Scientists:}%
		{http://www.johnny-lin.com/cdat_tips/}}
	This web site is a FAQ of sorts for people using Python and
	the Climate Data Analysis Tools (CDAT) in the earth sciences,
	and thus focuses on using Python to do science rather than
	the computer science aspects of the language.

\end{itemize}

The purpose of this package is to make the QTCM1 model easier to
use.  In order to interpret the results, however, you still need
to have a robust sense of what climate models can and cannot tell
you.  A starting point for the QTCM1 model is the brief description
of the model in Section~\ref{sec:brief_qtcm}.  After that, I would
read the original papers describing the model formulation and results
\cite{Neelin/Zeng:2000,Zeng/etal:2000}, and 
\latexhtml{papers citing the model formulation work.\footnote%
{http://scholar.google.com/scholar?hl=en\&lr=\&cites=14217886709842286738}}%
{\htmladdnormallink{papers citing the model formulation work}%
{http://scholar.google.com/scholar?hl=en&lr=&cites=14217886709842286738}.}
Being an intermediate-level model using the quasi-equilibrium assumption,
QTCM1 (and thus \mods{qtcm}) has distinct strengths and limitations; 
please be aware of them.


	\subsection{Typographic Conventions}

\begin{center}
\begin{tabular}{\tabletabulardims}
\cmd{commands} & to be typed at the command-line
	are rendered in a 
	blue, serif, fixed-width typewriter font
	(e.g., \cmd{make \_qtcm\_full\_365}). \\ \hline
\dumarg{dummy arguments} &
	coupled with code or screen display is rendered in a 
	serif, proportional, italicized font
	(e.g., \screen{Error-Value too long in} \dumarg{variable}). \\ \hline
\fn{file and directory names} & are rendered in a 
	sans-serif, italicized font
	(e.g., \fn{setbypy.F90}). \\ \hline
\screen{screen display} & is rendered in a 
	green, serif, fixed-width typewriter font. \\ \hline
\mods{module, method, and subroutine names} & are rendered in a 
	blue, serif, fixed-width typewriter font. \\ \hline
\vars{variable and attribute names} & are rendered in a 
	blue, serif, fixed-width typewriter font. \\ \hline
\class{class names} & are rendered in a 
	blue, serif, fixed-width typewriter font.
\end{tabular}
\end{center}

Blocks of code (usually commands, screen display, and module,
variable, and class names) are displayed in a blue, serif, fixed-width
typewriter font.


	\subsection{Terminology}

\begin{description}
\item[attribute and method references:]
	If there is any possibility of confusion, I will give the
	class that an attribute or method comes from when that
	attribute or method is referenced.  If no class is mentioned
	by name or context,
	assume that the attribute/method comes from the
	\class{Qtcm} class.

\item[``compiled QTCM1 model'':]
	This usually is used to denote when I'm talking about
	compiled Fortran QTCM1 routines and variables therein,
	as an extension module to the Python \mods{qtcm} package..
	Thus, ``compiled QTCM1 model \vars{u1}'' is the value
	of variable \vars{u1} in the Fortran model, not the
	value at the Python-level.  Sometimes I refer to the
	compiled QTCM1 model as just ``QTCM1'' or as
	``compiled QTCM1 Fortran model''.

\item[``pure-Fortran QTCM1'':]
	This refers to the Neelin-Zeng QTCM1 model in it's
	original Fortran form, not as an extension module to
	the Python \mods{qtcm} package.

\item[``Python-level'':]
	This usually denotes the value of a variable as an
	attribute of a \class{Qtcm} instance.  This variable
	is stored at the Python interpreter level.

\item[\class{Qtcm}:]
	This refers to the Python \class{Qtcm} class
	(note the capitalized first letter).

\item[\mods{qtcm}:]
	This refers to the Python \mods{qtcm} package.

\item[QTCM1 vs.\ QTCM:]
	Although the QTCM1 is currently the only version of a
	quasi-equilibrium tropical circulation model (QTCM), in
	principle one can construct a QTCM with any number of
	baroclinic modes.  In anticipation of this, the names of
	the Python package and class do not end in a numeral.  In
	this manual and the \mods{qtcm} docstrings, QTCM and QTCM1
	are used interchangably.
	Usually either of these phrases mean the quasi-equilibrium
	tropical circulation model in a generic sense, regardless
	of its form of implementation.
\end{description}




%---------------------------------------------------------------------
\section{Current Version Information and Acknowledgments}  \label{sec:ver}

\input{pkg_version_date.tex}
\input{pkg_author.tex}is the primary author of the package.

The \mods{qtcm} package is built upon the pure-Fortran QTCM1 model,
version 2.3 (August 2002), with a few minor changes.
Those changes are described in detail in
Section~\ref{sec:f90changes}.

The homepage for the \mods{qtcm} package is
\htmladdnormallink{http://www.johnny-lin.com/py\_pkgs/qtcm}%
	{http://www.johnny-lin.com/py_pkgs/qtcm}.
All Python code in this package, 
and the Fortran files \fn{setbypy.F90} and \fn{wrapcall.F90},
are \copyright\ 2003--2008 by 
\htmladdnormallinkfoot{Johnny Lin}%
		{http://www.johnny-lin.com} 
and constitutes a
library that is covered under the GNU Lesser General Public License
(LGPL):

\begin{quotation}
	This library is free software; you can redistribute it
	and/or modify it under the terms of the 
	\htmladdnormallinkfoot{GNU Lesser General Public License}%
		{http://www.gnu.org/copyleft/lesser.html} 
	as published by
	the Free Software Foundation; either version 2.1 of the
	License, or (at your option) any later version.

	This library is distributed in the hope that it will be
	useful, but WITHOUT ANY WARRANTY; without even the implied
	warranty of MERCHANTABILITY or FITNESS FOR A PARTICULAR
	PURPOSE. See the GNU Lesser General Public License for more
	details.

	You should have received a copy of the GNU Lesser General
	Public License along with this library; if not, write to
	the Free Software Foundation, Inc., 59 Temple Place, Suite
	330, Boston, MA 02111-1307 USA.

	You can contact Johnny Lin at his email address 
	or at North Park University, Physics Department,
	3225 W. Foster Ave., Chicago, IL 60625, USA.  
\end{quotation}

All other Fortran code in this package, as well as the makefiles,
are covered by licenses (if any) chosen by their respective owners.

This manual, in all forms (e.g., HTML, PDF, \LaTeX),
is part of the documentation of the \mods{qtcm} package 
and is \copyright\ 2007--2008 by Johnny Lin.
Permission is granted to copy, distribute and/or modify 
this document under the terms of the 
GNU Free Documentation License, Version 1.2 
or any later version published by the Free Software Foundation; 
with no Invariant Sections, no Front-Cover Texts, 
and no Back-Cover Texts. 
A copy of the license can be found 
\htmladdnormallinkfoot{here}{http://www.gnu.org/licenses/fdl.html}.

Transparent copies of this document are located online in
\latexhtml{%
\htmladdnormallinkfoot{PDF}%
	{http://www.johnny-lin.com/py\_pkgs/qtcm/doc/manual.pdf}}%
{\htmladdnormallink{PDF}%
	{http://www.johnny-lin.com/py_pkgs/qtcm/doc/manual.pdf}}
and
\latexhtml{%
\htmladdnormallinkfoot{HTML}%
	{http://www.johnny-lin.com/py\_pkgs/qtcm/doc/}}%
{\htmladdnormallink{HTML}%
	{http://www.johnny-lin.com/py_pkgs/qtcm/doc/}}
formats.
The \LaTeX\ source files are distributed with the \mods{qtcm}
distribution.
While the HTML version is nearly identical to the PDF
and \LaTeX\ versions, not every feature in the manual was successfully
converted.  This is particularly true with figures, which are
unnumbered in the HTML version and may be formatted differently
than the authoritative PDF version.
Thus, please consider the \LaTeX\ version as the authoritative
version.

\vspace{\baselineskip}

\emphpara{Acknowledgements:}
Thanks to David Neelin and Ning Zeng and the Climate Systems
Interactions Group at UCLA for their encouragement and help.
On the Python side,
thanks to Alexis Zubrow, Christian Dieterich, Rodrigo Caballero,
Michael Tobis, and Ray Pierrehumbert for steering me straight.
Early versions of some of this work was carried out 
at the University of Chicago Climate Systems Center, 
funded by the National Science Foundation (NSF) 
Information Technology Research Program under grant ATM-0121028. 
Any opinions, findings and conclusions or recommendations 
expressed in this material are those of the author and 
do not necessarily reflect the views of the NSF.

Intel\textregistered\ and
   Xeon\textregistered\ are registered trademarks of Intel Corporation.
Matlab\textregistered\ is a registered trademark of The MathWorks.
UNIX\textregistered\ is a registered trademark of The Open Group.




%---------------------------------------------------------------------
\section{Summary of Release History}

\begin{itemize}
\item 2008 Sep 12:  Version 0.1.2 released.  Summary of changes:
	\begin{itemize}
	\item Create \class{Qtcm} method \mods{get\_qtcm1\_item}.
		This method is effectively an alias of method 
		\mods{get\_qtcm\_item}.
	\item Create \class{Qtcm} method \mods{set\_qtcm1\_item}.
		This method is effectively an alias of method 
		\mods{set\_qtcm\_item}.
	\item Update User's Guide to phase out references to
		the \mods{get\_qtcm\_item}
		and \mods{set\_qtcm\_item} methods.  
		Adding the ``1'' to the method names makes the purpose
		of the methods clearer.
	\item Add unit tests to cover methods \mods{get\_qtcm1\_item} and
		\mods{set\_qtcm1\_item}.
	\end{itemize}

\item 2008 Jul 30:  Updates to the User's Guide (just the online versions,
        not the copies released with the tarball).

\item 2008 Jul 15:  First publicly available distribution 
	released (v0.1.1).
\end{itemize}




%---------------------------------------------------------------------
\section{A Brief Description of The QTCM1}   \label{sec:brief_qtcm}

This description is copied from Ch.\ 3 of Lin \cite{Lin:2000}, 
with minor revisions.
Model formulation is fully described in
Neelin \& Zeng \cite{Neelin/Zeng:2000} and model
results are described in Zeng et~al.\ \cite{Zeng/etal:2000}.
Neelin \& Zeng \cite{Neelin/Zeng:2000} is based upon v2.0 of QTCM1
and Zeng et~al.\ \cite{Zeng/etal:2000} is based on QTCM1 v2.1.
The 
\latexhtml{%
\htmladdnormallinkfoot{QTCM1 manual}%
	{http://www.atmos.ucla.edu/$\sim$csi/qtcm\_man/v2.3/qtcm\_manv2.3.pdf}}%
{\htmladdnormallink{QTCM1 manual}%
	{http://www.atmos.ucla.edu/~csi/qtcm_man/v2.3/qtcm_manv2.3.pdf}}
\cite{Neelin/etal:2002}
describes the details of model implementation.

QTCM1 differs from most full-scale GCMs primarily in how the vertical
temperature, humidity, and velocity structure of the atmosphere is
represented.  First, instead of representing the vertical structure
by finite-differenced levels, the model uses a Galerkin expansion
in the vertical.  The vertical basis functions are chosen according
to analytical solutions under convective quasi-equilibrium conditions,
so only a few need be retained.
Temperature and humidity are each described by separate
vertical basis functions ($a_1$ and $b_1$, respectively).
Low-level variations in the humidity basis
are larger than in the temperature basis.
For velocity, QTCM1 uses a single baroclinic basis function ($V_1$)
defined consistently with the temperature basis function,
as well as a barotropic velocity mode ($V_0$).
The vertical profiles of $a_1$, $b_1$, and $V_1$
are given in Figure~\ref{fig:qtcm.basis}.
Currently, QTCM1 does not include a separate
vertical degree of freedom describing the PBL.
The horizontal grid spacing of the model is 
$5.625^{\circ}$ longitude by $3.75^{\circ}$ latitude.


% <QTCM1 beta version vertical structure modes>
%
% (1) LaTeX version:
%
\begin{latexonly}
\begin{figure}
   \noindent
   \begin{minipage}[b]{.49\linewidth}
      \settowidth{\enumlabel}{(a) }%
      \setlength{\remainder}{\linewidth}% 
      \addtolength{\remainder}{-\enumlabel}
      {(a)}~\makebox[\remainder]{$a_1$ and $b_1$}
      \centering\includegraphics[width=\linewidth,viewport=58 72 389 344,clip]%
                    {figs/a1b1.pdf}
   \end{minipage}\hfill
   \begin{minipage}[b]{.49\linewidth}
      \settowidth{\enumlabel}{(b) }%
      \setlength{\remainder}{\linewidth}% 
      \addtolength{\remainder}{-\enumlabel}
      {(b)}~\makebox[\remainder]{$V_1$}

      \centering\includegraphics[width=\linewidth,viewport=58 72 389 346,clip]%
                    {figs/V1.pdf}
   \end{minipage}

   \caption{Vertical profiles of basis functions for
		(a) temperature $a_1$ (solid) and humidity $b_1$ (dashed) and
		(b) baroclinic component of
		horizontal velocity $V_1$.}
   \label{fig:qtcm.basis}
\end{figure}
\end{latexonly}

% (2) HTML replacement version:
%
\begin{htmlonly}
\label{fig:qtcm.basis}
\begin{center}
\htmladdimg{../latex/figs/a1b1.png}
\htmladdimg{../latex/figs/V1.png}

\htmlfigcaption{Figure \ref{fig:qtcm.basis}:  
	Vertical profiles of basis functions for
   	(a) temperature $a_1$ (solid) and humidity $b_1$ (dashed) and
   	(b) baroclinic component of
   	horizontal velocity $V_1$.}
\end{center}
\end{htmlonly}


These modes are chosen to accurately capture deep convective regions.
Outside deep convective regions the mode
is simply a highly truncated
Galerkin representation.  The system is much more tightly constrained than
a full-scale GCM, yet hopefully retains the essential dynamics and nonlinear
feedbacks.  The result is that QTCM1 is easier to diagnose than a GCM,
and is computationally fast (about 8 minutes per year on a Sun Ultra 2
workstation).  Zeng et al.\ \cite{Zeng/etal:2000} show results indicating
this intermediate-level model does a reasonable job simulating
tropical climatology and ENSO variability.  


Below is a summary of the main model equations \cite{Neelin/Zeng:2000}:
\begin{equation}
   \partial_t \mathbf{v}_1 
      + \D_{V1} (\mathbf{v}_0 , \mathbf{v}_1)
      + f \mathbf{k} \times \mathbf{v}_1
      =
   - \kappa \nabla T_1 
      - \epsilon_1 \mathbf{v}_1 
      - \epsilon_{01} \mathbf{v}_0
   \label{eqn:barocl_wind}
\end{equation}
\begin{equation}
   \partial_t \zeta_0 
      + \mathrm{curl}_z (\D_{V0} (\mathbf{v}_0 , \mathbf{v}_1))
      + \beta v_0
      =
   - \mathrm{curl}_z (\epsilon_0 \mathbf{v}_0)
      - \mathrm{curl}_z (\epsilon_{10} \mathbf{v}_1)
   \label{eqn:barotr_wind}
\end{equation}
\begin{equation}
   \aonehat (\partial_t + \D_{T1}) T_1 
      + M_{S1} \nabla \cdot {\bf v}_1 
      =
   \langle Q_c \rangle
      + (g/p_T) (-R^\uparrow_t -R^\downarrow_s + R^\uparrow_s + S_t - S_s + H)
   \label{eqn:temperature}
\end{equation}
\begin{equation}
   \bonehat (\partial_t + \D_{q1}) q_1 
      - M_{q1} \nabla \cdot {\bf v}_1 
      =
   \langle Q_q \rangle
      + (g/p_T) E
   \label{eqn:moisture}
\end{equation}
where (\ref{eqn:barocl_wind}) describes the baroclinic wind component,
      (\ref{eqn:barotr_wind}) describes the barotropic wind component,
      (\ref{eqn:temperature}) is the temperature equation, and
      (\ref{eqn:moisture}) is the moisture equation.

In the simplest formulation, the vertically integrated
convective heating and moisture sink
are assumed to be equal and opposite, so:
\begin{equation}
  -\langle Q_q \rangle = \langle Q_c \rangle 
                              = \epsilon^\ast_c (q_1 - T_1)
\end{equation}

For its convective parameterization for $Q_c$, this model uses the
Betts-Miller \cite{Betts/Miller:1986} moist convective
adjustment scheme, a scheme that is also used in some GCMs.
In the convective parameterization, the coefficient
$\epsilon^\ast_c$ is defined as:
\begin{equation}
   \epsilon^\ast_c 
      \equiv 
   \aonehat \bonehat (\aonehat + \bonehat)^{-1} \tau_c^{-1} 
      \mathcal{H}( \mathit{C}_{\mathrm{1}} )
\end{equation}
where $\mathcal{H}( \mathit{C}_{\mathrm{1}} )$ is zero for
$C_{1} \leq 0$, and one for $C_{1} > 0$, and $C_{1}$
is a measure of the convective available potential energy (CAPE),
projected onto the model's temperature and moisture basis functions.

Sensible heat ($H$) and evaporation ($E$) are given as
bulk-aerodynamic formulations:
\begin{equation}
   H
      =
   \rho_a C_D \mathrm{V}_s (T_s - (T_{rs} + a_{1s} T_1))
\end{equation}
\begin{equation}
   E
      =
   \rho_a C_D \mathrm{V}_s (q_\mathit{sat} (T_s) 
      - (q_{rs} + b_{1s} q_1))
\end{equation}

Longwave radiation components are denoted by $R$, and net shortwave
radiation is denoted by $S$.
The terms $\D_{V1}$ and $\D_{V0}$ are the advection-diffusion operators
for the momentum equations (projected onto $V_0$ and $V_1 (p)$,
respectively).
The terms $\D_{T1}$ and $\D_{q1}$ are the
advection-diffusion operators for the temperature and moisture
equations, respectively, using a vertical average projection.
The $\langle X \rangle$ and $\widehat{X}$ operators are
equivalent and denote vertically integration over the troposphere.
Please see Neelin \& Zeng \cite{Neelin/Zeng:2000} and 
Zeng et al.\ \cite{Zeng/etal:2000}
for a more complete description of equations and coefficients.







% ====== end file ======


\chapter{Installation and Configuration}    \label{ch:install}
	\section{Summary and Conventions}      \label{sec:install.sum}
	% ==========================================================================
% Installation Summary
%
% By Johnny Lin
% ==========================================================================


% ------ BODY -----
%

This section provides a summary of the steps needed to install
\mods{qtcm}, and a description of the naming conventions used in
this chapter.  If you have had a decent amount of experience with
Python and installing software on a Unix system, this section will
probably be all you need to read.  The installation steps are:

\begin{enumerate}
\item Install a Fortran compiler (see Section~\ref{sec:fort.compilers}
	for a list of compilers known to work).
	This compiler should be in a directory
	listed in your system path (e.g., \fn{/usr/bin}, etc.).

\item Install all required packages
	(see Section~\ref{sec:py.etc.pkgs} for details):
	Python,
	\mods{matplotlib} (plus the \mods{basemap} toolkit),
	NumPy (which includes \mods{f2py}),
	Scientific Python,
	\LaTeX,
	and
	netCDF.

	Python packages are required to be installed on your
	system in a directory listed in your \vars{sys.path},
	and the other packages/libraries are required to be in 
	standard directories listed in your system path 
	(e.g., \fn{/usr/bin}, \fn{/sw/include}, etc.).

	Make sure the executable for Python can be called at the
	Unix command line by typing both \cmd{python}.
	You might need to define a Unix alias
	that maps \cmd{python2.4} (or whichever version of Python
	you are using) to \cmd{python}.

\item \latexhtml{Download\footnote{http://www.johnny-lin.com/py\_pkgs/qtcm/}}%
        {\htmladdnormallink{Download}{http://www.johnny-lin.com/py_pkgs/qtcm/}}
	the \mods{qtcm} tarball and extract the distribution
	into a temporary directory for building purposes.
	\fn{\input{pkg_distro_dirname}}is the name of
	the \mods{qtcm} distribution directory;
	the number following the hyphen is the
	version number of the distribution.  \label{list:download.qtcm.sum}

	In this manual, the path to \fn{\input{pkg_distro_dirname}}will
	be called the ``\mods{qtcm} build path'' and be given as
	\fn{/buildpath}.  When you see \fn{/buildpath}, please substitute
	the actual temporary directory you created for building purposes.

\item The \mods{qtcm} distribution directory 
	\fn{\input{pkg_distro_dirname}}contains the following 
	principal sub-directories:
	\fn{doc}, \fn{lib}, \fn{src}, \fn{test}.
	Documentation is in \fn{doc},
	all the package modules are in \fn{lib},
	building of extension modules will take place in \fn{src},
	and testing of the package is done in \fn{test}.

\item Compile \mods{qtcm} extension modules in \fn{src}:
	Go to \fn{src}, copy the makefile from
	\fn{src/Makefiles} corresponding to your
	system into \fn{src}, rename to \fn{makefile},
	make changes to the makefile as needed,
	and execute:
	\begin{codeblock}
	\codeblockfont{%
	make clean \\
	make \_qtcm\_full\_365.so \\
	make \_qtcm\_parts\_365.so}
	\end{codeblock}
	If you executed the make commands in \fn{src,},
	the extension modules will be automatically placed in
	\fn{lib} in the \fn{\input{pkg_distro_dirname}}directory.
	See Section~\ref{sec:create.so} for details.
	\label{list:compile.so.sum}

\item Copy the entire contents of \fn{lib} in
	\fn{\input{pkg_distro_dirname}}(not \fn{lib} itself) 
	to a directory named
	\fn{qtcm} that is on your \mods{sys.path}.  For instance,
	for Mac OS X using Fink,
	many Python packages are located in a directory
	named \fn{/sw/\-lib/\-python2.4/\-site-packages}, or something
	similar, and this directory is on the system \mods{sys.path}.  
	If this is the case for your system, copy the
	contents of \fn{lib} into
	\fn{/sw/lib/\-python2.4/\-site-packages/\-qtcm}.
	(For Unix systems, the equivalent directory is usually
	\fn{/usr/\-local/\-lib/\-python2.4/\-site-packages}.)

\item Test the \mods{qtcm} distribution in \fn{test}:
	This step is optional and can take a while.
	Testing requires you to first generate a suite of benchmarks
	using the pure-Fortran QTCM1 model, then running the tests of
	\mods{qtcm} by typing:
	\begin{codeblock}
	\codeblockfont{%
python test\_all.py}
	\end{codeblock}
	at the Unix command line while in \fn{test}.
	See Section~\ref{sec:test.qtcm} for details.

\end{enumerate}

At some point, I will automate the installation using Python's
\htmladdnormallinkfoot{\mods{distutils}}{http://docs.python.org/dist/dist.html}
utilities.



% ===== end of file =====

	\section{Fortran Compiler}             \label{sec:fort.compilers}
	% ==========================================================================
% Fortran compilers
%
% By Johnny Lin
% ==========================================================================


% ------ BODY -----
%

You must have a Fortran compiler installed on your system in order
to compile \mods{qtcm}.  The compiler must be able to interface with
a pre-processor, as QTCM1 makes copious use of pre-processor directives.
\mods{qtcm} is known to work with the following Fortran compilers on the
following platforms:

\begin{center}
\begin{tabular}{l|l|l}
\textbf{Compiler}  & \textbf{Compiler Web Site} & \textbf{Platform(s)} \\ 
\hline
\mods{g95} & \htmladdnormallink{http://www.g95.org/}{http://www.g95.org/}  
	& Mac OS X \\
\end{tabular}
\end{center}

It will probably work with other platforms, but I haven't been able
to test platforms besides those listed above.  Note that \mods{g95}
is not \htmladdnormallink{GNU Fortran}{http://gcc.gnu.org/fortran/}
(\mods{gfortran}), the Fortran 95 compiler included with the more
recent versions of GCC.




% ===== end of file =====

	\section{Required Packages}            \label{sec:py.etc.pkgs}
	% ==========================================================================
% Python packages
%
% By Johnny Lin
% ==========================================================================


% ------ BODY -----
%

The following Python packages are required to be installed on your
system in a directory listed in your \vars{sys.path}:
\begin{itemize}
\item \htmladdnormallinkfoot{Python}%
	{http://www.python.org/}:  The Python programming language
	and interpreter.  Make sure you have a version recent enough
	to be compatible with all the needed Python packages.
\item \htmladdnormallinkfoot{\mods{matplotlib}}%
	{http://matplotlib.sourceforge.net/}:  Scientific plotting
	package, using Matlab-like syntax.  The \mods{basemap} toolkit
	for \mods{matplotlib} must also be installed.
\item \htmladdnormallinkfoot{NumPy}%
	{http://numpy.scipy.org/}:  The standard array package for
	Python.  The module name of NumPy imported in a Python 
	session is \mods{numpy}.
\item \htmladdnormallinkfoot{Scientific Python}%
	{http://dirac.cnrs-orleans.fr/plone/software/scientificpython/}:
	Has netCDF file operators, in addition to other routines
	of use in scientific computing.  The module name of
	Scientific Python imported in a Python session is
	\mods{Scientific}.
\end{itemize}

One other required Python package, \mods{f2py}, is now a part of the
NumPy package, and so installation of NumPy is sufficient to give
you both.

The package \htmladdnormallinkfoot{SciPy}{http://www.scipy.org},
which includes several Python-accessible scientific libraries, also
includes NumPy (and thus \mods{f2py}), so if you install SciPy,
you don't have to install NumPy again.  Note that SciPy is not the
same as Scientific Python; the names are confusing.

A few non-Python packages are also required:
\begin{itemize}
\item \LaTeX: A scientific typesetting program used by the 
	\class{Qtcm} instance method \mods{plotm} to handle 
	exponents and subscripts.  The most common Unix 
	distribution of \LaTeX\ is
	\htmladdnormallinkfoot{teTeX}{http://www.tug.org/teTeX}.

\item netCDF:  This set of libraries enables one to write datasets into
	a platform independent, binary format, with metdata attached.
	The \htmladdnormallinkfoot{netCDF 3.6.2 library}%
        	{http://www.unidata.ucar.edu/software/netcdf/}
	source code can be
\latexhtml{downloaded from UCAR\footnote{http://www.unidata.ucar.edu/downloads/netcdf/netcdf-3\_6\_2/}}%
        {\htmladdnormallink{downloaded from UCAR}{http://www.unidata.ucar.edu/downloads/netcdf/netcdf-3_6_2/}}.
\end{itemize}

For most Unix installations, the easiest way to install all the
above is via a package manager, for instance \mods{apt-get} in
Debian GNU/Linux, \mods{aptitude} or \mods{synaptic} in Ubuntu
GNU/Linux, and \mods{fink} in Mac OS X.  Of course, you can also
download a package's source code and build direct and/or install
using Python's
\htmladdnormallinkfoot{\mods{distutils}}{http://docs.python.org/dist/dist.html}
utilities.




% ===== end of file =====

	\section{Compiling Extension Modules}  \label{sec:create.so}
	% ==========================================================================
% Compiling extension modules
%
% By Johnny Lin
% ==========================================================================


% ------ BODY -----
%

The extension modules (\fn{.so} files) are imported and used by
\mods{qtcm} objects, and contain the Fortran QTCM1 model that is
called by the \mods{qtcm} Python wrappers.  These extension modules
are located in the \fn{lib} directory of the \mods{qtcm} distribution,
and, in general, need to be created only when the \mods{qtcm} package
is installed.

Two extension modules are created:  \fn{\_qtcm\_full\_365.so} and
\fn{\_qtcm\_parts\_365.so}.  Both modules define QTCM1 models where:

\begin{itemize}
\item A year is 365 days long 
	(makefile macro \vars{YEAR360} is off).
\item Model output is written to netCDF files
	(makefile macro \vars{NETCDFOUT} is on).
\item The atmospheric boundary layer model is used
	(makefile macro \vars{NO\_ABL} is off).
\item A global domain is used
	(makefile macro \vars{SPONGES} is off).
\item Topography effects due to induced divergence are not included
	(makefile macro \vars{TOPO} is off).
\item Coupling between atmosphere and ocean is through mean fluxes
	(makefile macro \vars{CPLMEAN} is off).
\item The mixed layer ocean model is not used
	(makefile macros \vars{MXL\_OCEAN} and \vars{BLEND\_SST} are both off).
\end{itemize}

(All other makefile macros not listed are also turned off.)
The only difference between these two extension modules is that the
``full'' module is used by \class{Qtcm} instances where
\vars{compiled\_form} is set to \vars{'full'}, and the ``parts''
module is used by \class{Qtcm} instances where \vars{compiled\_form}
is set to \vars{'parts'}.  See Section~\ref{sec:compiledform} for
details about the \vars{compiled\_form} attribute.

The extension modules are created through the following steps:
\begin{enumerate}
\item Go to the \mods{qtcm} distribution directory
	\fn{\input{pkg_distro_dirname}}located in
	your build path \fn{/buildpath}.  Go to the \fn{src}
	sub-directory.  This is where all the building of the
	extension modules will take place.

\item Copy the makefile that corresponds to your platform to
	the \fn{src} directory, and rename it \fn{makefile}.
	The \fn{Makefiles} sub-directory of \fn{src} contains
	makefiles for various platforms.

\item In \fn{makefile}, make the following changes:
	\begin{enumerate}
	\item Change the \vars{FC} environment variable as needed, 
		if your Fortran compiler is different.
	\item Change the \vars{FFLAGSM} environment variable, if the
		compiler flags listed are not supported by your
		compiler.
	\item Change the \vars{-I} and \vars{-L} parts of the
		\vars{NCINC} and \vars{NCLIB} environment
		variables so that the paths for the netCDF library and
		include files match your system's installation:
		\begin{codeblock}
		\codeblockfont{%
NCINC=-I/yourpath/netcdf/include \\
NCLIB=-L/yourpath/netcdf/lib -lnetcdf}
		\end{codeblock}
		Set \dumarg{yourpath} to the full path to the
		\fn{netcdf} directory where the \fn{include} and
		\fn{lib} sub-directories are that hold the netCDF
		libraries and include files.
		(You shouldn't have to change the \vars{-l} part of
		\vars{NCLIB}, since it is standard to name the netCDF
		library \fn{libnetcdf.a}.  But if you have a non-standard
		installation, change the \vars{-l} part too.)
	\end{enumerate}

\item At the Unix prompt, type:
\begin{codeblock}
\codeblockfont{%
\small
make clean \&\& make \_qtcm\_full\_365.so \&\& make \_qtcm\_parts\_365.so}
\end{codeblock}
	to clean up leftover files from previous compilations, and to
	compile the extension module shared object files
	\fn{\_qtcm\_full\_365.so} and \fn{\_qtcm\_parts\_365.so}.
\end{enumerate}

The makefile will automatically move the shared object files into
\fn{../lib}, overwriting any pre-existing files of the same name.
A detailed description of the makefile and using \mods{f2py} is
given in Section~\ref{sec:create.new.so}, if you wish to create a
different extension module.




% ===== end of file =====

	\section{Testing the Installation}     \label{sec:test.qtcm}
	% ==========================================================================
% Installation Summary
%
% By Johnny Lin
% ==========================================================================


% ------ BODY -----
%

The \mods{qtcm} distribution comes with a set of tests for the
package, using Python's \mods{unittest} package.  
Just to warn you, the tests take around an hour to run.
The tests will not work if the contents of \fn{lib}
after you've finished building \mods{qtcm} have not been copied
to a directory named \fn{qtcm} that is on your \mods{sys.path} path,
so make sure you've gone through all the install steps
(summarized in Section~\ref{sec:install.sum}) before you do these
tests.

\emphpara{NB:}  For these tests to work, both \cmd{python} and
\cmd{python2.4} must refer to the executable for the Python
installation on your system that you are using for running \mods{qtcm}.

The tests require a set of benchmark output files in the
\fn{test/benchmarks} directory in the
\fn{\input{pkg_distro_dirname}}directory (the output will be in
directories whose names begin with ``aquaplanet'' or ``landon'').
These output files are not included with the \mods{qtcm} distribution,
and must be created, by doing the following:

\begin{enumerate}
\item Goto the directory \fn{test/benchmarks/create/src} in the
	\fn{\input{pkg_distro_dirname}}\mods{qtcm} distribution directory,
	and copy the makefile from sub-directory \fn{Makesfiles},
	that corresponds to your system to the
	\fn{test/benchmarks/create/src} directory.  Rename the makefile 
	in \fn{test/benchmarks/create/src} to \fn{makefile}.

\item In \fn{makefile}, make the following changes:
        \begin{enumerate}
        \item Change the \vars{FC} environment variable as needed,
                if your Fortran compiler is different.
        \item Change the \vars{FFLAGSM} environment variable, if the
                compiler flags listed are not supported by your
                compiler.
        \item Change the \vars{-I} and \vars{-L} parts of the
                \vars{NCINC} and \vars{NCLIB} environment
                variables so that the paths for the netCDF library and
                include files match your system's installation:
                \begin{codeblock}
                \codeblockfont{%
NCINC=-I/yourpath/netcdf/include \\
NCLIB=-L/yourpath/netcdf/lib -lnetcdf}
                \end{codeblock}
                Set \dumarg{yourpath} to the full path to the
                \fn{netcdf} directory where the \fn{include} and
                \fn{lib} sub-directories are that hold the netCDF
                libraries and include files.
                (You shouldn't have to change the \vars{-l} part of
                \vars{NCLIB}, since it is standard to name the netCDF
                library \fn{libnetcdf.a}.  But if you have a non-standard
                installation, change the \vars{-l} part too.)
        \end{enumerate}

\item Go to the directory \fn{test/benchmarks/create} in the
	\fn{\input{pkg_distro_dirname}}\mods{qtcm} distribution directory.

\item Type \cmd{python create\_benchmarks.py} at the Unix command line
	to run the benchmark creation script.
\end{enumerate}

The created benchmarks will be located in 
\fn{test/benchmarks}, in directories with names related to the
run that was done, as described earlier.
The benchmarks are created using the
pure-Fortran QTCM1 model code,
version 2.3 (August 2002), with an altered makefile
(described above) and the following code change:
In all \fn{.F90} files, occurrences of:
        \begin{codeblock}
        \codeblockfont{%
        Character(len=130)}
        \end{codeblock}
        are changed to:
        \begin{codeblock}
        \codeblockfont{%
        Character(len=305)}
        \end{codeblock}
This enables the model to properly deal with longer filenames.
The number ``305'' is chosen to make search and replace easier.

Once the benchmarks are created, you can test the \mods{qtcm} package
by doing the following:
\begin{enumerate}
\item Go to the \fn{test} directory in the 
	\fn{\input{pkg_distro_dirname}}directory.
\item Type \cmd{python test\_all.py} at the Unix command line.
\end{enumerate}

If at the end of the test runs you see this message (or something similar):
\begin{codeblock}
\codeblockfont{%
\footnotesize
---------------------------------------------------------------------- \\
Ran 93 tests in 1244.205s \\
 \\
OK}
\end{codeblock}
then everything worked fine!  If you get any other message, the test(s) have
failed.



% ===== end of file =====

	\section{Model Performance}
	%=====================================================================
% Model Performance
%=====================================================================


% ----- BEGIN TEXT -----
%
%---------------------------------------------------------------------

The wall-clock time values below give the mean over three
separate 365 day aquaplanet runs,
using climatological sea surface temperature for lower boundary forcing.
NetCDF output is written daily, for both instantaneous and mean values.
The time step is 1200~sec, and the version of \mods{qtcm} used
is 0.1.1.
The horizontal grid spacing of all model versions is
$5.625^{\circ}$ longitude by $3.75^{\circ}$ latitude.
Values are in seconds:
\begin{center}
\begin{tabular}{p{0.5\linewidth}|c|c|c}
\textbf{System} & \textbf{Pure} & \textbf{Full} & \textbf{Parts} \\
\hline
Mac OS X:  MacBook 1.83 GHz Intel Core Duo running Mac OS X
	10.4.10 with 1 GB RAM
	(Python 2.4.3, NumPy 1.0.3, \mods{f2py} 2\_3816).
    & 152.59 & 153.63 & 158.94 \\
\hline
Ubuntu GNU/Linux:  Dell PowerEdge 860 with 2.66 GHz Quad Core Intel
	Xeon processors (64 bit) running Ubuntu 8.04.1 LTS
	(Python 2.5.2, NumPy 1.1.0, \mods{f2py} 2\_5237).
    & 43.73 & 44.79 & 47.45
\end{tabular}
\end{center}

``Pure'' refers to the pure-Fortran version of QTCM1.
``Full'' refers to a \mods{qtcm} run session with \vars{compiled\_form}
set to \vars{'full'}.  ``Parts'' refers to a \mods{qtcm} run session
with \vars{compiled\_form} set to \vars{'parts'}.
(Section~\ref{sec:compiledform} has details about the difference
between compiled forms.)

The \vars{'parts'} version of \mods{qtcm} gives Python the maximum
flexibility in accessing compiled QTCM1 model subroutines and
variables.  The price of that flexibility is an increase in
run time of approximately 4--9\% over the pure-Fortran version.
The difference in performance between the
\vars{'full'} version of \mods{qtcm} and the pure-Fortran version of
QTCM1 is between negligible and 3\% longer.

To make a timing for the pure-Fortran model, go to
\fn{test/benchmarks/timing/work} in \fn{/buildpath} and run the
\fn{timing\_365.sh} script in that directory.  That script runs the
QTCM1 model using \cmd{/usr/bin/time}, which at the end of the
script will output the amount of time it took to make the model
run.  Run the timing script three times and average the values to
obtain a time comparable to the above.

To make a timing for the \mods{qtcm} model, type \cmd{python
timing\_365.py} while in the \fn{test} directory in \fn{/buildpath}.
Three run sessions will be made for \vars{compiled\_form} equal to
\vars{'full'} and \vars{'parts'}, the times are averaged, and the
value are output at the end of the script.




% ====== end file ======

	\section{Installing in Mac OS X}       \label{sec:install.macosx}
	% ==========================================================================
% Description of installing in Mac OS X
%
% By Johnny Lin
% ==========================================================================


% ------ BODY -----
%
%------------------------------------------------------------------------
\subsection{Introduction}

This section describes issues and a summary of the installation steps
I followed to install \mods{qtcm} on a Mac running OS X.
It is a specific realization of the general installation
instructions found in Sections~\ref{sec:install.sum}--\ref{sec:test.qtcm}.
I first worked through these installation steps during June--July 2007,
with updates during July 2008.
The best way to go through this section is to go through
the summary of the installation steps in 
Section~\ref{sec:osx.install.summary},
and looking back to other sections as needed.




%------------------------------------------------------------------------
\subsection{Platform and Unix Dependencies}

This work was done on a MacBook 1.83 GHz Intel Core Duo running Mac OS X
10.4.11.  My machine has 1 GB RAM and 64 GB of disk in its main partition.

I recommend you turn-off your antivirus software before you
do the installs.  
Problems have been
\latexhtml{reported by Fink users\footnote%
		{http://finkproject.org/faq/usage-fink.php?phpLang=en\#kernel-panics}}%
	{\htmladdnormallink{reported by Fink users}%
		{http://finkproject.org/faq/usage-fink.php?phpLang=en#kernel-panics}}
using the Fink package manager with antivirus software enabled.

There are a variety of dependencies that are required to get your Mac
up-and-running as a scientific computing platform.  The most basic is
installing Apple's 
\htmladdnormallinkfoot{XCode}{http://developer.apple.com/tools/xcode/}
developer tools.\footnote%
	{The package should work in Mac OS X 10.4 with XCode 2.4.1 and higher;
	I've tried it with both 2.4.1 and XCode 2.5.  Note that
	XCode 3.1 only works on Mac OS X 10.5.}
This set of tools contains compilers and libraries
needed to do anything further.  You have to be a member of Apple's
Developer Connection, but registration is free.

Besides XCode, there are a variety of Unix libraries and utilities that you
need.  I first tried installing them by myself, from scratch, into
\fn{/usr/local}, but it was hard to keep track of all the dependencies.
A few that did work, and that I installed from their disk images, are:
\htmladdnormallinkfoot{MacTeX}{http://www.tug.org/mactex/}, 
\htmladdnormallinkfoot{MAMP}{http://www.mamp.info/}, and 
\htmladdnormallinkfoot{Tcl/Tk Aqua BI (Batteries Included)}%
	{http://tcltkaqua.sourceforge.net/}.\footnote%
		{Theoretically you can use Fink to install the equivalent
		of these packages, but I like the specific collection 
		found in these packages.  For instance, Tcl/Tk Aqua BI
		runs natively on the Mac.}

For everything else, thankfully, there's the
\htmladdnormallinkfoot{Fink Project}{http://www.finkproject.org/} which
uses a package manager built upon Debian tools to install ports of
Unix programs onto a Mac.  I just 
\htmladdnormallinkfoot{downloaded}%
	{http://www.finkproject.org/download/index.php?phpLang=en}
a binary version of the Fink 0.8.1 installer for Intel Macs,
installed Fink, and used its package management tools to install
(almost) everything else I needed.\footnote%
	{The one drawback of Fink is that it sometimes
	has stability problems.  In those cases, Fink provides
	command line suggestions to fix the problems, which sometimes
	will work.  If not, sometimes
\latexhtml{%
	deleting Fink and everything it installed,\footnote%
	{http://www.finkproject.org/faq/usage-fink.php?phpLang=en\#removing}}{%
\htmladdnormallink{deleting Fink and everything it installed}
	{http://www.finkproject.org/faq/usage-fink.php?phpLang=en#removing},}
	and starting afresh, will do the trick.
	It also appeared to me that sometimes when I installed 
	multiple packages
	via one \cmd{fink install} call, the installation did not work
	as well as when I installed only one package per call.}

Although you do not need anything besides a Fortran compiler and
the netCDF libraries to run QTCM1 in its pure-Fortran form, in order to
manipulate the model and use this Python version \mods{qtcm}, you
need to have Python installed.  The default Python that comes
with the Mac is a little old, so I used Fink to also install
Python 2.5 and related packages, including
\htmladdnormallinkfoot{matplotlib}{http://matplotlib.sourceforge.net/},
\htmladdnormallinkfoot{ScientificPython}{http://dirac.cnrs-orleans.fr/plone/software/scientificpython/},
and
\htmladdnormallinkfoot{SciPy}{http://www.scipy.org}
(see Section~\ref{sec:osx.summary} for details).




%------------------------------------------------------------------------
\subsection{Fortran Compiler}

There are a variety of high-quality, commercial Fortran compilers.
Unfortunately, because I do not have a research budget, I am not able
to use those compilers.  The 
\htmladdnormallinkfoot{GNU Compiler Collection}{http://gcc.gnu.org/}
(GCC) provides a suite of open-source compilers, some of which are the
standards of their language.  Most of the GCC compilers are installed
on your Mac when you install XCode.

\htmladdnormallinkfoot{GNU Fortran}{http://gcc.gnu.org/fortran/}
(\mods{gfortran}), is the Fortran 95 compiler included with the more
recent versions of GCC.
Unfortunately, I was not able to get it to compile QTCM1.
There is a second open-source Fortran compiler,
\htmladdnormallinkfoot{G95}{http://www.g95.org/} (\mods{g95}),
which some feel is farther along in its development than \mods{gfortran}.
I was able to successfully compile QTCM1 with \mods{g95} on my Mac.
I used Fink to install G95
(see Section~\ref{sec:osx.summary} for details).




%------------------------------------------------------------------------
\subsection{NetCDF Libraries}   \label{sec:netcdf}

For some reason, the netCDF libraries and include files
installed by Fink didn't correspond to the files needed
by the calling routines in \mods{qtcm}.  To solve this, I compiled
my own set of 
\htmladdnormallinkfoot{netCDF 3.6.2 libraries}%
	{http://www.unidata.ucar.edu/software/netcdf/}
using the tarball 
\latexhtml{downloaded from UCAR\footnote{http://www.unidata.ucar.edu/downloads/netcdf/netcdf-3\_6\_2/}}%
        {\htmladdnormallink{downloaded from UCAR}{http://www.unidata.ucar.edu/downloads/netcdf/netcdf-3_6_2/}}.

Once I uncompressed and untarred the package, and went into 
the top-level directory of the package, I built the package by typing
the following at the Unix prompt:

\begin{codeblock}
\codeblockfont{%
./configure --prefix=/Users/jlin/extra/netcdf \\
make check \\
make install}
\end{codeblock}

This installed the netCDF binaries, libraries, and include files into
sub-directories \fn{bin}, \fn{lib}, and \fn{include} in 
the directory specified by \vars{--prefix}.
If you want to install the netCDF libraries in the default
(usually \fn{/usr/local}), just leave out the \vars{--prefix}
option.

Note:  When you build netCDF, make sure the build directory
is not in the directory tree of \vars{--prefix}
(or the default directory \fn{/usr/local}).




%------------------------------------------------------------------------
\subsection{Makefile Configuration}  \label{sec:osx.makefile}

	\subsubsection{NetCDF}

In the \fn{src} directory in the \mods{qtcm} distribution, there is a
sub-directory \fn{Makefiles} that contains the makefiles for a
variety of platforms.  Edit the file \fn{makefile.osx\_g95}
so that the lines specifying the environment variables for the
netCDF libraries and include files:

\begin{codeblock}
\codeblockfont{%
NCINC=-I/Users/jlin/extra/netcdf/include \\
NCLIB=-L/Users/jlin/extra/netcdf/lib -lnetcdf}
\end{codeblock}

are changed to the path where your \emph{manually compiled} 
netCDF libraries and include files are.

Copy \fn{makefile.osx\_g95} from the \fn{Makefiles} sub-directory
in \fn{src} into \fn{src}.  
In other words, from the \mods{qtcm} distribution directory
(i.e., \fn{/buildpath}), at the Unix prompt execute:

\begin{codeblock}
\codeblockfont{%
cp src/Makefiles/makefile.osx\_g95 src/makefile}
\end{codeblock}


	\subsubsection{Linking Order}

Compilers in the GNU Compiler Collection (GCC) search libraries
and object files in the order they are listed in the command-line, 
\latexhtml{from left-to-right\footnote%
        {http://gcc.gnu.org/onlinedocs/gcc-4.1.2/gcc/Link-Options.html\#index-l-670}}%
        {\htmladdnormallink{from left-to-right}{http://gcc.gnu.org/onlinedocs/gcc-4.1.2/gcc/Link-Options.html#index-l-670}}.
Thus, if routines in \fn{b.o} call routines in \fn{a.o}, 
you must list the files in the order \fn{a.o b.o}.

For some reason, that isn't the case for \mods{g95}.  Thus, you will
find \mods{g95} makefile rules structured like the following
(below is part of the rule to create an executable (\fn{qtcm}) for
benchmark runs):

% --- Two versions of this rule, one for display in PDF and the other
%     for display in HTML:
%
\begin{latexonly}
\begin{codeblock}
\codeblockfont{%
qtcm: main.o \\
\hspace*{8ex}\$(FC)~-O~\$(NCINC)~-o~\$@ main.o~\$(QTCMLIB)~\$(NCLIB)}
\end{codeblock}
\end{latexonly}

\begin{htmlonly}
\begin{rawhtml}
<p><code><font color="blue">qtcm: main.o<br>
&nbsp;&nbsp;&nbsp;&nbsp;&nbsp;&nbsp;&nbsp;$(FC) -O $(NCINC) -o 
$@ main.o $(QTCMLIB) $(NCLIB)</font></code></p>
\end{rawhtml}
\end{htmlonly}

even though \fn{main.o} depends on the QTCM library 
(specified in macro setting \vars{\$(QTCMLIB)}), which in turn
depends on the netCDF library (specified in macro setting \vars{\$(NCLIB)}).


%------------------------------------------------------------------------
\subsection{Summary of Steps}   \label{sec:osx.install.summary}

The following summarizes all the steps I took to install
\mods{qtcm} in Mac OS X:

\begin{enumerate}
\item Install
	\htmladdnormallinkfoot{XCode 2.5}%
		{http://developer.apple.com/tools/xcode/}.

\item Install 
	\htmladdnormallinkfoot{MacTeX}{http://www.tug.org/mactex/}, 
	\htmladdnormallinkfoot{MAMP}{http://www.mamp.info/}, and 
	\htmladdnormallinkfoot{TCL/Tk Aqua BI (Batteries Included)}%
		{http://tcltkaqua.sourceforge.net/}.

\item Install
	\htmladdnormallinkfoot{Fink 0.8.1}%
		{http://www.finkproject.org/download/index.php?phpLang=en}.
	Make sure you
	\htmladdnormallink{set up your environment}%
		{http://www.finkproject.org/doc/users-guide/install.php\#setup}
	to enable you to use the packages you install with Fink
	(e.g. \vars{PATH} settings, etc.).
	Most of the time, that just means adding the line
	\cmd{source /sw/bin/init.csh} to your \fn{.cshrc} file (or the
	equivalent in your \fn{.bashrc}).

	Note that for many of the packages needed to run \mods{qtcm},
	you need to 
	\htmladdnormallink{configure Fink to download packages 
		from the unstable trees}%
	{http://www.finkproject.org/faq/usage-fink.php?phpLang=en\#unstable}.
	To do that, add \vars{unstable/main} and \vars{unstable/crypto}
	to the \vars{Trees:} line in \fn{/sw/etc/fink.conf}, and run:

	\begin{codeblock}
	\codeblockfont{fink selfupdate} \\
	\codeblockfont{fink index} \\
	\codeblockfont{fink scanpackages} \\
	\codeblockfont{fink update-all}
	\end{codeblock}

	When \cmd{selfupdate} runs, choose \cmd{rsync} for the
	self update method.  If you do not, Fink will not look in the
	unstable trees for packages.

\item Use Fink to install the \mods{g95} Fortran compiler.
	From a Unix prompt, type:

	\begin{codeblock}
	\codeblockfont{fink -$\,\!$-use-binary-dist install g95}
	\end{codeblock}

\item Use Fink to install Python 
	and the NumPy package (which \mods{f2py} is a part of).
	From a Unix prompt, type:

	\begin{codeblock}
	\codeblockfont{%
	fink -$\,\!$-use-binary-dist install python25 \\
	fink -$\,\!$-use-binary-dist install scipy-core-py25}
	\end{codeblock}

	(Numpy used to be called SciPy Core.)  If you want to
	install Python 2.4 instead, just change the ``25'' and ``py25'' above
	(and in later occurrences) to ``24'' and ``py24'', respectively.
	Note that Fink does not have a version of epydoc for Python 2.4,
	so if you wish to create documentation using epydoc, you will
	need to install Python 2.5.

\item Install teTeX and \LaTeX{2HTML} using Fink.
	From a Unix prompt, type:

	\begin{codeblock}
	\codeblockfont{fink -$\,\!$-use-binary-dist install tetex} \\
	\codeblockfont{fink -$\,\!$-use-binary-dist install latex2html}
	\end{codeblock}

	When prompted, choose ghostscript and ghostscript-fonts to
	satistfy the dependency (which should be the default options).
	I tried choosing system-ghostscript8, but Fink looks for
	ghostscript 8.51 and didn't recognize ghostscript 8.57 that
	was already installed in \fn{/usr/local} (via my MacTeX
	install).  \LaTeX{2HTML} has a package required by the
	\mods{qtcm} manual \LaTeX\ file.

\item Install additional programming and
	scientific packages and libraries using Fink.
	From a Unix prompt, type:

	\begin{codeblock}
	\codeblockfont{%
	fink -$\,\!$-use-binary-dist install scientificpython-py25 \\
	fink -$\,\!$-use-binary-dist install matplotlib-py25 \\
	fink -$\,\!$-use-binary-dist install matplotlib-basemap-py25 \\
	fink -$\,\!$-use-binary-dist install matplotlib-basemap-data-py25 \\
	fink -$\,\!$-use-binary-dist install xaw3d \\
	fink -$\,\!$-use-binary-dist install fftw fftw3 \\
	fink -$\,\!$-use-binary-dist install epydoc-py25 \\
	fink -$\,\!$-use-binary-dist install graphviz \\
	fink -$\,\!$-use-binary-dist install scipy-py25}
	\end{codeblock}

\item Manually install netCDF 3.6.2
	(see Section \ref{sec:netcdf}).

\item From this point on, you can follow the
	general instructions given in Section~\ref{sec:install.sum},
	starting with step~\ref{list:download.qtcm.sum}.
	Please do not ignore, however, Section~\ref{sec:install.macosx}'s
	Mac-specific details.

\end{enumerate}



% ===== end of file =====

	\section{Installing in Ubuntu}         \label{sec:install.ubuntu}
	% ==========================================================================
% Description of installing in Ubuntu
%
% By Johnny Lin
% ==========================================================================


% ------ BODY -----
%
%------------------------------------------------------------------------
\subsection{Introduction}

This section describes installation issues 
I followed to install \mods{qtcm} on my
Dell PowerEdge 860 running Ubuntu GNU/Linux 8.04.1 LTS (Hardy).
The machine has 2.66 GHz Quad Core Intel Xeon processors (64 bit),
4 GB RAM, and 677 GB of disk in its main partition.
This section is a specific realization of the general installation
instructions found in Sections~\ref{sec:install.sum}--\ref{sec:test.qtcm}.
I worked through these installation steps during July 2008.
The best way to go through this section is to go through
the summary of the installation steps in 
Section~\ref{sec:ubuntu.install.summary},
and looking back to other sections as needed.



%------------------------------------------------------------------------
\subsection{Fortran Compiler}     \label{sec:ubuntu.fort.install}

The easiest Fortran compiler to install in Ubuntu 8.04.1 is
\htmladdnormallinkfoot{GNU Fortran}{http://gcc.gnu.org/fortran/}
(\mods{gfortran}), the Fortran 95 compiler included with the more
recent versions of the GNU Compiler Collection (GCC); you can
use any package manager (e.g., \mods{apt-get}, \mods{aptitude})
to install it.
Unfortunately, I was not able to get it to compile QTCM1.
I was, however, able to successfully compile QTCM1 using
the second open-source Fortran compiler,
\htmladdnormallinkfoot{G95}{http://www.g95.org/} (\mods{g95}),
which some feel is farther along in its development than \mods{gfortran}.
G95, however, is not supported as an Ubuntu package, and so I had
to manually install it.

I downloaded the binary version of G95 v0.91 
(the Linux x86\_64/EMT64 with 32 bit default integers) 
using the following
\cmd{curl} command:\footnote%
	{I use \mods{curl} because I usually access my
	Ubuntu server via a terminal session.}

\begin{codeblock}
\codeblockfont{%
\small
curl -o g95.tgz http://ftp.g95.org/v0.91/g95-x86\_64-32-linux.tgz}
\end{codeblock}

which saves the \fn{.tgz} file as the local file \fn{g95.tgz}.
After that, I followed the G95 project's standard
\latexhtml{installation instructions\footnote%
	{http://g95.sourceforge.net/docs.html\#starting}}%
	{\htmladdnormallink{installation instructions}%
		{http://g95.sourceforge.net/docs.html#starting}}
to finish the install.\footnote%
	{The G95 installation instructions say you can put
	\fn{g95-install} anywhere, and make a link to the
	executable \mods{g95} in
	\fn{$\sim$/bin}.  I put \fn{g95-install} in
	\fn{/usr/local}, and while in \fn{/usr/local/bin}, 
	I put a link to the G95 executable using the command:
	\begin{codeblock}
	\codeblockfont{%
	sudo ln -s ../g95-install\_64/bin/x86\_64-suse-linux-gnu-g95 g95.}
	\end{codeblock}}
The regular Linux x86 version of G95
(in \fn{g95-x86-linux.tgz} from the G95 website) did not work on my
machine.




%------------------------------------------------------------------------
\subsection{NetCDF Libraries}   \label{sec:ubuntu.netcdf}

%Here things were very confusing for my machine, as I needed to
%install two versions of the
%\htmladdnormallinkfoot{netCDF}%
%	{http://www.unidata.ucar.edu/software/netcdf/}
%libraries and include files, one 
%for a successful compilation of the extension modules
%(as described in Section~\ref{sec:create.so}),
%and the other 
%for a successful run of the pure-Fortran QTCM1 model
%(used to create the testing benchmarks, as described in
%Section~\ref{sec:test.qtcm}).
%
%The first set of netCDF files (for the extension modules) are
%installed from Ubuntu's package management system.
%These are automatically installed when the \mods{python-netcdf}
%package is installed via an Ubuntu package manager
%(see Section~\ref{sec:ubuntu.install.summary}).
%The include files for this netCDF installation are 
%located in \fn{/usr/include}, and the libraries for this
%netCDF installation are location in \fn{/usr/lib}.

For some reason, the netCDF libraries and include files
installed from the Ubuntu packages do not
correspond to the files needed
by the calling routines in \mods{qtcm}.  To solve this, I compiled
my own set of
\htmladdnormallinkfoot{netCDF 3.6.2 libraries}%
        {http://www.unidata.ucar.edu/software/netcdf/}
using the tarball
\latexhtml{downloaded from UCAR\footnote{http://www.unidata.ucar.edu/downloads/netcdf/netcdf-3\_6\_2/}}%
        {\htmladdnormallink{downloaded from UCAR}{http://www.unidata.ucar.edu/downloads/netcdf/netcdf-3_6_2/}}.

Once I uncompressed and untarred the package, and went into
the top-level directory of the package, I built the package by typing
the following at the Unix prompt:

\begin{codeblock}
\codeblockfont{%
export FC=g95 \\
export FFLAGS="-O -fPIC" \\
export FFLAGS="-fPIC" \\
export F90FLAGS="-fPIC" \\
export CFLAGS="-fPIC" \\
export CXXFLAGS="-fPIC" \\
./configure \\
make check \\
sudo make install}
\end{codeblock}

(The \cmd{export} commands set environment variables for the
Fortran compiler and Fortran and other compiler flags.  The
\vars{-fPIC} flag enables the compilers to create
position independent code, needed for shared libraries in
Ubuntu on a 64 bit Intel processor.)

The above installs the netCDF binaries, libraries, and include files into
sub-directories \fn{bin}, \fn{lib}, and \fn{include} in 
\fn{/usr/local}, the default.
The include files for this netCDF installation are thus
located in \fn{/usr/local/include}, and the libraries for this
netCDF installation are location in \fn{/usr/local/lib}.
(If you want to specify a different installation
location, use the \vars{--prefix} option in \cmd{configure}.)
While you don't have to have root privileges during the configuration
and check steps, you do during the installation step if you're installing
into \fn{/usr/local} (thus the \cmd{sudo} in the last step).\footnote%
	{Note that when you build netCDF, make sure the build directory
	is not in the directory tree of \vars{--prefix}
	or the default directory \fn{/usr/local}.}

%Because there are two different netCDF installations used in the
%\mods{qtcm} package, the makefiles for creating the benchmarks
%and extensions files will have different \vars{NCLIB} and \vars{NCINC}
%environment variables (see Section~\ref{sec:ubuntu.makefile}).




%------------------------------------------------------------------------
\subsection{Makefile Configuration}  \label{sec:ubuntu.makefile}

	\subsubsection{NetCDF}

In the \fn{src} directory in the \mods{qtcm} distribution, there is a
sub-directory \fn{Makefiles} that contains the makefiles for a
variety of platforms.  Edit the file \fn{makefile.ubuntu\_64\_g95}
so that the lines specifying the environment variables for the
netCDF libraries and include files:

\begin{codeblock}
\codeblockfont{%
NCINC=-I/usr/local/include \\
NCLIB=-L/usr/local/lib -lnetcdf}
\end{codeblock}

are changed to the path where your manually compiled
netCDF libraries and include files are.

Copy \fn{makefile.ubuntu\_64\_g95} from the \fn{Makefiles} sub-directory
in \fn{src} into \fn{src}.  
In other words, from the \mods{qtcm} distribution directory
(i.e., \fn{/buildpath}), at the Unix prompt execute:

\begin{codeblock}
\codeblockfont{%
cp src/Makefiles/makefile.ubuntu\_64\_g95 src/makefile}
\end{codeblock}


	\subsubsection{Linking Order}

Compilers in the GNU Compiler Collection (GCC) search libraries
and object files in the order they are listed in the command-line,
\latexhtml{from left-to-right\footnote%
	{http://gcc.gnu.org/onlinedocs/gcc-4.1.2/gcc/Link-Options.html\#index-l-670}}%
	{\htmladdnormallink{from left-to-right}{http://gcc.gnu.org/onlinedocs/gcc-4.1.2/gcc/Link-Options.html#index-l-670}}.
Thus, if routines in \fn{b.o} call routines in \fn{a.o}, 
you must list the files in the order \fn{a.o b.o}.

For some reason, that isn't the case for \mods{g95}.  Thus, you will
find \mods{g95} makefile rules structured like the following
(below is part of the rule to create an executable (\fn{qtcm}) for
benchmark runs):

% --- Two versions of this rule, one for display in PDF and the other
%     for display in HTML:
%
\begin{latexonly}
\begin{codeblock}
\codeblockfont{%
qtcm: main.o \\
\hspace*{8ex}\$(FC)~-O~\$(NCINC)~-o~\$@ main.o~\$(QTCMLIB)~\$(NCLIB)}
\end{codeblock}
\end{latexonly}

\begin{htmlonly}
\begin{rawhtml}
<p><code><font color="blue">qtcm: main.o<br>
&nbsp;&nbsp;&nbsp;&nbsp;&nbsp;&nbsp;&nbsp;$(FC) -O $(NCINC) -o 
$@ main.o $(QTCMLIB) $(NCLIB)</font></code></p>
\end{rawhtml}
\end{htmlonly}

even though \fn{main.o} depends on the QTCM library 
(specified in macro setting \vars{QTCMLIB}), which in turn
depends on the netCDF library (specified in macro setting \vars{NCLIB}).


	\subsubsection{Shared Object PIC}   \label{sec:sopic}

In order to compile the model in Ubuntu on a 64 bit Intel processor,
the model and the netCDF library it is linked to needs to be
compiled to be 
\latexhtml{position independent code (PIC).\footnote%
		{http://www.gentoo.org/proj/en/base/amd64/howtos/index.xml?part=1\&chap=3}}%
	{\htmladdnormallink{position independent code (PIC)}%
		{http://www.gentoo.org/proj/en/base/amd64/howtos/index.xml?part=1&chap=3}.}
This is accomplished with the 
\htmladdnormallinkfoot{\cmd{-fPIC} flag}%
	{http://www.fortran-2000.com/ArnaudRecipes/sharedlib.html}.

In the \mods{qtcm} makefiles, the \cmd{-fPIC} flag is introduced in the
macro \vars{FFLAGSM}, for instance:
\begin{codeblock}
\codeblockfont{%
FFLAGSM = -O -fPIC}
\end{codeblock}
For makefiles used in creating extension modules, \cmd{-fPIC} must
be passed into the \mods{f2py} call.  To do so, put the flags:
\begin{codeblock}
\codeblockfont{%
--f90flags="-fPIC" --f77flags="-fPIC"}
\end{codeblock}
after the \vars{--fcompiler} flag in the \mods{f2py} calling line.

The \cmd{-fPIC} flag must also be used when compiling the netCDF
libraries, as described in Section~\ref{sec:ubuntu.netcdf}.
Failure to create PIC libraries in 64 bit Ubuntu can result in errors 
like the following when creating the \mods{qtcm} extension modules:
\begin{codeblock}
\codeblockfont{%
ld: /usr/local/lib/libnetcdf.a(fort-attio.o): relocation R\_X86\_64\_32 against `a local symbol' can not be used when making a shared object; recompile with -fPIC /usr/local/lib/libnetcdf.a: could not read symbols: Bad value}
\end{codeblock}




%------------------------------------------------------------------------
\subsection{Summary of Steps}      \label{sec:ubuntu.install.summary}

The following summarizes all the steps I took to install
\mods{qtcm} in
Ubuntu 8.04.1 LTS (Hardy) running on a
Quad Core Intel Xeon (64 bit) machine.
Note that while I use the \mods{aptitude} package manager, you are
free to use any manager of your choice (e.g., \mods{apt-get},
\mods{synaptic}, etc.):

\begin{enumerate}
\item Install the G95 Fortran compiler from the binary distribution.
	See Section~\ref{sec:ubuntu.fort.install} for details.

\item Use an Ubuntu package manager
	to install the following packages, by typing:
	\begin{codeblock}
	\codeblockfont{%
sudo aptitude update \\
sudo aptitude install curl \\
sudo aptitude install python-epydoc \\
sudo aptitude install python-matplotlib \\
sudo aptitude install python-netcdf \\
sudo aptitude install python-scientific \\
sudo aptitude install python-scipy \\
sudo aptitude install texlive}
	\end{codeblock}

	Installing \mods{python-scipy} will also install NumPy and
	\mods{f2py}, so you don't have to install the
	\mods{python-numpy} package separately.

	Early-on as I debugged my \mods{qtcm} install on Ubuntu,
	I encountered errors that I thought came from an 
	\htmladdnormallinkfoot{old version of NumPy}%
		{http://cens.ioc.ee/pipermail/f2py-users/2008-June/001617.html},
	and thus I replaced Ubuntu's packaged NumPy with NumPy 1.1.0
	built 
	\latexhtml{directly from source.\footnote%
			{http://sourceforge.net/project/showfiles.php?group\_id=1369\&package\_id=175103}}%
		{\htmladdnormallink{directly from source}{http://sourceforge.net/project/showfiles.php?group_id=1369&package_id=175103}.}
	(Note, you shouldn't install your new NumPy in the default
	location, which may cause problems later-on with Ubuntu's
	package manager.)
	Later on, I concluded the errors I had encountered were not
	because of the NumPy version, but by then I didn't want to
	try to reinstall NumPy again.
	So strictly speaking, the version of Numpy I used is not
	the one bundled with \mods{python-scipy}, but that shouldn't
	be a problem.

\item Manually install netCDF 3.6.2 from source
	(see Section \ref{sec:ubuntu.netcdf}).

\item Manually install the \mods{basemap} package of
	\mods{matplotlib}.  
	The source for the \mods{basemap} toolkit is
	available 
	\latexhtml{from Sourceforge\footnote%
			{http://sourceforge.net/project/showfiles.php?group\_id=80706}}%
		{\htmladdnormallink{from Sourceforge}%
			{http://sourceforge.net/project/showfiles.php?group_id=80706}}
	I obtained version 0.9.9.1 using the
	following \cmd{curl} command:
	\begin{codeblock}
	\codeblockfont{%
\scriptsize
curl -o basemap.tar.gz $\backslash$ \\
http://voxel.dl.sourceforge.net/sourceforge/matplotlib/basemap-0.9.9.1.tar.gz}
	\end{codeblock}

	The \fn{README} file in the \fn{basemap-0.9.9.1} directory has
	detailed installation instructions.  Note that you have to
	install the GEOS library first (\fn{README} has detailed
	directions on how to do that too).  To be on the safe-side,
	I would set the \vars{FC} environment variable to the G95
	compiler
	(e.g., with \cmd{export FC=g95} in Bash).

\item From this point on, you can follow the
	general instructions given in Section~\ref{sec:install.sum},
	starting with step~\ref{list:download.qtcm.sum}.
	Please do not ignore, however, Section~\ref{sec:install.ubuntu}'s
	Ubuntu-specific details.

\end{enumerate}



% ===== end of file =====


\chapter{Getting Started With \mods{qtcm}}  \label{ch:getting.started}
% ==========================================================================
% Getting Started With qtcm
%
% By Johnny Lin
% ==========================================================================


% ------ BODY -----
%
%---------------------------------------------------------------------
\section{Your First Model Run}

Figure~\ref{fig:my.first.run} shows an example of a script to make
a 30 day seasonal, aquaplanet model run, with run name ``test'',
starting from November 1, Year 1.


%--- Two versions, one for PDF, one for HTML:
\begin{latexonly}
\begin{figure}[htp]
\begin{center}
\begin{codeblock}
\codeblockfont{%
from qtcm import Qtcm \\
inputs = \{\} \\
inputs['runname'] = 'test' \\
inputs['landon'] = 0 \\
inputs['year0'] = 1 \\
inputs['month0'] = 11 \\
inputs['day0'] = 1 \\
inputs['lastday'] = 30 \\
inputs['mrestart'] = 0 \\
inputs['compiled\_form'] = 'parts' \\
model = Qtcm(**inputs) \\
model.run\_session()}
\end{codeblock}
\end{center}
\caption{An example of a simple \mods{qtcm} run.}
\label{fig:my.first.run}
\end{figure}
\end{latexonly}

\begin{htmlonly}
\label{fig:my.first.run}
\begin{center}
\htmlfigcaption{%
	\codeblockfont{%
from qtcm import Qtcm \\
inputs = \{\} \\
inputs['runname'] = 'test' \\
inputs['landon'] = 0 \\
inputs['year0'] = 1 \\
inputs['month0'] = 11 \\
inputs['day0'] = 1 \\
inputs['lastday'] = 30 \\
inputs['mrestart'] = 0 \\
inputs['compiled\_form'] = 'parts' \\
model = Qtcm(**inputs) \\
model.run\_session()}
	}

\htmlfigcaption{Figure~\ref{fig:my.first.run}:
	An example of a simple \mods{qtcm} run.}
\end{center}
\end{htmlonly}



The class describing the QTCM1 model is \class{Qtcm}.  An instance
of \class{Qtcm}, in this example \vars{model}, is created the same
way you create an instance of any class.  When instantiating an
instance of \class{Qtcm}, keyword parameters can be used to override
any default settings.  In the example above, the dictionary
\vars{inputs} specifying all keyword parameters is passed in on the
instantiation of \vars{model}.

The keyword parameter settings in
Figure~\ref{fig:my.first.run} have the following meanings:
\begin{itemize}
\item \vars{runname}:  This string (``test'') is used in the
	output filename.  QTCM1 writes mean and instantaneous
	output files to the directory given in \vars{model.outdir.value},
	with filenames 
	\fn{qm\_}\dumarg{runname}\fn{.nc} for mean output and
	\fn{qi\_}\dumarg{runname}\fn{.nc} for instantaneous output.

\item \vars{landon}: When set to ``0'', the land is turned off and
	the run is an aquaplanet run.  When set to ``1'', the land
	model is turned on.

\item \vars{year0}:  The year the run starts on.

\item \vars{month0}:  The month the run starts on (11 = November).

\item \vars{day0}: The day of the month the run starts on.

\item \vars{lastday}:  The model runs from day 1 to \vars{lastday}.

\item \vars{mrestart}:  When set to ``0'', the run starts from
	default initial conditions
	(see Section~\ref{sec:initial.variables} for a table of
	those values).
	When set to ``1'', the run starts from a restart file.

\item \vars{compiled\_form}:  This keyword sets what form the
	compiled QTCM1 model has, and its value is saved to
	the instance's \vars{compiled\_form} attribute.
	It is a string and can be set either to
	``parts'' or ``full''.  Most of the time, you will want
	to set it to \vars{'parts'}.
	This keyword is the only one
	that must be specified on instantiation; the model instance
	will at least instantiate
	using only the default settings for all the other keyword
	parameters (given in Appendix~\ref{app:defaults.values}).
	See Section~\ref{sec:compiledform} for details about
	what the \vars{compiled\_form} attribute controls.
\end{itemize}

By default, the \vars{SSTmode} attribute, which controls whether the
model will use climatological sea-surface temperatures (SST) 
or real SSTs, is set to the \vars{value} ``seasonal'', thus giving a
run with seasonal forcing at the lower-boundary over the ocean.

This example assumes that the boundary condition files, sea surface
temperature files, and the model output directories are as specified
in submodule \mods{defaults}.  Those values are described in
Section~\ref{sec:defaults.scalar}.




%---------------------------------------------------------------------
\section{Managing Directories}

Most of the time, your boundary condition files and output files
will not be in the locations specified in
Section~\ref{sec:defaults.scalar}, or in the directory your
Python script resides.  The easiest way to tell your \class{Qtcm} 
instance where your input/output files are is to pass them in
as keyword parameters on instantiation.


%--- Two versions, one for PDF, one for HTML:
\begin{latexonly}
\begin{figure}[htp]
\begin{codeblock}
\codeblockfont{%
\small
from qtcm import Qtcm \\
rundirname = 'test' \\
dirbasepath = os.path.join(os.getcwd(), rundirname) \\
inputs = \{\} \\
inputs['bnddir'] = os.path.join( os.getcwd(), 'bnddir', \\
\hspace*{40ex}'r64x42' ) \\
inputs['SSTdir'] = os.path.join( os.getcwd(), 'bnddir', \\
\hspace*{40ex}'r64x42', 'SST\_Reynolds' ) \\
inputs['outdir'] = dirbasepath \\
inputs['runname'] = rundirname \\
inputs['landon'] = 0 \\
inputs['year0'] = 1 \\
inputs['month0'] = 11 \\
inputs['day0'] = 1 \\
inputs['lastday'] = 30 \\
inputs['mrestart'] = 0 \\
inputs['compiled\_form'] = 'parts' \\
model = Qtcm(**inputs) \\
model.run\_session()}
\end{codeblock}

\caption{An example \mods{qtcm} run showing detailed description of
	input and output directories.}
\label{fig:manage.dir.example}
\end{figure}
\end{latexonly}

\begin{htmlonly}
\label{fig:manage.dir.example}
\begin{center}
\htmlfigcaption{%
	\codeblockfont{%
from qtcm import Qtcm \\
rundirname = 'test' \\
dirbasepath = os.path.join(os.getcwd(), rundirname) \\
inputs = \{\} \\
inputs['bnddir'] = os.path.join( os.getcwd(), 'bnddir', \\
\hspace*{40ex}'r64x42' ) \\
inputs['SSTdir'] = os.path.join( os.getcwd(), 'bnddir', \\
\hspace*{40ex}'r64x42', 'SST\_Reynolds' ) \\
inputs['outdir'] = dirbasepath \\
inputs['runname'] = rundirname \\
inputs['landon'] = 0 \\
inputs['year0'] = 1 \\
inputs['month0'] = 11 \\
inputs['day0'] = 1 \\
inputs['lastday'] = 30 \\
inputs['mrestart'] = 0 \\
inputs['compiled\_form'] = 'parts' \\
model = Qtcm(**inputs) \\
model.run\_session()}
	}

\htmlfigcaption{Figure~\ref{fig:manage.dir.example}:
	An example \mods{qtcm} run showing detailed description of
        input and output directories.}
\end{center}
\end{htmlonly}


Figure~\ref{fig:manage.dir.example} shows an example run where those
directories are explicitly specified; in all other aspects, the run
is identical to the one in Figure~\ref{fig:my.first.run}.
In Figure~\ref{fig:manage.dir.example}, output from the model is
directed to the directory described by string variable
\vars{dirbasepath}.  \vars{dirbasepath} is created by joining the
current working directory with the run name given in string variable
\vars{rundirname}.\footnote%
	{The Python \mods{os} module enables platform-independent
	handling of files and directories.  The \mods{os.path.join}
	function resolves paths without the programmer needing to know
	all the possible directory separation characters; the function
	chooses the correct separation character at runtime.  The
	\mods{os.getcwd} function returns the current working directory.}
Setting keyword parameter \vars{outdir} to \vars{dirbasepath} sends
output to \vars{dirbasepath}.  
Keywords \vars{bnddir} and \vars{SSTdir} specify the directories
where non-SST and SST boundary condition files, respectively, are
found.

Interestingly, the default version of QTCM1 does \emph{not} send
all output from the model to \vars{outdir}.  The restart file
\fn{qtcm\_}\dumarg{yyyymmdd}\fn{.restart} (where \dumarg{yyyymmdd}
is the year, month, and day of the model date when the restart
file was written) is written into the current working directory,
not the output directory.  Thus, if you do multiple runs, you'll
have to manually deal with the restart files that will proliferate.

Neither the QTCM1 model nor the \class{Qtcm} object
create the directories specified in \mods{bnddir}, \mods{SSTdir},
and \mods{outdir}.  Failure to do so will create an error.  I use
Python's file management tools to make sure the output directory
is created, and any old output files are deleted.  Here's an example
that does that, using the \vars{dirbasepath} and \vars{rundirname}
variables from Figure~\ref{fig:manage.dir.example}:

\begin{codeblock}
\codeblockfont{%
\small
if not os.path.exists(dirbasepath):  os.makedirs(dirbasepath) \\
qi\_file = os.path.join( dirbasepath, 'qi\_'+rundirname+'.nc' ) \\
qm\_file = os.path.join( dirbasepath, 'qm\_'+rundirname+'.nc' ) \\
if os.path.exists(qi\_file):   os.remove(qi\_file) \\
if os.path.exists(qm\_file):   os.remove(qm\_file)}
\end{codeblock}




%---------------------------------------------------------------------
\section{Model Field Variables}   \label{sec:field.variables.intro}

The term ``field'' variable refers to QTCM1 model variables that 
are accessible at both the compiled Fortran QTCM1 model-level as
well as the Python \class{Qtcm} instance-level.
Field variables are all instances of the \class{Field} class,
and are stored as attributes of the \class{Qtcm} instance.\footnote%
	{Note non-field variables can also be instances of \class{Field},
	and that \class{Qtcm} instances have other attributes that are
	not equal to \class{Field} instances.}

\class{Field} class instances have the following attributes:
\begin{itemize}
\item \vars{id}:  A string naming the field (e.g., ``Qc'', ``mrestart'').
	This string should contain no whitespace.
\item \vars{value}:  The value of the field.  Can be of any type, though
	typically is either a string or numeric scalar or a numeric array.
\item \vars{units}:  A string giving the units of the field.
\item \vars{long\_name}:  A string giving a description of the field.
\end{itemize}

\class{Field} instances also have methods to return the rank 
and typecode of \vars{value}.

Remember, if you want to access the value of a \class{Field} object,
make sure you access that object's \vars{value} attribute.  
Thus, for example,
to assign a variable \vars{foo} to the
\vars{lastday} value for a given
\class{Qtcm} instance \vars{model}, type the following:
\begin{codeblock}
\codeblockfont{%
foo = model.lastday.value}
\end{codeblock}

For scalars, this assignment sets \vars{foo} by value (i.e., a copy
of the value of attribute \vars{model.lastday} is set to \vars{foo}).
In general, however, Python assigns variables by reference.  Use
the \mods{copy} module if you truly want a copy of a field variable's
value (such as an array), rather than an alias.  For more details
about field variables, see Section~\ref{sec:field.variables}.




%---------------------------------------------------------------------
\section{Run Sessions}

	\subsection{What is a Run Session?}

A run session is a unit of simulation where the model is run from
day 1 of simulation to the day specified by the \vars{lastday}
attribute of a \class{Qtcm} instance.  A run session is a
``complete'' model run, at the beginning of which all compiled QTCM1
model variables are set to the values given at the Python-level,
and at the end of which restart files are written, the values
at the Python-level are overwritten by the values in the Fortran
model, and a Python-accessible snapshot is taken of the 
model variables that were written to the restart file.


	\subsection{Changing Variables}

Between run sessions, changing any field variable is as easy
as a Python assignment.  For instance, to change the atmosphere
mixed layer depth to 100~m, just type:
\begin{codeblock}
\codeblockfont{%
model.ziml.value = 100.0}
\end{codeblock}

When changing arrays, be careful to try to match the shape of the 
array.\footnote%
	{At the very least, match the rank of the array, which is required
	for the routines in \mods{setbypy} to properly choose which
	Fortran subroutine to use in reading the Python value.
	I haven't tested if only the rank is needed, however,
	for the passing to work, for a continuation run (my hunch is
	it won't).}
You can use the NumPy \mods{shape} function on a NumPy array to
check its shape.


	\subsection{Continuing a Model Run}  \label{sec:continuation.intro}

Figure~\ref{fig:continuation.example} shows an example of two run
sessions, where the second run session is a continuation of the
first.


%--- Two versions, one for PDF, one for HTML:
\begin{latexonly}
\begin{figure}[htp]
\begin{codeblock}
\codeblockfont{%
\small
inputs['year0'] = 1 \\
inputs['month0'] = 11 \\
inputs['day0'] = 1 \\
inputs['lastday'] = 10 \\
inputs['mrestart'] = 0 \\
inputs['compiled\_form'] = 'parts' \\ \\
model = Qtcm(**inputs) \\
model.run\_session() \\
model.u1.value = model.u1.value * 2.0 \\
model.init\_with\_instance\_state = True \\
model.run\_session(cont=30)}
\end{codeblock}

\caption{An example of two \mods{qtcm} run sessions where the second
	run session is a continuation of the first.  Assume 
	\vars{inputs} is a dictionary, and that earlier in the
	script the run name and
	all input and output directory names were added
	to the dictionary.}
\label{fig:continuation.example}
\end{figure}
\end{latexonly}

\begin{htmlonly}
\label{fig:continuation.example}
\begin{center}
\htmlfigcaption{%
	\codeblockfont{%
inputs['year0'] = 1 \\
inputs['month0'] = 11 \\
inputs['day0'] = 1 \\
inputs['lastday'] = 10 \\
inputs['mrestart'] = 0 \\
inputs['compiled\_form'] = 'parts' \\ \\
model = Qtcm(**inputs) \\
model.run\_session() \\
model.u1.value = model.u1.value * 2.0 \\
model.init\_with\_instance\_state = True \\
model.run\_session(cont=30)}
	}

\htmlfigcaption{Figure~\ref{fig:continuation.example}:
	An example of two \mods{qtcm} run sessions where the second
	run session is a continuation of the first.  Assume 
	\vars{inputs} is a dictionary, and that earlier in the
	script the run name and
	all input and output directory names were added
	to the dictionary.}
\end{center}
\end{htmlonly}


The first run session runs from day 1 to day 10.  The second
run session runs the model for another 30 days.  
Setting the \vars{init\_with\_instance\_state} of
\vars{model} to \vars{True} tells the model to use the
the values of the instance attributes 
(for prognostic variables, right-hand sides, and start date) 
are currently stored \vars{model}
as the initial values for the run\_session.\footnote%
	{Unless overridden, by default, 
	\vars{init\_with\_instance\_state} is set
	to True on \class{Qtcm} instance instantiation.}
The \vars{cont}
keyword in the second \mods{run\_session} call specifies a
continuation run, and the value gives the number of additional
days to run the model.

The set of runs described above would produce the exact same
results as if you had gone into the Fortran model after 10 days,
doubled the first baroclinic mode zonal velocity, and continued
the run for another 30 days.  With the Python example above, however,
you didn't need to know you were going to do that ahead of starting
the model run (which is what a compiled model requires you to do).
Section~\ref{sec:contination.run.sessions} describes continuation
runs in detail.


	\subsection{Passing Restart Snapshots Between Run Sessions}
					\label{sec:snapshot.intro}

The pure-Fortran QTCM1 uses a restart file to enable continuation
runs.  A \class{Qtcm} instance can also make use of that option,
through setting the \vars{mrestart} attribute value
(see Section~\ref{sec:contination.run.sessions} and
Neelin et al.\ \cite{Neelin/etal:2002} for details).  
It's easier, however, instead of using a restart file, to pass 
along a ``snapshot'' dictionary.

The \class{Qtcm} instance method \mods{make\_snapshot} copies the
variables that would be written out to a restart file into a
dictionary that is saves as the instance attribute \vars{snapshot}.
This snapshot can be saved separately, for later recall.  Note that
snapshots are automatically made at the end of a run session.

The following example shows a model \mods{run\_session} call,
following which the snapshot is saved to the variable
\vars{snapshot}:\footnote%
	{Remember Python assignment defaults to assignment by
	reference, so in this example the variable \vars{mysnapshot}
	is a pointer to the \vars{model.snapshot} attribute.
	(However, note that \vars{model.snapshot} itself is not a
	reference, but a distinct copy of those variables; to do
	otherwise would result in a non-static snapshot.)
	If the \vars{model.snapshot} attribute is dereferenced,
	then \vars{mysnapshot} will become the sole pointer to the
	dictionary.}

\begin{codeblock}
\codeblockfont{%
model.run\_session() \\
mysnapshot = model.snapshot}
\end{codeblock}

After taking the snapshot, you might continue the run a while, and
then decide to return to the snapshot you saved.  To do so, use
the \mods{sync\_set\_py\_values\_to\_snapshot}
method to reset the model instance values to
\vars{mysnapshot} before your next run session:
\begin{codeblock}
\codeblockfont{%
model.sync\_set\_py\_values\_to\_snapshot(snapshot=mysnapshot) \\
model.init\_with\_instance\_state = True \\
model.run\_session()}
\end{codeblock}

See Section~\ref{sec:snapshots} for details regarding the use of
snapshots, as well as for a list of what variables are saved in
a snapshot.




%---------------------------------------------------------------------
\section{Creating Multiple Models}

	\subsection{Model Instances}

Creating a new QTCM1 model is as simple as creating another
\class{Qtcm} instance.
For instance, to instantiate two QTCM1
models, \vars{model1} and \vars{model2}, type the following:

\begin{codeblock}
\codeblockfont{%
from qtcm import Qtcm \\
model1 = Qtcm(compiled\_form='parts') \\
model2 = Qtcm(compiled\_form='parts')}
\end{codeblock}

\vars{model1} and \vars{model2} do \emph{not} share any variables
in common, including the extension modules holding the Fortran
code.  In creating the instances, a copy of the extension modules
are saved in temporary directories.


	\subsection{Passing Snapshots To Other Models}

The snapshots described in Section~\ref{sec:snapshot.intro}
can also be passed around to other model instances,
enabling you to easily branch a model run:

\begin{codeblock}
\codeblockfont{%
model.run\_session() \\
mysnapshot = model.snapshot \\
model1.sync\_set\_py\_values\_to\_snapshot(snapshot=mysnapshot) \\
model2.sync\_set\_py\_values\_to\_snapshot(snapshot=mysnapshot) \\
model1.run\_session() \\
model2.run\_session()}
\end{codeblock}

The state of \vars{model} after its run session is used to start
\vars{model1} and \vars{model2}.  This is an easy way to save time
in spinning-up multiple models.




%---------------------------------------------------------------------
\section{Run Lists}		\label{sec:runlist.intro}

This feature of \class{Qtcm} objects is what really gives 
\class{Qtcm} model instances their flexibility.
A run list is a list of strings and dictionaries that specify
what routines to run in order to execute a particular part of
the model.  Each element of the run list specifies the method
or subroutine to execute, and the order of the elements specifies
their execution order.

For instance, the standard run list for initializing the the
atmospheric portion of the model is named ``qtcminit'', and
equals the following list:

\begin{latexonly}
\begin{codeblock}
\codeblockfont{%
\parbox{46ex}{\input{qtcminit_runlist}}}
\end{codeblock}
\end{latexonly}

\begin{htmlonly}
\begin{quotation}
\input{qtcminit_runlist}
\end{quotation}
\end{htmlonly}

This list is stored as an entry in the \vars{runlists} dictionary
(with key \vars{'qtcminit'}).
\vars{runlists} is an attribute of a \class{Qtcm} instance.
Table~\ref{tab:stnd.runlists} lists all standard run lists.

When the run list element in the list is a string, the string gives the
name of the routine to execute.  The routine has no parameter
list.  The routine can be a
compiled QTCM1 model subroutine for which an interface has been
written (e.g., \mods{\_\_qtcm.wrapcall.wparinit}), 
a method of the of the Python model instance 
(e.g., \mods{varinit}), or another run list
(e.g., \vars{atm\_physics1}).

When the run list element is a 1-element dictionary, the key of
the dictionary element is the name of the routine, and the value
of the dictionary element is a list specifying input parameters
to be passed to the routine on call.  Thus, the element:
\begin{codeblock}
\codeblockfont{%
{\{'\_\_qtcm.wrapcall.wtimemanager': [1]\}}}
\end{codeblock}
calls the \mods{\_\_qtcm.wrapcall.wtimemanager} routine, passing in
one input parameter, which in this case is the value 1.

If you want to change the order of the run list, just change the
order of the list.  To add or remove routines to be executed, just
add and remove their names from the run list.
Python provides a number of methods to manipulate
lists (e.g., \mods{append}).  Since lists are dynamic data types
in Python, you do not have to do any recompiling to implement
the change.

The \vars{compiled\_form} attribute must be set to \vars{'parts'}
in the \class{Qtcm} instance in order to take advantage of the run
lists feature of the class.  Run lists are not available for
\vars{compiled\_form\thinspace=\thinspace'full'}, because subroutine
calls are hardwired in the compiled QTCM1 model Fortran code in
that case.




%---------------------------------------------------------------------
\section{Model Output}			\label{sec:output.intro}

	\subsection{NetCDF Output}

Model output is written to netCDF files in the directory
specified by the \class{Qtcm} instance attribute \vars{outdir}.
Mean values are written to an output file beginning with
\fn{qm\_}, and instantaneous values are written to an output
file beginning with \fn{qi\_}.

The frequency of mean output is controlled by \vars{ntout}, and the
frequency of instantaneous output is controlled by \vars{ntouti}.
\vars{ntout.value} gives the number of days over which to average
(and if equals \vars{-30}, monthly means are calculated).
\vars{ntouti.value} gives the frequency in days that instantaneous
values are output (monthly if it equals \vars{-30}).  (See
Section~\ref{sec:initial.variables} for a description of other
output-control variables, and see the QTCM1 manual \cite{Neelin/etal:2002}
for a detailed description of how these variables control output.)

Figure~\ref{fig:netcdf.read} gives an example of a block of code
to read netCDF output, where \vars{datafn} is the netCDF filename, and
\vars{id} is the string name of the field variable (e.g.,
\vars{'u1'}, \vars{'T1'}, etc.).
(Note that the netCDF identifier for field variables is the same as
the name in \class{Qtcm}, except for the variables given in
Table~\ref{tab:qtcm.netcdf.ids}.)

In the code in Figure~\ref{fig:netcdf.read},
the array value is read into \vars{data}, and the longitude values, 
latitude values, and time values are read into variables
\vars{lon}, \vars{lat}, and \vars{time}, respectively.
As netCDF files also hold metadata, a description and the units
of the variable given by \vars{id}, and each dimension, are read
into variables ending in \vars{\_name} and \vars{\_units},
respectively.


%--- Two versions, one for PDF, one for HTML:
\begin{latexonly}
\begin{figure}[htp]
\begin{codeblock}
\codeblockfont{%
import numpy as N \\
import Scientific as S \\ \\
fileobj = S.NetCDFFile(datafn, mode='r') \\ \\
data = N.array(fileobj.variables[id].getValue()) \\
data\_name = fileobj.variables[id].long\_name \\
data\_units = fileobj.variables[id].units \\ \\
lat = N.array(fileobj.variables['lat'].getValue()) \\
lat\_name = fileobj.variables['lat'].long\_name \\
lat\_units = fileobj.variables['lat'].units \\ \\
lon = N.array(fileobj.variables['lon'].getValue()) \\
lon\_name = fileobj.variables['lon'].long\_name \\
lon\_units = fileobj.variables['lon'].units \\ \\
time = N.array(fileobj.variables['time'].getValue()) \\
time\_name = fileobj.variables['time'].long\_name \\
time\_units = fileobj.variables['time'].units \\ \\
fileobj.close()}
\end{codeblock}

\caption{Example of Python code to read netCDF output.
	See text for description.}
\label{fig:netcdf.read}
\end{figure}
\end{latexonly}

\begin{htmlonly}
\label{fig:netcdf.read}
\begin{center}
\htmlfigcaption{%
	\codeblockfont{%
import numpy as N \\
import Scientific as S \\ \\
fileobj = S.NetCDFFile(datafn, mode='r') \\ \\
data = N.array(fileobj.variables[id].getValue()) \\
data\_name = fileobj.variables[id].long\_name \\
data\_units = fileobj.variables[id].units \\ \\
lat = N.array(fileobj.variables['lat'].getValue()) \\
lat\_name = fileobj.variables['lat'].long\_name \\
lat\_units = fileobj.variables['lat'].units \\ \\
lon = N.array(fileobj.variables['lon'].getValue()) \\
lon\_name = fileobj.variables['lon'].long\_name \\
lon\_units = fileobj.variables['lon'].units \\ \\
time = N.array(fileobj.variables['time'].getValue()) \\
time\_name = fileobj.variables['time'].long\_name \\
time\_units = fileobj.variables['time'].units \\ \\
fileobj.close()}
	}

\htmlfigcaption{Figure~\ref{fig:netcdf.read}:
	Example of Python code to read netCDF output.
	See text for description.}
\end{center}
\end{htmlonly}





\begin{table}[tp]
\begin{center}
\begin{tabular}{l|l}
\textbf{\class{Qtcm} Attribute Name} & \textbf{NetCDF Output Name} \\
\hline
\vars{'Qc'}                & \vars{'Prec'} \\
\vars{'FLWut'}             & \vars{'OLR'} \\
\vars{'STYPE'}             & \vars{'stype'}
\end{tabular}
\end{center}
\caption{NetCDF output names for \class{Qtcm} field variables that
	are different from the \class{Qtcm} and compiled QTCM1 model
	variable names.  The netCDF names are case-sensitive.}
\label{tab:qtcm.netcdf.ids}
\end{table}


\emphpara{NB:}  All netCDF array output is dimensioned (time, latitude,
longitude) when read into Python using the \mods{Scientific} package.
This differs from the way \class{Qtcm} saves field variables, which
follows Fortran convention (longitude, latitude).  Please be careful
when relating the two types of arrays.
Section~\ref{sec:field.var.shape} for a discussion of why there is
this discrepancy.


	\subsection{Visualization}	\label{sec:viz.intro}

The \mods{plotm} method of \class{Qtcm} instances creates line
plots or contour plots, as appropriate, of model output of
average fields of run session(s) associated with the instance.
Some examples, assuming \vars{model} is an instance of \class{Qtcm}
and has already executed a run session:
\begin{itemize}
\item \cmd{model.plotm('Qc', lat=1.875)}:
	A time vs.\ longitude contour
          plot is made for the full range of time and longitude,
          at the latitude 1.875 deg N, for mean precipitation.

\item \cmd{model.plotm('Qc', time=10)}:
	A latitude vs.\ longitude contour plot of precipitation
	is made for the full spatial domain at day 10 of the model run.

\item \cmd{model.plotm('Evap', lat=1.875, lon=[100,200])}:  A contour
	plot of time vs.\ longitude of evaporation is made for the
          longitude points between 100 and 200 degrees E, at the
          latitude 1.875 deg N.  

\item \cmd{model.plotm('cl1', lat=1.875, lon=[100,200], time=20)}:
          A deep cloud amount vs.\ longitude line plot is made for
          the longitude points between 100 and 200 degrees east,
          at the latitude 1.875 deg N, at day 20 of the model run.
\end{itemize}

In these examples, the number of days over which the mean is taken
equals \vars{model.ntout.value}.
Also, the \mods{plotm} method automatically takes into account the
\class{Qtcm}/netCDF variable differences described in
Table~\ref{tab:qtcm.netcdf.ids}.



%---------------------------------------------------------------------
\section{Documentation}

Section~\ref{sec:ver} gives the online locations of the
transparent copies of this manual.  
Model formulation is fully described in
Neelin \& Zeng \cite{Neelin/Zeng:2000} and model
results are described in Zeng et~al.\ \cite{Zeng/etal:2000}
(\cite{Neelin/Zeng:2000} is based upon v2.0 of QTCM1
and \cite{Zeng/etal:2000} is based on QTCM1 v2.1).
Additional documentation you'll find useful include:

\begin{itemize}
\item \latexhtml{%
\htmladdnormallinkfoot{The \mods{qtcm} Package API Documentation}%
        {http://www.johnny-lin.com/py\_pkgs/qtcm/doc/html-api/}}%
{\htmladdnormallink{The \mods{qtcm} Package API Documentation}%
        {http://www.johnny-lin.com/py_pkgs/qtcm/doc/html-api/}}

\item \latexhtml{%
\htmladdnormallinkfoot{The Pure-Fortran QTCM1 Manual}%
        {http://www.atmos.ucla.edu/$\sim$csi/qtcm\_man/v2.3/qtcm\_manv2.3.pdf}}%
{\htmladdnormallink{The Pure-Fortran QTCM1 Manual}%
        {http://www.atmos.ucla.edu/~csi/qtcm_man/v2.3/qtcm_manv2.3.pdf}}
\cite{Neelin/etal:2002}

\end{itemize}



% ===== end of file =====


\chapter{Using \mods{qtcm}}                 \label{ch:using}
% ==========================================================================
% Using QTCM
%
% By Johnny Lin
% ==========================================================================


% ------ BODY -----
%
%---------------------------------------------------------------------
\section{Introduction}

Now that you've successfully run your first model instances, in
this chapter I provide detailed explanations regarding the features
of \mods{qtcm}.  I present these explanations in a documentary
rather than didactic fashion; my goal is to document how the features
work.  More details are given in the code docstrings.  At the end
of the chapter, in Section~\ref{sec:cookbook}, I provide a few
cookbook ideas/examples of ways to use the model.




%---------------------------------------------------------------------
\section{Model Instances}  \label{sec:model.instances}

An instance of a \class{Qtcm} model is created in \mods{qtcm} the same way
you create an instance of any class.
For instance, to instantiate two \class{Qtcm}
models, \vars{model1} and \vars{model2}, I type the following:

\begin{codeblock}
\codeblockfont{%
from qtcm import Qtcm \\
model1 = Qtcm(compiled\_form\thinspace=\thinspace'full') \\
model2 = Qtcm(compiled\_form\thinspace=\thinspace'parts')}
\end{codeblock}

In the above example, \vars{model1} uses the compiled QTCM1 model
that runs the model (essentially) using the Fortran driver,
while \vars{model2} uses the compiled QTCM1 model where execution
order and content all the way down to the atmospheric timestep level
is controlled by Python run lists.  (Section~\ref{sec:compiledform}
has more details about the difference between compiled forms.)

For each instance of \class{Qtcm}, copies of all needed extension
modules (e.g., \fn{.so} files) are copied to a temporary directory
that is automatically created by Python.  The full path name of
that directory is saved in the instance attribute \vars{sodir}.
These extension modules are then associated with the specific instance 
through private instance attributes,
and thus every instance of \class{Qtcm} has its own separate variable
and name space on both the Fortran and Python sides.\footnote%
	{The private instance attribute is \vars{\_\_qtcm}.
	See Section~\ref{sec:Qtcm.private.attrib} for details about 
	private \class{Qtcm} instance attributes.}
The temporary directory and all of its contents are deleted when the 
model instance is deleted.

On instantiation, \class{Qtcm} instances set all scalar field
variables to their default values as given in the submodule
\mods{defaults} (and listed in Section~\ref{sec:defaults.scalar}),
and assign the fields as instance attributes.  The instance attribute
\vars{init\_with\_instance\_state} is set to True by default, unless
overridden on instantiation.




%---------------------------------------------------------------------
\section{Initializing a Model Run}

In the pure-Fortran QTCM1, there are three broad
classes of initialized variables:
\begin{enumerate}
\item Those that are read-in using a namelist, 
\item Those that the are read-in from a restart file, and
\item Those that are set by assignment in the Fortran code.  
\end{enumerate}
These variables are a combination of scalars and arrays.

For \mods{qtcm}, interfaces were built so that all classes of
initialized variables that could be user-controlled are accessible
and changeable at the Python-level.  For \mods{qtcm},
the set of variables that could be changed is also expanded, to
include not just the first and second classes of pure-Fortran
QTCM1 initialized variables.  This was done to make \mods{qtcm}
more flexible.  All variables that can be passed between the
compiled QTCM1 model and Python model levels are called
field variables, and are described in detail in
Section~\ref{sec:field.variables}.

As it happens, all the namelist-set variables are scalars.  In the
pure-Fortran QTCM1, those variables are given default values prior
to reading in of the namelist.  To duplicate this functionality,
on model instantiation, all scalar fields are set to their default
values as given in the submodule \mods{defaults} and listed in
Section~\ref{sec:defaults.scalar}.  Most of the default values in
\mods{defaults} are the same as in the pure-Fortran QTCM1, but
there are a few differences.\footnote%
	{One difference being \vars{mrestart}, which in \vars{qtcm} 
	will have the value of 0 in contrast to the pure-Fortran 
	QTCM1 where the default is the 1.}
This setting of scalar defaults is the same for both
\vars{compiled\_form\thinspace=\thinspace'full'} and
\vars{compiled\_form\thinspace=\thinspace'parts'} instances.
Of course, all
\mods{qtcm} fields are user-controllable, both via keyword input
parameters at model instantiation as well as through direct
manipulation of the instance attribute that stores the field variable.

The pure-Fortran QTCM1 initialized prognostic variables and
right-hand sides are set in the Fortran subroutine \mods{varinit}.
Their they are read-in from a restart file or, as default,
set by assignment.
In \mods{qtcm}, the same variables are initialized by a \class{Qtcm}
instance method of the same name, \mods{varinit}, for the case when
\vars{compiled\_form\thinspace=\thinspace'parts'}.  For the case
of \vars{compiled\_form\thinspace=\thinspace'full'}, the compiled
QTCM1 subroutine that is the same as in the pure-Fortran case is
used, and that routine is inaccessible at the Python level.
See Section~\ref{sec:snapshots}'s listing of snapshot variables,
which also includes the prognostic variables and right-hand sides that
are set in \mods{varinit} (both Fortran and Python).




%---------------------------------------------------------------------
\section{The \vars{compiled\_form} Keyword}  \label{sec:compiledform}

The \mods{qtcm} package is a Python wrap of the Fortran routines
that make up QTCM1.  The wrapping layer adds flexibility and
functionality, but at the cost of speed.  Thus, I created two
types of extension modules from the Fortran QTCM1 code, one
which permits very little control over the compiled Fortran
\emph{routines} at the Python level, and one that allows the Python-level
to control model execution in the compiled QTCM1 model
all the way down to the atmospheric timestep level.\footnote%
	{That control is via run lists, which are described in
	Section~\ref{sec:runlists}.}
The former extension module corresponds to 
\vars{compiled\_form\thinspace=\thinspace'full'} and
the latter extension module to
\vars{compiled\_form\thinspace=\thinspace'parts'}.

For \vars{compiled\_form\thinspace=\thinspace'full'},
the compiled portion of the model encompasses (nearly) the
entire QTCM1 model as a whole.  Thus, the only compiled QTCM1 model
modules or subroutines that Python should interact with is
the \mods{driver} routine (which executes the entire model) and
the \mods{setbypy} module (which enables communication between the
compiled model and the Python-level of model fields.\footnote%
	{The \mods{setbypy} Python module is the wrap of the
	Fortran QTCM1 \mods{SetByPy} module.}

For \vars{compiled\_form\thinspace=\thinspace'parts'}, the compiled
portion of the model does not encompasses the model as a whole, but
rather is broken up into separate units (as appropriate) all the
way down to an atmosphere timestep.  Thus, compiled QTCM1 model
modules/subroutines that are accessible at the Python-level include
those that are executed within an atmosphere timestep on up.

Because the difference in compiled forms fundamentally affects how
the \class{Qtcm} instance facilitates Python-Fortran communication,
this attribute must be set on instantiation via a keyword input
parameter.

In the rest of this section, to avoid being verbose, when I
write \vars{'full'}, I mean the situation where
\vars{compiled\_form\thinspace=\thinspace'full'}.
Likewise, when I
write \vars{'parts'}, I mean the situation where
\vars{compiled\_form\thinspace=\thinspace'parts'}.


	\subsection{Initialization for 
			\vars{compiled\_form\thinspace=\thinspace'full'}}
				\label{sec:init.compiledform.full}

For a model run of this case, the \class{Qtcm} instance will
initialize the model using the Fortran \mods{varinit} subroutine
in the compiled QTCM1 model.  This subroutine does the following:

\begin{itemize}
\item If \vars{mrestart\thinspace=\thinspace1}, 
	the restart file is used to initialize all prognostic
	variables.  In terms of start date, the following rules are
	used:
	\begin{enumerate}
	\item Variable \vars{dateofmodel} is read from the restart file.
	\item If \vars{day0}, \vars{month0}, and \vars{year0}
		are negative, or otherwise
		invalid (e.g., \vars{month0} greater than 12), the invalid
		value is replaced with the
		day, month, and/or year of the day \emph{after} 
		that given by \vars{dateofmodel}.
		If the value of \vars{day0}, \vars{month0}, or \vars{year0}
		is not invalid in this sense, it is not replaced.
	\end{enumerate}
	Thus, if the restart file gives 
	\vars{dateofmodel} equal to 101102
	(year 10, month 11, day 2), and 
	\vars{day0\thinspace=\thinspace-1}, 
	\vars{month0\thinspace=\thinspace-1}, 
	\vars{year0\thinspace=\thinspace-1},
	and 
	\vars{mrestart\thinspace=\thinspace1}, 
	the model will start running from year 10, month 11, day 3.
	If \vars{dateofmodel} equals to 101102, and 
	\vars{day0\thinspace=\thinspace-1}, 
	\vars{month0\thinspace=\thinspace3}, 
	\vars{year0\thinspace=\thinspace-1},
	the model will start running from year 10, month 3, day 3.

\item If \vars{mrestart\thinspace=\thinspace0}, 
	all prognostic variables and right-hand sides are set to an
	initial value (which for most of those variables is zero).
	In terms of start date, \vars{day0} is set to 1 (and thus 
	the value of \vars{day0} previously input is ignored), and
	both \vars{month0} and \vars{year0}
	are set to 1 
	if their previously input values are invalid (where
	invalid means less than
	1, or, for \vars{month0}, greater than 12).
	Otherwise, \vars{month0} and \vars{year0} are left unchanged.
	Variable \vars{dateofmodel} has the value it had when the variable
	was declared (which is determined by the compiler and usually
	is zero; \vars{dateofmodel} will not be properly set until
	subroutine \mods{TimeManager} is called.

	Thus, if 
	\vars{day0\thinspace=\thinspace-1},
	\vars{month0\thinspace=\thinspace-1}, 
	\vars{year0\thinspace=\thinspace-1} is input into the model
	(say from a namelist) and 
	\vars{mrestart\thinspace=\thinspace0},
	the model will start running from year 1, month 1, day 1,
	and \vars{dateofmodel} at the exit of subroutine 
	\mods{varinit} will equal its compiler-set default.
	If 
	\vars{day0\thinspace=\thinspace14}, 
	\vars{month0\thinspace=\thinspace3}, 
	\vars{year0\thinspace=\thinspace11}, and 
	\vars{mrestart\thinspace=\thinspace0} on input into the
	model,
	the model will start running from year 11, month 3, day 1,
	and \vars{dateofmodel} at the exit of subroutine 
	\mods{varinit} will equal its compiler-set default.

	Note that \vars{dateofmodel}
	can thus be inconsistent with 
	\vars{month0} and \vars{year0} at the
	exit of subroutine \mods{varinit}.
\end{itemize}

This behavior with respect to initializing
the start date is different than in QTCM1 versions 1.0 and 2.1.
Please see the source code from those earlier QTCM1 versions for
details.




	\subsection{Initialization for 
			\vars{compiled\_form\thinspace=\thinspace'parts'}}
				\label{sec:init.compiledform.parts}

For \vars{'parts'} model, the methodology of how initialized
prognostic variables, right-hand sides, and start date related
variables are set is controlled by the \class{Qtcm} instance
attribute/flag \vars{init\_with\_instance\_state}.  The initialization
is (mostly) executed in the Python \vars{varinit} method in the
following way:

\begin{itemize}
\item If \vars{init\_with\_instance\_state} is False:
The method as described for
initialization for the 
\vars{'full'} case is generally
followed, with the exception that dateofmodel is set
to match \vars{day0}, \vars{month0}, \vars{year0}, prior to exit of 
\mods{varinit}.

\item If \vars{init\_with\_instance\_state} is True:
the model object will initialize the model based on the current
state of the model instance.  This enables you to set a model run
session's initial conditions based upon the state of the prognostic
variables and parameters stored at the Python level, which is
accessible at runtime.
\end{itemize}


Since the \vars{init\_with\_instance\_state\thinspace=\thinspace{False}}
case is mainly described by the initialization method for the
\vars{'full'} case, I refer the
reader to Section~\ref{sec:init.compiledform.full}.
For the case of \vars{init\_with\_instance\_state} is True, however,
the task is more complicated.  Specifically, for that case,
initialization includes the following:

\begin{enumerate}
\item If not currently defined,
	variable \vars{dateofmodel} is set to a default value of 0,
	which is specified in the module defaults.

\item The \vars{mrestart} flag is ignored for variable initialization.

\item All prognostic variables and right-hand sides
        are set to an
        initial value (which for most of those variables is zero),
	unless the variable is defined at the Python level, in which
	case the inital value is set to the Python level defined value.

\item If \vars{dateofmodel} is greater than 0, 
	\vars{day0}, \vars{month0}, and \vars{year0} are overwritten
        with values derived from \vars{dateofmodel} 
	in order to set the run to start
	the day \emph{after} \vars{dateofmodel}.

\item If \vars{dateofmodel} is less than or equal to 0, \vars{day0},
	\vars{month0}, and \vars{year0} are set to their respective
	instance attribute values, if valid.  For invalid instance
	attribute values, the invalid \vars{day0}, \vars{month0},
	and/or \vars{year0} is set to 1.

\item Variable \vars{dateofmodel} is recalculated
	and overwritten to match 
	\vars{day0}, \vars{month0}, \vars{year0}, prior to exit of 
	\mods{varinit}.
\end{enumerate}

As a result, for \vars{init\_with\_instance\_state} is True, the
way you indicate to the model that a run session is a brand-new run
is by setting, before the \mods{run\_session} method call,
\vars{dateofmodel} to a value less than or equal to 0, and \vars{day0},
\vars{model0}, and \vars{year0} to the day you want the model to
begin the run session.  To indicate to the model you wish to continue
a run, set \vars{dateofmodel} to the day \emph{before} you want the
model to start running from.

Examples:
\begin{itemize}
\item If \vars{day0\thinspace=\thinspace-1}, 
	\vars{month0\thinspace=\thinspace-1}, 
	\vars{year0\thinspace=\thinspace-1}, and
	\vars{dateofmodel\thinspace=\thinspace0} is input into 
	the model the model will start running from year 1, month 1, day 1,
	and 
	variable \vars{dateofmodel} at the exit of 
	subroutine \mods{varinit}
	will equal 10101.

\item If \vars{day0\thinspace=\thinspace14},
	\vars{month0\thinspace=\thinspace3}, 
	\vars{year0\thinspace=\thinspace11},
	and \vars{dateofmodel\thinspace=\thinspace0} is input into the
	model, the model will start running from year 11, month 3, day 14,
	and 
	variable \vars{dateofmodel} at the exit of 
	subroutine \mods{varinit} will equal
	110314.

\item If \vars{day0\thinspace=\thinspace14},
	\vars{month0\thinspace=\thinspace3}, 
	\vars{year0\thinspace=\thinspace11},
	and \vars{dateofmodel\thinspace=\thinspace341023} is input into the
	model, the model will start running from year 34, month 10, day 24,
	and at the exit of subroutine 
	\mods{varinit}, \vars{dateofmodel} will equal
	341024, with \vars{day0\thinspace=\thinspace24},
	\vars{month0\thinspace=\thinspace10}, and
	\vars{year0\thinspace=\thinspace34}.
\end{itemize}


	\subsection{Communication Between Python and Fortran-Levels}
				\label{sec:comm.py.fort.compiledform}

After initialization, the second major difference between a
\vars{'full'} and \vars{'parts'} model is how and when communication
between the Python and Fortran levels can occur.  For the \vars{'full'}
case, except for the passing in and out of variables before and after
a run session, all variable passing and subroutine calling happens in
the compiled QTCM1 model, with no control at the Python level.
For the \vars{'parts'} case, variables can be passed between the
Python and Fortran-levels at all levels down to the atmospheric
timestep, and many Fortran QTCM1 subroutines can be called from the
Python-level.  


		\subsubsection{Passing Variables}

For all \vars{compiled\_form} cases, variables are passed back and
forth between the Python \class{Qtcm} instance level and the
compiled QTCM1 model Fortran-level using the \class{Qtcm}
instance methods \mods{get\_qtcm1\_item} and \mods{set\_qtcm1\_item}:\footnote%
	{All Fortran routines used to pass variables back and forth are
	defined in the \mods{setbypy} module of the \fn{.so} extension
	module stored in the \class{Qtcm} instance variable \vars{\_\_qtcm}.
	All Fortran wrappers that enable Python to call compiled QTCM1 model
	subroutines are defined in the \mods{wrapcall} module stored in
	the \class{Qtcm} instance variable \vars{\_\_qtcm}.
	These modules are described in detail in 
	Sections~\ref{sec:setbypy} and~\ref{sec:wrapcall}, respectively.}

\begin{itemize}
\item \mods{get\_qtcm1\_item}(\dumarg{key}):
	Returns the value of the field variable given by the string
	\dumarg{key}.  If the compiled QTCM1 model variable given by
	\dumarg{key} is unreadable, the
        custom exception 
	\vars{FieldNotReadableFromCompiledModel} is thrown.
	The value returned is a copy of the value on the Fortran
	side, not a reference to the variable in memory.

\item \mods{set\_qtcm1\_item}:
	Sets the value of a field variable
	in the compiled QTCM1 model \emph{and at the Python-level,}
	automatically overriding any previous value at both levels.
	Thus, calling this method will change/create the \class{Qtcm}
	instance attribute corresponding to the field variable.
        When the compiled QTCM1 model variable is set, a copy of the
        Python value is passed to the Fortran model; the
	variable is \emph{not passed by reference.}
	This value comes from the \mods{set\_qtcm1\_item} calling
	parameter list, \emph{not} from the \class{Qtcm}
        instance attribute corresponding to the field variable.
\end{itemize}

The \mods{set\_qtcm1\_item} method has two calling forms, one with
one argument and the other with two arguments:
\begin{itemize}
\item One argument:  The method is called
	as \mods{set\_qtcm1\_item}(\dumarg{arg}), where \dumarg{arg} 
	is either a string giving the name of the field variable or 
	a \class{Field} instance.

\item Two arguments:  The method is called as
	\mods{set\_qtcm1\_item}(\dumarg{key}, \dumarg{value}), where
	\dumarg{key} is the string giving the name of the field variable
	and \dumarg{value} is the value to set the model field variable to
	(note \dumarg{value} can be a \class{Field} instance).
\end{itemize}
In either calling form, if no value given, the default value as defined
in module \mods{defaults} is used.

Some compiled QTCM1 model variables are not in a state where they
can be set.  An example is a compiled QTCM1 model pointer variable,
prior to the pointer being associated with a target (an attempt
to set would yield a bus error).  In such cases, the
\mods{set\_qtcm1\_item} method will throw a
\vars{FieldNotReadableFromCompiledModel} exception, nothing will
be set in the compiled QTCM1 model, and the Python counterpart
field variable (if it previously existed) would be left unchanged.\footnote%
	{We handle this situation in this way to enable the
	\class{Qtcm} instance to store variables
	even if the compiled model is not yet ready to accept them.}

Examples, typed in at a Python prompt, and
assuming that \vars{model} is a \class{Qtcm} instance:
\begin{itemize}
\item \cmd{dtvalue\thinspace=\thinspace{model.get\_qtcm1\_item('dt')}}:
	Retrieves the value of field variable \vars{dt} (timestep)
	from the compiled QTCM1 Fortran model and sets it to the
	Python variable \vars{dtvalue}.

\item \cmd{model.set\_qtcm1\_item('dt')}:
	Sets the value of field variable \vars{dt}
	in the compiled QTCM1 Fortran model to the default
	value (as given in \mods{defaults}),
	and sets the value of Python attribute \vars{model.dt} also to 
	that default value.  
	Remember that \vars{model.dt} is a \class{Field}
	instance.

\item \cmd{model.set\_qtcm1\_item('dt', 2000.)}:
	Sets the value of field variable \vars{dt}
	in the compiled QTCM1 Fortran model to 2000 (as a real),
	and sets the value of Python attribute \vars{model.dt} also to 2000.
\end{itemize}


		\subsubsection{Calling Compiled QTCM1 Model Subroutines}

All compiled QTCM1 model subroutines that can be called
(except \mods{driver} and \mods{varptrinit}) are in the
\mods{setbypy} or \mods{wrapcall} modules
of the \class{Qtcm} instance private attribute \vars{\_\_qtcm}.
(On \class{Qtcm} instance instantiation, \vars{\_\_qtcm} is set
to the \fn{.so} extension module that is the compiled QTCM1 Fortran model.)
Thus, to call \mods{wmconvct} in \mods{wrapcall} at the Python-level,
just type \cmd{model.\_\_qtcm.wrapcall.wmconvct()} (where \vars{model}
is a \class{Qtcm} instance).
For \mods{driver} and \mods{varptrinit}, these subroutines are not
contained in a \vars{\_\_qtcm} module, and thus can be called
directly (e.g., just type \cmd{model.\_\_qtcm.driver()}).
See Sections~\ref{sec:setbypy} and~\ref{sec:wrapcall} for more information
on the \mods{setbypy} and \mods{wrapcall} modules.

For the \vars{'full'} case, the only compiled QTCM1 model
subroutine you can usefully call during a run session is \mods{driver}.
For the \vars{'parts'} case, while you can essentially call any subroutine
given in a run list, you usually will not directly call a compiled QTCM1
model subroutine but will instead call it through including it in a
run list.  For example, if you have the following run list in a
\vars{'parts'} model:
\begin{codeblock}
\codeblockfont{%
[ 'qtcminit', '\_\_qtcm.wrapcall.woutpinit' ]}
\end{codeblock}
Running this list using the \class{Qtcm} instance method
\mods{run\_list} will result in \class{Qtcm} instance method
\mods{qtcminit} first being run, 
then the compiled QTCM1 Fortran model subroutine
\mods{woutpinit} in Fortran module \mods{wrapcall} being run.
See Section~\ref{sec:runlists} and
Table~\ref{tab:stnd.runlists} for a discussion and list of the
standard run lists that control routine execution content and order
in the \vars{'parts'} case.




%---------------------------------------------------------------------
\section{Restart and Continuation Run Sessions}
				\label{sec:contination.run.sessions}


	\subsection{Restart Runs In the Pure-Fortran QTCM1}
					\label{sec:puref90.restart}

To enable restart of a model run, the pure-Fortran QTCM1 model
writes out a restart file with the state of the prognostic variables
and select right-hand sides at that point in the run (for a list
of the variables, see Section~\ref{sec:snapshots}).  This binary
file can then be read in by later model runs.  The Fortran
\vars{mrestart} flag is passed in via a namelist; if \vars{mrestart}
is 1, the run uses the restart file (named \fn{qtcm.restart}).

One of the problems with using the restart file to do a continuation
run is that the continuation run will not be perfect.  In other words,
a 15~day run followed by a 25~day run based on the restart file 
generated at the end of the 15~day run will \emph{not} give the
exact same output as a continuous 40~day run.


	\subsection{Overview of Restart/Continuation Options In \mods{qtcm}}
					\label{sec:restart.options.list}

For a \class{Qtcm} instance, in contrast to the pure-Fortran QTCM1,
more than one method of continuation is available.
Thus, for a continuation run, you need to tell the model
``continue from what?''
The \class{Qtcm} class provides three choices for restart/continuing
a run:
\begin{enumerate} 
\item From a restart file:  Move/rename a QTCM1 restart file
        to the current working directory to \fn{qtcm.restart}.
	\label{list:continue.from.restart}

\item From a snapshot from another run session
	(see Sections~\ref{sec:snapshot.intro} and~\ref{sec:snapshots}).
	\label{list:continue.from.snapshot}

\item From the values of the \class{Qtcm} instance you will be
	calling \mods{run\_session} from.
	\label{list:continue.from.instance}
\end{enumerate}

Restart/continuation methods~\ref{list:continue.from.restart} 
and~\ref{list:continue.from.snapshot} both suffer from the
same problem as the pure-Fortran QTCM1 restart process:
They do not produce perfect restarts
(see Section~{sec:puref90.restart} for details).
In this section, I discuss the restart/continuation options
for each \vars{compiled\_form} option.

Methods~\ref{list:continue.from.restart}
and~\ref{list:continue.from.snapshot} are best used when making a
run session from a newly instantiated \class{Qtcm} instance.
Method~\ref{list:continue.from.instance} is best used when executing
a run session using a \class{Qtcm} instance that has already gone
through at least one run session.  Regardless of which method you
use, however, please note that anytime you execute a run session
using a \class{Qtcm} instance that already has made a previous run
session, some variables \emph{cannot be updated} between run sessions.
This feature is most noticeable with the output filename, and occurs
because the name persists in the compiled QTCM model, and is stored
in the extension module (\fn{.so} files in \vars{sodir}) associated
with the instance.  If you wish to control all variables possible
from the Python level (including output filename), you need do the
run session from a new model instance.


	\subsection{Restart/Continuation for 
		\vars{compiled\_form\thinspace=\thinspace'full'} 
		Model Instances}

The only option for restart when using
\vars{compiled\_form\thinspace=\thinspace'full'} model instances
is method~\ref{list:continue.from.restart}, to use a QTCM1 restart
file.\footnote%
	{The \vars{cont} keyword parameter in \mods{run\_session}
	and the value of the \vars{init\_with\_instance\_state}
	attribute have no effect if
	\vars{compiled\_form\thinspace=\thinspace'full'}.  With
	\vars{'full'}, the call to initialize variables all happens
	at the Fortran level (via the Fortran \mods{varinit}, not
	the Python \mods{varinit}), with no reference to the Python field
	states (or even existing Fortran field states, if present).}
To use this option, the value of the \vars{mrestart} 
attribute must equal 1, the restart file must be named
\fn{qtcm.restart}, and the restart file must be in the 
current working directory.
As with the pure-Fortran QTCM1 restart process, this method
does not produce perfect restarts.



	\subsection{Restart/Continuation for 
		\vars{compiled\_form\thinspace=\thinspace'parts'} 
		Model Instances}

For the \vars{compiled\_form\thinspace=\thinspace'parts'} case,
all three restart/continuation methods
described in Section~\ref{sec:restart.options.list} are
available.


		\subsubsection{Method~\ref{list:continue.from.restart}:
			From a QTCM1 Restart File}

To use the QTCM1 restart file mechanism, not only must the
\vars{mrestart} attribute have a value to 1, but the
\vars{init\_with\_instance\_state} flag also has to be \vars{False},
otherwise the \vars{mrestart} attribute value will be ignored.  
As with the pure-Fortran QTCM1 restart process, this method does not
produce perfect restarts.


		\subsubsection{Method~\ref{list:continue.from.snapshot}:
			From a \class{Qtcm} Instance Snapshot}

You can take snapshots of the model state of a \class{Qtcm} instance
by the \mods{make\_snapshot} instance method.  This snapshot saves
a copy of all the variables saved to a QTCM1 restart file (see
Section~\ref{sec:snapshots} for the full list of fields), which
then can be passed to other \class{Qtcm} instances for use in other
run sessions.

The key difference between this method and 
method~\ref{list:continue.from.instance} (described below)
is that \mods{run\_session} calls using the snapshot are done
\emph{without} the \vars{cont} keyword input parameter
(by default, \vars{cont} is False).  If the \vars{cont} keyword
is not False, it says the run session is a continuation run
that uses the state of the compiled QTCM1 model for all variables
that are not specified at, and read-in from,
the Python level.  If the \vars{cont} keyword
is False, the run session initializes as if it were a new run.

See Section~\ref{sec:snapshot.intro} for details and
an example of using snapshots to initialize a run session.
Note that as with the pure-Fortran QTCM1 restart process, this method 
does not produce perfect restarts.


		\subsubsection{Method~\ref{list:continue.from.instance}:
			From the Calling \class{Qtcm} Instance}

This method is used when you want to make a run session that is a
``true'' continuation run, i.e., one that uses the current state
of the compiled QTCM1 model for all variables that are not read-in
from the Python level (remember that \class{Qtcm} instances hold a
subset of the variables defined at the Fortran level).  
The key reason to use this method for a continuation run session
is that the continuation is byte-for-byte the same (if no fields
are changed) as if the run just went straight on through.  Thus,
the continuation would be perfect: A 15~day run followed by a 25~day
run using the same \class{Qtcm} instance with the \vars{cont} keyword
will give the exact same output as a continuous 40~day run.  This
is not the case when making a new instance and passing a restart
file or a snapshot, because a separate extension module is used for
those new instances.

Control of this method is accomplished through the \vars{cont}
keyword input parameter to the \mods{run\_session} method and the
\vars{init\_with\_instance\_state} attribute of a
\class{Qtcm} instance:

\begin{itemize}
\item \vars{cont}: If set to False, the run session is not a
	continuation of the previous run, but a new run session.
	If set to True, the run session is a continuation of the
	previous run session.  If set to an integer greater than
	zero, the run session is a continuation just like
	\vars{cont\thinspace=\thinspace{True}}, but the value
	\vars{cont} is set to is used for \vars{lastday} and replaces
	\vars{lastday.value} in the \class{Qtcm} instance.

\item \vars{init\_with\_instance\_state}:
	If True, for a \mods{run\_session} call using the
	\vars{cont} keyword, whatever the field values are in the Python
	instance are used in the run session.
	If False, model variables are set and initialized as described in
	Section~\ref{sec:init.compiledform.parts}.  In that case,
	previous compiled QTCM1 model values will likely be overwritten.
	Thus, if you want a continuation run that uses the state of
	all field variables except for those you explicitly change at
	the Python-level, make sure \vars{init\_with\_instance\_state}
	is True.
\end{itemize}

(Note that the \vars{cont} keyword has no effect if \vars{compiled\_form}
is \vars{'full'}.  The default value of \vars{cont} in a
\mods{run\_session} call is False.  The value of keyword \vars{cont}
is stored as private instance attribute \vars{\_cont}, in case you
really need to access it elsewhere; see
Section~\ref{sec:Qtcm.private.attrib} for more details).

The example described in Section~\ref{sec:continuation.intro} is
an example of method~\ref{list:continue.from.instance} in the list
above: The second run session is continued from the state of
\vars{model}, with the values of \vars{model}'s instance variables
overriding any values in the compiled QTCM1 model in initializing
the second run session.

This method has a few caveats worthy of note:
\begin{itemize}
\item The \vars{init\_with\_instance\_state} attribute value
	will have no effect unless the instance prognostic variables
	are set, i.e., unless a previous run session has been done.
	Another way to put it is for an initial run session right
	after a \class{Qtcm} instance is created, \mods{varinit}
	will use the same initial values for prognostic variables
	(defined in \mods{defaults} module variable
	\vars{init\_prognostic\_dict})\footnote%
		{\vars{init\_prognostic\_dict} is the dictionary giving
		the default initial values of each prognostic variable
		and right-hand side (as defined by the restart file 
		specification).}
	as it would with for both
	\vars{init\_with\_instance\_state} set to True or False).

\item Continuation run sessions using this method have to continue
	with the next day from wherever the last run session left
	off, contiguously.\footnote%
		{For continuation run sessions, you keep the 
		same extension module (the compiled \fn{.so} library),
		and all the values that define the state where it
		left off.}
	If you want to do a non-contiguous run,
	create a new \class{Qtcm} instance initialized with a
	snapshot instead of the continuation method describe in
	this section.
	will use restart rules to run a new model.  

\item When making a continuation run session using this method,
	you cannot change some variables, for instance,
	\vars{outdir} and any of the date related
	variables.  In fact, the only thing you should change for
	your continuation run session are the prognostic and
	diagnostic variables and \vars{lastday}.  This is because
	some variables cannot be updated between run sessions.
	As noted in Section~\ref{sec:restart.options.list},
	if you wish to control all variables possible
	from the Python level (including output filename), you need 
	to execute the run session from a new model instance.
\end{itemize}


	\subsection{Snapshots of a \class{Qtcm} Instance}
				\label{sec:snapshots}

The snapshot dictionary (briefly described in
Section~\ref{sec:snapshot.intro}), saved as the \class{Qtcm} instance
attribute \vars{snapshot}, and generated by the method
\mods{make\_snapshot}, saves the current state of the following
instance field variables:

\begin{center}
\input{snapshot_vars.tex}
\end{center}

These are the same variables saved to a QTCM1 restart file, and so
a snapshot duplicates the restart functionality in the Python
environment, but with more flexibility.  Since the \vars{snapshot}
dictionary is a Python variable like any other, you can manipulate
it and alter it to fit any condition you wish.




%---------------------------------------------------------------------
\section{Creating and Using Run Lists}  \label{sec:runlists}

Section~\ref{sec:runlist.intro} provides an introduction to the
role and use of run lists.  A run list is a list of methods, Fortran
subroutines, and other run lists that can be executed by the
\class{Qtcm} instance \mods{run\_list} method.  Run lists are stored
in the \class{Qtcm} instance attribute \vars{runlists}, which is a
dictionary of run lists.  The names of run lists should not be
preceeded by two underscores (though elements of a run list may be
very private variables), nor should names of run lists be the same
as any instance attribute.  Run lists are not available for
\vars{compiled\_form\thinspace=\thinspace'full'}.

The \mods{run\_list} method takes a single input parameter, a list,
and runs through that list of elements that specify other run lists
or instance method names to execute.  Methods with private attribute
names are automatically mangled as needed to become executable by
the method.  Note that if an item in the input run list is an
instance method, it should be the entire name (not including the
instance name) of the callable method, separated by periods as
appropriate.

Elements in a run list are either strings or 1-element dictionaries.
Consider the following example, where \vars{model} is a \class{Qtcm}
instance, and \mods{run\_list} is called using \vars{mylist} as
input:

\begin{codeblock}
\codeblockfont{%
model = Qtcm(\ldots) \\
mylist = [ \{'varinit':None\}, \\
\hspace*{13ex}'init\_model', \\
\hspace*{13ex}'\_\_qtcm.driver', \\
\hspace*{13ex}\{'set\_qtcm1\_item': ['outdir', '/home/jlin']\} ]
model.run\_list(mylist)}
\end{codeblock}

The first element in \vars{mylist} refers to a method that requires
no positional input parameters be passed in (as shown by the None).
The second and third elements in \vars{mylist} also refers to methods
that require no positional input parameters be passed in.  The last
element in \vars{mylist} refers to a method with two input parameters.
Note that while I use the term ``method'' to describe the elements,
the strings/keys do not have to be only Python instance methods.
The second element, for instance, refers to another run list, and
the third element refers to a compiled QTCM1 model subroutine (note
the \vars{\_\_qtcm} attribute).

When the \mods{run\_list} method is called, the items in the input
run list are called in the order given in the list.  For each
element,  the \mods{run\_list} method first checks if the string
or dictionary key name corresponds to the key of an entry in the
\class{Qtcm} instance attribute \vars{runlists}.  If so, \mods{run\_list}
is called using that run list (i.e., it is a ``recursive'' call).
If the string or dictionary key name does not refer to another run
list, the \mods{run\_list} method checks if the string or dictionary
key name is a method of the \class{Qtcm} instance, and if so the
method is called.  Any other value throws an exception.

If input parameters for a method are of class \class{Field}, the
\mods{run\_list} method first tries to pass the parameters into the
method as is, i.e., as Field object(s).  If that fails, the
\mods{run\_list } method  passes its parameters in as the \vars{value}
attribute of the \class{Field} object.

If you want a variable that is being passed into a run list to be
continuously updated, you have to set the parameter in the run list
to a \class{Field} instance that is a \class{Qtcm} instance attribute,
not just to the value of the field variable (or to a non-\class{Field}
object).  Otherwise, subsequent calls to that run list element will
not use the updated values as input parameters.

For instance, if you had a run list element:
\begin{codeblock}
\codeblockfont{%
\{'\_\_qtcm.timemanager':[model.coupling\_day,]\}}
\end{codeblock}
and \vars{model.coupling\_day} were an integer (it's not by default,
but pretend it was), then \mods{run\_list} calling
\mods{\_\_qtcm.timemanager} will pass in a scalar integer rather
than a binding to the variable \vars{model.coupling\_day}.  In such
a situation, if the variable \vars{model.coupling\_day} were updated
in time, the \mods{run\_list} call of \mods{\_\_qtcm.timemanager}
would not be updated in time.  This happens because when the
dictionary that is the run list element is created, the value of
list element(s) attached to the dictionary element is set to the
scalar value of \vars{model.coupling\_day} at that instant.

You can get around this feature by setting \class{Qtcm} instance
attributes that will change with model execution to \class{Field}
instances, and then referring to those attributes in the parameter
list in the run list element.  In that case:
\begin{codeblock}
\codeblockfont{%
\{'\_\_qtcm.timemanager':[model.coupling\_day,]\}}
\end{codeblock}
will use the current value of \vars{model.coupling\_day} anytime
\vars{\_\_qtcm.timemanager} is called by \mods{run\_list}, if
\vars{model.coupling\_day} is a \class{Field} object.

When \mods{run\_list}, encounters a calling input parameter that
is a \class{Field} object, it will first try to pass the entire
\class{Field} object to the method/routine being called.  If that
raises an exception, it will then try to pass just the value of the
entire \class{Field} object.  This is done to enable \mods{run\_list}
to be used for both pure-Python and compiled QTCM Fortran model
routines.  Fortran cannot handle \class{Field} objects as input
parameters, only values.

Table~\ref{tab:stnd.runlists} shows all standard run lists
stored in the \vars{runlists} attribute upon instantiation
of a \class{Qtcm} instance.

\begin{htmlonly}
\begin{table}[htp]
\begin{center}
\fbox{Empty placeholder block for table that would have gone here.}
\end{center}
\caption{Standard run lists stored in the \vars{runlists} 
	attribute upon instantiation of a \class{Qtcm} instance.
	The run list and list element names are stored as strings.
	\emphpara{This table is improperly reproduced in the
	HTML conversion.  Please see the PDF version for the table.}}
\label{tab:stnd.runlists}
\end{table}
\end{htmlonly}

\begin{latexonly}
\begin{table}[htp]
\input{runlists}
\caption{Standard run lists stored in the \vars{runlists} 
	attribute upon instantiation of a \class{Qtcm} instance.
	The run list and list element names are stored as strings.}
\label{tab:stnd.runlists}
\end{table}
\end{latexonly}

Of course, feel free to change the contents of any of the run lists
after instantiation, or to add additional run lists to the
\vars{runlists} attribute dictionary.  The ability to alter run
lists at runtime gives the \mods{qtcm} package much of its flexibility.




%---------------------------------------------------------------------
\section{Field Variables and the \class{Field} Class}
						\label{sec:field.variables}

The term ``field'' variable refers to QTCM1 model variables that 
are accessible at both the compiled Fortran QTCM1 model-level as
well as the Python \class{Qtcm} instance-level.
Field variables are all instances of the \class{Field} class
(though non-field variables can also be instances of \class{Field}).

Section~\ref{sec:field.variables.intro} gives a brief introduction to
the attributes and methods in a \class{Field} instance.
A nitty gritty description of the class is found in its docstrings.

	\subsection{Creating Field Variables}

To create a \class{Field} instance whose value is set to the
default, instantiate with the field id as the only positional
input argument.  Thus:

\begin{codeblock}
\codeblockfont{foo = Field('lastday')}
\end{codeblock}

will return \vars{foo} as a \class{Field} instance with \vars{foo.value}
set to the value listed in Section~\ref{sec:defaults.scalar}.
The value of all \class{Field} instances upon creation are specified
in the \mods{defaults} submodule of package \mods{qtcm}, and listed
in Sections~\ref{sec:defaults.scalar} and~\ref{sec:defaults.array}.

To create \class{Field} instances whose attributes are set different
from their defaults, you can specify the different settings in the
instantiation parameter list, or change the attributes once the
instance is created.  See the \class{Field} docstring for details.


	\subsection{Initial Field Variables}  \label{sec:initial.variables}

Field variables include both model parameters that do not change
for a \class{Qtcm} instance as well as prognostic variables that
do change during model integration.  As a result, many field variables
have values different from the default values listed in
Sections~\ref{sec:defaults.scalar} and~\ref{sec:defaults.array}.
In this section, I list the \emph{initial} values of all field
variables.  The ``initial'' values are the settings for \class{Qtcm}
field variables execution of the \mods{run\_session} method, but
prior to cycling through an atmosphere-ocean coupling timestep.
This is in contrast to ``default'' values, which the field variables
are given on instantiation, if no other value is specified.
Numerical values are rounded as per the conventions
of Python's \vars{\%g} format code.


		\subsubsection{Scalars}

For the fields that give the input/output directory names, and the
run name, the entry ``value varies'' is provided in the ``Value''
column.

\input{init_scalars}

		\subsubsection{Arrays}

\input{init_arrays}


	\subsection{Passing Fields Between the Python and Fortran-Levels}

Section~\ref{sec:comm.py.fort.compiledform} discusses the differences
between how the \vars{'full'} and \vars{'parts'} compiled forms
pass field variables between the Python and Fortran-levels.  That
discussion gives a detailed description of the methods used for
passing fields to and from the Python and Fortran-levels (i.e., the
\mods{get\_qtcm1\_item} and \mods{set\_qtcm1\_item} methods).

Please note the following regarding field variables as you pass them 
back and forth between the Python and Fortran-levels:
\begin{itemize}
\item Field variables with ghost latitudes, such as \vars{u1}, on
	the Python end are always the full variables (i.e., including
	the ghost latitudes).  On the Fortran end, variables like
	\vars{u1} also always have the ghost latitudes while in the
	model, but when stored as restart files, do not have the
	ghost latitudes; the end points are not saved in restart
	files or written to the netCDF output files.
	See the
	\latexhtml{%
\htmladdnormallinkfoot{QTCM1 manual}%
        {http://www.atmos.ucla.edu/$\sim$csi/qtcm\_man/v2.3/qtcm\_manv2.3.pdf}}%
{\htmladdnormallink{QTCM1 manual}%
        {http://www.atmos.ucla.edu/~csi/qtcm_man/v2.3/qtcm_manv2.3.pdf}}
	\cite{Neelin/etal:2002}
	for details about ghost latitudes.

\item You should assume there is only a full synchronizing between 
	compiled QTCM1 model and Python model field variables
	at the beginning and end of a run session.  

\item If you have a variable at the Python-level, but at the
	compiled QTCM1 Fortran model-level the variable is not
	readable, if you try to call \mods{set\_qtcm1\_item} on the
	variable, nothing is done, and the Python-level value is
	left alone.  If you have a compiled QTCM1 model variable,
	but no Python-level equivalent, if you call \mods{set\_qtcm1\_item}
	on the variable, the Python-level variable (as an attribute)
	is created.

\item To be precise, only compiled QTCM1 model variables can be
	passed pass back and forth between the Python and Fortran-levels;
	there are many \class{Qtcm} instance attributes that do not
	have any counterparts at the Fortran-level.\footnote%
		{I use the term ``field variables'' to refer to 
		compiled QTCM1 model variables that can be passed
		back and forth to the Python level.}

\item Although \vars{dayofmodel} is described in module \mods{setbypy}
	as an option for the \mods{get\_qtcm1\_item} and
	\mods{set\_qtcm1\_item} methods to operate on, in reality
	those methods cannot operate on \vars{dayofmodel}, but
	\vars{dayofmodel} is not defined in \mods{defaults}.\footnote%
		{All field variables must be defined in \mods{defaults} in
		order for the proper Fortran routine to be called
		according to the variable's type.}
\end{itemize}


	\subsection{Field Variable Shape}   \label{sec:field.var.shape}

Normally, Python arrays have a different dimension order than Fortran
arrays.  While Fortran arrays are dimensioned (col, row, slice),
with adjacent columns being contiguous, then rows, and then slices, Python
arrays are dimensioned (slice, row, col), with adjacent columns being
contiguous, then rows, and then slices.  Based on this, you would
think that everytime you passed an array between the Python and
Fortran-levels you would need to transpose the array.

Thankfully, we don't have to do this because \mods{f2py} handles
array dimension order transparently so we can refer to each element
the same way whether we're in Python or Fortran.  Thus, the array
\vars{Qc} in Fortran is dimensioned (longitude, latitude), (64,42)
by default, and the Python \class{Qtcm} instance attribute \vars{Qc}
has a \vars{value} attribute also dimensioned (longitude, latitude),
(64,42) by default.  And at both the Fortran and Python-levels, the
first longtude, second latitude element is referred to as \vars{Qc(1,2)}.

In contrast, however, netCDF output saved by the compiled QTCM1 model
and read into Python (using the \mods{Scientific} package) is
\emph{not} in Fortran array order.  Arrays read from netCDF output
into Python are in Python array order, and are dimensioned
(latitude, longitude) or (time, latitude, longitude).  The \class{Qtcm}
routines that manipulate netCDF data (e.g., \mods{plotm}), however,
automatically adjust for this, so you only need to be aware of this
when reading in output for your own analysis
(see Section~\ref{sec:model.output}).




%---------------------------------------------------------------------
\section{Model Output}			\label{sec:model.output}

Section~\ref{sec:output.intro} gives an overview of how to
use \mods{qtcm} model output to netCDF files.

All netCDF array output is dimensioned (time, latitude, longitude)
when read into Python using the \mods{Scientific} package.  This
differs from the way \class{Qtcm} saves field variables, which
follows Fortran convention (longitude, latitude).  Thus, the shapes
in Section~\ref{sec:initial.variables}, Appendix~\ref{app:defaults.values},
etc., are not the shapes of arrays read from the netCDF output.
See Section~\ref{sec:field.var.shape} for a discussion of why
there is this discrepancy.

Because netCDF files allow you to specify an ``unlimited'' dimension,
it is possible to close a netCDF file, reopen it, and add more
slices of data to the file.  Thus, continuous \class{Qtcm} run
sessions (i.e., those that use the \vars{cont} keyword input parameter
in the \mods{run\_session} method) will automatically append output
to the netCDF output files.

Field variables with ghost latitudes, such as \vars{u1}, on the
Python and Fortran ends are always the full variables (i.e., including
the ghost latitudes).  The ghost latitudes are not written to the
netCDF output files, however.
See the \latexhtml{%
\htmladdnormallinkfoot{QTCM1 manual}%
        {http://www.atmos.ucla.edu/$\sim$csi/qtcm\_man/v2.3/qtcm\_manv2.3.pdf}}%
{\htmladdnormallink{QTCM1 manual}%
        {http://www.atmos.ucla.edu/~csi/qtcm_man/v2.3/qtcm_manv2.3.pdf}}
	\cite{Neelin/etal:2002}
for details about ghost latitude structure.

\class{Qtcm} instances have a few built-in tools to visualization
model output.  These are briefly described in Section~\ref{sec:viz.intro}.
Note that the \mods{plotm} method is linked to a specific \class{Qtcm}
instance.  Do not use \mods{plotm} outside of the instance it is
linked to.  It must also be used only after a successful run session
(i.e., not in the middle of a run session).




%---------------------------------------------------------------------
\section{Miscellaneous}

A few miscellaneous items/issues about the model:
\begin{itemize}
\item The land model runs at same timestep as the atmosphere.

\item If the land model runs less often than 
	\mods{sflux} in \mods{physics1}, 
	the calculation of evaporation over the land 
	needs to be fixed in sflux.

\item The units of some field variables are not what you would expect.
	For instance, \vars{Qc} is in energy units, i.e., K, and not
	mm/day.
	See the
	\latexhtml{%
\htmladdnormallinkfoot{QTCM1 manual}%
        {http://www.atmos.ucla.edu/$\sim$csi/qtcm\_man/v2.3/qtcm\_manv2.3.pdf}}%
{\htmladdnormallink{QTCM1 manual}%
        {http://www.atmos.ucla.edu/~csi/qtcm_man/v2.3/qtcm_manv2.3.pdf}}
	\cite{Neelin/etal:2002}
	for details.
\end{itemize}




%---------------------------------------------------------------------
\section{Cookbook of Ways the Model Can Be Used}  \label{sec:cookbook}

This cookbook of a few ways to use the model is arranged by science
tasks, i.e., certain types of runs we want to do.  For some of the
examples below, I assume that the dictionary
\vars{inputs} is initially defined as given in
Figure~\ref{fig:defn.of.inputs}.  All examples assume that
\cmd{from qtcm import Qtcm} has already been executed.


%--- Two versions, one for PDF and the other for HTML:
\begin{latexonly}
\begin{figure}[tp]
\begin{codeblock}
\codeblockfont{%
inputs = \{\} \\
inputs['runname'] = 'test' \\
inputs['landon'] = 0 \\
inputs['year0'] = 1 \\
inputs['month0'] = 11 \\
inputs['day0'] = 1 \\
inputs['lastday'] = 30 \\
inputs['mrestart'] = 0 \\
inputs['init\_with\_instance\_state'] = True \\
inputs['compiled\_form'] = 'parts'}
\end{codeblock}

\caption{The initial definition of the \vars{inputs} dictionary for 
	examples given in Section~\ref{sec:cookbook}.  These settings
	imply that a run session will start on November 1, Year 1,
	last for 30 days, and will be an aquaplanet run.}
\label{fig:defn.of.inputs}
\end{figure}
\end{latexonly}

\begin{htmlonly}
\label{fig:defn.of.inputs}
\begin{center}
\htmlfigcaption{%
	\codeblockfont{%
inputs = \{\} \\
inputs['runname'] = 'test' \\
inputs['landon'] = 0 \\
inputs['year0'] = 1 \\
inputs['month0'] = 11 \\
inputs['day0'] = 1 \\
inputs['lastday'] = 30 \\
inputs['mrestart'] = 0 \\
inputs['init\_with\_instance\_state'] = True \\
inputs['compiled\_form'] = 'parts'}
	}

\htmlfigcaption{Figure~\ref{fig:defn.of.inputs}:
	The initial definition of the \vars{inputs} dictionary for 
	examples given in Section~\ref{sec:cookbook}.  These settings
	imply that a run session will start on November 1, Year 1,
	last for 30 days, and will be an aquaplanet run.}
\end{center}
\end{htmlonly}



\begin{description}
\item[Plain model run:]
	Here I just want to make a single model run.
	Tasks:  Instantiate a fresh model and execute a run session.
	The code to run the model is just:
	\begin{codeblock}
	\codeblockfont{%
inputs['init\_with\_instance\_state'] = False \\
model = Qtcm(**inputs) \\
model.run\_session()}
	\end{codeblock}
	where \vars{inputs} is initialized with the code in
	Figure~\ref{fig:defn.of.inputs}.


\item[Explore parameter space with a set of models:]
	Here I want to create an entire suite of separate models,
	in order to determine the sensitivity of the model to changing
	a parameter.
	To do this, I
	instantiate multiple fresh models, 
	and execute a run session for each instance, all within
	a \vars{for} loop:


%--- Two versions, because LaTeX2HTML does not correctly typeset
%    the hspace command:
\begin{latexonly}
	\begin{codeblock}
	\codeblockfont{%
import os \\
inputs['init\_with\_instance\_state'] = False \\
for i in xrange(0,1002,10): \\
\hspace*{5ex}iname = 'ziml-' + str(i) + 'm' \\
\hspace*{5ex}ipath = os.path.join('proc', iname) \\
\hspace*{5ex}os.makedirs(ipath) \\
\hspace*{5ex}model = Qtcm(**inputs) \\
\hspace*{5ex}model.ziml.value = float(i)  \\
\hspace*{5ex}model.runname.value = iname \\
\hspace*{5ex}model.outdir.value = ipath \\
\hspace*{5ex}model.run\_session() \\
\hspace*{5ex}del model}
	\end{codeblock}
\end{latexonly}

\begin{htmlonly}
\begin{center}
\htmlfigcaption{%
	\codeblockfont{%
import os \\
inputs['init\_with\_instance\_state'] = False \\
for i in xrange(0,1002,10): \\
\hspace*{5ex}iname = 'ziml-' + str(i) + 'm' \\
\hspace*{5ex}ipath = os.path.join('proc', iname) \\
\hspace*{5ex}os.makedirs(ipath) \\
\hspace*{5ex}model = Qtcm(**inputs) \\
\hspace*{5ex}model.ziml.value = float(i)  \\
\hspace*{5ex}model.runname.value = iname \\
\hspace*{5ex}model.outdir.value = ipath \\
\hspace*{5ex}model.run\_session() \\
\hspace*{5ex}del model}
	}
\end{center}
\end{htmlonly}


	The loop explores mixed-layer depth \vars{ziml} from 0~m to
        1000~m, in 10~m intervals.  I create the \vars{outdir}
	directory before every model call, since the compiled QTCM1 model
	requires the output directory exist, specifying the run name
	and output directory as the string \vars{iname}.
	The output directories are assumed to all be in the \fn{proc}
	sub-directory of the current working directory.
	\vars{inputs} is initialized with the code in
	Figure~\ref{fig:defn.of.inputs}.


\item[Conditionally explore parameter space:]
	Here I want to 
	conditionally explore the parameter space, on the basis of
	some mathematical criteria.
	To do this, I
	instantiate a model, evaluate results using
	that criteria, and run another fresh model depending on
	the results (passing the previous model state via a snapshot),
	all within a \vars{while} loop.
	Note that this type of investigation is very difficult to 
	automate if all you can use are shell scripts and
	Fortran.
	See Figure~\ref{fig:conditional.test.eg} for a detailed
	example.


\item[With interactive adjustments at run time:]
	The example in Figure~\ref{sec:continuation.intro}
	illustrates this type of run.  In this example,
	I instantiate a fresh model, execute a run session, analyze the
	output, change variables in the model instance, and then
	execute a continuation run session.


\item[Test alternative parameterizations:]
	I've already described how we can use run lists to arbitrarily
	change model execution order and content at run time.
	We can take advantage of Python's inheritance
	abilities, along with run lists, to simplify this.
	Figure~\ref{fig:alt.param.inherit.eg} provides an example of
	this use.

	Of course, you can use pre-processor directives and shell
	scripts to accomplish the same functionality seen in
	Figure~\ref{fig:alt.param.inherit.eg} using just Fortran.
	The Python solution, however, shortcuts the compile/linking
	step, and enables you to easily do run time swapping between
	subroutine choices based upon run time calculated
	tests (see Figure~\ref{fig:conditional.test.eg} for an
	example of such tests).
\end{description}




% --- Two versions of this block, one for display in PDF and the other
%     for display in HTML:
\begin{latexonly}
\begin{figure}[p]
	\begin{codeblock}
	\codeblockfont{%
\small
import os \\
import numpy as N \\
maxu1 = 0.0 \\
while maxu1 < 10.0: \\
\hspace*{5ex}iziml = 0.1 * maxu1 \\
\hspace*{5ex}iname = 'ziml-' + str(iziml) + 'm' \\
\hspace*{5ex}ipath = os.path.join('proc', iname) \\
\hspace*{5ex}os.makedirs(ipath) \\
\hspace*{5ex}model = Qtcm(**inputs) \\
\hspace*{5ex}try: \\
\hspace*{10ex}model.sync\_set\_py\_values\_to\_snapshot(snapshot=mysnapshot) \\
\hspace*{10ex}model.init\_with\_instance\_state = True \\
\hspace*{5ex}except: \\
\hspace*{10ex}model.init\_with\_instance\_state = False \\
\hspace*{5ex}model.ziml.value = iziml  \\
\hspace*{5ex}model.runname.value = iname \\
\hspace*{5ex}model.outdir.value = ipath \\
\hspace*{5ex}model.run\_session() \\
\hspace*{5ex}maxu1 = N.max(N.abs(model.u1.value)) \\
\hspace*{5ex}mysnapshot = model.snapshot \\
\hspace*{5ex}del model}
	\end{codeblock}

\caption{This code explores different values of
	mixed-layer depth \vars{ziml} for 30~day runs,
	as a function of maximum \vars{u1} magnitude,
	until it finds a case where the maximum \vars{u1} is
	greater than 10~m/s.  (The relationship between
	\vars{ziml} and the maximum of the speed of
	\vars{u1}, where 
	\vars{ziml\thinspace=\thinspace0.1\thinspace*\thinspace{maxu1}}, 
	is made up.)
	With each iteration, the new run uses the snapshot from
	a previous run to initialize (as well as the new value
	of \vars{ziml}); the \vars{try} statement is used to
	ensure the model works even if \vars{mysnapshot} is not
	defined (which is the case the first time around).
	The \vars{inputs} dictionary is initialized with the code in
	Figure~\ref{fig:defn.of.inputs}.}
\label{fig:conditional.test.eg}
\end{figure}
\end{latexonly}

\begin{htmlonly}
\label{fig:conditional.test.eg}
\begin{center}
\htmlfigcaption{%
	\codeblockfont{%
import os \\
import numpy as N \\
maxu1 = 0.0 \\
while maxu1 < 10.0: \\
\hspace*{5ex}iziml = 0.1 * maxu1 \\
\hspace*{5ex}iname = 'ziml-' + str(iziml) + 'm' \\
\hspace*{5ex}ipath = os.path.join('proc', iname) \\
\hspace*{5ex}os.makedirs(ipath) \\
\hspace*{5ex}model = Qtcm(**inputs) \\
\hspace*{5ex}try: \\
\hspace*{10ex}model.sync\_set\_py\_values\_to\_snapshot(snapshot=mysnapshot) \\
\hspace*{10ex}model.init\_with\_instance\_state = True \\
\hspace*{5ex}except: \\
\hspace*{10ex}model.init\_with\_instance\_state = False \\
\hspace*{5ex}model.ziml.value = iziml  \\
\hspace*{5ex}model.runname.value = iname \\
\hspace*{5ex}model.outdir.value = ipath \\
\hspace*{5ex}model.run\_session() \\
\hspace*{5ex}maxu1 = N.max(N.abs(model.u1.value)) \\
\hspace*{5ex}mysnapshot = model.snapshot \\
\hspace*{5ex}del model}
	}

\htmlfigcaption{Figure \ref{fig:conditional.test.eg}:
	This code explores different values of
	mixed-layer depth \vars{ziml} for 30~day runs,
	as a function of maximum \vars{u1} magnitude,
	until it finds a case where the maximum \vars{u1} is
	greater than 10~m/s.  (The relationship between
	\vars{ziml} and the maximum of the speed of
	\vars{u1}, where 
	\vars{ziml\thinspace=\thinspace0.1\thinspace*\thinspace{maxu1}}, 
	is made up.)
	With each iteration, the new run uses the snapshot from
	a previous run to initialize (as well as the new value
	of \vars{ziml}); the \vars{try} statement is used to
	ensure the model works even if \vars{mysnapshot} is not
	defined (which is the case the first time around).
	The \vars{inputs} dictionary is initialized with the code in
	Figure~\ref{fig:defn.of.inputs}.}
\end{center}
\end{htmlonly}


% --- Two versions of this block, one for display in PDF and the other
%     for display in HTML:
\begin{latexonly}
\begin{figure}[p]
\begin{center}
	\begin{codeblock}
	\codeblockfont{%
\small
import os \\
\\
class NewQtcm(Qtcm): \\
\hspace*{5ex}def cloud0(self):\\
\hspace*{10ex}[\ldots] \\
\hspace*{5ex}def cloud1(self):\\
\hspace*{10ex}[\ldots] \\
\hspace*{5ex}def cloud2(self):\\
\hspace*{10ex}[\ldots] \\
\hspace*{5ex}[\ldots] \\
\\
inputs['init\_with\_instance\_state'] = False \\
for i in xrange(10): \\
\hspace*{5ex}iname = 'cloudroutine-' + str(i)  \\
\hspace*{5ex}ipath = os.path.join('proc', iname) \\
\hspace*{5ex}os.makedirs(ipath) \\
\hspace*{5ex}model = NewQtcm(**inputs) \\
\hspace*{5ex}model.runlists['atm\_physics1'][1] = 'cloud' + str(i) \\
\hspace*{5ex}model.runname.value = iname \\
\hspace*{5ex}model.outdir.value = ipath \\
\hspace*{5ex}model.run\_session() \\
\hspace*{5ex}del model}
	\end{codeblock}
\end{center}

\caption{Let's say we have 9 different cloud physics schemes we wish
	to try out in 9 different runs.  The easiest way to do this
	is to create a new class \class{NewQtcm} that
	inherits everything from \class{Qtcm}, and to which we'll
	add the additional cloud schemes (\vars{cloud0}, \vars{cloud1},
	etc.).
	In the \vars{for} loop, I change the cloud model
	run list entry in the run list that governs
	atmospheric physics at one instant to whatever the cloud
	model is at this point in the loop.
	The \vars{inputs} dictionary is initialized with the code in
	Figure~\ref{fig:defn.of.inputs}.
	Of course, we could do the same thing by running the 9
	models separately, but this set-up makes it easy to do
	hypothesis testing with these 9 models.  For instance, we
	can create a test by which we will choose which of the 9
	models to use:  Within this framework, the selection of
	those models can be altered by changing a string.}
\label{fig:alt.param.inherit.eg}
\end{figure}
\end{latexonly}

\begin{htmlonly}
\label{fig:alt.param.inherit.eg}
\begin{center}
\htmlfigcaption{%
	\codeblockfont{%
import os \\
\\
class NewQtcm(Qtcm): \\
\hspace*{5ex}def cloud0(self):\\
\hspace*{10ex}[\ldots] \\
\hspace*{5ex}def cloud1(self):\\
\hspace*{10ex}[\ldots] \\
\hspace*{5ex}def cloud2(self):\\
\hspace*{10ex}[\ldots] \\
\hspace*{5ex}[\ldots] \\
\\
inputs['init\_with\_instance\_state'] = False \\
for i in xrange(10): \\
\hspace*{5ex}iname = 'cloudroutine-' + str(i)  \\
\hspace*{5ex}ipath = os.path.join('proc', iname) \\
\hspace*{5ex}os.makedirs(ipath) \\
\hspace*{5ex}model = NewQtcm(**inputs) \\
\hspace*{5ex}model.runlists['atm\_physics1'][1] = 'cloud' + str(i) \\
\hspace*{5ex}model.runname.value = iname \\
\hspace*{5ex}model.outdir.value = ipath \\
\hspace*{5ex}model.run\_session() \\
\hspace*{5ex}del model}
	}

\htmlfigcaption{Figure \ref{fig:alt.param.inherit.eg}:
	Let's say we have 9 different cloud physics schemes we wish
	to try out in 9 different runs.  The easiest way to do this
	is to create a new class \class{NewQtcm} that
	inherits everything from \class{Qtcm}, and to which we'll
	add the additional cloud schemes (\vars{cloud0}, \vars{cloud1},
	etc.).
	In the \vars{for} loop, I change the cloud model
	run list entry in the run list that governs
	atmospheric physics at one instant to whatever the cloud
	model is at this point in the loop.
	The \vars{inputs} dictionary is initialized with the code in
	Figure~\ref{fig:defn.of.inputs}.
	Of course, we could do the same thing by running the 9
	models separately, but this set-up makes it easy to do
	hypothesis testing with these 9 models.  For instance, we
	can create a test by which we will choose which of the 9
	models to use:  Within this framework, the selection of
	those models can be altered by changing a string.}
\end{center}
\end{htmlonly}




% ===== end of file =====


%@@@\chapter{Combining \code{qtcm} with \code{CliMT}}
%@@@% ==========================================================================
% CliMT
%
% By Johnny Lin
% ==========================================================================


% ------ BODY -----
%
\section{General Tutorial on CliMT}


General notes of things I think I may have observed about
\code{Parameters} objects:
\begin{itemize}
\item You can treat a \code{Parameters} instance as a dictionary, where
	the key is the name of the field, because \code{\_\_getitem\_\_},
	etc.\ have been defined for the instance.  However, the values,
	units, and long names of the fields are stored in dictionaries
	assigned to \code{value}, \code{units}, and \code{long\_name},
	keyed to the field name (a string).
\end{itemize}


General notes of things I think I may have observed about
\code{Components} objects:
\begin{itemize}
\item All variables and quantities, whether they be physical fields,
	filenames, or metadata,
	are stored as attributes in the \code{Components} instance.
\item \code{Components} have these special attributes:
        \code{Required},
        \code{Prognostic},
	and
        \code{Diagnostic},
	which are lists that contain the names of describe whether
\item Scalar parameters in \code{Component} objects
	are stored as an instance of the \code{Parameters}
	class, under the attribute \code{Params}.
\end{itemize}


General notes of things I think I may have observed about
\code{Federation} objects:
\begin{itemize}
\item \code{Federation} objects hold the \code{Components} instances
	in a list assigned to the attribute \code{list}.
\item \code{Federation} attributes
        \code{Required},
        and
	\code{Prognostic},
	are unions of the same attributes of the constituent
	\code{Components} objects.
\end{itemize}






% ===== end of file =====


\chapter{Troubleshooting}                   \label{ch:trouble}
% ==========================================================================
% Troubleshooting
%
% By Johnny Lin
% ==========================================================================


% ------ BODY -----
%
\section{Error Messages Produced by \mods{qtcm}}

\begin{description}
\item[\screen{Error-Value too long in SetbyPy module getitem\_str for}
	\dumarg{key}:]
	This message is produced by the Fortran
	subroutine \mods{getitem\_str}
	in the module \mods{SetbyPy} in the compiled QTCM1 Fortran code.
	The code is in the file \fn{setbypy.F90}.  This error occurs when
	the Fortran variable whose name is given by the string \dumarg{key}
	has a value that is greater than the local parameter
	\vars{maxitemlen} in \mods{getitem\_str}.  To fix this, you have
	to go into \fn{setbypy.F90} and change the value of
	\vars{maxitemlen}.

\item[\screen{Error-real\_rank1\_array should be deallocated}:]
	Fortran module \mods{SetByPy}'s subroutine
	\mods{getitem\_real\_array} generates this message
	(or a similar message for other ranks) if the Fortran
	variable for the input \dumarg{key} are allocated on entry
	to the routine.  This may indicate the user has written another
	Fortran routine to access the \mods{real\_rank1\_array} variable
	outside of the standard interfaces..

\item[\screen{Error-Bad call to SetbyPy module \ldots}:]
	Often times, this error occurs because a get or set routine
	in \mods{SetByPy} tried to act on a variable for which the
	corresponding input \dumarg{key} is not defined.  The solution
	is to add that case in the if/then construct for the get and set
	routines in \mods{SetByPy} and rebuild the extension modules.
\end{description}


\section{Other Errors}

\begin{description}
\item[Python cannot find some packages:]
	This error often happens when the version of Python in which
	you have installed all your packages is not the version that
	is called at the Unix command line by typing in \cmd{python}.
	To get around this, 
        define a Unix alias
        that maps \cmd{python2.4} (or whichever version of Python
	has all your packages installed) to \cmd{python}.  If you
	have multiple Python's installed on your system, you might
	have to use a more specific name for the Python executable.
	As a result, you may have to change the test scripts in
	\fn{test} in the \mods{qtcm} distribution directory.

\item[\mods{get\_qtcm1\_item} and compiled QTCM1 model pointer
	variables:]
	If you try to use the \mods{get\_qtcm1\_item} method on a compiled
	QTCM1 model pointer variable 
	(i.e., \vars{u1}, \vars{v1}, \vars{q1}, \vars{T1}),
	 before the compiled
	model \mods{varinit} subroutine is run, you'll get a bus error
	with no additional message.

\item[Mismatch between Python and Fortran array field variables:]
	You change an array field variable on the Python side, but
	it seems like the wrong elements are changed on the Fortran
	side.  Or you type in the same index address for accessing a
	\mods{qtcm} netCDF output array as well as its \class{Qtcm}
	instance attribute counterpart, and find you get different
	answers.  Some possible reasons and fixes:

	\begin{itemize}
	\item This will occur if you haven't accounted for the
		difference in how field variables are saved at the
		Python-level, Fortran-level, and in a netCDF file.
		All netCDF array output is dimensioned (time,
		latitude, longitude) when read into Python using
		the \mods{Scientific} package.  This differs from
		the way \class{Qtcm} saves field variables, \emph{both}
		at the Python- and Fortran-levels, which follows
		Fortran convention (longitude, latitude).

		Note that the way \class{Qtcm} saves field variables
		at the Python- and Fortran-levels is different than
		the default way Python and Fortran save arrays.
		Section~\ref{sec:field.var.shape} for more information.

	\item You may have forgotten that array indices in Python start at
		0, while indices in Fortran (generally) start at 1.
		Also, ranges in Python are exclusive at the upper-bound,
		while ranges in Fortran are inclusive at the upper-bound.
		(Both Python and Fortran array indice ranges are inclusive
		at the lower-bound.)

	\item You may have forgotten some field variables have
		ghost latitudes, and thus there are extra latitude bands
		when the array is stored as a Python or Fortran field
		variable, but there are \emph{no} extra latitude bands
		when the array is stored as netCDF output (the QTCM1
		output routines strip off the ghost latitudes when
		writing those field variables out).
	        See the
        \latexhtml{%
\htmladdnormallinkfoot{QTCM1 manual}%
        {http://www.atmos.ucla.edu/$\sim$csi/qtcm\_man/v2.3/qtcm\_manv2.3.pdf}}%
{\htmladdnormallink{QTCM1 manual}%
        {http://www.atmos.ucla.edu/~csi/qtcm_man/v2.3/qtcm_manv2.3.pdf}}
        \cite{Neelin/etal:2002}
        for details about ghost latitudes.

		The safest and easiest way to tell whether the variable has a
		ghost latitudes is to look at its shape.
		A call to the \class{Qtcm} instance
		method \mods{get\_qtcm1\_item} will give you the array,
		and the use of NumPy's \mods{shape} function will give you
		the shape.
	\end{itemize}
\end{description}




% ===== end of file =====


\chapter{Developer Notes}                   \label{ch:devnotes}
% ==========================================================================
% Using QTCM
%
% By Johnny Lin
% ==========================================================================


% ------ BODY -----
%

%---------------------------------------------------------------------
\section{Introduction}

This section describes programming practices and issues related to
the \mods{qtcm} package that might be of interest to users wishing
to add/change code in the package.
Please see the package
\latexhtml{API documentation,%
		\footnote{http://www.johnny-lin.com/py\_pkgs/qtcm/doc/html-api/}
		which includes the source code}%
        {\htmladdnormallink{API documentation}%
		{http://www.johnny-lin.com/py\_pkgs/qtcm/doc/html-api/},
		which includes the source code},
for details.




%---------------------------------------------------------------------
\section{Changes to QTCM1 Fortran Files}  \label{sec:f90changes}

The source code used to generate the shared object files used
in this Python package is unchanged
from the pure-Fortran QTCM1 model source code, except in the
following ways:

\begin{itemize}
\item The suffix of all source code files 
	has been changed from \fn{.f90} to \fn{.F90}, 
	in order to ensure the compiler preprocesses 
	the source code.  Some compilers use the capitalization to
	tell whether or not to run the code through a preprocessor.

\item In all \fn{.F90} files, occurrences of:
	\begin{codeblock}
	\codeblockfont{%
	Character(len=130)}
	\end{codeblock}
	are changed to:
	\begin{codeblock}
	\codeblockfont{%
	Character(len=305)}
	\end{codeblock}
	This enables the model to properly deal with longer filenames.
	The number ``305'' is chosen to make search and replace easier.

\item In \fn{qtcmpar.F90}, the 
	\vars{eps\_c} variable is changed from an unchangable
	parameter to a changeable real, 
	so that it can be changed in the model at runtime.

\item All occurrences of an underscore (``\_'') character in a
	subroutine or function name are removed.  The
	presence of the underscore messes up the dynamic lookup
	mechanism for the \mods{f2py} generated extension module.
	The following names are changed, both in subroutine definitions
	and calls:
	\begin{itemize}
	\item \mods{out\_restart} to \mods{outrestart},
	\item \mods{save\_bartr} to \mods{savebartr},
	\item \mods{grad\_phis} to \mods{gradphis}.
	\end{itemize}

\item \fn{driver.F90} is changed so that program
	\mods{driver} becomes a subroutine, and 
	subroutine \mods{driverinit} is deleted (along with
	all calls to it) because basic model initialization is
	handled at the Python level.

\item In \fn{clrad.F90}, subroutine \mods{cloud}, the first
	\vars{COUNTCAP} preprocessor macro, a comment line for
	that ifdef is moved to prevent a warning message during
	building with \mods{f2py}.

\item The order of subroutine \mods{qtcminit} is changed.  The original
	pure-Fortran QTCM1 \mods{qtcminit} code has the following
	calling sequence:

	\begin{codeblock}
        \codeblockfont{%
Call parinit            !Initialize model parameters \\
Call varinit            !Initialize variables \\
Call TimeManager(1)     !mm set model time \\
Call bndinit            !input boundary datasets \\
Call physics1           !diagnostic fields for initial condition}
	\end{codeblock}

	For the \mods{qtcm} package, I've altered this order so
	\mods{bndinit} comes after \mods{parinit} but before \mods{varinit}:
	\begin{codeblock}
        \codeblockfont{%
Call parinit            !Initialize model parameters \\
Call bndinit            !input boundary datasets \\
Call varinit            !Initialize variables \\
Call TimeManager(1)     !mm set model time  \\
Call physics1           !diagnostic fields for initial condition}
	\end{codeblock}

	This is done because \vars{STYPE} is not read in for the
	\vars{landon} \vars{True} case until \mods{bndinit}, but
	in \mods{varinit} \vars{STYPE} is used to calculate the
	original values of \vars{WD} for the non-restart case.  This
	also corrects the conflicting order found in the pure-Fortran
	QTCM1 manual (compare pp.\ 29 and 32).  As far as I can
	tell, \mods{bndinit} has no dependencies that require it
	to come after \mods{timemanager} or \mods{varinit}.

\end{itemize}

In addition, the Fortran files \fn{setbypy.F90}, \fn{wrapcall.F90},
and \fn{varptrinit.F90} are added.  The routines in these files, 
however, just add more flexibility and functionality to the model;
they do not automatically affect any model computations.  See
Section~\ref{sec:newf90} for details.




%---------------------------------------------------------------------
\section{New Interfaces and Fortran Functionality}  \label{sec:newf90}

As described in Section~\ref{sec:f90changes}, the Fortran files
\fn{setbypy.F90}, \fn{wrapcall.F90}, and \fn{varptrinit.F90} are
added to the QTCM1 source directory.  The first two files define the Fortran
90 modules (\mods{SetbyPy} and \mods{WrapCall}) needed to interface
the Python and Fortran levels.  The last file defines a new Fortran
subroutine \mods{varptrinit} that associates QTCM1 model pointer
variables at the Fortran level.  In a pure-Fortran run of QTCM1,
this occurs in subroutine \mods{varinit}; for a
\vars{compiled\_form\thinspace=\thinspace'parts'} run, since the
functionality of the Fortran \mods{varinit} is now in the Python
\mods{varinit} method, a separate Fortran pointer association
subroutine needed to be defined.  The Fortran subroutine \mods{varptrinit}
is called as the \mods{varptrinit} function of the 
\vars{compiled\_form\thinspace=\thinspace'parts'}
\fn{.so} extension module.


	\subsection{Fortran Module \mods{SetbyPy}}   \label{sec:setbypy}

		\subsubsection{Design Description}

This module defines functions and subroutines used to read variables
from the Fortran-level to the Python-level, and in setting Fortran-level
variables using the Python-level values.  These routines are used
by \class{Qtcm} methods \mods{get\_qtcm1\_item} and \mods{set\_qtcm1\_item}
(and dependencies thereof) to ``get'' and ``set'' the Fortran-level
variables.  Note that the Fortran module \mods{SetbyPy} is referred
to in lowercase at the Python level, i.e., as the
attribute \vars{\_\_.qtcm.setbypy} of a \class{Qtcm} instance.

Because Fortran variables are not dynamically typed, separate Fortran
functions and subroutines need to be defined to get and set variables
of different types.\footnote%
	{The \mods{interface} construct in Fortran 90 is supposed to
	provide a way to create a single interface to multiple
	routines, e.g.:
	\begin{codeblock}
	\codeblockfont{%
Interface setitem \\
\hspace*{3ex}Module Procedure setitem\_real, setitem\_int, setitem\_str \\
End Interface}
	\end{codeblock}
	This construct, however, causes a bus error
	(Mac OS X 10.4, Intel).  Thus, I put the
	same functionality in the Python code.}
The \class{Qtcm} methods \mods{get\_qtcm1\_item}
and \mods{set\_qtcm1\_item} know which one of the Fortran routines
to call on the basis of the type and rank of the value for the field
variable in the \mods{defaults} submodule.  This is why all field
variables need to have defaults defined in \mods{defaults}.  For
array variables, the field variable defaults also provide the rank
of the Fortran-level variable being gotten or set.  However, the
array default values do \emph{not} have to have the same shape as
the Fortran-level variables; on the Python-side, variable shape
adjusts to what is declared on the Fortran-side.  
Thus, if you change the resolution of
the compiled QTCM1 model, you do not have to change the dimensions
of the field variable values in \mods{defaults}.

The \class{Qtcm} method \mods{get\_qtcm1\_item} directly calls
the \mods{SetByPy} routines.
The \class{Qtcm} method \mods{set\_qtcm1\_item} makes use of
private instance methods that make the calls to the \mods{SetByPy} routines.

For scalar field variables, \mods{SetByPy} provides functions and
subroutines that provide the value of the variable on output.
For array field variables, \mods{SetByPy}
dynamic \emph{module} arrays are used to pass array
variables in and out; I could not get the 
\mods{SetByPy} Fortran routines to set
locally defined dynamic arrays (that is, locally within a function or
subroutine).\footnote%
	{I tried to implement Fortran subroutine
	\mods{getitem\_real\_array} using traditional array 
	dimension passing 
	(e.g., \code{subroutine foo(nx, ny, a)}) as well
	as declaring the allocatable array inside the subroutine, 
	but neither option worked on my \mods{f2py} (version 2\_3816) 
	and Python (version 2.4.3).}
In the \mods{SetByPy} module, these dynamic arrays
are defined as follows:

\begin{codeblock}
\codeblockfont{%
Real, allocatable, dimension(:) :: real\_rank1\_array \\
Real, allocatable, dimension(:,:) :: real\_rank2\_array \\
Real, allocatable, dimension(:,:,:) :: real\_rank3\_array}
\end{codeblock}

For all field variables, scalar or array, the \mods{SetByPy} module
has a fourth module variable defined, \vars{is\_readable}, that the
Fortran get and set routines will set to \vars{.TRUE.} if the
variable is readable and \vars{.FALSE.} if not (it's declared as a
logical variable).  This Fortran variable can be used to prevent
Python from accessing pointer variables that aren't yet associated
to targets.

In general, \mods{SetByPy} routines make use of Fortran constructs
to enable them to accomodate all possible
variables of a given type and shape.  However, 
for string scalars, the \mods{SetByPy} function \mods{getitem\_str}
has to have a return value of a predefined length, in order to
work properly.  That length is given by the parameter
\vars{maxitemlen} and is set to 505 (the value is chosen to
be larger than all filename variables described in
Section~\ref{sec:f90changes} and to be easily found in
the \fn{.F90} files).


		\subsubsection{Module Structure}

If you're a Fortran programmer, you can probably get all the information
in this section from just reading the \fn{setbypy.F90} file directly.
This description of the module structure, however, permits me to highlight
what you need to do if you want to make additional compiled QTCM1 variables
accessible to Python \class{Qtcm} objects.

\begin{itemize}
\item All \mods{Use} statements are given in the beginning of 
	the \mods{SetByPy} module.  These statements cover
	nearly all of the QTCM1 Fortran
	modules that contain variables of interest.  If the
	QTCM1 variable you're interested in isn't in a module
	listed here, you'll have to add your own
	\mods{Use} statement of that module here.

\item Next comes the definitions for the
	\vars{real\_rank1\_array},
	\vars{real\_rank2\_array}, and
	\vars{real\_rank3\_array} dynamic array variables, and
	the \vars{is\_readable} boolean variable.

\item The \mods{Contains} block of the module defines the module
	routines called by the \class{Qtcm} instance methods to
	set and get the compiled QTCM1 model variables.  The
	routines are:
	\begin{itemize}
	\item Function \mods{getitem\_real}
	\item Subroutine \mods{getitem\_real\_array}
	\item Function \mods{getitem\_int}
	\item Function \mods{getitem\_str}
	\item Subroutine \mods{setitem\_real}
	\item Subroutine \mods{setitem\_real\_array}
	\item Subroutine \mods{setitem\_int}
	\item Subroutine \mods{setitem\_str}
	\end{itemize}

\end{itemize}

Each of the routines in the module \mods{Contains} block is essentially
a list of \mods{if}/\mods{elseif} statements.  The list tests for the
name of the variable of interest (a string), and gets or sets the
compiled QTCM1 model variable corresponding to that name.  For pointer
array variables, a test is also made as to whether or not the variable
has been associated.  If not, the variable is not readable
and \vars{is\_readable} is set to \vars{.FALSE.}\ accordingly.

If you wish to add another compiled QTCM1 model variable to be
accessible to \class{Qtcm} instance methods \mods{get\_qtcm1\_item}
and \mods{set\_qtcm1\_item}, just add another \mods{if}/\mods{else\-if},
like the other \mods{if}/\mods{elseif} blocks, in the Fortran set
and get routines corresponding to the QTCM1 variable type (scalar
vs.\ array, and real, integer, or string).  On the Python side, add
an entry in \mods{defaults} corresponding to the new field variable
you've created access to.  I would strongly recommend making the
Python name of your new field variable
(given in \mods{defaults}) be the same as the compiled
QTCM1 model variable name.



	\subsection{Fortran Module \mods{WrapCall}}   \label{sec:wrapcall}

Most of the time, if you want to call a compiled QTCM1 model subroutine
from the Python level, you will use the version of the subroutine that
is found in this Fortran module.  
Note that the Fortran module \mods{WrapCall} is referred
to in lowercase at the Python level, i.e., as the
attribute \vars{\_\_.qtcm.wrapcall} of a \class{Qtcm} instance.

All the routines in this module do is wrap one of the compiled QTCM1
model routines.  For instance, \mods{WrapCall} subroutine
\mods{wadvcttq} is defined as just:

% --- Two versions of this block, one for display in PDF and the other
%     for display in HTML:
%
\begin{latexonly}
\begin{codeblock}
\codeblockfont{%
Subroutine wadvcttq \\
\hspace*{3ex}Call advcttq \\
End Subroutine wadvcttq}
\end{codeblock}
\end{latexonly}

\begin{htmlonly}
\begin{rawhtml}
<p><code><font color="blue">Subroutine wadvcttq<br>
&nbsp;&nbsp;&nbsp;Call advcttq<br>
End Subroutine wadvcttq</font></code></p>
\end{rawhtml}
\end{htmlonly}

All subroutines in this module begin with ``w'', with the rest of
the name being the Fortran QTCM1 subroutine name.  The calling
interface for the ``w'' version is the same as the Fortran QTCM1
original version.  There are no subroutines in this module that do
not have an exact counterpart in the Fortran QTCM1 code, and thus
this module's subroutines sole purpose is to call other subroutines
in the compiled QTCM1 model.

These wrapper routines are needed because \mods{f2py}, for some
reason I can't figure out, will not properly wrap Fortran routines
(that are then callable at the Python level) that create local
arrays using parameters obtained through a Fortran \mods{use}
statment.  Thus, as an example, a Fortran subroutine \mods{foo}
with the following definition:

% --- Two versions of this block, one for display in PDF and the other
%     for display in HTML:
%
\begin{latexonly}
\begin{codeblock}
\codeblockfont{%
subroutine foo \\
\hspace*{3ex}use dimensions \\
\hspace*{3ex}real a(nx,ny) \\
\hspace*{3ex}[\ldots] \\
end subroutine foo}
\end{codeblock}
\end{latexonly}

\begin{htmlonly}
\begin{rawhtml}
<p><code><font color="blue">
subroutine foo<br>
&nbsp;&nbsp;&nbsp;use dimensions<br>
&nbsp;&nbsp;&nbsp;real a(nx,ny)<br>
&nbsp;&nbsp;&nbsp;[\ldots]<br>
end subroutine foo
</font></code></p>
\end{rawhtml}
\end{htmlonly}


where \vars{nx} and \vars{ny} are defined in the module vars{dimensions},
will return an error, with the result that the extension module
will not be created, or an extension modules that yields output
that is different from running the pure-Fortran version of QTCM1.

By wrapping these calls into this file, I also avoid having to
separate out the Fortran QTCM1 subroutines into separate \fn{.F90}
files.  For Fortran subroutines that you want callable from the
Python level, \mods{f2py} seems to require each Fortran subroutine
to be in its own file of the same name (e.g., the version of
\fn{driver.F90} for this package). If several Fortran subroutines
are all found in a single \fn{.F90} files, \mods{f2py} seems unable
to create wrappers for those subroutines.




%---------------------------------------------------------------------
\section{Python \mods{qtcm} and Pure-Fortran QTCM1 Differences}

This section describes differences between how the \mods{qtcm}
package and the pure-Fortran QTCM1 assign some varables.  A list
of changes to the QTCM1 Fortran Files for use in the \mods{qtcm}
package is found in Section~\ref{sec:f90changes}.


	\subsection{QTCM1 \mods{driverinit}}   \label{sec:driverinit.diffs}

In the pure-Fortran version of QTCM1, by default, the following variables are
set by reference (as given below), not by value, in the \mods{driverinit}
routine:\footnote%
	{In the pure-Fortran version of QTCM1, this routine is found
	in \fn{driver.F90}.}
\begin{codeblock}
\codeblockfont{%
lastday\thinspace=\thinspace{daysperyear} \\
viscxu0\thinspace=\thinspace{viscU} \\
viscyu0\thinspace=\thinspace{viscU} \\
visc4x\thinspace=\thinspace{viscU} \\
visc4y\thinspace=\thinspace{viscU} \\
viscxu1\thinspace=\thinspace{viscU} \\
viscyu1\thinspace=\thinspace{viscU} \\
viscxT\thinspace=\thinspace{viscT} \\
viscyT\thinspace=\thinspace{viscT} \\
viscxq\thinspace=\thinspace{viscQ} \\
viscyq\thinspace=\thinspace{viscQ}}
\end{codeblock}

Thus, in pure-Fortran QTCM1, if you change \vars{daysperyear},
\vars{viscU}, etc.
and recompile (as needed), you will automatically change 
\vars{lastday}, \vars{viscxu0}, etc.
(Though, in the pure-Fortran QTCM1, the default values may be overwritten by
namelist input values.)

The \mods{driverinit} routine is eliminated
in the Python \code{qtcm} package.  Instead, inital values 
of field variables are specified in the \mods{defaults} submodule
and set by value to attributes of the \code{Qtcm} instance.
Thus, for instance, in a \class{Qtcm} instance, \code{lastday} 
is set to \code{365} by default, not to some variable
\vars{daysperyear}.  For the diffusion and viscosity terms,
the \class{Qtcm} instance attributes corresponding to those
terms are set to literals.\footnote%
	{Those literals are defined by \mods{defaults} private
	module variables \vars{\_\_viscT}, \vars{\_\_viscQ},
	and \vars{\_\_viscU}.}

In contrast, in the pure-Fortran QTCM1,
\mods{driverinit} declares local
variables \code{viscU}, \code{viscT}, and \code{viscQ},
and reads values into those variables via the input namelist.
Those values are then used to set
\vars{viscxu0}, \vars{viscyu0}, etc., as described above.
In pure-Fortran QTCM1, \code{viscU}, \code{viscT}, and \code{viscQ}
are not directly accessed anywhere else in the model.
Thus, \code{viscU}, \code{viscT}, and \code{viscQ} are not
defined as field variables in the \code{qtcm} package, and
\class{Qtcm} instances do not have attributes corresponding
to those names.
Additionally, if you wish to change a viscosity parameter
\vars{visc*} (given above), the parameter for each direction
must be set one-by-one even if the flow is isotropic.


	\subsection{The \mods{varinit} Routine}

One of the functions of the pure-Fortran QTCM1 \mods{varinit}
subroutine is to associate the pointer variables \vars{u1}, \vars{v1},
\vars{q1}, and \vars{T1}.  For the extension modules in the \mods{qtcm}
package, a Fortran subroutine \mods{varptrinit} is added that can
also do this association.  This subroutine is called in the
\class{Qtcm} instance method
\latexhtml{\mods{varinit}%
		\footnote{http://www.johnny-lin.com/py\_docs/qtcm/doc/html-api/qtcm.qtcm.Qtcm-class.html\#varinit}}%
	{\htmladdnormallink{\mods{varinit}}{http://www.johnny-lin.com/py_docs/qtcm/doc/html-api/qtcm.qtcm.Qtcm-class.html#varinit}}
(which duplicates and
extends the function of its pure-Fortran counterpart, enabling
alternative ways of handling restart).

The \mods{varptrinit} is not accessed via \mods{wrapcall}.  Remember
that \mods{wrapcall} contains only those routines that were in the
original pure-Fortran QTCM1 code, and that we want to have access
to at the Python level.


	\subsection{The \mods{qtcm} Method of \class{Qtcm}}

The \class{Qtcm} method \mods{qtcm} duplicates the functionality
of the \mods{qtcm} subroutine in the pure-Fortran QTCM1 model.
There are a few differences, however.  First, the \mods{qtcm} method
for \class{Qtcm} instances does not include a call to \mods{cplmean},
which uses mean surface flux for air-sea coupling.  This state is
consistent with the pure-Fortran QTCM1 pre-processor macro
\vars{CPLMEAN} being off.  Thus, if you wish to use mean surface
flux for air-sea coupling, you will have to revise the \mods{qtcm}
method of \class{Qtcm} to call \mods{cplmean}.  You'll also have to
check for any other code additions needed that are associated with
the \vars{CPLMEAN} macro.

Second, the \mods{qtcm} method for \class{Qtcm} instances does not
include the option of not using the atmospheric boundary layer
model.  This is consistent with macro \vars{NO\_ABL} being off.  If
you wish to have no atmospheric boundary layer model, change the
run list \vars{atm\_bartr\_mode} so that the \mods{wsavebartr} and
\mods{wgradphis} routines are not called.  You'll also have to check
for any other code additions needed that are associated with the
\vars{NO\_ABL} macro.



	\subsection{Miscellaneous Differences}

\begin{itemize}
\item In Python \class{Qtcm} instances,
	\vars{dateofmodel} is set to 0 by default.  
	In contrast, in the compiled QTCM1 model,
	the default (i.e., initial value) is calculated from 
	\vars{day0}, \vars{month0}, and \vars{year0}.
	See Section~\ref{sec:init.compiledform.full} for details.

\item The \class{Qtcm} instance attribute
	\vars{\_\_qtcm} is not copyable using \mods{copy.deepcopy}.

\item In general, when executing a \class{Qtcm} instance method, 
	if you change a \class{Qtcm} instance attribute 
	that has a counterpart in the compiled QTCM1 model,
	the compiled QTCM1 counterpart is not changed until the
	end of the method.  Likewise, if you call a compiled QTCM1 model
	subroutine and change a compiled QTCM1 model variable with
	a \class{Qtcm} instance counterpart, the \class{Qtcm}
	instance counterpart is not changed until the end of the
	subroutine.

\item In general, even though some of the compiled QTCM1 model
	Fortran subroutines/functions have counterparts in \class{Qtcm}
	that duplicate the former's functionality, the Fortran
	versions are kept intact so that the
	\vars{compiled\_form\thinspace=\thinspace'full'} case will work.
\end{itemize}




%---------------------------------------------------------------------
\section{Considerations When Adding Fortran Code}

In this section I describe issues to consider if you wish to add
your own compiled code to the package as separate extension modules.
(This is different from creating new standard extension modules,
which is described in Section~\ref{sec:create.new.so}.):

\begin{itemize}
\item The \class{Qtcm} class assumes that the directory path 
	to the original shared object file is the same as for the 
	\mods{package\_version} module.

\item If you want to be able to pass other Fortran variables 
	in and out to/from Python, please see the 
	Section~\ref{sec:setbypy}
	discussion of the Fotran \mods{SetByPy} module.

\item Fortran and Python routines to get and set compiled QTCM1 model
	arrays are currently written only for floating point array.

\item If you ever change 
	\class{Qtcm} instance method
	\mods{\_set\_qtcm\_array\_item\_in\_model}
	to work with non-floating point values, you will also
	have to change the array handling section in 
	\mods{set\_qtcm1\_item}.

\item The restart mechanism in the pure-Fortran QTCM1 model is 
	\emph{not} bit-for-bit correct.  Thus, if you compare the final
	output from a 40 day run with a 30 day run restarted from
	a 10 day run, the output will not be the same.
	This behavior has been duplicated in \class{Qtcm} 
	instances when the \vars{mrestart} flag is used
	and applicable.

\item When creating new extension modules using the \fn{src} makefile,
	be sure you first use the \cmd{make clean} command to clean-up
	any old files.

\end{itemize}




%---------------------------------------------------------------------
\section{Creating New Standard Extension Modules}   \label{sec:create.new.so}

The steps involved in creating the standard extension modules (e.g.,
\fn{\_qtcm\_full\_365.so}, etc.) on installation are given in
Section~\ref{sec:create.so}.  The makefile provided in \fn{/buildpath/src}
uses a Fortran compiler to create the object code, runs \mods{f2py}
to create the shared object file in \fn{src}, and moves the shared
object files into \fn{../lib}, overwriting any pre-existing files
of the same name.  In this section, I describe the makefile and
\mods{f2py} in a little more detail, in case you wish to create
standard extension modules with additions from the ones the default
makefile creates.


	\subsection{Makefile Rules}    \label{sec:makefile.rules}

This section describes the rules of the
makefile found in the \fn{src} directory
of the \mods{qtcm} distribution.  
This makefile is used by the Python package to create the extension
module (\fn{.so} files) imported and used by \mods{qtcm} objects
(as described in Section~\ref{sec:create.so}).
The makefile will, in general, be used only during \mods{qtcm}
installation, but if you wish to recompile the QTCM1 libraries
and make changes in the Python extension module,
you'll want to use/change this makefile.

\begin{description}
\item[clean] Removes old files in preparation for compiling new
	extension modules.

\item[libqtcm.a] Creates library \fn{libqtcm.a} that contains all
	QTCM1 object files in the directory \fn{src},, except
	\fn{setbypy.o}, \fn{wrapcall.o}, \fn{varptrinit.o}, and
	\fn{driver.o}.  This archive is compiled with the netCDF
	libraries.  Previous versions of \fn{libqtcm.a} are overwritten.

\item[\_qtcm\_full\_365.so] Creates the extension module
	\fn{\_qtcm\_full\_365.so}.  \mods{f2py} is run on applicable code
	in \fn{src}, and the extension module is moved to \fn{../lib}.
	Any previous versions of \fn{../lib/\_qtcm\_full\_365.so}
	are overwritten.

\item[\_qtcm\_parts\_365.so] Creates the extension module
	\fn{\_qtcm\_parts\_365.so}.  \mods{f2py} is run on applicable code
	in \fn{src}, and the extension module is moved to \fn{../lib}.
	Any previous versions of \fn{../lib/\_qtcm\_parts\_365.so}
	are overwritten.

\end{description}



	\subsection{Using \mods{f2py}}      \label{sec:using.f2py}

This section briefly describes how \mods{f2py} is used in the
makefile during the creation of the extension modules.
\htmladdnormallink{\mods{F2py}}{http://cens.ioc.ee/projects/f2py2e/} is a
program that generates shared object libraries that allow you to call
Fortran routines in Python.  \mods{F2py} comes with Python's
\htmladdnormallink{NumPy}{http://numpy.scipy.org/}
array handling package, so you do not need to install anything
extra if you have NumPy already installed.

To create the extension modules in \mods{qtcm} using
the makefile described in Section~\ref{sec:makefile.rules},
I use a method similar to the
\latexhtml{``Quick and Smart Way,''\footnote%
{http://cens.ioc.ee/projects/f2py2e/usersguide/index.html\#the-quick-and-smart-way}}%
{\htmladdnormallink{``Quick and Smart Way''}%
{http://cens.ioc.ee/projects/f2py2e/usersguide/index.html#the-quick-and-smart-way}}
described in the \mods{f2py} manual.
For the \fn{\_qtcm\_full\_365.so} extension module, the 
\mods{f2py} call is:

\begin{codeblock}
\codeblockfont{%
f2py --fcompiler=\$(FC) -c -m \_qtcm\_full\_365 driver.F90 $\backslash$ \\
\hspace*{10ex}setbypy.F90 libqtcm.a \$(NCLIB)}
\end{codeblock}

and for the \fn{\_qtcm\_parts\_365.so} extension module, the call is:

\begin{codeblock}
\codeblockfont{%
f2py --fcompiler=\$(FC) -c -m \_qtcm\_parts\_365 $\backslash$ \\
\hspace*{10ex}varptrinit.F90 wrapcall.F90 setbypy.F90 $\backslash$ \\
\hspace*{10ex}libqtcm.a \$(NCLIB)}
\end{codeblock}

For both calls, \vars{FC} and \vars{NCLIB} are the environment
variables in the makefile specifying the Fortran compiler and netCDF
libraries, respectively.  The \vars{-m} flag specifies the extension
module name (without the \fn{.so} suffix).  The \fn{.F90} files
specify the files that have modules and routines that will be
accessible at the extension module level, and the rest of the Fortran
files in QTCM1 are compiled and archived in a library \fn{libqtcm.a}.
For \mods{f2py} to work properly,
the \fn{.F90} files may define \emph{only one} module or routine.

If you add Fortran files containing new modules, and you wish those
modules to be accessible at the Python level, compile your new code
with \mods{f2py}.  If we have a file of such new code, \fn{newcode.F90},
the \mods{f2py} call to create the \fn{\_qtcm\_parts\_365.so}
extension module will become:

\begin{codeblock}
\codeblockfont{%
f2py --fcompiler=\$(FC) -c -m \_qtcm\_parts\_365 $\backslash$ \\
\hspace*{10ex}varptrinit.F90 wrapcall.F90 setbypy.F90 $\backslash$ \\
\hspace*{10ex}newcode.F90 $\backslash$ \\
\hspace*{10ex}libqtcm.a \$(NCLIB)}
\end{codeblock}

If you write new Fortran code for the compiled QTCM1 model that
will \emph{not} be accessed from the Python-level, just add the
object code filename to the variable \vars{QTCMOBJS} in the
makefile; you don't have to do anything else.  If you are adding
Fortran code to existing Fortran modules, it's even easier:  You
don't need change the makefile.  Note that for 64 bit processor
machines, you may have to use \mods{f2py} with the \cmd{-fPIC} flag;
see Section~\ref{sec:sopic} for details on how the lines above will
change.


	\subsection{Two Examples}

\emphpara{A Function:}
Let's say you have written a piece of Fortran code called
\fn{myfunction.F90} that contains one function called
\mods{myfunction}, and you want to have this function
callable from the Python level through the \class{Qtcm} 
instance method \mods{\_\_qtcm.myfunction}.  Do the following:

\begin{enumerate}
\item Move \fn{myfunction.F90} to \fn{src} in the \mods{qtcm}
	distribution directory \fn{/buildpath}.

\item Add \cmd{myfunction.o} to the end of the object file list lines
	after the target names
	\vars{\_qtcm\_full\_365.so} and
	\vars{\_qtcm\_parts\_365.so}.

\item In the
	\vars{\_qtcm\_full\_365.so} and
	\vars{\_qtcm\_parts\_365.so} target descriptions,
	add \cmd{myfunction.F90} to the 
	beginning of the list of \fn{.F90} names 
	in the \mods{f2py} lines.
\end{enumerate}


\emphpara{A Module:} 
Let's say you have written a piece of Fortran code called
\fn{mymodule.F90} that contains the Fortran module \mods{MyModule}
containing multiple routines and variables.  You want to have those
routines and variables callable from the Python level through the
\class{Qtcm} instance attribute \mods{\_\_qtcm.mymodule}.  The steps
to add \mods{MyModule} to the extension modules are exactly the
same as for a single function, with \cmd{mymodule} being
substituted in the makefile everywhere you have \cmd{myfunction}.




%---------------------------------------------------------------------
\section{Attributes and Methods in \class{Qtcm} Instances}

In this section I describe some attributes, particularly private ones,
that may be of interest to developers.
As is the convention in Python, private
attributes and methods are prepended by one or two underscores,
with two underscores being the ``more'' private attribute.
Please see the package
\latexhtml{API documentation%
		\footnote{http://www.johnny-lin.com/py\_pkgs/qtcm/doc/html-api/}}
        {\htmladdnormallink{API documentation}%
		{http://www.johnny-lin.com/py\_pkgs/qtcm/doc/html-api/}}
for details about all variables, including private variables.


	\subsection{Public \mods{num\_settings} Submodule Attributes/Methods}

\begin{itemize}
\item \vars{typecode}:  This module function returns the
	type code of the data array passed in as its argument.

\item \vars{typecodes}:  This dictionary is the same as the
	NumPy (or Numeric and \mods{numarray})
	dictionary \vars{typecodes}, except that the character
	\vars{'S'} and \vars{'c'} are added to the
	\vars{typecodes['Character']} entry, if absent.  This
	functionality is added because I found 
	\vars{typecodes['Character']} had different values in
	Mac OS X and Ubuntu GNU/Linux.
\end{itemize}


	\subsection{Private \mods{qtcm} Submodule Attributes}

This submodule of the package \mods{qtcm} is the module that defines
the \class{Qtcm} class.

\begin{itemize}
\item \vars{\_init\_prog\_dict}:  This dictionary contains
	the default values of all prognostic variables and 
	right-hand sides that can be initialized.  In the
	submodule \mods{qtcm}, it is set to
	the \vars{init\_prognostic\_dict} module variable in
	submodule \mods{defaults}.

\item \vars{\_init\_vars\_keys}:  List of all keys in
	\vars{\_init\_prog\_dict}, plus \vars{'dateofmodel'}
	and \vars{'title'}.  These names correspond to the
	field variables that are usually written out into a
	restart file.

\item \vars{\_test\_field}:  \class{Field} object instance used 
	in type tests.
\end{itemize}



	\subsection{Private \class{Qtcm} Attributes}  
					\label{sec:Qtcm.private.attrib}

\begin{itemize}
\item \vars{\_cont}:  A boolean attribute that is \vars{True}
	if the run session is a continuation run session and
	\vars{False} if not.  Set the value passed in by
	the keyword \vars{cont} when the \mods{run\_session}
	method is executed.

\item \vars{\_monlen}:  Integer array of the number of days in 
	each month, assuming a 365~day year.

\item \vars{\_\_qtcm}:  The extension module that is the
	compiled QTCM1 Fortran model for this instance.
	This attribute is unique for every instance:  The
	extension module \fn{.so} file is first copied to
	a temporary directory (given by the \vars{sodir}
	instance attribute) and then imported to the
	\class{Qtcm} instance.
	This private attribute is set on instantiation.

\item \vars{\_qtcm\_fields\_ids}:  Field ids for all default 
	field variables, set on instantiation.

\item \vars{\_runlists\_long\_names}:  Dictionary holding the
	descriptions of the standard run lists.  The keys of
	the dictionary are the names of the standard run lists.
\end{itemize}




%---------------------------------------------------------------------
\section{Creating Documentation}

The distribution of \mods{qtcm} comes with the full set of
documentation in readable form (PDF and HTML).  The documentation
consists of two kinds:  this User's Guide and the API documentation.
The User's Guide is written in \LaTeX.  The PDF version is generated
directly from \LaTeX, and the HTML version is created by
\LaTeX{2}HTML.

I use the \fn{make\_docs} shell script in \fn{doc} creates all these
documents.  Briefly, that script does the following:

\begin{itemize}
\item In the \fn{doc/latex} directory, uses \cmd{python} to
	run \fn{code\_to\_latex.py}, which generates the
	\LaTeX\ files describing the current \mods{qtcm} 
	package settings, including text in the manual which gives
	all uses of the current version number.

\item \LaTeX\ is run on the \LaTeX\ files in the \fn{doc/latex} directory.
	The PDF generated by the run is moved from \fn{doc/latex} to
	\fn{doc}.

\item \LaTeX{2}HTML is run on the \LaTeX\ files in \fn{doc/latex}.
	The HTML files generated by the run are moved to \fn{doc/html}.

\item \mods{epydoc} is run on the \mods{qtcm} package libraries.
	This is run in \fn{doc}, to make use of the \fn{epydoc}
	configuration file present there.  The syntax from the
	command line is:

\begin{codeblock}
\codeblockfont{%
epydoc -v --config epydocrc [name]}
\end{codeblock}
\vars{[name]} is either \cmd{qtcm}, if the \mods{qtcm} package is
installed in a directory listed in \vars{sys.path}, or 
\vars{[name]} is the name of the directory the \mods{qtcm} package is
located in (e.g., \fn{/usr/lib/python2.4/site-packages/qtcm}).

\end{itemize}

The \fn{make\_docs} script cannot be used without customizing it
to your system, so please \emphpara{DO NOT USE IT} if you do
not know what you are doing.  You could easily wipe out all your
documentation by mistake.





% ===== end of file =====


\chapter{Future Work}                       \label{ch:future}
% ==========================================================================
% Future
%
% By Johnny Lin
% ==========================================================================


% ------ BODY -----
%
This section describes the features and fixes I plan to work on
in this package.  The most urgent items are listed closer to the
begining of the lists.

\begin{itemize}
\item Add \code{implicit none} top setbypy.F90.

\item Check through Fortran routines that have arguments, to make sure
	f2py is properly understanding the intentions
	(i.e., in, out, inout) of the variables, since we're using the
	``quick way'' of making shared object libraries using f2py.
	The \fn{utilities.F90} file has a number of Fortran routines
	with arguments.

\item Cite:  Peterson, P. (2009) 
	F2PY: a tool for connecting Fortran and Python programs, 
	\emph{Int. J. Computational Science and Engineering,}
	Vol.\ 4, No.\ 4, pp.\ 296--305 for f2py.

\item Create a method like \mods{calc\_derived('T100')} which would
	primarily operate on a data file and provide a derived variable
	such as the temperature at 100 hPa, as given in this example.
	Figure out where to put the parameters (V1s, etc.) that are
	needed to make such a calculation.  As attributes?  Create a
	method to write the quantity out to an output file?
	Perhaps make an ability to calculate these values at heights
	at a given time each day during a run session?

\item Automate the installation using Python's
\htmladdnormallinkfoot{\mods{distutils}}{http://docs.python.org/dist/dist.html}
	utilities.

\item Describe a way of using job control (either via the operating system
	or IPython's \mods{jobctrl} module) 
	to do a quick-and-dirty parallelization of multiple
	\class{Qtcm} instance run sessions.  Or use some sort of threading
	to fire up two simulataneously running models.  Check that the
	simultaneously running models have different memory space.

\item Add capability for \fn{create\_benchmark.py} to overwrite
	existing benchmark files.

\item Make \vars{compiled\_form} set to \vars{'parts'} as the
	default instantiation.  Change documentation accordingly.

\item Currently, the \class{Qtcm} \mods{plotm} method works only on
	3-D output (time, latitude, longitude).  Some of the fields
	in the netCDF output files are 2-D.  Add the capability to
	\mods{plot\_netcdf\_output} in the \mods{plot} submodule
	to handle 2-D fields.

\item Add documentation about removing temporary files.
	Add documentation in Section~\ref{sec:model.instances}
	of details of what occurs during instantiation of 
	a \class{Qtcm} instance.

\item Add the units and long names for all field variables in the
	\mods{defaults} module.

\item Create a keyword to automatically change precipitation and
	evaporation units to mm/day (or similar).

\item Add ability to calculate and plot fields at different pressure
	levels.  Create another module like defaults that specifies
	the vertical fields and gives the equation to use to calculate
	those fields; call the module ``derivfields'' or something
	similar.

\item Throughout the \mods{qtcm} package I use the condition
	\mods{N.rank(}\dumarg{arg}\mods{)\thinspace=\thinspace0} 
	to test whether
	\dumarg{arg} is a scalar.  This works fine for \mods{numpy}
	objects, but it does not work properly for
	\mods{Numeric} and \mods{numarray} arrays.  In those
	array packages, \mods{rank('abc')} returns the value~1.
	This is not a problem, as long as everyone has \mods{numpy},
	but in order to make the package interoperable, I need to
	find a better way of testing for scalars.  The definitions
	of isscalar need to be changed in \mods{num\_settings}.

\item \mods{num\_settings} needs to be changed to truly enable me
	to test whether \mods{qtcm} works for 
	\mods{numarray} and \mods{Numeric} arrays.  The tests
	do not do this right now, because \mods{num\_settings}
	defaults to \mods{numpy}, if it exists.

\item Create makefiles for other platforms.
 
\item A few fields (e.g., \vars{u1}) have data for extra latitude bands,
	due to the use of ``ghost latitudes'' as part of the
	implementation of the numerics.  Details are found in the 
\latexhtml{%
\htmladdnormallinkfoot{QTCM1 manual}%
        {http://www.atmos.ucla.edu/$\sim$csi/qtcm\_man/v2.3/qtcm\_manv2.3.pdf}}%
{\htmladdnormallink{QTCM1 manual}%
        {http://www.atmos.ucla.edu/~csi/qtcm_man/v2.3/qtcm_manv2.3.pdf}}
\cite{Neelin/etal:2002}.

	Though adjusting to this idiosyncracy is not that difficult, 
	in the future I hope to implement a method of handing
	fields with ghost latitudes so that they have the same
	dimensions as the other gridded output variables.  In order
	to do this, I plan to write a Python method to read the
	Fortran generated binary restart file.

\item Change the \mods{set\_qtcm\_item} method so that it can 
	automatically accomodate setting Fortran real variables
	if integer values are input.

\item Currently, the \mods{get\_item\_qtcm} and 
	\mods{set\_item\_qtcm} methods will not work
	on integer and character arrays, only scalars and real arrays.
	Add that missing functionality to those methods.

\item Currently, the \mods{make\_snapshot} method duplicates the
	functionality of the pure-Fortran QTCM1 restart file mechanism.
	However, the restart file mechanism itself does not do a true
	restart.  A continuous run does not provide the same results
	as two runs over the same period, joined by the restart file.

	To see whether saving more variables would do the trick,
	I altered \mods{make\_snapshot} to store all Python level
	variables (i.e., \vars{self.\_qtcm\_fields\_ids}).  However,
	the restart failing described above still continued.  In the
	future, I hope to figure out exactly how many variables are
	needed in order to make the restart feature do a true
	restart.

\item Add a test of using the \vars{mrestart\thinspace=\thinspace1}
	restart option.  Does the \fn{qtcm.restart} file need to be
	in the current working directory or another?

\item Add a test in the unit test scripts to
	confirm that the \vars{init\_with\_instance\_state}
	attribute setting only has an effect if 
	\vars{compiled\_form\thinspace=\thinspace'parts'}.

\item Document \vars{tmppreview} keyword in \mods{plot.plot\_ncdf\_output}.

\item Confirm and document that
	for netCDF output, time is model time since dd-mm-yyyy.

\item Add to the \mods{plotm} method the ability to
	plot as text onto the figure the
	runname string and the calling line
	for the plotm method.

\item Couple with the
	\latexhtml{CliMT\footnote{http://maths.ucd.ie/$\sim$rca/climt/}}%
	{\htmladdnormallink{CliMT}{http://maths.ucd.ie/~rca/climt/}}
	climate modeling toolkit.

\item Enable Python to set \vars{arr1name}, etc., which are string
	variables at the Python level.  I haven't really thought through
	how \vars{arr1} variables work with the Python \class{Qtcm}
	instance.

\item Possible:  In the \class{Qtcm} method
	\mods{\_\_setattr\_\_}, add a test to raise an exception
	if the instance tries to set \vars{viscU}, \vars{viscT},
	or \vars{viscQ} as attributes.  Also create a method
	\code{isotropic\_visc} that will set all viscosity parameters
	non-dependent on direction.  See Section~\ref{sec:driverinit.diffs}
	for details.

\item Go through the manual and create HTML-only versions of tables
	that have table numbers (use a similar construct as in
	figure environments).

\item Go through documentation to check that
	output variable names are capitalized consistently.

\item Create way to redirect stdout.

\item Create a step method to run an arbitrary number of timesteps at
	the atmosphere level.

\end{itemize}


% ===== end of file =====





% ----- BACK MATTER OF THE DOCUMENT -----
%
\normalsize
\pagebreak
\bibliographystyle{plain}
\bibliography{/Users/jlin/work/res/bib/master}

%- Uncomment the input line below and comment out the \bibliographystyle
%  and \bibliography lines if you're running this without the master.bib 
%  BibTeX database
%% ==========================================================================
% Manual for QTCM Python Package
%
% Usage:
% - If you are running this on your own system, you will not have a copy of
%   my master.bib BibTeX database.  To run this, you'll have to comment out:
%
%      \bibliographystyle{chicago-jl}
%      \bibliography{/Users/jlin/work/res/bib/master}
%
%   and comment back in:
%
%      \input{manual.bbl}
%
%   in this file.  Then you can use pdflatex on this file to get the PDF of
%   the manual.  These 3 lines are in the back matter of the document.
%
% Revision Notes:
% - By Johnny Lin, North Park University, http://www.johnny-lin.com/
% - The chicago BibTeX style is unrecognized by latex2html, so I use
%   the plain style.
% ==========================================================================


% ------ DOCUMENT DEFINITIONS ------
%
\documentclass[12pt]{book}
\usepackage{color}
\usepackage{html}
\usepackage{graphicx}
\usepackage{textcomp}
%\usepackage{comment}    %- Unrecognized by latex2html; its use causes errors
%\usepackage{fancyvrb}   %- Unrecognized by latex2html; its use causes errors


%- Packages unrecognized by latex2html, but causes no error:
%
%\usepackage[letterpaper,margin=1in,includefoot]{geometry}
\usepackage[letterpaper,margin=1.25in]{geometry}
\usepackage{bibnames}
\usepackage{longtable}
\usepackage{multirow}


%+ Comment out explicity margin settings since use package geometry:
%\setlength{\topmargin}{0in}
%\setlength{\headheight}{0in}
%\setlength{\headsep}{0in}
%\setlength{\oddsidemargin}{0in}
%\setlength{\evensidemargin}{0in}
%\setlength{\textheight}{8.5in}
%\setlength{\textwidth}{6.5in}




% ------ COMMANDS AND LENGTHS ------
%
% --- Define colors:  Have to do this because for some reason LaTeX
%     sometimes looks for "BLUE" instead of "blue" and complains when
%     "BLUE" isn't found.
%
\definecolor{Blue}{rgb}{0,0,1}
\definecolor{BLUE}{rgb}{0,0,1}
\definecolor{green}{rgb}{0,0.6,0}
\definecolor{Green}{rgb}{0,0.6,0}
\definecolor{GREEN}{rgb}{0,0.6,0}


% --- Format code blocks.  Currently set to print out the code in just 
%     typewriter font with no box.  Will work the same for pdflatex 
%     and latex2html:
%
%     codeblock:  Environment for blocks of computer code or internet 
%       addresses.
%     codeblockfont:  Sets font for codeblocks.
%
\newenvironment{codeblock}%
	{\begin{quotation}\begin{minipage}[t]{0.9\textwidth}}%
	{\end{minipage}\end{quotation}}
	%{\begin{flushleft}}%
	%{\end{flushleft}}
\newcommand{\codeblockfont}[1]{\textcolor{blue}{\texttt{#1}}}
%     *** Version that only works for pdflatex that puts a box around 
%         the block and centers it (commented out).  Note that using
%         fancyvrb is the better way of creating such a boxed section
%         of code, but fancyvrb isn't recognized by latex2html:
%\newenvironment{codeblock}%
%	{\begin{center}\begin{tabular}{|c|} \hline \\ }%
%	{\\ \\ \hline \end{tabular}\end{center}}
%\newcommand{\codeblockfont}[1]{\parbox{0.8\textwidth}{\texttt{#1}}}


% --- Text titling/emphasis settings:
%
%     emphpara:  Emphasis for the first phrase or sentence of a 
%         paragraph.
%     booktitle:  Formats book titles.
%     tabletitle:  Title for an item block in the information table.
%     paratitle:  Title for a paragraph in an item block in the
%         information table.
%     emphdate:  Emphasize date in paragraph text.
%
%     cmd:  Commands
%     dumarg:  Dummy arguments
%     codearg:  Same as dumarg.
%     fn:  File and directory names
%     screen:  Screen display
%     vars:  Variable and attribute names
%     mods:  Module, subroutine, and method names
%     class:  Class names
%     code:  Generic code (avoid using this)
%
\newcommand{\emphpara}[1]{\textbf{#1}}
\newcommand{\booktitle}[1]{\textit{#1}}
%\newcommand{\tabletitle}[1]{\textsf{\textbf{#1}}}
\newcommand{\paratitle}[1]{\textit{#1}}
\newcommand{\emphdate}[1]{\textbf{#1}}

\newcommand{\code}[1]{\textcolor{blue}{\texttt{#1}}}
\newcommand{\cmd}[1]{\textcolor{blue}{\texttt{#1}}}
\newcommand{\dumarg}[1]{\textit{#1}}
\newcommand{\codearg}[1]{\textit{#1}}
\newcommand{\fn}[1]{\textsf{\textit{#1}}}
\newcommand{\screen}[1]{\textcolor{green}{\texttt{#1}}}
\newcommand{\vars}[1]{\textcolor{blue}{\texttt{#1}}}
\newcommand{\class}[1]{\textcolor{blue}{\texttt{#1}}}
\newcommand{\mods}[1]{\textcolor{blue}{\texttt{#1}}}


% --- Special table formatting:
%
%     tabletitlewidth:  Width for title field of an item block in the 
%         information table.
%     tablebodywidth:  Width for body field of an item block in the 
%         information table.
%     tabletabulardims:  Dimensions for the information table, used in
%         the tabular command.
%     tableitemlinespace:  Vertical spacing between item blocks in the
%         information table.
%     infotitle and infotext:  Used for two-column sub-information 
%         tables found in the body field of the information table.  
%         These are not global lengths but have values specific to the 
%         local context in which they're used.
%
\newlength{\tabletitlewidth}
\settowidth{\tabletitlewidth}{file and directory names}

\newlength{\tablebodywidth}
\setlength{\tablebodywidth}{0.9\textwidth}
\addtolength{\tablebodywidth}{-4ex}
\addtolength{\tablebodywidth}{-\tabletitlewidth}

\newcommand{\tabletabulardims}%
	{p{\tabletitlewidth}@{\hspace{4ex}}p{\tablebodywidth}}

\newcommand{\tableitemlinespace}{\baselineskip}
\newlength{\infotitle}
\newlength{\infotext}


% --- Lengths for formatting:
%
\newlength{\remainder}        % length to describe the residual of the
                              %   linewidth minus \enumlabel
\newlength{\enumlabel}        % length to describe figure sub-label width
                              %   (e.g. "(a)")


% --- TtH stuff:
%
%\def\tthdump#1{#1}


% --- LaTeX2HTML stuff:
%
%     htmlfigcaption:  Formatting for HTML replacement figure captions.
%
\newcommand{\htmlfigcaption}[1]{\parbox[c]{70ex}{\footnotesize{#1}}}


% --- Some book title abbreviations:
%
%     rute:  Booktitle for Rute User's.
%     linuxnut:  Booktitle for Linux in a Nutshell.
%     pynut:  Booktitle for Python in a Nutshell.
%
\newcommand{\rute}{\booktitle{Rute User's}}
\newcommand{\linuxnut}{\booktitle{Linux in a Nutshell}}
\newcommand{\pynut}{\booktitle{Python in a Nutshell}}


% --- Define special characters ---
%
\newcommand{\aonehat}{\ensuremath{\widehat{a_1}}}
\newcommand{\bonehat}{\ensuremath{\widehat{b_1}}}
\newcommand{\D}{\ensuremath{\mathcal{D}}}
\def\BibTeX{B\kern-.03em i\kern-.03em b\kern-.15em\TeX}




% ------ BEGINNING OF DOCUMENT TEXT ------
%
\begin{document}

    

    
% ------ TITLE AND TOC ------
%
\title{\mods{qtcm} User's Guide}
\author{Johnny Wei-Bing Lin\thanks{Physics Department, North Park University,
	3225 W.\ Foster Ave., Chicago, IL  60625, USA}}
\date{\today}
\maketitle
\tableofcontents




% ------ BODY ------
%
\chapter{Introduction}
\input{intro}

\chapter{Installation and Configuration}    \label{ch:install}
	\section{Summary and Conventions}      \label{sec:install.sum}
	\input{install_sum}
	\section{Fortran Compiler}             \label{sec:fort.compilers}
	\input{install_fort}
	\section{Required Packages}            \label{sec:py.etc.pkgs}
	\input{install_pkgs}
	\section{Compiling Extension Modules}  \label{sec:create.so}
	\input{compile_so}
	\section{Testing the Installation}     \label{sec:test.qtcm}
	\input{test_qtcm}
	\section{Model Performance}
	\input{perform}
	\section{Installing in Mac OS X}       \label{sec:install.macosx}
	\input{qtcm_in_macosx}
	\section{Installing in Ubuntu}         \label{sec:install.ubuntu}
	\input{qtcm_in_ubuntu}

\chapter{Getting Started With \mods{qtcm}}  \label{ch:getting.started}
\input{started}

\chapter{Using \mods{qtcm}}                 \label{ch:using}
\input{using}

%@@@\chapter{Combining \code{qtcm} with \code{CliMT}}
%@@@\input{climt}

\chapter{Troubleshooting}                   \label{ch:trouble}
\input{trouble}

\chapter{Developer Notes}                   \label{ch:devnotes}
\input{devnotes}

\chapter{Future Work}                       \label{ch:future}
\input{future}




% ----- BACK MATTER OF THE DOCUMENT -----
%
\normalsize
\pagebreak
\bibliographystyle{plain}
\bibliography{/Users/jlin/work/res/bib/master}

%- Uncomment the input line below and comment out the \bibliographystyle
%  and \bibliography lines if you're running this without the master.bib 
%  BibTeX database
%\input{manual.bbl}        

\appendix
\chapter{Field Settings in \mods{defaults}}  \label{app:defaults.values}
\input{defaults}




% ------ END OF DOCUMENT TEXT ------
%
\end{document}


% ===== end of file =====
        

\appendix
\chapter{Field Settings in \mods{defaults}}  \label{app:defaults.values}
% ==========================================================================
% Appendix:  Defaults from the submodule defaults
%
% By Johnny Lin
% ==========================================================================


% ------ BODY -----
%
%---------------------------------------------------------------------------
\section{Scalar Field Variables}  \label{sec:defaults.scalar}

This table lists the default settings for scalar \mods{qtcm} fields
as set by the \mods{defaults} submodule.  All fields are of class
\class{Field}.  Numerical values are rounded as per the conventions
of Python's \vars{\%g} format code.
To create a \class{Field} instance whose value is set to the
default, instantiate with the field id as the argument

\input{defaults_scalars}




%---------------------------------------------------------------------------
\section{Array Field Variables}   \label{sec:defaults.array}

This table lists the default settings for array \mods{qtcm} fields
as set by the \mods{defaults} submodule.  All fields are of class
\class{Field}.  Numerical values are rounded as per the conventions
of Python's \vars{\%g} format code.

\input{defaults_arrays}




% ===== end of file =====





% ------ END OF DOCUMENT TEXT ------
%
\end{document}


% ===== end of file =====

%
%   in this file.  Then you can use pdflatex on this file to get the PDF of
%   the manual.  These 3 lines are in the back matter of the document.
%
% Revision Notes:
% - By Johnny Lin, North Park University, http://www.johnny-lin.com/
% - The chicago BibTeX style is unrecognized by latex2html, so I use
%   the plain style.
% ==========================================================================


% ------ DOCUMENT DEFINITIONS ------
%
\documentclass[12pt]{book}
\usepackage{color}
\usepackage{html}
\usepackage{graphicx}
\usepackage{textcomp}
%\usepackage{comment}    %- Unrecognized by latex2html; its use causes errors
%\usepackage{fancyvrb}   %- Unrecognized by latex2html; its use causes errors


%- Packages unrecognized by latex2html, but causes no error:
%
%\usepackage[letterpaper,margin=1in,includefoot]{geometry}
\usepackage[letterpaper,margin=1.25in]{geometry}
\usepackage{bibnames}
\usepackage{longtable}
\usepackage{multirow}


%+ Comment out explicity margin settings since use package geometry:
%\setlength{\topmargin}{0in}
%\setlength{\headheight}{0in}
%\setlength{\headsep}{0in}
%\setlength{\oddsidemargin}{0in}
%\setlength{\evensidemargin}{0in}
%\setlength{\textheight}{8.5in}
%\setlength{\textwidth}{6.5in}




% ------ COMMANDS AND LENGTHS ------
%
% --- Define colors:  Have to do this because for some reason LaTeX
%     sometimes looks for "BLUE" instead of "blue" and complains when
%     "BLUE" isn't found.
%
\definecolor{Blue}{rgb}{0,0,1}
\definecolor{BLUE}{rgb}{0,0,1}
\definecolor{green}{rgb}{0,0.6,0}
\definecolor{Green}{rgb}{0,0.6,0}
\definecolor{GREEN}{rgb}{0,0.6,0}


% --- Format code blocks.  Currently set to print out the code in just 
%     typewriter font with no box.  Will work the same for pdflatex 
%     and latex2html:
%
%     codeblock:  Environment for blocks of computer code or internet 
%       addresses.
%     codeblockfont:  Sets font for codeblocks.
%
\newenvironment{codeblock}%
	{\begin{quotation}\begin{minipage}[t]{0.9\textwidth}}%
	{\end{minipage}\end{quotation}}
	%{\begin{flushleft}}%
	%{\end{flushleft}}
\newcommand{\codeblockfont}[1]{\textcolor{blue}{\texttt{#1}}}
%     *** Version that only works for pdflatex that puts a box around 
%         the block and centers it (commented out).  Note that using
%         fancyvrb is the better way of creating such a boxed section
%         of code, but fancyvrb isn't recognized by latex2html:
%\newenvironment{codeblock}%
%	{\begin{center}\begin{tabular}{|c|} \hline \\ }%
%	{\\ \\ \hline \end{tabular}\end{center}}
%\newcommand{\codeblockfont}[1]{\parbox{0.8\textwidth}{\texttt{#1}}}


% --- Text titling/emphasis settings:
%
%     emphpara:  Emphasis for the first phrase or sentence of a 
%         paragraph.
%     booktitle:  Formats book titles.
%     tabletitle:  Title for an item block in the information table.
%     paratitle:  Title for a paragraph in an item block in the
%         information table.
%     emphdate:  Emphasize date in paragraph text.
%
%     cmd:  Commands
%     dumarg:  Dummy arguments
%     codearg:  Same as dumarg.
%     fn:  File and directory names
%     screen:  Screen display
%     vars:  Variable and attribute names
%     mods:  Module, subroutine, and method names
%     class:  Class names
%     code:  Generic code (avoid using this)
%
\newcommand{\emphpara}[1]{\textbf{#1}}
\newcommand{\booktitle}[1]{\textit{#1}}
%\newcommand{\tabletitle}[1]{\textsf{\textbf{#1}}}
\newcommand{\paratitle}[1]{\textit{#1}}
\newcommand{\emphdate}[1]{\textbf{#1}}

\newcommand{\code}[1]{\textcolor{blue}{\texttt{#1}}}
\newcommand{\cmd}[1]{\textcolor{blue}{\texttt{#1}}}
\newcommand{\dumarg}[1]{\textit{#1}}
\newcommand{\codearg}[1]{\textit{#1}}
\newcommand{\fn}[1]{\textsf{\textit{#1}}}
\newcommand{\screen}[1]{\textcolor{green}{\texttt{#1}}}
\newcommand{\vars}[1]{\textcolor{blue}{\texttt{#1}}}
\newcommand{\class}[1]{\textcolor{blue}{\texttt{#1}}}
\newcommand{\mods}[1]{\textcolor{blue}{\texttt{#1}}}


% --- Special table formatting:
%
%     tabletitlewidth:  Width for title field of an item block in the 
%         information table.
%     tablebodywidth:  Width for body field of an item block in the 
%         information table.
%     tabletabulardims:  Dimensions for the information table, used in
%         the tabular command.
%     tableitemlinespace:  Vertical spacing between item blocks in the
%         information table.
%     infotitle and infotext:  Used for two-column sub-information 
%         tables found in the body field of the information table.  
%         These are not global lengths but have values specific to the 
%         local context in which they're used.
%
\newlength{\tabletitlewidth}
\settowidth{\tabletitlewidth}{file and directory names}

\newlength{\tablebodywidth}
\setlength{\tablebodywidth}{0.9\textwidth}
\addtolength{\tablebodywidth}{-4ex}
\addtolength{\tablebodywidth}{-\tabletitlewidth}

\newcommand{\tabletabulardims}%
	{p{\tabletitlewidth}@{\hspace{4ex}}p{\tablebodywidth}}

\newcommand{\tableitemlinespace}{\baselineskip}
\newlength{\infotitle}
\newlength{\infotext}


% --- Lengths for formatting:
%
\newlength{\remainder}        % length to describe the residual of the
                              %   linewidth minus \enumlabel
\newlength{\enumlabel}        % length to describe figure sub-label width
                              %   (e.g. "(a)")


% --- TtH stuff:
%
%\def\tthdump#1{#1}


% --- LaTeX2HTML stuff:
%
%     htmlfigcaption:  Formatting for HTML replacement figure captions.
%
\newcommand{\htmlfigcaption}[1]{\parbox[c]{70ex}{\footnotesize{#1}}}


% --- Some book title abbreviations:
%
%     rute:  Booktitle for Rute User's.
%     linuxnut:  Booktitle for Linux in a Nutshell.
%     pynut:  Booktitle for Python in a Nutshell.
%
\newcommand{\rute}{\booktitle{Rute User's}}
\newcommand{\linuxnut}{\booktitle{Linux in a Nutshell}}
\newcommand{\pynut}{\booktitle{Python in a Nutshell}}


% --- Define special characters ---
%
\newcommand{\aonehat}{\ensuremath{\widehat{a_1}}}
\newcommand{\bonehat}{\ensuremath{\widehat{b_1}}}
\newcommand{\D}{\ensuremath{\mathcal{D}}}
\def\BibTeX{B\kern-.03em i\kern-.03em b\kern-.15em\TeX}




% ------ BEGINNING OF DOCUMENT TEXT ------
%
\begin{document}

    

    
% ------ TITLE AND TOC ------
%
\title{\mods{qtcm} User's Guide}
\author{Johnny Wei-Bing Lin\thanks{Physics Department, North Park University,
	3225 W.\ Foster Ave., Chicago, IL  60625, USA}}
\date{\today}
\maketitle
\tableofcontents




% ------ BODY ------
%
\chapter{Introduction}
%=====================================================================
% Introduction
%=====================================================================


% ----- BEGIN TEXT -----
%
%---------------------------------------------------------------------
\section{How to Read This Manual}

\emphpara{Most users:} 
Just read 
(1) the installation instructions in Chapter~\ref{ch:install},
(2) Chapter~\ref{ch:getting.started},
which tells you all you need to get started using \mods{qtcm}, and
(3) examples in Section~\ref{sec:cookbook} that give a feel
for how you can use the model.

\emphpara{Users having problems:}
Chapter~\ref{ch:trouble} provides troubleshooting tips for
a few problems.
The detailed description of how the package functions, 
in Chapter~\ref{ch:using}, will probably be more useful.

\emphpara{Developers:}
If you want to change the source code, please read
Chapter~\ref{ch:devnotes}.  Chapter~\ref{ch:future} describes
all the things I'd like to do to improve the package, but haven't
gotten to yet.




%---------------------------------------------------------------------
\section{About the Package}

The single-baroclinic mode
Neelin-Zeng Quasi-Equilibrium Tropical Circulation Model
\latexhtml{(QTCM1)\footnote{http://www.atmos.ucla.edu/$\sim$csi}}%
	{\htmladdnormallink{(QTCM1)}{http://www.atmos.ucla.edu/~csi}}
is a primitive equation-based intermediate-level atmospheric model
that focuses on simulating the tropical atmosphere.  Being more
complicated than a simple model, the model has full non-linearity
with a basic representation of baroclinic instability,
includes a radiative-convective feedback package, and includes a
simple land soil moisture routine (but does not include topography).
A brief, but more detailed, description of QTCM1 is given in
Section~\ref{sec:brief_qtcm}.

\htmladdnormallinkfoot{Python}{http://www.python.org}
is an interpreted, object-oriented, multi-platform,
open-source language that is used in a variety of software applications,
ranging from game development to bioinformatics.
In climate studies, Python has been used as the core language for
data analysis
(e.g., \htmladdnormallinkfoot{Climate Data Analysis Tools}{http://cdat.sf.net}),
visualization
(e.g., \htmladdnormallinkfoot{Matplotlib}{http://matplotlib.sf.net}),
and 
modeling
(e.g., \htmladdnormallinkfoot{PyCCSM}{http://code.google.com/p/pyccsm/}).

In comparison to traditional compiled languages like Fortran,
Python's lack of a separate compile step greatly simplifies the
debugging and testing phases of development, because code snippets
can be testing as code is written.
Python's extensive suite of higher-level tools (e.g., statistics,
visualization, string and file manipulation) accessible at runtime 
enables modeling and analysis to occur concurrently.  

The \mods{qtcm} package is an implementation of the Neelin-Zeng
QTCM1 in a Python object-oriented environment.  The conversion
package
\htmladdnormallinkfoot{\mods{f2py}}{http://cens.ioc.ee/projects/f2py2e/} is
used to wrap the QTCM1 Fortran model routines and manage model
execution using Python datatypes and utilities.  The result is a
modeling package where order and choice of subroutine execution can
be altered at runtime.  Model analysis and visualization can also
be integrated with model execution at runtime.




%---------------------------------------------------------------------
\section{Conventions In This Manual}

	\subsection{Audience}

In this manual I assume you have a rudimentary knowledge of Python.
Thus, I do not describe basic Python data types (e.g., dictionaries,
lists), object framework and syntax (e.g., classes, methods,
attributes, instantiation), module and package importing.  If you
need to brush up (or learn) Python, I'd recommend the following
resources:

\begin{itemize}
\item \htmladdnormallinkfoot{Python Tutorial:}{http://docs.python.org/tut/}
	This tutorial was written by Guido van Rossum, Python's original
	author.

\item \htmladdnormallinkfoot{Instant Hacking:}%
	{http://www.hetland.org/python/instant-hacking.php}
	Learn how to program with Python.

\item \htmladdnormallinkfoot{Dive Into Python:}%
	{http://diveintopython.org/index.html}
	Reasonably complete and cohesive. The entire book is available for 
	free online.

\item \htmladdnormallinkfoot{Handbook of the Physics Computing Course:}%
	{http://www.pentangle.net/python/handbook/}
	Written for a science audience.  Recommended.

\item \latexhtml{CDAT/Python Tips for Earth Scientists:\footnote%
	{http://www.johnny-lin.com/cdat\_tips/}}%
	{\htmladdnormallink{CDAT/Python Tips for Earth Scientists:}%
		{http://www.johnny-lin.com/cdat_tips/}}
	This web site is a FAQ of sorts for people using Python and
	the Climate Data Analysis Tools (CDAT) in the earth sciences,
	and thus focuses on using Python to do science rather than
	the computer science aspects of the language.

\end{itemize}

The purpose of this package is to make the QTCM1 model easier to
use.  In order to interpret the results, however, you still need
to have a robust sense of what climate models can and cannot tell
you.  A starting point for the QTCM1 model is the brief description
of the model in Section~\ref{sec:brief_qtcm}.  After that, I would
read the original papers describing the model formulation and results
\cite{Neelin/Zeng:2000,Zeng/etal:2000}, and 
\latexhtml{papers citing the model formulation work.\footnote%
{http://scholar.google.com/scholar?hl=en\&lr=\&cites=14217886709842286738}}%
{\htmladdnormallink{papers citing the model formulation work}%
{http://scholar.google.com/scholar?hl=en&lr=&cites=14217886709842286738}.}
Being an intermediate-level model using the quasi-equilibrium assumption,
QTCM1 (and thus \mods{qtcm}) has distinct strengths and limitations; 
please be aware of them.


	\subsection{Typographic Conventions}

\begin{center}
\begin{tabular}{\tabletabulardims}
\cmd{commands} & to be typed at the command-line
	are rendered in a 
	blue, serif, fixed-width typewriter font
	(e.g., \cmd{make \_qtcm\_full\_365}). \\ \hline
\dumarg{dummy arguments} &
	coupled with code or screen display is rendered in a 
	serif, proportional, italicized font
	(e.g., \screen{Error-Value too long in} \dumarg{variable}). \\ \hline
\fn{file and directory names} & are rendered in a 
	sans-serif, italicized font
	(e.g., \fn{setbypy.F90}). \\ \hline
\screen{screen display} & is rendered in a 
	green, serif, fixed-width typewriter font. \\ \hline
\mods{module, method, and subroutine names} & are rendered in a 
	blue, serif, fixed-width typewriter font. \\ \hline
\vars{variable and attribute names} & are rendered in a 
	blue, serif, fixed-width typewriter font. \\ \hline
\class{class names} & are rendered in a 
	blue, serif, fixed-width typewriter font.
\end{tabular}
\end{center}

Blocks of code (usually commands, screen display, and module,
variable, and class names) are displayed in a blue, serif, fixed-width
typewriter font.


	\subsection{Terminology}

\begin{description}
\item[attribute and method references:]
	If there is any possibility of confusion, I will give the
	class that an attribute or method comes from when that
	attribute or method is referenced.  If no class is mentioned
	by name or context,
	assume that the attribute/method comes from the
	\class{Qtcm} class.

\item[``compiled QTCM1 model'':]
	This usually is used to denote when I'm talking about
	compiled Fortran QTCM1 routines and variables therein,
	as an extension module to the Python \mods{qtcm} package..
	Thus, ``compiled QTCM1 model \vars{u1}'' is the value
	of variable \vars{u1} in the Fortran model, not the
	value at the Python-level.  Sometimes I refer to the
	compiled QTCM1 model as just ``QTCM1'' or as
	``compiled QTCM1 Fortran model''.

\item[``pure-Fortran QTCM1'':]
	This refers to the Neelin-Zeng QTCM1 model in it's
	original Fortran form, not as an extension module to
	the Python \mods{qtcm} package.

\item[``Python-level'':]
	This usually denotes the value of a variable as an
	attribute of a \class{Qtcm} instance.  This variable
	is stored at the Python interpreter level.

\item[\class{Qtcm}:]
	This refers to the Python \class{Qtcm} class
	(note the capitalized first letter).

\item[\mods{qtcm}:]
	This refers to the Python \mods{qtcm} package.

\item[QTCM1 vs.\ QTCM:]
	Although the QTCM1 is currently the only version of a
	quasi-equilibrium tropical circulation model (QTCM), in
	principle one can construct a QTCM with any number of
	baroclinic modes.  In anticipation of this, the names of
	the Python package and class do not end in a numeral.  In
	this manual and the \mods{qtcm} docstrings, QTCM and QTCM1
	are used interchangably.
	Usually either of these phrases mean the quasi-equilibrium
	tropical circulation model in a generic sense, regardless
	of its form of implementation.
\end{description}




%---------------------------------------------------------------------
\section{Current Version Information and Acknowledgments}  \label{sec:ver}

% This file is automatically generated by
    % code_to_latex.py.

This manual describes version 0.1.2 (dated September 12, 2008), of package \mods{qtcm}.
Johnny Linis the primary author of the package.

The \mods{qtcm} package is built upon the pure-Fortran QTCM1 model,
version 2.3 (August 2002), with a few minor changes.
Those changes are described in detail in
Section~\ref{sec:f90changes}.

The homepage for the \mods{qtcm} package is
\htmladdnormallink{http://www.johnny-lin.com/py\_pkgs/qtcm}%
	{http://www.johnny-lin.com/py_pkgs/qtcm}.
All Python code in this package, 
and the Fortran files \fn{setbypy.F90} and \fn{wrapcall.F90},
are \copyright\ 2003--2008 by 
\htmladdnormallinkfoot{Johnny Lin}%
		{http://www.johnny-lin.com} 
and constitutes a
library that is covered under the GNU Lesser General Public License
(LGPL):

\begin{quotation}
	This library is free software; you can redistribute it
	and/or modify it under the terms of the 
	\htmladdnormallinkfoot{GNU Lesser General Public License}%
		{http://www.gnu.org/copyleft/lesser.html} 
	as published by
	the Free Software Foundation; either version 2.1 of the
	License, or (at your option) any later version.

	This library is distributed in the hope that it will be
	useful, but WITHOUT ANY WARRANTY; without even the implied
	warranty of MERCHANTABILITY or FITNESS FOR A PARTICULAR
	PURPOSE. See the GNU Lesser General Public License for more
	details.

	You should have received a copy of the GNU Lesser General
	Public License along with this library; if not, write to
	the Free Software Foundation, Inc., 59 Temple Place, Suite
	330, Boston, MA 02111-1307 USA.

	You can contact Johnny Lin at his email address 
	or at North Park University, Physics Department,
	3225 W. Foster Ave., Chicago, IL 60625, USA.  
\end{quotation}

All other Fortran code in this package, as well as the makefiles,
are covered by licenses (if any) chosen by their respective owners.

This manual, in all forms (e.g., HTML, PDF, \LaTeX),
is part of the documentation of the \mods{qtcm} package 
and is \copyright\ 2007--2008 by Johnny Lin.
Permission is granted to copy, distribute and/or modify 
this document under the terms of the 
GNU Free Documentation License, Version 1.2 
or any later version published by the Free Software Foundation; 
with no Invariant Sections, no Front-Cover Texts, 
and no Back-Cover Texts. 
A copy of the license can be found 
\htmladdnormallinkfoot{here}{http://www.gnu.org/licenses/fdl.html}.

Transparent copies of this document are located online in
\latexhtml{%
\htmladdnormallinkfoot{PDF}%
	{http://www.johnny-lin.com/py\_pkgs/qtcm/doc/manual.pdf}}%
{\htmladdnormallink{PDF}%
	{http://www.johnny-lin.com/py_pkgs/qtcm/doc/manual.pdf}}
and
\latexhtml{%
\htmladdnormallinkfoot{HTML}%
	{http://www.johnny-lin.com/py\_pkgs/qtcm/doc/}}%
{\htmladdnormallink{HTML}%
	{http://www.johnny-lin.com/py_pkgs/qtcm/doc/}}
formats.
The \LaTeX\ source files are distributed with the \mods{qtcm}
distribution.
While the HTML version is nearly identical to the PDF
and \LaTeX\ versions, not every feature in the manual was successfully
converted.  This is particularly true with figures, which are
unnumbered in the HTML version and may be formatted differently
than the authoritative PDF version.
Thus, please consider the \LaTeX\ version as the authoritative
version.

\vspace{\baselineskip}

\emphpara{Acknowledgements:}
Thanks to David Neelin and Ning Zeng and the Climate Systems
Interactions Group at UCLA for their encouragement and help.
On the Python side,
thanks to Alexis Zubrow, Christian Dieterich, Rodrigo Caballero,
Michael Tobis, and Ray Pierrehumbert for steering me straight.
Early versions of some of this work was carried out 
at the University of Chicago Climate Systems Center, 
funded by the National Science Foundation (NSF) 
Information Technology Research Program under grant ATM-0121028. 
Any opinions, findings and conclusions or recommendations 
expressed in this material are those of the author and 
do not necessarily reflect the views of the NSF.

Intel\textregistered\ and
   Xeon\textregistered\ are registered trademarks of Intel Corporation.
Matlab\textregistered\ is a registered trademark of The MathWorks.
UNIX\textregistered\ is a registered trademark of The Open Group.




%---------------------------------------------------------------------
\section{Summary of Release History}

\begin{itemize}
\item 2008 Sep 12:  Version 0.1.2 released.  Summary of changes:
	\begin{itemize}
	\item Create \class{Qtcm} method \mods{get\_qtcm1\_item}.
		This method is effectively an alias of method 
		\mods{get\_qtcm\_item}.
	\item Create \class{Qtcm} method \mods{set\_qtcm1\_item}.
		This method is effectively an alias of method 
		\mods{set\_qtcm\_item}.
	\item Update User's Guide to phase out references to
		the \mods{get\_qtcm\_item}
		and \mods{set\_qtcm\_item} methods.  
		Adding the ``1'' to the method names makes the purpose
		of the methods clearer.
	\item Add unit tests to cover methods \mods{get\_qtcm1\_item} and
		\mods{set\_qtcm1\_item}.
	\end{itemize}

\item 2008 Jul 30:  Updates to the User's Guide (just the online versions,
        not the copies released with the tarball).

\item 2008 Jul 15:  First publicly available distribution 
	released (v0.1.1).
\end{itemize}




%---------------------------------------------------------------------
\section{A Brief Description of The QTCM1}   \label{sec:brief_qtcm}

This description is copied from Ch.\ 3 of Lin \cite{Lin:2000}, 
with minor revisions.
Model formulation is fully described in
Neelin \& Zeng \cite{Neelin/Zeng:2000} and model
results are described in Zeng et~al.\ \cite{Zeng/etal:2000}.
Neelin \& Zeng \cite{Neelin/Zeng:2000} is based upon v2.0 of QTCM1
and Zeng et~al.\ \cite{Zeng/etal:2000} is based on QTCM1 v2.1.
The 
\latexhtml{%
\htmladdnormallinkfoot{QTCM1 manual}%
	{http://www.atmos.ucla.edu/$\sim$csi/qtcm\_man/v2.3/qtcm\_manv2.3.pdf}}%
{\htmladdnormallink{QTCM1 manual}%
	{http://www.atmos.ucla.edu/~csi/qtcm_man/v2.3/qtcm_manv2.3.pdf}}
\cite{Neelin/etal:2002}
describes the details of model implementation.

QTCM1 differs from most full-scale GCMs primarily in how the vertical
temperature, humidity, and velocity structure of the atmosphere is
represented.  First, instead of representing the vertical structure
by finite-differenced levels, the model uses a Galerkin expansion
in the vertical.  The vertical basis functions are chosen according
to analytical solutions under convective quasi-equilibrium conditions,
so only a few need be retained.
Temperature and humidity are each described by separate
vertical basis functions ($a_1$ and $b_1$, respectively).
Low-level variations in the humidity basis
are larger than in the temperature basis.
For velocity, QTCM1 uses a single baroclinic basis function ($V_1$)
defined consistently with the temperature basis function,
as well as a barotropic velocity mode ($V_0$).
The vertical profiles of $a_1$, $b_1$, and $V_1$
are given in Figure~\ref{fig:qtcm.basis}.
Currently, QTCM1 does not include a separate
vertical degree of freedom describing the PBL.
The horizontal grid spacing of the model is 
$5.625^{\circ}$ longitude by $3.75^{\circ}$ latitude.


% <QTCM1 beta version vertical structure modes>
%
% (1) LaTeX version:
%
\begin{latexonly}
\begin{figure}
   \noindent
   \begin{minipage}[b]{.49\linewidth}
      \settowidth{\enumlabel}{(a) }%
      \setlength{\remainder}{\linewidth}% 
      \addtolength{\remainder}{-\enumlabel}
      {(a)}~\makebox[\remainder]{$a_1$ and $b_1$}
      \centering\includegraphics[width=\linewidth,viewport=58 72 389 344,clip]%
                    {figs/a1b1.pdf}
   \end{minipage}\hfill
   \begin{minipage}[b]{.49\linewidth}
      \settowidth{\enumlabel}{(b) }%
      \setlength{\remainder}{\linewidth}% 
      \addtolength{\remainder}{-\enumlabel}
      {(b)}~\makebox[\remainder]{$V_1$}

      \centering\includegraphics[width=\linewidth,viewport=58 72 389 346,clip]%
                    {figs/V1.pdf}
   \end{minipage}

   \caption{Vertical profiles of basis functions for
		(a) temperature $a_1$ (solid) and humidity $b_1$ (dashed) and
		(b) baroclinic component of
		horizontal velocity $V_1$.}
   \label{fig:qtcm.basis}
\end{figure}
\end{latexonly}

% (2) HTML replacement version:
%
\begin{htmlonly}
\label{fig:qtcm.basis}
\begin{center}
\htmladdimg{../latex/figs/a1b1.png}
\htmladdimg{../latex/figs/V1.png}

\htmlfigcaption{Figure \ref{fig:qtcm.basis}:  
	Vertical profiles of basis functions for
   	(a) temperature $a_1$ (solid) and humidity $b_1$ (dashed) and
   	(b) baroclinic component of
   	horizontal velocity $V_1$.}
\end{center}
\end{htmlonly}


These modes are chosen to accurately capture deep convective regions.
Outside deep convective regions the mode
is simply a highly truncated
Galerkin representation.  The system is much more tightly constrained than
a full-scale GCM, yet hopefully retains the essential dynamics and nonlinear
feedbacks.  The result is that QTCM1 is easier to diagnose than a GCM,
and is computationally fast (about 8 minutes per year on a Sun Ultra 2
workstation).  Zeng et al.\ \cite{Zeng/etal:2000} show results indicating
this intermediate-level model does a reasonable job simulating
tropical climatology and ENSO variability.  


Below is a summary of the main model equations \cite{Neelin/Zeng:2000}:
\begin{equation}
   \partial_t \mathbf{v}_1 
      + \D_{V1} (\mathbf{v}_0 , \mathbf{v}_1)
      + f \mathbf{k} \times \mathbf{v}_1
      =
   - \kappa \nabla T_1 
      - \epsilon_1 \mathbf{v}_1 
      - \epsilon_{01} \mathbf{v}_0
   \label{eqn:barocl_wind}
\end{equation}
\begin{equation}
   \partial_t \zeta_0 
      + \mathrm{curl}_z (\D_{V0} (\mathbf{v}_0 , \mathbf{v}_1))
      + \beta v_0
      =
   - \mathrm{curl}_z (\epsilon_0 \mathbf{v}_0)
      - \mathrm{curl}_z (\epsilon_{10} \mathbf{v}_1)
   \label{eqn:barotr_wind}
\end{equation}
\begin{equation}
   \aonehat (\partial_t + \D_{T1}) T_1 
      + M_{S1} \nabla \cdot {\bf v}_1 
      =
   \langle Q_c \rangle
      + (g/p_T) (-R^\uparrow_t -R^\downarrow_s + R^\uparrow_s + S_t - S_s + H)
   \label{eqn:temperature}
\end{equation}
\begin{equation}
   \bonehat (\partial_t + \D_{q1}) q_1 
      - M_{q1} \nabla \cdot {\bf v}_1 
      =
   \langle Q_q \rangle
      + (g/p_T) E
   \label{eqn:moisture}
\end{equation}
where (\ref{eqn:barocl_wind}) describes the baroclinic wind component,
      (\ref{eqn:barotr_wind}) describes the barotropic wind component,
      (\ref{eqn:temperature}) is the temperature equation, and
      (\ref{eqn:moisture}) is the moisture equation.

In the simplest formulation, the vertically integrated
convective heating and moisture sink
are assumed to be equal and opposite, so:
\begin{equation}
  -\langle Q_q \rangle = \langle Q_c \rangle 
                              = \epsilon^\ast_c (q_1 - T_1)
\end{equation}

For its convective parameterization for $Q_c$, this model uses the
Betts-Miller \cite{Betts/Miller:1986} moist convective
adjustment scheme, a scheme that is also used in some GCMs.
In the convective parameterization, the coefficient
$\epsilon^\ast_c$ is defined as:
\begin{equation}
   \epsilon^\ast_c 
      \equiv 
   \aonehat \bonehat (\aonehat + \bonehat)^{-1} \tau_c^{-1} 
      \mathcal{H}( \mathit{C}_{\mathrm{1}} )
\end{equation}
where $\mathcal{H}( \mathit{C}_{\mathrm{1}} )$ is zero for
$C_{1} \leq 0$, and one for $C_{1} > 0$, and $C_{1}$
is a measure of the convective available potential energy (CAPE),
projected onto the model's temperature and moisture basis functions.

Sensible heat ($H$) and evaporation ($E$) are given as
bulk-aerodynamic formulations:
\begin{equation}
   H
      =
   \rho_a C_D \mathrm{V}_s (T_s - (T_{rs} + a_{1s} T_1))
\end{equation}
\begin{equation}
   E
      =
   \rho_a C_D \mathrm{V}_s (q_\mathit{sat} (T_s) 
      - (q_{rs} + b_{1s} q_1))
\end{equation}

Longwave radiation components are denoted by $R$, and net shortwave
radiation is denoted by $S$.
The terms $\D_{V1}$ and $\D_{V0}$ are the advection-diffusion operators
for the momentum equations (projected onto $V_0$ and $V_1 (p)$,
respectively).
The terms $\D_{T1}$ and $\D_{q1}$ are the
advection-diffusion operators for the temperature and moisture
equations, respectively, using a vertical average projection.
The $\langle X \rangle$ and $\widehat{X}$ operators are
equivalent and denote vertically integration over the troposphere.
Please see Neelin \& Zeng \cite{Neelin/Zeng:2000} and 
Zeng et al.\ \cite{Zeng/etal:2000}
for a more complete description of equations and coefficients.







% ====== end file ======


\chapter{Installation and Configuration}    \label{ch:install}
	\section{Summary and Conventions}      \label{sec:install.sum}
	% ==========================================================================
% Installation Summary
%
% By Johnny Lin
% ==========================================================================


% ------ BODY -----
%

This section provides a summary of the steps needed to install
\mods{qtcm}, and a description of the naming conventions used in
this chapter.  If you have had a decent amount of experience with
Python and installing software on a Unix system, this section will
probably be all you need to read.  The installation steps are:

\begin{enumerate}
\item Install a Fortran compiler (see Section~\ref{sec:fort.compilers}
	for a list of compilers known to work).
	This compiler should be in a directory
	listed in your system path (e.g., \fn{/usr/bin}, etc.).

\item Install all required packages
	(see Section~\ref{sec:py.etc.pkgs} for details):
	Python,
	\mods{matplotlib} (plus the \mods{basemap} toolkit),
	NumPy (which includes \mods{f2py}),
	Scientific Python,
	\LaTeX,
	and
	netCDF.

	Python packages are required to be installed on your
	system in a directory listed in your \vars{sys.path},
	and the other packages/libraries are required to be in 
	standard directories listed in your system path 
	(e.g., \fn{/usr/bin}, \fn{/sw/include}, etc.).

	Make sure the executable for Python can be called at the
	Unix command line by typing both \cmd{python}.
	You might need to define a Unix alias
	that maps \cmd{python2.4} (or whichever version of Python
	you are using) to \cmd{python}.

\item \latexhtml{Download\footnote{http://www.johnny-lin.com/py\_pkgs/qtcm/}}%
        {\htmladdnormallink{Download}{http://www.johnny-lin.com/py_pkgs/qtcm/}}
	the \mods{qtcm} tarball and extract the distribution
	into a temporary directory for building purposes.
	\fn{qtcm-0.1.2}is the name of
	the \mods{qtcm} distribution directory;
	the number following the hyphen is the
	version number of the distribution.  \label{list:download.qtcm.sum}

	In this manual, the path to \fn{qtcm-0.1.2}will
	be called the ``\mods{qtcm} build path'' and be given as
	\fn{/buildpath}.  When you see \fn{/buildpath}, please substitute
	the actual temporary directory you created for building purposes.

\item The \mods{qtcm} distribution directory 
	\fn{qtcm-0.1.2}contains the following 
	principal sub-directories:
	\fn{doc}, \fn{lib}, \fn{src}, \fn{test}.
	Documentation is in \fn{doc},
	all the package modules are in \fn{lib},
	building of extension modules will take place in \fn{src},
	and testing of the package is done in \fn{test}.

\item Compile \mods{qtcm} extension modules in \fn{src}:
	Go to \fn{src}, copy the makefile from
	\fn{src/Makefiles} corresponding to your
	system into \fn{src}, rename to \fn{makefile},
	make changes to the makefile as needed,
	and execute:
	\begin{codeblock}
	\codeblockfont{%
	make clean \\
	make \_qtcm\_full\_365.so \\
	make \_qtcm\_parts\_365.so}
	\end{codeblock}
	If you executed the make commands in \fn{src,},
	the extension modules will be automatically placed in
	\fn{lib} in the \fn{qtcm-0.1.2}directory.
	See Section~\ref{sec:create.so} for details.
	\label{list:compile.so.sum}

\item Copy the entire contents of \fn{lib} in
	\fn{qtcm-0.1.2}(not \fn{lib} itself) 
	to a directory named
	\fn{qtcm} that is on your \mods{sys.path}.  For instance,
	for Mac OS X using Fink,
	many Python packages are located in a directory
	named \fn{/sw/\-lib/\-python2.4/\-site-packages}, or something
	similar, and this directory is on the system \mods{sys.path}.  
	If this is the case for your system, copy the
	contents of \fn{lib} into
	\fn{/sw/lib/\-python2.4/\-site-packages/\-qtcm}.
	(For Unix systems, the equivalent directory is usually
	\fn{/usr/\-local/\-lib/\-python2.4/\-site-packages}.)

\item Test the \mods{qtcm} distribution in \fn{test}:
	This step is optional and can take a while.
	Testing requires you to first generate a suite of benchmarks
	using the pure-Fortran QTCM1 model, then running the tests of
	\mods{qtcm} by typing:
	\begin{codeblock}
	\codeblockfont{%
python test\_all.py}
	\end{codeblock}
	at the Unix command line while in \fn{test}.
	See Section~\ref{sec:test.qtcm} for details.

\end{enumerate}

At some point, I will automate the installation using Python's
\htmladdnormallinkfoot{\mods{distutils}}{http://docs.python.org/dist/dist.html}
utilities.



% ===== end of file =====

	\section{Fortran Compiler}             \label{sec:fort.compilers}
	% ==========================================================================
% Fortran compilers
%
% By Johnny Lin
% ==========================================================================


% ------ BODY -----
%

You must have a Fortran compiler installed on your system in order
to compile \mods{qtcm}.  The compiler must be able to interface with
a pre-processor, as QTCM1 makes copious use of pre-processor directives.
\mods{qtcm} is known to work with the following Fortran compilers on the
following platforms:

\begin{center}
\begin{tabular}{l|l|l}
\textbf{Compiler}  & \textbf{Compiler Web Site} & \textbf{Platform(s)} \\ 
\hline
\mods{g95} & \htmladdnormallink{http://www.g95.org/}{http://www.g95.org/}  
	& Mac OS X \\
\end{tabular}
\end{center}

It will probably work with other platforms, but I haven't been able
to test platforms besides those listed above.  Note that \mods{g95}
is not \htmladdnormallink{GNU Fortran}{http://gcc.gnu.org/fortran/}
(\mods{gfortran}), the Fortran 95 compiler included with the more
recent versions of GCC.




% ===== end of file =====

	\section{Required Packages}            \label{sec:py.etc.pkgs}
	% ==========================================================================
% Python packages
%
% By Johnny Lin
% ==========================================================================


% ------ BODY -----
%

The following Python packages are required to be installed on your
system in a directory listed in your \vars{sys.path}:
\begin{itemize}
\item \htmladdnormallinkfoot{Python}%
	{http://www.python.org/}:  The Python programming language
	and interpreter.  Make sure you have a version recent enough
	to be compatible with all the needed Python packages.
\item \htmladdnormallinkfoot{\mods{matplotlib}}%
	{http://matplotlib.sourceforge.net/}:  Scientific plotting
	package, using Matlab-like syntax.  The \mods{basemap} toolkit
	for \mods{matplotlib} must also be installed.
\item \htmladdnormallinkfoot{NumPy}%
	{http://numpy.scipy.org/}:  The standard array package for
	Python.  The module name of NumPy imported in a Python 
	session is \mods{numpy}.
\item \htmladdnormallinkfoot{Scientific Python}%
	{http://dirac.cnrs-orleans.fr/plone/software/scientificpython/}:
	Has netCDF file operators, in addition to other routines
	of use in scientific computing.  The module name of
	Scientific Python imported in a Python session is
	\mods{Scientific}.
\end{itemize}

One other required Python package, \mods{f2py}, is now a part of the
NumPy package, and so installation of NumPy is sufficient to give
you both.

The package \htmladdnormallinkfoot{SciPy}{http://www.scipy.org},
which includes several Python-accessible scientific libraries, also
includes NumPy (and thus \mods{f2py}), so if you install SciPy,
you don't have to install NumPy again.  Note that SciPy is not the
same as Scientific Python; the names are confusing.

A few non-Python packages are also required:
\begin{itemize}
\item \LaTeX: A scientific typesetting program used by the 
	\class{Qtcm} instance method \mods{plotm} to handle 
	exponents and subscripts.  The most common Unix 
	distribution of \LaTeX\ is
	\htmladdnormallinkfoot{teTeX}{http://www.tug.org/teTeX}.

\item netCDF:  This set of libraries enables one to write datasets into
	a platform independent, binary format, with metdata attached.
	The \htmladdnormallinkfoot{netCDF 3.6.2 library}%
        	{http://www.unidata.ucar.edu/software/netcdf/}
	source code can be
\latexhtml{downloaded from UCAR\footnote{http://www.unidata.ucar.edu/downloads/netcdf/netcdf-3\_6\_2/}}%
        {\htmladdnormallink{downloaded from UCAR}{http://www.unidata.ucar.edu/downloads/netcdf/netcdf-3_6_2/}}.
\end{itemize}

For most Unix installations, the easiest way to install all the
above is via a package manager, for instance \mods{apt-get} in
Debian GNU/Linux, \mods{aptitude} or \mods{synaptic} in Ubuntu
GNU/Linux, and \mods{fink} in Mac OS X.  Of course, you can also
download a package's source code and build direct and/or install
using Python's
\htmladdnormallinkfoot{\mods{distutils}}{http://docs.python.org/dist/dist.html}
utilities.




% ===== end of file =====

	\section{Compiling Extension Modules}  \label{sec:create.so}
	% ==========================================================================
% Compiling extension modules
%
% By Johnny Lin
% ==========================================================================


% ------ BODY -----
%

The extension modules (\fn{.so} files) are imported and used by
\mods{qtcm} objects, and contain the Fortran QTCM1 model that is
called by the \mods{qtcm} Python wrappers.  These extension modules
are located in the \fn{lib} directory of the \mods{qtcm} distribution,
and, in general, need to be created only when the \mods{qtcm} package
is installed.

Two extension modules are created:  \fn{\_qtcm\_full\_365.so} and
\fn{\_qtcm\_parts\_365.so}.  Both modules define QTCM1 models where:

\begin{itemize}
\item A year is 365 days long 
	(makefile macro \vars{YEAR360} is off).
\item Model output is written to netCDF files
	(makefile macro \vars{NETCDFOUT} is on).
\item The atmospheric boundary layer model is used
	(makefile macro \vars{NO\_ABL} is off).
\item A global domain is used
	(makefile macro \vars{SPONGES} is off).
\item Topography effects due to induced divergence are not included
	(makefile macro \vars{TOPO} is off).
\item Coupling between atmosphere and ocean is through mean fluxes
	(makefile macro \vars{CPLMEAN} is off).
\item The mixed layer ocean model is not used
	(makefile macros \vars{MXL\_OCEAN} and \vars{BLEND\_SST} are both off).
\end{itemize}

(All other makefile macros not listed are also turned off.)
The only difference between these two extension modules is that the
``full'' module is used by \class{Qtcm} instances where
\vars{compiled\_form} is set to \vars{'full'}, and the ``parts''
module is used by \class{Qtcm} instances where \vars{compiled\_form}
is set to \vars{'parts'}.  See Section~\ref{sec:compiledform} for
details about the \vars{compiled\_form} attribute.

The extension modules are created through the following steps:
\begin{enumerate}
\item Go to the \mods{qtcm} distribution directory
	\fn{qtcm-0.1.2}located in
	your build path \fn{/buildpath}.  Go to the \fn{src}
	sub-directory.  This is where all the building of the
	extension modules will take place.

\item Copy the makefile that corresponds to your platform to
	the \fn{src} directory, and rename it \fn{makefile}.
	The \fn{Makefiles} sub-directory of \fn{src} contains
	makefiles for various platforms.

\item In \fn{makefile}, make the following changes:
	\begin{enumerate}
	\item Change the \vars{FC} environment variable as needed, 
		if your Fortran compiler is different.
	\item Change the \vars{FFLAGSM} environment variable, if the
		compiler flags listed are not supported by your
		compiler.
	\item Change the \vars{-I} and \vars{-L} parts of the
		\vars{NCINC} and \vars{NCLIB} environment
		variables so that the paths for the netCDF library and
		include files match your system's installation:
		\begin{codeblock}
		\codeblockfont{%
NCINC=-I/yourpath/netcdf/include \\
NCLIB=-L/yourpath/netcdf/lib -lnetcdf}
		\end{codeblock}
		Set \dumarg{yourpath} to the full path to the
		\fn{netcdf} directory where the \fn{include} and
		\fn{lib} sub-directories are that hold the netCDF
		libraries and include files.
		(You shouldn't have to change the \vars{-l} part of
		\vars{NCLIB}, since it is standard to name the netCDF
		library \fn{libnetcdf.a}.  But if you have a non-standard
		installation, change the \vars{-l} part too.)
	\end{enumerate}

\item At the Unix prompt, type:
\begin{codeblock}
\codeblockfont{%
\small
make clean \&\& make \_qtcm\_full\_365.so \&\& make \_qtcm\_parts\_365.so}
\end{codeblock}
	to clean up leftover files from previous compilations, and to
	compile the extension module shared object files
	\fn{\_qtcm\_full\_365.so} and \fn{\_qtcm\_parts\_365.so}.
\end{enumerate}

The makefile will automatically move the shared object files into
\fn{../lib}, overwriting any pre-existing files of the same name.
A detailed description of the makefile and using \mods{f2py} is
given in Section~\ref{sec:create.new.so}, if you wish to create a
different extension module.




% ===== end of file =====

	\section{Testing the Installation}     \label{sec:test.qtcm}
	% ==========================================================================
% Installation Summary
%
% By Johnny Lin
% ==========================================================================


% ------ BODY -----
%

The \mods{qtcm} distribution comes with a set of tests for the
package, using Python's \mods{unittest} package.  
Just to warn you, the tests take around an hour to run.
The tests will not work if the contents of \fn{lib}
after you've finished building \mods{qtcm} have not been copied
to a directory named \fn{qtcm} that is on your \mods{sys.path} path,
so make sure you've gone through all the install steps
(summarized in Section~\ref{sec:install.sum}) before you do these
tests.

\emphpara{NB:}  For these tests to work, both \cmd{python} and
\cmd{python2.4} must refer to the executable for the Python
installation on your system that you are using for running \mods{qtcm}.

The tests require a set of benchmark output files in the
\fn{test/benchmarks} directory in the
\fn{qtcm-0.1.2}directory (the output will be in
directories whose names begin with ``aquaplanet'' or ``landon'').
These output files are not included with the \mods{qtcm} distribution,
and must be created, by doing the following:

\begin{enumerate}
\item Goto the directory \fn{test/benchmarks/create/src} in the
	\fn{qtcm-0.1.2}\mods{qtcm} distribution directory,
	and copy the makefile from sub-directory \fn{Makesfiles},
	that corresponds to your system to the
	\fn{test/benchmarks/create/src} directory.  Rename the makefile 
	in \fn{test/benchmarks/create/src} to \fn{makefile}.

\item In \fn{makefile}, make the following changes:
        \begin{enumerate}
        \item Change the \vars{FC} environment variable as needed,
                if your Fortran compiler is different.
        \item Change the \vars{FFLAGSM} environment variable, if the
                compiler flags listed are not supported by your
                compiler.
        \item Change the \vars{-I} and \vars{-L} parts of the
                \vars{NCINC} and \vars{NCLIB} environment
                variables so that the paths for the netCDF library and
                include files match your system's installation:
                \begin{codeblock}
                \codeblockfont{%
NCINC=-I/yourpath/netcdf/include \\
NCLIB=-L/yourpath/netcdf/lib -lnetcdf}
                \end{codeblock}
                Set \dumarg{yourpath} to the full path to the
                \fn{netcdf} directory where the \fn{include} and
                \fn{lib} sub-directories are that hold the netCDF
                libraries and include files.
                (You shouldn't have to change the \vars{-l} part of
                \vars{NCLIB}, since it is standard to name the netCDF
                library \fn{libnetcdf.a}.  But if you have a non-standard
                installation, change the \vars{-l} part too.)
        \end{enumerate}

\item Go to the directory \fn{test/benchmarks/create} in the
	\fn{qtcm-0.1.2}\mods{qtcm} distribution directory.

\item Type \cmd{python create\_benchmarks.py} at the Unix command line
	to run the benchmark creation script.
\end{enumerate}

The created benchmarks will be located in 
\fn{test/benchmarks}, in directories with names related to the
run that was done, as described earlier.
The benchmarks are created using the
pure-Fortran QTCM1 model code,
version 2.3 (August 2002), with an altered makefile
(described above) and the following code change:
In all \fn{.F90} files, occurrences of:
        \begin{codeblock}
        \codeblockfont{%
        Character(len=130)}
        \end{codeblock}
        are changed to:
        \begin{codeblock}
        \codeblockfont{%
        Character(len=305)}
        \end{codeblock}
This enables the model to properly deal with longer filenames.
The number ``305'' is chosen to make search and replace easier.

Once the benchmarks are created, you can test the \mods{qtcm} package
by doing the following:
\begin{enumerate}
\item Go to the \fn{test} directory in the 
	\fn{qtcm-0.1.2}directory.
\item Type \cmd{python test\_all.py} at the Unix command line.
\end{enumerate}

If at the end of the test runs you see this message (or something similar):
\begin{codeblock}
\codeblockfont{%
\footnotesize
---------------------------------------------------------------------- \\
Ran 93 tests in 1244.205s \\
 \\
OK}
\end{codeblock}
then everything worked fine!  If you get any other message, the test(s) have
failed.



% ===== end of file =====

	\section{Model Performance}
	%=====================================================================
% Model Performance
%=====================================================================


% ----- BEGIN TEXT -----
%
%---------------------------------------------------------------------

The wall-clock time values below give the mean over three
separate 365 day aquaplanet runs,
using climatological sea surface temperature for lower boundary forcing.
NetCDF output is written daily, for both instantaneous and mean values.
The time step is 1200~sec, and the version of \mods{qtcm} used
is 0.1.1.
The horizontal grid spacing of all model versions is
$5.625^{\circ}$ longitude by $3.75^{\circ}$ latitude.
Values are in seconds:
\begin{center}
\begin{tabular}{p{0.5\linewidth}|c|c|c}
\textbf{System} & \textbf{Pure} & \textbf{Full} & \textbf{Parts} \\
\hline
Mac OS X:  MacBook 1.83 GHz Intel Core Duo running Mac OS X
	10.4.10 with 1 GB RAM
	(Python 2.4.3, NumPy 1.0.3, \mods{f2py} 2\_3816).
    & 152.59 & 153.63 & 158.94 \\
\hline
Ubuntu GNU/Linux:  Dell PowerEdge 860 with 2.66 GHz Quad Core Intel
	Xeon processors (64 bit) running Ubuntu 8.04.1 LTS
	(Python 2.5.2, NumPy 1.1.0, \mods{f2py} 2\_5237).
    & 43.73 & 44.79 & 47.45
\end{tabular}
\end{center}

``Pure'' refers to the pure-Fortran version of QTCM1.
``Full'' refers to a \mods{qtcm} run session with \vars{compiled\_form}
set to \vars{'full'}.  ``Parts'' refers to a \mods{qtcm} run session
with \vars{compiled\_form} set to \vars{'parts'}.
(Section~\ref{sec:compiledform} has details about the difference
between compiled forms.)

The \vars{'parts'} version of \mods{qtcm} gives Python the maximum
flexibility in accessing compiled QTCM1 model subroutines and
variables.  The price of that flexibility is an increase in
run time of approximately 4--9\% over the pure-Fortran version.
The difference in performance between the
\vars{'full'} version of \mods{qtcm} and the pure-Fortran version of
QTCM1 is between negligible and 3\% longer.

To make a timing for the pure-Fortran model, go to
\fn{test/benchmarks/timing/work} in \fn{/buildpath} and run the
\fn{timing\_365.sh} script in that directory.  That script runs the
QTCM1 model using \cmd{/usr/bin/time}, which at the end of the
script will output the amount of time it took to make the model
run.  Run the timing script three times and average the values to
obtain a time comparable to the above.

To make a timing for the \mods{qtcm} model, type \cmd{python
timing\_365.py} while in the \fn{test} directory in \fn{/buildpath}.
Three run sessions will be made for \vars{compiled\_form} equal to
\vars{'full'} and \vars{'parts'}, the times are averaged, and the
value are output at the end of the script.




% ====== end file ======

	\section{Installing in Mac OS X}       \label{sec:install.macosx}
	% ==========================================================================
% Description of installing in Mac OS X
%
% By Johnny Lin
% ==========================================================================


% ------ BODY -----
%
%------------------------------------------------------------------------
\subsection{Introduction}

This section describes issues and a summary of the installation steps
I followed to install \mods{qtcm} on a Mac running OS X.
It is a specific realization of the general installation
instructions found in Sections~\ref{sec:install.sum}--\ref{sec:test.qtcm}.
I first worked through these installation steps during June--July 2007,
with updates during July 2008.
The best way to go through this section is to go through
the summary of the installation steps in 
Section~\ref{sec:osx.install.summary},
and looking back to other sections as needed.




%------------------------------------------------------------------------
\subsection{Platform and Unix Dependencies}

This work was done on a MacBook 1.83 GHz Intel Core Duo running Mac OS X
10.4.11.  My machine has 1 GB RAM and 64 GB of disk in its main partition.

I recommend you turn-off your antivirus software before you
do the installs.  
Problems have been
\latexhtml{reported by Fink users\footnote%
		{http://finkproject.org/faq/usage-fink.php?phpLang=en\#kernel-panics}}%
	{\htmladdnormallink{reported by Fink users}%
		{http://finkproject.org/faq/usage-fink.php?phpLang=en#kernel-panics}}
using the Fink package manager with antivirus software enabled.

There are a variety of dependencies that are required to get your Mac
up-and-running as a scientific computing platform.  The most basic is
installing Apple's 
\htmladdnormallinkfoot{XCode}{http://developer.apple.com/tools/xcode/}
developer tools.\footnote%
	{The package should work in Mac OS X 10.4 with XCode 2.4.1 and higher;
	I've tried it with both 2.4.1 and XCode 2.5.  Note that
	XCode 3.1 only works on Mac OS X 10.5.}
This set of tools contains compilers and libraries
needed to do anything further.  You have to be a member of Apple's
Developer Connection, but registration is free.

Besides XCode, there are a variety of Unix libraries and utilities that you
need.  I first tried installing them by myself, from scratch, into
\fn{/usr/local}, but it was hard to keep track of all the dependencies.
A few that did work, and that I installed from their disk images, are:
\htmladdnormallinkfoot{MacTeX}{http://www.tug.org/mactex/}, 
\htmladdnormallinkfoot{MAMP}{http://www.mamp.info/}, and 
\htmladdnormallinkfoot{Tcl/Tk Aqua BI (Batteries Included)}%
	{http://tcltkaqua.sourceforge.net/}.\footnote%
		{Theoretically you can use Fink to install the equivalent
		of these packages, but I like the specific collection 
		found in these packages.  For instance, Tcl/Tk Aqua BI
		runs natively on the Mac.}

For everything else, thankfully, there's the
\htmladdnormallinkfoot{Fink Project}{http://www.finkproject.org/} which
uses a package manager built upon Debian tools to install ports of
Unix programs onto a Mac.  I just 
\htmladdnormallinkfoot{downloaded}%
	{http://www.finkproject.org/download/index.php?phpLang=en}
a binary version of the Fink 0.8.1 installer for Intel Macs,
installed Fink, and used its package management tools to install
(almost) everything else I needed.\footnote%
	{The one drawback of Fink is that it sometimes
	has stability problems.  In those cases, Fink provides
	command line suggestions to fix the problems, which sometimes
	will work.  If not, sometimes
\latexhtml{%
	deleting Fink and everything it installed,\footnote%
	{http://www.finkproject.org/faq/usage-fink.php?phpLang=en\#removing}}{%
\htmladdnormallink{deleting Fink and everything it installed}
	{http://www.finkproject.org/faq/usage-fink.php?phpLang=en#removing},}
	and starting afresh, will do the trick.
	It also appeared to me that sometimes when I installed 
	multiple packages
	via one \cmd{fink install} call, the installation did not work
	as well as when I installed only one package per call.}

Although you do not need anything besides a Fortran compiler and
the netCDF libraries to run QTCM1 in its pure-Fortran form, in order to
manipulate the model and use this Python version \mods{qtcm}, you
need to have Python installed.  The default Python that comes
with the Mac is a little old, so I used Fink to also install
Python 2.5 and related packages, including
\htmladdnormallinkfoot{matplotlib}{http://matplotlib.sourceforge.net/},
\htmladdnormallinkfoot{ScientificPython}{http://dirac.cnrs-orleans.fr/plone/software/scientificpython/},
and
\htmladdnormallinkfoot{SciPy}{http://www.scipy.org}
(see Section~\ref{sec:osx.summary} for details).




%------------------------------------------------------------------------
\subsection{Fortran Compiler}

There are a variety of high-quality, commercial Fortran compilers.
Unfortunately, because I do not have a research budget, I am not able
to use those compilers.  The 
\htmladdnormallinkfoot{GNU Compiler Collection}{http://gcc.gnu.org/}
(GCC) provides a suite of open-source compilers, some of which are the
standards of their language.  Most of the GCC compilers are installed
on your Mac when you install XCode.

\htmladdnormallinkfoot{GNU Fortran}{http://gcc.gnu.org/fortran/}
(\mods{gfortran}), is the Fortran 95 compiler included with the more
recent versions of GCC.
Unfortunately, I was not able to get it to compile QTCM1.
There is a second open-source Fortran compiler,
\htmladdnormallinkfoot{G95}{http://www.g95.org/} (\mods{g95}),
which some feel is farther along in its development than \mods{gfortran}.
I was able to successfully compile QTCM1 with \mods{g95} on my Mac.
I used Fink to install G95
(see Section~\ref{sec:osx.summary} for details).




%------------------------------------------------------------------------
\subsection{NetCDF Libraries}   \label{sec:netcdf}

For some reason, the netCDF libraries and include files
installed by Fink didn't correspond to the files needed
by the calling routines in \mods{qtcm}.  To solve this, I compiled
my own set of 
\htmladdnormallinkfoot{netCDF 3.6.2 libraries}%
	{http://www.unidata.ucar.edu/software/netcdf/}
using the tarball 
\latexhtml{downloaded from UCAR\footnote{http://www.unidata.ucar.edu/downloads/netcdf/netcdf-3\_6\_2/}}%
        {\htmladdnormallink{downloaded from UCAR}{http://www.unidata.ucar.edu/downloads/netcdf/netcdf-3_6_2/}}.

Once I uncompressed and untarred the package, and went into 
the top-level directory of the package, I built the package by typing
the following at the Unix prompt:

\begin{codeblock}
\codeblockfont{%
./configure --prefix=/Users/jlin/extra/netcdf \\
make check \\
make install}
\end{codeblock}

This installed the netCDF binaries, libraries, and include files into
sub-directories \fn{bin}, \fn{lib}, and \fn{include} in 
the directory specified by \vars{--prefix}.
If you want to install the netCDF libraries in the default
(usually \fn{/usr/local}), just leave out the \vars{--prefix}
option.

Note:  When you build netCDF, make sure the build directory
is not in the directory tree of \vars{--prefix}
(or the default directory \fn{/usr/local}).




%------------------------------------------------------------------------
\subsection{Makefile Configuration}  \label{sec:osx.makefile}

	\subsubsection{NetCDF}

In the \fn{src} directory in the \mods{qtcm} distribution, there is a
sub-directory \fn{Makefiles} that contains the makefiles for a
variety of platforms.  Edit the file \fn{makefile.osx\_g95}
so that the lines specifying the environment variables for the
netCDF libraries and include files:

\begin{codeblock}
\codeblockfont{%
NCINC=-I/Users/jlin/extra/netcdf/include \\
NCLIB=-L/Users/jlin/extra/netcdf/lib -lnetcdf}
\end{codeblock}

are changed to the path where your \emph{manually compiled} 
netCDF libraries and include files are.

Copy \fn{makefile.osx\_g95} from the \fn{Makefiles} sub-directory
in \fn{src} into \fn{src}.  
In other words, from the \mods{qtcm} distribution directory
(i.e., \fn{/buildpath}), at the Unix prompt execute:

\begin{codeblock}
\codeblockfont{%
cp src/Makefiles/makefile.osx\_g95 src/makefile}
\end{codeblock}


	\subsubsection{Linking Order}

Compilers in the GNU Compiler Collection (GCC) search libraries
and object files in the order they are listed in the command-line, 
\latexhtml{from left-to-right\footnote%
        {http://gcc.gnu.org/onlinedocs/gcc-4.1.2/gcc/Link-Options.html\#index-l-670}}%
        {\htmladdnormallink{from left-to-right}{http://gcc.gnu.org/onlinedocs/gcc-4.1.2/gcc/Link-Options.html#index-l-670}}.
Thus, if routines in \fn{b.o} call routines in \fn{a.o}, 
you must list the files in the order \fn{a.o b.o}.

For some reason, that isn't the case for \mods{g95}.  Thus, you will
find \mods{g95} makefile rules structured like the following
(below is part of the rule to create an executable (\fn{qtcm}) for
benchmark runs):

% --- Two versions of this rule, one for display in PDF and the other
%     for display in HTML:
%
\begin{latexonly}
\begin{codeblock}
\codeblockfont{%
qtcm: main.o \\
\hspace*{8ex}\$(FC)~-O~\$(NCINC)~-o~\$@ main.o~\$(QTCMLIB)~\$(NCLIB)}
\end{codeblock}
\end{latexonly}

\begin{htmlonly}
\begin{rawhtml}
<p><code><font color="blue">qtcm: main.o<br>
&nbsp;&nbsp;&nbsp;&nbsp;&nbsp;&nbsp;&nbsp;$(FC) -O $(NCINC) -o 
$@ main.o $(QTCMLIB) $(NCLIB)</font></code></p>
\end{rawhtml}
\end{htmlonly}

even though \fn{main.o} depends on the QTCM library 
(specified in macro setting \vars{\$(QTCMLIB)}), which in turn
depends on the netCDF library (specified in macro setting \vars{\$(NCLIB)}).


%------------------------------------------------------------------------
\subsection{Summary of Steps}   \label{sec:osx.install.summary}

The following summarizes all the steps I took to install
\mods{qtcm} in Mac OS X:

\begin{enumerate}
\item Install
	\htmladdnormallinkfoot{XCode 2.5}%
		{http://developer.apple.com/tools/xcode/}.

\item Install 
	\htmladdnormallinkfoot{MacTeX}{http://www.tug.org/mactex/}, 
	\htmladdnormallinkfoot{MAMP}{http://www.mamp.info/}, and 
	\htmladdnormallinkfoot{TCL/Tk Aqua BI (Batteries Included)}%
		{http://tcltkaqua.sourceforge.net/}.

\item Install
	\htmladdnormallinkfoot{Fink 0.8.1}%
		{http://www.finkproject.org/download/index.php?phpLang=en}.
	Make sure you
	\htmladdnormallink{set up your environment}%
		{http://www.finkproject.org/doc/users-guide/install.php\#setup}
	to enable you to use the packages you install with Fink
	(e.g. \vars{PATH} settings, etc.).
	Most of the time, that just means adding the line
	\cmd{source /sw/bin/init.csh} to your \fn{.cshrc} file (or the
	equivalent in your \fn{.bashrc}).

	Note that for many of the packages needed to run \mods{qtcm},
	you need to 
	\htmladdnormallink{configure Fink to download packages 
		from the unstable trees}%
	{http://www.finkproject.org/faq/usage-fink.php?phpLang=en\#unstable}.
	To do that, add \vars{unstable/main} and \vars{unstable/crypto}
	to the \vars{Trees:} line in \fn{/sw/etc/fink.conf}, and run:

	\begin{codeblock}
	\codeblockfont{fink selfupdate} \\
	\codeblockfont{fink index} \\
	\codeblockfont{fink scanpackages} \\
	\codeblockfont{fink update-all}
	\end{codeblock}

	When \cmd{selfupdate} runs, choose \cmd{rsync} for the
	self update method.  If you do not, Fink will not look in the
	unstable trees for packages.

\item Use Fink to install the \mods{g95} Fortran compiler.
	From a Unix prompt, type:

	\begin{codeblock}
	\codeblockfont{fink -$\,\!$-use-binary-dist install g95}
	\end{codeblock}

\item Use Fink to install Python 
	and the NumPy package (which \mods{f2py} is a part of).
	From a Unix prompt, type:

	\begin{codeblock}
	\codeblockfont{%
	fink -$\,\!$-use-binary-dist install python25 \\
	fink -$\,\!$-use-binary-dist install scipy-core-py25}
	\end{codeblock}

	(Numpy used to be called SciPy Core.)  If you want to
	install Python 2.4 instead, just change the ``25'' and ``py25'' above
	(and in later occurrences) to ``24'' and ``py24'', respectively.
	Note that Fink does not have a version of epydoc for Python 2.4,
	so if you wish to create documentation using epydoc, you will
	need to install Python 2.5.

\item Install teTeX and \LaTeX{2HTML} using Fink.
	From a Unix prompt, type:

	\begin{codeblock}
	\codeblockfont{fink -$\,\!$-use-binary-dist install tetex} \\
	\codeblockfont{fink -$\,\!$-use-binary-dist install latex2html}
	\end{codeblock}

	When prompted, choose ghostscript and ghostscript-fonts to
	satistfy the dependency (which should be the default options).
	I tried choosing system-ghostscript8, but Fink looks for
	ghostscript 8.51 and didn't recognize ghostscript 8.57 that
	was already installed in \fn{/usr/local} (via my MacTeX
	install).  \LaTeX{2HTML} has a package required by the
	\mods{qtcm} manual \LaTeX\ file.

\item Install additional programming and
	scientific packages and libraries using Fink.
	From a Unix prompt, type:

	\begin{codeblock}
	\codeblockfont{%
	fink -$\,\!$-use-binary-dist install scientificpython-py25 \\
	fink -$\,\!$-use-binary-dist install matplotlib-py25 \\
	fink -$\,\!$-use-binary-dist install matplotlib-basemap-py25 \\
	fink -$\,\!$-use-binary-dist install matplotlib-basemap-data-py25 \\
	fink -$\,\!$-use-binary-dist install xaw3d \\
	fink -$\,\!$-use-binary-dist install fftw fftw3 \\
	fink -$\,\!$-use-binary-dist install epydoc-py25 \\
	fink -$\,\!$-use-binary-dist install graphviz \\
	fink -$\,\!$-use-binary-dist install scipy-py25}
	\end{codeblock}

\item Manually install netCDF 3.6.2
	(see Section \ref{sec:netcdf}).

\item From this point on, you can follow the
	general instructions given in Section~\ref{sec:install.sum},
	starting with step~\ref{list:download.qtcm.sum}.
	Please do not ignore, however, Section~\ref{sec:install.macosx}'s
	Mac-specific details.

\end{enumerate}



% ===== end of file =====

	\section{Installing in Ubuntu}         \label{sec:install.ubuntu}
	% ==========================================================================
% Description of installing in Ubuntu
%
% By Johnny Lin
% ==========================================================================


% ------ BODY -----
%
%------------------------------------------------------------------------
\subsection{Introduction}

This section describes installation issues 
I followed to install \mods{qtcm} on my
Dell PowerEdge 860 running Ubuntu GNU/Linux 8.04.1 LTS (Hardy).
The machine has 2.66 GHz Quad Core Intel Xeon processors (64 bit),
4 GB RAM, and 677 GB of disk in its main partition.
This section is a specific realization of the general installation
instructions found in Sections~\ref{sec:install.sum}--\ref{sec:test.qtcm}.
I worked through these installation steps during July 2008.
The best way to go through this section is to go through
the summary of the installation steps in 
Section~\ref{sec:ubuntu.install.summary},
and looking back to other sections as needed.



%------------------------------------------------------------------------
\subsection{Fortran Compiler}     \label{sec:ubuntu.fort.install}

The easiest Fortran compiler to install in Ubuntu 8.04.1 is
\htmladdnormallinkfoot{GNU Fortran}{http://gcc.gnu.org/fortran/}
(\mods{gfortran}), the Fortran 95 compiler included with the more
recent versions of the GNU Compiler Collection (GCC); you can
use any package manager (e.g., \mods{apt-get}, \mods{aptitude})
to install it.
Unfortunately, I was not able to get it to compile QTCM1.
I was, however, able to successfully compile QTCM1 using
the second open-source Fortran compiler,
\htmladdnormallinkfoot{G95}{http://www.g95.org/} (\mods{g95}),
which some feel is farther along in its development than \mods{gfortran}.
G95, however, is not supported as an Ubuntu package, and so I had
to manually install it.

I downloaded the binary version of G95 v0.91 
(the Linux x86\_64/EMT64 with 32 bit default integers) 
using the following
\cmd{curl} command:\footnote%
	{I use \mods{curl} because I usually access my
	Ubuntu server via a terminal session.}

\begin{codeblock}
\codeblockfont{%
\small
curl -o g95.tgz http://ftp.g95.org/v0.91/g95-x86\_64-32-linux.tgz}
\end{codeblock}

which saves the \fn{.tgz} file as the local file \fn{g95.tgz}.
After that, I followed the G95 project's standard
\latexhtml{installation instructions\footnote%
	{http://g95.sourceforge.net/docs.html\#starting}}%
	{\htmladdnormallink{installation instructions}%
		{http://g95.sourceforge.net/docs.html#starting}}
to finish the install.\footnote%
	{The G95 installation instructions say you can put
	\fn{g95-install} anywhere, and make a link to the
	executable \mods{g95} in
	\fn{$\sim$/bin}.  I put \fn{g95-install} in
	\fn{/usr/local}, and while in \fn{/usr/local/bin}, 
	I put a link to the G95 executable using the command:
	\begin{codeblock}
	\codeblockfont{%
	sudo ln -s ../g95-install\_64/bin/x86\_64-suse-linux-gnu-g95 g95.}
	\end{codeblock}}
The regular Linux x86 version of G95
(in \fn{g95-x86-linux.tgz} from the G95 website) did not work on my
machine.




%------------------------------------------------------------------------
\subsection{NetCDF Libraries}   \label{sec:ubuntu.netcdf}

%Here things were very confusing for my machine, as I needed to
%install two versions of the
%\htmladdnormallinkfoot{netCDF}%
%	{http://www.unidata.ucar.edu/software/netcdf/}
%libraries and include files, one 
%for a successful compilation of the extension modules
%(as described in Section~\ref{sec:create.so}),
%and the other 
%for a successful run of the pure-Fortran QTCM1 model
%(used to create the testing benchmarks, as described in
%Section~\ref{sec:test.qtcm}).
%
%The first set of netCDF files (for the extension modules) are
%installed from Ubuntu's package management system.
%These are automatically installed when the \mods{python-netcdf}
%package is installed via an Ubuntu package manager
%(see Section~\ref{sec:ubuntu.install.summary}).
%The include files for this netCDF installation are 
%located in \fn{/usr/include}, and the libraries for this
%netCDF installation are location in \fn{/usr/lib}.

For some reason, the netCDF libraries and include files
installed from the Ubuntu packages do not
correspond to the files needed
by the calling routines in \mods{qtcm}.  To solve this, I compiled
my own set of
\htmladdnormallinkfoot{netCDF 3.6.2 libraries}%
        {http://www.unidata.ucar.edu/software/netcdf/}
using the tarball
\latexhtml{downloaded from UCAR\footnote{http://www.unidata.ucar.edu/downloads/netcdf/netcdf-3\_6\_2/}}%
        {\htmladdnormallink{downloaded from UCAR}{http://www.unidata.ucar.edu/downloads/netcdf/netcdf-3_6_2/}}.

Once I uncompressed and untarred the package, and went into
the top-level directory of the package, I built the package by typing
the following at the Unix prompt:

\begin{codeblock}
\codeblockfont{%
export FC=g95 \\
export FFLAGS="-O -fPIC" \\
export FFLAGS="-fPIC" \\
export F90FLAGS="-fPIC" \\
export CFLAGS="-fPIC" \\
export CXXFLAGS="-fPIC" \\
./configure \\
make check \\
sudo make install}
\end{codeblock}

(The \cmd{export} commands set environment variables for the
Fortran compiler and Fortran and other compiler flags.  The
\vars{-fPIC} flag enables the compilers to create
position independent code, needed for shared libraries in
Ubuntu on a 64 bit Intel processor.)

The above installs the netCDF binaries, libraries, and include files into
sub-directories \fn{bin}, \fn{lib}, and \fn{include} in 
\fn{/usr/local}, the default.
The include files for this netCDF installation are thus
located in \fn{/usr/local/include}, and the libraries for this
netCDF installation are location in \fn{/usr/local/lib}.
(If you want to specify a different installation
location, use the \vars{--prefix} option in \cmd{configure}.)
While you don't have to have root privileges during the configuration
and check steps, you do during the installation step if you're installing
into \fn{/usr/local} (thus the \cmd{sudo} in the last step).\footnote%
	{Note that when you build netCDF, make sure the build directory
	is not in the directory tree of \vars{--prefix}
	or the default directory \fn{/usr/local}.}

%Because there are two different netCDF installations used in the
%\mods{qtcm} package, the makefiles for creating the benchmarks
%and extensions files will have different \vars{NCLIB} and \vars{NCINC}
%environment variables (see Section~\ref{sec:ubuntu.makefile}).




%------------------------------------------------------------------------
\subsection{Makefile Configuration}  \label{sec:ubuntu.makefile}

	\subsubsection{NetCDF}

In the \fn{src} directory in the \mods{qtcm} distribution, there is a
sub-directory \fn{Makefiles} that contains the makefiles for a
variety of platforms.  Edit the file \fn{makefile.ubuntu\_64\_g95}
so that the lines specifying the environment variables for the
netCDF libraries and include files:

\begin{codeblock}
\codeblockfont{%
NCINC=-I/usr/local/include \\
NCLIB=-L/usr/local/lib -lnetcdf}
\end{codeblock}

are changed to the path where your manually compiled
netCDF libraries and include files are.

Copy \fn{makefile.ubuntu\_64\_g95} from the \fn{Makefiles} sub-directory
in \fn{src} into \fn{src}.  
In other words, from the \mods{qtcm} distribution directory
(i.e., \fn{/buildpath}), at the Unix prompt execute:

\begin{codeblock}
\codeblockfont{%
cp src/Makefiles/makefile.ubuntu\_64\_g95 src/makefile}
\end{codeblock}


	\subsubsection{Linking Order}

Compilers in the GNU Compiler Collection (GCC) search libraries
and object files in the order they are listed in the command-line,
\latexhtml{from left-to-right\footnote%
	{http://gcc.gnu.org/onlinedocs/gcc-4.1.2/gcc/Link-Options.html\#index-l-670}}%
	{\htmladdnormallink{from left-to-right}{http://gcc.gnu.org/onlinedocs/gcc-4.1.2/gcc/Link-Options.html#index-l-670}}.
Thus, if routines in \fn{b.o} call routines in \fn{a.o}, 
you must list the files in the order \fn{a.o b.o}.

For some reason, that isn't the case for \mods{g95}.  Thus, you will
find \mods{g95} makefile rules structured like the following
(below is part of the rule to create an executable (\fn{qtcm}) for
benchmark runs):

% --- Two versions of this rule, one for display in PDF and the other
%     for display in HTML:
%
\begin{latexonly}
\begin{codeblock}
\codeblockfont{%
qtcm: main.o \\
\hspace*{8ex}\$(FC)~-O~\$(NCINC)~-o~\$@ main.o~\$(QTCMLIB)~\$(NCLIB)}
\end{codeblock}
\end{latexonly}

\begin{htmlonly}
\begin{rawhtml}
<p><code><font color="blue">qtcm: main.o<br>
&nbsp;&nbsp;&nbsp;&nbsp;&nbsp;&nbsp;&nbsp;$(FC) -O $(NCINC) -o 
$@ main.o $(QTCMLIB) $(NCLIB)</font></code></p>
\end{rawhtml}
\end{htmlonly}

even though \fn{main.o} depends on the QTCM library 
(specified in macro setting \vars{QTCMLIB}), which in turn
depends on the netCDF library (specified in macro setting \vars{NCLIB}).


	\subsubsection{Shared Object PIC}   \label{sec:sopic}

In order to compile the model in Ubuntu on a 64 bit Intel processor,
the model and the netCDF library it is linked to needs to be
compiled to be 
\latexhtml{position independent code (PIC).\footnote%
		{http://www.gentoo.org/proj/en/base/amd64/howtos/index.xml?part=1\&chap=3}}%
	{\htmladdnormallink{position independent code (PIC)}%
		{http://www.gentoo.org/proj/en/base/amd64/howtos/index.xml?part=1&chap=3}.}
This is accomplished with the 
\htmladdnormallinkfoot{\cmd{-fPIC} flag}%
	{http://www.fortran-2000.com/ArnaudRecipes/sharedlib.html}.

In the \mods{qtcm} makefiles, the \cmd{-fPIC} flag is introduced in the
macro \vars{FFLAGSM}, for instance:
\begin{codeblock}
\codeblockfont{%
FFLAGSM = -O -fPIC}
\end{codeblock}
For makefiles used in creating extension modules, \cmd{-fPIC} must
be passed into the \mods{f2py} call.  To do so, put the flags:
\begin{codeblock}
\codeblockfont{%
--f90flags="-fPIC" --f77flags="-fPIC"}
\end{codeblock}
after the \vars{--fcompiler} flag in the \mods{f2py} calling line.

The \cmd{-fPIC} flag must also be used when compiling the netCDF
libraries, as described in Section~\ref{sec:ubuntu.netcdf}.
Failure to create PIC libraries in 64 bit Ubuntu can result in errors 
like the following when creating the \mods{qtcm} extension modules:
\begin{codeblock}
\codeblockfont{%
ld: /usr/local/lib/libnetcdf.a(fort-attio.o): relocation R\_X86\_64\_32 against `a local symbol' can not be used when making a shared object; recompile with -fPIC /usr/local/lib/libnetcdf.a: could not read symbols: Bad value}
\end{codeblock}




%------------------------------------------------------------------------
\subsection{Summary of Steps}      \label{sec:ubuntu.install.summary}

The following summarizes all the steps I took to install
\mods{qtcm} in
Ubuntu 8.04.1 LTS (Hardy) running on a
Quad Core Intel Xeon (64 bit) machine.
Note that while I use the \mods{aptitude} package manager, you are
free to use any manager of your choice (e.g., \mods{apt-get},
\mods{synaptic}, etc.):

\begin{enumerate}
\item Install the G95 Fortran compiler from the binary distribution.
	See Section~\ref{sec:ubuntu.fort.install} for details.

\item Use an Ubuntu package manager
	to install the following packages, by typing:
	\begin{codeblock}
	\codeblockfont{%
sudo aptitude update \\
sudo aptitude install curl \\
sudo aptitude install python-epydoc \\
sudo aptitude install python-matplotlib \\
sudo aptitude install python-netcdf \\
sudo aptitude install python-scientific \\
sudo aptitude install python-scipy \\
sudo aptitude install texlive}
	\end{codeblock}

	Installing \mods{python-scipy} will also install NumPy and
	\mods{f2py}, so you don't have to install the
	\mods{python-numpy} package separately.

	Early-on as I debugged my \mods{qtcm} install on Ubuntu,
	I encountered errors that I thought came from an 
	\htmladdnormallinkfoot{old version of NumPy}%
		{http://cens.ioc.ee/pipermail/f2py-users/2008-June/001617.html},
	and thus I replaced Ubuntu's packaged NumPy with NumPy 1.1.0
	built 
	\latexhtml{directly from source.\footnote%
			{http://sourceforge.net/project/showfiles.php?group\_id=1369\&package\_id=175103}}%
		{\htmladdnormallink{directly from source}{http://sourceforge.net/project/showfiles.php?group_id=1369&package_id=175103}.}
	(Note, you shouldn't install your new NumPy in the default
	location, which may cause problems later-on with Ubuntu's
	package manager.)
	Later on, I concluded the errors I had encountered were not
	because of the NumPy version, but by then I didn't want to
	try to reinstall NumPy again.
	So strictly speaking, the version of Numpy I used is not
	the one bundled with \mods{python-scipy}, but that shouldn't
	be a problem.

\item Manually install netCDF 3.6.2 from source
	(see Section \ref{sec:ubuntu.netcdf}).

\item Manually install the \mods{basemap} package of
	\mods{matplotlib}.  
	The source for the \mods{basemap} toolkit is
	available 
	\latexhtml{from Sourceforge\footnote%
			{http://sourceforge.net/project/showfiles.php?group\_id=80706}}%
		{\htmladdnormallink{from Sourceforge}%
			{http://sourceforge.net/project/showfiles.php?group_id=80706}}
	I obtained version 0.9.9.1 using the
	following \cmd{curl} command:
	\begin{codeblock}
	\codeblockfont{%
\scriptsize
curl -o basemap.tar.gz $\backslash$ \\
http://voxel.dl.sourceforge.net/sourceforge/matplotlib/basemap-0.9.9.1.tar.gz}
	\end{codeblock}

	The \fn{README} file in the \fn{basemap-0.9.9.1} directory has
	detailed installation instructions.  Note that you have to
	install the GEOS library first (\fn{README} has detailed
	directions on how to do that too).  To be on the safe-side,
	I would set the \vars{FC} environment variable to the G95
	compiler
	(e.g., with \cmd{export FC=g95} in Bash).

\item From this point on, you can follow the
	general instructions given in Section~\ref{sec:install.sum},
	starting with step~\ref{list:download.qtcm.sum}.
	Please do not ignore, however, Section~\ref{sec:install.ubuntu}'s
	Ubuntu-specific details.

\end{enumerate}



% ===== end of file =====


\chapter{Getting Started With \mods{qtcm}}  \label{ch:getting.started}
% ==========================================================================
% Getting Started With qtcm
%
% By Johnny Lin
% ==========================================================================


% ------ BODY -----
%
%---------------------------------------------------------------------
\section{Your First Model Run}

Figure~\ref{fig:my.first.run} shows an example of a script to make
a 30 day seasonal, aquaplanet model run, with run name ``test'',
starting from November 1, Year 1.


%--- Two versions, one for PDF, one for HTML:
\begin{latexonly}
\begin{figure}[htp]
\begin{center}
\begin{codeblock}
\codeblockfont{%
from qtcm import Qtcm \\
inputs = \{\} \\
inputs['runname'] = 'test' \\
inputs['landon'] = 0 \\
inputs['year0'] = 1 \\
inputs['month0'] = 11 \\
inputs['day0'] = 1 \\
inputs['lastday'] = 30 \\
inputs['mrestart'] = 0 \\
inputs['compiled\_form'] = 'parts' \\
model = Qtcm(**inputs) \\
model.run\_session()}
\end{codeblock}
\end{center}
\caption{An example of a simple \mods{qtcm} run.}
\label{fig:my.first.run}
\end{figure}
\end{latexonly}

\begin{htmlonly}
\label{fig:my.first.run}
\begin{center}
\htmlfigcaption{%
	\codeblockfont{%
from qtcm import Qtcm \\
inputs = \{\} \\
inputs['runname'] = 'test' \\
inputs['landon'] = 0 \\
inputs['year0'] = 1 \\
inputs['month0'] = 11 \\
inputs['day0'] = 1 \\
inputs['lastday'] = 30 \\
inputs['mrestart'] = 0 \\
inputs['compiled\_form'] = 'parts' \\
model = Qtcm(**inputs) \\
model.run\_session()}
	}

\htmlfigcaption{Figure~\ref{fig:my.first.run}:
	An example of a simple \mods{qtcm} run.}
\end{center}
\end{htmlonly}



The class describing the QTCM1 model is \class{Qtcm}.  An instance
of \class{Qtcm}, in this example \vars{model}, is created the same
way you create an instance of any class.  When instantiating an
instance of \class{Qtcm}, keyword parameters can be used to override
any default settings.  In the example above, the dictionary
\vars{inputs} specifying all keyword parameters is passed in on the
instantiation of \vars{model}.

The keyword parameter settings in
Figure~\ref{fig:my.first.run} have the following meanings:
\begin{itemize}
\item \vars{runname}:  This string (``test'') is used in the
	output filename.  QTCM1 writes mean and instantaneous
	output files to the directory given in \vars{model.outdir.value},
	with filenames 
	\fn{qm\_}\dumarg{runname}\fn{.nc} for mean output and
	\fn{qi\_}\dumarg{runname}\fn{.nc} for instantaneous output.

\item \vars{landon}: When set to ``0'', the land is turned off and
	the run is an aquaplanet run.  When set to ``1'', the land
	model is turned on.

\item \vars{year0}:  The year the run starts on.

\item \vars{month0}:  The month the run starts on (11 = November).

\item \vars{day0}: The day of the month the run starts on.

\item \vars{lastday}:  The model runs from day 1 to \vars{lastday}.

\item \vars{mrestart}:  When set to ``0'', the run starts from
	default initial conditions
	(see Section~\ref{sec:initial.variables} for a table of
	those values).
	When set to ``1'', the run starts from a restart file.

\item \vars{compiled\_form}:  This keyword sets what form the
	compiled QTCM1 model has, and its value is saved to
	the instance's \vars{compiled\_form} attribute.
	It is a string and can be set either to
	``parts'' or ``full''.  Most of the time, you will want
	to set it to \vars{'parts'}.
	This keyword is the only one
	that must be specified on instantiation; the model instance
	will at least instantiate
	using only the default settings for all the other keyword
	parameters (given in Appendix~\ref{app:defaults.values}).
	See Section~\ref{sec:compiledform} for details about
	what the \vars{compiled\_form} attribute controls.
\end{itemize}

By default, the \vars{SSTmode} attribute, which controls whether the
model will use climatological sea-surface temperatures (SST) 
or real SSTs, is set to the \vars{value} ``seasonal'', thus giving a
run with seasonal forcing at the lower-boundary over the ocean.

This example assumes that the boundary condition files, sea surface
temperature files, and the model output directories are as specified
in submodule \mods{defaults}.  Those values are described in
Section~\ref{sec:defaults.scalar}.




%---------------------------------------------------------------------
\section{Managing Directories}

Most of the time, your boundary condition files and output files
will not be in the locations specified in
Section~\ref{sec:defaults.scalar}, or in the directory your
Python script resides.  The easiest way to tell your \class{Qtcm} 
instance where your input/output files are is to pass them in
as keyword parameters on instantiation.


%--- Two versions, one for PDF, one for HTML:
\begin{latexonly}
\begin{figure}[htp]
\begin{codeblock}
\codeblockfont{%
\small
from qtcm import Qtcm \\
rundirname = 'test' \\
dirbasepath = os.path.join(os.getcwd(), rundirname) \\
inputs = \{\} \\
inputs['bnddir'] = os.path.join( os.getcwd(), 'bnddir', \\
\hspace*{40ex}'r64x42' ) \\
inputs['SSTdir'] = os.path.join( os.getcwd(), 'bnddir', \\
\hspace*{40ex}'r64x42', 'SST\_Reynolds' ) \\
inputs['outdir'] = dirbasepath \\
inputs['runname'] = rundirname \\
inputs['landon'] = 0 \\
inputs['year0'] = 1 \\
inputs['month0'] = 11 \\
inputs['day0'] = 1 \\
inputs['lastday'] = 30 \\
inputs['mrestart'] = 0 \\
inputs['compiled\_form'] = 'parts' \\
model = Qtcm(**inputs) \\
model.run\_session()}
\end{codeblock}

\caption{An example \mods{qtcm} run showing detailed description of
	input and output directories.}
\label{fig:manage.dir.example}
\end{figure}
\end{latexonly}

\begin{htmlonly}
\label{fig:manage.dir.example}
\begin{center}
\htmlfigcaption{%
	\codeblockfont{%
from qtcm import Qtcm \\
rundirname = 'test' \\
dirbasepath = os.path.join(os.getcwd(), rundirname) \\
inputs = \{\} \\
inputs['bnddir'] = os.path.join( os.getcwd(), 'bnddir', \\
\hspace*{40ex}'r64x42' ) \\
inputs['SSTdir'] = os.path.join( os.getcwd(), 'bnddir', \\
\hspace*{40ex}'r64x42', 'SST\_Reynolds' ) \\
inputs['outdir'] = dirbasepath \\
inputs['runname'] = rundirname \\
inputs['landon'] = 0 \\
inputs['year0'] = 1 \\
inputs['month0'] = 11 \\
inputs['day0'] = 1 \\
inputs['lastday'] = 30 \\
inputs['mrestart'] = 0 \\
inputs['compiled\_form'] = 'parts' \\
model = Qtcm(**inputs) \\
model.run\_session()}
	}

\htmlfigcaption{Figure~\ref{fig:manage.dir.example}:
	An example \mods{qtcm} run showing detailed description of
        input and output directories.}
\end{center}
\end{htmlonly}


Figure~\ref{fig:manage.dir.example} shows an example run where those
directories are explicitly specified; in all other aspects, the run
is identical to the one in Figure~\ref{fig:my.first.run}.
In Figure~\ref{fig:manage.dir.example}, output from the model is
directed to the directory described by string variable
\vars{dirbasepath}.  \vars{dirbasepath} is created by joining the
current working directory with the run name given in string variable
\vars{rundirname}.\footnote%
	{The Python \mods{os} module enables platform-independent
	handling of files and directories.  The \mods{os.path.join}
	function resolves paths without the programmer needing to know
	all the possible directory separation characters; the function
	chooses the correct separation character at runtime.  The
	\mods{os.getcwd} function returns the current working directory.}
Setting keyword parameter \vars{outdir} to \vars{dirbasepath} sends
output to \vars{dirbasepath}.  
Keywords \vars{bnddir} and \vars{SSTdir} specify the directories
where non-SST and SST boundary condition files, respectively, are
found.

Interestingly, the default version of QTCM1 does \emph{not} send
all output from the model to \vars{outdir}.  The restart file
\fn{qtcm\_}\dumarg{yyyymmdd}\fn{.restart} (where \dumarg{yyyymmdd}
is the year, month, and day of the model date when the restart
file was written) is written into the current working directory,
not the output directory.  Thus, if you do multiple runs, you'll
have to manually deal with the restart files that will proliferate.

Neither the QTCM1 model nor the \class{Qtcm} object
create the directories specified in \mods{bnddir}, \mods{SSTdir},
and \mods{outdir}.  Failure to do so will create an error.  I use
Python's file management tools to make sure the output directory
is created, and any old output files are deleted.  Here's an example
that does that, using the \vars{dirbasepath} and \vars{rundirname}
variables from Figure~\ref{fig:manage.dir.example}:

\begin{codeblock}
\codeblockfont{%
\small
if not os.path.exists(dirbasepath):  os.makedirs(dirbasepath) \\
qi\_file = os.path.join( dirbasepath, 'qi\_'+rundirname+'.nc' ) \\
qm\_file = os.path.join( dirbasepath, 'qm\_'+rundirname+'.nc' ) \\
if os.path.exists(qi\_file):   os.remove(qi\_file) \\
if os.path.exists(qm\_file):   os.remove(qm\_file)}
\end{codeblock}




%---------------------------------------------------------------------
\section{Model Field Variables}   \label{sec:field.variables.intro}

The term ``field'' variable refers to QTCM1 model variables that 
are accessible at both the compiled Fortran QTCM1 model-level as
well as the Python \class{Qtcm} instance-level.
Field variables are all instances of the \class{Field} class,
and are stored as attributes of the \class{Qtcm} instance.\footnote%
	{Note non-field variables can also be instances of \class{Field},
	and that \class{Qtcm} instances have other attributes that are
	not equal to \class{Field} instances.}

\class{Field} class instances have the following attributes:
\begin{itemize}
\item \vars{id}:  A string naming the field (e.g., ``Qc'', ``mrestart'').
	This string should contain no whitespace.
\item \vars{value}:  The value of the field.  Can be of any type, though
	typically is either a string or numeric scalar or a numeric array.
\item \vars{units}:  A string giving the units of the field.
\item \vars{long\_name}:  A string giving a description of the field.
\end{itemize}

\class{Field} instances also have methods to return the rank 
and typecode of \vars{value}.

Remember, if you want to access the value of a \class{Field} object,
make sure you access that object's \vars{value} attribute.  
Thus, for example,
to assign a variable \vars{foo} to the
\vars{lastday} value for a given
\class{Qtcm} instance \vars{model}, type the following:
\begin{codeblock}
\codeblockfont{%
foo = model.lastday.value}
\end{codeblock}

For scalars, this assignment sets \vars{foo} by value (i.e., a copy
of the value of attribute \vars{model.lastday} is set to \vars{foo}).
In general, however, Python assigns variables by reference.  Use
the \mods{copy} module if you truly want a copy of a field variable's
value (such as an array), rather than an alias.  For more details
about field variables, see Section~\ref{sec:field.variables}.




%---------------------------------------------------------------------
\section{Run Sessions}

	\subsection{What is a Run Session?}

A run session is a unit of simulation where the model is run from
day 1 of simulation to the day specified by the \vars{lastday}
attribute of a \class{Qtcm} instance.  A run session is a
``complete'' model run, at the beginning of which all compiled QTCM1
model variables are set to the values given at the Python-level,
and at the end of which restart files are written, the values
at the Python-level are overwritten by the values in the Fortran
model, and a Python-accessible snapshot is taken of the 
model variables that were written to the restart file.


	\subsection{Changing Variables}

Between run sessions, changing any field variable is as easy
as a Python assignment.  For instance, to change the atmosphere
mixed layer depth to 100~m, just type:
\begin{codeblock}
\codeblockfont{%
model.ziml.value = 100.0}
\end{codeblock}

When changing arrays, be careful to try to match the shape of the 
array.\footnote%
	{At the very least, match the rank of the array, which is required
	for the routines in \mods{setbypy} to properly choose which
	Fortran subroutine to use in reading the Python value.
	I haven't tested if only the rank is needed, however,
	for the passing to work, for a continuation run (my hunch is
	it won't).}
You can use the NumPy \mods{shape} function on a NumPy array to
check its shape.


	\subsection{Continuing a Model Run}  \label{sec:continuation.intro}

Figure~\ref{fig:continuation.example} shows an example of two run
sessions, where the second run session is a continuation of the
first.


%--- Two versions, one for PDF, one for HTML:
\begin{latexonly}
\begin{figure}[htp]
\begin{codeblock}
\codeblockfont{%
\small
inputs['year0'] = 1 \\
inputs['month0'] = 11 \\
inputs['day0'] = 1 \\
inputs['lastday'] = 10 \\
inputs['mrestart'] = 0 \\
inputs['compiled\_form'] = 'parts' \\ \\
model = Qtcm(**inputs) \\
model.run\_session() \\
model.u1.value = model.u1.value * 2.0 \\
model.init\_with\_instance\_state = True \\
model.run\_session(cont=30)}
\end{codeblock}

\caption{An example of two \mods{qtcm} run sessions where the second
	run session is a continuation of the first.  Assume 
	\vars{inputs} is a dictionary, and that earlier in the
	script the run name and
	all input and output directory names were added
	to the dictionary.}
\label{fig:continuation.example}
\end{figure}
\end{latexonly}

\begin{htmlonly}
\label{fig:continuation.example}
\begin{center}
\htmlfigcaption{%
	\codeblockfont{%
inputs['year0'] = 1 \\
inputs['month0'] = 11 \\
inputs['day0'] = 1 \\
inputs['lastday'] = 10 \\
inputs['mrestart'] = 0 \\
inputs['compiled\_form'] = 'parts' \\ \\
model = Qtcm(**inputs) \\
model.run\_session() \\
model.u1.value = model.u1.value * 2.0 \\
model.init\_with\_instance\_state = True \\
model.run\_session(cont=30)}
	}

\htmlfigcaption{Figure~\ref{fig:continuation.example}:
	An example of two \mods{qtcm} run sessions where the second
	run session is a continuation of the first.  Assume 
	\vars{inputs} is a dictionary, and that earlier in the
	script the run name and
	all input and output directory names were added
	to the dictionary.}
\end{center}
\end{htmlonly}


The first run session runs from day 1 to day 10.  The second
run session runs the model for another 30 days.  
Setting the \vars{init\_with\_instance\_state} of
\vars{model} to \vars{True} tells the model to use the
the values of the instance attributes 
(for prognostic variables, right-hand sides, and start date) 
are currently stored \vars{model}
as the initial values for the run\_session.\footnote%
	{Unless overridden, by default, 
	\vars{init\_with\_instance\_state} is set
	to True on \class{Qtcm} instance instantiation.}
The \vars{cont}
keyword in the second \mods{run\_session} call specifies a
continuation run, and the value gives the number of additional
days to run the model.

The set of runs described above would produce the exact same
results as if you had gone into the Fortran model after 10 days,
doubled the first baroclinic mode zonal velocity, and continued
the run for another 30 days.  With the Python example above, however,
you didn't need to know you were going to do that ahead of starting
the model run (which is what a compiled model requires you to do).
Section~\ref{sec:contination.run.sessions} describes continuation
runs in detail.


	\subsection{Passing Restart Snapshots Between Run Sessions}
					\label{sec:snapshot.intro}

The pure-Fortran QTCM1 uses a restart file to enable continuation
runs.  A \class{Qtcm} instance can also make use of that option,
through setting the \vars{mrestart} attribute value
(see Section~\ref{sec:contination.run.sessions} and
Neelin et al.\ \cite{Neelin/etal:2002} for details).  
It's easier, however, instead of using a restart file, to pass 
along a ``snapshot'' dictionary.

The \class{Qtcm} instance method \mods{make\_snapshot} copies the
variables that would be written out to a restart file into a
dictionary that is saves as the instance attribute \vars{snapshot}.
This snapshot can be saved separately, for later recall.  Note that
snapshots are automatically made at the end of a run session.

The following example shows a model \mods{run\_session} call,
following which the snapshot is saved to the variable
\vars{snapshot}:\footnote%
	{Remember Python assignment defaults to assignment by
	reference, so in this example the variable \vars{mysnapshot}
	is a pointer to the \vars{model.snapshot} attribute.
	(However, note that \vars{model.snapshot} itself is not a
	reference, but a distinct copy of those variables; to do
	otherwise would result in a non-static snapshot.)
	If the \vars{model.snapshot} attribute is dereferenced,
	then \vars{mysnapshot} will become the sole pointer to the
	dictionary.}

\begin{codeblock}
\codeblockfont{%
model.run\_session() \\
mysnapshot = model.snapshot}
\end{codeblock}

After taking the snapshot, you might continue the run a while, and
then decide to return to the snapshot you saved.  To do so, use
the \mods{sync\_set\_py\_values\_to\_snapshot}
method to reset the model instance values to
\vars{mysnapshot} before your next run session:
\begin{codeblock}
\codeblockfont{%
model.sync\_set\_py\_values\_to\_snapshot(snapshot=mysnapshot) \\
model.init\_with\_instance\_state = True \\
model.run\_session()}
\end{codeblock}

See Section~\ref{sec:snapshots} for details regarding the use of
snapshots, as well as for a list of what variables are saved in
a snapshot.




%---------------------------------------------------------------------
\section{Creating Multiple Models}

	\subsection{Model Instances}

Creating a new QTCM1 model is as simple as creating another
\class{Qtcm} instance.
For instance, to instantiate two QTCM1
models, \vars{model1} and \vars{model2}, type the following:

\begin{codeblock}
\codeblockfont{%
from qtcm import Qtcm \\
model1 = Qtcm(compiled\_form='parts') \\
model2 = Qtcm(compiled\_form='parts')}
\end{codeblock}

\vars{model1} and \vars{model2} do \emph{not} share any variables
in common, including the extension modules holding the Fortran
code.  In creating the instances, a copy of the extension modules
are saved in temporary directories.


	\subsection{Passing Snapshots To Other Models}

The snapshots described in Section~\ref{sec:snapshot.intro}
can also be passed around to other model instances,
enabling you to easily branch a model run:

\begin{codeblock}
\codeblockfont{%
model.run\_session() \\
mysnapshot = model.snapshot \\
model1.sync\_set\_py\_values\_to\_snapshot(snapshot=mysnapshot) \\
model2.sync\_set\_py\_values\_to\_snapshot(snapshot=mysnapshot) \\
model1.run\_session() \\
model2.run\_session()}
\end{codeblock}

The state of \vars{model} after its run session is used to start
\vars{model1} and \vars{model2}.  This is an easy way to save time
in spinning-up multiple models.




%---------------------------------------------------------------------
\section{Run Lists}		\label{sec:runlist.intro}

This feature of \class{Qtcm} objects is what really gives 
\class{Qtcm} model instances their flexibility.
A run list is a list of strings and dictionaries that specify
what routines to run in order to execute a particular part of
the model.  Each element of the run list specifies the method
or subroutine to execute, and the order of the elements specifies
their execution order.

For instance, the standard run list for initializing the the
atmospheric portion of the model is named ``qtcminit'', and
equals the following list:

\begin{latexonly}
\begin{codeblock}
\codeblockfont{%
\parbox{46ex}{% This file is automatically generated by
% code_to_latex.py.  It lists the code of the qtcminit
% runlist.

['\_\_qtcm.wrapcall.wparinit', \\
 '\_\_qtcm.wrapcall.wbndinit', \\
 'varinit', \\
 \{'\_\_qtcm.wrapcall.wtimemanager': [1]\}, \\
 'atm\_physics1']}}
\end{codeblock}
\end{latexonly}

\begin{htmlonly}
\begin{quotation}
% This file is automatically generated by
% code_to_latex.py.  It lists the code of the qtcminit
% runlist.

['\_\_qtcm.wrapcall.wparinit', \\
 '\_\_qtcm.wrapcall.wbndinit', \\
 'varinit', \\
 \{'\_\_qtcm.wrapcall.wtimemanager': [1]\}, \\
 'atm\_physics1']
\end{quotation}
\end{htmlonly}

This list is stored as an entry in the \vars{runlists} dictionary
(with key \vars{'qtcminit'}).
\vars{runlists} is an attribute of a \class{Qtcm} instance.
Table~\ref{tab:stnd.runlists} lists all standard run lists.

When the run list element in the list is a string, the string gives the
name of the routine to execute.  The routine has no parameter
list.  The routine can be a
compiled QTCM1 model subroutine for which an interface has been
written (e.g., \mods{\_\_qtcm.wrapcall.wparinit}), 
a method of the of the Python model instance 
(e.g., \mods{varinit}), or another run list
(e.g., \vars{atm\_physics1}).

When the run list element is a 1-element dictionary, the key of
the dictionary element is the name of the routine, and the value
of the dictionary element is a list specifying input parameters
to be passed to the routine on call.  Thus, the element:
\begin{codeblock}
\codeblockfont{%
{\{'\_\_qtcm.wrapcall.wtimemanager': [1]\}}}
\end{codeblock}
calls the \mods{\_\_qtcm.wrapcall.wtimemanager} routine, passing in
one input parameter, which in this case is the value 1.

If you want to change the order of the run list, just change the
order of the list.  To add or remove routines to be executed, just
add and remove their names from the run list.
Python provides a number of methods to manipulate
lists (e.g., \mods{append}).  Since lists are dynamic data types
in Python, you do not have to do any recompiling to implement
the change.

The \vars{compiled\_form} attribute must be set to \vars{'parts'}
in the \class{Qtcm} instance in order to take advantage of the run
lists feature of the class.  Run lists are not available for
\vars{compiled\_form\thinspace=\thinspace'full'}, because subroutine
calls are hardwired in the compiled QTCM1 model Fortran code in
that case.




%---------------------------------------------------------------------
\section{Model Output}			\label{sec:output.intro}

	\subsection{NetCDF Output}

Model output is written to netCDF files in the directory
specified by the \class{Qtcm} instance attribute \vars{outdir}.
Mean values are written to an output file beginning with
\fn{qm\_}, and instantaneous values are written to an output
file beginning with \fn{qi\_}.

The frequency of mean output is controlled by \vars{ntout}, and the
frequency of instantaneous output is controlled by \vars{ntouti}.
\vars{ntout.value} gives the number of days over which to average
(and if equals \vars{-30}, monthly means are calculated).
\vars{ntouti.value} gives the frequency in days that instantaneous
values are output (monthly if it equals \vars{-30}).  (See
Section~\ref{sec:initial.variables} for a description of other
output-control variables, and see the QTCM1 manual \cite{Neelin/etal:2002}
for a detailed description of how these variables control output.)

Figure~\ref{fig:netcdf.read} gives an example of a block of code
to read netCDF output, where \vars{datafn} is the netCDF filename, and
\vars{id} is the string name of the field variable (e.g.,
\vars{'u1'}, \vars{'T1'}, etc.).
(Note that the netCDF identifier for field variables is the same as
the name in \class{Qtcm}, except for the variables given in
Table~\ref{tab:qtcm.netcdf.ids}.)

In the code in Figure~\ref{fig:netcdf.read},
the array value is read into \vars{data}, and the longitude values, 
latitude values, and time values are read into variables
\vars{lon}, \vars{lat}, and \vars{time}, respectively.
As netCDF files also hold metadata, a description and the units
of the variable given by \vars{id}, and each dimension, are read
into variables ending in \vars{\_name} and \vars{\_units},
respectively.


%--- Two versions, one for PDF, one for HTML:
\begin{latexonly}
\begin{figure}[htp]
\begin{codeblock}
\codeblockfont{%
import numpy as N \\
import Scientific as S \\ \\
fileobj = S.NetCDFFile(datafn, mode='r') \\ \\
data = N.array(fileobj.variables[id].getValue()) \\
data\_name = fileobj.variables[id].long\_name \\
data\_units = fileobj.variables[id].units \\ \\
lat = N.array(fileobj.variables['lat'].getValue()) \\
lat\_name = fileobj.variables['lat'].long\_name \\
lat\_units = fileobj.variables['lat'].units \\ \\
lon = N.array(fileobj.variables['lon'].getValue()) \\
lon\_name = fileobj.variables['lon'].long\_name \\
lon\_units = fileobj.variables['lon'].units \\ \\
time = N.array(fileobj.variables['time'].getValue()) \\
time\_name = fileobj.variables['time'].long\_name \\
time\_units = fileobj.variables['time'].units \\ \\
fileobj.close()}
\end{codeblock}

\caption{Example of Python code to read netCDF output.
	See text for description.}
\label{fig:netcdf.read}
\end{figure}
\end{latexonly}

\begin{htmlonly}
\label{fig:netcdf.read}
\begin{center}
\htmlfigcaption{%
	\codeblockfont{%
import numpy as N \\
import Scientific as S \\ \\
fileobj = S.NetCDFFile(datafn, mode='r') \\ \\
data = N.array(fileobj.variables[id].getValue()) \\
data\_name = fileobj.variables[id].long\_name \\
data\_units = fileobj.variables[id].units \\ \\
lat = N.array(fileobj.variables['lat'].getValue()) \\
lat\_name = fileobj.variables['lat'].long\_name \\
lat\_units = fileobj.variables['lat'].units \\ \\
lon = N.array(fileobj.variables['lon'].getValue()) \\
lon\_name = fileobj.variables['lon'].long\_name \\
lon\_units = fileobj.variables['lon'].units \\ \\
time = N.array(fileobj.variables['time'].getValue()) \\
time\_name = fileobj.variables['time'].long\_name \\
time\_units = fileobj.variables['time'].units \\ \\
fileobj.close()}
	}

\htmlfigcaption{Figure~\ref{fig:netcdf.read}:
	Example of Python code to read netCDF output.
	See text for description.}
\end{center}
\end{htmlonly}





\begin{table}[tp]
\begin{center}
\begin{tabular}{l|l}
\textbf{\class{Qtcm} Attribute Name} & \textbf{NetCDF Output Name} \\
\hline
\vars{'Qc'}                & \vars{'Prec'} \\
\vars{'FLWut'}             & \vars{'OLR'} \\
\vars{'STYPE'}             & \vars{'stype'}
\end{tabular}
\end{center}
\caption{NetCDF output names for \class{Qtcm} field variables that
	are different from the \class{Qtcm} and compiled QTCM1 model
	variable names.  The netCDF names are case-sensitive.}
\label{tab:qtcm.netcdf.ids}
\end{table}


\emphpara{NB:}  All netCDF array output is dimensioned (time, latitude,
longitude) when read into Python using the \mods{Scientific} package.
This differs from the way \class{Qtcm} saves field variables, which
follows Fortran convention (longitude, latitude).  Please be careful
when relating the two types of arrays.
Section~\ref{sec:field.var.shape} for a discussion of why there is
this discrepancy.


	\subsection{Visualization}	\label{sec:viz.intro}

The \mods{plotm} method of \class{Qtcm} instances creates line
plots or contour plots, as appropriate, of model output of
average fields of run session(s) associated with the instance.
Some examples, assuming \vars{model} is an instance of \class{Qtcm}
and has already executed a run session:
\begin{itemize}
\item \cmd{model.plotm('Qc', lat=1.875)}:
	A time vs.\ longitude contour
          plot is made for the full range of time and longitude,
          at the latitude 1.875 deg N, for mean precipitation.

\item \cmd{model.plotm('Qc', time=10)}:
	A latitude vs.\ longitude contour plot of precipitation
	is made for the full spatial domain at day 10 of the model run.

\item \cmd{model.plotm('Evap', lat=1.875, lon=[100,200])}:  A contour
	plot of time vs.\ longitude of evaporation is made for the
          longitude points between 100 and 200 degrees E, at the
          latitude 1.875 deg N.  

\item \cmd{model.plotm('cl1', lat=1.875, lon=[100,200], time=20)}:
          A deep cloud amount vs.\ longitude line plot is made for
          the longitude points between 100 and 200 degrees east,
          at the latitude 1.875 deg N, at day 20 of the model run.
\end{itemize}

In these examples, the number of days over which the mean is taken
equals \vars{model.ntout.value}.
Also, the \mods{plotm} method automatically takes into account the
\class{Qtcm}/netCDF variable differences described in
Table~\ref{tab:qtcm.netcdf.ids}.



%---------------------------------------------------------------------
\section{Documentation}

Section~\ref{sec:ver} gives the online locations of the
transparent copies of this manual.  
Model formulation is fully described in
Neelin \& Zeng \cite{Neelin/Zeng:2000} and model
results are described in Zeng et~al.\ \cite{Zeng/etal:2000}
(\cite{Neelin/Zeng:2000} is based upon v2.0 of QTCM1
and \cite{Zeng/etal:2000} is based on QTCM1 v2.1).
Additional documentation you'll find useful include:

\begin{itemize}
\item \latexhtml{%
\htmladdnormallinkfoot{The \mods{qtcm} Package API Documentation}%
        {http://www.johnny-lin.com/py\_pkgs/qtcm/doc/html-api/}}%
{\htmladdnormallink{The \mods{qtcm} Package API Documentation}%
        {http://www.johnny-lin.com/py_pkgs/qtcm/doc/html-api/}}

\item \latexhtml{%
\htmladdnormallinkfoot{The Pure-Fortran QTCM1 Manual}%
        {http://www.atmos.ucla.edu/$\sim$csi/qtcm\_man/v2.3/qtcm\_manv2.3.pdf}}%
{\htmladdnormallink{The Pure-Fortran QTCM1 Manual}%
        {http://www.atmos.ucla.edu/~csi/qtcm_man/v2.3/qtcm_manv2.3.pdf}}
\cite{Neelin/etal:2002}

\end{itemize}



% ===== end of file =====


\chapter{Using \mods{qtcm}}                 \label{ch:using}
% ==========================================================================
% Using QTCM
%
% By Johnny Lin
% ==========================================================================


% ------ BODY -----
%
%---------------------------------------------------------------------
\section{Introduction}

Now that you've successfully run your first model instances, in
this chapter I provide detailed explanations regarding the features
of \mods{qtcm}.  I present these explanations in a documentary
rather than didactic fashion; my goal is to document how the features
work.  More details are given in the code docstrings.  At the end
of the chapter, in Section~\ref{sec:cookbook}, I provide a few
cookbook ideas/examples of ways to use the model.




%---------------------------------------------------------------------
\section{Model Instances}  \label{sec:model.instances}

An instance of a \class{Qtcm} model is created in \mods{qtcm} the same way
you create an instance of any class.
For instance, to instantiate two \class{Qtcm}
models, \vars{model1} and \vars{model2}, I type the following:

\begin{codeblock}
\codeblockfont{%
from qtcm import Qtcm \\
model1 = Qtcm(compiled\_form\thinspace=\thinspace'full') \\
model2 = Qtcm(compiled\_form\thinspace=\thinspace'parts')}
\end{codeblock}

In the above example, \vars{model1} uses the compiled QTCM1 model
that runs the model (essentially) using the Fortran driver,
while \vars{model2} uses the compiled QTCM1 model where execution
order and content all the way down to the atmospheric timestep level
is controlled by Python run lists.  (Section~\ref{sec:compiledform}
has more details about the difference between compiled forms.)

For each instance of \class{Qtcm}, copies of all needed extension
modules (e.g., \fn{.so} files) are copied to a temporary directory
that is automatically created by Python.  The full path name of
that directory is saved in the instance attribute \vars{sodir}.
These extension modules are then associated with the specific instance 
through private instance attributes,
and thus every instance of \class{Qtcm} has its own separate variable
and name space on both the Fortran and Python sides.\footnote%
	{The private instance attribute is \vars{\_\_qtcm}.
	See Section~\ref{sec:Qtcm.private.attrib} for details about 
	private \class{Qtcm} instance attributes.}
The temporary directory and all of its contents are deleted when the 
model instance is deleted.

On instantiation, \class{Qtcm} instances set all scalar field
variables to their default values as given in the submodule
\mods{defaults} (and listed in Section~\ref{sec:defaults.scalar}),
and assign the fields as instance attributes.  The instance attribute
\vars{init\_with\_instance\_state} is set to True by default, unless
overridden on instantiation.




%---------------------------------------------------------------------
\section{Initializing a Model Run}

In the pure-Fortran QTCM1, there are three broad
classes of initialized variables:
\begin{enumerate}
\item Those that are read-in using a namelist, 
\item Those that the are read-in from a restart file, and
\item Those that are set by assignment in the Fortran code.  
\end{enumerate}
These variables are a combination of scalars and arrays.

For \mods{qtcm}, interfaces were built so that all classes of
initialized variables that could be user-controlled are accessible
and changeable at the Python-level.  For \mods{qtcm},
the set of variables that could be changed is also expanded, to
include not just the first and second classes of pure-Fortran
QTCM1 initialized variables.  This was done to make \mods{qtcm}
more flexible.  All variables that can be passed between the
compiled QTCM1 model and Python model levels are called
field variables, and are described in detail in
Section~\ref{sec:field.variables}.

As it happens, all the namelist-set variables are scalars.  In the
pure-Fortran QTCM1, those variables are given default values prior
to reading in of the namelist.  To duplicate this functionality,
on model instantiation, all scalar fields are set to their default
values as given in the submodule \mods{defaults} and listed in
Section~\ref{sec:defaults.scalar}.  Most of the default values in
\mods{defaults} are the same as in the pure-Fortran QTCM1, but
there are a few differences.\footnote%
	{One difference being \vars{mrestart}, which in \vars{qtcm} 
	will have the value of 0 in contrast to the pure-Fortran 
	QTCM1 where the default is the 1.}
This setting of scalar defaults is the same for both
\vars{compiled\_form\thinspace=\thinspace'full'} and
\vars{compiled\_form\thinspace=\thinspace'parts'} instances.
Of course, all
\mods{qtcm} fields are user-controllable, both via keyword input
parameters at model instantiation as well as through direct
manipulation of the instance attribute that stores the field variable.

The pure-Fortran QTCM1 initialized prognostic variables and
right-hand sides are set in the Fortran subroutine \mods{varinit}.
Their they are read-in from a restart file or, as default,
set by assignment.
In \mods{qtcm}, the same variables are initialized by a \class{Qtcm}
instance method of the same name, \mods{varinit}, for the case when
\vars{compiled\_form\thinspace=\thinspace'parts'}.  For the case
of \vars{compiled\_form\thinspace=\thinspace'full'}, the compiled
QTCM1 subroutine that is the same as in the pure-Fortran case is
used, and that routine is inaccessible at the Python level.
See Section~\ref{sec:snapshots}'s listing of snapshot variables,
which also includes the prognostic variables and right-hand sides that
are set in \mods{varinit} (both Fortran and Python).




%---------------------------------------------------------------------
\section{The \vars{compiled\_form} Keyword}  \label{sec:compiledform}

The \mods{qtcm} package is a Python wrap of the Fortran routines
that make up QTCM1.  The wrapping layer adds flexibility and
functionality, but at the cost of speed.  Thus, I created two
types of extension modules from the Fortran QTCM1 code, one
which permits very little control over the compiled Fortran
\emph{routines} at the Python level, and one that allows the Python-level
to control model execution in the compiled QTCM1 model
all the way down to the atmospheric timestep level.\footnote%
	{That control is via run lists, which are described in
	Section~\ref{sec:runlists}.}
The former extension module corresponds to 
\vars{compiled\_form\thinspace=\thinspace'full'} and
the latter extension module to
\vars{compiled\_form\thinspace=\thinspace'parts'}.

For \vars{compiled\_form\thinspace=\thinspace'full'},
the compiled portion of the model encompasses (nearly) the
entire QTCM1 model as a whole.  Thus, the only compiled QTCM1 model
modules or subroutines that Python should interact with is
the \mods{driver} routine (which executes the entire model) and
the \mods{setbypy} module (which enables communication between the
compiled model and the Python-level of model fields.\footnote%
	{The \mods{setbypy} Python module is the wrap of the
	Fortran QTCM1 \mods{SetByPy} module.}

For \vars{compiled\_form\thinspace=\thinspace'parts'}, the compiled
portion of the model does not encompasses the model as a whole, but
rather is broken up into separate units (as appropriate) all the
way down to an atmosphere timestep.  Thus, compiled QTCM1 model
modules/subroutines that are accessible at the Python-level include
those that are executed within an atmosphere timestep on up.

Because the difference in compiled forms fundamentally affects how
the \class{Qtcm} instance facilitates Python-Fortran communication,
this attribute must be set on instantiation via a keyword input
parameter.

In the rest of this section, to avoid being verbose, when I
write \vars{'full'}, I mean the situation where
\vars{compiled\_form\thinspace=\thinspace'full'}.
Likewise, when I
write \vars{'parts'}, I mean the situation where
\vars{compiled\_form\thinspace=\thinspace'parts'}.


	\subsection{Initialization for 
			\vars{compiled\_form\thinspace=\thinspace'full'}}
				\label{sec:init.compiledform.full}

For a model run of this case, the \class{Qtcm} instance will
initialize the model using the Fortran \mods{varinit} subroutine
in the compiled QTCM1 model.  This subroutine does the following:

\begin{itemize}
\item If \vars{mrestart\thinspace=\thinspace1}, 
	the restart file is used to initialize all prognostic
	variables.  In terms of start date, the following rules are
	used:
	\begin{enumerate}
	\item Variable \vars{dateofmodel} is read from the restart file.
	\item If \vars{day0}, \vars{month0}, and \vars{year0}
		are negative, or otherwise
		invalid (e.g., \vars{month0} greater than 12), the invalid
		value is replaced with the
		day, month, and/or year of the day \emph{after} 
		that given by \vars{dateofmodel}.
		If the value of \vars{day0}, \vars{month0}, or \vars{year0}
		is not invalid in this sense, it is not replaced.
	\end{enumerate}
	Thus, if the restart file gives 
	\vars{dateofmodel} equal to 101102
	(year 10, month 11, day 2), and 
	\vars{day0\thinspace=\thinspace-1}, 
	\vars{month0\thinspace=\thinspace-1}, 
	\vars{year0\thinspace=\thinspace-1},
	and 
	\vars{mrestart\thinspace=\thinspace1}, 
	the model will start running from year 10, month 11, day 3.
	If \vars{dateofmodel} equals to 101102, and 
	\vars{day0\thinspace=\thinspace-1}, 
	\vars{month0\thinspace=\thinspace3}, 
	\vars{year0\thinspace=\thinspace-1},
	the model will start running from year 10, month 3, day 3.

\item If \vars{mrestart\thinspace=\thinspace0}, 
	all prognostic variables and right-hand sides are set to an
	initial value (which for most of those variables is zero).
	In terms of start date, \vars{day0} is set to 1 (and thus 
	the value of \vars{day0} previously input is ignored), and
	both \vars{month0} and \vars{year0}
	are set to 1 
	if their previously input values are invalid (where
	invalid means less than
	1, or, for \vars{month0}, greater than 12).
	Otherwise, \vars{month0} and \vars{year0} are left unchanged.
	Variable \vars{dateofmodel} has the value it had when the variable
	was declared (which is determined by the compiler and usually
	is zero; \vars{dateofmodel} will not be properly set until
	subroutine \mods{TimeManager} is called.

	Thus, if 
	\vars{day0\thinspace=\thinspace-1},
	\vars{month0\thinspace=\thinspace-1}, 
	\vars{year0\thinspace=\thinspace-1} is input into the model
	(say from a namelist) and 
	\vars{mrestart\thinspace=\thinspace0},
	the model will start running from year 1, month 1, day 1,
	and \vars{dateofmodel} at the exit of subroutine 
	\mods{varinit} will equal its compiler-set default.
	If 
	\vars{day0\thinspace=\thinspace14}, 
	\vars{month0\thinspace=\thinspace3}, 
	\vars{year0\thinspace=\thinspace11}, and 
	\vars{mrestart\thinspace=\thinspace0} on input into the
	model,
	the model will start running from year 11, month 3, day 1,
	and \vars{dateofmodel} at the exit of subroutine 
	\mods{varinit} will equal its compiler-set default.

	Note that \vars{dateofmodel}
	can thus be inconsistent with 
	\vars{month0} and \vars{year0} at the
	exit of subroutine \mods{varinit}.
\end{itemize}

This behavior with respect to initializing
the start date is different than in QTCM1 versions 1.0 and 2.1.
Please see the source code from those earlier QTCM1 versions for
details.




	\subsection{Initialization for 
			\vars{compiled\_form\thinspace=\thinspace'parts'}}
				\label{sec:init.compiledform.parts}

For \vars{'parts'} model, the methodology of how initialized
prognostic variables, right-hand sides, and start date related
variables are set is controlled by the \class{Qtcm} instance
attribute/flag \vars{init\_with\_instance\_state}.  The initialization
is (mostly) executed in the Python \vars{varinit} method in the
following way:

\begin{itemize}
\item If \vars{init\_with\_instance\_state} is False:
The method as described for
initialization for the 
\vars{'full'} case is generally
followed, with the exception that dateofmodel is set
to match \vars{day0}, \vars{month0}, \vars{year0}, prior to exit of 
\mods{varinit}.

\item If \vars{init\_with\_instance\_state} is True:
the model object will initialize the model based on the current
state of the model instance.  This enables you to set a model run
session's initial conditions based upon the state of the prognostic
variables and parameters stored at the Python level, which is
accessible at runtime.
\end{itemize}


Since the \vars{init\_with\_instance\_state\thinspace=\thinspace{False}}
case is mainly described by the initialization method for the
\vars{'full'} case, I refer the
reader to Section~\ref{sec:init.compiledform.full}.
For the case of \vars{init\_with\_instance\_state} is True, however,
the task is more complicated.  Specifically, for that case,
initialization includes the following:

\begin{enumerate}
\item If not currently defined,
	variable \vars{dateofmodel} is set to a default value of 0,
	which is specified in the module defaults.

\item The \vars{mrestart} flag is ignored for variable initialization.

\item All prognostic variables and right-hand sides
        are set to an
        initial value (which for most of those variables is zero),
	unless the variable is defined at the Python level, in which
	case the inital value is set to the Python level defined value.

\item If \vars{dateofmodel} is greater than 0, 
	\vars{day0}, \vars{month0}, and \vars{year0} are overwritten
        with values derived from \vars{dateofmodel} 
	in order to set the run to start
	the day \emph{after} \vars{dateofmodel}.

\item If \vars{dateofmodel} is less than or equal to 0, \vars{day0},
	\vars{month0}, and \vars{year0} are set to their respective
	instance attribute values, if valid.  For invalid instance
	attribute values, the invalid \vars{day0}, \vars{month0},
	and/or \vars{year0} is set to 1.

\item Variable \vars{dateofmodel} is recalculated
	and overwritten to match 
	\vars{day0}, \vars{month0}, \vars{year0}, prior to exit of 
	\mods{varinit}.
\end{enumerate}

As a result, for \vars{init\_with\_instance\_state} is True, the
way you indicate to the model that a run session is a brand-new run
is by setting, before the \mods{run\_session} method call,
\vars{dateofmodel} to a value less than or equal to 0, and \vars{day0},
\vars{model0}, and \vars{year0} to the day you want the model to
begin the run session.  To indicate to the model you wish to continue
a run, set \vars{dateofmodel} to the day \emph{before} you want the
model to start running from.

Examples:
\begin{itemize}
\item If \vars{day0\thinspace=\thinspace-1}, 
	\vars{month0\thinspace=\thinspace-1}, 
	\vars{year0\thinspace=\thinspace-1}, and
	\vars{dateofmodel\thinspace=\thinspace0} is input into 
	the model the model will start running from year 1, month 1, day 1,
	and 
	variable \vars{dateofmodel} at the exit of 
	subroutine \mods{varinit}
	will equal 10101.

\item If \vars{day0\thinspace=\thinspace14},
	\vars{month0\thinspace=\thinspace3}, 
	\vars{year0\thinspace=\thinspace11},
	and \vars{dateofmodel\thinspace=\thinspace0} is input into the
	model, the model will start running from year 11, month 3, day 14,
	and 
	variable \vars{dateofmodel} at the exit of 
	subroutine \mods{varinit} will equal
	110314.

\item If \vars{day0\thinspace=\thinspace14},
	\vars{month0\thinspace=\thinspace3}, 
	\vars{year0\thinspace=\thinspace11},
	and \vars{dateofmodel\thinspace=\thinspace341023} is input into the
	model, the model will start running from year 34, month 10, day 24,
	and at the exit of subroutine 
	\mods{varinit}, \vars{dateofmodel} will equal
	341024, with \vars{day0\thinspace=\thinspace24},
	\vars{month0\thinspace=\thinspace10}, and
	\vars{year0\thinspace=\thinspace34}.
\end{itemize}


	\subsection{Communication Between Python and Fortran-Levels}
				\label{sec:comm.py.fort.compiledform}

After initialization, the second major difference between a
\vars{'full'} and \vars{'parts'} model is how and when communication
between the Python and Fortran levels can occur.  For the \vars{'full'}
case, except for the passing in and out of variables before and after
a run session, all variable passing and subroutine calling happens in
the compiled QTCM1 model, with no control at the Python level.
For the \vars{'parts'} case, variables can be passed between the
Python and Fortran-levels at all levels down to the atmospheric
timestep, and many Fortran QTCM1 subroutines can be called from the
Python-level.  


		\subsubsection{Passing Variables}

For all \vars{compiled\_form} cases, variables are passed back and
forth between the Python \class{Qtcm} instance level and the
compiled QTCM1 model Fortran-level using the \class{Qtcm}
instance methods \mods{get\_qtcm1\_item} and \mods{set\_qtcm1\_item}:\footnote%
	{All Fortran routines used to pass variables back and forth are
	defined in the \mods{setbypy} module of the \fn{.so} extension
	module stored in the \class{Qtcm} instance variable \vars{\_\_qtcm}.
	All Fortran wrappers that enable Python to call compiled QTCM1 model
	subroutines are defined in the \mods{wrapcall} module stored in
	the \class{Qtcm} instance variable \vars{\_\_qtcm}.
	These modules are described in detail in 
	Sections~\ref{sec:setbypy} and~\ref{sec:wrapcall}, respectively.}

\begin{itemize}
\item \mods{get\_qtcm1\_item}(\dumarg{key}):
	Returns the value of the field variable given by the string
	\dumarg{key}.  If the compiled QTCM1 model variable given by
	\dumarg{key} is unreadable, the
        custom exception 
	\vars{FieldNotReadableFromCompiledModel} is thrown.
	The value returned is a copy of the value on the Fortran
	side, not a reference to the variable in memory.

\item \mods{set\_qtcm1\_item}:
	Sets the value of a field variable
	in the compiled QTCM1 model \emph{and at the Python-level,}
	automatically overriding any previous value at both levels.
	Thus, calling this method will change/create the \class{Qtcm}
	instance attribute corresponding to the field variable.
        When the compiled QTCM1 model variable is set, a copy of the
        Python value is passed to the Fortran model; the
	variable is \emph{not passed by reference.}
	This value comes from the \mods{set\_qtcm1\_item} calling
	parameter list, \emph{not} from the \class{Qtcm}
        instance attribute corresponding to the field variable.
\end{itemize}

The \mods{set\_qtcm1\_item} method has two calling forms, one with
one argument and the other with two arguments:
\begin{itemize}
\item One argument:  The method is called
	as \mods{set\_qtcm1\_item}(\dumarg{arg}), where \dumarg{arg} 
	is either a string giving the name of the field variable or 
	a \class{Field} instance.

\item Two arguments:  The method is called as
	\mods{set\_qtcm1\_item}(\dumarg{key}, \dumarg{value}), where
	\dumarg{key} is the string giving the name of the field variable
	and \dumarg{value} is the value to set the model field variable to
	(note \dumarg{value} can be a \class{Field} instance).
\end{itemize}
In either calling form, if no value given, the default value as defined
in module \mods{defaults} is used.

Some compiled QTCM1 model variables are not in a state where they
can be set.  An example is a compiled QTCM1 model pointer variable,
prior to the pointer being associated with a target (an attempt
to set would yield a bus error).  In such cases, the
\mods{set\_qtcm1\_item} method will throw a
\vars{FieldNotReadableFromCompiledModel} exception, nothing will
be set in the compiled QTCM1 model, and the Python counterpart
field variable (if it previously existed) would be left unchanged.\footnote%
	{We handle this situation in this way to enable the
	\class{Qtcm} instance to store variables
	even if the compiled model is not yet ready to accept them.}

Examples, typed in at a Python prompt, and
assuming that \vars{model} is a \class{Qtcm} instance:
\begin{itemize}
\item \cmd{dtvalue\thinspace=\thinspace{model.get\_qtcm1\_item('dt')}}:
	Retrieves the value of field variable \vars{dt} (timestep)
	from the compiled QTCM1 Fortran model and sets it to the
	Python variable \vars{dtvalue}.

\item \cmd{model.set\_qtcm1\_item('dt')}:
	Sets the value of field variable \vars{dt}
	in the compiled QTCM1 Fortran model to the default
	value (as given in \mods{defaults}),
	and sets the value of Python attribute \vars{model.dt} also to 
	that default value.  
	Remember that \vars{model.dt} is a \class{Field}
	instance.

\item \cmd{model.set\_qtcm1\_item('dt', 2000.)}:
	Sets the value of field variable \vars{dt}
	in the compiled QTCM1 Fortran model to 2000 (as a real),
	and sets the value of Python attribute \vars{model.dt} also to 2000.
\end{itemize}


		\subsubsection{Calling Compiled QTCM1 Model Subroutines}

All compiled QTCM1 model subroutines that can be called
(except \mods{driver} and \mods{varptrinit}) are in the
\mods{setbypy} or \mods{wrapcall} modules
of the \class{Qtcm} instance private attribute \vars{\_\_qtcm}.
(On \class{Qtcm} instance instantiation, \vars{\_\_qtcm} is set
to the \fn{.so} extension module that is the compiled QTCM1 Fortran model.)
Thus, to call \mods{wmconvct} in \mods{wrapcall} at the Python-level,
just type \cmd{model.\_\_qtcm.wrapcall.wmconvct()} (where \vars{model}
is a \class{Qtcm} instance).
For \mods{driver} and \mods{varptrinit}, these subroutines are not
contained in a \vars{\_\_qtcm} module, and thus can be called
directly (e.g., just type \cmd{model.\_\_qtcm.driver()}).
See Sections~\ref{sec:setbypy} and~\ref{sec:wrapcall} for more information
on the \mods{setbypy} and \mods{wrapcall} modules.

For the \vars{'full'} case, the only compiled QTCM1 model
subroutine you can usefully call during a run session is \mods{driver}.
For the \vars{'parts'} case, while you can essentially call any subroutine
given in a run list, you usually will not directly call a compiled QTCM1
model subroutine but will instead call it through including it in a
run list.  For example, if you have the following run list in a
\vars{'parts'} model:
\begin{codeblock}
\codeblockfont{%
[ 'qtcminit', '\_\_qtcm.wrapcall.woutpinit' ]}
\end{codeblock}
Running this list using the \class{Qtcm} instance method
\mods{run\_list} will result in \class{Qtcm} instance method
\mods{qtcminit} first being run, 
then the compiled QTCM1 Fortran model subroutine
\mods{woutpinit} in Fortran module \mods{wrapcall} being run.
See Section~\ref{sec:runlists} and
Table~\ref{tab:stnd.runlists} for a discussion and list of the
standard run lists that control routine execution content and order
in the \vars{'parts'} case.




%---------------------------------------------------------------------
\section{Restart and Continuation Run Sessions}
				\label{sec:contination.run.sessions}


	\subsection{Restart Runs In the Pure-Fortran QTCM1}
					\label{sec:puref90.restart}

To enable restart of a model run, the pure-Fortran QTCM1 model
writes out a restart file with the state of the prognostic variables
and select right-hand sides at that point in the run (for a list
of the variables, see Section~\ref{sec:snapshots}).  This binary
file can then be read in by later model runs.  The Fortran
\vars{mrestart} flag is passed in via a namelist; if \vars{mrestart}
is 1, the run uses the restart file (named \fn{qtcm.restart}).

One of the problems with using the restart file to do a continuation
run is that the continuation run will not be perfect.  In other words,
a 15~day run followed by a 25~day run based on the restart file 
generated at the end of the 15~day run will \emph{not} give the
exact same output as a continuous 40~day run.


	\subsection{Overview of Restart/Continuation Options In \mods{qtcm}}
					\label{sec:restart.options.list}

For a \class{Qtcm} instance, in contrast to the pure-Fortran QTCM1,
more than one method of continuation is available.
Thus, for a continuation run, you need to tell the model
``continue from what?''
The \class{Qtcm} class provides three choices for restart/continuing
a run:
\begin{enumerate} 
\item From a restart file:  Move/rename a QTCM1 restart file
        to the current working directory to \fn{qtcm.restart}.
	\label{list:continue.from.restart}

\item From a snapshot from another run session
	(see Sections~\ref{sec:snapshot.intro} and~\ref{sec:snapshots}).
	\label{list:continue.from.snapshot}

\item From the values of the \class{Qtcm} instance you will be
	calling \mods{run\_session} from.
	\label{list:continue.from.instance}
\end{enumerate}

Restart/continuation methods~\ref{list:continue.from.restart} 
and~\ref{list:continue.from.snapshot} both suffer from the
same problem as the pure-Fortran QTCM1 restart process:
They do not produce perfect restarts
(see Section~{sec:puref90.restart} for details).
In this section, I discuss the restart/continuation options
for each \vars{compiled\_form} option.

Methods~\ref{list:continue.from.restart}
and~\ref{list:continue.from.snapshot} are best used when making a
run session from a newly instantiated \class{Qtcm} instance.
Method~\ref{list:continue.from.instance} is best used when executing
a run session using a \class{Qtcm} instance that has already gone
through at least one run session.  Regardless of which method you
use, however, please note that anytime you execute a run session
using a \class{Qtcm} instance that already has made a previous run
session, some variables \emph{cannot be updated} between run sessions.
This feature is most noticeable with the output filename, and occurs
because the name persists in the compiled QTCM model, and is stored
in the extension module (\fn{.so} files in \vars{sodir}) associated
with the instance.  If you wish to control all variables possible
from the Python level (including output filename), you need do the
run session from a new model instance.


	\subsection{Restart/Continuation for 
		\vars{compiled\_form\thinspace=\thinspace'full'} 
		Model Instances}

The only option for restart when using
\vars{compiled\_form\thinspace=\thinspace'full'} model instances
is method~\ref{list:continue.from.restart}, to use a QTCM1 restart
file.\footnote%
	{The \vars{cont} keyword parameter in \mods{run\_session}
	and the value of the \vars{init\_with\_instance\_state}
	attribute have no effect if
	\vars{compiled\_form\thinspace=\thinspace'full'}.  With
	\vars{'full'}, the call to initialize variables all happens
	at the Fortran level (via the Fortran \mods{varinit}, not
	the Python \mods{varinit}), with no reference to the Python field
	states (or even existing Fortran field states, if present).}
To use this option, the value of the \vars{mrestart} 
attribute must equal 1, the restart file must be named
\fn{qtcm.restart}, and the restart file must be in the 
current working directory.
As with the pure-Fortran QTCM1 restart process, this method
does not produce perfect restarts.



	\subsection{Restart/Continuation for 
		\vars{compiled\_form\thinspace=\thinspace'parts'} 
		Model Instances}

For the \vars{compiled\_form\thinspace=\thinspace'parts'} case,
all three restart/continuation methods
described in Section~\ref{sec:restart.options.list} are
available.


		\subsubsection{Method~\ref{list:continue.from.restart}:
			From a QTCM1 Restart File}

To use the QTCM1 restart file mechanism, not only must the
\vars{mrestart} attribute have a value to 1, but the
\vars{init\_with\_instance\_state} flag also has to be \vars{False},
otherwise the \vars{mrestart} attribute value will be ignored.  
As with the pure-Fortran QTCM1 restart process, this method does not
produce perfect restarts.


		\subsubsection{Method~\ref{list:continue.from.snapshot}:
			From a \class{Qtcm} Instance Snapshot}

You can take snapshots of the model state of a \class{Qtcm} instance
by the \mods{make\_snapshot} instance method.  This snapshot saves
a copy of all the variables saved to a QTCM1 restart file (see
Section~\ref{sec:snapshots} for the full list of fields), which
then can be passed to other \class{Qtcm} instances for use in other
run sessions.

The key difference between this method and 
method~\ref{list:continue.from.instance} (described below)
is that \mods{run\_session} calls using the snapshot are done
\emph{without} the \vars{cont} keyword input parameter
(by default, \vars{cont} is False).  If the \vars{cont} keyword
is not False, it says the run session is a continuation run
that uses the state of the compiled QTCM1 model for all variables
that are not specified at, and read-in from,
the Python level.  If the \vars{cont} keyword
is False, the run session initializes as if it were a new run.

See Section~\ref{sec:snapshot.intro} for details and
an example of using snapshots to initialize a run session.
Note that as with the pure-Fortran QTCM1 restart process, this method 
does not produce perfect restarts.


		\subsubsection{Method~\ref{list:continue.from.instance}:
			From the Calling \class{Qtcm} Instance}

This method is used when you want to make a run session that is a
``true'' continuation run, i.e., one that uses the current state
of the compiled QTCM1 model for all variables that are not read-in
from the Python level (remember that \class{Qtcm} instances hold a
subset of the variables defined at the Fortran level).  
The key reason to use this method for a continuation run session
is that the continuation is byte-for-byte the same (if no fields
are changed) as if the run just went straight on through.  Thus,
the continuation would be perfect: A 15~day run followed by a 25~day
run using the same \class{Qtcm} instance with the \vars{cont} keyword
will give the exact same output as a continuous 40~day run.  This
is not the case when making a new instance and passing a restart
file or a snapshot, because a separate extension module is used for
those new instances.

Control of this method is accomplished through the \vars{cont}
keyword input parameter to the \mods{run\_session} method and the
\vars{init\_with\_instance\_state} attribute of a
\class{Qtcm} instance:

\begin{itemize}
\item \vars{cont}: If set to False, the run session is not a
	continuation of the previous run, but a new run session.
	If set to True, the run session is a continuation of the
	previous run session.  If set to an integer greater than
	zero, the run session is a continuation just like
	\vars{cont\thinspace=\thinspace{True}}, but the value
	\vars{cont} is set to is used for \vars{lastday} and replaces
	\vars{lastday.value} in the \class{Qtcm} instance.

\item \vars{init\_with\_instance\_state}:
	If True, for a \mods{run\_session} call using the
	\vars{cont} keyword, whatever the field values are in the Python
	instance are used in the run session.
	If False, model variables are set and initialized as described in
	Section~\ref{sec:init.compiledform.parts}.  In that case,
	previous compiled QTCM1 model values will likely be overwritten.
	Thus, if you want a continuation run that uses the state of
	all field variables except for those you explicitly change at
	the Python-level, make sure \vars{init\_with\_instance\_state}
	is True.
\end{itemize}

(Note that the \vars{cont} keyword has no effect if \vars{compiled\_form}
is \vars{'full'}.  The default value of \vars{cont} in a
\mods{run\_session} call is False.  The value of keyword \vars{cont}
is stored as private instance attribute \vars{\_cont}, in case you
really need to access it elsewhere; see
Section~\ref{sec:Qtcm.private.attrib} for more details).

The example described in Section~\ref{sec:continuation.intro} is
an example of method~\ref{list:continue.from.instance} in the list
above: The second run session is continued from the state of
\vars{model}, with the values of \vars{model}'s instance variables
overriding any values in the compiled QTCM1 model in initializing
the second run session.

This method has a few caveats worthy of note:
\begin{itemize}
\item The \vars{init\_with\_instance\_state} attribute value
	will have no effect unless the instance prognostic variables
	are set, i.e., unless a previous run session has been done.
	Another way to put it is for an initial run session right
	after a \class{Qtcm} instance is created, \mods{varinit}
	will use the same initial values for prognostic variables
	(defined in \mods{defaults} module variable
	\vars{init\_prognostic\_dict})\footnote%
		{\vars{init\_prognostic\_dict} is the dictionary giving
		the default initial values of each prognostic variable
		and right-hand side (as defined by the restart file 
		specification).}
	as it would with for both
	\vars{init\_with\_instance\_state} set to True or False).

\item Continuation run sessions using this method have to continue
	with the next day from wherever the last run session left
	off, contiguously.\footnote%
		{For continuation run sessions, you keep the 
		same extension module (the compiled \fn{.so} library),
		and all the values that define the state where it
		left off.}
	If you want to do a non-contiguous run,
	create a new \class{Qtcm} instance initialized with a
	snapshot instead of the continuation method describe in
	this section.
	will use restart rules to run a new model.  

\item When making a continuation run session using this method,
	you cannot change some variables, for instance,
	\vars{outdir} and any of the date related
	variables.  In fact, the only thing you should change for
	your continuation run session are the prognostic and
	diagnostic variables and \vars{lastday}.  This is because
	some variables cannot be updated between run sessions.
	As noted in Section~\ref{sec:restart.options.list},
	if you wish to control all variables possible
	from the Python level (including output filename), you need 
	to execute the run session from a new model instance.
\end{itemize}


	\subsection{Snapshots of a \class{Qtcm} Instance}
				\label{sec:snapshots}

The snapshot dictionary (briefly described in
Section~\ref{sec:snapshot.intro}), saved as the \class{Qtcm} instance
attribute \vars{snapshot}, and generated by the method
\mods{make\_snapshot}, saves the current state of the following
instance field variables:

\begin{center}
% This file is automatically generated by
% code_to_latex.py.  It lists all the snapshot variables.


\begin{longtable}{l|c|c|p{0.4\linewidth}}
\textbf{Field} & \textbf{Shape} &
                                \textbf{Units} & \textbf{Description} \\
\hline
\endhead
    \vars{T1} & (64, 44) & K &  \\
\vars{Ts} & (64, 42) & K & Surface temperature \\
\vars{WD} & (64, 42) &  &  \\
\vars{dateofmodel} &      &  & Date of model coded as an integer as yyyymmdd \\
\vars{psi0} & (64, 43) &  &  \\
\vars{q1} & (64, 44) & K &  \\
\vars{rhsu0bar} & (3,) &  &  \\
\vars{rhsvort0} & (64, 42, 3) &  &  \\
\vars{title} &      &  & A descriptive title \\
\vars{u0} & (64, 44) & m/s & Barotropic zonal wind \\
\vars{u0bar} &      &  &  \\
\vars{u1} & (64, 44) & m/s & Current time step baroclinic zonal wind \\
\vars{v0} & (64, 43) & m/s & Barotropic meridional wind \\
\vars{v1} & (64, 43) & m/s &  \\
\vars{vort0} & (64, 42) &  &  \\
\end{longtable}
\end{center}

These are the same variables saved to a QTCM1 restart file, and so
a snapshot duplicates the restart functionality in the Python
environment, but with more flexibility.  Since the \vars{snapshot}
dictionary is a Python variable like any other, you can manipulate
it and alter it to fit any condition you wish.




%---------------------------------------------------------------------
\section{Creating and Using Run Lists}  \label{sec:runlists}

Section~\ref{sec:runlist.intro} provides an introduction to the
role and use of run lists.  A run list is a list of methods, Fortran
subroutines, and other run lists that can be executed by the
\class{Qtcm} instance \mods{run\_list} method.  Run lists are stored
in the \class{Qtcm} instance attribute \vars{runlists}, which is a
dictionary of run lists.  The names of run lists should not be
preceeded by two underscores (though elements of a run list may be
very private variables), nor should names of run lists be the same
as any instance attribute.  Run lists are not available for
\vars{compiled\_form\thinspace=\thinspace'full'}.

The \mods{run\_list} method takes a single input parameter, a list,
and runs through that list of elements that specify other run lists
or instance method names to execute.  Methods with private attribute
names are automatically mangled as needed to become executable by
the method.  Note that if an item in the input run list is an
instance method, it should be the entire name (not including the
instance name) of the callable method, separated by periods as
appropriate.

Elements in a run list are either strings or 1-element dictionaries.
Consider the following example, where \vars{model} is a \class{Qtcm}
instance, and \mods{run\_list} is called using \vars{mylist} as
input:

\begin{codeblock}
\codeblockfont{%
model = Qtcm(\ldots) \\
mylist = [ \{'varinit':None\}, \\
\hspace*{13ex}'init\_model', \\
\hspace*{13ex}'\_\_qtcm.driver', \\
\hspace*{13ex}\{'set\_qtcm1\_item': ['outdir', '/home/jlin']\} ]
model.run\_list(mylist)}
\end{codeblock}

The first element in \vars{mylist} refers to a method that requires
no positional input parameters be passed in (as shown by the None).
The second and third elements in \vars{mylist} also refers to methods
that require no positional input parameters be passed in.  The last
element in \vars{mylist} refers to a method with two input parameters.
Note that while I use the term ``method'' to describe the elements,
the strings/keys do not have to be only Python instance methods.
The second element, for instance, refers to another run list, and
the third element refers to a compiled QTCM1 model subroutine (note
the \vars{\_\_qtcm} attribute).

When the \mods{run\_list} method is called, the items in the input
run list are called in the order given in the list.  For each
element,  the \mods{run\_list} method first checks if the string
or dictionary key name corresponds to the key of an entry in the
\class{Qtcm} instance attribute \vars{runlists}.  If so, \mods{run\_list}
is called using that run list (i.e., it is a ``recursive'' call).
If the string or dictionary key name does not refer to another run
list, the \mods{run\_list} method checks if the string or dictionary
key name is a method of the \class{Qtcm} instance, and if so the
method is called.  Any other value throws an exception.

If input parameters for a method are of class \class{Field}, the
\mods{run\_list} method first tries to pass the parameters into the
method as is, i.e., as Field object(s).  If that fails, the
\mods{run\_list } method  passes its parameters in as the \vars{value}
attribute of the \class{Field} object.

If you want a variable that is being passed into a run list to be
continuously updated, you have to set the parameter in the run list
to a \class{Field} instance that is a \class{Qtcm} instance attribute,
not just to the value of the field variable (or to a non-\class{Field}
object).  Otherwise, subsequent calls to that run list element will
not use the updated values as input parameters.

For instance, if you had a run list element:
\begin{codeblock}
\codeblockfont{%
\{'\_\_qtcm.timemanager':[model.coupling\_day,]\}}
\end{codeblock}
and \vars{model.coupling\_day} were an integer (it's not by default,
but pretend it was), then \mods{run\_list} calling
\mods{\_\_qtcm.timemanager} will pass in a scalar integer rather
than a binding to the variable \vars{model.coupling\_day}.  In such
a situation, if the variable \vars{model.coupling\_day} were updated
in time, the \mods{run\_list} call of \mods{\_\_qtcm.timemanager}
would not be updated in time.  This happens because when the
dictionary that is the run list element is created, the value of
list element(s) attached to the dictionary element is set to the
scalar value of \vars{model.coupling\_day} at that instant.

You can get around this feature by setting \class{Qtcm} instance
attributes that will change with model execution to \class{Field}
instances, and then referring to those attributes in the parameter
list in the run list element.  In that case:
\begin{codeblock}
\codeblockfont{%
\{'\_\_qtcm.timemanager':[model.coupling\_day,]\}}
\end{codeblock}
will use the current value of \vars{model.coupling\_day} anytime
\vars{\_\_qtcm.timemanager} is called by \mods{run\_list}, if
\vars{model.coupling\_day} is a \class{Field} object.

When \mods{run\_list}, encounters a calling input parameter that
is a \class{Field} object, it will first try to pass the entire
\class{Field} object to the method/routine being called.  If that
raises an exception, it will then try to pass just the value of the
entire \class{Field} object.  This is done to enable \mods{run\_list}
to be used for both pure-Python and compiled QTCM Fortran model
routines.  Fortran cannot handle \class{Field} objects as input
parameters, only values.

Table~\ref{tab:stnd.runlists} shows all standard run lists
stored in the \vars{runlists} attribute upon instantiation
of a \class{Qtcm} instance.

\begin{htmlonly}
\begin{table}[htp]
\begin{center}
\fbox{Empty placeholder block for table that would have gone here.}
\end{center}
\caption{Standard run lists stored in the \vars{runlists} 
	attribute upon instantiation of a \class{Qtcm} instance.
	The run list and list element names are stored as strings.
	\emphpara{This table is improperly reproduced in the
	HTML conversion.  Please see the PDF version for the table.}}
\label{tab:stnd.runlists}
\end{table}
\end{htmlonly}

\begin{latexonly}
\begin{table}[htp]
% This file is automatically generated by
% code_to_latex.py.  It lists all the standard 
% runlists in the class.


\begin{longtable}{l|lc}
\textbf{Run List Name/Description} & \textbf{List Element(s) Name(s)} & 
                                                   \textbf{\# Arg(s)} \\
\hline
\endhead
\multirow{3}{*}{\parbox{0.4\linewidth}{atm\_bartr\_mode (calculate the atmospheric barotropic mode at the barotropic timestep)}} & \_\_qtcm.wrapcall.wsavebartr & None \\
         & \_\_qtcm.wrapcall.wbartr & None \\
         & \_\_qtcm.wrapcall.wgradphis & None \\
\hline
\multirow{5}{*}{\parbox{0.4\linewidth}{atm\_oc\_step (calculate the atmosphere and ocean models at a coupling timestep)}} & \_first\_method\_at\_atm\_oc\_step & None \\
         & \_\_qtcm.wrapcall.wtimemanager & 1\\
         & \_\_qtcm.wrapcall.wocean & 2\\
         & qtcm & None \\
         & \_\_qtcm.wrapcall.woutpall & None \\
\hline
\multirow{5}{*}{\parbox{0.4\linewidth}{atm\_physics1 (calculate atmospheric physics at one instant)}} & \_\_qtcm.wrapcall.wmconvct & None \\
         & \_\_qtcm.wrapcall.wcloud & None \\
         & \_\_qtcm.wrapcall.wradsw & None \\
         & \_\_qtcm.wrapcall.wradlw & None \\
         & \_\_qtcm.wrapcall.wsflux & None \\
\hline
\multirow{8}{*}{\parbox{0.4\linewidth}{atm\_step (calculate the entire atmosphere at one atmosphere timestep)}} & atm\_physics1 & None \\
         & \_\_qtcm.wrapcall.wsland1 & None \\
         & \_\_qtcm.wrapcall.wadvctuv & None \\
         & \_\_qtcm.wrapcall.wadvcttq & None \\
         & \_\_qtcm.wrapcall.wdffus & None \\
         & \_\_qtcm.wrapcall.wbarcl & None \\
         & \_bartropic\_mode\_at\_atm\_step & None \\
         & \_\_qtcm.wrapcall.wvarmean & None \\
\hline
\multirow{3}{*}{\parbox{0.4\linewidth}{init\_model (initialize the entire model, i.e., the atmosphere and ocean components and output)}} & qtcminit & None \\
         & \_\_qtcm.wrapcall.woceaninit & None \\
         & \_\_qtcm.wrapcall.woutpinit & None \\
\hline
\multirow{5}{*}{\parbox{0.4\linewidth}{qtcminit (initialize the atmosphere portion of the entire model)}} & \_\_qtcm.wrapcall.wparinit & None \\
         & \_\_qtcm.wrapcall.wbndinit & None \\
         & varinit & None \\
         & \_\_qtcm.wrapcall.wtimemanager & 1\\
         & atm\_physics1 & None \\
\end{longtable}
\caption{Standard run lists stored in the \vars{runlists} 
	attribute upon instantiation of a \class{Qtcm} instance.
	The run list and list element names are stored as strings.}
\label{tab:stnd.runlists}
\end{table}
\end{latexonly}

Of course, feel free to change the contents of any of the run lists
after instantiation, or to add additional run lists to the
\vars{runlists} attribute dictionary.  The ability to alter run
lists at runtime gives the \mods{qtcm} package much of its flexibility.




%---------------------------------------------------------------------
\section{Field Variables and the \class{Field} Class}
						\label{sec:field.variables}

The term ``field'' variable refers to QTCM1 model variables that 
are accessible at both the compiled Fortran QTCM1 model-level as
well as the Python \class{Qtcm} instance-level.
Field variables are all instances of the \class{Field} class
(though non-field variables can also be instances of \class{Field}).

Section~\ref{sec:field.variables.intro} gives a brief introduction to
the attributes and methods in a \class{Field} instance.
A nitty gritty description of the class is found in its docstrings.

	\subsection{Creating Field Variables}

To create a \class{Field} instance whose value is set to the
default, instantiate with the field id as the only positional
input argument.  Thus:

\begin{codeblock}
\codeblockfont{foo = Field('lastday')}
\end{codeblock}

will return \vars{foo} as a \class{Field} instance with \vars{foo.value}
set to the value listed in Section~\ref{sec:defaults.scalar}.
The value of all \class{Field} instances upon creation are specified
in the \mods{defaults} submodule of package \mods{qtcm}, and listed
in Sections~\ref{sec:defaults.scalar} and~\ref{sec:defaults.array}.

To create \class{Field} instances whose attributes are set different
from their defaults, you can specify the different settings in the
instantiation parameter list, or change the attributes once the
instance is created.  See the \class{Field} docstring for details.


	\subsection{Initial Field Variables}  \label{sec:initial.variables}

Field variables include both model parameters that do not change
for a \class{Qtcm} instance as well as prognostic variables that
do change during model integration.  As a result, many field variables
have values different from the default values listed in
Sections~\ref{sec:defaults.scalar} and~\ref{sec:defaults.array}.
In this section, I list the \emph{initial} values of all field
variables.  The ``initial'' values are the settings for \class{Qtcm}
field variables execution of the \mods{run\_session} method, but
prior to cycling through an atmosphere-ocean coupling timestep.
This is in contrast to ``default'' values, which the field variables
are given on instantiation, if no other value is specified.
Numerical values are rounded as per the conventions
of Python's \vars{\%g} format code.


		\subsubsection{Scalars}

For the fields that give the input/output directory names, and the
run name, the entry ``value varies'' is provided in the ``Value''
column.

% This file is automatically generated by the script
% code_to_latex.py in the doc/latex directory.  It is based
% upon the values found after model initialization, and should
% not be hand-edited if you want the values to correspond to
% the values in a Qtcm instance, for compiled_form='parts'.


\begin{longtable}{l|c|c|p{0.42\linewidth}}
\textbf{Field} & \textbf{Value} & \textbf{Units} & 
                                \textbf{Description} \\
\hline
\endhead
\vars{SSTdir} & value varies &  & Where SST files are \\
\vars{SSTmode} & seasonal &  & Decide what kind of SST to use \\
\vars{VVsmin} & 4.5 & m/s & Minimum wind speed for fluxes \\
\vars{bnddir} & value varies &  & Boundary data other than SST \\
\vars{dateofmodel} & 10101 &  & Date of model coded as an integer as yyyymmdd \\
\vars{day0} & 1 & dy & Starting day; if $<$ 0 use day in restart \\
\vars{dt} & 1200 & s & Time step \\
\vars{eps\_c} & 0.000138889 & 1/s & 1/tau\_c NZ (5.7) \\
\vars{interval} & 1 & dy & Atmosphere-ocean coupling interval \\
\vars{it} & 1 &  & Time of day in time steps \\
\vars{landon} & 1 &  & If not 1: land = ocean with fake SST \\
\vars{lastday} & 0 & dy & Last day of integration \\
\vars{month0} & 1 & mo & Starting month; if $<$ 0 use mo in restart \\
\vars{mrestart} & 0 &  & =1: restart using qtcm.restart \\
\vars{mt0} & 1 &  & Barotropic timestep every mt0 timesteps \\
\vars{nastep} & 1 &  & Number of atmosphere time steps within one air-sea coupling interval \\
\vars{noout} & 0 & dy & No output for the first noout days \\
\vars{nooutr} & 0 & dy & No restart file for the first nooutr days \\
\vars{ntout} & -30 & dy & Monthly mean output \\
\vars{ntouti} & 0 & dy & Monthly instantaneous data output \\
\vars{ntoutr} & 0 & dy & Restart file only at end of model run \\
\vars{outdir} & value varies &  & Where output goes to \\
\vars{runname} & value varies &  & String for an output filename \\
\vars{title} & value varies &  & A descriptive title \\
\vars{u0bar} & 0 &  &  \\
\vars{visc4x} & 700000 & m$^2$/s & Del 4 viscocity parameter in x \\
\vars{visc4y} & 700000 & m$^2$/s & Del 4 viscocity parameter in y \\
\vars{viscxT} & 1.2e+06 & m$^2$/s & Temperature diffusion parameter in x \\
\vars{viscxq} & 1.2e+06 & m$^2$/s & Humidity diffusion parameter in x \\
\vars{viscxu0} & 700000 & m$^2$/s & Viscocity parameter for u0 in x \\
\vars{viscxu1} & 700000 & m$^2$/s & Viscocity parameter for u1 in x \\
\vars{viscyT} & 1.2e+06 & m$^2$/s & Temperature diffusion parameter in y \\
\vars{viscyq} & 1.2e+06 & m$^2$/s & Humidity diffusion parameter in y \\
\vars{viscyu0} & 700000 & m$^2$/s & Viscocity parameter for u0 in y \\
\vars{viscyu1} & 700000 & m$^2$/s & Viscocity parameter for u1 in y \\
\vars{weml} & 0.01 & m/s & Mixed layer entrainment velocity \\
\vars{year0} & 1 & yr & Starting year; if $<$ 0 use year in restart \\
\vars{ziml} & 500 & m & Atmosphere mixed layer depth $\sim$ cloud base \\
\end{longtable}


		\subsubsection{Arrays}

% This file is automatically generated by the script
% code_to_latex.py in the doc/latex directory.  It is based
% upon the values found after model initialization, and should
% not be hand-edited if you want the values to correspond to
% the values in a Qtcm instance, for compiled_form='parts'.


\begin{longtable}{l|c|c|c|c|p{0.3\linewidth}}
\textbf{Field} & \textbf{Shape} & \textbf{Max} & \textbf{Min} &
                                \textbf{Units} & \textbf{Description} \\
\hline
\endhead
\vars{Evap} & (64, 42) & 1502.56 & 223.552 &  &  \\
\vars{FLW} & (64, 42) & 74.5136 & 74.5136 &  &  \\
\vars{FLWds} & (64, 42) & 206.424 & 206.424 &  &  \\
\vars{FLWus} & (64, 42) & 429.708 & 429.708 &  &  \\
\vars{FLWut} & (64, 42) & 148.771 & 148.771 &  &  \\
\vars{FSW} & (64, 42) & 147.767 & 0 &  &  \\
\vars{FSWds} & (64, 42) & 410.895 & -6.99713 &  &  \\
\vars{FSWus} & (64, 42) & 356.831 & -4.49983 &  &  \\
\vars{FSWut} & (64, 42) & 332.431 & 0 &  &  \\
\vars{FTs} & (64, 42) & 930.115 & 138.383 &  &  \\
\vars{Qc} & (64, 42) & 0 & 0 & K & Precipitation \\
\vars{S0} & (64, 42) & 534.264 & 0 &  &  \\
\vars{STYPE} & (64, 42) & 3 & 0 &  & Surface type; ocean or vegetation type over land \\
\vars{T1} & (64, 44) & -100 & -100 & K &  \\
\vars{Ts} & (64, 42) & 295 & 295 & K & Surface temperature \\
\vars{WD} & (64, 42) & 350 & 0 &  &  \\
\vars{WD0} & (4,) & 500 & 0 &  & Field capacity SIB2/CSU (approximately) \\
\vars{arr1} & (64, 42) & 0 & 0 &  & Auxiliary optional output array 1 \\
\vars{arr2} & (64, 42) & 0 & 0 &  & Auxiliary optional output array 2 \\
\vars{arr3} & (64, 42) & 0.138699 & 0.138699 &  & Auxiliary optional output array 3 \\
\vars{arr4} & (64, 42) & 0 & 0 &  & Auxiliary optional output array 4 \\
\vars{arr5} & (64, 42) & 0 & 0 &  & Auxiliary optional output array 5 \\
\vars{arr6} & (64, 42) & 0 & 0 &  & Auxiliary optional output array 6 \\
\vars{arr7} & (64, 42) & 0 & 0 &  & Auxiliary optional output array 7 \\
\vars{arr8} & (64, 42) & 0 & 0 &  & Auxiliary optional output array 8 \\
\vars{psi0} & (64, 43) & 0 & 0 &  &  \\
\vars{q1} & (64, 44) & -50 & -50 & K &  \\
\vars{rhsu0bar} & (3,) & 0 & 0 &  &  \\
\vars{rhsvort0} & (64, 42, 3) & 0 & 0 &  &  \\
\vars{taux} & (64, 42) & 0 & 0 &  &  \\
\vars{tauy} & (64, 42) & 0 & 0 &  &  \\
\vars{u0} & (64, 44) & 0 & 0 & m/s & Barotropic zonal wind \\
\vars{u1} & (64, 44) & 0 & 0 & m/s & Current time step baroclinic zonal wind \\
\vars{v0} & (64, 43) & 0 & 0 & m/s & Barotropic meridional wind \\
\vars{v1} & (64, 43) & 0 & 0 & m/s &  \\
\vars{vort0} & (64, 42) & 0 & 0 &  &  \\
\end{longtable}



	\subsection{Passing Fields Between the Python and Fortran-Levels}

Section~\ref{sec:comm.py.fort.compiledform} discusses the differences
between how the \vars{'full'} and \vars{'parts'} compiled forms
pass field variables between the Python and Fortran-levels.  That
discussion gives a detailed description of the methods used for
passing fields to and from the Python and Fortran-levels (i.e., the
\mods{get\_qtcm1\_item} and \mods{set\_qtcm1\_item} methods).

Please note the following regarding field variables as you pass them 
back and forth between the Python and Fortran-levels:
\begin{itemize}
\item Field variables with ghost latitudes, such as \vars{u1}, on
	the Python end are always the full variables (i.e., including
	the ghost latitudes).  On the Fortran end, variables like
	\vars{u1} also always have the ghost latitudes while in the
	model, but when stored as restart files, do not have the
	ghost latitudes; the end points are not saved in restart
	files or written to the netCDF output files.
	See the
	\latexhtml{%
\htmladdnormallinkfoot{QTCM1 manual}%
        {http://www.atmos.ucla.edu/$\sim$csi/qtcm\_man/v2.3/qtcm\_manv2.3.pdf}}%
{\htmladdnormallink{QTCM1 manual}%
        {http://www.atmos.ucla.edu/~csi/qtcm_man/v2.3/qtcm_manv2.3.pdf}}
	\cite{Neelin/etal:2002}
	for details about ghost latitudes.

\item You should assume there is only a full synchronizing between 
	compiled QTCM1 model and Python model field variables
	at the beginning and end of a run session.  

\item If you have a variable at the Python-level, but at the
	compiled QTCM1 Fortran model-level the variable is not
	readable, if you try to call \mods{set\_qtcm1\_item} on the
	variable, nothing is done, and the Python-level value is
	left alone.  If you have a compiled QTCM1 model variable,
	but no Python-level equivalent, if you call \mods{set\_qtcm1\_item}
	on the variable, the Python-level variable (as an attribute)
	is created.

\item To be precise, only compiled QTCM1 model variables can be
	passed pass back and forth between the Python and Fortran-levels;
	there are many \class{Qtcm} instance attributes that do not
	have any counterparts at the Fortran-level.\footnote%
		{I use the term ``field variables'' to refer to 
		compiled QTCM1 model variables that can be passed
		back and forth to the Python level.}

\item Although \vars{dayofmodel} is described in module \mods{setbypy}
	as an option for the \mods{get\_qtcm1\_item} and
	\mods{set\_qtcm1\_item} methods to operate on, in reality
	those methods cannot operate on \vars{dayofmodel}, but
	\vars{dayofmodel} is not defined in \mods{defaults}.\footnote%
		{All field variables must be defined in \mods{defaults} in
		order for the proper Fortran routine to be called
		according to the variable's type.}
\end{itemize}


	\subsection{Field Variable Shape}   \label{sec:field.var.shape}

Normally, Python arrays have a different dimension order than Fortran
arrays.  While Fortran arrays are dimensioned (col, row, slice),
with adjacent columns being contiguous, then rows, and then slices, Python
arrays are dimensioned (slice, row, col), with adjacent columns being
contiguous, then rows, and then slices.  Based on this, you would
think that everytime you passed an array between the Python and
Fortran-levels you would need to transpose the array.

Thankfully, we don't have to do this because \mods{f2py} handles
array dimension order transparently so we can refer to each element
the same way whether we're in Python or Fortran.  Thus, the array
\vars{Qc} in Fortran is dimensioned (longitude, latitude), (64,42)
by default, and the Python \class{Qtcm} instance attribute \vars{Qc}
has a \vars{value} attribute also dimensioned (longitude, latitude),
(64,42) by default.  And at both the Fortran and Python-levels, the
first longtude, second latitude element is referred to as \vars{Qc(1,2)}.

In contrast, however, netCDF output saved by the compiled QTCM1 model
and read into Python (using the \mods{Scientific} package) is
\emph{not} in Fortran array order.  Arrays read from netCDF output
into Python are in Python array order, and are dimensioned
(latitude, longitude) or (time, latitude, longitude).  The \class{Qtcm}
routines that manipulate netCDF data (e.g., \mods{plotm}), however,
automatically adjust for this, so you only need to be aware of this
when reading in output for your own analysis
(see Section~\ref{sec:model.output}).




%---------------------------------------------------------------------
\section{Model Output}			\label{sec:model.output}

Section~\ref{sec:output.intro} gives an overview of how to
use \mods{qtcm} model output to netCDF files.

All netCDF array output is dimensioned (time, latitude, longitude)
when read into Python using the \mods{Scientific} package.  This
differs from the way \class{Qtcm} saves field variables, which
follows Fortran convention (longitude, latitude).  Thus, the shapes
in Section~\ref{sec:initial.variables}, Appendix~\ref{app:defaults.values},
etc., are not the shapes of arrays read from the netCDF output.
See Section~\ref{sec:field.var.shape} for a discussion of why
there is this discrepancy.

Because netCDF files allow you to specify an ``unlimited'' dimension,
it is possible to close a netCDF file, reopen it, and add more
slices of data to the file.  Thus, continuous \class{Qtcm} run
sessions (i.e., those that use the \vars{cont} keyword input parameter
in the \mods{run\_session} method) will automatically append output
to the netCDF output files.

Field variables with ghost latitudes, such as \vars{u1}, on the
Python and Fortran ends are always the full variables (i.e., including
the ghost latitudes).  The ghost latitudes are not written to the
netCDF output files, however.
See the \latexhtml{%
\htmladdnormallinkfoot{QTCM1 manual}%
        {http://www.atmos.ucla.edu/$\sim$csi/qtcm\_man/v2.3/qtcm\_manv2.3.pdf}}%
{\htmladdnormallink{QTCM1 manual}%
        {http://www.atmos.ucla.edu/~csi/qtcm_man/v2.3/qtcm_manv2.3.pdf}}
	\cite{Neelin/etal:2002}
for details about ghost latitude structure.

\class{Qtcm} instances have a few built-in tools to visualization
model output.  These are briefly described in Section~\ref{sec:viz.intro}.
Note that the \mods{plotm} method is linked to a specific \class{Qtcm}
instance.  Do not use \mods{plotm} outside of the instance it is
linked to.  It must also be used only after a successful run session
(i.e., not in the middle of a run session).




%---------------------------------------------------------------------
\section{Miscellaneous}

A few miscellaneous items/issues about the model:
\begin{itemize}
\item The land model runs at same timestep as the atmosphere.

\item If the land model runs less often than 
	\mods{sflux} in \mods{physics1}, 
	the calculation of evaporation over the land 
	needs to be fixed in sflux.

\item The units of some field variables are not what you would expect.
	For instance, \vars{Qc} is in energy units, i.e., K, and not
	mm/day.
	See the
	\latexhtml{%
\htmladdnormallinkfoot{QTCM1 manual}%
        {http://www.atmos.ucla.edu/$\sim$csi/qtcm\_man/v2.3/qtcm\_manv2.3.pdf}}%
{\htmladdnormallink{QTCM1 manual}%
        {http://www.atmos.ucla.edu/~csi/qtcm_man/v2.3/qtcm_manv2.3.pdf}}
	\cite{Neelin/etal:2002}
	for details.
\end{itemize}




%---------------------------------------------------------------------
\section{Cookbook of Ways the Model Can Be Used}  \label{sec:cookbook}

This cookbook of a few ways to use the model is arranged by science
tasks, i.e., certain types of runs we want to do.  For some of the
examples below, I assume that the dictionary
\vars{inputs} is initially defined as given in
Figure~\ref{fig:defn.of.inputs}.  All examples assume that
\cmd{from qtcm import Qtcm} has already been executed.


%--- Two versions, one for PDF and the other for HTML:
\begin{latexonly}
\begin{figure}[tp]
\begin{codeblock}
\codeblockfont{%
inputs = \{\} \\
inputs['runname'] = 'test' \\
inputs['landon'] = 0 \\
inputs['year0'] = 1 \\
inputs['month0'] = 11 \\
inputs['day0'] = 1 \\
inputs['lastday'] = 30 \\
inputs['mrestart'] = 0 \\
inputs['init\_with\_instance\_state'] = True \\
inputs['compiled\_form'] = 'parts'}
\end{codeblock}

\caption{The initial definition of the \vars{inputs} dictionary for 
	examples given in Section~\ref{sec:cookbook}.  These settings
	imply that a run session will start on November 1, Year 1,
	last for 30 days, and will be an aquaplanet run.}
\label{fig:defn.of.inputs}
\end{figure}
\end{latexonly}

\begin{htmlonly}
\label{fig:defn.of.inputs}
\begin{center}
\htmlfigcaption{%
	\codeblockfont{%
inputs = \{\} \\
inputs['runname'] = 'test' \\
inputs['landon'] = 0 \\
inputs['year0'] = 1 \\
inputs['month0'] = 11 \\
inputs['day0'] = 1 \\
inputs['lastday'] = 30 \\
inputs['mrestart'] = 0 \\
inputs['init\_with\_instance\_state'] = True \\
inputs['compiled\_form'] = 'parts'}
	}

\htmlfigcaption{Figure~\ref{fig:defn.of.inputs}:
	The initial definition of the \vars{inputs} dictionary for 
	examples given in Section~\ref{sec:cookbook}.  These settings
	imply that a run session will start on November 1, Year 1,
	last for 30 days, and will be an aquaplanet run.}
\end{center}
\end{htmlonly}



\begin{description}
\item[Plain model run:]
	Here I just want to make a single model run.
	Tasks:  Instantiate a fresh model and execute a run session.
	The code to run the model is just:
	\begin{codeblock}
	\codeblockfont{%
inputs['init\_with\_instance\_state'] = False \\
model = Qtcm(**inputs) \\
model.run\_session()}
	\end{codeblock}
	where \vars{inputs} is initialized with the code in
	Figure~\ref{fig:defn.of.inputs}.


\item[Explore parameter space with a set of models:]
	Here I want to create an entire suite of separate models,
	in order to determine the sensitivity of the model to changing
	a parameter.
	To do this, I
	instantiate multiple fresh models, 
	and execute a run session for each instance, all within
	a \vars{for} loop:


%--- Two versions, because LaTeX2HTML does not correctly typeset
%    the hspace command:
\begin{latexonly}
	\begin{codeblock}
	\codeblockfont{%
import os \\
inputs['init\_with\_instance\_state'] = False \\
for i in xrange(0,1002,10): \\
\hspace*{5ex}iname = 'ziml-' + str(i) + 'm' \\
\hspace*{5ex}ipath = os.path.join('proc', iname) \\
\hspace*{5ex}os.makedirs(ipath) \\
\hspace*{5ex}model = Qtcm(**inputs) \\
\hspace*{5ex}model.ziml.value = float(i)  \\
\hspace*{5ex}model.runname.value = iname \\
\hspace*{5ex}model.outdir.value = ipath \\
\hspace*{5ex}model.run\_session() \\
\hspace*{5ex}del model}
	\end{codeblock}
\end{latexonly}

\begin{htmlonly}
\begin{center}
\htmlfigcaption{%
	\codeblockfont{%
import os \\
inputs['init\_with\_instance\_state'] = False \\
for i in xrange(0,1002,10): \\
\hspace*{5ex}iname = 'ziml-' + str(i) + 'm' \\
\hspace*{5ex}ipath = os.path.join('proc', iname) \\
\hspace*{5ex}os.makedirs(ipath) \\
\hspace*{5ex}model = Qtcm(**inputs) \\
\hspace*{5ex}model.ziml.value = float(i)  \\
\hspace*{5ex}model.runname.value = iname \\
\hspace*{5ex}model.outdir.value = ipath \\
\hspace*{5ex}model.run\_session() \\
\hspace*{5ex}del model}
	}
\end{center}
\end{htmlonly}


	The loop explores mixed-layer depth \vars{ziml} from 0~m to
        1000~m, in 10~m intervals.  I create the \vars{outdir}
	directory before every model call, since the compiled QTCM1 model
	requires the output directory exist, specifying the run name
	and output directory as the string \vars{iname}.
	The output directories are assumed to all be in the \fn{proc}
	sub-directory of the current working directory.
	\vars{inputs} is initialized with the code in
	Figure~\ref{fig:defn.of.inputs}.


\item[Conditionally explore parameter space:]
	Here I want to 
	conditionally explore the parameter space, on the basis of
	some mathematical criteria.
	To do this, I
	instantiate a model, evaluate results using
	that criteria, and run another fresh model depending on
	the results (passing the previous model state via a snapshot),
	all within a \vars{while} loop.
	Note that this type of investigation is very difficult to 
	automate if all you can use are shell scripts and
	Fortran.
	See Figure~\ref{fig:conditional.test.eg} for a detailed
	example.


\item[With interactive adjustments at run time:]
	The example in Figure~\ref{sec:continuation.intro}
	illustrates this type of run.  In this example,
	I instantiate a fresh model, execute a run session, analyze the
	output, change variables in the model instance, and then
	execute a continuation run session.


\item[Test alternative parameterizations:]
	I've already described how we can use run lists to arbitrarily
	change model execution order and content at run time.
	We can take advantage of Python's inheritance
	abilities, along with run lists, to simplify this.
	Figure~\ref{fig:alt.param.inherit.eg} provides an example of
	this use.

	Of course, you can use pre-processor directives and shell
	scripts to accomplish the same functionality seen in
	Figure~\ref{fig:alt.param.inherit.eg} using just Fortran.
	The Python solution, however, shortcuts the compile/linking
	step, and enables you to easily do run time swapping between
	subroutine choices based upon run time calculated
	tests (see Figure~\ref{fig:conditional.test.eg} for an
	example of such tests).
\end{description}




% --- Two versions of this block, one for display in PDF and the other
%     for display in HTML:
\begin{latexonly}
\begin{figure}[p]
	\begin{codeblock}
	\codeblockfont{%
\small
import os \\
import numpy as N \\
maxu1 = 0.0 \\
while maxu1 < 10.0: \\
\hspace*{5ex}iziml = 0.1 * maxu1 \\
\hspace*{5ex}iname = 'ziml-' + str(iziml) + 'm' \\
\hspace*{5ex}ipath = os.path.join('proc', iname) \\
\hspace*{5ex}os.makedirs(ipath) \\
\hspace*{5ex}model = Qtcm(**inputs) \\
\hspace*{5ex}try: \\
\hspace*{10ex}model.sync\_set\_py\_values\_to\_snapshot(snapshot=mysnapshot) \\
\hspace*{10ex}model.init\_with\_instance\_state = True \\
\hspace*{5ex}except: \\
\hspace*{10ex}model.init\_with\_instance\_state = False \\
\hspace*{5ex}model.ziml.value = iziml  \\
\hspace*{5ex}model.runname.value = iname \\
\hspace*{5ex}model.outdir.value = ipath \\
\hspace*{5ex}model.run\_session() \\
\hspace*{5ex}maxu1 = N.max(N.abs(model.u1.value)) \\
\hspace*{5ex}mysnapshot = model.snapshot \\
\hspace*{5ex}del model}
	\end{codeblock}

\caption{This code explores different values of
	mixed-layer depth \vars{ziml} for 30~day runs,
	as a function of maximum \vars{u1} magnitude,
	until it finds a case where the maximum \vars{u1} is
	greater than 10~m/s.  (The relationship between
	\vars{ziml} and the maximum of the speed of
	\vars{u1}, where 
	\vars{ziml\thinspace=\thinspace0.1\thinspace*\thinspace{maxu1}}, 
	is made up.)
	With each iteration, the new run uses the snapshot from
	a previous run to initialize (as well as the new value
	of \vars{ziml}); the \vars{try} statement is used to
	ensure the model works even if \vars{mysnapshot} is not
	defined (which is the case the first time around).
	The \vars{inputs} dictionary is initialized with the code in
	Figure~\ref{fig:defn.of.inputs}.}
\label{fig:conditional.test.eg}
\end{figure}
\end{latexonly}

\begin{htmlonly}
\label{fig:conditional.test.eg}
\begin{center}
\htmlfigcaption{%
	\codeblockfont{%
import os \\
import numpy as N \\
maxu1 = 0.0 \\
while maxu1 < 10.0: \\
\hspace*{5ex}iziml = 0.1 * maxu1 \\
\hspace*{5ex}iname = 'ziml-' + str(iziml) + 'm' \\
\hspace*{5ex}ipath = os.path.join('proc', iname) \\
\hspace*{5ex}os.makedirs(ipath) \\
\hspace*{5ex}model = Qtcm(**inputs) \\
\hspace*{5ex}try: \\
\hspace*{10ex}model.sync\_set\_py\_values\_to\_snapshot(snapshot=mysnapshot) \\
\hspace*{10ex}model.init\_with\_instance\_state = True \\
\hspace*{5ex}except: \\
\hspace*{10ex}model.init\_with\_instance\_state = False \\
\hspace*{5ex}model.ziml.value = iziml  \\
\hspace*{5ex}model.runname.value = iname \\
\hspace*{5ex}model.outdir.value = ipath \\
\hspace*{5ex}model.run\_session() \\
\hspace*{5ex}maxu1 = N.max(N.abs(model.u1.value)) \\
\hspace*{5ex}mysnapshot = model.snapshot \\
\hspace*{5ex}del model}
	}

\htmlfigcaption{Figure \ref{fig:conditional.test.eg}:
	This code explores different values of
	mixed-layer depth \vars{ziml} for 30~day runs,
	as a function of maximum \vars{u1} magnitude,
	until it finds a case where the maximum \vars{u1} is
	greater than 10~m/s.  (The relationship between
	\vars{ziml} and the maximum of the speed of
	\vars{u1}, where 
	\vars{ziml\thinspace=\thinspace0.1\thinspace*\thinspace{maxu1}}, 
	is made up.)
	With each iteration, the new run uses the snapshot from
	a previous run to initialize (as well as the new value
	of \vars{ziml}); the \vars{try} statement is used to
	ensure the model works even if \vars{mysnapshot} is not
	defined (which is the case the first time around).
	The \vars{inputs} dictionary is initialized with the code in
	Figure~\ref{fig:defn.of.inputs}.}
\end{center}
\end{htmlonly}


% --- Two versions of this block, one for display in PDF and the other
%     for display in HTML:
\begin{latexonly}
\begin{figure}[p]
\begin{center}
	\begin{codeblock}
	\codeblockfont{%
\small
import os \\
\\
class NewQtcm(Qtcm): \\
\hspace*{5ex}def cloud0(self):\\
\hspace*{10ex}[\ldots] \\
\hspace*{5ex}def cloud1(self):\\
\hspace*{10ex}[\ldots] \\
\hspace*{5ex}def cloud2(self):\\
\hspace*{10ex}[\ldots] \\
\hspace*{5ex}[\ldots] \\
\\
inputs['init\_with\_instance\_state'] = False \\
for i in xrange(10): \\
\hspace*{5ex}iname = 'cloudroutine-' + str(i)  \\
\hspace*{5ex}ipath = os.path.join('proc', iname) \\
\hspace*{5ex}os.makedirs(ipath) \\
\hspace*{5ex}model = NewQtcm(**inputs) \\
\hspace*{5ex}model.runlists['atm\_physics1'][1] = 'cloud' + str(i) \\
\hspace*{5ex}model.runname.value = iname \\
\hspace*{5ex}model.outdir.value = ipath \\
\hspace*{5ex}model.run\_session() \\
\hspace*{5ex}del model}
	\end{codeblock}
\end{center}

\caption{Let's say we have 9 different cloud physics schemes we wish
	to try out in 9 different runs.  The easiest way to do this
	is to create a new class \class{NewQtcm} that
	inherits everything from \class{Qtcm}, and to which we'll
	add the additional cloud schemes (\vars{cloud0}, \vars{cloud1},
	etc.).
	In the \vars{for} loop, I change the cloud model
	run list entry in the run list that governs
	atmospheric physics at one instant to whatever the cloud
	model is at this point in the loop.
	The \vars{inputs} dictionary is initialized with the code in
	Figure~\ref{fig:defn.of.inputs}.
	Of course, we could do the same thing by running the 9
	models separately, but this set-up makes it easy to do
	hypothesis testing with these 9 models.  For instance, we
	can create a test by which we will choose which of the 9
	models to use:  Within this framework, the selection of
	those models can be altered by changing a string.}
\label{fig:alt.param.inherit.eg}
\end{figure}
\end{latexonly}

\begin{htmlonly}
\label{fig:alt.param.inherit.eg}
\begin{center}
\htmlfigcaption{%
	\codeblockfont{%
import os \\
\\
class NewQtcm(Qtcm): \\
\hspace*{5ex}def cloud0(self):\\
\hspace*{10ex}[\ldots] \\
\hspace*{5ex}def cloud1(self):\\
\hspace*{10ex}[\ldots] \\
\hspace*{5ex}def cloud2(self):\\
\hspace*{10ex}[\ldots] \\
\hspace*{5ex}[\ldots] \\
\\
inputs['init\_with\_instance\_state'] = False \\
for i in xrange(10): \\
\hspace*{5ex}iname = 'cloudroutine-' + str(i)  \\
\hspace*{5ex}ipath = os.path.join('proc', iname) \\
\hspace*{5ex}os.makedirs(ipath) \\
\hspace*{5ex}model = NewQtcm(**inputs) \\
\hspace*{5ex}model.runlists['atm\_physics1'][1] = 'cloud' + str(i) \\
\hspace*{5ex}model.runname.value = iname \\
\hspace*{5ex}model.outdir.value = ipath \\
\hspace*{5ex}model.run\_session() \\
\hspace*{5ex}del model}
	}

\htmlfigcaption{Figure \ref{fig:alt.param.inherit.eg}:
	Let's say we have 9 different cloud physics schemes we wish
	to try out in 9 different runs.  The easiest way to do this
	is to create a new class \class{NewQtcm} that
	inherits everything from \class{Qtcm}, and to which we'll
	add the additional cloud schemes (\vars{cloud0}, \vars{cloud1},
	etc.).
	In the \vars{for} loop, I change the cloud model
	run list entry in the run list that governs
	atmospheric physics at one instant to whatever the cloud
	model is at this point in the loop.
	The \vars{inputs} dictionary is initialized with the code in
	Figure~\ref{fig:defn.of.inputs}.
	Of course, we could do the same thing by running the 9
	models separately, but this set-up makes it easy to do
	hypothesis testing with these 9 models.  For instance, we
	can create a test by which we will choose which of the 9
	models to use:  Within this framework, the selection of
	those models can be altered by changing a string.}
\end{center}
\end{htmlonly}




% ===== end of file =====


%@@@\chapter{Combining \code{qtcm} with \code{CliMT}}
%@@@% ==========================================================================
% CliMT
%
% By Johnny Lin
% ==========================================================================


% ------ BODY -----
%
\section{General Tutorial on CliMT}


General notes of things I think I may have observed about
\code{Parameters} objects:
\begin{itemize}
\item You can treat a \code{Parameters} instance as a dictionary, where
	the key is the name of the field, because \code{\_\_getitem\_\_},
	etc.\ have been defined for the instance.  However, the values,
	units, and long names of the fields are stored in dictionaries
	assigned to \code{value}, \code{units}, and \code{long\_name},
	keyed to the field name (a string).
\end{itemize}


General notes of things I think I may have observed about
\code{Components} objects:
\begin{itemize}
\item All variables and quantities, whether they be physical fields,
	filenames, or metadata,
	are stored as attributes in the \code{Components} instance.
\item \code{Components} have these special attributes:
        \code{Required},
        \code{Prognostic},
	and
        \code{Diagnostic},
	which are lists that contain the names of describe whether
\item Scalar parameters in \code{Component} objects
	are stored as an instance of the \code{Parameters}
	class, under the attribute \code{Params}.
\end{itemize}


General notes of things I think I may have observed about
\code{Federation} objects:
\begin{itemize}
\item \code{Federation} objects hold the \code{Components} instances
	in a list assigned to the attribute \code{list}.
\item \code{Federation} attributes
        \code{Required},
        and
	\code{Prognostic},
	are unions of the same attributes of the constituent
	\code{Components} objects.
\end{itemize}






% ===== end of file =====


\chapter{Troubleshooting}                   \label{ch:trouble}
% ==========================================================================
% Troubleshooting
%
% By Johnny Lin
% ==========================================================================


% ------ BODY -----
%
\section{Error Messages Produced by \mods{qtcm}}

\begin{description}
\item[\screen{Error-Value too long in SetbyPy module getitem\_str for}
	\dumarg{key}:]
	This message is produced by the Fortran
	subroutine \mods{getitem\_str}
	in the module \mods{SetbyPy} in the compiled QTCM1 Fortran code.
	The code is in the file \fn{setbypy.F90}.  This error occurs when
	the Fortran variable whose name is given by the string \dumarg{key}
	has a value that is greater than the local parameter
	\vars{maxitemlen} in \mods{getitem\_str}.  To fix this, you have
	to go into \fn{setbypy.F90} and change the value of
	\vars{maxitemlen}.

\item[\screen{Error-real\_rank1\_array should be deallocated}:]
	Fortran module \mods{SetByPy}'s subroutine
	\mods{getitem\_real\_array} generates this message
	(or a similar message for other ranks) if the Fortran
	variable for the input \dumarg{key} are allocated on entry
	to the routine.  This may indicate the user has written another
	Fortran routine to access the \mods{real\_rank1\_array} variable
	outside of the standard interfaces..

\item[\screen{Error-Bad call to SetbyPy module \ldots}:]
	Often times, this error occurs because a get or set routine
	in \mods{SetByPy} tried to act on a variable for which the
	corresponding input \dumarg{key} is not defined.  The solution
	is to add that case in the if/then construct for the get and set
	routines in \mods{SetByPy} and rebuild the extension modules.
\end{description}


\section{Other Errors}

\begin{description}
\item[Python cannot find some packages:]
	This error often happens when the version of Python in which
	you have installed all your packages is not the version that
	is called at the Unix command line by typing in \cmd{python}.
	To get around this, 
        define a Unix alias
        that maps \cmd{python2.4} (or whichever version of Python
	has all your packages installed) to \cmd{python}.  If you
	have multiple Python's installed on your system, you might
	have to use a more specific name for the Python executable.
	As a result, you may have to change the test scripts in
	\fn{test} in the \mods{qtcm} distribution directory.

\item[\mods{get\_qtcm1\_item} and compiled QTCM1 model pointer
	variables:]
	If you try to use the \mods{get\_qtcm1\_item} method on a compiled
	QTCM1 model pointer variable 
	(i.e., \vars{u1}, \vars{v1}, \vars{q1}, \vars{T1}),
	 before the compiled
	model \mods{varinit} subroutine is run, you'll get a bus error
	with no additional message.

\item[Mismatch between Python and Fortran array field variables:]
	You change an array field variable on the Python side, but
	it seems like the wrong elements are changed on the Fortran
	side.  Or you type in the same index address for accessing a
	\mods{qtcm} netCDF output array as well as its \class{Qtcm}
	instance attribute counterpart, and find you get different
	answers.  Some possible reasons and fixes:

	\begin{itemize}
	\item This will occur if you haven't accounted for the
		difference in how field variables are saved at the
		Python-level, Fortran-level, and in a netCDF file.
		All netCDF array output is dimensioned (time,
		latitude, longitude) when read into Python using
		the \mods{Scientific} package.  This differs from
		the way \class{Qtcm} saves field variables, \emph{both}
		at the Python- and Fortran-levels, which follows
		Fortran convention (longitude, latitude).

		Note that the way \class{Qtcm} saves field variables
		at the Python- and Fortran-levels is different than
		the default way Python and Fortran save arrays.
		Section~\ref{sec:field.var.shape} for more information.

	\item You may have forgotten that array indices in Python start at
		0, while indices in Fortran (generally) start at 1.
		Also, ranges in Python are exclusive at the upper-bound,
		while ranges in Fortran are inclusive at the upper-bound.
		(Both Python and Fortran array indice ranges are inclusive
		at the lower-bound.)

	\item You may have forgotten some field variables have
		ghost latitudes, and thus there are extra latitude bands
		when the array is stored as a Python or Fortran field
		variable, but there are \emph{no} extra latitude bands
		when the array is stored as netCDF output (the QTCM1
		output routines strip off the ghost latitudes when
		writing those field variables out).
	        See the
        \latexhtml{%
\htmladdnormallinkfoot{QTCM1 manual}%
        {http://www.atmos.ucla.edu/$\sim$csi/qtcm\_man/v2.3/qtcm\_manv2.3.pdf}}%
{\htmladdnormallink{QTCM1 manual}%
        {http://www.atmos.ucla.edu/~csi/qtcm_man/v2.3/qtcm_manv2.3.pdf}}
        \cite{Neelin/etal:2002}
        for details about ghost latitudes.

		The safest and easiest way to tell whether the variable has a
		ghost latitudes is to look at its shape.
		A call to the \class{Qtcm} instance
		method \mods{get\_qtcm1\_item} will give you the array,
		and the use of NumPy's \mods{shape} function will give you
		the shape.
	\end{itemize}
\end{description}




% ===== end of file =====


\chapter{Developer Notes}                   \label{ch:devnotes}
% ==========================================================================
% Using QTCM
%
% By Johnny Lin
% ==========================================================================


% ------ BODY -----
%

%---------------------------------------------------------------------
\section{Introduction}

This section describes programming practices and issues related to
the \mods{qtcm} package that might be of interest to users wishing
to add/change code in the package.
Please see the package
\latexhtml{API documentation,%
		\footnote{http://www.johnny-lin.com/py\_pkgs/qtcm/doc/html-api/}
		which includes the source code}%
        {\htmladdnormallink{API documentation}%
		{http://www.johnny-lin.com/py\_pkgs/qtcm/doc/html-api/},
		which includes the source code},
for details.




%---------------------------------------------------------------------
\section{Changes to QTCM1 Fortran Files}  \label{sec:f90changes}

The source code used to generate the shared object files used
in this Python package is unchanged
from the pure-Fortran QTCM1 model source code, except in the
following ways:

\begin{itemize}
\item The suffix of all source code files 
	has been changed from \fn{.f90} to \fn{.F90}, 
	in order to ensure the compiler preprocesses 
	the source code.  Some compilers use the capitalization to
	tell whether or not to run the code through a preprocessor.

\item In all \fn{.F90} files, occurrences of:
	\begin{codeblock}
	\codeblockfont{%
	Character(len=130)}
	\end{codeblock}
	are changed to:
	\begin{codeblock}
	\codeblockfont{%
	Character(len=305)}
	\end{codeblock}
	This enables the model to properly deal with longer filenames.
	The number ``305'' is chosen to make search and replace easier.

\item In \fn{qtcmpar.F90}, the 
	\vars{eps\_c} variable is changed from an unchangable
	parameter to a changeable real, 
	so that it can be changed in the model at runtime.

\item All occurrences of an underscore (``\_'') character in a
	subroutine or function name are removed.  The
	presence of the underscore messes up the dynamic lookup
	mechanism for the \mods{f2py} generated extension module.
	The following names are changed, both in subroutine definitions
	and calls:
	\begin{itemize}
	\item \mods{out\_restart} to \mods{outrestart},
	\item \mods{save\_bartr} to \mods{savebartr},
	\item \mods{grad\_phis} to \mods{gradphis}.
	\end{itemize}

\item \fn{driver.F90} is changed so that program
	\mods{driver} becomes a subroutine, and 
	subroutine \mods{driverinit} is deleted (along with
	all calls to it) because basic model initialization is
	handled at the Python level.

\item In \fn{clrad.F90}, subroutine \mods{cloud}, the first
	\vars{COUNTCAP} preprocessor macro, a comment line for
	that ifdef is moved to prevent a warning message during
	building with \mods{f2py}.

\item The order of subroutine \mods{qtcminit} is changed.  The original
	pure-Fortran QTCM1 \mods{qtcminit} code has the following
	calling sequence:

	\begin{codeblock}
        \codeblockfont{%
Call parinit            !Initialize model parameters \\
Call varinit            !Initialize variables \\
Call TimeManager(1)     !mm set model time \\
Call bndinit            !input boundary datasets \\
Call physics1           !diagnostic fields for initial condition}
	\end{codeblock}

	For the \mods{qtcm} package, I've altered this order so
	\mods{bndinit} comes after \mods{parinit} but before \mods{varinit}:
	\begin{codeblock}
        \codeblockfont{%
Call parinit            !Initialize model parameters \\
Call bndinit            !input boundary datasets \\
Call varinit            !Initialize variables \\
Call TimeManager(1)     !mm set model time  \\
Call physics1           !diagnostic fields for initial condition}
	\end{codeblock}

	This is done because \vars{STYPE} is not read in for the
	\vars{landon} \vars{True} case until \mods{bndinit}, but
	in \mods{varinit} \vars{STYPE} is used to calculate the
	original values of \vars{WD} for the non-restart case.  This
	also corrects the conflicting order found in the pure-Fortran
	QTCM1 manual (compare pp.\ 29 and 32).  As far as I can
	tell, \mods{bndinit} has no dependencies that require it
	to come after \mods{timemanager} or \mods{varinit}.

\end{itemize}

In addition, the Fortran files \fn{setbypy.F90}, \fn{wrapcall.F90},
and \fn{varptrinit.F90} are added.  The routines in these files, 
however, just add more flexibility and functionality to the model;
they do not automatically affect any model computations.  See
Section~\ref{sec:newf90} for details.




%---------------------------------------------------------------------
\section{New Interfaces and Fortran Functionality}  \label{sec:newf90}

As described in Section~\ref{sec:f90changes}, the Fortran files
\fn{setbypy.F90}, \fn{wrapcall.F90}, and \fn{varptrinit.F90} are
added to the QTCM1 source directory.  The first two files define the Fortran
90 modules (\mods{SetbyPy} and \mods{WrapCall}) needed to interface
the Python and Fortran levels.  The last file defines a new Fortran
subroutine \mods{varptrinit} that associates QTCM1 model pointer
variables at the Fortran level.  In a pure-Fortran run of QTCM1,
this occurs in subroutine \mods{varinit}; for a
\vars{compiled\_form\thinspace=\thinspace'parts'} run, since the
functionality of the Fortran \mods{varinit} is now in the Python
\mods{varinit} method, a separate Fortran pointer association
subroutine needed to be defined.  The Fortran subroutine \mods{varptrinit}
is called as the \mods{varptrinit} function of the 
\vars{compiled\_form\thinspace=\thinspace'parts'}
\fn{.so} extension module.


	\subsection{Fortran Module \mods{SetbyPy}}   \label{sec:setbypy}

		\subsubsection{Design Description}

This module defines functions and subroutines used to read variables
from the Fortran-level to the Python-level, and in setting Fortran-level
variables using the Python-level values.  These routines are used
by \class{Qtcm} methods \mods{get\_qtcm1\_item} and \mods{set\_qtcm1\_item}
(and dependencies thereof) to ``get'' and ``set'' the Fortran-level
variables.  Note that the Fortran module \mods{SetbyPy} is referred
to in lowercase at the Python level, i.e., as the
attribute \vars{\_\_.qtcm.setbypy} of a \class{Qtcm} instance.

Because Fortran variables are not dynamically typed, separate Fortran
functions and subroutines need to be defined to get and set variables
of different types.\footnote%
	{The \mods{interface} construct in Fortran 90 is supposed to
	provide a way to create a single interface to multiple
	routines, e.g.:
	\begin{codeblock}
	\codeblockfont{%
Interface setitem \\
\hspace*{3ex}Module Procedure setitem\_real, setitem\_int, setitem\_str \\
End Interface}
	\end{codeblock}
	This construct, however, causes a bus error
	(Mac OS X 10.4, Intel).  Thus, I put the
	same functionality in the Python code.}
The \class{Qtcm} methods \mods{get\_qtcm1\_item}
and \mods{set\_qtcm1\_item} know which one of the Fortran routines
to call on the basis of the type and rank of the value for the field
variable in the \mods{defaults} submodule.  This is why all field
variables need to have defaults defined in \mods{defaults}.  For
array variables, the field variable defaults also provide the rank
of the Fortran-level variable being gotten or set.  However, the
array default values do \emph{not} have to have the same shape as
the Fortran-level variables; on the Python-side, variable shape
adjusts to what is declared on the Fortran-side.  
Thus, if you change the resolution of
the compiled QTCM1 model, you do not have to change the dimensions
of the field variable values in \mods{defaults}.

The \class{Qtcm} method \mods{get\_qtcm1\_item} directly calls
the \mods{SetByPy} routines.
The \class{Qtcm} method \mods{set\_qtcm1\_item} makes use of
private instance methods that make the calls to the \mods{SetByPy} routines.

For scalar field variables, \mods{SetByPy} provides functions and
subroutines that provide the value of the variable on output.
For array field variables, \mods{SetByPy}
dynamic \emph{module} arrays are used to pass array
variables in and out; I could not get the 
\mods{SetByPy} Fortran routines to set
locally defined dynamic arrays (that is, locally within a function or
subroutine).\footnote%
	{I tried to implement Fortran subroutine
	\mods{getitem\_real\_array} using traditional array 
	dimension passing 
	(e.g., \code{subroutine foo(nx, ny, a)}) as well
	as declaring the allocatable array inside the subroutine, 
	but neither option worked on my \mods{f2py} (version 2\_3816) 
	and Python (version 2.4.3).}
In the \mods{SetByPy} module, these dynamic arrays
are defined as follows:

\begin{codeblock}
\codeblockfont{%
Real, allocatable, dimension(:) :: real\_rank1\_array \\
Real, allocatable, dimension(:,:) :: real\_rank2\_array \\
Real, allocatable, dimension(:,:,:) :: real\_rank3\_array}
\end{codeblock}

For all field variables, scalar or array, the \mods{SetByPy} module
has a fourth module variable defined, \vars{is\_readable}, that the
Fortran get and set routines will set to \vars{.TRUE.} if the
variable is readable and \vars{.FALSE.} if not (it's declared as a
logical variable).  This Fortran variable can be used to prevent
Python from accessing pointer variables that aren't yet associated
to targets.

In general, \mods{SetByPy} routines make use of Fortran constructs
to enable them to accomodate all possible
variables of a given type and shape.  However, 
for string scalars, the \mods{SetByPy} function \mods{getitem\_str}
has to have a return value of a predefined length, in order to
work properly.  That length is given by the parameter
\vars{maxitemlen} and is set to 505 (the value is chosen to
be larger than all filename variables described in
Section~\ref{sec:f90changes} and to be easily found in
the \fn{.F90} files).


		\subsubsection{Module Structure}

If you're a Fortran programmer, you can probably get all the information
in this section from just reading the \fn{setbypy.F90} file directly.
This description of the module structure, however, permits me to highlight
what you need to do if you want to make additional compiled QTCM1 variables
accessible to Python \class{Qtcm} objects.

\begin{itemize}
\item All \mods{Use} statements are given in the beginning of 
	the \mods{SetByPy} module.  These statements cover
	nearly all of the QTCM1 Fortran
	modules that contain variables of interest.  If the
	QTCM1 variable you're interested in isn't in a module
	listed here, you'll have to add your own
	\mods{Use} statement of that module here.

\item Next comes the definitions for the
	\vars{real\_rank1\_array},
	\vars{real\_rank2\_array}, and
	\vars{real\_rank3\_array} dynamic array variables, and
	the \vars{is\_readable} boolean variable.

\item The \mods{Contains} block of the module defines the module
	routines called by the \class{Qtcm} instance methods to
	set and get the compiled QTCM1 model variables.  The
	routines are:
	\begin{itemize}
	\item Function \mods{getitem\_real}
	\item Subroutine \mods{getitem\_real\_array}
	\item Function \mods{getitem\_int}
	\item Function \mods{getitem\_str}
	\item Subroutine \mods{setitem\_real}
	\item Subroutine \mods{setitem\_real\_array}
	\item Subroutine \mods{setitem\_int}
	\item Subroutine \mods{setitem\_str}
	\end{itemize}

\end{itemize}

Each of the routines in the module \mods{Contains} block is essentially
a list of \mods{if}/\mods{elseif} statements.  The list tests for the
name of the variable of interest (a string), and gets or sets the
compiled QTCM1 model variable corresponding to that name.  For pointer
array variables, a test is also made as to whether or not the variable
has been associated.  If not, the variable is not readable
and \vars{is\_readable} is set to \vars{.FALSE.}\ accordingly.

If you wish to add another compiled QTCM1 model variable to be
accessible to \class{Qtcm} instance methods \mods{get\_qtcm1\_item}
and \mods{set\_qtcm1\_item}, just add another \mods{if}/\mods{else\-if},
like the other \mods{if}/\mods{elseif} blocks, in the Fortran set
and get routines corresponding to the QTCM1 variable type (scalar
vs.\ array, and real, integer, or string).  On the Python side, add
an entry in \mods{defaults} corresponding to the new field variable
you've created access to.  I would strongly recommend making the
Python name of your new field variable
(given in \mods{defaults}) be the same as the compiled
QTCM1 model variable name.



	\subsection{Fortran Module \mods{WrapCall}}   \label{sec:wrapcall}

Most of the time, if you want to call a compiled QTCM1 model subroutine
from the Python level, you will use the version of the subroutine that
is found in this Fortran module.  
Note that the Fortran module \mods{WrapCall} is referred
to in lowercase at the Python level, i.e., as the
attribute \vars{\_\_.qtcm.wrapcall} of a \class{Qtcm} instance.

All the routines in this module do is wrap one of the compiled QTCM1
model routines.  For instance, \mods{WrapCall} subroutine
\mods{wadvcttq} is defined as just:

% --- Two versions of this block, one for display in PDF and the other
%     for display in HTML:
%
\begin{latexonly}
\begin{codeblock}
\codeblockfont{%
Subroutine wadvcttq \\
\hspace*{3ex}Call advcttq \\
End Subroutine wadvcttq}
\end{codeblock}
\end{latexonly}

\begin{htmlonly}
\begin{rawhtml}
<p><code><font color="blue">Subroutine wadvcttq<br>
&nbsp;&nbsp;&nbsp;Call advcttq<br>
End Subroutine wadvcttq</font></code></p>
\end{rawhtml}
\end{htmlonly}

All subroutines in this module begin with ``w'', with the rest of
the name being the Fortran QTCM1 subroutine name.  The calling
interface for the ``w'' version is the same as the Fortran QTCM1
original version.  There are no subroutines in this module that do
not have an exact counterpart in the Fortran QTCM1 code, and thus
this module's subroutines sole purpose is to call other subroutines
in the compiled QTCM1 model.

These wrapper routines are needed because \mods{f2py}, for some
reason I can't figure out, will not properly wrap Fortran routines
(that are then callable at the Python level) that create local
arrays using parameters obtained through a Fortran \mods{use}
statment.  Thus, as an example, a Fortran subroutine \mods{foo}
with the following definition:

% --- Two versions of this block, one for display in PDF and the other
%     for display in HTML:
%
\begin{latexonly}
\begin{codeblock}
\codeblockfont{%
subroutine foo \\
\hspace*{3ex}use dimensions \\
\hspace*{3ex}real a(nx,ny) \\
\hspace*{3ex}[\ldots] \\
end subroutine foo}
\end{codeblock}
\end{latexonly}

\begin{htmlonly}
\begin{rawhtml}
<p><code><font color="blue">
subroutine foo<br>
&nbsp;&nbsp;&nbsp;use dimensions<br>
&nbsp;&nbsp;&nbsp;real a(nx,ny)<br>
&nbsp;&nbsp;&nbsp;[\ldots]<br>
end subroutine foo
</font></code></p>
\end{rawhtml}
\end{htmlonly}


where \vars{nx} and \vars{ny} are defined in the module vars{dimensions},
will return an error, with the result that the extension module
will not be created, or an extension modules that yields output
that is different from running the pure-Fortran version of QTCM1.

By wrapping these calls into this file, I also avoid having to
separate out the Fortran QTCM1 subroutines into separate \fn{.F90}
files.  For Fortran subroutines that you want callable from the
Python level, \mods{f2py} seems to require each Fortran subroutine
to be in its own file of the same name (e.g., the version of
\fn{driver.F90} for this package). If several Fortran subroutines
are all found in a single \fn{.F90} files, \mods{f2py} seems unable
to create wrappers for those subroutines.




%---------------------------------------------------------------------
\section{Python \mods{qtcm} and Pure-Fortran QTCM1 Differences}

This section describes differences between how the \mods{qtcm}
package and the pure-Fortran QTCM1 assign some varables.  A list
of changes to the QTCM1 Fortran Files for use in the \mods{qtcm}
package is found in Section~\ref{sec:f90changes}.


	\subsection{QTCM1 \mods{driverinit}}   \label{sec:driverinit.diffs}

In the pure-Fortran version of QTCM1, by default, the following variables are
set by reference (as given below), not by value, in the \mods{driverinit}
routine:\footnote%
	{In the pure-Fortran version of QTCM1, this routine is found
	in \fn{driver.F90}.}
\begin{codeblock}
\codeblockfont{%
lastday\thinspace=\thinspace{daysperyear} \\
viscxu0\thinspace=\thinspace{viscU} \\
viscyu0\thinspace=\thinspace{viscU} \\
visc4x\thinspace=\thinspace{viscU} \\
visc4y\thinspace=\thinspace{viscU} \\
viscxu1\thinspace=\thinspace{viscU} \\
viscyu1\thinspace=\thinspace{viscU} \\
viscxT\thinspace=\thinspace{viscT} \\
viscyT\thinspace=\thinspace{viscT} \\
viscxq\thinspace=\thinspace{viscQ} \\
viscyq\thinspace=\thinspace{viscQ}}
\end{codeblock}

Thus, in pure-Fortran QTCM1, if you change \vars{daysperyear},
\vars{viscU}, etc.
and recompile (as needed), you will automatically change 
\vars{lastday}, \vars{viscxu0}, etc.
(Though, in the pure-Fortran QTCM1, the default values may be overwritten by
namelist input values.)

The \mods{driverinit} routine is eliminated
in the Python \code{qtcm} package.  Instead, inital values 
of field variables are specified in the \mods{defaults} submodule
and set by value to attributes of the \code{Qtcm} instance.
Thus, for instance, in a \class{Qtcm} instance, \code{lastday} 
is set to \code{365} by default, not to some variable
\vars{daysperyear}.  For the diffusion and viscosity terms,
the \class{Qtcm} instance attributes corresponding to those
terms are set to literals.\footnote%
	{Those literals are defined by \mods{defaults} private
	module variables \vars{\_\_viscT}, \vars{\_\_viscQ},
	and \vars{\_\_viscU}.}

In contrast, in the pure-Fortran QTCM1,
\mods{driverinit} declares local
variables \code{viscU}, \code{viscT}, and \code{viscQ},
and reads values into those variables via the input namelist.
Those values are then used to set
\vars{viscxu0}, \vars{viscyu0}, etc., as described above.
In pure-Fortran QTCM1, \code{viscU}, \code{viscT}, and \code{viscQ}
are not directly accessed anywhere else in the model.
Thus, \code{viscU}, \code{viscT}, and \code{viscQ} are not
defined as field variables in the \code{qtcm} package, and
\class{Qtcm} instances do not have attributes corresponding
to those names.
Additionally, if you wish to change a viscosity parameter
\vars{visc*} (given above), the parameter for each direction
must be set one-by-one even if the flow is isotropic.


	\subsection{The \mods{varinit} Routine}

One of the functions of the pure-Fortran QTCM1 \mods{varinit}
subroutine is to associate the pointer variables \vars{u1}, \vars{v1},
\vars{q1}, and \vars{T1}.  For the extension modules in the \mods{qtcm}
package, a Fortran subroutine \mods{varptrinit} is added that can
also do this association.  This subroutine is called in the
\class{Qtcm} instance method
\latexhtml{\mods{varinit}%
		\footnote{http://www.johnny-lin.com/py\_docs/qtcm/doc/html-api/qtcm.qtcm.Qtcm-class.html\#varinit}}%
	{\htmladdnormallink{\mods{varinit}}{http://www.johnny-lin.com/py_docs/qtcm/doc/html-api/qtcm.qtcm.Qtcm-class.html#varinit}}
(which duplicates and
extends the function of its pure-Fortran counterpart, enabling
alternative ways of handling restart).

The \mods{varptrinit} is not accessed via \mods{wrapcall}.  Remember
that \mods{wrapcall} contains only those routines that were in the
original pure-Fortran QTCM1 code, and that we want to have access
to at the Python level.


	\subsection{The \mods{qtcm} Method of \class{Qtcm}}

The \class{Qtcm} method \mods{qtcm} duplicates the functionality
of the \mods{qtcm} subroutine in the pure-Fortran QTCM1 model.
There are a few differences, however.  First, the \mods{qtcm} method
for \class{Qtcm} instances does not include a call to \mods{cplmean},
which uses mean surface flux for air-sea coupling.  This state is
consistent with the pure-Fortran QTCM1 pre-processor macro
\vars{CPLMEAN} being off.  Thus, if you wish to use mean surface
flux for air-sea coupling, you will have to revise the \mods{qtcm}
method of \class{Qtcm} to call \mods{cplmean}.  You'll also have to
check for any other code additions needed that are associated with
the \vars{CPLMEAN} macro.

Second, the \mods{qtcm} method for \class{Qtcm} instances does not
include the option of not using the atmospheric boundary layer
model.  This is consistent with macro \vars{NO\_ABL} being off.  If
you wish to have no atmospheric boundary layer model, change the
run list \vars{atm\_bartr\_mode} so that the \mods{wsavebartr} and
\mods{wgradphis} routines are not called.  You'll also have to check
for any other code additions needed that are associated with the
\vars{NO\_ABL} macro.



	\subsection{Miscellaneous Differences}

\begin{itemize}
\item In Python \class{Qtcm} instances,
	\vars{dateofmodel} is set to 0 by default.  
	In contrast, in the compiled QTCM1 model,
	the default (i.e., initial value) is calculated from 
	\vars{day0}, \vars{month0}, and \vars{year0}.
	See Section~\ref{sec:init.compiledform.full} for details.

\item The \class{Qtcm} instance attribute
	\vars{\_\_qtcm} is not copyable using \mods{copy.deepcopy}.

\item In general, when executing a \class{Qtcm} instance method, 
	if you change a \class{Qtcm} instance attribute 
	that has a counterpart in the compiled QTCM1 model,
	the compiled QTCM1 counterpart is not changed until the
	end of the method.  Likewise, if you call a compiled QTCM1 model
	subroutine and change a compiled QTCM1 model variable with
	a \class{Qtcm} instance counterpart, the \class{Qtcm}
	instance counterpart is not changed until the end of the
	subroutine.

\item In general, even though some of the compiled QTCM1 model
	Fortran subroutines/functions have counterparts in \class{Qtcm}
	that duplicate the former's functionality, the Fortran
	versions are kept intact so that the
	\vars{compiled\_form\thinspace=\thinspace'full'} case will work.
\end{itemize}




%---------------------------------------------------------------------
\section{Considerations When Adding Fortran Code}

In this section I describe issues to consider if you wish to add
your own compiled code to the package as separate extension modules.
(This is different from creating new standard extension modules,
which is described in Section~\ref{sec:create.new.so}.):

\begin{itemize}
\item The \class{Qtcm} class assumes that the directory path 
	to the original shared object file is the same as for the 
	\mods{package\_version} module.

\item If you want to be able to pass other Fortran variables 
	in and out to/from Python, please see the 
	Section~\ref{sec:setbypy}
	discussion of the Fotran \mods{SetByPy} module.

\item Fortran and Python routines to get and set compiled QTCM1 model
	arrays are currently written only for floating point array.

\item If you ever change 
	\class{Qtcm} instance method
	\mods{\_set\_qtcm\_array\_item\_in\_model}
	to work with non-floating point values, you will also
	have to change the array handling section in 
	\mods{set\_qtcm1\_item}.

\item The restart mechanism in the pure-Fortran QTCM1 model is 
	\emph{not} bit-for-bit correct.  Thus, if you compare the final
	output from a 40 day run with a 30 day run restarted from
	a 10 day run, the output will not be the same.
	This behavior has been duplicated in \class{Qtcm} 
	instances when the \vars{mrestart} flag is used
	and applicable.

\item When creating new extension modules using the \fn{src} makefile,
	be sure you first use the \cmd{make clean} command to clean-up
	any old files.

\end{itemize}




%---------------------------------------------------------------------
\section{Creating New Standard Extension Modules}   \label{sec:create.new.so}

The steps involved in creating the standard extension modules (e.g.,
\fn{\_qtcm\_full\_365.so}, etc.) on installation are given in
Section~\ref{sec:create.so}.  The makefile provided in \fn{/buildpath/src}
uses a Fortran compiler to create the object code, runs \mods{f2py}
to create the shared object file in \fn{src}, and moves the shared
object files into \fn{../lib}, overwriting any pre-existing files
of the same name.  In this section, I describe the makefile and
\mods{f2py} in a little more detail, in case you wish to create
standard extension modules with additions from the ones the default
makefile creates.


	\subsection{Makefile Rules}    \label{sec:makefile.rules}

This section describes the rules of the
makefile found in the \fn{src} directory
of the \mods{qtcm} distribution.  
This makefile is used by the Python package to create the extension
module (\fn{.so} files) imported and used by \mods{qtcm} objects
(as described in Section~\ref{sec:create.so}).
The makefile will, in general, be used only during \mods{qtcm}
installation, but if you wish to recompile the QTCM1 libraries
and make changes in the Python extension module,
you'll want to use/change this makefile.

\begin{description}
\item[clean] Removes old files in preparation for compiling new
	extension modules.

\item[libqtcm.a] Creates library \fn{libqtcm.a} that contains all
	QTCM1 object files in the directory \fn{src},, except
	\fn{setbypy.o}, \fn{wrapcall.o}, \fn{varptrinit.o}, and
	\fn{driver.o}.  This archive is compiled with the netCDF
	libraries.  Previous versions of \fn{libqtcm.a} are overwritten.

\item[\_qtcm\_full\_365.so] Creates the extension module
	\fn{\_qtcm\_full\_365.so}.  \mods{f2py} is run on applicable code
	in \fn{src}, and the extension module is moved to \fn{../lib}.
	Any previous versions of \fn{../lib/\_qtcm\_full\_365.so}
	are overwritten.

\item[\_qtcm\_parts\_365.so] Creates the extension module
	\fn{\_qtcm\_parts\_365.so}.  \mods{f2py} is run on applicable code
	in \fn{src}, and the extension module is moved to \fn{../lib}.
	Any previous versions of \fn{../lib/\_qtcm\_parts\_365.so}
	are overwritten.

\end{description}



	\subsection{Using \mods{f2py}}      \label{sec:using.f2py}

This section briefly describes how \mods{f2py} is used in the
makefile during the creation of the extension modules.
\htmladdnormallink{\mods{F2py}}{http://cens.ioc.ee/projects/f2py2e/} is a
program that generates shared object libraries that allow you to call
Fortran routines in Python.  \mods{F2py} comes with Python's
\htmladdnormallink{NumPy}{http://numpy.scipy.org/}
array handling package, so you do not need to install anything
extra if you have NumPy already installed.

To create the extension modules in \mods{qtcm} using
the makefile described in Section~\ref{sec:makefile.rules},
I use a method similar to the
\latexhtml{``Quick and Smart Way,''\footnote%
{http://cens.ioc.ee/projects/f2py2e/usersguide/index.html\#the-quick-and-smart-way}}%
{\htmladdnormallink{``Quick and Smart Way''}%
{http://cens.ioc.ee/projects/f2py2e/usersguide/index.html#the-quick-and-smart-way}}
described in the \mods{f2py} manual.
For the \fn{\_qtcm\_full\_365.so} extension module, the 
\mods{f2py} call is:

\begin{codeblock}
\codeblockfont{%
f2py --fcompiler=\$(FC) -c -m \_qtcm\_full\_365 driver.F90 $\backslash$ \\
\hspace*{10ex}setbypy.F90 libqtcm.a \$(NCLIB)}
\end{codeblock}

and for the \fn{\_qtcm\_parts\_365.so} extension module, the call is:

\begin{codeblock}
\codeblockfont{%
f2py --fcompiler=\$(FC) -c -m \_qtcm\_parts\_365 $\backslash$ \\
\hspace*{10ex}varptrinit.F90 wrapcall.F90 setbypy.F90 $\backslash$ \\
\hspace*{10ex}libqtcm.a \$(NCLIB)}
\end{codeblock}

For both calls, \vars{FC} and \vars{NCLIB} are the environment
variables in the makefile specifying the Fortran compiler and netCDF
libraries, respectively.  The \vars{-m} flag specifies the extension
module name (without the \fn{.so} suffix).  The \fn{.F90} files
specify the files that have modules and routines that will be
accessible at the extension module level, and the rest of the Fortran
files in QTCM1 are compiled and archived in a library \fn{libqtcm.a}.
For \mods{f2py} to work properly,
the \fn{.F90} files may define \emph{only one} module or routine.

If you add Fortran files containing new modules, and you wish those
modules to be accessible at the Python level, compile your new code
with \mods{f2py}.  If we have a file of such new code, \fn{newcode.F90},
the \mods{f2py} call to create the \fn{\_qtcm\_parts\_365.so}
extension module will become:

\begin{codeblock}
\codeblockfont{%
f2py --fcompiler=\$(FC) -c -m \_qtcm\_parts\_365 $\backslash$ \\
\hspace*{10ex}varptrinit.F90 wrapcall.F90 setbypy.F90 $\backslash$ \\
\hspace*{10ex}newcode.F90 $\backslash$ \\
\hspace*{10ex}libqtcm.a \$(NCLIB)}
\end{codeblock}

If you write new Fortran code for the compiled QTCM1 model that
will \emph{not} be accessed from the Python-level, just add the
object code filename to the variable \vars{QTCMOBJS} in the
makefile; you don't have to do anything else.  If you are adding
Fortran code to existing Fortran modules, it's even easier:  You
don't need change the makefile.  Note that for 64 bit processor
machines, you may have to use \mods{f2py} with the \cmd{-fPIC} flag;
see Section~\ref{sec:sopic} for details on how the lines above will
change.


	\subsection{Two Examples}

\emphpara{A Function:}
Let's say you have written a piece of Fortran code called
\fn{myfunction.F90} that contains one function called
\mods{myfunction}, and you want to have this function
callable from the Python level through the \class{Qtcm} 
instance method \mods{\_\_qtcm.myfunction}.  Do the following:

\begin{enumerate}
\item Move \fn{myfunction.F90} to \fn{src} in the \mods{qtcm}
	distribution directory \fn{/buildpath}.

\item Add \cmd{myfunction.o} to the end of the object file list lines
	after the target names
	\vars{\_qtcm\_full\_365.so} and
	\vars{\_qtcm\_parts\_365.so}.

\item In the
	\vars{\_qtcm\_full\_365.so} and
	\vars{\_qtcm\_parts\_365.so} target descriptions,
	add \cmd{myfunction.F90} to the 
	beginning of the list of \fn{.F90} names 
	in the \mods{f2py} lines.
\end{enumerate}


\emphpara{A Module:} 
Let's say you have written a piece of Fortran code called
\fn{mymodule.F90} that contains the Fortran module \mods{MyModule}
containing multiple routines and variables.  You want to have those
routines and variables callable from the Python level through the
\class{Qtcm} instance attribute \mods{\_\_qtcm.mymodule}.  The steps
to add \mods{MyModule} to the extension modules are exactly the
same as for a single function, with \cmd{mymodule} being
substituted in the makefile everywhere you have \cmd{myfunction}.




%---------------------------------------------------------------------
\section{Attributes and Methods in \class{Qtcm} Instances}

In this section I describe some attributes, particularly private ones,
that may be of interest to developers.
As is the convention in Python, private
attributes and methods are prepended by one or two underscores,
with two underscores being the ``more'' private attribute.
Please see the package
\latexhtml{API documentation%
		\footnote{http://www.johnny-lin.com/py\_pkgs/qtcm/doc/html-api/}}
        {\htmladdnormallink{API documentation}%
		{http://www.johnny-lin.com/py\_pkgs/qtcm/doc/html-api/}}
for details about all variables, including private variables.


	\subsection{Public \mods{num\_settings} Submodule Attributes/Methods}

\begin{itemize}
\item \vars{typecode}:  This module function returns the
	type code of the data array passed in as its argument.

\item \vars{typecodes}:  This dictionary is the same as the
	NumPy (or Numeric and \mods{numarray})
	dictionary \vars{typecodes}, except that the character
	\vars{'S'} and \vars{'c'} are added to the
	\vars{typecodes['Character']} entry, if absent.  This
	functionality is added because I found 
	\vars{typecodes['Character']} had different values in
	Mac OS X and Ubuntu GNU/Linux.
\end{itemize}


	\subsection{Private \mods{qtcm} Submodule Attributes}

This submodule of the package \mods{qtcm} is the module that defines
the \class{Qtcm} class.

\begin{itemize}
\item \vars{\_init\_prog\_dict}:  This dictionary contains
	the default values of all prognostic variables and 
	right-hand sides that can be initialized.  In the
	submodule \mods{qtcm}, it is set to
	the \vars{init\_prognostic\_dict} module variable in
	submodule \mods{defaults}.

\item \vars{\_init\_vars\_keys}:  List of all keys in
	\vars{\_init\_prog\_dict}, plus \vars{'dateofmodel'}
	and \vars{'title'}.  These names correspond to the
	field variables that are usually written out into a
	restart file.

\item \vars{\_test\_field}:  \class{Field} object instance used 
	in type tests.
\end{itemize}



	\subsection{Private \class{Qtcm} Attributes}  
					\label{sec:Qtcm.private.attrib}

\begin{itemize}
\item \vars{\_cont}:  A boolean attribute that is \vars{True}
	if the run session is a continuation run session and
	\vars{False} if not.  Set the value passed in by
	the keyword \vars{cont} when the \mods{run\_session}
	method is executed.

\item \vars{\_monlen}:  Integer array of the number of days in 
	each month, assuming a 365~day year.

\item \vars{\_\_qtcm}:  The extension module that is the
	compiled QTCM1 Fortran model for this instance.
	This attribute is unique for every instance:  The
	extension module \fn{.so} file is first copied to
	a temporary directory (given by the \vars{sodir}
	instance attribute) and then imported to the
	\class{Qtcm} instance.
	This private attribute is set on instantiation.

\item \vars{\_qtcm\_fields\_ids}:  Field ids for all default 
	field variables, set on instantiation.

\item \vars{\_runlists\_long\_names}:  Dictionary holding the
	descriptions of the standard run lists.  The keys of
	the dictionary are the names of the standard run lists.
\end{itemize}




%---------------------------------------------------------------------
\section{Creating Documentation}

The distribution of \mods{qtcm} comes with the full set of
documentation in readable form (PDF and HTML).  The documentation
consists of two kinds:  this User's Guide and the API documentation.
The User's Guide is written in \LaTeX.  The PDF version is generated
directly from \LaTeX, and the HTML version is created by
\LaTeX{2}HTML.

I use the \fn{make\_docs} shell script in \fn{doc} creates all these
documents.  Briefly, that script does the following:

\begin{itemize}
\item In the \fn{doc/latex} directory, uses \cmd{python} to
	run \fn{code\_to\_latex.py}, which generates the
	\LaTeX\ files describing the current \mods{qtcm} 
	package settings, including text in the manual which gives
	all uses of the current version number.

\item \LaTeX\ is run on the \LaTeX\ files in the \fn{doc/latex} directory.
	The PDF generated by the run is moved from \fn{doc/latex} to
	\fn{doc}.

\item \LaTeX{2}HTML is run on the \LaTeX\ files in \fn{doc/latex}.
	The HTML files generated by the run are moved to \fn{doc/html}.

\item \mods{epydoc} is run on the \mods{qtcm} package libraries.
	This is run in \fn{doc}, to make use of the \fn{epydoc}
	configuration file present there.  The syntax from the
	command line is:

\begin{codeblock}
\codeblockfont{%
epydoc -v --config epydocrc [name]}
\end{codeblock}
\vars{[name]} is either \cmd{qtcm}, if the \mods{qtcm} package is
installed in a directory listed in \vars{sys.path}, or 
\vars{[name]} is the name of the directory the \mods{qtcm} package is
located in (e.g., \fn{/usr/lib/python2.4/site-packages/qtcm}).

\end{itemize}

The \fn{make\_docs} script cannot be used without customizing it
to your system, so please \emphpara{DO NOT USE IT} if you do
not know what you are doing.  You could easily wipe out all your
documentation by mistake.





% ===== end of file =====


\chapter{Future Work}                       \label{ch:future}
% ==========================================================================
% Future
%
% By Johnny Lin
% ==========================================================================


% ------ BODY -----
%
This section describes the features and fixes I plan to work on
in this package.  The most urgent items are listed closer to the
begining of the lists.

\begin{itemize}
\item Add \code{implicit none} top setbypy.F90.

\item Check through Fortran routines that have arguments, to make sure
	f2py is properly understanding the intentions
	(i.e., in, out, inout) of the variables, since we're using the
	``quick way'' of making shared object libraries using f2py.
	The \fn{utilities.F90} file has a number of Fortran routines
	with arguments.

\item Cite:  Peterson, P. (2009) 
	F2PY: a tool for connecting Fortran and Python programs, 
	\emph{Int. J. Computational Science and Engineering,}
	Vol.\ 4, No.\ 4, pp.\ 296--305 for f2py.

\item Create a method like \mods{calc\_derived('T100')} which would
	primarily operate on a data file and provide a derived variable
	such as the temperature at 100 hPa, as given in this example.
	Figure out where to put the parameters (V1s, etc.) that are
	needed to make such a calculation.  As attributes?  Create a
	method to write the quantity out to an output file?
	Perhaps make an ability to calculate these values at heights
	at a given time each day during a run session?

\item Automate the installation using Python's
\htmladdnormallinkfoot{\mods{distutils}}{http://docs.python.org/dist/dist.html}
	utilities.

\item Describe a way of using job control (either via the operating system
	or IPython's \mods{jobctrl} module) 
	to do a quick-and-dirty parallelization of multiple
	\class{Qtcm} instance run sessions.  Or use some sort of threading
	to fire up two simulataneously running models.  Check that the
	simultaneously running models have different memory space.

\item Add capability for \fn{create\_benchmark.py} to overwrite
	existing benchmark files.

\item Make \vars{compiled\_form} set to \vars{'parts'} as the
	default instantiation.  Change documentation accordingly.

\item Currently, the \class{Qtcm} \mods{plotm} method works only on
	3-D output (time, latitude, longitude).  Some of the fields
	in the netCDF output files are 2-D.  Add the capability to
	\mods{plot\_netcdf\_output} in the \mods{plot} submodule
	to handle 2-D fields.

\item Add documentation about removing temporary files.
	Add documentation in Section~\ref{sec:model.instances}
	of details of what occurs during instantiation of 
	a \class{Qtcm} instance.

\item Add the units and long names for all field variables in the
	\mods{defaults} module.

\item Create a keyword to automatically change precipitation and
	evaporation units to mm/day (or similar).

\item Add ability to calculate and plot fields at different pressure
	levels.  Create another module like defaults that specifies
	the vertical fields and gives the equation to use to calculate
	those fields; call the module ``derivfields'' or something
	similar.

\item Throughout the \mods{qtcm} package I use the condition
	\mods{N.rank(}\dumarg{arg}\mods{)\thinspace=\thinspace0} 
	to test whether
	\dumarg{arg} is a scalar.  This works fine for \mods{numpy}
	objects, but it does not work properly for
	\mods{Numeric} and \mods{numarray} arrays.  In those
	array packages, \mods{rank('abc')} returns the value~1.
	This is not a problem, as long as everyone has \mods{numpy},
	but in order to make the package interoperable, I need to
	find a better way of testing for scalars.  The definitions
	of isscalar need to be changed in \mods{num\_settings}.

\item \mods{num\_settings} needs to be changed to truly enable me
	to test whether \mods{qtcm} works for 
	\mods{numarray} and \mods{Numeric} arrays.  The tests
	do not do this right now, because \mods{num\_settings}
	defaults to \mods{numpy}, if it exists.

\item Create makefiles for other platforms.
 
\item A few fields (e.g., \vars{u1}) have data for extra latitude bands,
	due to the use of ``ghost latitudes'' as part of the
	implementation of the numerics.  Details are found in the 
\latexhtml{%
\htmladdnormallinkfoot{QTCM1 manual}%
        {http://www.atmos.ucla.edu/$\sim$csi/qtcm\_man/v2.3/qtcm\_manv2.3.pdf}}%
{\htmladdnormallink{QTCM1 manual}%
        {http://www.atmos.ucla.edu/~csi/qtcm_man/v2.3/qtcm_manv2.3.pdf}}
\cite{Neelin/etal:2002}.

	Though adjusting to this idiosyncracy is not that difficult, 
	in the future I hope to implement a method of handing
	fields with ghost latitudes so that they have the same
	dimensions as the other gridded output variables.  In order
	to do this, I plan to write a Python method to read the
	Fortran generated binary restart file.

\item Change the \mods{set\_qtcm\_item} method so that it can 
	automatically accomodate setting Fortran real variables
	if integer values are input.

\item Currently, the \mods{get\_item\_qtcm} and 
	\mods{set\_item\_qtcm} methods will not work
	on integer and character arrays, only scalars and real arrays.
	Add that missing functionality to those methods.

\item Currently, the \mods{make\_snapshot} method duplicates the
	functionality of the pure-Fortran QTCM1 restart file mechanism.
	However, the restart file mechanism itself does not do a true
	restart.  A continuous run does not provide the same results
	as two runs over the same period, joined by the restart file.

	To see whether saving more variables would do the trick,
	I altered \mods{make\_snapshot} to store all Python level
	variables (i.e., \vars{self.\_qtcm\_fields\_ids}).  However,
	the restart failing described above still continued.  In the
	future, I hope to figure out exactly how many variables are
	needed in order to make the restart feature do a true
	restart.

\item Add a test of using the \vars{mrestart\thinspace=\thinspace1}
	restart option.  Does the \fn{qtcm.restart} file need to be
	in the current working directory or another?

\item Add a test in the unit test scripts to
	confirm that the \vars{init\_with\_instance\_state}
	attribute setting only has an effect if 
	\vars{compiled\_form\thinspace=\thinspace'parts'}.

\item Document \vars{tmppreview} keyword in \mods{plot.plot\_ncdf\_output}.

\item Confirm and document that
	for netCDF output, time is model time since dd-mm-yyyy.

\item Add to the \mods{plotm} method the ability to
	plot as text onto the figure the
	runname string and the calling line
	for the plotm method.

\item Couple with the
	\latexhtml{CliMT\footnote{http://maths.ucd.ie/$\sim$rca/climt/}}%
	{\htmladdnormallink{CliMT}{http://maths.ucd.ie/~rca/climt/}}
	climate modeling toolkit.

\item Enable Python to set \vars{arr1name}, etc., which are string
	variables at the Python level.  I haven't really thought through
	how \vars{arr1} variables work with the Python \class{Qtcm}
	instance.

\item Possible:  In the \class{Qtcm} method
	\mods{\_\_setattr\_\_}, add a test to raise an exception
	if the instance tries to set \vars{viscU}, \vars{viscT},
	or \vars{viscQ} as attributes.  Also create a method
	\code{isotropic\_visc} that will set all viscosity parameters
	non-dependent on direction.  See Section~\ref{sec:driverinit.diffs}
	for details.

\item Go through the manual and create HTML-only versions of tables
	that have table numbers (use a similar construct as in
	figure environments).

\item Go through documentation to check that
	output variable names are capitalized consistently.

\item Create way to redirect stdout.

\item Create a step method to run an arbitrary number of timesteps at
	the atmosphere level.

\end{itemize}


% ===== end of file =====





% ----- BACK MATTER OF THE DOCUMENT -----
%
\normalsize
\pagebreak
\bibliographystyle{plain}
\bibliography{/Users/jlin/work/res/bib/master}

%- Uncomment the input line below and comment out the \bibliographystyle
%  and \bibliography lines if you're running this without the master.bib 
%  BibTeX database
%% ==========================================================================
% Manual for QTCM Python Package
%
% Usage:
% - If you are running this on your own system, you will not have a copy of
%   my master.bib BibTeX database.  To run this, you'll have to comment out:
%
%      \bibliographystyle{chicago-jl}
%      \bibliography{/Users/jlin/work/res/bib/master}
%
%   and comment back in:
%
%      % ==========================================================================
% Manual for QTCM Python Package
%
% Usage:
% - If you are running this on your own system, you will not have a copy of
%   my master.bib BibTeX database.  To run this, you'll have to comment out:
%
%      \bibliographystyle{chicago-jl}
%      \bibliography{/Users/jlin/work/res/bib/master}
%
%   and comment back in:
%
%      \input{manual.bbl}
%
%   in this file.  Then you can use pdflatex on this file to get the PDF of
%   the manual.  These 3 lines are in the back matter of the document.
%
% Revision Notes:
% - By Johnny Lin, North Park University, http://www.johnny-lin.com/
% - The chicago BibTeX style is unrecognized by latex2html, so I use
%   the plain style.
% ==========================================================================


% ------ DOCUMENT DEFINITIONS ------
%
\documentclass[12pt]{book}
\usepackage{color}
\usepackage{html}
\usepackage{graphicx}
\usepackage{textcomp}
%\usepackage{comment}    %- Unrecognized by latex2html; its use causes errors
%\usepackage{fancyvrb}   %- Unrecognized by latex2html; its use causes errors


%- Packages unrecognized by latex2html, but causes no error:
%
%\usepackage[letterpaper,margin=1in,includefoot]{geometry}
\usepackage[letterpaper,margin=1.25in]{geometry}
\usepackage{bibnames}
\usepackage{longtable}
\usepackage{multirow}


%+ Comment out explicity margin settings since use package geometry:
%\setlength{\topmargin}{0in}
%\setlength{\headheight}{0in}
%\setlength{\headsep}{0in}
%\setlength{\oddsidemargin}{0in}
%\setlength{\evensidemargin}{0in}
%\setlength{\textheight}{8.5in}
%\setlength{\textwidth}{6.5in}




% ------ COMMANDS AND LENGTHS ------
%
% --- Define colors:  Have to do this because for some reason LaTeX
%     sometimes looks for "BLUE" instead of "blue" and complains when
%     "BLUE" isn't found.
%
\definecolor{Blue}{rgb}{0,0,1}
\definecolor{BLUE}{rgb}{0,0,1}
\definecolor{green}{rgb}{0,0.6,0}
\definecolor{Green}{rgb}{0,0.6,0}
\definecolor{GREEN}{rgb}{0,0.6,0}


% --- Format code blocks.  Currently set to print out the code in just 
%     typewriter font with no box.  Will work the same for pdflatex 
%     and latex2html:
%
%     codeblock:  Environment for blocks of computer code or internet 
%       addresses.
%     codeblockfont:  Sets font for codeblocks.
%
\newenvironment{codeblock}%
	{\begin{quotation}\begin{minipage}[t]{0.9\textwidth}}%
	{\end{minipage}\end{quotation}}
	%{\begin{flushleft}}%
	%{\end{flushleft}}
\newcommand{\codeblockfont}[1]{\textcolor{blue}{\texttt{#1}}}
%     *** Version that only works for pdflatex that puts a box around 
%         the block and centers it (commented out).  Note that using
%         fancyvrb is the better way of creating such a boxed section
%         of code, but fancyvrb isn't recognized by latex2html:
%\newenvironment{codeblock}%
%	{\begin{center}\begin{tabular}{|c|} \hline \\ }%
%	{\\ \\ \hline \end{tabular}\end{center}}
%\newcommand{\codeblockfont}[1]{\parbox{0.8\textwidth}{\texttt{#1}}}


% --- Text titling/emphasis settings:
%
%     emphpara:  Emphasis for the first phrase or sentence of a 
%         paragraph.
%     booktitle:  Formats book titles.
%     tabletitle:  Title for an item block in the information table.
%     paratitle:  Title for a paragraph in an item block in the
%         information table.
%     emphdate:  Emphasize date in paragraph text.
%
%     cmd:  Commands
%     dumarg:  Dummy arguments
%     codearg:  Same as dumarg.
%     fn:  File and directory names
%     screen:  Screen display
%     vars:  Variable and attribute names
%     mods:  Module, subroutine, and method names
%     class:  Class names
%     code:  Generic code (avoid using this)
%
\newcommand{\emphpara}[1]{\textbf{#1}}
\newcommand{\booktitle}[1]{\textit{#1}}
%\newcommand{\tabletitle}[1]{\textsf{\textbf{#1}}}
\newcommand{\paratitle}[1]{\textit{#1}}
\newcommand{\emphdate}[1]{\textbf{#1}}

\newcommand{\code}[1]{\textcolor{blue}{\texttt{#1}}}
\newcommand{\cmd}[1]{\textcolor{blue}{\texttt{#1}}}
\newcommand{\dumarg}[1]{\textit{#1}}
\newcommand{\codearg}[1]{\textit{#1}}
\newcommand{\fn}[1]{\textsf{\textit{#1}}}
\newcommand{\screen}[1]{\textcolor{green}{\texttt{#1}}}
\newcommand{\vars}[1]{\textcolor{blue}{\texttt{#1}}}
\newcommand{\class}[1]{\textcolor{blue}{\texttt{#1}}}
\newcommand{\mods}[1]{\textcolor{blue}{\texttt{#1}}}


% --- Special table formatting:
%
%     tabletitlewidth:  Width for title field of an item block in the 
%         information table.
%     tablebodywidth:  Width for body field of an item block in the 
%         information table.
%     tabletabulardims:  Dimensions for the information table, used in
%         the tabular command.
%     tableitemlinespace:  Vertical spacing between item blocks in the
%         information table.
%     infotitle and infotext:  Used for two-column sub-information 
%         tables found in the body field of the information table.  
%         These are not global lengths but have values specific to the 
%         local context in which they're used.
%
\newlength{\tabletitlewidth}
\settowidth{\tabletitlewidth}{file and directory names}

\newlength{\tablebodywidth}
\setlength{\tablebodywidth}{0.9\textwidth}
\addtolength{\tablebodywidth}{-4ex}
\addtolength{\tablebodywidth}{-\tabletitlewidth}

\newcommand{\tabletabulardims}%
	{p{\tabletitlewidth}@{\hspace{4ex}}p{\tablebodywidth}}

\newcommand{\tableitemlinespace}{\baselineskip}
\newlength{\infotitle}
\newlength{\infotext}


% --- Lengths for formatting:
%
\newlength{\remainder}        % length to describe the residual of the
                              %   linewidth minus \enumlabel
\newlength{\enumlabel}        % length to describe figure sub-label width
                              %   (e.g. "(a)")


% --- TtH stuff:
%
%\def\tthdump#1{#1}


% --- LaTeX2HTML stuff:
%
%     htmlfigcaption:  Formatting for HTML replacement figure captions.
%
\newcommand{\htmlfigcaption}[1]{\parbox[c]{70ex}{\footnotesize{#1}}}


% --- Some book title abbreviations:
%
%     rute:  Booktitle for Rute User's.
%     linuxnut:  Booktitle for Linux in a Nutshell.
%     pynut:  Booktitle for Python in a Nutshell.
%
\newcommand{\rute}{\booktitle{Rute User's}}
\newcommand{\linuxnut}{\booktitle{Linux in a Nutshell}}
\newcommand{\pynut}{\booktitle{Python in a Nutshell}}


% --- Define special characters ---
%
\newcommand{\aonehat}{\ensuremath{\widehat{a_1}}}
\newcommand{\bonehat}{\ensuremath{\widehat{b_1}}}
\newcommand{\D}{\ensuremath{\mathcal{D}}}
\def\BibTeX{B\kern-.03em i\kern-.03em b\kern-.15em\TeX}




% ------ BEGINNING OF DOCUMENT TEXT ------
%
\begin{document}

    

    
% ------ TITLE AND TOC ------
%
\title{\mods{qtcm} User's Guide}
\author{Johnny Wei-Bing Lin\thanks{Physics Department, North Park University,
	3225 W.\ Foster Ave., Chicago, IL  60625, USA}}
\date{\today}
\maketitle
\tableofcontents




% ------ BODY ------
%
\chapter{Introduction}
\input{intro}

\chapter{Installation and Configuration}    \label{ch:install}
	\section{Summary and Conventions}      \label{sec:install.sum}
	\input{install_sum}
	\section{Fortran Compiler}             \label{sec:fort.compilers}
	\input{install_fort}
	\section{Required Packages}            \label{sec:py.etc.pkgs}
	\input{install_pkgs}
	\section{Compiling Extension Modules}  \label{sec:create.so}
	\input{compile_so}
	\section{Testing the Installation}     \label{sec:test.qtcm}
	\input{test_qtcm}
	\section{Model Performance}
	\input{perform}
	\section{Installing in Mac OS X}       \label{sec:install.macosx}
	\input{qtcm_in_macosx}
	\section{Installing in Ubuntu}         \label{sec:install.ubuntu}
	\input{qtcm_in_ubuntu}

\chapter{Getting Started With \mods{qtcm}}  \label{ch:getting.started}
\input{started}

\chapter{Using \mods{qtcm}}                 \label{ch:using}
\input{using}

%@@@\chapter{Combining \code{qtcm} with \code{CliMT}}
%@@@\input{climt}

\chapter{Troubleshooting}                   \label{ch:trouble}
\input{trouble}

\chapter{Developer Notes}                   \label{ch:devnotes}
\input{devnotes}

\chapter{Future Work}                       \label{ch:future}
\input{future}




% ----- BACK MATTER OF THE DOCUMENT -----
%
\normalsize
\pagebreak
\bibliographystyle{plain}
\bibliography{/Users/jlin/work/res/bib/master}

%- Uncomment the input line below and comment out the \bibliographystyle
%  and \bibliography lines if you're running this without the master.bib 
%  BibTeX database
%\input{manual.bbl}        

\appendix
\chapter{Field Settings in \mods{defaults}}  \label{app:defaults.values}
\input{defaults}




% ------ END OF DOCUMENT TEXT ------
%
\end{document}


% ===== end of file =====

%
%   in this file.  Then you can use pdflatex on this file to get the PDF of
%   the manual.  These 3 lines are in the back matter of the document.
%
% Revision Notes:
% - By Johnny Lin, North Park University, http://www.johnny-lin.com/
% - The chicago BibTeX style is unrecognized by latex2html, so I use
%   the plain style.
% ==========================================================================


% ------ DOCUMENT DEFINITIONS ------
%
\documentclass[12pt]{book}
\usepackage{color}
\usepackage{html}
\usepackage{graphicx}
\usepackage{textcomp}
%\usepackage{comment}    %- Unrecognized by latex2html; its use causes errors
%\usepackage{fancyvrb}   %- Unrecognized by latex2html; its use causes errors


%- Packages unrecognized by latex2html, but causes no error:
%
%\usepackage[letterpaper,margin=1in,includefoot]{geometry}
\usepackage[letterpaper,margin=1.25in]{geometry}
\usepackage{bibnames}
\usepackage{longtable}
\usepackage{multirow}


%+ Comment out explicity margin settings since use package geometry:
%\setlength{\topmargin}{0in}
%\setlength{\headheight}{0in}
%\setlength{\headsep}{0in}
%\setlength{\oddsidemargin}{0in}
%\setlength{\evensidemargin}{0in}
%\setlength{\textheight}{8.5in}
%\setlength{\textwidth}{6.5in}




% ------ COMMANDS AND LENGTHS ------
%
% --- Define colors:  Have to do this because for some reason LaTeX
%     sometimes looks for "BLUE" instead of "blue" and complains when
%     "BLUE" isn't found.
%
\definecolor{Blue}{rgb}{0,0,1}
\definecolor{BLUE}{rgb}{0,0,1}
\definecolor{green}{rgb}{0,0.6,0}
\definecolor{Green}{rgb}{0,0.6,0}
\definecolor{GREEN}{rgb}{0,0.6,0}


% --- Format code blocks.  Currently set to print out the code in just 
%     typewriter font with no box.  Will work the same for pdflatex 
%     and latex2html:
%
%     codeblock:  Environment for blocks of computer code or internet 
%       addresses.
%     codeblockfont:  Sets font for codeblocks.
%
\newenvironment{codeblock}%
	{\begin{quotation}\begin{minipage}[t]{0.9\textwidth}}%
	{\end{minipage}\end{quotation}}
	%{\begin{flushleft}}%
	%{\end{flushleft}}
\newcommand{\codeblockfont}[1]{\textcolor{blue}{\texttt{#1}}}
%     *** Version that only works for pdflatex that puts a box around 
%         the block and centers it (commented out).  Note that using
%         fancyvrb is the better way of creating such a boxed section
%         of code, but fancyvrb isn't recognized by latex2html:
%\newenvironment{codeblock}%
%	{\begin{center}\begin{tabular}{|c|} \hline \\ }%
%	{\\ \\ \hline \end{tabular}\end{center}}
%\newcommand{\codeblockfont}[1]{\parbox{0.8\textwidth}{\texttt{#1}}}


% --- Text titling/emphasis settings:
%
%     emphpara:  Emphasis for the first phrase or sentence of a 
%         paragraph.
%     booktitle:  Formats book titles.
%     tabletitle:  Title for an item block in the information table.
%     paratitle:  Title for a paragraph in an item block in the
%         information table.
%     emphdate:  Emphasize date in paragraph text.
%
%     cmd:  Commands
%     dumarg:  Dummy arguments
%     codearg:  Same as dumarg.
%     fn:  File and directory names
%     screen:  Screen display
%     vars:  Variable and attribute names
%     mods:  Module, subroutine, and method names
%     class:  Class names
%     code:  Generic code (avoid using this)
%
\newcommand{\emphpara}[1]{\textbf{#1}}
\newcommand{\booktitle}[1]{\textit{#1}}
%\newcommand{\tabletitle}[1]{\textsf{\textbf{#1}}}
\newcommand{\paratitle}[1]{\textit{#1}}
\newcommand{\emphdate}[1]{\textbf{#1}}

\newcommand{\code}[1]{\textcolor{blue}{\texttt{#1}}}
\newcommand{\cmd}[1]{\textcolor{blue}{\texttt{#1}}}
\newcommand{\dumarg}[1]{\textit{#1}}
\newcommand{\codearg}[1]{\textit{#1}}
\newcommand{\fn}[1]{\textsf{\textit{#1}}}
\newcommand{\screen}[1]{\textcolor{green}{\texttt{#1}}}
\newcommand{\vars}[1]{\textcolor{blue}{\texttt{#1}}}
\newcommand{\class}[1]{\textcolor{blue}{\texttt{#1}}}
\newcommand{\mods}[1]{\textcolor{blue}{\texttt{#1}}}


% --- Special table formatting:
%
%     tabletitlewidth:  Width for title field of an item block in the 
%         information table.
%     tablebodywidth:  Width for body field of an item block in the 
%         information table.
%     tabletabulardims:  Dimensions for the information table, used in
%         the tabular command.
%     tableitemlinespace:  Vertical spacing between item blocks in the
%         information table.
%     infotitle and infotext:  Used for two-column sub-information 
%         tables found in the body field of the information table.  
%         These are not global lengths but have values specific to the 
%         local context in which they're used.
%
\newlength{\tabletitlewidth}
\settowidth{\tabletitlewidth}{file and directory names}

\newlength{\tablebodywidth}
\setlength{\tablebodywidth}{0.9\textwidth}
\addtolength{\tablebodywidth}{-4ex}
\addtolength{\tablebodywidth}{-\tabletitlewidth}

\newcommand{\tabletabulardims}%
	{p{\tabletitlewidth}@{\hspace{4ex}}p{\tablebodywidth}}

\newcommand{\tableitemlinespace}{\baselineskip}
\newlength{\infotitle}
\newlength{\infotext}


% --- Lengths for formatting:
%
\newlength{\remainder}        % length to describe the residual of the
                              %   linewidth minus \enumlabel
\newlength{\enumlabel}        % length to describe figure sub-label width
                              %   (e.g. "(a)")


% --- TtH stuff:
%
%\def\tthdump#1{#1}


% --- LaTeX2HTML stuff:
%
%     htmlfigcaption:  Formatting for HTML replacement figure captions.
%
\newcommand{\htmlfigcaption}[1]{\parbox[c]{70ex}{\footnotesize{#1}}}


% --- Some book title abbreviations:
%
%     rute:  Booktitle for Rute User's.
%     linuxnut:  Booktitle for Linux in a Nutshell.
%     pynut:  Booktitle for Python in a Nutshell.
%
\newcommand{\rute}{\booktitle{Rute User's}}
\newcommand{\linuxnut}{\booktitle{Linux in a Nutshell}}
\newcommand{\pynut}{\booktitle{Python in a Nutshell}}


% --- Define special characters ---
%
\newcommand{\aonehat}{\ensuremath{\widehat{a_1}}}
\newcommand{\bonehat}{\ensuremath{\widehat{b_1}}}
\newcommand{\D}{\ensuremath{\mathcal{D}}}
\def\BibTeX{B\kern-.03em i\kern-.03em b\kern-.15em\TeX}




% ------ BEGINNING OF DOCUMENT TEXT ------
%
\begin{document}

    

    
% ------ TITLE AND TOC ------
%
\title{\mods{qtcm} User's Guide}
\author{Johnny Wei-Bing Lin\thanks{Physics Department, North Park University,
	3225 W.\ Foster Ave., Chicago, IL  60625, USA}}
\date{\today}
\maketitle
\tableofcontents




% ------ BODY ------
%
\chapter{Introduction}
%=====================================================================
% Introduction
%=====================================================================


% ----- BEGIN TEXT -----
%
%---------------------------------------------------------------------
\section{How to Read This Manual}

\emphpara{Most users:} 
Just read 
(1) the installation instructions in Chapter~\ref{ch:install},
(2) Chapter~\ref{ch:getting.started},
which tells you all you need to get started using \mods{qtcm}, and
(3) examples in Section~\ref{sec:cookbook} that give a feel
for how you can use the model.

\emphpara{Users having problems:}
Chapter~\ref{ch:trouble} provides troubleshooting tips for
a few problems.
The detailed description of how the package functions, 
in Chapter~\ref{ch:using}, will probably be more useful.

\emphpara{Developers:}
If you want to change the source code, please read
Chapter~\ref{ch:devnotes}.  Chapter~\ref{ch:future} describes
all the things I'd like to do to improve the package, but haven't
gotten to yet.




%---------------------------------------------------------------------
\section{About the Package}

The single-baroclinic mode
Neelin-Zeng Quasi-Equilibrium Tropical Circulation Model
\latexhtml{(QTCM1)\footnote{http://www.atmos.ucla.edu/$\sim$csi}}%
	{\htmladdnormallink{(QTCM1)}{http://www.atmos.ucla.edu/~csi}}
is a primitive equation-based intermediate-level atmospheric model
that focuses on simulating the tropical atmosphere.  Being more
complicated than a simple model, the model has full non-linearity
with a basic representation of baroclinic instability,
includes a radiative-convective feedback package, and includes a
simple land soil moisture routine (but does not include topography).
A brief, but more detailed, description of QTCM1 is given in
Section~\ref{sec:brief_qtcm}.

\htmladdnormallinkfoot{Python}{http://www.python.org}
is an interpreted, object-oriented, multi-platform,
open-source language that is used in a variety of software applications,
ranging from game development to bioinformatics.
In climate studies, Python has been used as the core language for
data analysis
(e.g., \htmladdnormallinkfoot{Climate Data Analysis Tools}{http://cdat.sf.net}),
visualization
(e.g., \htmladdnormallinkfoot{Matplotlib}{http://matplotlib.sf.net}),
and 
modeling
(e.g., \htmladdnormallinkfoot{PyCCSM}{http://code.google.com/p/pyccsm/}).

In comparison to traditional compiled languages like Fortran,
Python's lack of a separate compile step greatly simplifies the
debugging and testing phases of development, because code snippets
can be testing as code is written.
Python's extensive suite of higher-level tools (e.g., statistics,
visualization, string and file manipulation) accessible at runtime 
enables modeling and analysis to occur concurrently.  

The \mods{qtcm} package is an implementation of the Neelin-Zeng
QTCM1 in a Python object-oriented environment.  The conversion
package
\htmladdnormallinkfoot{\mods{f2py}}{http://cens.ioc.ee/projects/f2py2e/} is
used to wrap the QTCM1 Fortran model routines and manage model
execution using Python datatypes and utilities.  The result is a
modeling package where order and choice of subroutine execution can
be altered at runtime.  Model analysis and visualization can also
be integrated with model execution at runtime.




%---------------------------------------------------------------------
\section{Conventions In This Manual}

	\subsection{Audience}

In this manual I assume you have a rudimentary knowledge of Python.
Thus, I do not describe basic Python data types (e.g., dictionaries,
lists), object framework and syntax (e.g., classes, methods,
attributes, instantiation), module and package importing.  If you
need to brush up (or learn) Python, I'd recommend the following
resources:

\begin{itemize}
\item \htmladdnormallinkfoot{Python Tutorial:}{http://docs.python.org/tut/}
	This tutorial was written by Guido van Rossum, Python's original
	author.

\item \htmladdnormallinkfoot{Instant Hacking:}%
	{http://www.hetland.org/python/instant-hacking.php}
	Learn how to program with Python.

\item \htmladdnormallinkfoot{Dive Into Python:}%
	{http://diveintopython.org/index.html}
	Reasonably complete and cohesive. The entire book is available for 
	free online.

\item \htmladdnormallinkfoot{Handbook of the Physics Computing Course:}%
	{http://www.pentangle.net/python/handbook/}
	Written for a science audience.  Recommended.

\item \latexhtml{CDAT/Python Tips for Earth Scientists:\footnote%
	{http://www.johnny-lin.com/cdat\_tips/}}%
	{\htmladdnormallink{CDAT/Python Tips for Earth Scientists:}%
		{http://www.johnny-lin.com/cdat_tips/}}
	This web site is a FAQ of sorts for people using Python and
	the Climate Data Analysis Tools (CDAT) in the earth sciences,
	and thus focuses on using Python to do science rather than
	the computer science aspects of the language.

\end{itemize}

The purpose of this package is to make the QTCM1 model easier to
use.  In order to interpret the results, however, you still need
to have a robust sense of what climate models can and cannot tell
you.  A starting point for the QTCM1 model is the brief description
of the model in Section~\ref{sec:brief_qtcm}.  After that, I would
read the original papers describing the model formulation and results
\cite{Neelin/Zeng:2000,Zeng/etal:2000}, and 
\latexhtml{papers citing the model formulation work.\footnote%
{http://scholar.google.com/scholar?hl=en\&lr=\&cites=14217886709842286738}}%
{\htmladdnormallink{papers citing the model formulation work}%
{http://scholar.google.com/scholar?hl=en&lr=&cites=14217886709842286738}.}
Being an intermediate-level model using the quasi-equilibrium assumption,
QTCM1 (and thus \mods{qtcm}) has distinct strengths and limitations; 
please be aware of them.


	\subsection{Typographic Conventions}

\begin{center}
\begin{tabular}{\tabletabulardims}
\cmd{commands} & to be typed at the command-line
	are rendered in a 
	blue, serif, fixed-width typewriter font
	(e.g., \cmd{make \_qtcm\_full\_365}). \\ \hline
\dumarg{dummy arguments} &
	coupled with code or screen display is rendered in a 
	serif, proportional, italicized font
	(e.g., \screen{Error-Value too long in} \dumarg{variable}). \\ \hline
\fn{file and directory names} & are rendered in a 
	sans-serif, italicized font
	(e.g., \fn{setbypy.F90}). \\ \hline
\screen{screen display} & is rendered in a 
	green, serif, fixed-width typewriter font. \\ \hline
\mods{module, method, and subroutine names} & are rendered in a 
	blue, serif, fixed-width typewriter font. \\ \hline
\vars{variable and attribute names} & are rendered in a 
	blue, serif, fixed-width typewriter font. \\ \hline
\class{class names} & are rendered in a 
	blue, serif, fixed-width typewriter font.
\end{tabular}
\end{center}

Blocks of code (usually commands, screen display, and module,
variable, and class names) are displayed in a blue, serif, fixed-width
typewriter font.


	\subsection{Terminology}

\begin{description}
\item[attribute and method references:]
	If there is any possibility of confusion, I will give the
	class that an attribute or method comes from when that
	attribute or method is referenced.  If no class is mentioned
	by name or context,
	assume that the attribute/method comes from the
	\class{Qtcm} class.

\item[``compiled QTCM1 model'':]
	This usually is used to denote when I'm talking about
	compiled Fortran QTCM1 routines and variables therein,
	as an extension module to the Python \mods{qtcm} package..
	Thus, ``compiled QTCM1 model \vars{u1}'' is the value
	of variable \vars{u1} in the Fortran model, not the
	value at the Python-level.  Sometimes I refer to the
	compiled QTCM1 model as just ``QTCM1'' or as
	``compiled QTCM1 Fortran model''.

\item[``pure-Fortran QTCM1'':]
	This refers to the Neelin-Zeng QTCM1 model in it's
	original Fortran form, not as an extension module to
	the Python \mods{qtcm} package.

\item[``Python-level'':]
	This usually denotes the value of a variable as an
	attribute of a \class{Qtcm} instance.  This variable
	is stored at the Python interpreter level.

\item[\class{Qtcm}:]
	This refers to the Python \class{Qtcm} class
	(note the capitalized first letter).

\item[\mods{qtcm}:]
	This refers to the Python \mods{qtcm} package.

\item[QTCM1 vs.\ QTCM:]
	Although the QTCM1 is currently the only version of a
	quasi-equilibrium tropical circulation model (QTCM), in
	principle one can construct a QTCM with any number of
	baroclinic modes.  In anticipation of this, the names of
	the Python package and class do not end in a numeral.  In
	this manual and the \mods{qtcm} docstrings, QTCM and QTCM1
	are used interchangably.
	Usually either of these phrases mean the quasi-equilibrium
	tropical circulation model in a generic sense, regardless
	of its form of implementation.
\end{description}




%---------------------------------------------------------------------
\section{Current Version Information and Acknowledgments}  \label{sec:ver}

\input{pkg_version_date.tex}
\input{pkg_author.tex}is the primary author of the package.

The \mods{qtcm} package is built upon the pure-Fortran QTCM1 model,
version 2.3 (August 2002), with a few minor changes.
Those changes are described in detail in
Section~\ref{sec:f90changes}.

The homepage for the \mods{qtcm} package is
\htmladdnormallink{http://www.johnny-lin.com/py\_pkgs/qtcm}%
	{http://www.johnny-lin.com/py_pkgs/qtcm}.
All Python code in this package, 
and the Fortran files \fn{setbypy.F90} and \fn{wrapcall.F90},
are \copyright\ 2003--2008 by 
\htmladdnormallinkfoot{Johnny Lin}%
		{http://www.johnny-lin.com} 
and constitutes a
library that is covered under the GNU Lesser General Public License
(LGPL):

\begin{quotation}
	This library is free software; you can redistribute it
	and/or modify it under the terms of the 
	\htmladdnormallinkfoot{GNU Lesser General Public License}%
		{http://www.gnu.org/copyleft/lesser.html} 
	as published by
	the Free Software Foundation; either version 2.1 of the
	License, or (at your option) any later version.

	This library is distributed in the hope that it will be
	useful, but WITHOUT ANY WARRANTY; without even the implied
	warranty of MERCHANTABILITY or FITNESS FOR A PARTICULAR
	PURPOSE. See the GNU Lesser General Public License for more
	details.

	You should have received a copy of the GNU Lesser General
	Public License along with this library; if not, write to
	the Free Software Foundation, Inc., 59 Temple Place, Suite
	330, Boston, MA 02111-1307 USA.

	You can contact Johnny Lin at his email address 
	or at North Park University, Physics Department,
	3225 W. Foster Ave., Chicago, IL 60625, USA.  
\end{quotation}

All other Fortran code in this package, as well as the makefiles,
are covered by licenses (if any) chosen by their respective owners.

This manual, in all forms (e.g., HTML, PDF, \LaTeX),
is part of the documentation of the \mods{qtcm} package 
and is \copyright\ 2007--2008 by Johnny Lin.
Permission is granted to copy, distribute and/or modify 
this document under the terms of the 
GNU Free Documentation License, Version 1.2 
or any later version published by the Free Software Foundation; 
with no Invariant Sections, no Front-Cover Texts, 
and no Back-Cover Texts. 
A copy of the license can be found 
\htmladdnormallinkfoot{here}{http://www.gnu.org/licenses/fdl.html}.

Transparent copies of this document are located online in
\latexhtml{%
\htmladdnormallinkfoot{PDF}%
	{http://www.johnny-lin.com/py\_pkgs/qtcm/doc/manual.pdf}}%
{\htmladdnormallink{PDF}%
	{http://www.johnny-lin.com/py_pkgs/qtcm/doc/manual.pdf}}
and
\latexhtml{%
\htmladdnormallinkfoot{HTML}%
	{http://www.johnny-lin.com/py\_pkgs/qtcm/doc/}}%
{\htmladdnormallink{HTML}%
	{http://www.johnny-lin.com/py_pkgs/qtcm/doc/}}
formats.
The \LaTeX\ source files are distributed with the \mods{qtcm}
distribution.
While the HTML version is nearly identical to the PDF
and \LaTeX\ versions, not every feature in the manual was successfully
converted.  This is particularly true with figures, which are
unnumbered in the HTML version and may be formatted differently
than the authoritative PDF version.
Thus, please consider the \LaTeX\ version as the authoritative
version.

\vspace{\baselineskip}

\emphpara{Acknowledgements:}
Thanks to David Neelin and Ning Zeng and the Climate Systems
Interactions Group at UCLA for their encouragement and help.
On the Python side,
thanks to Alexis Zubrow, Christian Dieterich, Rodrigo Caballero,
Michael Tobis, and Ray Pierrehumbert for steering me straight.
Early versions of some of this work was carried out 
at the University of Chicago Climate Systems Center, 
funded by the National Science Foundation (NSF) 
Information Technology Research Program under grant ATM-0121028. 
Any opinions, findings and conclusions or recommendations 
expressed in this material are those of the author and 
do not necessarily reflect the views of the NSF.

Intel\textregistered\ and
   Xeon\textregistered\ are registered trademarks of Intel Corporation.
Matlab\textregistered\ is a registered trademark of The MathWorks.
UNIX\textregistered\ is a registered trademark of The Open Group.




%---------------------------------------------------------------------
\section{Summary of Release History}

\begin{itemize}
\item 2008 Sep 12:  Version 0.1.2 released.  Summary of changes:
	\begin{itemize}
	\item Create \class{Qtcm} method \mods{get\_qtcm1\_item}.
		This method is effectively an alias of method 
		\mods{get\_qtcm\_item}.
	\item Create \class{Qtcm} method \mods{set\_qtcm1\_item}.
		This method is effectively an alias of method 
		\mods{set\_qtcm\_item}.
	\item Update User's Guide to phase out references to
		the \mods{get\_qtcm\_item}
		and \mods{set\_qtcm\_item} methods.  
		Adding the ``1'' to the method names makes the purpose
		of the methods clearer.
	\item Add unit tests to cover methods \mods{get\_qtcm1\_item} and
		\mods{set\_qtcm1\_item}.
	\end{itemize}

\item 2008 Jul 30:  Updates to the User's Guide (just the online versions,
        not the copies released with the tarball).

\item 2008 Jul 15:  First publicly available distribution 
	released (v0.1.1).
\end{itemize}




%---------------------------------------------------------------------
\section{A Brief Description of The QTCM1}   \label{sec:brief_qtcm}

This description is copied from Ch.\ 3 of Lin \cite{Lin:2000}, 
with minor revisions.
Model formulation is fully described in
Neelin \& Zeng \cite{Neelin/Zeng:2000} and model
results are described in Zeng et~al.\ \cite{Zeng/etal:2000}.
Neelin \& Zeng \cite{Neelin/Zeng:2000} is based upon v2.0 of QTCM1
and Zeng et~al.\ \cite{Zeng/etal:2000} is based on QTCM1 v2.1.
The 
\latexhtml{%
\htmladdnormallinkfoot{QTCM1 manual}%
	{http://www.atmos.ucla.edu/$\sim$csi/qtcm\_man/v2.3/qtcm\_manv2.3.pdf}}%
{\htmladdnormallink{QTCM1 manual}%
	{http://www.atmos.ucla.edu/~csi/qtcm_man/v2.3/qtcm_manv2.3.pdf}}
\cite{Neelin/etal:2002}
describes the details of model implementation.

QTCM1 differs from most full-scale GCMs primarily in how the vertical
temperature, humidity, and velocity structure of the atmosphere is
represented.  First, instead of representing the vertical structure
by finite-differenced levels, the model uses a Galerkin expansion
in the vertical.  The vertical basis functions are chosen according
to analytical solutions under convective quasi-equilibrium conditions,
so only a few need be retained.
Temperature and humidity are each described by separate
vertical basis functions ($a_1$ and $b_1$, respectively).
Low-level variations in the humidity basis
are larger than in the temperature basis.
For velocity, QTCM1 uses a single baroclinic basis function ($V_1$)
defined consistently with the temperature basis function,
as well as a barotropic velocity mode ($V_0$).
The vertical profiles of $a_1$, $b_1$, and $V_1$
are given in Figure~\ref{fig:qtcm.basis}.
Currently, QTCM1 does not include a separate
vertical degree of freedom describing the PBL.
The horizontal grid spacing of the model is 
$5.625^{\circ}$ longitude by $3.75^{\circ}$ latitude.


% <QTCM1 beta version vertical structure modes>
%
% (1) LaTeX version:
%
\begin{latexonly}
\begin{figure}
   \noindent
   \begin{minipage}[b]{.49\linewidth}
      \settowidth{\enumlabel}{(a) }%
      \setlength{\remainder}{\linewidth}% 
      \addtolength{\remainder}{-\enumlabel}
      {(a)}~\makebox[\remainder]{$a_1$ and $b_1$}
      \centering\includegraphics[width=\linewidth,viewport=58 72 389 344,clip]%
                    {figs/a1b1.pdf}
   \end{minipage}\hfill
   \begin{minipage}[b]{.49\linewidth}
      \settowidth{\enumlabel}{(b) }%
      \setlength{\remainder}{\linewidth}% 
      \addtolength{\remainder}{-\enumlabel}
      {(b)}~\makebox[\remainder]{$V_1$}

      \centering\includegraphics[width=\linewidth,viewport=58 72 389 346,clip]%
                    {figs/V1.pdf}
   \end{minipage}

   \caption{Vertical profiles of basis functions for
		(a) temperature $a_1$ (solid) and humidity $b_1$ (dashed) and
		(b) baroclinic component of
		horizontal velocity $V_1$.}
   \label{fig:qtcm.basis}
\end{figure}
\end{latexonly}

% (2) HTML replacement version:
%
\begin{htmlonly}
\label{fig:qtcm.basis}
\begin{center}
\htmladdimg{../latex/figs/a1b1.png}
\htmladdimg{../latex/figs/V1.png}

\htmlfigcaption{Figure \ref{fig:qtcm.basis}:  
	Vertical profiles of basis functions for
   	(a) temperature $a_1$ (solid) and humidity $b_1$ (dashed) and
   	(b) baroclinic component of
   	horizontal velocity $V_1$.}
\end{center}
\end{htmlonly}


These modes are chosen to accurately capture deep convective regions.
Outside deep convective regions the mode
is simply a highly truncated
Galerkin representation.  The system is much more tightly constrained than
a full-scale GCM, yet hopefully retains the essential dynamics and nonlinear
feedbacks.  The result is that QTCM1 is easier to diagnose than a GCM,
and is computationally fast (about 8 minutes per year on a Sun Ultra 2
workstation).  Zeng et al.\ \cite{Zeng/etal:2000} show results indicating
this intermediate-level model does a reasonable job simulating
tropical climatology and ENSO variability.  


Below is a summary of the main model equations \cite{Neelin/Zeng:2000}:
\begin{equation}
   \partial_t \mathbf{v}_1 
      + \D_{V1} (\mathbf{v}_0 , \mathbf{v}_1)
      + f \mathbf{k} \times \mathbf{v}_1
      =
   - \kappa \nabla T_1 
      - \epsilon_1 \mathbf{v}_1 
      - \epsilon_{01} \mathbf{v}_0
   \label{eqn:barocl_wind}
\end{equation}
\begin{equation}
   \partial_t \zeta_0 
      + \mathrm{curl}_z (\D_{V0} (\mathbf{v}_0 , \mathbf{v}_1))
      + \beta v_0
      =
   - \mathrm{curl}_z (\epsilon_0 \mathbf{v}_0)
      - \mathrm{curl}_z (\epsilon_{10} \mathbf{v}_1)
   \label{eqn:barotr_wind}
\end{equation}
\begin{equation}
   \aonehat (\partial_t + \D_{T1}) T_1 
      + M_{S1} \nabla \cdot {\bf v}_1 
      =
   \langle Q_c \rangle
      + (g/p_T) (-R^\uparrow_t -R^\downarrow_s + R^\uparrow_s + S_t - S_s + H)
   \label{eqn:temperature}
\end{equation}
\begin{equation}
   \bonehat (\partial_t + \D_{q1}) q_1 
      - M_{q1} \nabla \cdot {\bf v}_1 
      =
   \langle Q_q \rangle
      + (g/p_T) E
   \label{eqn:moisture}
\end{equation}
where (\ref{eqn:barocl_wind}) describes the baroclinic wind component,
      (\ref{eqn:barotr_wind}) describes the barotropic wind component,
      (\ref{eqn:temperature}) is the temperature equation, and
      (\ref{eqn:moisture}) is the moisture equation.

In the simplest formulation, the vertically integrated
convective heating and moisture sink
are assumed to be equal and opposite, so:
\begin{equation}
  -\langle Q_q \rangle = \langle Q_c \rangle 
                              = \epsilon^\ast_c (q_1 - T_1)
\end{equation}

For its convective parameterization for $Q_c$, this model uses the
Betts-Miller \cite{Betts/Miller:1986} moist convective
adjustment scheme, a scheme that is also used in some GCMs.
In the convective parameterization, the coefficient
$\epsilon^\ast_c$ is defined as:
\begin{equation}
   \epsilon^\ast_c 
      \equiv 
   \aonehat \bonehat (\aonehat + \bonehat)^{-1} \tau_c^{-1} 
      \mathcal{H}( \mathit{C}_{\mathrm{1}} )
\end{equation}
where $\mathcal{H}( \mathit{C}_{\mathrm{1}} )$ is zero for
$C_{1} \leq 0$, and one for $C_{1} > 0$, and $C_{1}$
is a measure of the convective available potential energy (CAPE),
projected onto the model's temperature and moisture basis functions.

Sensible heat ($H$) and evaporation ($E$) are given as
bulk-aerodynamic formulations:
\begin{equation}
   H
      =
   \rho_a C_D \mathrm{V}_s (T_s - (T_{rs} + a_{1s} T_1))
\end{equation}
\begin{equation}
   E
      =
   \rho_a C_D \mathrm{V}_s (q_\mathit{sat} (T_s) 
      - (q_{rs} + b_{1s} q_1))
\end{equation}

Longwave radiation components are denoted by $R$, and net shortwave
radiation is denoted by $S$.
The terms $\D_{V1}$ and $\D_{V0}$ are the advection-diffusion operators
for the momentum equations (projected onto $V_0$ and $V_1 (p)$,
respectively).
The terms $\D_{T1}$ and $\D_{q1}$ are the
advection-diffusion operators for the temperature and moisture
equations, respectively, using a vertical average projection.
The $\langle X \rangle$ and $\widehat{X}$ operators are
equivalent and denote vertically integration over the troposphere.
Please see Neelin \& Zeng \cite{Neelin/Zeng:2000} and 
Zeng et al.\ \cite{Zeng/etal:2000}
for a more complete description of equations and coefficients.







% ====== end file ======


\chapter{Installation and Configuration}    \label{ch:install}
	\section{Summary and Conventions}      \label{sec:install.sum}
	% ==========================================================================
% Installation Summary
%
% By Johnny Lin
% ==========================================================================


% ------ BODY -----
%

This section provides a summary of the steps needed to install
\mods{qtcm}, and a description of the naming conventions used in
this chapter.  If you have had a decent amount of experience with
Python and installing software on a Unix system, this section will
probably be all you need to read.  The installation steps are:

\begin{enumerate}
\item Install a Fortran compiler (see Section~\ref{sec:fort.compilers}
	for a list of compilers known to work).
	This compiler should be in a directory
	listed in your system path (e.g., \fn{/usr/bin}, etc.).

\item Install all required packages
	(see Section~\ref{sec:py.etc.pkgs} for details):
	Python,
	\mods{matplotlib} (plus the \mods{basemap} toolkit),
	NumPy (which includes \mods{f2py}),
	Scientific Python,
	\LaTeX,
	and
	netCDF.

	Python packages are required to be installed on your
	system in a directory listed in your \vars{sys.path},
	and the other packages/libraries are required to be in 
	standard directories listed in your system path 
	(e.g., \fn{/usr/bin}, \fn{/sw/include}, etc.).

	Make sure the executable for Python can be called at the
	Unix command line by typing both \cmd{python}.
	You might need to define a Unix alias
	that maps \cmd{python2.4} (or whichever version of Python
	you are using) to \cmd{python}.

\item \latexhtml{Download\footnote{http://www.johnny-lin.com/py\_pkgs/qtcm/}}%
        {\htmladdnormallink{Download}{http://www.johnny-lin.com/py_pkgs/qtcm/}}
	the \mods{qtcm} tarball and extract the distribution
	into a temporary directory for building purposes.
	\fn{\input{pkg_distro_dirname}}is the name of
	the \mods{qtcm} distribution directory;
	the number following the hyphen is the
	version number of the distribution.  \label{list:download.qtcm.sum}

	In this manual, the path to \fn{\input{pkg_distro_dirname}}will
	be called the ``\mods{qtcm} build path'' and be given as
	\fn{/buildpath}.  When you see \fn{/buildpath}, please substitute
	the actual temporary directory you created for building purposes.

\item The \mods{qtcm} distribution directory 
	\fn{\input{pkg_distro_dirname}}contains the following 
	principal sub-directories:
	\fn{doc}, \fn{lib}, \fn{src}, \fn{test}.
	Documentation is in \fn{doc},
	all the package modules are in \fn{lib},
	building of extension modules will take place in \fn{src},
	and testing of the package is done in \fn{test}.

\item Compile \mods{qtcm} extension modules in \fn{src}:
	Go to \fn{src}, copy the makefile from
	\fn{src/Makefiles} corresponding to your
	system into \fn{src}, rename to \fn{makefile},
	make changes to the makefile as needed,
	and execute:
	\begin{codeblock}
	\codeblockfont{%
	make clean \\
	make \_qtcm\_full\_365.so \\
	make \_qtcm\_parts\_365.so}
	\end{codeblock}
	If you executed the make commands in \fn{src,},
	the extension modules will be automatically placed in
	\fn{lib} in the \fn{\input{pkg_distro_dirname}}directory.
	See Section~\ref{sec:create.so} for details.
	\label{list:compile.so.sum}

\item Copy the entire contents of \fn{lib} in
	\fn{\input{pkg_distro_dirname}}(not \fn{lib} itself) 
	to a directory named
	\fn{qtcm} that is on your \mods{sys.path}.  For instance,
	for Mac OS X using Fink,
	many Python packages are located in a directory
	named \fn{/sw/\-lib/\-python2.4/\-site-packages}, or something
	similar, and this directory is on the system \mods{sys.path}.  
	If this is the case for your system, copy the
	contents of \fn{lib} into
	\fn{/sw/lib/\-python2.4/\-site-packages/\-qtcm}.
	(For Unix systems, the equivalent directory is usually
	\fn{/usr/\-local/\-lib/\-python2.4/\-site-packages}.)

\item Test the \mods{qtcm} distribution in \fn{test}:
	This step is optional and can take a while.
	Testing requires you to first generate a suite of benchmarks
	using the pure-Fortran QTCM1 model, then running the tests of
	\mods{qtcm} by typing:
	\begin{codeblock}
	\codeblockfont{%
python test\_all.py}
	\end{codeblock}
	at the Unix command line while in \fn{test}.
	See Section~\ref{sec:test.qtcm} for details.

\end{enumerate}

At some point, I will automate the installation using Python's
\htmladdnormallinkfoot{\mods{distutils}}{http://docs.python.org/dist/dist.html}
utilities.



% ===== end of file =====

	\section{Fortran Compiler}             \label{sec:fort.compilers}
	% ==========================================================================
% Fortran compilers
%
% By Johnny Lin
% ==========================================================================


% ------ BODY -----
%

You must have a Fortran compiler installed on your system in order
to compile \mods{qtcm}.  The compiler must be able to interface with
a pre-processor, as QTCM1 makes copious use of pre-processor directives.
\mods{qtcm} is known to work with the following Fortran compilers on the
following platforms:

\begin{center}
\begin{tabular}{l|l|l}
\textbf{Compiler}  & \textbf{Compiler Web Site} & \textbf{Platform(s)} \\ 
\hline
\mods{g95} & \htmladdnormallink{http://www.g95.org/}{http://www.g95.org/}  
	& Mac OS X \\
\end{tabular}
\end{center}

It will probably work with other platforms, but I haven't been able
to test platforms besides those listed above.  Note that \mods{g95}
is not \htmladdnormallink{GNU Fortran}{http://gcc.gnu.org/fortran/}
(\mods{gfortran}), the Fortran 95 compiler included with the more
recent versions of GCC.




% ===== end of file =====

	\section{Required Packages}            \label{sec:py.etc.pkgs}
	% ==========================================================================
% Python packages
%
% By Johnny Lin
% ==========================================================================


% ------ BODY -----
%

The following Python packages are required to be installed on your
system in a directory listed in your \vars{sys.path}:
\begin{itemize}
\item \htmladdnormallinkfoot{Python}%
	{http://www.python.org/}:  The Python programming language
	and interpreter.  Make sure you have a version recent enough
	to be compatible with all the needed Python packages.
\item \htmladdnormallinkfoot{\mods{matplotlib}}%
	{http://matplotlib.sourceforge.net/}:  Scientific plotting
	package, using Matlab-like syntax.  The \mods{basemap} toolkit
	for \mods{matplotlib} must also be installed.
\item \htmladdnormallinkfoot{NumPy}%
	{http://numpy.scipy.org/}:  The standard array package for
	Python.  The module name of NumPy imported in a Python 
	session is \mods{numpy}.
\item \htmladdnormallinkfoot{Scientific Python}%
	{http://dirac.cnrs-orleans.fr/plone/software/scientificpython/}:
	Has netCDF file operators, in addition to other routines
	of use in scientific computing.  The module name of
	Scientific Python imported in a Python session is
	\mods{Scientific}.
\end{itemize}

One other required Python package, \mods{f2py}, is now a part of the
NumPy package, and so installation of NumPy is sufficient to give
you both.

The package \htmladdnormallinkfoot{SciPy}{http://www.scipy.org},
which includes several Python-accessible scientific libraries, also
includes NumPy (and thus \mods{f2py}), so if you install SciPy,
you don't have to install NumPy again.  Note that SciPy is not the
same as Scientific Python; the names are confusing.

A few non-Python packages are also required:
\begin{itemize}
\item \LaTeX: A scientific typesetting program used by the 
	\class{Qtcm} instance method \mods{plotm} to handle 
	exponents and subscripts.  The most common Unix 
	distribution of \LaTeX\ is
	\htmladdnormallinkfoot{teTeX}{http://www.tug.org/teTeX}.

\item netCDF:  This set of libraries enables one to write datasets into
	a platform independent, binary format, with metdata attached.
	The \htmladdnormallinkfoot{netCDF 3.6.2 library}%
        	{http://www.unidata.ucar.edu/software/netcdf/}
	source code can be
\latexhtml{downloaded from UCAR\footnote{http://www.unidata.ucar.edu/downloads/netcdf/netcdf-3\_6\_2/}}%
        {\htmladdnormallink{downloaded from UCAR}{http://www.unidata.ucar.edu/downloads/netcdf/netcdf-3_6_2/}}.
\end{itemize}

For most Unix installations, the easiest way to install all the
above is via a package manager, for instance \mods{apt-get} in
Debian GNU/Linux, \mods{aptitude} or \mods{synaptic} in Ubuntu
GNU/Linux, and \mods{fink} in Mac OS X.  Of course, you can also
download a package's source code and build direct and/or install
using Python's
\htmladdnormallinkfoot{\mods{distutils}}{http://docs.python.org/dist/dist.html}
utilities.




% ===== end of file =====

	\section{Compiling Extension Modules}  \label{sec:create.so}
	% ==========================================================================
% Compiling extension modules
%
% By Johnny Lin
% ==========================================================================


% ------ BODY -----
%

The extension modules (\fn{.so} files) are imported and used by
\mods{qtcm} objects, and contain the Fortran QTCM1 model that is
called by the \mods{qtcm} Python wrappers.  These extension modules
are located in the \fn{lib} directory of the \mods{qtcm} distribution,
and, in general, need to be created only when the \mods{qtcm} package
is installed.

Two extension modules are created:  \fn{\_qtcm\_full\_365.so} and
\fn{\_qtcm\_parts\_365.so}.  Both modules define QTCM1 models where:

\begin{itemize}
\item A year is 365 days long 
	(makefile macro \vars{YEAR360} is off).
\item Model output is written to netCDF files
	(makefile macro \vars{NETCDFOUT} is on).
\item The atmospheric boundary layer model is used
	(makefile macro \vars{NO\_ABL} is off).
\item A global domain is used
	(makefile macro \vars{SPONGES} is off).
\item Topography effects due to induced divergence are not included
	(makefile macro \vars{TOPO} is off).
\item Coupling between atmosphere and ocean is through mean fluxes
	(makefile macro \vars{CPLMEAN} is off).
\item The mixed layer ocean model is not used
	(makefile macros \vars{MXL\_OCEAN} and \vars{BLEND\_SST} are both off).
\end{itemize}

(All other makefile macros not listed are also turned off.)
The only difference between these two extension modules is that the
``full'' module is used by \class{Qtcm} instances where
\vars{compiled\_form} is set to \vars{'full'}, and the ``parts''
module is used by \class{Qtcm} instances where \vars{compiled\_form}
is set to \vars{'parts'}.  See Section~\ref{sec:compiledform} for
details about the \vars{compiled\_form} attribute.

The extension modules are created through the following steps:
\begin{enumerate}
\item Go to the \mods{qtcm} distribution directory
	\fn{\input{pkg_distro_dirname}}located in
	your build path \fn{/buildpath}.  Go to the \fn{src}
	sub-directory.  This is where all the building of the
	extension modules will take place.

\item Copy the makefile that corresponds to your platform to
	the \fn{src} directory, and rename it \fn{makefile}.
	The \fn{Makefiles} sub-directory of \fn{src} contains
	makefiles for various platforms.

\item In \fn{makefile}, make the following changes:
	\begin{enumerate}
	\item Change the \vars{FC} environment variable as needed, 
		if your Fortran compiler is different.
	\item Change the \vars{FFLAGSM} environment variable, if the
		compiler flags listed are not supported by your
		compiler.
	\item Change the \vars{-I} and \vars{-L} parts of the
		\vars{NCINC} and \vars{NCLIB} environment
		variables so that the paths for the netCDF library and
		include files match your system's installation:
		\begin{codeblock}
		\codeblockfont{%
NCINC=-I/yourpath/netcdf/include \\
NCLIB=-L/yourpath/netcdf/lib -lnetcdf}
		\end{codeblock}
		Set \dumarg{yourpath} to the full path to the
		\fn{netcdf} directory where the \fn{include} and
		\fn{lib} sub-directories are that hold the netCDF
		libraries and include files.
		(You shouldn't have to change the \vars{-l} part of
		\vars{NCLIB}, since it is standard to name the netCDF
		library \fn{libnetcdf.a}.  But if you have a non-standard
		installation, change the \vars{-l} part too.)
	\end{enumerate}

\item At the Unix prompt, type:
\begin{codeblock}
\codeblockfont{%
\small
make clean \&\& make \_qtcm\_full\_365.so \&\& make \_qtcm\_parts\_365.so}
\end{codeblock}
	to clean up leftover files from previous compilations, and to
	compile the extension module shared object files
	\fn{\_qtcm\_full\_365.so} and \fn{\_qtcm\_parts\_365.so}.
\end{enumerate}

The makefile will automatically move the shared object files into
\fn{../lib}, overwriting any pre-existing files of the same name.
A detailed description of the makefile and using \mods{f2py} is
given in Section~\ref{sec:create.new.so}, if you wish to create a
different extension module.




% ===== end of file =====

	\section{Testing the Installation}     \label{sec:test.qtcm}
	% ==========================================================================
% Installation Summary
%
% By Johnny Lin
% ==========================================================================


% ------ BODY -----
%

The \mods{qtcm} distribution comes with a set of tests for the
package, using Python's \mods{unittest} package.  
Just to warn you, the tests take around an hour to run.
The tests will not work if the contents of \fn{lib}
after you've finished building \mods{qtcm} have not been copied
to a directory named \fn{qtcm} that is on your \mods{sys.path} path,
so make sure you've gone through all the install steps
(summarized in Section~\ref{sec:install.sum}) before you do these
tests.

\emphpara{NB:}  For these tests to work, both \cmd{python} and
\cmd{python2.4} must refer to the executable for the Python
installation on your system that you are using for running \mods{qtcm}.

The tests require a set of benchmark output files in the
\fn{test/benchmarks} directory in the
\fn{\input{pkg_distro_dirname}}directory (the output will be in
directories whose names begin with ``aquaplanet'' or ``landon'').
These output files are not included with the \mods{qtcm} distribution,
and must be created, by doing the following:

\begin{enumerate}
\item Goto the directory \fn{test/benchmarks/create/src} in the
	\fn{\input{pkg_distro_dirname}}\mods{qtcm} distribution directory,
	and copy the makefile from sub-directory \fn{Makesfiles},
	that corresponds to your system to the
	\fn{test/benchmarks/create/src} directory.  Rename the makefile 
	in \fn{test/benchmarks/create/src} to \fn{makefile}.

\item In \fn{makefile}, make the following changes:
        \begin{enumerate}
        \item Change the \vars{FC} environment variable as needed,
                if your Fortran compiler is different.
        \item Change the \vars{FFLAGSM} environment variable, if the
                compiler flags listed are not supported by your
                compiler.
        \item Change the \vars{-I} and \vars{-L} parts of the
                \vars{NCINC} and \vars{NCLIB} environment
                variables so that the paths for the netCDF library and
                include files match your system's installation:
                \begin{codeblock}
                \codeblockfont{%
NCINC=-I/yourpath/netcdf/include \\
NCLIB=-L/yourpath/netcdf/lib -lnetcdf}
                \end{codeblock}
                Set \dumarg{yourpath} to the full path to the
                \fn{netcdf} directory where the \fn{include} and
                \fn{lib} sub-directories are that hold the netCDF
                libraries and include files.
                (You shouldn't have to change the \vars{-l} part of
                \vars{NCLIB}, since it is standard to name the netCDF
                library \fn{libnetcdf.a}.  But if you have a non-standard
                installation, change the \vars{-l} part too.)
        \end{enumerate}

\item Go to the directory \fn{test/benchmarks/create} in the
	\fn{\input{pkg_distro_dirname}}\mods{qtcm} distribution directory.

\item Type \cmd{python create\_benchmarks.py} at the Unix command line
	to run the benchmark creation script.
\end{enumerate}

The created benchmarks will be located in 
\fn{test/benchmarks}, in directories with names related to the
run that was done, as described earlier.
The benchmarks are created using the
pure-Fortran QTCM1 model code,
version 2.3 (August 2002), with an altered makefile
(described above) and the following code change:
In all \fn{.F90} files, occurrences of:
        \begin{codeblock}
        \codeblockfont{%
        Character(len=130)}
        \end{codeblock}
        are changed to:
        \begin{codeblock}
        \codeblockfont{%
        Character(len=305)}
        \end{codeblock}
This enables the model to properly deal with longer filenames.
The number ``305'' is chosen to make search and replace easier.

Once the benchmarks are created, you can test the \mods{qtcm} package
by doing the following:
\begin{enumerate}
\item Go to the \fn{test} directory in the 
	\fn{\input{pkg_distro_dirname}}directory.
\item Type \cmd{python test\_all.py} at the Unix command line.
\end{enumerate}

If at the end of the test runs you see this message (or something similar):
\begin{codeblock}
\codeblockfont{%
\footnotesize
---------------------------------------------------------------------- \\
Ran 93 tests in 1244.205s \\
 \\
OK}
\end{codeblock}
then everything worked fine!  If you get any other message, the test(s) have
failed.



% ===== end of file =====

	\section{Model Performance}
	%=====================================================================
% Model Performance
%=====================================================================


% ----- BEGIN TEXT -----
%
%---------------------------------------------------------------------

The wall-clock time values below give the mean over three
separate 365 day aquaplanet runs,
using climatological sea surface temperature for lower boundary forcing.
NetCDF output is written daily, for both instantaneous and mean values.
The time step is 1200~sec, and the version of \mods{qtcm} used
is 0.1.1.
The horizontal grid spacing of all model versions is
$5.625^{\circ}$ longitude by $3.75^{\circ}$ latitude.
Values are in seconds:
\begin{center}
\begin{tabular}{p{0.5\linewidth}|c|c|c}
\textbf{System} & \textbf{Pure} & \textbf{Full} & \textbf{Parts} \\
\hline
Mac OS X:  MacBook 1.83 GHz Intel Core Duo running Mac OS X
	10.4.10 with 1 GB RAM
	(Python 2.4.3, NumPy 1.0.3, \mods{f2py} 2\_3816).
    & 152.59 & 153.63 & 158.94 \\
\hline
Ubuntu GNU/Linux:  Dell PowerEdge 860 with 2.66 GHz Quad Core Intel
	Xeon processors (64 bit) running Ubuntu 8.04.1 LTS
	(Python 2.5.2, NumPy 1.1.0, \mods{f2py} 2\_5237).
    & 43.73 & 44.79 & 47.45
\end{tabular}
\end{center}

``Pure'' refers to the pure-Fortran version of QTCM1.
``Full'' refers to a \mods{qtcm} run session with \vars{compiled\_form}
set to \vars{'full'}.  ``Parts'' refers to a \mods{qtcm} run session
with \vars{compiled\_form} set to \vars{'parts'}.
(Section~\ref{sec:compiledform} has details about the difference
between compiled forms.)

The \vars{'parts'} version of \mods{qtcm} gives Python the maximum
flexibility in accessing compiled QTCM1 model subroutines and
variables.  The price of that flexibility is an increase in
run time of approximately 4--9\% over the pure-Fortran version.
The difference in performance between the
\vars{'full'} version of \mods{qtcm} and the pure-Fortran version of
QTCM1 is between negligible and 3\% longer.

To make a timing for the pure-Fortran model, go to
\fn{test/benchmarks/timing/work} in \fn{/buildpath} and run the
\fn{timing\_365.sh} script in that directory.  That script runs the
QTCM1 model using \cmd{/usr/bin/time}, which at the end of the
script will output the amount of time it took to make the model
run.  Run the timing script three times and average the values to
obtain a time comparable to the above.

To make a timing for the \mods{qtcm} model, type \cmd{python
timing\_365.py} while in the \fn{test} directory in \fn{/buildpath}.
Three run sessions will be made for \vars{compiled\_form} equal to
\vars{'full'} and \vars{'parts'}, the times are averaged, and the
value are output at the end of the script.




% ====== end file ======

	\section{Installing in Mac OS X}       \label{sec:install.macosx}
	% ==========================================================================
% Description of installing in Mac OS X
%
% By Johnny Lin
% ==========================================================================


% ------ BODY -----
%
%------------------------------------------------------------------------
\subsection{Introduction}

This section describes issues and a summary of the installation steps
I followed to install \mods{qtcm} on a Mac running OS X.
It is a specific realization of the general installation
instructions found in Sections~\ref{sec:install.sum}--\ref{sec:test.qtcm}.
I first worked through these installation steps during June--July 2007,
with updates during July 2008.
The best way to go through this section is to go through
the summary of the installation steps in 
Section~\ref{sec:osx.install.summary},
and looking back to other sections as needed.




%------------------------------------------------------------------------
\subsection{Platform and Unix Dependencies}

This work was done on a MacBook 1.83 GHz Intel Core Duo running Mac OS X
10.4.11.  My machine has 1 GB RAM and 64 GB of disk in its main partition.

I recommend you turn-off your antivirus software before you
do the installs.  
Problems have been
\latexhtml{reported by Fink users\footnote%
		{http://finkproject.org/faq/usage-fink.php?phpLang=en\#kernel-panics}}%
	{\htmladdnormallink{reported by Fink users}%
		{http://finkproject.org/faq/usage-fink.php?phpLang=en#kernel-panics}}
using the Fink package manager with antivirus software enabled.

There are a variety of dependencies that are required to get your Mac
up-and-running as a scientific computing platform.  The most basic is
installing Apple's 
\htmladdnormallinkfoot{XCode}{http://developer.apple.com/tools/xcode/}
developer tools.\footnote%
	{The package should work in Mac OS X 10.4 with XCode 2.4.1 and higher;
	I've tried it with both 2.4.1 and XCode 2.5.  Note that
	XCode 3.1 only works on Mac OS X 10.5.}
This set of tools contains compilers and libraries
needed to do anything further.  You have to be a member of Apple's
Developer Connection, but registration is free.

Besides XCode, there are a variety of Unix libraries and utilities that you
need.  I first tried installing them by myself, from scratch, into
\fn{/usr/local}, but it was hard to keep track of all the dependencies.
A few that did work, and that I installed from their disk images, are:
\htmladdnormallinkfoot{MacTeX}{http://www.tug.org/mactex/}, 
\htmladdnormallinkfoot{MAMP}{http://www.mamp.info/}, and 
\htmladdnormallinkfoot{Tcl/Tk Aqua BI (Batteries Included)}%
	{http://tcltkaqua.sourceforge.net/}.\footnote%
		{Theoretically you can use Fink to install the equivalent
		of these packages, but I like the specific collection 
		found in these packages.  For instance, Tcl/Tk Aqua BI
		runs natively on the Mac.}

For everything else, thankfully, there's the
\htmladdnormallinkfoot{Fink Project}{http://www.finkproject.org/} which
uses a package manager built upon Debian tools to install ports of
Unix programs onto a Mac.  I just 
\htmladdnormallinkfoot{downloaded}%
	{http://www.finkproject.org/download/index.php?phpLang=en}
a binary version of the Fink 0.8.1 installer for Intel Macs,
installed Fink, and used its package management tools to install
(almost) everything else I needed.\footnote%
	{The one drawback of Fink is that it sometimes
	has stability problems.  In those cases, Fink provides
	command line suggestions to fix the problems, which sometimes
	will work.  If not, sometimes
\latexhtml{%
	deleting Fink and everything it installed,\footnote%
	{http://www.finkproject.org/faq/usage-fink.php?phpLang=en\#removing}}{%
\htmladdnormallink{deleting Fink and everything it installed}
	{http://www.finkproject.org/faq/usage-fink.php?phpLang=en#removing},}
	and starting afresh, will do the trick.
	It also appeared to me that sometimes when I installed 
	multiple packages
	via one \cmd{fink install} call, the installation did not work
	as well as when I installed only one package per call.}

Although you do not need anything besides a Fortran compiler and
the netCDF libraries to run QTCM1 in its pure-Fortran form, in order to
manipulate the model and use this Python version \mods{qtcm}, you
need to have Python installed.  The default Python that comes
with the Mac is a little old, so I used Fink to also install
Python 2.5 and related packages, including
\htmladdnormallinkfoot{matplotlib}{http://matplotlib.sourceforge.net/},
\htmladdnormallinkfoot{ScientificPython}{http://dirac.cnrs-orleans.fr/plone/software/scientificpython/},
and
\htmladdnormallinkfoot{SciPy}{http://www.scipy.org}
(see Section~\ref{sec:osx.summary} for details).




%------------------------------------------------------------------------
\subsection{Fortran Compiler}

There are a variety of high-quality, commercial Fortran compilers.
Unfortunately, because I do not have a research budget, I am not able
to use those compilers.  The 
\htmladdnormallinkfoot{GNU Compiler Collection}{http://gcc.gnu.org/}
(GCC) provides a suite of open-source compilers, some of which are the
standards of their language.  Most of the GCC compilers are installed
on your Mac when you install XCode.

\htmladdnormallinkfoot{GNU Fortran}{http://gcc.gnu.org/fortran/}
(\mods{gfortran}), is the Fortran 95 compiler included with the more
recent versions of GCC.
Unfortunately, I was not able to get it to compile QTCM1.
There is a second open-source Fortran compiler,
\htmladdnormallinkfoot{G95}{http://www.g95.org/} (\mods{g95}),
which some feel is farther along in its development than \mods{gfortran}.
I was able to successfully compile QTCM1 with \mods{g95} on my Mac.
I used Fink to install G95
(see Section~\ref{sec:osx.summary} for details).




%------------------------------------------------------------------------
\subsection{NetCDF Libraries}   \label{sec:netcdf}

For some reason, the netCDF libraries and include files
installed by Fink didn't correspond to the files needed
by the calling routines in \mods{qtcm}.  To solve this, I compiled
my own set of 
\htmladdnormallinkfoot{netCDF 3.6.2 libraries}%
	{http://www.unidata.ucar.edu/software/netcdf/}
using the tarball 
\latexhtml{downloaded from UCAR\footnote{http://www.unidata.ucar.edu/downloads/netcdf/netcdf-3\_6\_2/}}%
        {\htmladdnormallink{downloaded from UCAR}{http://www.unidata.ucar.edu/downloads/netcdf/netcdf-3_6_2/}}.

Once I uncompressed and untarred the package, and went into 
the top-level directory of the package, I built the package by typing
the following at the Unix prompt:

\begin{codeblock}
\codeblockfont{%
./configure --prefix=/Users/jlin/extra/netcdf \\
make check \\
make install}
\end{codeblock}

This installed the netCDF binaries, libraries, and include files into
sub-directories \fn{bin}, \fn{lib}, and \fn{include} in 
the directory specified by \vars{--prefix}.
If you want to install the netCDF libraries in the default
(usually \fn{/usr/local}), just leave out the \vars{--prefix}
option.

Note:  When you build netCDF, make sure the build directory
is not in the directory tree of \vars{--prefix}
(or the default directory \fn{/usr/local}).




%------------------------------------------------------------------------
\subsection{Makefile Configuration}  \label{sec:osx.makefile}

	\subsubsection{NetCDF}

In the \fn{src} directory in the \mods{qtcm} distribution, there is a
sub-directory \fn{Makefiles} that contains the makefiles for a
variety of platforms.  Edit the file \fn{makefile.osx\_g95}
so that the lines specifying the environment variables for the
netCDF libraries and include files:

\begin{codeblock}
\codeblockfont{%
NCINC=-I/Users/jlin/extra/netcdf/include \\
NCLIB=-L/Users/jlin/extra/netcdf/lib -lnetcdf}
\end{codeblock}

are changed to the path where your \emph{manually compiled} 
netCDF libraries and include files are.

Copy \fn{makefile.osx\_g95} from the \fn{Makefiles} sub-directory
in \fn{src} into \fn{src}.  
In other words, from the \mods{qtcm} distribution directory
(i.e., \fn{/buildpath}), at the Unix prompt execute:

\begin{codeblock}
\codeblockfont{%
cp src/Makefiles/makefile.osx\_g95 src/makefile}
\end{codeblock}


	\subsubsection{Linking Order}

Compilers in the GNU Compiler Collection (GCC) search libraries
and object files in the order they are listed in the command-line, 
\latexhtml{from left-to-right\footnote%
        {http://gcc.gnu.org/onlinedocs/gcc-4.1.2/gcc/Link-Options.html\#index-l-670}}%
        {\htmladdnormallink{from left-to-right}{http://gcc.gnu.org/onlinedocs/gcc-4.1.2/gcc/Link-Options.html#index-l-670}}.
Thus, if routines in \fn{b.o} call routines in \fn{a.o}, 
you must list the files in the order \fn{a.o b.o}.

For some reason, that isn't the case for \mods{g95}.  Thus, you will
find \mods{g95} makefile rules structured like the following
(below is part of the rule to create an executable (\fn{qtcm}) for
benchmark runs):

% --- Two versions of this rule, one for display in PDF and the other
%     for display in HTML:
%
\begin{latexonly}
\begin{codeblock}
\codeblockfont{%
qtcm: main.o \\
\hspace*{8ex}\$(FC)~-O~\$(NCINC)~-o~\$@ main.o~\$(QTCMLIB)~\$(NCLIB)}
\end{codeblock}
\end{latexonly}

\begin{htmlonly}
\begin{rawhtml}
<p><code><font color="blue">qtcm: main.o<br>
&nbsp;&nbsp;&nbsp;&nbsp;&nbsp;&nbsp;&nbsp;$(FC) -O $(NCINC) -o 
$@ main.o $(QTCMLIB) $(NCLIB)</font></code></p>
\end{rawhtml}
\end{htmlonly}

even though \fn{main.o} depends on the QTCM library 
(specified in macro setting \vars{\$(QTCMLIB)}), which in turn
depends on the netCDF library (specified in macro setting \vars{\$(NCLIB)}).


%------------------------------------------------------------------------
\subsection{Summary of Steps}   \label{sec:osx.install.summary}

The following summarizes all the steps I took to install
\mods{qtcm} in Mac OS X:

\begin{enumerate}
\item Install
	\htmladdnormallinkfoot{XCode 2.5}%
		{http://developer.apple.com/tools/xcode/}.

\item Install 
	\htmladdnormallinkfoot{MacTeX}{http://www.tug.org/mactex/}, 
	\htmladdnormallinkfoot{MAMP}{http://www.mamp.info/}, and 
	\htmladdnormallinkfoot{TCL/Tk Aqua BI (Batteries Included)}%
		{http://tcltkaqua.sourceforge.net/}.

\item Install
	\htmladdnormallinkfoot{Fink 0.8.1}%
		{http://www.finkproject.org/download/index.php?phpLang=en}.
	Make sure you
	\htmladdnormallink{set up your environment}%
		{http://www.finkproject.org/doc/users-guide/install.php\#setup}
	to enable you to use the packages you install with Fink
	(e.g. \vars{PATH} settings, etc.).
	Most of the time, that just means adding the line
	\cmd{source /sw/bin/init.csh} to your \fn{.cshrc} file (or the
	equivalent in your \fn{.bashrc}).

	Note that for many of the packages needed to run \mods{qtcm},
	you need to 
	\htmladdnormallink{configure Fink to download packages 
		from the unstable trees}%
	{http://www.finkproject.org/faq/usage-fink.php?phpLang=en\#unstable}.
	To do that, add \vars{unstable/main} and \vars{unstable/crypto}
	to the \vars{Trees:} line in \fn{/sw/etc/fink.conf}, and run:

	\begin{codeblock}
	\codeblockfont{fink selfupdate} \\
	\codeblockfont{fink index} \\
	\codeblockfont{fink scanpackages} \\
	\codeblockfont{fink update-all}
	\end{codeblock}

	When \cmd{selfupdate} runs, choose \cmd{rsync} for the
	self update method.  If you do not, Fink will not look in the
	unstable trees for packages.

\item Use Fink to install the \mods{g95} Fortran compiler.
	From a Unix prompt, type:

	\begin{codeblock}
	\codeblockfont{fink -$\,\!$-use-binary-dist install g95}
	\end{codeblock}

\item Use Fink to install Python 
	and the NumPy package (which \mods{f2py} is a part of).
	From a Unix prompt, type:

	\begin{codeblock}
	\codeblockfont{%
	fink -$\,\!$-use-binary-dist install python25 \\
	fink -$\,\!$-use-binary-dist install scipy-core-py25}
	\end{codeblock}

	(Numpy used to be called SciPy Core.)  If you want to
	install Python 2.4 instead, just change the ``25'' and ``py25'' above
	(and in later occurrences) to ``24'' and ``py24'', respectively.
	Note that Fink does not have a version of epydoc for Python 2.4,
	so if you wish to create documentation using epydoc, you will
	need to install Python 2.5.

\item Install teTeX and \LaTeX{2HTML} using Fink.
	From a Unix prompt, type:

	\begin{codeblock}
	\codeblockfont{fink -$\,\!$-use-binary-dist install tetex} \\
	\codeblockfont{fink -$\,\!$-use-binary-dist install latex2html}
	\end{codeblock}

	When prompted, choose ghostscript and ghostscript-fonts to
	satistfy the dependency (which should be the default options).
	I tried choosing system-ghostscript8, but Fink looks for
	ghostscript 8.51 and didn't recognize ghostscript 8.57 that
	was already installed in \fn{/usr/local} (via my MacTeX
	install).  \LaTeX{2HTML} has a package required by the
	\mods{qtcm} manual \LaTeX\ file.

\item Install additional programming and
	scientific packages and libraries using Fink.
	From a Unix prompt, type:

	\begin{codeblock}
	\codeblockfont{%
	fink -$\,\!$-use-binary-dist install scientificpython-py25 \\
	fink -$\,\!$-use-binary-dist install matplotlib-py25 \\
	fink -$\,\!$-use-binary-dist install matplotlib-basemap-py25 \\
	fink -$\,\!$-use-binary-dist install matplotlib-basemap-data-py25 \\
	fink -$\,\!$-use-binary-dist install xaw3d \\
	fink -$\,\!$-use-binary-dist install fftw fftw3 \\
	fink -$\,\!$-use-binary-dist install epydoc-py25 \\
	fink -$\,\!$-use-binary-dist install graphviz \\
	fink -$\,\!$-use-binary-dist install scipy-py25}
	\end{codeblock}

\item Manually install netCDF 3.6.2
	(see Section \ref{sec:netcdf}).

\item From this point on, you can follow the
	general instructions given in Section~\ref{sec:install.sum},
	starting with step~\ref{list:download.qtcm.sum}.
	Please do not ignore, however, Section~\ref{sec:install.macosx}'s
	Mac-specific details.

\end{enumerate}



% ===== end of file =====

	\section{Installing in Ubuntu}         \label{sec:install.ubuntu}
	% ==========================================================================
% Description of installing in Ubuntu
%
% By Johnny Lin
% ==========================================================================


% ------ BODY -----
%
%------------------------------------------------------------------------
\subsection{Introduction}

This section describes installation issues 
I followed to install \mods{qtcm} on my
Dell PowerEdge 860 running Ubuntu GNU/Linux 8.04.1 LTS (Hardy).
The machine has 2.66 GHz Quad Core Intel Xeon processors (64 bit),
4 GB RAM, and 677 GB of disk in its main partition.
This section is a specific realization of the general installation
instructions found in Sections~\ref{sec:install.sum}--\ref{sec:test.qtcm}.
I worked through these installation steps during July 2008.
The best way to go through this section is to go through
the summary of the installation steps in 
Section~\ref{sec:ubuntu.install.summary},
and looking back to other sections as needed.



%------------------------------------------------------------------------
\subsection{Fortran Compiler}     \label{sec:ubuntu.fort.install}

The easiest Fortran compiler to install in Ubuntu 8.04.1 is
\htmladdnormallinkfoot{GNU Fortran}{http://gcc.gnu.org/fortran/}
(\mods{gfortran}), the Fortran 95 compiler included with the more
recent versions of the GNU Compiler Collection (GCC); you can
use any package manager (e.g., \mods{apt-get}, \mods{aptitude})
to install it.
Unfortunately, I was not able to get it to compile QTCM1.
I was, however, able to successfully compile QTCM1 using
the second open-source Fortran compiler,
\htmladdnormallinkfoot{G95}{http://www.g95.org/} (\mods{g95}),
which some feel is farther along in its development than \mods{gfortran}.
G95, however, is not supported as an Ubuntu package, and so I had
to manually install it.

I downloaded the binary version of G95 v0.91 
(the Linux x86\_64/EMT64 with 32 bit default integers) 
using the following
\cmd{curl} command:\footnote%
	{I use \mods{curl} because I usually access my
	Ubuntu server via a terminal session.}

\begin{codeblock}
\codeblockfont{%
\small
curl -o g95.tgz http://ftp.g95.org/v0.91/g95-x86\_64-32-linux.tgz}
\end{codeblock}

which saves the \fn{.tgz} file as the local file \fn{g95.tgz}.
After that, I followed the G95 project's standard
\latexhtml{installation instructions\footnote%
	{http://g95.sourceforge.net/docs.html\#starting}}%
	{\htmladdnormallink{installation instructions}%
		{http://g95.sourceforge.net/docs.html#starting}}
to finish the install.\footnote%
	{The G95 installation instructions say you can put
	\fn{g95-install} anywhere, and make a link to the
	executable \mods{g95} in
	\fn{$\sim$/bin}.  I put \fn{g95-install} in
	\fn{/usr/local}, and while in \fn{/usr/local/bin}, 
	I put a link to the G95 executable using the command:
	\begin{codeblock}
	\codeblockfont{%
	sudo ln -s ../g95-install\_64/bin/x86\_64-suse-linux-gnu-g95 g95.}
	\end{codeblock}}
The regular Linux x86 version of G95
(in \fn{g95-x86-linux.tgz} from the G95 website) did not work on my
machine.




%------------------------------------------------------------------------
\subsection{NetCDF Libraries}   \label{sec:ubuntu.netcdf}

%Here things were very confusing for my machine, as I needed to
%install two versions of the
%\htmladdnormallinkfoot{netCDF}%
%	{http://www.unidata.ucar.edu/software/netcdf/}
%libraries and include files, one 
%for a successful compilation of the extension modules
%(as described in Section~\ref{sec:create.so}),
%and the other 
%for a successful run of the pure-Fortran QTCM1 model
%(used to create the testing benchmarks, as described in
%Section~\ref{sec:test.qtcm}).
%
%The first set of netCDF files (for the extension modules) are
%installed from Ubuntu's package management system.
%These are automatically installed when the \mods{python-netcdf}
%package is installed via an Ubuntu package manager
%(see Section~\ref{sec:ubuntu.install.summary}).
%The include files for this netCDF installation are 
%located in \fn{/usr/include}, and the libraries for this
%netCDF installation are location in \fn{/usr/lib}.

For some reason, the netCDF libraries and include files
installed from the Ubuntu packages do not
correspond to the files needed
by the calling routines in \mods{qtcm}.  To solve this, I compiled
my own set of
\htmladdnormallinkfoot{netCDF 3.6.2 libraries}%
        {http://www.unidata.ucar.edu/software/netcdf/}
using the tarball
\latexhtml{downloaded from UCAR\footnote{http://www.unidata.ucar.edu/downloads/netcdf/netcdf-3\_6\_2/}}%
        {\htmladdnormallink{downloaded from UCAR}{http://www.unidata.ucar.edu/downloads/netcdf/netcdf-3_6_2/}}.

Once I uncompressed and untarred the package, and went into
the top-level directory of the package, I built the package by typing
the following at the Unix prompt:

\begin{codeblock}
\codeblockfont{%
export FC=g95 \\
export FFLAGS="-O -fPIC" \\
export FFLAGS="-fPIC" \\
export F90FLAGS="-fPIC" \\
export CFLAGS="-fPIC" \\
export CXXFLAGS="-fPIC" \\
./configure \\
make check \\
sudo make install}
\end{codeblock}

(The \cmd{export} commands set environment variables for the
Fortran compiler and Fortran and other compiler flags.  The
\vars{-fPIC} flag enables the compilers to create
position independent code, needed for shared libraries in
Ubuntu on a 64 bit Intel processor.)

The above installs the netCDF binaries, libraries, and include files into
sub-directories \fn{bin}, \fn{lib}, and \fn{include} in 
\fn{/usr/local}, the default.
The include files for this netCDF installation are thus
located in \fn{/usr/local/include}, and the libraries for this
netCDF installation are location in \fn{/usr/local/lib}.
(If you want to specify a different installation
location, use the \vars{--prefix} option in \cmd{configure}.)
While you don't have to have root privileges during the configuration
and check steps, you do during the installation step if you're installing
into \fn{/usr/local} (thus the \cmd{sudo} in the last step).\footnote%
	{Note that when you build netCDF, make sure the build directory
	is not in the directory tree of \vars{--prefix}
	or the default directory \fn{/usr/local}.}

%Because there are two different netCDF installations used in the
%\mods{qtcm} package, the makefiles for creating the benchmarks
%and extensions files will have different \vars{NCLIB} and \vars{NCINC}
%environment variables (see Section~\ref{sec:ubuntu.makefile}).




%------------------------------------------------------------------------
\subsection{Makefile Configuration}  \label{sec:ubuntu.makefile}

	\subsubsection{NetCDF}

In the \fn{src} directory in the \mods{qtcm} distribution, there is a
sub-directory \fn{Makefiles} that contains the makefiles for a
variety of platforms.  Edit the file \fn{makefile.ubuntu\_64\_g95}
so that the lines specifying the environment variables for the
netCDF libraries and include files:

\begin{codeblock}
\codeblockfont{%
NCINC=-I/usr/local/include \\
NCLIB=-L/usr/local/lib -lnetcdf}
\end{codeblock}

are changed to the path where your manually compiled
netCDF libraries and include files are.

Copy \fn{makefile.ubuntu\_64\_g95} from the \fn{Makefiles} sub-directory
in \fn{src} into \fn{src}.  
In other words, from the \mods{qtcm} distribution directory
(i.e., \fn{/buildpath}), at the Unix prompt execute:

\begin{codeblock}
\codeblockfont{%
cp src/Makefiles/makefile.ubuntu\_64\_g95 src/makefile}
\end{codeblock}


	\subsubsection{Linking Order}

Compilers in the GNU Compiler Collection (GCC) search libraries
and object files in the order they are listed in the command-line,
\latexhtml{from left-to-right\footnote%
	{http://gcc.gnu.org/onlinedocs/gcc-4.1.2/gcc/Link-Options.html\#index-l-670}}%
	{\htmladdnormallink{from left-to-right}{http://gcc.gnu.org/onlinedocs/gcc-4.1.2/gcc/Link-Options.html#index-l-670}}.
Thus, if routines in \fn{b.o} call routines in \fn{a.o}, 
you must list the files in the order \fn{a.o b.o}.

For some reason, that isn't the case for \mods{g95}.  Thus, you will
find \mods{g95} makefile rules structured like the following
(below is part of the rule to create an executable (\fn{qtcm}) for
benchmark runs):

% --- Two versions of this rule, one for display in PDF and the other
%     for display in HTML:
%
\begin{latexonly}
\begin{codeblock}
\codeblockfont{%
qtcm: main.o \\
\hspace*{8ex}\$(FC)~-O~\$(NCINC)~-o~\$@ main.o~\$(QTCMLIB)~\$(NCLIB)}
\end{codeblock}
\end{latexonly}

\begin{htmlonly}
\begin{rawhtml}
<p><code><font color="blue">qtcm: main.o<br>
&nbsp;&nbsp;&nbsp;&nbsp;&nbsp;&nbsp;&nbsp;$(FC) -O $(NCINC) -o 
$@ main.o $(QTCMLIB) $(NCLIB)</font></code></p>
\end{rawhtml}
\end{htmlonly}

even though \fn{main.o} depends on the QTCM library 
(specified in macro setting \vars{QTCMLIB}), which in turn
depends on the netCDF library (specified in macro setting \vars{NCLIB}).


	\subsubsection{Shared Object PIC}   \label{sec:sopic}

In order to compile the model in Ubuntu on a 64 bit Intel processor,
the model and the netCDF library it is linked to needs to be
compiled to be 
\latexhtml{position independent code (PIC).\footnote%
		{http://www.gentoo.org/proj/en/base/amd64/howtos/index.xml?part=1\&chap=3}}%
	{\htmladdnormallink{position independent code (PIC)}%
		{http://www.gentoo.org/proj/en/base/amd64/howtos/index.xml?part=1&chap=3}.}
This is accomplished with the 
\htmladdnormallinkfoot{\cmd{-fPIC} flag}%
	{http://www.fortran-2000.com/ArnaudRecipes/sharedlib.html}.

In the \mods{qtcm} makefiles, the \cmd{-fPIC} flag is introduced in the
macro \vars{FFLAGSM}, for instance:
\begin{codeblock}
\codeblockfont{%
FFLAGSM = -O -fPIC}
\end{codeblock}
For makefiles used in creating extension modules, \cmd{-fPIC} must
be passed into the \mods{f2py} call.  To do so, put the flags:
\begin{codeblock}
\codeblockfont{%
--f90flags="-fPIC" --f77flags="-fPIC"}
\end{codeblock}
after the \vars{--fcompiler} flag in the \mods{f2py} calling line.

The \cmd{-fPIC} flag must also be used when compiling the netCDF
libraries, as described in Section~\ref{sec:ubuntu.netcdf}.
Failure to create PIC libraries in 64 bit Ubuntu can result in errors 
like the following when creating the \mods{qtcm} extension modules:
\begin{codeblock}
\codeblockfont{%
ld: /usr/local/lib/libnetcdf.a(fort-attio.o): relocation R\_X86\_64\_32 against `a local symbol' can not be used when making a shared object; recompile with -fPIC /usr/local/lib/libnetcdf.a: could not read symbols: Bad value}
\end{codeblock}




%------------------------------------------------------------------------
\subsection{Summary of Steps}      \label{sec:ubuntu.install.summary}

The following summarizes all the steps I took to install
\mods{qtcm} in
Ubuntu 8.04.1 LTS (Hardy) running on a
Quad Core Intel Xeon (64 bit) machine.
Note that while I use the \mods{aptitude} package manager, you are
free to use any manager of your choice (e.g., \mods{apt-get},
\mods{synaptic}, etc.):

\begin{enumerate}
\item Install the G95 Fortran compiler from the binary distribution.
	See Section~\ref{sec:ubuntu.fort.install} for details.

\item Use an Ubuntu package manager
	to install the following packages, by typing:
	\begin{codeblock}
	\codeblockfont{%
sudo aptitude update \\
sudo aptitude install curl \\
sudo aptitude install python-epydoc \\
sudo aptitude install python-matplotlib \\
sudo aptitude install python-netcdf \\
sudo aptitude install python-scientific \\
sudo aptitude install python-scipy \\
sudo aptitude install texlive}
	\end{codeblock}

	Installing \mods{python-scipy} will also install NumPy and
	\mods{f2py}, so you don't have to install the
	\mods{python-numpy} package separately.

	Early-on as I debugged my \mods{qtcm} install on Ubuntu,
	I encountered errors that I thought came from an 
	\htmladdnormallinkfoot{old version of NumPy}%
		{http://cens.ioc.ee/pipermail/f2py-users/2008-June/001617.html},
	and thus I replaced Ubuntu's packaged NumPy with NumPy 1.1.0
	built 
	\latexhtml{directly from source.\footnote%
			{http://sourceforge.net/project/showfiles.php?group\_id=1369\&package\_id=175103}}%
		{\htmladdnormallink{directly from source}{http://sourceforge.net/project/showfiles.php?group_id=1369&package_id=175103}.}
	(Note, you shouldn't install your new NumPy in the default
	location, which may cause problems later-on with Ubuntu's
	package manager.)
	Later on, I concluded the errors I had encountered were not
	because of the NumPy version, but by then I didn't want to
	try to reinstall NumPy again.
	So strictly speaking, the version of Numpy I used is not
	the one bundled with \mods{python-scipy}, but that shouldn't
	be a problem.

\item Manually install netCDF 3.6.2 from source
	(see Section \ref{sec:ubuntu.netcdf}).

\item Manually install the \mods{basemap} package of
	\mods{matplotlib}.  
	The source for the \mods{basemap} toolkit is
	available 
	\latexhtml{from Sourceforge\footnote%
			{http://sourceforge.net/project/showfiles.php?group\_id=80706}}%
		{\htmladdnormallink{from Sourceforge}%
			{http://sourceforge.net/project/showfiles.php?group_id=80706}}
	I obtained version 0.9.9.1 using the
	following \cmd{curl} command:
	\begin{codeblock}
	\codeblockfont{%
\scriptsize
curl -o basemap.tar.gz $\backslash$ \\
http://voxel.dl.sourceforge.net/sourceforge/matplotlib/basemap-0.9.9.1.tar.gz}
	\end{codeblock}

	The \fn{README} file in the \fn{basemap-0.9.9.1} directory has
	detailed installation instructions.  Note that you have to
	install the GEOS library first (\fn{README} has detailed
	directions on how to do that too).  To be on the safe-side,
	I would set the \vars{FC} environment variable to the G95
	compiler
	(e.g., with \cmd{export FC=g95} in Bash).

\item From this point on, you can follow the
	general instructions given in Section~\ref{sec:install.sum},
	starting with step~\ref{list:download.qtcm.sum}.
	Please do not ignore, however, Section~\ref{sec:install.ubuntu}'s
	Ubuntu-specific details.

\end{enumerate}



% ===== end of file =====


\chapter{Getting Started With \mods{qtcm}}  \label{ch:getting.started}
% ==========================================================================
% Getting Started With qtcm
%
% By Johnny Lin
% ==========================================================================


% ------ BODY -----
%
%---------------------------------------------------------------------
\section{Your First Model Run}

Figure~\ref{fig:my.first.run} shows an example of a script to make
a 30 day seasonal, aquaplanet model run, with run name ``test'',
starting from November 1, Year 1.


%--- Two versions, one for PDF, one for HTML:
\begin{latexonly}
\begin{figure}[htp]
\begin{center}
\begin{codeblock}
\codeblockfont{%
from qtcm import Qtcm \\
inputs = \{\} \\
inputs['runname'] = 'test' \\
inputs['landon'] = 0 \\
inputs['year0'] = 1 \\
inputs['month0'] = 11 \\
inputs['day0'] = 1 \\
inputs['lastday'] = 30 \\
inputs['mrestart'] = 0 \\
inputs['compiled\_form'] = 'parts' \\
model = Qtcm(**inputs) \\
model.run\_session()}
\end{codeblock}
\end{center}
\caption{An example of a simple \mods{qtcm} run.}
\label{fig:my.first.run}
\end{figure}
\end{latexonly}

\begin{htmlonly}
\label{fig:my.first.run}
\begin{center}
\htmlfigcaption{%
	\codeblockfont{%
from qtcm import Qtcm \\
inputs = \{\} \\
inputs['runname'] = 'test' \\
inputs['landon'] = 0 \\
inputs['year0'] = 1 \\
inputs['month0'] = 11 \\
inputs['day0'] = 1 \\
inputs['lastday'] = 30 \\
inputs['mrestart'] = 0 \\
inputs['compiled\_form'] = 'parts' \\
model = Qtcm(**inputs) \\
model.run\_session()}
	}

\htmlfigcaption{Figure~\ref{fig:my.first.run}:
	An example of a simple \mods{qtcm} run.}
\end{center}
\end{htmlonly}



The class describing the QTCM1 model is \class{Qtcm}.  An instance
of \class{Qtcm}, in this example \vars{model}, is created the same
way you create an instance of any class.  When instantiating an
instance of \class{Qtcm}, keyword parameters can be used to override
any default settings.  In the example above, the dictionary
\vars{inputs} specifying all keyword parameters is passed in on the
instantiation of \vars{model}.

The keyword parameter settings in
Figure~\ref{fig:my.first.run} have the following meanings:
\begin{itemize}
\item \vars{runname}:  This string (``test'') is used in the
	output filename.  QTCM1 writes mean and instantaneous
	output files to the directory given in \vars{model.outdir.value},
	with filenames 
	\fn{qm\_}\dumarg{runname}\fn{.nc} for mean output and
	\fn{qi\_}\dumarg{runname}\fn{.nc} for instantaneous output.

\item \vars{landon}: When set to ``0'', the land is turned off and
	the run is an aquaplanet run.  When set to ``1'', the land
	model is turned on.

\item \vars{year0}:  The year the run starts on.

\item \vars{month0}:  The month the run starts on (11 = November).

\item \vars{day0}: The day of the month the run starts on.

\item \vars{lastday}:  The model runs from day 1 to \vars{lastday}.

\item \vars{mrestart}:  When set to ``0'', the run starts from
	default initial conditions
	(see Section~\ref{sec:initial.variables} for a table of
	those values).
	When set to ``1'', the run starts from a restart file.

\item \vars{compiled\_form}:  This keyword sets what form the
	compiled QTCM1 model has, and its value is saved to
	the instance's \vars{compiled\_form} attribute.
	It is a string and can be set either to
	``parts'' or ``full''.  Most of the time, you will want
	to set it to \vars{'parts'}.
	This keyword is the only one
	that must be specified on instantiation; the model instance
	will at least instantiate
	using only the default settings for all the other keyword
	parameters (given in Appendix~\ref{app:defaults.values}).
	See Section~\ref{sec:compiledform} for details about
	what the \vars{compiled\_form} attribute controls.
\end{itemize}

By default, the \vars{SSTmode} attribute, which controls whether the
model will use climatological sea-surface temperatures (SST) 
or real SSTs, is set to the \vars{value} ``seasonal'', thus giving a
run with seasonal forcing at the lower-boundary over the ocean.

This example assumes that the boundary condition files, sea surface
temperature files, and the model output directories are as specified
in submodule \mods{defaults}.  Those values are described in
Section~\ref{sec:defaults.scalar}.




%---------------------------------------------------------------------
\section{Managing Directories}

Most of the time, your boundary condition files and output files
will not be in the locations specified in
Section~\ref{sec:defaults.scalar}, or in the directory your
Python script resides.  The easiest way to tell your \class{Qtcm} 
instance where your input/output files are is to pass them in
as keyword parameters on instantiation.


%--- Two versions, one for PDF, one for HTML:
\begin{latexonly}
\begin{figure}[htp]
\begin{codeblock}
\codeblockfont{%
\small
from qtcm import Qtcm \\
rundirname = 'test' \\
dirbasepath = os.path.join(os.getcwd(), rundirname) \\
inputs = \{\} \\
inputs['bnddir'] = os.path.join( os.getcwd(), 'bnddir', \\
\hspace*{40ex}'r64x42' ) \\
inputs['SSTdir'] = os.path.join( os.getcwd(), 'bnddir', \\
\hspace*{40ex}'r64x42', 'SST\_Reynolds' ) \\
inputs['outdir'] = dirbasepath \\
inputs['runname'] = rundirname \\
inputs['landon'] = 0 \\
inputs['year0'] = 1 \\
inputs['month0'] = 11 \\
inputs['day0'] = 1 \\
inputs['lastday'] = 30 \\
inputs['mrestart'] = 0 \\
inputs['compiled\_form'] = 'parts' \\
model = Qtcm(**inputs) \\
model.run\_session()}
\end{codeblock}

\caption{An example \mods{qtcm} run showing detailed description of
	input and output directories.}
\label{fig:manage.dir.example}
\end{figure}
\end{latexonly}

\begin{htmlonly}
\label{fig:manage.dir.example}
\begin{center}
\htmlfigcaption{%
	\codeblockfont{%
from qtcm import Qtcm \\
rundirname = 'test' \\
dirbasepath = os.path.join(os.getcwd(), rundirname) \\
inputs = \{\} \\
inputs['bnddir'] = os.path.join( os.getcwd(), 'bnddir', \\
\hspace*{40ex}'r64x42' ) \\
inputs['SSTdir'] = os.path.join( os.getcwd(), 'bnddir', \\
\hspace*{40ex}'r64x42', 'SST\_Reynolds' ) \\
inputs['outdir'] = dirbasepath \\
inputs['runname'] = rundirname \\
inputs['landon'] = 0 \\
inputs['year0'] = 1 \\
inputs['month0'] = 11 \\
inputs['day0'] = 1 \\
inputs['lastday'] = 30 \\
inputs['mrestart'] = 0 \\
inputs['compiled\_form'] = 'parts' \\
model = Qtcm(**inputs) \\
model.run\_session()}
	}

\htmlfigcaption{Figure~\ref{fig:manage.dir.example}:
	An example \mods{qtcm} run showing detailed description of
        input and output directories.}
\end{center}
\end{htmlonly}


Figure~\ref{fig:manage.dir.example} shows an example run where those
directories are explicitly specified; in all other aspects, the run
is identical to the one in Figure~\ref{fig:my.first.run}.
In Figure~\ref{fig:manage.dir.example}, output from the model is
directed to the directory described by string variable
\vars{dirbasepath}.  \vars{dirbasepath} is created by joining the
current working directory with the run name given in string variable
\vars{rundirname}.\footnote%
	{The Python \mods{os} module enables platform-independent
	handling of files and directories.  The \mods{os.path.join}
	function resolves paths without the programmer needing to know
	all the possible directory separation characters; the function
	chooses the correct separation character at runtime.  The
	\mods{os.getcwd} function returns the current working directory.}
Setting keyword parameter \vars{outdir} to \vars{dirbasepath} sends
output to \vars{dirbasepath}.  
Keywords \vars{bnddir} and \vars{SSTdir} specify the directories
where non-SST and SST boundary condition files, respectively, are
found.

Interestingly, the default version of QTCM1 does \emph{not} send
all output from the model to \vars{outdir}.  The restart file
\fn{qtcm\_}\dumarg{yyyymmdd}\fn{.restart} (where \dumarg{yyyymmdd}
is the year, month, and day of the model date when the restart
file was written) is written into the current working directory,
not the output directory.  Thus, if you do multiple runs, you'll
have to manually deal with the restart files that will proliferate.

Neither the QTCM1 model nor the \class{Qtcm} object
create the directories specified in \mods{bnddir}, \mods{SSTdir},
and \mods{outdir}.  Failure to do so will create an error.  I use
Python's file management tools to make sure the output directory
is created, and any old output files are deleted.  Here's an example
that does that, using the \vars{dirbasepath} and \vars{rundirname}
variables from Figure~\ref{fig:manage.dir.example}:

\begin{codeblock}
\codeblockfont{%
\small
if not os.path.exists(dirbasepath):  os.makedirs(dirbasepath) \\
qi\_file = os.path.join( dirbasepath, 'qi\_'+rundirname+'.nc' ) \\
qm\_file = os.path.join( dirbasepath, 'qm\_'+rundirname+'.nc' ) \\
if os.path.exists(qi\_file):   os.remove(qi\_file) \\
if os.path.exists(qm\_file):   os.remove(qm\_file)}
\end{codeblock}




%---------------------------------------------------------------------
\section{Model Field Variables}   \label{sec:field.variables.intro}

The term ``field'' variable refers to QTCM1 model variables that 
are accessible at both the compiled Fortran QTCM1 model-level as
well as the Python \class{Qtcm} instance-level.
Field variables are all instances of the \class{Field} class,
and are stored as attributes of the \class{Qtcm} instance.\footnote%
	{Note non-field variables can also be instances of \class{Field},
	and that \class{Qtcm} instances have other attributes that are
	not equal to \class{Field} instances.}

\class{Field} class instances have the following attributes:
\begin{itemize}
\item \vars{id}:  A string naming the field (e.g., ``Qc'', ``mrestart'').
	This string should contain no whitespace.
\item \vars{value}:  The value of the field.  Can be of any type, though
	typically is either a string or numeric scalar or a numeric array.
\item \vars{units}:  A string giving the units of the field.
\item \vars{long\_name}:  A string giving a description of the field.
\end{itemize}

\class{Field} instances also have methods to return the rank 
and typecode of \vars{value}.

Remember, if you want to access the value of a \class{Field} object,
make sure you access that object's \vars{value} attribute.  
Thus, for example,
to assign a variable \vars{foo} to the
\vars{lastday} value for a given
\class{Qtcm} instance \vars{model}, type the following:
\begin{codeblock}
\codeblockfont{%
foo = model.lastday.value}
\end{codeblock}

For scalars, this assignment sets \vars{foo} by value (i.e., a copy
of the value of attribute \vars{model.lastday} is set to \vars{foo}).
In general, however, Python assigns variables by reference.  Use
the \mods{copy} module if you truly want a copy of a field variable's
value (such as an array), rather than an alias.  For more details
about field variables, see Section~\ref{sec:field.variables}.




%---------------------------------------------------------------------
\section{Run Sessions}

	\subsection{What is a Run Session?}

A run session is a unit of simulation where the model is run from
day 1 of simulation to the day specified by the \vars{lastday}
attribute of a \class{Qtcm} instance.  A run session is a
``complete'' model run, at the beginning of which all compiled QTCM1
model variables are set to the values given at the Python-level,
and at the end of which restart files are written, the values
at the Python-level are overwritten by the values in the Fortran
model, and a Python-accessible snapshot is taken of the 
model variables that were written to the restart file.


	\subsection{Changing Variables}

Between run sessions, changing any field variable is as easy
as a Python assignment.  For instance, to change the atmosphere
mixed layer depth to 100~m, just type:
\begin{codeblock}
\codeblockfont{%
model.ziml.value = 100.0}
\end{codeblock}

When changing arrays, be careful to try to match the shape of the 
array.\footnote%
	{At the very least, match the rank of the array, which is required
	for the routines in \mods{setbypy} to properly choose which
	Fortran subroutine to use in reading the Python value.
	I haven't tested if only the rank is needed, however,
	for the passing to work, for a continuation run (my hunch is
	it won't).}
You can use the NumPy \mods{shape} function on a NumPy array to
check its shape.


	\subsection{Continuing a Model Run}  \label{sec:continuation.intro}

Figure~\ref{fig:continuation.example} shows an example of two run
sessions, where the second run session is a continuation of the
first.


%--- Two versions, one for PDF, one for HTML:
\begin{latexonly}
\begin{figure}[htp]
\begin{codeblock}
\codeblockfont{%
\small
inputs['year0'] = 1 \\
inputs['month0'] = 11 \\
inputs['day0'] = 1 \\
inputs['lastday'] = 10 \\
inputs['mrestart'] = 0 \\
inputs['compiled\_form'] = 'parts' \\ \\
model = Qtcm(**inputs) \\
model.run\_session() \\
model.u1.value = model.u1.value * 2.0 \\
model.init\_with\_instance\_state = True \\
model.run\_session(cont=30)}
\end{codeblock}

\caption{An example of two \mods{qtcm} run sessions where the second
	run session is a continuation of the first.  Assume 
	\vars{inputs} is a dictionary, and that earlier in the
	script the run name and
	all input and output directory names were added
	to the dictionary.}
\label{fig:continuation.example}
\end{figure}
\end{latexonly}

\begin{htmlonly}
\label{fig:continuation.example}
\begin{center}
\htmlfigcaption{%
	\codeblockfont{%
inputs['year0'] = 1 \\
inputs['month0'] = 11 \\
inputs['day0'] = 1 \\
inputs['lastday'] = 10 \\
inputs['mrestart'] = 0 \\
inputs['compiled\_form'] = 'parts' \\ \\
model = Qtcm(**inputs) \\
model.run\_session() \\
model.u1.value = model.u1.value * 2.0 \\
model.init\_with\_instance\_state = True \\
model.run\_session(cont=30)}
	}

\htmlfigcaption{Figure~\ref{fig:continuation.example}:
	An example of two \mods{qtcm} run sessions where the second
	run session is a continuation of the first.  Assume 
	\vars{inputs} is a dictionary, and that earlier in the
	script the run name and
	all input and output directory names were added
	to the dictionary.}
\end{center}
\end{htmlonly}


The first run session runs from day 1 to day 10.  The second
run session runs the model for another 30 days.  
Setting the \vars{init\_with\_instance\_state} of
\vars{model} to \vars{True} tells the model to use the
the values of the instance attributes 
(for prognostic variables, right-hand sides, and start date) 
are currently stored \vars{model}
as the initial values for the run\_session.\footnote%
	{Unless overridden, by default, 
	\vars{init\_with\_instance\_state} is set
	to True on \class{Qtcm} instance instantiation.}
The \vars{cont}
keyword in the second \mods{run\_session} call specifies a
continuation run, and the value gives the number of additional
days to run the model.

The set of runs described above would produce the exact same
results as if you had gone into the Fortran model after 10 days,
doubled the first baroclinic mode zonal velocity, and continued
the run for another 30 days.  With the Python example above, however,
you didn't need to know you were going to do that ahead of starting
the model run (which is what a compiled model requires you to do).
Section~\ref{sec:contination.run.sessions} describes continuation
runs in detail.


	\subsection{Passing Restart Snapshots Between Run Sessions}
					\label{sec:snapshot.intro}

The pure-Fortran QTCM1 uses a restart file to enable continuation
runs.  A \class{Qtcm} instance can also make use of that option,
through setting the \vars{mrestart} attribute value
(see Section~\ref{sec:contination.run.sessions} and
Neelin et al.\ \cite{Neelin/etal:2002} for details).  
It's easier, however, instead of using a restart file, to pass 
along a ``snapshot'' dictionary.

The \class{Qtcm} instance method \mods{make\_snapshot} copies the
variables that would be written out to a restart file into a
dictionary that is saves as the instance attribute \vars{snapshot}.
This snapshot can be saved separately, for later recall.  Note that
snapshots are automatically made at the end of a run session.

The following example shows a model \mods{run\_session} call,
following which the snapshot is saved to the variable
\vars{snapshot}:\footnote%
	{Remember Python assignment defaults to assignment by
	reference, so in this example the variable \vars{mysnapshot}
	is a pointer to the \vars{model.snapshot} attribute.
	(However, note that \vars{model.snapshot} itself is not a
	reference, but a distinct copy of those variables; to do
	otherwise would result in a non-static snapshot.)
	If the \vars{model.snapshot} attribute is dereferenced,
	then \vars{mysnapshot} will become the sole pointer to the
	dictionary.}

\begin{codeblock}
\codeblockfont{%
model.run\_session() \\
mysnapshot = model.snapshot}
\end{codeblock}

After taking the snapshot, you might continue the run a while, and
then decide to return to the snapshot you saved.  To do so, use
the \mods{sync\_set\_py\_values\_to\_snapshot}
method to reset the model instance values to
\vars{mysnapshot} before your next run session:
\begin{codeblock}
\codeblockfont{%
model.sync\_set\_py\_values\_to\_snapshot(snapshot=mysnapshot) \\
model.init\_with\_instance\_state = True \\
model.run\_session()}
\end{codeblock}

See Section~\ref{sec:snapshots} for details regarding the use of
snapshots, as well as for a list of what variables are saved in
a snapshot.




%---------------------------------------------------------------------
\section{Creating Multiple Models}

	\subsection{Model Instances}

Creating a new QTCM1 model is as simple as creating another
\class{Qtcm} instance.
For instance, to instantiate two QTCM1
models, \vars{model1} and \vars{model2}, type the following:

\begin{codeblock}
\codeblockfont{%
from qtcm import Qtcm \\
model1 = Qtcm(compiled\_form='parts') \\
model2 = Qtcm(compiled\_form='parts')}
\end{codeblock}

\vars{model1} and \vars{model2} do \emph{not} share any variables
in common, including the extension modules holding the Fortran
code.  In creating the instances, a copy of the extension modules
are saved in temporary directories.


	\subsection{Passing Snapshots To Other Models}

The snapshots described in Section~\ref{sec:snapshot.intro}
can also be passed around to other model instances,
enabling you to easily branch a model run:

\begin{codeblock}
\codeblockfont{%
model.run\_session() \\
mysnapshot = model.snapshot \\
model1.sync\_set\_py\_values\_to\_snapshot(snapshot=mysnapshot) \\
model2.sync\_set\_py\_values\_to\_snapshot(snapshot=mysnapshot) \\
model1.run\_session() \\
model2.run\_session()}
\end{codeblock}

The state of \vars{model} after its run session is used to start
\vars{model1} and \vars{model2}.  This is an easy way to save time
in spinning-up multiple models.




%---------------------------------------------------------------------
\section{Run Lists}		\label{sec:runlist.intro}

This feature of \class{Qtcm} objects is what really gives 
\class{Qtcm} model instances their flexibility.
A run list is a list of strings and dictionaries that specify
what routines to run in order to execute a particular part of
the model.  Each element of the run list specifies the method
or subroutine to execute, and the order of the elements specifies
their execution order.

For instance, the standard run list for initializing the the
atmospheric portion of the model is named ``qtcminit'', and
equals the following list:

\begin{latexonly}
\begin{codeblock}
\codeblockfont{%
\parbox{46ex}{\input{qtcminit_runlist}}}
\end{codeblock}
\end{latexonly}

\begin{htmlonly}
\begin{quotation}
\input{qtcminit_runlist}
\end{quotation}
\end{htmlonly}

This list is stored as an entry in the \vars{runlists} dictionary
(with key \vars{'qtcminit'}).
\vars{runlists} is an attribute of a \class{Qtcm} instance.
Table~\ref{tab:stnd.runlists} lists all standard run lists.

When the run list element in the list is a string, the string gives the
name of the routine to execute.  The routine has no parameter
list.  The routine can be a
compiled QTCM1 model subroutine for which an interface has been
written (e.g., \mods{\_\_qtcm.wrapcall.wparinit}), 
a method of the of the Python model instance 
(e.g., \mods{varinit}), or another run list
(e.g., \vars{atm\_physics1}).

When the run list element is a 1-element dictionary, the key of
the dictionary element is the name of the routine, and the value
of the dictionary element is a list specifying input parameters
to be passed to the routine on call.  Thus, the element:
\begin{codeblock}
\codeblockfont{%
{\{'\_\_qtcm.wrapcall.wtimemanager': [1]\}}}
\end{codeblock}
calls the \mods{\_\_qtcm.wrapcall.wtimemanager} routine, passing in
one input parameter, which in this case is the value 1.

If you want to change the order of the run list, just change the
order of the list.  To add or remove routines to be executed, just
add and remove their names from the run list.
Python provides a number of methods to manipulate
lists (e.g., \mods{append}).  Since lists are dynamic data types
in Python, you do not have to do any recompiling to implement
the change.

The \vars{compiled\_form} attribute must be set to \vars{'parts'}
in the \class{Qtcm} instance in order to take advantage of the run
lists feature of the class.  Run lists are not available for
\vars{compiled\_form\thinspace=\thinspace'full'}, because subroutine
calls are hardwired in the compiled QTCM1 model Fortran code in
that case.




%---------------------------------------------------------------------
\section{Model Output}			\label{sec:output.intro}

	\subsection{NetCDF Output}

Model output is written to netCDF files in the directory
specified by the \class{Qtcm} instance attribute \vars{outdir}.
Mean values are written to an output file beginning with
\fn{qm\_}, and instantaneous values are written to an output
file beginning with \fn{qi\_}.

The frequency of mean output is controlled by \vars{ntout}, and the
frequency of instantaneous output is controlled by \vars{ntouti}.
\vars{ntout.value} gives the number of days over which to average
(and if equals \vars{-30}, monthly means are calculated).
\vars{ntouti.value} gives the frequency in days that instantaneous
values are output (monthly if it equals \vars{-30}).  (See
Section~\ref{sec:initial.variables} for a description of other
output-control variables, and see the QTCM1 manual \cite{Neelin/etal:2002}
for a detailed description of how these variables control output.)

Figure~\ref{fig:netcdf.read} gives an example of a block of code
to read netCDF output, where \vars{datafn} is the netCDF filename, and
\vars{id} is the string name of the field variable (e.g.,
\vars{'u1'}, \vars{'T1'}, etc.).
(Note that the netCDF identifier for field variables is the same as
the name in \class{Qtcm}, except for the variables given in
Table~\ref{tab:qtcm.netcdf.ids}.)

In the code in Figure~\ref{fig:netcdf.read},
the array value is read into \vars{data}, and the longitude values, 
latitude values, and time values are read into variables
\vars{lon}, \vars{lat}, and \vars{time}, respectively.
As netCDF files also hold metadata, a description and the units
of the variable given by \vars{id}, and each dimension, are read
into variables ending in \vars{\_name} and \vars{\_units},
respectively.


%--- Two versions, one for PDF, one for HTML:
\begin{latexonly}
\begin{figure}[htp]
\begin{codeblock}
\codeblockfont{%
import numpy as N \\
import Scientific as S \\ \\
fileobj = S.NetCDFFile(datafn, mode='r') \\ \\
data = N.array(fileobj.variables[id].getValue()) \\
data\_name = fileobj.variables[id].long\_name \\
data\_units = fileobj.variables[id].units \\ \\
lat = N.array(fileobj.variables['lat'].getValue()) \\
lat\_name = fileobj.variables['lat'].long\_name \\
lat\_units = fileobj.variables['lat'].units \\ \\
lon = N.array(fileobj.variables['lon'].getValue()) \\
lon\_name = fileobj.variables['lon'].long\_name \\
lon\_units = fileobj.variables['lon'].units \\ \\
time = N.array(fileobj.variables['time'].getValue()) \\
time\_name = fileobj.variables['time'].long\_name \\
time\_units = fileobj.variables['time'].units \\ \\
fileobj.close()}
\end{codeblock}

\caption{Example of Python code to read netCDF output.
	See text for description.}
\label{fig:netcdf.read}
\end{figure}
\end{latexonly}

\begin{htmlonly}
\label{fig:netcdf.read}
\begin{center}
\htmlfigcaption{%
	\codeblockfont{%
import numpy as N \\
import Scientific as S \\ \\
fileobj = S.NetCDFFile(datafn, mode='r') \\ \\
data = N.array(fileobj.variables[id].getValue()) \\
data\_name = fileobj.variables[id].long\_name \\
data\_units = fileobj.variables[id].units \\ \\
lat = N.array(fileobj.variables['lat'].getValue()) \\
lat\_name = fileobj.variables['lat'].long\_name \\
lat\_units = fileobj.variables['lat'].units \\ \\
lon = N.array(fileobj.variables['lon'].getValue()) \\
lon\_name = fileobj.variables['lon'].long\_name \\
lon\_units = fileobj.variables['lon'].units \\ \\
time = N.array(fileobj.variables['time'].getValue()) \\
time\_name = fileobj.variables['time'].long\_name \\
time\_units = fileobj.variables['time'].units \\ \\
fileobj.close()}
	}

\htmlfigcaption{Figure~\ref{fig:netcdf.read}:
	Example of Python code to read netCDF output.
	See text for description.}
\end{center}
\end{htmlonly}





\begin{table}[tp]
\begin{center}
\begin{tabular}{l|l}
\textbf{\class{Qtcm} Attribute Name} & \textbf{NetCDF Output Name} \\
\hline
\vars{'Qc'}                & \vars{'Prec'} \\
\vars{'FLWut'}             & \vars{'OLR'} \\
\vars{'STYPE'}             & \vars{'stype'}
\end{tabular}
\end{center}
\caption{NetCDF output names for \class{Qtcm} field variables that
	are different from the \class{Qtcm} and compiled QTCM1 model
	variable names.  The netCDF names are case-sensitive.}
\label{tab:qtcm.netcdf.ids}
\end{table}


\emphpara{NB:}  All netCDF array output is dimensioned (time, latitude,
longitude) when read into Python using the \mods{Scientific} package.
This differs from the way \class{Qtcm} saves field variables, which
follows Fortran convention (longitude, latitude).  Please be careful
when relating the two types of arrays.
Section~\ref{sec:field.var.shape} for a discussion of why there is
this discrepancy.


	\subsection{Visualization}	\label{sec:viz.intro}

The \mods{plotm} method of \class{Qtcm} instances creates line
plots or contour plots, as appropriate, of model output of
average fields of run session(s) associated with the instance.
Some examples, assuming \vars{model} is an instance of \class{Qtcm}
and has already executed a run session:
\begin{itemize}
\item \cmd{model.plotm('Qc', lat=1.875)}:
	A time vs.\ longitude contour
          plot is made for the full range of time and longitude,
          at the latitude 1.875 deg N, for mean precipitation.

\item \cmd{model.plotm('Qc', time=10)}:
	A latitude vs.\ longitude contour plot of precipitation
	is made for the full spatial domain at day 10 of the model run.

\item \cmd{model.plotm('Evap', lat=1.875, lon=[100,200])}:  A contour
	plot of time vs.\ longitude of evaporation is made for the
          longitude points between 100 and 200 degrees E, at the
          latitude 1.875 deg N.  

\item \cmd{model.plotm('cl1', lat=1.875, lon=[100,200], time=20)}:
          A deep cloud amount vs.\ longitude line plot is made for
          the longitude points between 100 and 200 degrees east,
          at the latitude 1.875 deg N, at day 20 of the model run.
\end{itemize}

In these examples, the number of days over which the mean is taken
equals \vars{model.ntout.value}.
Also, the \mods{plotm} method automatically takes into account the
\class{Qtcm}/netCDF variable differences described in
Table~\ref{tab:qtcm.netcdf.ids}.



%---------------------------------------------------------------------
\section{Documentation}

Section~\ref{sec:ver} gives the online locations of the
transparent copies of this manual.  
Model formulation is fully described in
Neelin \& Zeng \cite{Neelin/Zeng:2000} and model
results are described in Zeng et~al.\ \cite{Zeng/etal:2000}
(\cite{Neelin/Zeng:2000} is based upon v2.0 of QTCM1
and \cite{Zeng/etal:2000} is based on QTCM1 v2.1).
Additional documentation you'll find useful include:

\begin{itemize}
\item \latexhtml{%
\htmladdnormallinkfoot{The \mods{qtcm} Package API Documentation}%
        {http://www.johnny-lin.com/py\_pkgs/qtcm/doc/html-api/}}%
{\htmladdnormallink{The \mods{qtcm} Package API Documentation}%
        {http://www.johnny-lin.com/py_pkgs/qtcm/doc/html-api/}}

\item \latexhtml{%
\htmladdnormallinkfoot{The Pure-Fortran QTCM1 Manual}%
        {http://www.atmos.ucla.edu/$\sim$csi/qtcm\_man/v2.3/qtcm\_manv2.3.pdf}}%
{\htmladdnormallink{The Pure-Fortran QTCM1 Manual}%
        {http://www.atmos.ucla.edu/~csi/qtcm_man/v2.3/qtcm_manv2.3.pdf}}
\cite{Neelin/etal:2002}

\end{itemize}



% ===== end of file =====


\chapter{Using \mods{qtcm}}                 \label{ch:using}
% ==========================================================================
% Using QTCM
%
% By Johnny Lin
% ==========================================================================


% ------ BODY -----
%
%---------------------------------------------------------------------
\section{Introduction}

Now that you've successfully run your first model instances, in
this chapter I provide detailed explanations regarding the features
of \mods{qtcm}.  I present these explanations in a documentary
rather than didactic fashion; my goal is to document how the features
work.  More details are given in the code docstrings.  At the end
of the chapter, in Section~\ref{sec:cookbook}, I provide a few
cookbook ideas/examples of ways to use the model.




%---------------------------------------------------------------------
\section{Model Instances}  \label{sec:model.instances}

An instance of a \class{Qtcm} model is created in \mods{qtcm} the same way
you create an instance of any class.
For instance, to instantiate two \class{Qtcm}
models, \vars{model1} and \vars{model2}, I type the following:

\begin{codeblock}
\codeblockfont{%
from qtcm import Qtcm \\
model1 = Qtcm(compiled\_form\thinspace=\thinspace'full') \\
model2 = Qtcm(compiled\_form\thinspace=\thinspace'parts')}
\end{codeblock}

In the above example, \vars{model1} uses the compiled QTCM1 model
that runs the model (essentially) using the Fortran driver,
while \vars{model2} uses the compiled QTCM1 model where execution
order and content all the way down to the atmospheric timestep level
is controlled by Python run lists.  (Section~\ref{sec:compiledform}
has more details about the difference between compiled forms.)

For each instance of \class{Qtcm}, copies of all needed extension
modules (e.g., \fn{.so} files) are copied to a temporary directory
that is automatically created by Python.  The full path name of
that directory is saved in the instance attribute \vars{sodir}.
These extension modules are then associated with the specific instance 
through private instance attributes,
and thus every instance of \class{Qtcm} has its own separate variable
and name space on both the Fortran and Python sides.\footnote%
	{The private instance attribute is \vars{\_\_qtcm}.
	See Section~\ref{sec:Qtcm.private.attrib} for details about 
	private \class{Qtcm} instance attributes.}
The temporary directory and all of its contents are deleted when the 
model instance is deleted.

On instantiation, \class{Qtcm} instances set all scalar field
variables to their default values as given in the submodule
\mods{defaults} (and listed in Section~\ref{sec:defaults.scalar}),
and assign the fields as instance attributes.  The instance attribute
\vars{init\_with\_instance\_state} is set to True by default, unless
overridden on instantiation.




%---------------------------------------------------------------------
\section{Initializing a Model Run}

In the pure-Fortran QTCM1, there are three broad
classes of initialized variables:
\begin{enumerate}
\item Those that are read-in using a namelist, 
\item Those that the are read-in from a restart file, and
\item Those that are set by assignment in the Fortran code.  
\end{enumerate}
These variables are a combination of scalars and arrays.

For \mods{qtcm}, interfaces were built so that all classes of
initialized variables that could be user-controlled are accessible
and changeable at the Python-level.  For \mods{qtcm},
the set of variables that could be changed is also expanded, to
include not just the first and second classes of pure-Fortran
QTCM1 initialized variables.  This was done to make \mods{qtcm}
more flexible.  All variables that can be passed between the
compiled QTCM1 model and Python model levels are called
field variables, and are described in detail in
Section~\ref{sec:field.variables}.

As it happens, all the namelist-set variables are scalars.  In the
pure-Fortran QTCM1, those variables are given default values prior
to reading in of the namelist.  To duplicate this functionality,
on model instantiation, all scalar fields are set to their default
values as given in the submodule \mods{defaults} and listed in
Section~\ref{sec:defaults.scalar}.  Most of the default values in
\mods{defaults} are the same as in the pure-Fortran QTCM1, but
there are a few differences.\footnote%
	{One difference being \vars{mrestart}, which in \vars{qtcm} 
	will have the value of 0 in contrast to the pure-Fortran 
	QTCM1 where the default is the 1.}
This setting of scalar defaults is the same for both
\vars{compiled\_form\thinspace=\thinspace'full'} and
\vars{compiled\_form\thinspace=\thinspace'parts'} instances.
Of course, all
\mods{qtcm} fields are user-controllable, both via keyword input
parameters at model instantiation as well as through direct
manipulation of the instance attribute that stores the field variable.

The pure-Fortran QTCM1 initialized prognostic variables and
right-hand sides are set in the Fortran subroutine \mods{varinit}.
Their they are read-in from a restart file or, as default,
set by assignment.
In \mods{qtcm}, the same variables are initialized by a \class{Qtcm}
instance method of the same name, \mods{varinit}, for the case when
\vars{compiled\_form\thinspace=\thinspace'parts'}.  For the case
of \vars{compiled\_form\thinspace=\thinspace'full'}, the compiled
QTCM1 subroutine that is the same as in the pure-Fortran case is
used, and that routine is inaccessible at the Python level.
See Section~\ref{sec:snapshots}'s listing of snapshot variables,
which also includes the prognostic variables and right-hand sides that
are set in \mods{varinit} (both Fortran and Python).




%---------------------------------------------------------------------
\section{The \vars{compiled\_form} Keyword}  \label{sec:compiledform}

The \mods{qtcm} package is a Python wrap of the Fortran routines
that make up QTCM1.  The wrapping layer adds flexibility and
functionality, but at the cost of speed.  Thus, I created two
types of extension modules from the Fortran QTCM1 code, one
which permits very little control over the compiled Fortran
\emph{routines} at the Python level, and one that allows the Python-level
to control model execution in the compiled QTCM1 model
all the way down to the atmospheric timestep level.\footnote%
	{That control is via run lists, which are described in
	Section~\ref{sec:runlists}.}
The former extension module corresponds to 
\vars{compiled\_form\thinspace=\thinspace'full'} and
the latter extension module to
\vars{compiled\_form\thinspace=\thinspace'parts'}.

For \vars{compiled\_form\thinspace=\thinspace'full'},
the compiled portion of the model encompasses (nearly) the
entire QTCM1 model as a whole.  Thus, the only compiled QTCM1 model
modules or subroutines that Python should interact with is
the \mods{driver} routine (which executes the entire model) and
the \mods{setbypy} module (which enables communication between the
compiled model and the Python-level of model fields.\footnote%
	{The \mods{setbypy} Python module is the wrap of the
	Fortran QTCM1 \mods{SetByPy} module.}

For \vars{compiled\_form\thinspace=\thinspace'parts'}, the compiled
portion of the model does not encompasses the model as a whole, but
rather is broken up into separate units (as appropriate) all the
way down to an atmosphere timestep.  Thus, compiled QTCM1 model
modules/subroutines that are accessible at the Python-level include
those that are executed within an atmosphere timestep on up.

Because the difference in compiled forms fundamentally affects how
the \class{Qtcm} instance facilitates Python-Fortran communication,
this attribute must be set on instantiation via a keyword input
parameter.

In the rest of this section, to avoid being verbose, when I
write \vars{'full'}, I mean the situation where
\vars{compiled\_form\thinspace=\thinspace'full'}.
Likewise, when I
write \vars{'parts'}, I mean the situation where
\vars{compiled\_form\thinspace=\thinspace'parts'}.


	\subsection{Initialization for 
			\vars{compiled\_form\thinspace=\thinspace'full'}}
				\label{sec:init.compiledform.full}

For a model run of this case, the \class{Qtcm} instance will
initialize the model using the Fortran \mods{varinit} subroutine
in the compiled QTCM1 model.  This subroutine does the following:

\begin{itemize}
\item If \vars{mrestart\thinspace=\thinspace1}, 
	the restart file is used to initialize all prognostic
	variables.  In terms of start date, the following rules are
	used:
	\begin{enumerate}
	\item Variable \vars{dateofmodel} is read from the restart file.
	\item If \vars{day0}, \vars{month0}, and \vars{year0}
		are negative, or otherwise
		invalid (e.g., \vars{month0} greater than 12), the invalid
		value is replaced with the
		day, month, and/or year of the day \emph{after} 
		that given by \vars{dateofmodel}.
		If the value of \vars{day0}, \vars{month0}, or \vars{year0}
		is not invalid in this sense, it is not replaced.
	\end{enumerate}
	Thus, if the restart file gives 
	\vars{dateofmodel} equal to 101102
	(year 10, month 11, day 2), and 
	\vars{day0\thinspace=\thinspace-1}, 
	\vars{month0\thinspace=\thinspace-1}, 
	\vars{year0\thinspace=\thinspace-1},
	and 
	\vars{mrestart\thinspace=\thinspace1}, 
	the model will start running from year 10, month 11, day 3.
	If \vars{dateofmodel} equals to 101102, and 
	\vars{day0\thinspace=\thinspace-1}, 
	\vars{month0\thinspace=\thinspace3}, 
	\vars{year0\thinspace=\thinspace-1},
	the model will start running from year 10, month 3, day 3.

\item If \vars{mrestart\thinspace=\thinspace0}, 
	all prognostic variables and right-hand sides are set to an
	initial value (which for most of those variables is zero).
	In terms of start date, \vars{day0} is set to 1 (and thus 
	the value of \vars{day0} previously input is ignored), and
	both \vars{month0} and \vars{year0}
	are set to 1 
	if their previously input values are invalid (where
	invalid means less than
	1, or, for \vars{month0}, greater than 12).
	Otherwise, \vars{month0} and \vars{year0} are left unchanged.
	Variable \vars{dateofmodel} has the value it had when the variable
	was declared (which is determined by the compiler and usually
	is zero; \vars{dateofmodel} will not be properly set until
	subroutine \mods{TimeManager} is called.

	Thus, if 
	\vars{day0\thinspace=\thinspace-1},
	\vars{month0\thinspace=\thinspace-1}, 
	\vars{year0\thinspace=\thinspace-1} is input into the model
	(say from a namelist) and 
	\vars{mrestart\thinspace=\thinspace0},
	the model will start running from year 1, month 1, day 1,
	and \vars{dateofmodel} at the exit of subroutine 
	\mods{varinit} will equal its compiler-set default.
	If 
	\vars{day0\thinspace=\thinspace14}, 
	\vars{month0\thinspace=\thinspace3}, 
	\vars{year0\thinspace=\thinspace11}, and 
	\vars{mrestart\thinspace=\thinspace0} on input into the
	model,
	the model will start running from year 11, month 3, day 1,
	and \vars{dateofmodel} at the exit of subroutine 
	\mods{varinit} will equal its compiler-set default.

	Note that \vars{dateofmodel}
	can thus be inconsistent with 
	\vars{month0} and \vars{year0} at the
	exit of subroutine \mods{varinit}.
\end{itemize}

This behavior with respect to initializing
the start date is different than in QTCM1 versions 1.0 and 2.1.
Please see the source code from those earlier QTCM1 versions for
details.




	\subsection{Initialization for 
			\vars{compiled\_form\thinspace=\thinspace'parts'}}
				\label{sec:init.compiledform.parts}

For \vars{'parts'} model, the methodology of how initialized
prognostic variables, right-hand sides, and start date related
variables are set is controlled by the \class{Qtcm} instance
attribute/flag \vars{init\_with\_instance\_state}.  The initialization
is (mostly) executed in the Python \vars{varinit} method in the
following way:

\begin{itemize}
\item If \vars{init\_with\_instance\_state} is False:
The method as described for
initialization for the 
\vars{'full'} case is generally
followed, with the exception that dateofmodel is set
to match \vars{day0}, \vars{month0}, \vars{year0}, prior to exit of 
\mods{varinit}.

\item If \vars{init\_with\_instance\_state} is True:
the model object will initialize the model based on the current
state of the model instance.  This enables you to set a model run
session's initial conditions based upon the state of the prognostic
variables and parameters stored at the Python level, which is
accessible at runtime.
\end{itemize}


Since the \vars{init\_with\_instance\_state\thinspace=\thinspace{False}}
case is mainly described by the initialization method for the
\vars{'full'} case, I refer the
reader to Section~\ref{sec:init.compiledform.full}.
For the case of \vars{init\_with\_instance\_state} is True, however,
the task is more complicated.  Specifically, for that case,
initialization includes the following:

\begin{enumerate}
\item If not currently defined,
	variable \vars{dateofmodel} is set to a default value of 0,
	which is specified in the module defaults.

\item The \vars{mrestart} flag is ignored for variable initialization.

\item All prognostic variables and right-hand sides
        are set to an
        initial value (which for most of those variables is zero),
	unless the variable is defined at the Python level, in which
	case the inital value is set to the Python level defined value.

\item If \vars{dateofmodel} is greater than 0, 
	\vars{day0}, \vars{month0}, and \vars{year0} are overwritten
        with values derived from \vars{dateofmodel} 
	in order to set the run to start
	the day \emph{after} \vars{dateofmodel}.

\item If \vars{dateofmodel} is less than or equal to 0, \vars{day0},
	\vars{month0}, and \vars{year0} are set to their respective
	instance attribute values, if valid.  For invalid instance
	attribute values, the invalid \vars{day0}, \vars{month0},
	and/or \vars{year0} is set to 1.

\item Variable \vars{dateofmodel} is recalculated
	and overwritten to match 
	\vars{day0}, \vars{month0}, \vars{year0}, prior to exit of 
	\mods{varinit}.
\end{enumerate}

As a result, for \vars{init\_with\_instance\_state} is True, the
way you indicate to the model that a run session is a brand-new run
is by setting, before the \mods{run\_session} method call,
\vars{dateofmodel} to a value less than or equal to 0, and \vars{day0},
\vars{model0}, and \vars{year0} to the day you want the model to
begin the run session.  To indicate to the model you wish to continue
a run, set \vars{dateofmodel} to the day \emph{before} you want the
model to start running from.

Examples:
\begin{itemize}
\item If \vars{day0\thinspace=\thinspace-1}, 
	\vars{month0\thinspace=\thinspace-1}, 
	\vars{year0\thinspace=\thinspace-1}, and
	\vars{dateofmodel\thinspace=\thinspace0} is input into 
	the model the model will start running from year 1, month 1, day 1,
	and 
	variable \vars{dateofmodel} at the exit of 
	subroutine \mods{varinit}
	will equal 10101.

\item If \vars{day0\thinspace=\thinspace14},
	\vars{month0\thinspace=\thinspace3}, 
	\vars{year0\thinspace=\thinspace11},
	and \vars{dateofmodel\thinspace=\thinspace0} is input into the
	model, the model will start running from year 11, month 3, day 14,
	and 
	variable \vars{dateofmodel} at the exit of 
	subroutine \mods{varinit} will equal
	110314.

\item If \vars{day0\thinspace=\thinspace14},
	\vars{month0\thinspace=\thinspace3}, 
	\vars{year0\thinspace=\thinspace11},
	and \vars{dateofmodel\thinspace=\thinspace341023} is input into the
	model, the model will start running from year 34, month 10, day 24,
	and at the exit of subroutine 
	\mods{varinit}, \vars{dateofmodel} will equal
	341024, with \vars{day0\thinspace=\thinspace24},
	\vars{month0\thinspace=\thinspace10}, and
	\vars{year0\thinspace=\thinspace34}.
\end{itemize}


	\subsection{Communication Between Python and Fortran-Levels}
				\label{sec:comm.py.fort.compiledform}

After initialization, the second major difference between a
\vars{'full'} and \vars{'parts'} model is how and when communication
between the Python and Fortran levels can occur.  For the \vars{'full'}
case, except for the passing in and out of variables before and after
a run session, all variable passing and subroutine calling happens in
the compiled QTCM1 model, with no control at the Python level.
For the \vars{'parts'} case, variables can be passed between the
Python and Fortran-levels at all levels down to the atmospheric
timestep, and many Fortran QTCM1 subroutines can be called from the
Python-level.  


		\subsubsection{Passing Variables}

For all \vars{compiled\_form} cases, variables are passed back and
forth between the Python \class{Qtcm} instance level and the
compiled QTCM1 model Fortran-level using the \class{Qtcm}
instance methods \mods{get\_qtcm1\_item} and \mods{set\_qtcm1\_item}:\footnote%
	{All Fortran routines used to pass variables back and forth are
	defined in the \mods{setbypy} module of the \fn{.so} extension
	module stored in the \class{Qtcm} instance variable \vars{\_\_qtcm}.
	All Fortran wrappers that enable Python to call compiled QTCM1 model
	subroutines are defined in the \mods{wrapcall} module stored in
	the \class{Qtcm} instance variable \vars{\_\_qtcm}.
	These modules are described in detail in 
	Sections~\ref{sec:setbypy} and~\ref{sec:wrapcall}, respectively.}

\begin{itemize}
\item \mods{get\_qtcm1\_item}(\dumarg{key}):
	Returns the value of the field variable given by the string
	\dumarg{key}.  If the compiled QTCM1 model variable given by
	\dumarg{key} is unreadable, the
        custom exception 
	\vars{FieldNotReadableFromCompiledModel} is thrown.
	The value returned is a copy of the value on the Fortran
	side, not a reference to the variable in memory.

\item \mods{set\_qtcm1\_item}:
	Sets the value of a field variable
	in the compiled QTCM1 model \emph{and at the Python-level,}
	automatically overriding any previous value at both levels.
	Thus, calling this method will change/create the \class{Qtcm}
	instance attribute corresponding to the field variable.
        When the compiled QTCM1 model variable is set, a copy of the
        Python value is passed to the Fortran model; the
	variable is \emph{not passed by reference.}
	This value comes from the \mods{set\_qtcm1\_item} calling
	parameter list, \emph{not} from the \class{Qtcm}
        instance attribute corresponding to the field variable.
\end{itemize}

The \mods{set\_qtcm1\_item} method has two calling forms, one with
one argument and the other with two arguments:
\begin{itemize}
\item One argument:  The method is called
	as \mods{set\_qtcm1\_item}(\dumarg{arg}), where \dumarg{arg} 
	is either a string giving the name of the field variable or 
	a \class{Field} instance.

\item Two arguments:  The method is called as
	\mods{set\_qtcm1\_item}(\dumarg{key}, \dumarg{value}), where
	\dumarg{key} is the string giving the name of the field variable
	and \dumarg{value} is the value to set the model field variable to
	(note \dumarg{value} can be a \class{Field} instance).
\end{itemize}
In either calling form, if no value given, the default value as defined
in module \mods{defaults} is used.

Some compiled QTCM1 model variables are not in a state where they
can be set.  An example is a compiled QTCM1 model pointer variable,
prior to the pointer being associated with a target (an attempt
to set would yield a bus error).  In such cases, the
\mods{set\_qtcm1\_item} method will throw a
\vars{FieldNotReadableFromCompiledModel} exception, nothing will
be set in the compiled QTCM1 model, and the Python counterpart
field variable (if it previously existed) would be left unchanged.\footnote%
	{We handle this situation in this way to enable the
	\class{Qtcm} instance to store variables
	even if the compiled model is not yet ready to accept them.}

Examples, typed in at a Python prompt, and
assuming that \vars{model} is a \class{Qtcm} instance:
\begin{itemize}
\item \cmd{dtvalue\thinspace=\thinspace{model.get\_qtcm1\_item('dt')}}:
	Retrieves the value of field variable \vars{dt} (timestep)
	from the compiled QTCM1 Fortran model and sets it to the
	Python variable \vars{dtvalue}.

\item \cmd{model.set\_qtcm1\_item('dt')}:
	Sets the value of field variable \vars{dt}
	in the compiled QTCM1 Fortran model to the default
	value (as given in \mods{defaults}),
	and sets the value of Python attribute \vars{model.dt} also to 
	that default value.  
	Remember that \vars{model.dt} is a \class{Field}
	instance.

\item \cmd{model.set\_qtcm1\_item('dt', 2000.)}:
	Sets the value of field variable \vars{dt}
	in the compiled QTCM1 Fortran model to 2000 (as a real),
	and sets the value of Python attribute \vars{model.dt} also to 2000.
\end{itemize}


		\subsubsection{Calling Compiled QTCM1 Model Subroutines}

All compiled QTCM1 model subroutines that can be called
(except \mods{driver} and \mods{varptrinit}) are in the
\mods{setbypy} or \mods{wrapcall} modules
of the \class{Qtcm} instance private attribute \vars{\_\_qtcm}.
(On \class{Qtcm} instance instantiation, \vars{\_\_qtcm} is set
to the \fn{.so} extension module that is the compiled QTCM1 Fortran model.)
Thus, to call \mods{wmconvct} in \mods{wrapcall} at the Python-level,
just type \cmd{model.\_\_qtcm.wrapcall.wmconvct()} (where \vars{model}
is a \class{Qtcm} instance).
For \mods{driver} and \mods{varptrinit}, these subroutines are not
contained in a \vars{\_\_qtcm} module, and thus can be called
directly (e.g., just type \cmd{model.\_\_qtcm.driver()}).
See Sections~\ref{sec:setbypy} and~\ref{sec:wrapcall} for more information
on the \mods{setbypy} and \mods{wrapcall} modules.

For the \vars{'full'} case, the only compiled QTCM1 model
subroutine you can usefully call during a run session is \mods{driver}.
For the \vars{'parts'} case, while you can essentially call any subroutine
given in a run list, you usually will not directly call a compiled QTCM1
model subroutine but will instead call it through including it in a
run list.  For example, if you have the following run list in a
\vars{'parts'} model:
\begin{codeblock}
\codeblockfont{%
[ 'qtcminit', '\_\_qtcm.wrapcall.woutpinit' ]}
\end{codeblock}
Running this list using the \class{Qtcm} instance method
\mods{run\_list} will result in \class{Qtcm} instance method
\mods{qtcminit} first being run, 
then the compiled QTCM1 Fortran model subroutine
\mods{woutpinit} in Fortran module \mods{wrapcall} being run.
See Section~\ref{sec:runlists} and
Table~\ref{tab:stnd.runlists} for a discussion and list of the
standard run lists that control routine execution content and order
in the \vars{'parts'} case.




%---------------------------------------------------------------------
\section{Restart and Continuation Run Sessions}
				\label{sec:contination.run.sessions}


	\subsection{Restart Runs In the Pure-Fortran QTCM1}
					\label{sec:puref90.restart}

To enable restart of a model run, the pure-Fortran QTCM1 model
writes out a restart file with the state of the prognostic variables
and select right-hand sides at that point in the run (for a list
of the variables, see Section~\ref{sec:snapshots}).  This binary
file can then be read in by later model runs.  The Fortran
\vars{mrestart} flag is passed in via a namelist; if \vars{mrestart}
is 1, the run uses the restart file (named \fn{qtcm.restart}).

One of the problems with using the restart file to do a continuation
run is that the continuation run will not be perfect.  In other words,
a 15~day run followed by a 25~day run based on the restart file 
generated at the end of the 15~day run will \emph{not} give the
exact same output as a continuous 40~day run.


	\subsection{Overview of Restart/Continuation Options In \mods{qtcm}}
					\label{sec:restart.options.list}

For a \class{Qtcm} instance, in contrast to the pure-Fortran QTCM1,
more than one method of continuation is available.
Thus, for a continuation run, you need to tell the model
``continue from what?''
The \class{Qtcm} class provides three choices for restart/continuing
a run:
\begin{enumerate} 
\item From a restart file:  Move/rename a QTCM1 restart file
        to the current working directory to \fn{qtcm.restart}.
	\label{list:continue.from.restart}

\item From a snapshot from another run session
	(see Sections~\ref{sec:snapshot.intro} and~\ref{sec:snapshots}).
	\label{list:continue.from.snapshot}

\item From the values of the \class{Qtcm} instance you will be
	calling \mods{run\_session} from.
	\label{list:continue.from.instance}
\end{enumerate}

Restart/continuation methods~\ref{list:continue.from.restart} 
and~\ref{list:continue.from.snapshot} both suffer from the
same problem as the pure-Fortran QTCM1 restart process:
They do not produce perfect restarts
(see Section~{sec:puref90.restart} for details).
In this section, I discuss the restart/continuation options
for each \vars{compiled\_form} option.

Methods~\ref{list:continue.from.restart}
and~\ref{list:continue.from.snapshot} are best used when making a
run session from a newly instantiated \class{Qtcm} instance.
Method~\ref{list:continue.from.instance} is best used when executing
a run session using a \class{Qtcm} instance that has already gone
through at least one run session.  Regardless of which method you
use, however, please note that anytime you execute a run session
using a \class{Qtcm} instance that already has made a previous run
session, some variables \emph{cannot be updated} between run sessions.
This feature is most noticeable with the output filename, and occurs
because the name persists in the compiled QTCM model, and is stored
in the extension module (\fn{.so} files in \vars{sodir}) associated
with the instance.  If you wish to control all variables possible
from the Python level (including output filename), you need do the
run session from a new model instance.


	\subsection{Restart/Continuation for 
		\vars{compiled\_form\thinspace=\thinspace'full'} 
		Model Instances}

The only option for restart when using
\vars{compiled\_form\thinspace=\thinspace'full'} model instances
is method~\ref{list:continue.from.restart}, to use a QTCM1 restart
file.\footnote%
	{The \vars{cont} keyword parameter in \mods{run\_session}
	and the value of the \vars{init\_with\_instance\_state}
	attribute have no effect if
	\vars{compiled\_form\thinspace=\thinspace'full'}.  With
	\vars{'full'}, the call to initialize variables all happens
	at the Fortran level (via the Fortran \mods{varinit}, not
	the Python \mods{varinit}), with no reference to the Python field
	states (or even existing Fortran field states, if present).}
To use this option, the value of the \vars{mrestart} 
attribute must equal 1, the restart file must be named
\fn{qtcm.restart}, and the restart file must be in the 
current working directory.
As with the pure-Fortran QTCM1 restart process, this method
does not produce perfect restarts.



	\subsection{Restart/Continuation for 
		\vars{compiled\_form\thinspace=\thinspace'parts'} 
		Model Instances}

For the \vars{compiled\_form\thinspace=\thinspace'parts'} case,
all three restart/continuation methods
described in Section~\ref{sec:restart.options.list} are
available.


		\subsubsection{Method~\ref{list:continue.from.restart}:
			From a QTCM1 Restart File}

To use the QTCM1 restart file mechanism, not only must the
\vars{mrestart} attribute have a value to 1, but the
\vars{init\_with\_instance\_state} flag also has to be \vars{False},
otherwise the \vars{mrestart} attribute value will be ignored.  
As with the pure-Fortran QTCM1 restart process, this method does not
produce perfect restarts.


		\subsubsection{Method~\ref{list:continue.from.snapshot}:
			From a \class{Qtcm} Instance Snapshot}

You can take snapshots of the model state of a \class{Qtcm} instance
by the \mods{make\_snapshot} instance method.  This snapshot saves
a copy of all the variables saved to a QTCM1 restart file (see
Section~\ref{sec:snapshots} for the full list of fields), which
then can be passed to other \class{Qtcm} instances for use in other
run sessions.

The key difference between this method and 
method~\ref{list:continue.from.instance} (described below)
is that \mods{run\_session} calls using the snapshot are done
\emph{without} the \vars{cont} keyword input parameter
(by default, \vars{cont} is False).  If the \vars{cont} keyword
is not False, it says the run session is a continuation run
that uses the state of the compiled QTCM1 model for all variables
that are not specified at, and read-in from,
the Python level.  If the \vars{cont} keyword
is False, the run session initializes as if it were a new run.

See Section~\ref{sec:snapshot.intro} for details and
an example of using snapshots to initialize a run session.
Note that as with the pure-Fortran QTCM1 restart process, this method 
does not produce perfect restarts.


		\subsubsection{Method~\ref{list:continue.from.instance}:
			From the Calling \class{Qtcm} Instance}

This method is used when you want to make a run session that is a
``true'' continuation run, i.e., one that uses the current state
of the compiled QTCM1 model for all variables that are not read-in
from the Python level (remember that \class{Qtcm} instances hold a
subset of the variables defined at the Fortran level).  
The key reason to use this method for a continuation run session
is that the continuation is byte-for-byte the same (if no fields
are changed) as if the run just went straight on through.  Thus,
the continuation would be perfect: A 15~day run followed by a 25~day
run using the same \class{Qtcm} instance with the \vars{cont} keyword
will give the exact same output as a continuous 40~day run.  This
is not the case when making a new instance and passing a restart
file or a snapshot, because a separate extension module is used for
those new instances.

Control of this method is accomplished through the \vars{cont}
keyword input parameter to the \mods{run\_session} method and the
\vars{init\_with\_instance\_state} attribute of a
\class{Qtcm} instance:

\begin{itemize}
\item \vars{cont}: If set to False, the run session is not a
	continuation of the previous run, but a new run session.
	If set to True, the run session is a continuation of the
	previous run session.  If set to an integer greater than
	zero, the run session is a continuation just like
	\vars{cont\thinspace=\thinspace{True}}, but the value
	\vars{cont} is set to is used for \vars{lastday} and replaces
	\vars{lastday.value} in the \class{Qtcm} instance.

\item \vars{init\_with\_instance\_state}:
	If True, for a \mods{run\_session} call using the
	\vars{cont} keyword, whatever the field values are in the Python
	instance are used in the run session.
	If False, model variables are set and initialized as described in
	Section~\ref{sec:init.compiledform.parts}.  In that case,
	previous compiled QTCM1 model values will likely be overwritten.
	Thus, if you want a continuation run that uses the state of
	all field variables except for those you explicitly change at
	the Python-level, make sure \vars{init\_with\_instance\_state}
	is True.
\end{itemize}

(Note that the \vars{cont} keyword has no effect if \vars{compiled\_form}
is \vars{'full'}.  The default value of \vars{cont} in a
\mods{run\_session} call is False.  The value of keyword \vars{cont}
is stored as private instance attribute \vars{\_cont}, in case you
really need to access it elsewhere; see
Section~\ref{sec:Qtcm.private.attrib} for more details).

The example described in Section~\ref{sec:continuation.intro} is
an example of method~\ref{list:continue.from.instance} in the list
above: The second run session is continued from the state of
\vars{model}, with the values of \vars{model}'s instance variables
overriding any values in the compiled QTCM1 model in initializing
the second run session.

This method has a few caveats worthy of note:
\begin{itemize}
\item The \vars{init\_with\_instance\_state} attribute value
	will have no effect unless the instance prognostic variables
	are set, i.e., unless a previous run session has been done.
	Another way to put it is for an initial run session right
	after a \class{Qtcm} instance is created, \mods{varinit}
	will use the same initial values for prognostic variables
	(defined in \mods{defaults} module variable
	\vars{init\_prognostic\_dict})\footnote%
		{\vars{init\_prognostic\_dict} is the dictionary giving
		the default initial values of each prognostic variable
		and right-hand side (as defined by the restart file 
		specification).}
	as it would with for both
	\vars{init\_with\_instance\_state} set to True or False).

\item Continuation run sessions using this method have to continue
	with the next day from wherever the last run session left
	off, contiguously.\footnote%
		{For continuation run sessions, you keep the 
		same extension module (the compiled \fn{.so} library),
		and all the values that define the state where it
		left off.}
	If you want to do a non-contiguous run,
	create a new \class{Qtcm} instance initialized with a
	snapshot instead of the continuation method describe in
	this section.
	will use restart rules to run a new model.  

\item When making a continuation run session using this method,
	you cannot change some variables, for instance,
	\vars{outdir} and any of the date related
	variables.  In fact, the only thing you should change for
	your continuation run session are the prognostic and
	diagnostic variables and \vars{lastday}.  This is because
	some variables cannot be updated between run sessions.
	As noted in Section~\ref{sec:restart.options.list},
	if you wish to control all variables possible
	from the Python level (including output filename), you need 
	to execute the run session from a new model instance.
\end{itemize}


	\subsection{Snapshots of a \class{Qtcm} Instance}
				\label{sec:snapshots}

The snapshot dictionary (briefly described in
Section~\ref{sec:snapshot.intro}), saved as the \class{Qtcm} instance
attribute \vars{snapshot}, and generated by the method
\mods{make\_snapshot}, saves the current state of the following
instance field variables:

\begin{center}
\input{snapshot_vars.tex}
\end{center}

These are the same variables saved to a QTCM1 restart file, and so
a snapshot duplicates the restart functionality in the Python
environment, but with more flexibility.  Since the \vars{snapshot}
dictionary is a Python variable like any other, you can manipulate
it and alter it to fit any condition you wish.




%---------------------------------------------------------------------
\section{Creating and Using Run Lists}  \label{sec:runlists}

Section~\ref{sec:runlist.intro} provides an introduction to the
role and use of run lists.  A run list is a list of methods, Fortran
subroutines, and other run lists that can be executed by the
\class{Qtcm} instance \mods{run\_list} method.  Run lists are stored
in the \class{Qtcm} instance attribute \vars{runlists}, which is a
dictionary of run lists.  The names of run lists should not be
preceeded by two underscores (though elements of a run list may be
very private variables), nor should names of run lists be the same
as any instance attribute.  Run lists are not available for
\vars{compiled\_form\thinspace=\thinspace'full'}.

The \mods{run\_list} method takes a single input parameter, a list,
and runs through that list of elements that specify other run lists
or instance method names to execute.  Methods with private attribute
names are automatically mangled as needed to become executable by
the method.  Note that if an item in the input run list is an
instance method, it should be the entire name (not including the
instance name) of the callable method, separated by periods as
appropriate.

Elements in a run list are either strings or 1-element dictionaries.
Consider the following example, where \vars{model} is a \class{Qtcm}
instance, and \mods{run\_list} is called using \vars{mylist} as
input:

\begin{codeblock}
\codeblockfont{%
model = Qtcm(\ldots) \\
mylist = [ \{'varinit':None\}, \\
\hspace*{13ex}'init\_model', \\
\hspace*{13ex}'\_\_qtcm.driver', \\
\hspace*{13ex}\{'set\_qtcm1\_item': ['outdir', '/home/jlin']\} ]
model.run\_list(mylist)}
\end{codeblock}

The first element in \vars{mylist} refers to a method that requires
no positional input parameters be passed in (as shown by the None).
The second and third elements in \vars{mylist} also refers to methods
that require no positional input parameters be passed in.  The last
element in \vars{mylist} refers to a method with two input parameters.
Note that while I use the term ``method'' to describe the elements,
the strings/keys do not have to be only Python instance methods.
The second element, for instance, refers to another run list, and
the third element refers to a compiled QTCM1 model subroutine (note
the \vars{\_\_qtcm} attribute).

When the \mods{run\_list} method is called, the items in the input
run list are called in the order given in the list.  For each
element,  the \mods{run\_list} method first checks if the string
or dictionary key name corresponds to the key of an entry in the
\class{Qtcm} instance attribute \vars{runlists}.  If so, \mods{run\_list}
is called using that run list (i.e., it is a ``recursive'' call).
If the string or dictionary key name does not refer to another run
list, the \mods{run\_list} method checks if the string or dictionary
key name is a method of the \class{Qtcm} instance, and if so the
method is called.  Any other value throws an exception.

If input parameters for a method are of class \class{Field}, the
\mods{run\_list} method first tries to pass the parameters into the
method as is, i.e., as Field object(s).  If that fails, the
\mods{run\_list } method  passes its parameters in as the \vars{value}
attribute of the \class{Field} object.

If you want a variable that is being passed into a run list to be
continuously updated, you have to set the parameter in the run list
to a \class{Field} instance that is a \class{Qtcm} instance attribute,
not just to the value of the field variable (or to a non-\class{Field}
object).  Otherwise, subsequent calls to that run list element will
not use the updated values as input parameters.

For instance, if you had a run list element:
\begin{codeblock}
\codeblockfont{%
\{'\_\_qtcm.timemanager':[model.coupling\_day,]\}}
\end{codeblock}
and \vars{model.coupling\_day} were an integer (it's not by default,
but pretend it was), then \mods{run\_list} calling
\mods{\_\_qtcm.timemanager} will pass in a scalar integer rather
than a binding to the variable \vars{model.coupling\_day}.  In such
a situation, if the variable \vars{model.coupling\_day} were updated
in time, the \mods{run\_list} call of \mods{\_\_qtcm.timemanager}
would not be updated in time.  This happens because when the
dictionary that is the run list element is created, the value of
list element(s) attached to the dictionary element is set to the
scalar value of \vars{model.coupling\_day} at that instant.

You can get around this feature by setting \class{Qtcm} instance
attributes that will change with model execution to \class{Field}
instances, and then referring to those attributes in the parameter
list in the run list element.  In that case:
\begin{codeblock}
\codeblockfont{%
\{'\_\_qtcm.timemanager':[model.coupling\_day,]\}}
\end{codeblock}
will use the current value of \vars{model.coupling\_day} anytime
\vars{\_\_qtcm.timemanager} is called by \mods{run\_list}, if
\vars{model.coupling\_day} is a \class{Field} object.

When \mods{run\_list}, encounters a calling input parameter that
is a \class{Field} object, it will first try to pass the entire
\class{Field} object to the method/routine being called.  If that
raises an exception, it will then try to pass just the value of the
entire \class{Field} object.  This is done to enable \mods{run\_list}
to be used for both pure-Python and compiled QTCM Fortran model
routines.  Fortran cannot handle \class{Field} objects as input
parameters, only values.

Table~\ref{tab:stnd.runlists} shows all standard run lists
stored in the \vars{runlists} attribute upon instantiation
of a \class{Qtcm} instance.

\begin{htmlonly}
\begin{table}[htp]
\begin{center}
\fbox{Empty placeholder block for table that would have gone here.}
\end{center}
\caption{Standard run lists stored in the \vars{runlists} 
	attribute upon instantiation of a \class{Qtcm} instance.
	The run list and list element names are stored as strings.
	\emphpara{This table is improperly reproduced in the
	HTML conversion.  Please see the PDF version for the table.}}
\label{tab:stnd.runlists}
\end{table}
\end{htmlonly}

\begin{latexonly}
\begin{table}[htp]
\input{runlists}
\caption{Standard run lists stored in the \vars{runlists} 
	attribute upon instantiation of a \class{Qtcm} instance.
	The run list and list element names are stored as strings.}
\label{tab:stnd.runlists}
\end{table}
\end{latexonly}

Of course, feel free to change the contents of any of the run lists
after instantiation, or to add additional run lists to the
\vars{runlists} attribute dictionary.  The ability to alter run
lists at runtime gives the \mods{qtcm} package much of its flexibility.




%---------------------------------------------------------------------
\section{Field Variables and the \class{Field} Class}
						\label{sec:field.variables}

The term ``field'' variable refers to QTCM1 model variables that 
are accessible at both the compiled Fortran QTCM1 model-level as
well as the Python \class{Qtcm} instance-level.
Field variables are all instances of the \class{Field} class
(though non-field variables can also be instances of \class{Field}).

Section~\ref{sec:field.variables.intro} gives a brief introduction to
the attributes and methods in a \class{Field} instance.
A nitty gritty description of the class is found in its docstrings.

	\subsection{Creating Field Variables}

To create a \class{Field} instance whose value is set to the
default, instantiate with the field id as the only positional
input argument.  Thus:

\begin{codeblock}
\codeblockfont{foo = Field('lastday')}
\end{codeblock}

will return \vars{foo} as a \class{Field} instance with \vars{foo.value}
set to the value listed in Section~\ref{sec:defaults.scalar}.
The value of all \class{Field} instances upon creation are specified
in the \mods{defaults} submodule of package \mods{qtcm}, and listed
in Sections~\ref{sec:defaults.scalar} and~\ref{sec:defaults.array}.

To create \class{Field} instances whose attributes are set different
from their defaults, you can specify the different settings in the
instantiation parameter list, or change the attributes once the
instance is created.  See the \class{Field} docstring for details.


	\subsection{Initial Field Variables}  \label{sec:initial.variables}

Field variables include both model parameters that do not change
for a \class{Qtcm} instance as well as prognostic variables that
do change during model integration.  As a result, many field variables
have values different from the default values listed in
Sections~\ref{sec:defaults.scalar} and~\ref{sec:defaults.array}.
In this section, I list the \emph{initial} values of all field
variables.  The ``initial'' values are the settings for \class{Qtcm}
field variables execution of the \mods{run\_session} method, but
prior to cycling through an atmosphere-ocean coupling timestep.
This is in contrast to ``default'' values, which the field variables
are given on instantiation, if no other value is specified.
Numerical values are rounded as per the conventions
of Python's \vars{\%g} format code.


		\subsubsection{Scalars}

For the fields that give the input/output directory names, and the
run name, the entry ``value varies'' is provided in the ``Value''
column.

\input{init_scalars}

		\subsubsection{Arrays}

\input{init_arrays}


	\subsection{Passing Fields Between the Python and Fortran-Levels}

Section~\ref{sec:comm.py.fort.compiledform} discusses the differences
between how the \vars{'full'} and \vars{'parts'} compiled forms
pass field variables between the Python and Fortran-levels.  That
discussion gives a detailed description of the methods used for
passing fields to and from the Python and Fortran-levels (i.e., the
\mods{get\_qtcm1\_item} and \mods{set\_qtcm1\_item} methods).

Please note the following regarding field variables as you pass them 
back and forth between the Python and Fortran-levels:
\begin{itemize}
\item Field variables with ghost latitudes, such as \vars{u1}, on
	the Python end are always the full variables (i.e., including
	the ghost latitudes).  On the Fortran end, variables like
	\vars{u1} also always have the ghost latitudes while in the
	model, but when stored as restart files, do not have the
	ghost latitudes; the end points are not saved in restart
	files or written to the netCDF output files.
	See the
	\latexhtml{%
\htmladdnormallinkfoot{QTCM1 manual}%
        {http://www.atmos.ucla.edu/$\sim$csi/qtcm\_man/v2.3/qtcm\_manv2.3.pdf}}%
{\htmladdnormallink{QTCM1 manual}%
        {http://www.atmos.ucla.edu/~csi/qtcm_man/v2.3/qtcm_manv2.3.pdf}}
	\cite{Neelin/etal:2002}
	for details about ghost latitudes.

\item You should assume there is only a full synchronizing between 
	compiled QTCM1 model and Python model field variables
	at the beginning and end of a run session.  

\item If you have a variable at the Python-level, but at the
	compiled QTCM1 Fortran model-level the variable is not
	readable, if you try to call \mods{set\_qtcm1\_item} on the
	variable, nothing is done, and the Python-level value is
	left alone.  If you have a compiled QTCM1 model variable,
	but no Python-level equivalent, if you call \mods{set\_qtcm1\_item}
	on the variable, the Python-level variable (as an attribute)
	is created.

\item To be precise, only compiled QTCM1 model variables can be
	passed pass back and forth between the Python and Fortran-levels;
	there are many \class{Qtcm} instance attributes that do not
	have any counterparts at the Fortran-level.\footnote%
		{I use the term ``field variables'' to refer to 
		compiled QTCM1 model variables that can be passed
		back and forth to the Python level.}

\item Although \vars{dayofmodel} is described in module \mods{setbypy}
	as an option for the \mods{get\_qtcm1\_item} and
	\mods{set\_qtcm1\_item} methods to operate on, in reality
	those methods cannot operate on \vars{dayofmodel}, but
	\vars{dayofmodel} is not defined in \mods{defaults}.\footnote%
		{All field variables must be defined in \mods{defaults} in
		order for the proper Fortran routine to be called
		according to the variable's type.}
\end{itemize}


	\subsection{Field Variable Shape}   \label{sec:field.var.shape}

Normally, Python arrays have a different dimension order than Fortran
arrays.  While Fortran arrays are dimensioned (col, row, slice),
with adjacent columns being contiguous, then rows, and then slices, Python
arrays are dimensioned (slice, row, col), with adjacent columns being
contiguous, then rows, and then slices.  Based on this, you would
think that everytime you passed an array between the Python and
Fortran-levels you would need to transpose the array.

Thankfully, we don't have to do this because \mods{f2py} handles
array dimension order transparently so we can refer to each element
the same way whether we're in Python or Fortran.  Thus, the array
\vars{Qc} in Fortran is dimensioned (longitude, latitude), (64,42)
by default, and the Python \class{Qtcm} instance attribute \vars{Qc}
has a \vars{value} attribute also dimensioned (longitude, latitude),
(64,42) by default.  And at both the Fortran and Python-levels, the
first longtude, second latitude element is referred to as \vars{Qc(1,2)}.

In contrast, however, netCDF output saved by the compiled QTCM1 model
and read into Python (using the \mods{Scientific} package) is
\emph{not} in Fortran array order.  Arrays read from netCDF output
into Python are in Python array order, and are dimensioned
(latitude, longitude) or (time, latitude, longitude).  The \class{Qtcm}
routines that manipulate netCDF data (e.g., \mods{plotm}), however,
automatically adjust for this, so you only need to be aware of this
when reading in output for your own analysis
(see Section~\ref{sec:model.output}).




%---------------------------------------------------------------------
\section{Model Output}			\label{sec:model.output}

Section~\ref{sec:output.intro} gives an overview of how to
use \mods{qtcm} model output to netCDF files.

All netCDF array output is dimensioned (time, latitude, longitude)
when read into Python using the \mods{Scientific} package.  This
differs from the way \class{Qtcm} saves field variables, which
follows Fortran convention (longitude, latitude).  Thus, the shapes
in Section~\ref{sec:initial.variables}, Appendix~\ref{app:defaults.values},
etc., are not the shapes of arrays read from the netCDF output.
See Section~\ref{sec:field.var.shape} for a discussion of why
there is this discrepancy.

Because netCDF files allow you to specify an ``unlimited'' dimension,
it is possible to close a netCDF file, reopen it, and add more
slices of data to the file.  Thus, continuous \class{Qtcm} run
sessions (i.e., those that use the \vars{cont} keyword input parameter
in the \mods{run\_session} method) will automatically append output
to the netCDF output files.

Field variables with ghost latitudes, such as \vars{u1}, on the
Python and Fortran ends are always the full variables (i.e., including
the ghost latitudes).  The ghost latitudes are not written to the
netCDF output files, however.
See the \latexhtml{%
\htmladdnormallinkfoot{QTCM1 manual}%
        {http://www.atmos.ucla.edu/$\sim$csi/qtcm\_man/v2.3/qtcm\_manv2.3.pdf}}%
{\htmladdnormallink{QTCM1 manual}%
        {http://www.atmos.ucla.edu/~csi/qtcm_man/v2.3/qtcm_manv2.3.pdf}}
	\cite{Neelin/etal:2002}
for details about ghost latitude structure.

\class{Qtcm} instances have a few built-in tools to visualization
model output.  These are briefly described in Section~\ref{sec:viz.intro}.
Note that the \mods{plotm} method is linked to a specific \class{Qtcm}
instance.  Do not use \mods{plotm} outside of the instance it is
linked to.  It must also be used only after a successful run session
(i.e., not in the middle of a run session).




%---------------------------------------------------------------------
\section{Miscellaneous}

A few miscellaneous items/issues about the model:
\begin{itemize}
\item The land model runs at same timestep as the atmosphere.

\item If the land model runs less often than 
	\mods{sflux} in \mods{physics1}, 
	the calculation of evaporation over the land 
	needs to be fixed in sflux.

\item The units of some field variables are not what you would expect.
	For instance, \vars{Qc} is in energy units, i.e., K, and not
	mm/day.
	See the
	\latexhtml{%
\htmladdnormallinkfoot{QTCM1 manual}%
        {http://www.atmos.ucla.edu/$\sim$csi/qtcm\_man/v2.3/qtcm\_manv2.3.pdf}}%
{\htmladdnormallink{QTCM1 manual}%
        {http://www.atmos.ucla.edu/~csi/qtcm_man/v2.3/qtcm_manv2.3.pdf}}
	\cite{Neelin/etal:2002}
	for details.
\end{itemize}




%---------------------------------------------------------------------
\section{Cookbook of Ways the Model Can Be Used}  \label{sec:cookbook}

This cookbook of a few ways to use the model is arranged by science
tasks, i.e., certain types of runs we want to do.  For some of the
examples below, I assume that the dictionary
\vars{inputs} is initially defined as given in
Figure~\ref{fig:defn.of.inputs}.  All examples assume that
\cmd{from qtcm import Qtcm} has already been executed.


%--- Two versions, one for PDF and the other for HTML:
\begin{latexonly}
\begin{figure}[tp]
\begin{codeblock}
\codeblockfont{%
inputs = \{\} \\
inputs['runname'] = 'test' \\
inputs['landon'] = 0 \\
inputs['year0'] = 1 \\
inputs['month0'] = 11 \\
inputs['day0'] = 1 \\
inputs['lastday'] = 30 \\
inputs['mrestart'] = 0 \\
inputs['init\_with\_instance\_state'] = True \\
inputs['compiled\_form'] = 'parts'}
\end{codeblock}

\caption{The initial definition of the \vars{inputs} dictionary for 
	examples given in Section~\ref{sec:cookbook}.  These settings
	imply that a run session will start on November 1, Year 1,
	last for 30 days, and will be an aquaplanet run.}
\label{fig:defn.of.inputs}
\end{figure}
\end{latexonly}

\begin{htmlonly}
\label{fig:defn.of.inputs}
\begin{center}
\htmlfigcaption{%
	\codeblockfont{%
inputs = \{\} \\
inputs['runname'] = 'test' \\
inputs['landon'] = 0 \\
inputs['year0'] = 1 \\
inputs['month0'] = 11 \\
inputs['day0'] = 1 \\
inputs['lastday'] = 30 \\
inputs['mrestart'] = 0 \\
inputs['init\_with\_instance\_state'] = True \\
inputs['compiled\_form'] = 'parts'}
	}

\htmlfigcaption{Figure~\ref{fig:defn.of.inputs}:
	The initial definition of the \vars{inputs} dictionary for 
	examples given in Section~\ref{sec:cookbook}.  These settings
	imply that a run session will start on November 1, Year 1,
	last for 30 days, and will be an aquaplanet run.}
\end{center}
\end{htmlonly}



\begin{description}
\item[Plain model run:]
	Here I just want to make a single model run.
	Tasks:  Instantiate a fresh model and execute a run session.
	The code to run the model is just:
	\begin{codeblock}
	\codeblockfont{%
inputs['init\_with\_instance\_state'] = False \\
model = Qtcm(**inputs) \\
model.run\_session()}
	\end{codeblock}
	where \vars{inputs} is initialized with the code in
	Figure~\ref{fig:defn.of.inputs}.


\item[Explore parameter space with a set of models:]
	Here I want to create an entire suite of separate models,
	in order to determine the sensitivity of the model to changing
	a parameter.
	To do this, I
	instantiate multiple fresh models, 
	and execute a run session for each instance, all within
	a \vars{for} loop:


%--- Two versions, because LaTeX2HTML does not correctly typeset
%    the hspace command:
\begin{latexonly}
	\begin{codeblock}
	\codeblockfont{%
import os \\
inputs['init\_with\_instance\_state'] = False \\
for i in xrange(0,1002,10): \\
\hspace*{5ex}iname = 'ziml-' + str(i) + 'm' \\
\hspace*{5ex}ipath = os.path.join('proc', iname) \\
\hspace*{5ex}os.makedirs(ipath) \\
\hspace*{5ex}model = Qtcm(**inputs) \\
\hspace*{5ex}model.ziml.value = float(i)  \\
\hspace*{5ex}model.runname.value = iname \\
\hspace*{5ex}model.outdir.value = ipath \\
\hspace*{5ex}model.run\_session() \\
\hspace*{5ex}del model}
	\end{codeblock}
\end{latexonly}

\begin{htmlonly}
\begin{center}
\htmlfigcaption{%
	\codeblockfont{%
import os \\
inputs['init\_with\_instance\_state'] = False \\
for i in xrange(0,1002,10): \\
\hspace*{5ex}iname = 'ziml-' + str(i) + 'm' \\
\hspace*{5ex}ipath = os.path.join('proc', iname) \\
\hspace*{5ex}os.makedirs(ipath) \\
\hspace*{5ex}model = Qtcm(**inputs) \\
\hspace*{5ex}model.ziml.value = float(i)  \\
\hspace*{5ex}model.runname.value = iname \\
\hspace*{5ex}model.outdir.value = ipath \\
\hspace*{5ex}model.run\_session() \\
\hspace*{5ex}del model}
	}
\end{center}
\end{htmlonly}


	The loop explores mixed-layer depth \vars{ziml} from 0~m to
        1000~m, in 10~m intervals.  I create the \vars{outdir}
	directory before every model call, since the compiled QTCM1 model
	requires the output directory exist, specifying the run name
	and output directory as the string \vars{iname}.
	The output directories are assumed to all be in the \fn{proc}
	sub-directory of the current working directory.
	\vars{inputs} is initialized with the code in
	Figure~\ref{fig:defn.of.inputs}.


\item[Conditionally explore parameter space:]
	Here I want to 
	conditionally explore the parameter space, on the basis of
	some mathematical criteria.
	To do this, I
	instantiate a model, evaluate results using
	that criteria, and run another fresh model depending on
	the results (passing the previous model state via a snapshot),
	all within a \vars{while} loop.
	Note that this type of investigation is very difficult to 
	automate if all you can use are shell scripts and
	Fortran.
	See Figure~\ref{fig:conditional.test.eg} for a detailed
	example.


\item[With interactive adjustments at run time:]
	The example in Figure~\ref{sec:continuation.intro}
	illustrates this type of run.  In this example,
	I instantiate a fresh model, execute a run session, analyze the
	output, change variables in the model instance, and then
	execute a continuation run session.


\item[Test alternative parameterizations:]
	I've already described how we can use run lists to arbitrarily
	change model execution order and content at run time.
	We can take advantage of Python's inheritance
	abilities, along with run lists, to simplify this.
	Figure~\ref{fig:alt.param.inherit.eg} provides an example of
	this use.

	Of course, you can use pre-processor directives and shell
	scripts to accomplish the same functionality seen in
	Figure~\ref{fig:alt.param.inherit.eg} using just Fortran.
	The Python solution, however, shortcuts the compile/linking
	step, and enables you to easily do run time swapping between
	subroutine choices based upon run time calculated
	tests (see Figure~\ref{fig:conditional.test.eg} for an
	example of such tests).
\end{description}




% --- Two versions of this block, one for display in PDF and the other
%     for display in HTML:
\begin{latexonly}
\begin{figure}[p]
	\begin{codeblock}
	\codeblockfont{%
\small
import os \\
import numpy as N \\
maxu1 = 0.0 \\
while maxu1 < 10.0: \\
\hspace*{5ex}iziml = 0.1 * maxu1 \\
\hspace*{5ex}iname = 'ziml-' + str(iziml) + 'm' \\
\hspace*{5ex}ipath = os.path.join('proc', iname) \\
\hspace*{5ex}os.makedirs(ipath) \\
\hspace*{5ex}model = Qtcm(**inputs) \\
\hspace*{5ex}try: \\
\hspace*{10ex}model.sync\_set\_py\_values\_to\_snapshot(snapshot=mysnapshot) \\
\hspace*{10ex}model.init\_with\_instance\_state = True \\
\hspace*{5ex}except: \\
\hspace*{10ex}model.init\_with\_instance\_state = False \\
\hspace*{5ex}model.ziml.value = iziml  \\
\hspace*{5ex}model.runname.value = iname \\
\hspace*{5ex}model.outdir.value = ipath \\
\hspace*{5ex}model.run\_session() \\
\hspace*{5ex}maxu1 = N.max(N.abs(model.u1.value)) \\
\hspace*{5ex}mysnapshot = model.snapshot \\
\hspace*{5ex}del model}
	\end{codeblock}

\caption{This code explores different values of
	mixed-layer depth \vars{ziml} for 30~day runs,
	as a function of maximum \vars{u1} magnitude,
	until it finds a case where the maximum \vars{u1} is
	greater than 10~m/s.  (The relationship between
	\vars{ziml} and the maximum of the speed of
	\vars{u1}, where 
	\vars{ziml\thinspace=\thinspace0.1\thinspace*\thinspace{maxu1}}, 
	is made up.)
	With each iteration, the new run uses the snapshot from
	a previous run to initialize (as well as the new value
	of \vars{ziml}); the \vars{try} statement is used to
	ensure the model works even if \vars{mysnapshot} is not
	defined (which is the case the first time around).
	The \vars{inputs} dictionary is initialized with the code in
	Figure~\ref{fig:defn.of.inputs}.}
\label{fig:conditional.test.eg}
\end{figure}
\end{latexonly}

\begin{htmlonly}
\label{fig:conditional.test.eg}
\begin{center}
\htmlfigcaption{%
	\codeblockfont{%
import os \\
import numpy as N \\
maxu1 = 0.0 \\
while maxu1 < 10.0: \\
\hspace*{5ex}iziml = 0.1 * maxu1 \\
\hspace*{5ex}iname = 'ziml-' + str(iziml) + 'm' \\
\hspace*{5ex}ipath = os.path.join('proc', iname) \\
\hspace*{5ex}os.makedirs(ipath) \\
\hspace*{5ex}model = Qtcm(**inputs) \\
\hspace*{5ex}try: \\
\hspace*{10ex}model.sync\_set\_py\_values\_to\_snapshot(snapshot=mysnapshot) \\
\hspace*{10ex}model.init\_with\_instance\_state = True \\
\hspace*{5ex}except: \\
\hspace*{10ex}model.init\_with\_instance\_state = False \\
\hspace*{5ex}model.ziml.value = iziml  \\
\hspace*{5ex}model.runname.value = iname \\
\hspace*{5ex}model.outdir.value = ipath \\
\hspace*{5ex}model.run\_session() \\
\hspace*{5ex}maxu1 = N.max(N.abs(model.u1.value)) \\
\hspace*{5ex}mysnapshot = model.snapshot \\
\hspace*{5ex}del model}
	}

\htmlfigcaption{Figure \ref{fig:conditional.test.eg}:
	This code explores different values of
	mixed-layer depth \vars{ziml} for 30~day runs,
	as a function of maximum \vars{u1} magnitude,
	until it finds a case where the maximum \vars{u1} is
	greater than 10~m/s.  (The relationship between
	\vars{ziml} and the maximum of the speed of
	\vars{u1}, where 
	\vars{ziml\thinspace=\thinspace0.1\thinspace*\thinspace{maxu1}}, 
	is made up.)
	With each iteration, the new run uses the snapshot from
	a previous run to initialize (as well as the new value
	of \vars{ziml}); the \vars{try} statement is used to
	ensure the model works even if \vars{mysnapshot} is not
	defined (which is the case the first time around).
	The \vars{inputs} dictionary is initialized with the code in
	Figure~\ref{fig:defn.of.inputs}.}
\end{center}
\end{htmlonly}


% --- Two versions of this block, one for display in PDF and the other
%     for display in HTML:
\begin{latexonly}
\begin{figure}[p]
\begin{center}
	\begin{codeblock}
	\codeblockfont{%
\small
import os \\
\\
class NewQtcm(Qtcm): \\
\hspace*{5ex}def cloud0(self):\\
\hspace*{10ex}[\ldots] \\
\hspace*{5ex}def cloud1(self):\\
\hspace*{10ex}[\ldots] \\
\hspace*{5ex}def cloud2(self):\\
\hspace*{10ex}[\ldots] \\
\hspace*{5ex}[\ldots] \\
\\
inputs['init\_with\_instance\_state'] = False \\
for i in xrange(10): \\
\hspace*{5ex}iname = 'cloudroutine-' + str(i)  \\
\hspace*{5ex}ipath = os.path.join('proc', iname) \\
\hspace*{5ex}os.makedirs(ipath) \\
\hspace*{5ex}model = NewQtcm(**inputs) \\
\hspace*{5ex}model.runlists['atm\_physics1'][1] = 'cloud' + str(i) \\
\hspace*{5ex}model.runname.value = iname \\
\hspace*{5ex}model.outdir.value = ipath \\
\hspace*{5ex}model.run\_session() \\
\hspace*{5ex}del model}
	\end{codeblock}
\end{center}

\caption{Let's say we have 9 different cloud physics schemes we wish
	to try out in 9 different runs.  The easiest way to do this
	is to create a new class \class{NewQtcm} that
	inherits everything from \class{Qtcm}, and to which we'll
	add the additional cloud schemes (\vars{cloud0}, \vars{cloud1},
	etc.).
	In the \vars{for} loop, I change the cloud model
	run list entry in the run list that governs
	atmospheric physics at one instant to whatever the cloud
	model is at this point in the loop.
	The \vars{inputs} dictionary is initialized with the code in
	Figure~\ref{fig:defn.of.inputs}.
	Of course, we could do the same thing by running the 9
	models separately, but this set-up makes it easy to do
	hypothesis testing with these 9 models.  For instance, we
	can create a test by which we will choose which of the 9
	models to use:  Within this framework, the selection of
	those models can be altered by changing a string.}
\label{fig:alt.param.inherit.eg}
\end{figure}
\end{latexonly}

\begin{htmlonly}
\label{fig:alt.param.inherit.eg}
\begin{center}
\htmlfigcaption{%
	\codeblockfont{%
import os \\
\\
class NewQtcm(Qtcm): \\
\hspace*{5ex}def cloud0(self):\\
\hspace*{10ex}[\ldots] \\
\hspace*{5ex}def cloud1(self):\\
\hspace*{10ex}[\ldots] \\
\hspace*{5ex}def cloud2(self):\\
\hspace*{10ex}[\ldots] \\
\hspace*{5ex}[\ldots] \\
\\
inputs['init\_with\_instance\_state'] = False \\
for i in xrange(10): \\
\hspace*{5ex}iname = 'cloudroutine-' + str(i)  \\
\hspace*{5ex}ipath = os.path.join('proc', iname) \\
\hspace*{5ex}os.makedirs(ipath) \\
\hspace*{5ex}model = NewQtcm(**inputs) \\
\hspace*{5ex}model.runlists['atm\_physics1'][1] = 'cloud' + str(i) \\
\hspace*{5ex}model.runname.value = iname \\
\hspace*{5ex}model.outdir.value = ipath \\
\hspace*{5ex}model.run\_session() \\
\hspace*{5ex}del model}
	}

\htmlfigcaption{Figure \ref{fig:alt.param.inherit.eg}:
	Let's say we have 9 different cloud physics schemes we wish
	to try out in 9 different runs.  The easiest way to do this
	is to create a new class \class{NewQtcm} that
	inherits everything from \class{Qtcm}, and to which we'll
	add the additional cloud schemes (\vars{cloud0}, \vars{cloud1},
	etc.).
	In the \vars{for} loop, I change the cloud model
	run list entry in the run list that governs
	atmospheric physics at one instant to whatever the cloud
	model is at this point in the loop.
	The \vars{inputs} dictionary is initialized with the code in
	Figure~\ref{fig:defn.of.inputs}.
	Of course, we could do the same thing by running the 9
	models separately, but this set-up makes it easy to do
	hypothesis testing with these 9 models.  For instance, we
	can create a test by which we will choose which of the 9
	models to use:  Within this framework, the selection of
	those models can be altered by changing a string.}
\end{center}
\end{htmlonly}




% ===== end of file =====


%@@@\chapter{Combining \code{qtcm} with \code{CliMT}}
%@@@% ==========================================================================
% CliMT
%
% By Johnny Lin
% ==========================================================================


% ------ BODY -----
%
\section{General Tutorial on CliMT}


General notes of things I think I may have observed about
\code{Parameters} objects:
\begin{itemize}
\item You can treat a \code{Parameters} instance as a dictionary, where
	the key is the name of the field, because \code{\_\_getitem\_\_},
	etc.\ have been defined for the instance.  However, the values,
	units, and long names of the fields are stored in dictionaries
	assigned to \code{value}, \code{units}, and \code{long\_name},
	keyed to the field name (a string).
\end{itemize}


General notes of things I think I may have observed about
\code{Components} objects:
\begin{itemize}
\item All variables and quantities, whether they be physical fields,
	filenames, or metadata,
	are stored as attributes in the \code{Components} instance.
\item \code{Components} have these special attributes:
        \code{Required},
        \code{Prognostic},
	and
        \code{Diagnostic},
	which are lists that contain the names of describe whether
\item Scalar parameters in \code{Component} objects
	are stored as an instance of the \code{Parameters}
	class, under the attribute \code{Params}.
\end{itemize}


General notes of things I think I may have observed about
\code{Federation} objects:
\begin{itemize}
\item \code{Federation} objects hold the \code{Components} instances
	in a list assigned to the attribute \code{list}.
\item \code{Federation} attributes
        \code{Required},
        and
	\code{Prognostic},
	are unions of the same attributes of the constituent
	\code{Components} objects.
\end{itemize}






% ===== end of file =====


\chapter{Troubleshooting}                   \label{ch:trouble}
% ==========================================================================
% Troubleshooting
%
% By Johnny Lin
% ==========================================================================


% ------ BODY -----
%
\section{Error Messages Produced by \mods{qtcm}}

\begin{description}
\item[\screen{Error-Value too long in SetbyPy module getitem\_str for}
	\dumarg{key}:]
	This message is produced by the Fortran
	subroutine \mods{getitem\_str}
	in the module \mods{SetbyPy} in the compiled QTCM1 Fortran code.
	The code is in the file \fn{setbypy.F90}.  This error occurs when
	the Fortran variable whose name is given by the string \dumarg{key}
	has a value that is greater than the local parameter
	\vars{maxitemlen} in \mods{getitem\_str}.  To fix this, you have
	to go into \fn{setbypy.F90} and change the value of
	\vars{maxitemlen}.

\item[\screen{Error-real\_rank1\_array should be deallocated}:]
	Fortran module \mods{SetByPy}'s subroutine
	\mods{getitem\_real\_array} generates this message
	(or a similar message for other ranks) if the Fortran
	variable for the input \dumarg{key} are allocated on entry
	to the routine.  This may indicate the user has written another
	Fortran routine to access the \mods{real\_rank1\_array} variable
	outside of the standard interfaces..

\item[\screen{Error-Bad call to SetbyPy module \ldots}:]
	Often times, this error occurs because a get or set routine
	in \mods{SetByPy} tried to act on a variable for which the
	corresponding input \dumarg{key} is not defined.  The solution
	is to add that case in the if/then construct for the get and set
	routines in \mods{SetByPy} and rebuild the extension modules.
\end{description}


\section{Other Errors}

\begin{description}
\item[Python cannot find some packages:]
	This error often happens when the version of Python in which
	you have installed all your packages is not the version that
	is called at the Unix command line by typing in \cmd{python}.
	To get around this, 
        define a Unix alias
        that maps \cmd{python2.4} (or whichever version of Python
	has all your packages installed) to \cmd{python}.  If you
	have multiple Python's installed on your system, you might
	have to use a more specific name for the Python executable.
	As a result, you may have to change the test scripts in
	\fn{test} in the \mods{qtcm} distribution directory.

\item[\mods{get\_qtcm1\_item} and compiled QTCM1 model pointer
	variables:]
	If you try to use the \mods{get\_qtcm1\_item} method on a compiled
	QTCM1 model pointer variable 
	(i.e., \vars{u1}, \vars{v1}, \vars{q1}, \vars{T1}),
	 before the compiled
	model \mods{varinit} subroutine is run, you'll get a bus error
	with no additional message.

\item[Mismatch between Python and Fortran array field variables:]
	You change an array field variable on the Python side, but
	it seems like the wrong elements are changed on the Fortran
	side.  Or you type in the same index address for accessing a
	\mods{qtcm} netCDF output array as well as its \class{Qtcm}
	instance attribute counterpart, and find you get different
	answers.  Some possible reasons and fixes:

	\begin{itemize}
	\item This will occur if you haven't accounted for the
		difference in how field variables are saved at the
		Python-level, Fortran-level, and in a netCDF file.
		All netCDF array output is dimensioned (time,
		latitude, longitude) when read into Python using
		the \mods{Scientific} package.  This differs from
		the way \class{Qtcm} saves field variables, \emph{both}
		at the Python- and Fortran-levels, which follows
		Fortran convention (longitude, latitude).

		Note that the way \class{Qtcm} saves field variables
		at the Python- and Fortran-levels is different than
		the default way Python and Fortran save arrays.
		Section~\ref{sec:field.var.shape} for more information.

	\item You may have forgotten that array indices in Python start at
		0, while indices in Fortran (generally) start at 1.
		Also, ranges in Python are exclusive at the upper-bound,
		while ranges in Fortran are inclusive at the upper-bound.
		(Both Python and Fortran array indice ranges are inclusive
		at the lower-bound.)

	\item You may have forgotten some field variables have
		ghost latitudes, and thus there are extra latitude bands
		when the array is stored as a Python or Fortran field
		variable, but there are \emph{no} extra latitude bands
		when the array is stored as netCDF output (the QTCM1
		output routines strip off the ghost latitudes when
		writing those field variables out).
	        See the
        \latexhtml{%
\htmladdnormallinkfoot{QTCM1 manual}%
        {http://www.atmos.ucla.edu/$\sim$csi/qtcm\_man/v2.3/qtcm\_manv2.3.pdf}}%
{\htmladdnormallink{QTCM1 manual}%
        {http://www.atmos.ucla.edu/~csi/qtcm_man/v2.3/qtcm_manv2.3.pdf}}
        \cite{Neelin/etal:2002}
        for details about ghost latitudes.

		The safest and easiest way to tell whether the variable has a
		ghost latitudes is to look at its shape.
		A call to the \class{Qtcm} instance
		method \mods{get\_qtcm1\_item} will give you the array,
		and the use of NumPy's \mods{shape} function will give you
		the shape.
	\end{itemize}
\end{description}




% ===== end of file =====


\chapter{Developer Notes}                   \label{ch:devnotes}
% ==========================================================================
% Using QTCM
%
% By Johnny Lin
% ==========================================================================


% ------ BODY -----
%

%---------------------------------------------------------------------
\section{Introduction}

This section describes programming practices and issues related to
the \mods{qtcm} package that might be of interest to users wishing
to add/change code in the package.
Please see the package
\latexhtml{API documentation,%
		\footnote{http://www.johnny-lin.com/py\_pkgs/qtcm/doc/html-api/}
		which includes the source code}%
        {\htmladdnormallink{API documentation}%
		{http://www.johnny-lin.com/py\_pkgs/qtcm/doc/html-api/},
		which includes the source code},
for details.




%---------------------------------------------------------------------
\section{Changes to QTCM1 Fortran Files}  \label{sec:f90changes}

The source code used to generate the shared object files used
in this Python package is unchanged
from the pure-Fortran QTCM1 model source code, except in the
following ways:

\begin{itemize}
\item The suffix of all source code files 
	has been changed from \fn{.f90} to \fn{.F90}, 
	in order to ensure the compiler preprocesses 
	the source code.  Some compilers use the capitalization to
	tell whether or not to run the code through a preprocessor.

\item In all \fn{.F90} files, occurrences of:
	\begin{codeblock}
	\codeblockfont{%
	Character(len=130)}
	\end{codeblock}
	are changed to:
	\begin{codeblock}
	\codeblockfont{%
	Character(len=305)}
	\end{codeblock}
	This enables the model to properly deal with longer filenames.
	The number ``305'' is chosen to make search and replace easier.

\item In \fn{qtcmpar.F90}, the 
	\vars{eps\_c} variable is changed from an unchangable
	parameter to a changeable real, 
	so that it can be changed in the model at runtime.

\item All occurrences of an underscore (``\_'') character in a
	subroutine or function name are removed.  The
	presence of the underscore messes up the dynamic lookup
	mechanism for the \mods{f2py} generated extension module.
	The following names are changed, both in subroutine definitions
	and calls:
	\begin{itemize}
	\item \mods{out\_restart} to \mods{outrestart},
	\item \mods{save\_bartr} to \mods{savebartr},
	\item \mods{grad\_phis} to \mods{gradphis}.
	\end{itemize}

\item \fn{driver.F90} is changed so that program
	\mods{driver} becomes a subroutine, and 
	subroutine \mods{driverinit} is deleted (along with
	all calls to it) because basic model initialization is
	handled at the Python level.

\item In \fn{clrad.F90}, subroutine \mods{cloud}, the first
	\vars{COUNTCAP} preprocessor macro, a comment line for
	that ifdef is moved to prevent a warning message during
	building with \mods{f2py}.

\item The order of subroutine \mods{qtcminit} is changed.  The original
	pure-Fortran QTCM1 \mods{qtcminit} code has the following
	calling sequence:

	\begin{codeblock}
        \codeblockfont{%
Call parinit            !Initialize model parameters \\
Call varinit            !Initialize variables \\
Call TimeManager(1)     !mm set model time \\
Call bndinit            !input boundary datasets \\
Call physics1           !diagnostic fields for initial condition}
	\end{codeblock}

	For the \mods{qtcm} package, I've altered this order so
	\mods{bndinit} comes after \mods{parinit} but before \mods{varinit}:
	\begin{codeblock}
        \codeblockfont{%
Call parinit            !Initialize model parameters \\
Call bndinit            !input boundary datasets \\
Call varinit            !Initialize variables \\
Call TimeManager(1)     !mm set model time  \\
Call physics1           !diagnostic fields for initial condition}
	\end{codeblock}

	This is done because \vars{STYPE} is not read in for the
	\vars{landon} \vars{True} case until \mods{bndinit}, but
	in \mods{varinit} \vars{STYPE} is used to calculate the
	original values of \vars{WD} for the non-restart case.  This
	also corrects the conflicting order found in the pure-Fortran
	QTCM1 manual (compare pp.\ 29 and 32).  As far as I can
	tell, \mods{bndinit} has no dependencies that require it
	to come after \mods{timemanager} or \mods{varinit}.

\end{itemize}

In addition, the Fortran files \fn{setbypy.F90}, \fn{wrapcall.F90},
and \fn{varptrinit.F90} are added.  The routines in these files, 
however, just add more flexibility and functionality to the model;
they do not automatically affect any model computations.  See
Section~\ref{sec:newf90} for details.




%---------------------------------------------------------------------
\section{New Interfaces and Fortran Functionality}  \label{sec:newf90}

As described in Section~\ref{sec:f90changes}, the Fortran files
\fn{setbypy.F90}, \fn{wrapcall.F90}, and \fn{varptrinit.F90} are
added to the QTCM1 source directory.  The first two files define the Fortran
90 modules (\mods{SetbyPy} and \mods{WrapCall}) needed to interface
the Python and Fortran levels.  The last file defines a new Fortran
subroutine \mods{varptrinit} that associates QTCM1 model pointer
variables at the Fortran level.  In a pure-Fortran run of QTCM1,
this occurs in subroutine \mods{varinit}; for a
\vars{compiled\_form\thinspace=\thinspace'parts'} run, since the
functionality of the Fortran \mods{varinit} is now in the Python
\mods{varinit} method, a separate Fortran pointer association
subroutine needed to be defined.  The Fortran subroutine \mods{varptrinit}
is called as the \mods{varptrinit} function of the 
\vars{compiled\_form\thinspace=\thinspace'parts'}
\fn{.so} extension module.


	\subsection{Fortran Module \mods{SetbyPy}}   \label{sec:setbypy}

		\subsubsection{Design Description}

This module defines functions and subroutines used to read variables
from the Fortran-level to the Python-level, and in setting Fortran-level
variables using the Python-level values.  These routines are used
by \class{Qtcm} methods \mods{get\_qtcm1\_item} and \mods{set\_qtcm1\_item}
(and dependencies thereof) to ``get'' and ``set'' the Fortran-level
variables.  Note that the Fortran module \mods{SetbyPy} is referred
to in lowercase at the Python level, i.e., as the
attribute \vars{\_\_.qtcm.setbypy} of a \class{Qtcm} instance.

Because Fortran variables are not dynamically typed, separate Fortran
functions and subroutines need to be defined to get and set variables
of different types.\footnote%
	{The \mods{interface} construct in Fortran 90 is supposed to
	provide a way to create a single interface to multiple
	routines, e.g.:
	\begin{codeblock}
	\codeblockfont{%
Interface setitem \\
\hspace*{3ex}Module Procedure setitem\_real, setitem\_int, setitem\_str \\
End Interface}
	\end{codeblock}
	This construct, however, causes a bus error
	(Mac OS X 10.4, Intel).  Thus, I put the
	same functionality in the Python code.}
The \class{Qtcm} methods \mods{get\_qtcm1\_item}
and \mods{set\_qtcm1\_item} know which one of the Fortran routines
to call on the basis of the type and rank of the value for the field
variable in the \mods{defaults} submodule.  This is why all field
variables need to have defaults defined in \mods{defaults}.  For
array variables, the field variable defaults also provide the rank
of the Fortran-level variable being gotten or set.  However, the
array default values do \emph{not} have to have the same shape as
the Fortran-level variables; on the Python-side, variable shape
adjusts to what is declared on the Fortran-side.  
Thus, if you change the resolution of
the compiled QTCM1 model, you do not have to change the dimensions
of the field variable values in \mods{defaults}.

The \class{Qtcm} method \mods{get\_qtcm1\_item} directly calls
the \mods{SetByPy} routines.
The \class{Qtcm} method \mods{set\_qtcm1\_item} makes use of
private instance methods that make the calls to the \mods{SetByPy} routines.

For scalar field variables, \mods{SetByPy} provides functions and
subroutines that provide the value of the variable on output.
For array field variables, \mods{SetByPy}
dynamic \emph{module} arrays are used to pass array
variables in and out; I could not get the 
\mods{SetByPy} Fortran routines to set
locally defined dynamic arrays (that is, locally within a function or
subroutine).\footnote%
	{I tried to implement Fortran subroutine
	\mods{getitem\_real\_array} using traditional array 
	dimension passing 
	(e.g., \code{subroutine foo(nx, ny, a)}) as well
	as declaring the allocatable array inside the subroutine, 
	but neither option worked on my \mods{f2py} (version 2\_3816) 
	and Python (version 2.4.3).}
In the \mods{SetByPy} module, these dynamic arrays
are defined as follows:

\begin{codeblock}
\codeblockfont{%
Real, allocatable, dimension(:) :: real\_rank1\_array \\
Real, allocatable, dimension(:,:) :: real\_rank2\_array \\
Real, allocatable, dimension(:,:,:) :: real\_rank3\_array}
\end{codeblock}

For all field variables, scalar or array, the \mods{SetByPy} module
has a fourth module variable defined, \vars{is\_readable}, that the
Fortran get and set routines will set to \vars{.TRUE.} if the
variable is readable and \vars{.FALSE.} if not (it's declared as a
logical variable).  This Fortran variable can be used to prevent
Python from accessing pointer variables that aren't yet associated
to targets.

In general, \mods{SetByPy} routines make use of Fortran constructs
to enable them to accomodate all possible
variables of a given type and shape.  However, 
for string scalars, the \mods{SetByPy} function \mods{getitem\_str}
has to have a return value of a predefined length, in order to
work properly.  That length is given by the parameter
\vars{maxitemlen} and is set to 505 (the value is chosen to
be larger than all filename variables described in
Section~\ref{sec:f90changes} and to be easily found in
the \fn{.F90} files).


		\subsubsection{Module Structure}

If you're a Fortran programmer, you can probably get all the information
in this section from just reading the \fn{setbypy.F90} file directly.
This description of the module structure, however, permits me to highlight
what you need to do if you want to make additional compiled QTCM1 variables
accessible to Python \class{Qtcm} objects.

\begin{itemize}
\item All \mods{Use} statements are given in the beginning of 
	the \mods{SetByPy} module.  These statements cover
	nearly all of the QTCM1 Fortran
	modules that contain variables of interest.  If the
	QTCM1 variable you're interested in isn't in a module
	listed here, you'll have to add your own
	\mods{Use} statement of that module here.

\item Next comes the definitions for the
	\vars{real\_rank1\_array},
	\vars{real\_rank2\_array}, and
	\vars{real\_rank3\_array} dynamic array variables, and
	the \vars{is\_readable} boolean variable.

\item The \mods{Contains} block of the module defines the module
	routines called by the \class{Qtcm} instance methods to
	set and get the compiled QTCM1 model variables.  The
	routines are:
	\begin{itemize}
	\item Function \mods{getitem\_real}
	\item Subroutine \mods{getitem\_real\_array}
	\item Function \mods{getitem\_int}
	\item Function \mods{getitem\_str}
	\item Subroutine \mods{setitem\_real}
	\item Subroutine \mods{setitem\_real\_array}
	\item Subroutine \mods{setitem\_int}
	\item Subroutine \mods{setitem\_str}
	\end{itemize}

\end{itemize}

Each of the routines in the module \mods{Contains} block is essentially
a list of \mods{if}/\mods{elseif} statements.  The list tests for the
name of the variable of interest (a string), and gets or sets the
compiled QTCM1 model variable corresponding to that name.  For pointer
array variables, a test is also made as to whether or not the variable
has been associated.  If not, the variable is not readable
and \vars{is\_readable} is set to \vars{.FALSE.}\ accordingly.

If you wish to add another compiled QTCM1 model variable to be
accessible to \class{Qtcm} instance methods \mods{get\_qtcm1\_item}
and \mods{set\_qtcm1\_item}, just add another \mods{if}/\mods{else\-if},
like the other \mods{if}/\mods{elseif} blocks, in the Fortran set
and get routines corresponding to the QTCM1 variable type (scalar
vs.\ array, and real, integer, or string).  On the Python side, add
an entry in \mods{defaults} corresponding to the new field variable
you've created access to.  I would strongly recommend making the
Python name of your new field variable
(given in \mods{defaults}) be the same as the compiled
QTCM1 model variable name.



	\subsection{Fortran Module \mods{WrapCall}}   \label{sec:wrapcall}

Most of the time, if you want to call a compiled QTCM1 model subroutine
from the Python level, you will use the version of the subroutine that
is found in this Fortran module.  
Note that the Fortran module \mods{WrapCall} is referred
to in lowercase at the Python level, i.e., as the
attribute \vars{\_\_.qtcm.wrapcall} of a \class{Qtcm} instance.

All the routines in this module do is wrap one of the compiled QTCM1
model routines.  For instance, \mods{WrapCall} subroutine
\mods{wadvcttq} is defined as just:

% --- Two versions of this block, one for display in PDF and the other
%     for display in HTML:
%
\begin{latexonly}
\begin{codeblock}
\codeblockfont{%
Subroutine wadvcttq \\
\hspace*{3ex}Call advcttq \\
End Subroutine wadvcttq}
\end{codeblock}
\end{latexonly}

\begin{htmlonly}
\begin{rawhtml}
<p><code><font color="blue">Subroutine wadvcttq<br>
&nbsp;&nbsp;&nbsp;Call advcttq<br>
End Subroutine wadvcttq</font></code></p>
\end{rawhtml}
\end{htmlonly}

All subroutines in this module begin with ``w'', with the rest of
the name being the Fortran QTCM1 subroutine name.  The calling
interface for the ``w'' version is the same as the Fortran QTCM1
original version.  There are no subroutines in this module that do
not have an exact counterpart in the Fortran QTCM1 code, and thus
this module's subroutines sole purpose is to call other subroutines
in the compiled QTCM1 model.

These wrapper routines are needed because \mods{f2py}, for some
reason I can't figure out, will not properly wrap Fortran routines
(that are then callable at the Python level) that create local
arrays using parameters obtained through a Fortran \mods{use}
statment.  Thus, as an example, a Fortran subroutine \mods{foo}
with the following definition:

% --- Two versions of this block, one for display in PDF and the other
%     for display in HTML:
%
\begin{latexonly}
\begin{codeblock}
\codeblockfont{%
subroutine foo \\
\hspace*{3ex}use dimensions \\
\hspace*{3ex}real a(nx,ny) \\
\hspace*{3ex}[\ldots] \\
end subroutine foo}
\end{codeblock}
\end{latexonly}

\begin{htmlonly}
\begin{rawhtml}
<p><code><font color="blue">
subroutine foo<br>
&nbsp;&nbsp;&nbsp;use dimensions<br>
&nbsp;&nbsp;&nbsp;real a(nx,ny)<br>
&nbsp;&nbsp;&nbsp;[\ldots]<br>
end subroutine foo
</font></code></p>
\end{rawhtml}
\end{htmlonly}


where \vars{nx} and \vars{ny} are defined in the module vars{dimensions},
will return an error, with the result that the extension module
will not be created, or an extension modules that yields output
that is different from running the pure-Fortran version of QTCM1.

By wrapping these calls into this file, I also avoid having to
separate out the Fortran QTCM1 subroutines into separate \fn{.F90}
files.  For Fortran subroutines that you want callable from the
Python level, \mods{f2py} seems to require each Fortran subroutine
to be in its own file of the same name (e.g., the version of
\fn{driver.F90} for this package). If several Fortran subroutines
are all found in a single \fn{.F90} files, \mods{f2py} seems unable
to create wrappers for those subroutines.




%---------------------------------------------------------------------
\section{Python \mods{qtcm} and Pure-Fortran QTCM1 Differences}

This section describes differences between how the \mods{qtcm}
package and the pure-Fortran QTCM1 assign some varables.  A list
of changes to the QTCM1 Fortran Files for use in the \mods{qtcm}
package is found in Section~\ref{sec:f90changes}.


	\subsection{QTCM1 \mods{driverinit}}   \label{sec:driverinit.diffs}

In the pure-Fortran version of QTCM1, by default, the following variables are
set by reference (as given below), not by value, in the \mods{driverinit}
routine:\footnote%
	{In the pure-Fortran version of QTCM1, this routine is found
	in \fn{driver.F90}.}
\begin{codeblock}
\codeblockfont{%
lastday\thinspace=\thinspace{daysperyear} \\
viscxu0\thinspace=\thinspace{viscU} \\
viscyu0\thinspace=\thinspace{viscU} \\
visc4x\thinspace=\thinspace{viscU} \\
visc4y\thinspace=\thinspace{viscU} \\
viscxu1\thinspace=\thinspace{viscU} \\
viscyu1\thinspace=\thinspace{viscU} \\
viscxT\thinspace=\thinspace{viscT} \\
viscyT\thinspace=\thinspace{viscT} \\
viscxq\thinspace=\thinspace{viscQ} \\
viscyq\thinspace=\thinspace{viscQ}}
\end{codeblock}

Thus, in pure-Fortran QTCM1, if you change \vars{daysperyear},
\vars{viscU}, etc.
and recompile (as needed), you will automatically change 
\vars{lastday}, \vars{viscxu0}, etc.
(Though, in the pure-Fortran QTCM1, the default values may be overwritten by
namelist input values.)

The \mods{driverinit} routine is eliminated
in the Python \code{qtcm} package.  Instead, inital values 
of field variables are specified in the \mods{defaults} submodule
and set by value to attributes of the \code{Qtcm} instance.
Thus, for instance, in a \class{Qtcm} instance, \code{lastday} 
is set to \code{365} by default, not to some variable
\vars{daysperyear}.  For the diffusion and viscosity terms,
the \class{Qtcm} instance attributes corresponding to those
terms are set to literals.\footnote%
	{Those literals are defined by \mods{defaults} private
	module variables \vars{\_\_viscT}, \vars{\_\_viscQ},
	and \vars{\_\_viscU}.}

In contrast, in the pure-Fortran QTCM1,
\mods{driverinit} declares local
variables \code{viscU}, \code{viscT}, and \code{viscQ},
and reads values into those variables via the input namelist.
Those values are then used to set
\vars{viscxu0}, \vars{viscyu0}, etc., as described above.
In pure-Fortran QTCM1, \code{viscU}, \code{viscT}, and \code{viscQ}
are not directly accessed anywhere else in the model.
Thus, \code{viscU}, \code{viscT}, and \code{viscQ} are not
defined as field variables in the \code{qtcm} package, and
\class{Qtcm} instances do not have attributes corresponding
to those names.
Additionally, if you wish to change a viscosity parameter
\vars{visc*} (given above), the parameter for each direction
must be set one-by-one even if the flow is isotropic.


	\subsection{The \mods{varinit} Routine}

One of the functions of the pure-Fortran QTCM1 \mods{varinit}
subroutine is to associate the pointer variables \vars{u1}, \vars{v1},
\vars{q1}, and \vars{T1}.  For the extension modules in the \mods{qtcm}
package, a Fortran subroutine \mods{varptrinit} is added that can
also do this association.  This subroutine is called in the
\class{Qtcm} instance method
\latexhtml{\mods{varinit}%
		\footnote{http://www.johnny-lin.com/py\_docs/qtcm/doc/html-api/qtcm.qtcm.Qtcm-class.html\#varinit}}%
	{\htmladdnormallink{\mods{varinit}}{http://www.johnny-lin.com/py_docs/qtcm/doc/html-api/qtcm.qtcm.Qtcm-class.html#varinit}}
(which duplicates and
extends the function of its pure-Fortran counterpart, enabling
alternative ways of handling restart).

The \mods{varptrinit} is not accessed via \mods{wrapcall}.  Remember
that \mods{wrapcall} contains only those routines that were in the
original pure-Fortran QTCM1 code, and that we want to have access
to at the Python level.


	\subsection{The \mods{qtcm} Method of \class{Qtcm}}

The \class{Qtcm} method \mods{qtcm} duplicates the functionality
of the \mods{qtcm} subroutine in the pure-Fortran QTCM1 model.
There are a few differences, however.  First, the \mods{qtcm} method
for \class{Qtcm} instances does not include a call to \mods{cplmean},
which uses mean surface flux for air-sea coupling.  This state is
consistent with the pure-Fortran QTCM1 pre-processor macro
\vars{CPLMEAN} being off.  Thus, if you wish to use mean surface
flux for air-sea coupling, you will have to revise the \mods{qtcm}
method of \class{Qtcm} to call \mods{cplmean}.  You'll also have to
check for any other code additions needed that are associated with
the \vars{CPLMEAN} macro.

Second, the \mods{qtcm} method for \class{Qtcm} instances does not
include the option of not using the atmospheric boundary layer
model.  This is consistent with macro \vars{NO\_ABL} being off.  If
you wish to have no atmospheric boundary layer model, change the
run list \vars{atm\_bartr\_mode} so that the \mods{wsavebartr} and
\mods{wgradphis} routines are not called.  You'll also have to check
for any other code additions needed that are associated with the
\vars{NO\_ABL} macro.



	\subsection{Miscellaneous Differences}

\begin{itemize}
\item In Python \class{Qtcm} instances,
	\vars{dateofmodel} is set to 0 by default.  
	In contrast, in the compiled QTCM1 model,
	the default (i.e., initial value) is calculated from 
	\vars{day0}, \vars{month0}, and \vars{year0}.
	See Section~\ref{sec:init.compiledform.full} for details.

\item The \class{Qtcm} instance attribute
	\vars{\_\_qtcm} is not copyable using \mods{copy.deepcopy}.

\item In general, when executing a \class{Qtcm} instance method, 
	if you change a \class{Qtcm} instance attribute 
	that has a counterpart in the compiled QTCM1 model,
	the compiled QTCM1 counterpart is not changed until the
	end of the method.  Likewise, if you call a compiled QTCM1 model
	subroutine and change a compiled QTCM1 model variable with
	a \class{Qtcm} instance counterpart, the \class{Qtcm}
	instance counterpart is not changed until the end of the
	subroutine.

\item In general, even though some of the compiled QTCM1 model
	Fortran subroutines/functions have counterparts in \class{Qtcm}
	that duplicate the former's functionality, the Fortran
	versions are kept intact so that the
	\vars{compiled\_form\thinspace=\thinspace'full'} case will work.
\end{itemize}




%---------------------------------------------------------------------
\section{Considerations When Adding Fortran Code}

In this section I describe issues to consider if you wish to add
your own compiled code to the package as separate extension modules.
(This is different from creating new standard extension modules,
which is described in Section~\ref{sec:create.new.so}.):

\begin{itemize}
\item The \class{Qtcm} class assumes that the directory path 
	to the original shared object file is the same as for the 
	\mods{package\_version} module.

\item If you want to be able to pass other Fortran variables 
	in and out to/from Python, please see the 
	Section~\ref{sec:setbypy}
	discussion of the Fotran \mods{SetByPy} module.

\item Fortran and Python routines to get and set compiled QTCM1 model
	arrays are currently written only for floating point array.

\item If you ever change 
	\class{Qtcm} instance method
	\mods{\_set\_qtcm\_array\_item\_in\_model}
	to work with non-floating point values, you will also
	have to change the array handling section in 
	\mods{set\_qtcm1\_item}.

\item The restart mechanism in the pure-Fortran QTCM1 model is 
	\emph{not} bit-for-bit correct.  Thus, if you compare the final
	output from a 40 day run with a 30 day run restarted from
	a 10 day run, the output will not be the same.
	This behavior has been duplicated in \class{Qtcm} 
	instances when the \vars{mrestart} flag is used
	and applicable.

\item When creating new extension modules using the \fn{src} makefile,
	be sure you first use the \cmd{make clean} command to clean-up
	any old files.

\end{itemize}




%---------------------------------------------------------------------
\section{Creating New Standard Extension Modules}   \label{sec:create.new.so}

The steps involved in creating the standard extension modules (e.g.,
\fn{\_qtcm\_full\_365.so}, etc.) on installation are given in
Section~\ref{sec:create.so}.  The makefile provided in \fn{/buildpath/src}
uses a Fortran compiler to create the object code, runs \mods{f2py}
to create the shared object file in \fn{src}, and moves the shared
object files into \fn{../lib}, overwriting any pre-existing files
of the same name.  In this section, I describe the makefile and
\mods{f2py} in a little more detail, in case you wish to create
standard extension modules with additions from the ones the default
makefile creates.


	\subsection{Makefile Rules}    \label{sec:makefile.rules}

This section describes the rules of the
makefile found in the \fn{src} directory
of the \mods{qtcm} distribution.  
This makefile is used by the Python package to create the extension
module (\fn{.so} files) imported and used by \mods{qtcm} objects
(as described in Section~\ref{sec:create.so}).
The makefile will, in general, be used only during \mods{qtcm}
installation, but if you wish to recompile the QTCM1 libraries
and make changes in the Python extension module,
you'll want to use/change this makefile.

\begin{description}
\item[clean] Removes old files in preparation for compiling new
	extension modules.

\item[libqtcm.a] Creates library \fn{libqtcm.a} that contains all
	QTCM1 object files in the directory \fn{src},, except
	\fn{setbypy.o}, \fn{wrapcall.o}, \fn{varptrinit.o}, and
	\fn{driver.o}.  This archive is compiled with the netCDF
	libraries.  Previous versions of \fn{libqtcm.a} are overwritten.

\item[\_qtcm\_full\_365.so] Creates the extension module
	\fn{\_qtcm\_full\_365.so}.  \mods{f2py} is run on applicable code
	in \fn{src}, and the extension module is moved to \fn{../lib}.
	Any previous versions of \fn{../lib/\_qtcm\_full\_365.so}
	are overwritten.

\item[\_qtcm\_parts\_365.so] Creates the extension module
	\fn{\_qtcm\_parts\_365.so}.  \mods{f2py} is run on applicable code
	in \fn{src}, and the extension module is moved to \fn{../lib}.
	Any previous versions of \fn{../lib/\_qtcm\_parts\_365.so}
	are overwritten.

\end{description}



	\subsection{Using \mods{f2py}}      \label{sec:using.f2py}

This section briefly describes how \mods{f2py} is used in the
makefile during the creation of the extension modules.
\htmladdnormallink{\mods{F2py}}{http://cens.ioc.ee/projects/f2py2e/} is a
program that generates shared object libraries that allow you to call
Fortran routines in Python.  \mods{F2py} comes with Python's
\htmladdnormallink{NumPy}{http://numpy.scipy.org/}
array handling package, so you do not need to install anything
extra if you have NumPy already installed.

To create the extension modules in \mods{qtcm} using
the makefile described in Section~\ref{sec:makefile.rules},
I use a method similar to the
\latexhtml{``Quick and Smart Way,''\footnote%
{http://cens.ioc.ee/projects/f2py2e/usersguide/index.html\#the-quick-and-smart-way}}%
{\htmladdnormallink{``Quick and Smart Way''}%
{http://cens.ioc.ee/projects/f2py2e/usersguide/index.html#the-quick-and-smart-way}}
described in the \mods{f2py} manual.
For the \fn{\_qtcm\_full\_365.so} extension module, the 
\mods{f2py} call is:

\begin{codeblock}
\codeblockfont{%
f2py --fcompiler=\$(FC) -c -m \_qtcm\_full\_365 driver.F90 $\backslash$ \\
\hspace*{10ex}setbypy.F90 libqtcm.a \$(NCLIB)}
\end{codeblock}

and for the \fn{\_qtcm\_parts\_365.so} extension module, the call is:

\begin{codeblock}
\codeblockfont{%
f2py --fcompiler=\$(FC) -c -m \_qtcm\_parts\_365 $\backslash$ \\
\hspace*{10ex}varptrinit.F90 wrapcall.F90 setbypy.F90 $\backslash$ \\
\hspace*{10ex}libqtcm.a \$(NCLIB)}
\end{codeblock}

For both calls, \vars{FC} and \vars{NCLIB} are the environment
variables in the makefile specifying the Fortran compiler and netCDF
libraries, respectively.  The \vars{-m} flag specifies the extension
module name (without the \fn{.so} suffix).  The \fn{.F90} files
specify the files that have modules and routines that will be
accessible at the extension module level, and the rest of the Fortran
files in QTCM1 are compiled and archived in a library \fn{libqtcm.a}.
For \mods{f2py} to work properly,
the \fn{.F90} files may define \emph{only one} module or routine.

If you add Fortran files containing new modules, and you wish those
modules to be accessible at the Python level, compile your new code
with \mods{f2py}.  If we have a file of such new code, \fn{newcode.F90},
the \mods{f2py} call to create the \fn{\_qtcm\_parts\_365.so}
extension module will become:

\begin{codeblock}
\codeblockfont{%
f2py --fcompiler=\$(FC) -c -m \_qtcm\_parts\_365 $\backslash$ \\
\hspace*{10ex}varptrinit.F90 wrapcall.F90 setbypy.F90 $\backslash$ \\
\hspace*{10ex}newcode.F90 $\backslash$ \\
\hspace*{10ex}libqtcm.a \$(NCLIB)}
\end{codeblock}

If you write new Fortran code for the compiled QTCM1 model that
will \emph{not} be accessed from the Python-level, just add the
object code filename to the variable \vars{QTCMOBJS} in the
makefile; you don't have to do anything else.  If you are adding
Fortran code to existing Fortran modules, it's even easier:  You
don't need change the makefile.  Note that for 64 bit processor
machines, you may have to use \mods{f2py} with the \cmd{-fPIC} flag;
see Section~\ref{sec:sopic} for details on how the lines above will
change.


	\subsection{Two Examples}

\emphpara{A Function:}
Let's say you have written a piece of Fortran code called
\fn{myfunction.F90} that contains one function called
\mods{myfunction}, and you want to have this function
callable from the Python level through the \class{Qtcm} 
instance method \mods{\_\_qtcm.myfunction}.  Do the following:

\begin{enumerate}
\item Move \fn{myfunction.F90} to \fn{src} in the \mods{qtcm}
	distribution directory \fn{/buildpath}.

\item Add \cmd{myfunction.o} to the end of the object file list lines
	after the target names
	\vars{\_qtcm\_full\_365.so} and
	\vars{\_qtcm\_parts\_365.so}.

\item In the
	\vars{\_qtcm\_full\_365.so} and
	\vars{\_qtcm\_parts\_365.so} target descriptions,
	add \cmd{myfunction.F90} to the 
	beginning of the list of \fn{.F90} names 
	in the \mods{f2py} lines.
\end{enumerate}


\emphpara{A Module:} 
Let's say you have written a piece of Fortran code called
\fn{mymodule.F90} that contains the Fortran module \mods{MyModule}
containing multiple routines and variables.  You want to have those
routines and variables callable from the Python level through the
\class{Qtcm} instance attribute \mods{\_\_qtcm.mymodule}.  The steps
to add \mods{MyModule} to the extension modules are exactly the
same as for a single function, with \cmd{mymodule} being
substituted in the makefile everywhere you have \cmd{myfunction}.




%---------------------------------------------------------------------
\section{Attributes and Methods in \class{Qtcm} Instances}

In this section I describe some attributes, particularly private ones,
that may be of interest to developers.
As is the convention in Python, private
attributes and methods are prepended by one or two underscores,
with two underscores being the ``more'' private attribute.
Please see the package
\latexhtml{API documentation%
		\footnote{http://www.johnny-lin.com/py\_pkgs/qtcm/doc/html-api/}}
        {\htmladdnormallink{API documentation}%
		{http://www.johnny-lin.com/py\_pkgs/qtcm/doc/html-api/}}
for details about all variables, including private variables.


	\subsection{Public \mods{num\_settings} Submodule Attributes/Methods}

\begin{itemize}
\item \vars{typecode}:  This module function returns the
	type code of the data array passed in as its argument.

\item \vars{typecodes}:  This dictionary is the same as the
	NumPy (or Numeric and \mods{numarray})
	dictionary \vars{typecodes}, except that the character
	\vars{'S'} and \vars{'c'} are added to the
	\vars{typecodes['Character']} entry, if absent.  This
	functionality is added because I found 
	\vars{typecodes['Character']} had different values in
	Mac OS X and Ubuntu GNU/Linux.
\end{itemize}


	\subsection{Private \mods{qtcm} Submodule Attributes}

This submodule of the package \mods{qtcm} is the module that defines
the \class{Qtcm} class.

\begin{itemize}
\item \vars{\_init\_prog\_dict}:  This dictionary contains
	the default values of all prognostic variables and 
	right-hand sides that can be initialized.  In the
	submodule \mods{qtcm}, it is set to
	the \vars{init\_prognostic\_dict} module variable in
	submodule \mods{defaults}.

\item \vars{\_init\_vars\_keys}:  List of all keys in
	\vars{\_init\_prog\_dict}, plus \vars{'dateofmodel'}
	and \vars{'title'}.  These names correspond to the
	field variables that are usually written out into a
	restart file.

\item \vars{\_test\_field}:  \class{Field} object instance used 
	in type tests.
\end{itemize}



	\subsection{Private \class{Qtcm} Attributes}  
					\label{sec:Qtcm.private.attrib}

\begin{itemize}
\item \vars{\_cont}:  A boolean attribute that is \vars{True}
	if the run session is a continuation run session and
	\vars{False} if not.  Set the value passed in by
	the keyword \vars{cont} when the \mods{run\_session}
	method is executed.

\item \vars{\_monlen}:  Integer array of the number of days in 
	each month, assuming a 365~day year.

\item \vars{\_\_qtcm}:  The extension module that is the
	compiled QTCM1 Fortran model for this instance.
	This attribute is unique for every instance:  The
	extension module \fn{.so} file is first copied to
	a temporary directory (given by the \vars{sodir}
	instance attribute) and then imported to the
	\class{Qtcm} instance.
	This private attribute is set on instantiation.

\item \vars{\_qtcm\_fields\_ids}:  Field ids for all default 
	field variables, set on instantiation.

\item \vars{\_runlists\_long\_names}:  Dictionary holding the
	descriptions of the standard run lists.  The keys of
	the dictionary are the names of the standard run lists.
\end{itemize}




%---------------------------------------------------------------------
\section{Creating Documentation}

The distribution of \mods{qtcm} comes with the full set of
documentation in readable form (PDF and HTML).  The documentation
consists of two kinds:  this User's Guide and the API documentation.
The User's Guide is written in \LaTeX.  The PDF version is generated
directly from \LaTeX, and the HTML version is created by
\LaTeX{2}HTML.

I use the \fn{make\_docs} shell script in \fn{doc} creates all these
documents.  Briefly, that script does the following:

\begin{itemize}
\item In the \fn{doc/latex} directory, uses \cmd{python} to
	run \fn{code\_to\_latex.py}, which generates the
	\LaTeX\ files describing the current \mods{qtcm} 
	package settings, including text in the manual which gives
	all uses of the current version number.

\item \LaTeX\ is run on the \LaTeX\ files in the \fn{doc/latex} directory.
	The PDF generated by the run is moved from \fn{doc/latex} to
	\fn{doc}.

\item \LaTeX{2}HTML is run on the \LaTeX\ files in \fn{doc/latex}.
	The HTML files generated by the run are moved to \fn{doc/html}.

\item \mods{epydoc} is run on the \mods{qtcm} package libraries.
	This is run in \fn{doc}, to make use of the \fn{epydoc}
	configuration file present there.  The syntax from the
	command line is:

\begin{codeblock}
\codeblockfont{%
epydoc -v --config epydocrc [name]}
\end{codeblock}
\vars{[name]} is either \cmd{qtcm}, if the \mods{qtcm} package is
installed in a directory listed in \vars{sys.path}, or 
\vars{[name]} is the name of the directory the \mods{qtcm} package is
located in (e.g., \fn{/usr/lib/python2.4/site-packages/qtcm}).

\end{itemize}

The \fn{make\_docs} script cannot be used without customizing it
to your system, so please \emphpara{DO NOT USE IT} if you do
not know what you are doing.  You could easily wipe out all your
documentation by mistake.





% ===== end of file =====


\chapter{Future Work}                       \label{ch:future}
% ==========================================================================
% Future
%
% By Johnny Lin
% ==========================================================================


% ------ BODY -----
%
This section describes the features and fixes I plan to work on
in this package.  The most urgent items are listed closer to the
begining of the lists.

\begin{itemize}
\item Add \code{implicit none} top setbypy.F90.

\item Check through Fortran routines that have arguments, to make sure
	f2py is properly understanding the intentions
	(i.e., in, out, inout) of the variables, since we're using the
	``quick way'' of making shared object libraries using f2py.
	The \fn{utilities.F90} file has a number of Fortran routines
	with arguments.

\item Cite:  Peterson, P. (2009) 
	F2PY: a tool for connecting Fortran and Python programs, 
	\emph{Int. J. Computational Science and Engineering,}
	Vol.\ 4, No.\ 4, pp.\ 296--305 for f2py.

\item Create a method like \mods{calc\_derived('T100')} which would
	primarily operate on a data file and provide a derived variable
	such as the temperature at 100 hPa, as given in this example.
	Figure out where to put the parameters (V1s, etc.) that are
	needed to make such a calculation.  As attributes?  Create a
	method to write the quantity out to an output file?
	Perhaps make an ability to calculate these values at heights
	at a given time each day during a run session?

\item Automate the installation using Python's
\htmladdnormallinkfoot{\mods{distutils}}{http://docs.python.org/dist/dist.html}
	utilities.

\item Describe a way of using job control (either via the operating system
	or IPython's \mods{jobctrl} module) 
	to do a quick-and-dirty parallelization of multiple
	\class{Qtcm} instance run sessions.  Or use some sort of threading
	to fire up two simulataneously running models.  Check that the
	simultaneously running models have different memory space.

\item Add capability for \fn{create\_benchmark.py} to overwrite
	existing benchmark files.

\item Make \vars{compiled\_form} set to \vars{'parts'} as the
	default instantiation.  Change documentation accordingly.

\item Currently, the \class{Qtcm} \mods{plotm} method works only on
	3-D output (time, latitude, longitude).  Some of the fields
	in the netCDF output files are 2-D.  Add the capability to
	\mods{plot\_netcdf\_output} in the \mods{plot} submodule
	to handle 2-D fields.

\item Add documentation about removing temporary files.
	Add documentation in Section~\ref{sec:model.instances}
	of details of what occurs during instantiation of 
	a \class{Qtcm} instance.

\item Add the units and long names for all field variables in the
	\mods{defaults} module.

\item Create a keyword to automatically change precipitation and
	evaporation units to mm/day (or similar).

\item Add ability to calculate and plot fields at different pressure
	levels.  Create another module like defaults that specifies
	the vertical fields and gives the equation to use to calculate
	those fields; call the module ``derivfields'' or something
	similar.

\item Throughout the \mods{qtcm} package I use the condition
	\mods{N.rank(}\dumarg{arg}\mods{)\thinspace=\thinspace0} 
	to test whether
	\dumarg{arg} is a scalar.  This works fine for \mods{numpy}
	objects, but it does not work properly for
	\mods{Numeric} and \mods{numarray} arrays.  In those
	array packages, \mods{rank('abc')} returns the value~1.
	This is not a problem, as long as everyone has \mods{numpy},
	but in order to make the package interoperable, I need to
	find a better way of testing for scalars.  The definitions
	of isscalar need to be changed in \mods{num\_settings}.

\item \mods{num\_settings} needs to be changed to truly enable me
	to test whether \mods{qtcm} works for 
	\mods{numarray} and \mods{Numeric} arrays.  The tests
	do not do this right now, because \mods{num\_settings}
	defaults to \mods{numpy}, if it exists.

\item Create makefiles for other platforms.
 
\item A few fields (e.g., \vars{u1}) have data for extra latitude bands,
	due to the use of ``ghost latitudes'' as part of the
	implementation of the numerics.  Details are found in the 
\latexhtml{%
\htmladdnormallinkfoot{QTCM1 manual}%
        {http://www.atmos.ucla.edu/$\sim$csi/qtcm\_man/v2.3/qtcm\_manv2.3.pdf}}%
{\htmladdnormallink{QTCM1 manual}%
        {http://www.atmos.ucla.edu/~csi/qtcm_man/v2.3/qtcm_manv2.3.pdf}}
\cite{Neelin/etal:2002}.

	Though adjusting to this idiosyncracy is not that difficult, 
	in the future I hope to implement a method of handing
	fields with ghost latitudes so that they have the same
	dimensions as the other gridded output variables.  In order
	to do this, I plan to write a Python method to read the
	Fortran generated binary restart file.

\item Change the \mods{set\_qtcm\_item} method so that it can 
	automatically accomodate setting Fortran real variables
	if integer values are input.

\item Currently, the \mods{get\_item\_qtcm} and 
	\mods{set\_item\_qtcm} methods will not work
	on integer and character arrays, only scalars and real arrays.
	Add that missing functionality to those methods.

\item Currently, the \mods{make\_snapshot} method duplicates the
	functionality of the pure-Fortran QTCM1 restart file mechanism.
	However, the restart file mechanism itself does not do a true
	restart.  A continuous run does not provide the same results
	as two runs over the same period, joined by the restart file.

	To see whether saving more variables would do the trick,
	I altered \mods{make\_snapshot} to store all Python level
	variables (i.e., \vars{self.\_qtcm\_fields\_ids}).  However,
	the restart failing described above still continued.  In the
	future, I hope to figure out exactly how many variables are
	needed in order to make the restart feature do a true
	restart.

\item Add a test of using the \vars{mrestart\thinspace=\thinspace1}
	restart option.  Does the \fn{qtcm.restart} file need to be
	in the current working directory or another?

\item Add a test in the unit test scripts to
	confirm that the \vars{init\_with\_instance\_state}
	attribute setting only has an effect if 
	\vars{compiled\_form\thinspace=\thinspace'parts'}.

\item Document \vars{tmppreview} keyword in \mods{plot.plot\_ncdf\_output}.

\item Confirm and document that
	for netCDF output, time is model time since dd-mm-yyyy.

\item Add to the \mods{plotm} method the ability to
	plot as text onto the figure the
	runname string and the calling line
	for the plotm method.

\item Couple with the
	\latexhtml{CliMT\footnote{http://maths.ucd.ie/$\sim$rca/climt/}}%
	{\htmladdnormallink{CliMT}{http://maths.ucd.ie/~rca/climt/}}
	climate modeling toolkit.

\item Enable Python to set \vars{arr1name}, etc., which are string
	variables at the Python level.  I haven't really thought through
	how \vars{arr1} variables work with the Python \class{Qtcm}
	instance.

\item Possible:  In the \class{Qtcm} method
	\mods{\_\_setattr\_\_}, add a test to raise an exception
	if the instance tries to set \vars{viscU}, \vars{viscT},
	or \vars{viscQ} as attributes.  Also create a method
	\code{isotropic\_visc} that will set all viscosity parameters
	non-dependent on direction.  See Section~\ref{sec:driverinit.diffs}
	for details.

\item Go through the manual and create HTML-only versions of tables
	that have table numbers (use a similar construct as in
	figure environments).

\item Go through documentation to check that
	output variable names are capitalized consistently.

\item Create way to redirect stdout.

\item Create a step method to run an arbitrary number of timesteps at
	the atmosphere level.

\end{itemize}


% ===== end of file =====





% ----- BACK MATTER OF THE DOCUMENT -----
%
\normalsize
\pagebreak
\bibliographystyle{plain}
\bibliography{/Users/jlin/work/res/bib/master}

%- Uncomment the input line below and comment out the \bibliographystyle
%  and \bibliography lines if you're running this without the master.bib 
%  BibTeX database
%% ==========================================================================
% Manual for QTCM Python Package
%
% Usage:
% - If you are running this on your own system, you will not have a copy of
%   my master.bib BibTeX database.  To run this, you'll have to comment out:
%
%      \bibliographystyle{chicago-jl}
%      \bibliography{/Users/jlin/work/res/bib/master}
%
%   and comment back in:
%
%      \input{manual.bbl}
%
%   in this file.  Then you can use pdflatex on this file to get the PDF of
%   the manual.  These 3 lines are in the back matter of the document.
%
% Revision Notes:
% - By Johnny Lin, North Park University, http://www.johnny-lin.com/
% - The chicago BibTeX style is unrecognized by latex2html, so I use
%   the plain style.
% ==========================================================================


% ------ DOCUMENT DEFINITIONS ------
%
\documentclass[12pt]{book}
\usepackage{color}
\usepackage{html}
\usepackage{graphicx}
\usepackage{textcomp}
%\usepackage{comment}    %- Unrecognized by latex2html; its use causes errors
%\usepackage{fancyvrb}   %- Unrecognized by latex2html; its use causes errors


%- Packages unrecognized by latex2html, but causes no error:
%
%\usepackage[letterpaper,margin=1in,includefoot]{geometry}
\usepackage[letterpaper,margin=1.25in]{geometry}
\usepackage{bibnames}
\usepackage{longtable}
\usepackage{multirow}


%+ Comment out explicity margin settings since use package geometry:
%\setlength{\topmargin}{0in}
%\setlength{\headheight}{0in}
%\setlength{\headsep}{0in}
%\setlength{\oddsidemargin}{0in}
%\setlength{\evensidemargin}{0in}
%\setlength{\textheight}{8.5in}
%\setlength{\textwidth}{6.5in}




% ------ COMMANDS AND LENGTHS ------
%
% --- Define colors:  Have to do this because for some reason LaTeX
%     sometimes looks for "BLUE" instead of "blue" and complains when
%     "BLUE" isn't found.
%
\definecolor{Blue}{rgb}{0,0,1}
\definecolor{BLUE}{rgb}{0,0,1}
\definecolor{green}{rgb}{0,0.6,0}
\definecolor{Green}{rgb}{0,0.6,0}
\definecolor{GREEN}{rgb}{0,0.6,0}


% --- Format code blocks.  Currently set to print out the code in just 
%     typewriter font with no box.  Will work the same for pdflatex 
%     and latex2html:
%
%     codeblock:  Environment for blocks of computer code or internet 
%       addresses.
%     codeblockfont:  Sets font for codeblocks.
%
\newenvironment{codeblock}%
	{\begin{quotation}\begin{minipage}[t]{0.9\textwidth}}%
	{\end{minipage}\end{quotation}}
	%{\begin{flushleft}}%
	%{\end{flushleft}}
\newcommand{\codeblockfont}[1]{\textcolor{blue}{\texttt{#1}}}
%     *** Version that only works for pdflatex that puts a box around 
%         the block and centers it (commented out).  Note that using
%         fancyvrb is the better way of creating such a boxed section
%         of code, but fancyvrb isn't recognized by latex2html:
%\newenvironment{codeblock}%
%	{\begin{center}\begin{tabular}{|c|} \hline \\ }%
%	{\\ \\ \hline \end{tabular}\end{center}}
%\newcommand{\codeblockfont}[1]{\parbox{0.8\textwidth}{\texttt{#1}}}


% --- Text titling/emphasis settings:
%
%     emphpara:  Emphasis for the first phrase or sentence of a 
%         paragraph.
%     booktitle:  Formats book titles.
%     tabletitle:  Title for an item block in the information table.
%     paratitle:  Title for a paragraph in an item block in the
%         information table.
%     emphdate:  Emphasize date in paragraph text.
%
%     cmd:  Commands
%     dumarg:  Dummy arguments
%     codearg:  Same as dumarg.
%     fn:  File and directory names
%     screen:  Screen display
%     vars:  Variable and attribute names
%     mods:  Module, subroutine, and method names
%     class:  Class names
%     code:  Generic code (avoid using this)
%
\newcommand{\emphpara}[1]{\textbf{#1}}
\newcommand{\booktitle}[1]{\textit{#1}}
%\newcommand{\tabletitle}[1]{\textsf{\textbf{#1}}}
\newcommand{\paratitle}[1]{\textit{#1}}
\newcommand{\emphdate}[1]{\textbf{#1}}

\newcommand{\code}[1]{\textcolor{blue}{\texttt{#1}}}
\newcommand{\cmd}[1]{\textcolor{blue}{\texttt{#1}}}
\newcommand{\dumarg}[1]{\textit{#1}}
\newcommand{\codearg}[1]{\textit{#1}}
\newcommand{\fn}[1]{\textsf{\textit{#1}}}
\newcommand{\screen}[1]{\textcolor{green}{\texttt{#1}}}
\newcommand{\vars}[1]{\textcolor{blue}{\texttt{#1}}}
\newcommand{\class}[1]{\textcolor{blue}{\texttt{#1}}}
\newcommand{\mods}[1]{\textcolor{blue}{\texttt{#1}}}


% --- Special table formatting:
%
%     tabletitlewidth:  Width for title field of an item block in the 
%         information table.
%     tablebodywidth:  Width for body field of an item block in the 
%         information table.
%     tabletabulardims:  Dimensions for the information table, used in
%         the tabular command.
%     tableitemlinespace:  Vertical spacing between item blocks in the
%         information table.
%     infotitle and infotext:  Used for two-column sub-information 
%         tables found in the body field of the information table.  
%         These are not global lengths but have values specific to the 
%         local context in which they're used.
%
\newlength{\tabletitlewidth}
\settowidth{\tabletitlewidth}{file and directory names}

\newlength{\tablebodywidth}
\setlength{\tablebodywidth}{0.9\textwidth}
\addtolength{\tablebodywidth}{-4ex}
\addtolength{\tablebodywidth}{-\tabletitlewidth}

\newcommand{\tabletabulardims}%
	{p{\tabletitlewidth}@{\hspace{4ex}}p{\tablebodywidth}}

\newcommand{\tableitemlinespace}{\baselineskip}
\newlength{\infotitle}
\newlength{\infotext}


% --- Lengths for formatting:
%
\newlength{\remainder}        % length to describe the residual of the
                              %   linewidth minus \enumlabel
\newlength{\enumlabel}        % length to describe figure sub-label width
                              %   (e.g. "(a)")


% --- TtH stuff:
%
%\def\tthdump#1{#1}


% --- LaTeX2HTML stuff:
%
%     htmlfigcaption:  Formatting for HTML replacement figure captions.
%
\newcommand{\htmlfigcaption}[1]{\parbox[c]{70ex}{\footnotesize{#1}}}


% --- Some book title abbreviations:
%
%     rute:  Booktitle for Rute User's.
%     linuxnut:  Booktitle for Linux in a Nutshell.
%     pynut:  Booktitle for Python in a Nutshell.
%
\newcommand{\rute}{\booktitle{Rute User's}}
\newcommand{\linuxnut}{\booktitle{Linux in a Nutshell}}
\newcommand{\pynut}{\booktitle{Python in a Nutshell}}


% --- Define special characters ---
%
\newcommand{\aonehat}{\ensuremath{\widehat{a_1}}}
\newcommand{\bonehat}{\ensuremath{\widehat{b_1}}}
\newcommand{\D}{\ensuremath{\mathcal{D}}}
\def\BibTeX{B\kern-.03em i\kern-.03em b\kern-.15em\TeX}




% ------ BEGINNING OF DOCUMENT TEXT ------
%
\begin{document}

    

    
% ------ TITLE AND TOC ------
%
\title{\mods{qtcm} User's Guide}
\author{Johnny Wei-Bing Lin\thanks{Physics Department, North Park University,
	3225 W.\ Foster Ave., Chicago, IL  60625, USA}}
\date{\today}
\maketitle
\tableofcontents




% ------ BODY ------
%
\chapter{Introduction}
\input{intro}

\chapter{Installation and Configuration}    \label{ch:install}
	\section{Summary and Conventions}      \label{sec:install.sum}
	\input{install_sum}
	\section{Fortran Compiler}             \label{sec:fort.compilers}
	\input{install_fort}
	\section{Required Packages}            \label{sec:py.etc.pkgs}
	\input{install_pkgs}
	\section{Compiling Extension Modules}  \label{sec:create.so}
	\input{compile_so}
	\section{Testing the Installation}     \label{sec:test.qtcm}
	\input{test_qtcm}
	\section{Model Performance}
	\input{perform}
	\section{Installing in Mac OS X}       \label{sec:install.macosx}
	\input{qtcm_in_macosx}
	\section{Installing in Ubuntu}         \label{sec:install.ubuntu}
	\input{qtcm_in_ubuntu}

\chapter{Getting Started With \mods{qtcm}}  \label{ch:getting.started}
\input{started}

\chapter{Using \mods{qtcm}}                 \label{ch:using}
\input{using}

%@@@\chapter{Combining \code{qtcm} with \code{CliMT}}
%@@@\input{climt}

\chapter{Troubleshooting}                   \label{ch:trouble}
\input{trouble}

\chapter{Developer Notes}                   \label{ch:devnotes}
\input{devnotes}

\chapter{Future Work}                       \label{ch:future}
\input{future}




% ----- BACK MATTER OF THE DOCUMENT -----
%
\normalsize
\pagebreak
\bibliographystyle{plain}
\bibliography{/Users/jlin/work/res/bib/master}

%- Uncomment the input line below and comment out the \bibliographystyle
%  and \bibliography lines if you're running this without the master.bib 
%  BibTeX database
%\input{manual.bbl}        

\appendix
\chapter{Field Settings in \mods{defaults}}  \label{app:defaults.values}
\input{defaults}




% ------ END OF DOCUMENT TEXT ------
%
\end{document}


% ===== end of file =====
        

\appendix
\chapter{Field Settings in \mods{defaults}}  \label{app:defaults.values}
% ==========================================================================
% Appendix:  Defaults from the submodule defaults
%
% By Johnny Lin
% ==========================================================================


% ------ BODY -----
%
%---------------------------------------------------------------------------
\section{Scalar Field Variables}  \label{sec:defaults.scalar}

This table lists the default settings for scalar \mods{qtcm} fields
as set by the \mods{defaults} submodule.  All fields are of class
\class{Field}.  Numerical values are rounded as per the conventions
of Python's \vars{\%g} format code.
To create a \class{Field} instance whose value is set to the
default, instantiate with the field id as the argument

\input{defaults_scalars}




%---------------------------------------------------------------------------
\section{Array Field Variables}   \label{sec:defaults.array}

This table lists the default settings for array \mods{qtcm} fields
as set by the \mods{defaults} submodule.  All fields are of class
\class{Field}.  Numerical values are rounded as per the conventions
of Python's \vars{\%g} format code.

\input{defaults_arrays}




% ===== end of file =====





% ------ END OF DOCUMENT TEXT ------
%
\end{document}


% ===== end of file =====
        

\appendix
\chapter{Field Settings in \mods{defaults}}  \label{app:defaults.values}
% ==========================================================================
% Appendix:  Defaults from the submodule defaults
%
% By Johnny Lin
% ==========================================================================


% ------ BODY -----
%
%---------------------------------------------------------------------------
\section{Scalar Field Variables}  \label{sec:defaults.scalar}

This table lists the default settings for scalar \mods{qtcm} fields
as set by the \mods{defaults} submodule.  All fields are of class
\class{Field}.  Numerical values are rounded as per the conventions
of Python's \vars{\%g} format code.
To create a \class{Field} instance whose value is set to the
default, instantiate with the field id as the argument

% This file is automatically generated by the script
% code_to_latex.py in the doc/latex directory.  It is based upon
% the values found in the qtcm.defaults submodule, and should
% not be hand-edited if you want the values to correspond to
% the values in the qtcm.defaults submodule.
        

\begin{longtable}{l|c|c|p{0.30\linewidth}}
\textbf{Field} & \textbf{Value} & \textbf{Units} & 
                                \textbf{Description} \\
\hline
\endhead
\vars{SSTdir} & ../bnddata/SST\_Reynolds &  & Where SST files are \\
\vars{SSTmode} & seasonal &  & Decide what kind of SST to use \\
\vars{VVsmin} & 4.5 & m/s & Minimum wind speed for fluxes \\
\vars{bnddir} & ../bnddata &  & Boundary data other than SST \\
\vars{dateofmodel} & 0 &  & Date of model coded as an integer as yyyymmdd \\
\vars{day0} & -1 & dy & Starting day; if $<$ 0 use day in restart \\
\vars{dt} & 1200 & s & Time step \\
\vars{eps\_c} & 0.000138889 & 1/s & 1/tau\_c NZ (5.7) \\
\vars{interval} & 1 & dy & Atmosphere-ocean coupling interval \\
\vars{it} & 1 &  & Time of day in time steps \\
\vars{landon} & 1 &  & If not 1: land = ocean with fake SST \\
\vars{lastday} & 365 & dy & Last day of integration \\
\vars{month0} & -1 & mo & Starting month; if $<$ 0 use mo in restart \\
\vars{mrestart} & 0 &  & =1: restart using qtcm.restart \\
\vars{mt0} & 1 &  & Barotropic timestep every mt0 timesteps \\
\vars{nastep} & 1 &  & Number of atmosphere time steps within one air-sea coupling interval \\
\vars{noout} & 0 & dy & No output for the first noout days \\
\vars{nooutr} & 0 & dy & No restart file for the first nooutr days \\
\vars{ntout} & -30 & dy & Monthly mean output \\
\vars{ntouti} & 0 & dy & Monthly instantaneous data output \\
\vars{ntoutr} & 0 & dy & Restart file only at end of model run \\
\vars{outdir} & ../proc/qtcm\_output &  & Where output goes to \\
\vars{runname} & runname &  & String for an output filename \\
\vars{title} & QTCM default title &  & A descriptive title \\
\vars{u0bar} & 0 &  &  \\
\vars{visc4x} & 700000 & m$^2$/s & Del 4 viscocity parameter in x \\
\vars{visc4y} & 700000 & m$^2$/s & Del 4 viscocity parameter in y \\
\vars{viscxT} & 1.2e+06 & m$^2$/s & Temperature diffusion parameter in x \\
\vars{viscxq} & 1.2e+06 & m$^2$/s & Humidity diffusion parameter in x \\
\vars{viscxu0} & 700000 & m$^2$/s & Viscocity parameter for u0 in x \\
\vars{viscxu1} & 700000 & m$^2$/s & Viscocity parameter for u1 in x \\
\vars{viscyT} & 1.2e+06 & m$^2$/s & Temperature diffusion parameter in y \\
\vars{viscyq} & 1.2e+06 & m$^2$/s & Humidity diffusion parameter in y \\
\vars{viscyu0} & 700000 & m$^2$/s & Viscocity parameter for u0 in y \\
\vars{viscyu1} & 700000 & m$^2$/s & Viscocity parameter for u1 in y \\
\vars{weml} & 0.01 & m/s & Mixed layer entrainment velocity \\
\vars{year0} & 0 & yr & Starting year; if $<$ 0 use year in restart \\
\vars{ziml} & 500 & m & Atmosphere mixed layer depth $\sim$ cloud base \\
\end{longtable}





%---------------------------------------------------------------------------
\section{Array Field Variables}   \label{sec:defaults.array}

This table lists the default settings for array \mods{qtcm} fields
as set by the \mods{defaults} submodule.  All fields are of class
\class{Field}.  Numerical values are rounded as per the conventions
of Python's \vars{\%g} format code.

% This file is automatically generated by the script
% code_to_latex.py in the doc/latex directory.  It is based upon
% the values found in the qtcm.defaults submodule, and should
% not be hand-edited if you want the values to correspond to
% the values in the qtcm.defaults submodule.
        

\begin{longtable}{l|c|c|c|c|p{0.37\linewidth}}
\textbf{Field} & \textbf{Shape} & \textbf{Max} & \textbf{Min} &
                                \textbf{Units} & \textbf{Description} \\
\hline
\endhead
\vars{Evap} & (1, 1) & 0 & 0 &  &  \\
\vars{FLW} & (1, 1) & 0 & 0 &  &  \\
\vars{FLWds} & (1, 1) & 0 & 0 &  &  \\
\vars{FLWus} & (1, 1) & 0 & 0 &  &  \\
\vars{FLWut} & (1, 1) & 0 & 0 &  &  \\
\vars{FSW} & (1, 1) & 0 & 0 &  &  \\
\vars{FSWds} & (1, 1) & 0 & 0 &  &  \\
\vars{FSWus} & (1, 1) & 0 & 0 &  &  \\
\vars{FSWut} & (1, 1) & 0 & 0 &  &  \\
\vars{FTs} & (1, 1) & 0 & 0 &  &  \\
\vars{Qc} & (1, 1) & 0 & 0 & K & Precipitation \\
\vars{S0} & (1, 1) & 0 & 0 &  &  \\
\vars{STYPE} & (1, 1) & 0 & 0 &  & Surface type; ocean or vegetation type over land \\
\vars{T1} & (1, 1) & 0 & 0 & K &  \\
\vars{Ts} & (1, 1) & 0 & 0 & K & Surface temperature \\
\vars{WD} & (1, 1) & 0 & 0 &  &  \\
\vars{WD0} & (1,) & 0 & 0 &  & Field capacity SIB2/CSU (approximately) \\
\vars{arr1} & (1, 1) & 0 & 0 &  & Auxiliary optional output array 1 \\
\vars{arr2} & (1, 1) & 0 & 0 &  & Auxiliary optional output array 2 \\
\vars{arr3} & (1, 1) & 0 & 0 &  & Auxiliary optional output array 3 \\
\vars{arr4} & (1, 1) & 0 & 0 &  & Auxiliary optional output array 4 \\
\vars{arr5} & (1, 1) & 0 & 0 &  & Auxiliary optional output array 5 \\
\vars{arr6} & (1, 1) & 0 & 0 &  & Auxiliary optional output array 6 \\
\vars{arr7} & (1, 1) & 0 & 0 &  & Auxiliary optional output array 7 \\
\vars{arr8} & (1, 1) & 0 & 0 &  & Auxiliary optional output array 8 \\
\vars{psi0} & (1, 1) & 0 & 0 &  &  \\
\vars{q1} & (1, 1) & 0 & 0 & K &  \\
\vars{rhsu0bar} & (1,) & 0 & 0 &  &  \\
\vars{rhsvort0} & (1, 1, 1) & 0 & 0 &  &  \\
\vars{taux} & (1, 1) & 0 & 0 &  &  \\
\vars{tauy} & (1, 1) & 0 & 0 &  &  \\
\vars{u0} & (1, 1) & 0 & 0 & m/s & Barotropic zonal wind \\
\vars{u1} & (1, 1) & 0 & 0 & m/s & Current time step baroclinic zonal wind \\
\vars{v0} & (1, 1) & 0 & 0 & m/s & Barotropic meridional wind \\
\vars{v1} & (1, 1) & 0 & 0 & m/s &  \\
\vars{vort0} & (1, 1) & 0 & 0 &  &  \\
\end{longtable}





% ===== end of file =====





% ------ END OF DOCUMENT TEXT ------
%
\end{document}


% ===== end of file =====

%
%   in this file.  Then you can use pdflatex on this file to get the PDF of
%   the manual.  These 3 lines are in the back matter of the document.
%
% Revision Notes:
% - By Johnny Lin, North Park University, http://www.johnny-lin.com/
% - The chicago BibTeX style is unrecognized by latex2html, so I use
%   the plain style.
% ==========================================================================


% ------ DOCUMENT DEFINITIONS ------
%
\documentclass[12pt]{book}
\usepackage{color}
\usepackage{html}
\usepackage{graphicx}
\usepackage{textcomp}
%\usepackage{comment}    %- Unrecognized by latex2html; its use causes errors
%\usepackage{fancyvrb}   %- Unrecognized by latex2html; its use causes errors


%- Packages unrecognized by latex2html, but causes no error:
%
%\usepackage[letterpaper,margin=1in,includefoot]{geometry}
\usepackage[letterpaper,margin=1.25in]{geometry}
\usepackage{bibnames}
\usepackage{longtable}
\usepackage{multirow}


%+ Comment out explicity margin settings since use package geometry:
%\setlength{\topmargin}{0in}
%\setlength{\headheight}{0in}
%\setlength{\headsep}{0in}
%\setlength{\oddsidemargin}{0in}
%\setlength{\evensidemargin}{0in}
%\setlength{\textheight}{8.5in}
%\setlength{\textwidth}{6.5in}




% ------ COMMANDS AND LENGTHS ------
%
% --- Define colors:  Have to do this because for some reason LaTeX
%     sometimes looks for "BLUE" instead of "blue" and complains when
%     "BLUE" isn't found.
%
\definecolor{Blue}{rgb}{0,0,1}
\definecolor{BLUE}{rgb}{0,0,1}
\definecolor{green}{rgb}{0,0.6,0}
\definecolor{Green}{rgb}{0,0.6,0}
\definecolor{GREEN}{rgb}{0,0.6,0}


% --- Format code blocks.  Currently set to print out the code in just 
%     typewriter font with no box.  Will work the same for pdflatex 
%     and latex2html:
%
%     codeblock:  Environment for blocks of computer code or internet 
%       addresses.
%     codeblockfont:  Sets font for codeblocks.
%
\newenvironment{codeblock}%
	{\begin{quotation}\begin{minipage}[t]{0.9\textwidth}}%
	{\end{minipage}\end{quotation}}
	%{\begin{flushleft}}%
	%{\end{flushleft}}
\newcommand{\codeblockfont}[1]{\textcolor{blue}{\texttt{#1}}}
%     *** Version that only works for pdflatex that puts a box around 
%         the block and centers it (commented out).  Note that using
%         fancyvrb is the better way of creating such a boxed section
%         of code, but fancyvrb isn't recognized by latex2html:
%\newenvironment{codeblock}%
%	{\begin{center}\begin{tabular}{|c|} \hline \\ }%
%	{\\ \\ \hline \end{tabular}\end{center}}
%\newcommand{\codeblockfont}[1]{\parbox{0.8\textwidth}{\texttt{#1}}}


% --- Text titling/emphasis settings:
%
%     emphpara:  Emphasis for the first phrase or sentence of a 
%         paragraph.
%     booktitle:  Formats book titles.
%     tabletitle:  Title for an item block in the information table.
%     paratitle:  Title for a paragraph in an item block in the
%         information table.
%     emphdate:  Emphasize date in paragraph text.
%
%     cmd:  Commands
%     dumarg:  Dummy arguments
%     codearg:  Same as dumarg.
%     fn:  File and directory names
%     screen:  Screen display
%     vars:  Variable and attribute names
%     mods:  Module, subroutine, and method names
%     class:  Class names
%     code:  Generic code (avoid using this)
%
\newcommand{\emphpara}[1]{\textbf{#1}}
\newcommand{\booktitle}[1]{\textit{#1}}
%\newcommand{\tabletitle}[1]{\textsf{\textbf{#1}}}
\newcommand{\paratitle}[1]{\textit{#1}}
\newcommand{\emphdate}[1]{\textbf{#1}}

\newcommand{\code}[1]{\textcolor{blue}{\texttt{#1}}}
\newcommand{\cmd}[1]{\textcolor{blue}{\texttt{#1}}}
\newcommand{\dumarg}[1]{\textit{#1}}
\newcommand{\codearg}[1]{\textit{#1}}
\newcommand{\fn}[1]{\textsf{\textit{#1}}}
\newcommand{\screen}[1]{\textcolor{green}{\texttt{#1}}}
\newcommand{\vars}[1]{\textcolor{blue}{\texttt{#1}}}
\newcommand{\class}[1]{\textcolor{blue}{\texttt{#1}}}
\newcommand{\mods}[1]{\textcolor{blue}{\texttt{#1}}}


% --- Special table formatting:
%
%     tabletitlewidth:  Width for title field of an item block in the 
%         information table.
%     tablebodywidth:  Width for body field of an item block in the 
%         information table.
%     tabletabulardims:  Dimensions for the information table, used in
%         the tabular command.
%     tableitemlinespace:  Vertical spacing between item blocks in the
%         information table.
%     infotitle and infotext:  Used for two-column sub-information 
%         tables found in the body field of the information table.  
%         These are not global lengths but have values specific to the 
%         local context in which they're used.
%
\newlength{\tabletitlewidth}
\settowidth{\tabletitlewidth}{file and directory names}

\newlength{\tablebodywidth}
\setlength{\tablebodywidth}{0.9\textwidth}
\addtolength{\tablebodywidth}{-4ex}
\addtolength{\tablebodywidth}{-\tabletitlewidth}

\newcommand{\tabletabulardims}%
	{p{\tabletitlewidth}@{\hspace{4ex}}p{\tablebodywidth}}

\newcommand{\tableitemlinespace}{\baselineskip}
\newlength{\infotitle}
\newlength{\infotext}


% --- Lengths for formatting:
%
\newlength{\remainder}        % length to describe the residual of the
                              %   linewidth minus \enumlabel
\newlength{\enumlabel}        % length to describe figure sub-label width
                              %   (e.g. "(a)")


% --- TtH stuff:
%
%\def\tthdump#1{#1}


% --- LaTeX2HTML stuff:
%
%     htmlfigcaption:  Formatting for HTML replacement figure captions.
%
\newcommand{\htmlfigcaption}[1]{\parbox[c]{70ex}{\footnotesize{#1}}}


% --- Some book title abbreviations:
%
%     rute:  Booktitle for Rute User's.
%     linuxnut:  Booktitle for Linux in a Nutshell.
%     pynut:  Booktitle for Python in a Nutshell.
%
\newcommand{\rute}{\booktitle{Rute User's}}
\newcommand{\linuxnut}{\booktitle{Linux in a Nutshell}}
\newcommand{\pynut}{\booktitle{Python in a Nutshell}}


% --- Define special characters ---
%
\newcommand{\aonehat}{\ensuremath{\widehat{a_1}}}
\newcommand{\bonehat}{\ensuremath{\widehat{b_1}}}
\newcommand{\D}{\ensuremath{\mathcal{D}}}
\def\BibTeX{B\kern-.03em i\kern-.03em b\kern-.15em\TeX}




% ------ BEGINNING OF DOCUMENT TEXT ------
%
\begin{document}

    

    
% ------ TITLE AND TOC ------
%
\title{\mods{qtcm} User's Guide}
\author{Johnny Wei-Bing Lin\thanks{Physics Department, North Park University,
	3225 W.\ Foster Ave., Chicago, IL  60625, USA}}
\date{\today}
\maketitle
\tableofcontents




% ------ BODY ------
%
\chapter{Introduction}
%=====================================================================
% Introduction
%=====================================================================


% ----- BEGIN TEXT -----
%
%---------------------------------------------------------------------
\section{How to Read This Manual}

\emphpara{Most users:} 
Just read 
(1) the installation instructions in Chapter~\ref{ch:install},
(2) Chapter~\ref{ch:getting.started},
which tells you all you need to get started using \mods{qtcm}, and
(3) examples in Section~\ref{sec:cookbook} that give a feel
for how you can use the model.

\emphpara{Users having problems:}
Chapter~\ref{ch:trouble} provides troubleshooting tips for
a few problems.
The detailed description of how the package functions, 
in Chapter~\ref{ch:using}, will probably be more useful.

\emphpara{Developers:}
If you want to change the source code, please read
Chapter~\ref{ch:devnotes}.  Chapter~\ref{ch:future} describes
all the things I'd like to do to improve the package, but haven't
gotten to yet.




%---------------------------------------------------------------------
\section{About the Package}

The single-baroclinic mode
Neelin-Zeng Quasi-Equilibrium Tropical Circulation Model
\latexhtml{(QTCM1)\footnote{http://www.atmos.ucla.edu/$\sim$csi}}%
	{\htmladdnormallink{(QTCM1)}{http://www.atmos.ucla.edu/~csi}}
is a primitive equation-based intermediate-level atmospheric model
that focuses on simulating the tropical atmosphere.  Being more
complicated than a simple model, the model has full non-linearity
with a basic representation of baroclinic instability,
includes a radiative-convective feedback package, and includes a
simple land soil moisture routine (but does not include topography).
A brief, but more detailed, description of QTCM1 is given in
Section~\ref{sec:brief_qtcm}.

\htmladdnormallinkfoot{Python}{http://www.python.org}
is an interpreted, object-oriented, multi-platform,
open-source language that is used in a variety of software applications,
ranging from game development to bioinformatics.
In climate studies, Python has been used as the core language for
data analysis
(e.g., \htmladdnormallinkfoot{Climate Data Analysis Tools}{http://cdat.sf.net}),
visualization
(e.g., \htmladdnormallinkfoot{Matplotlib}{http://matplotlib.sf.net}),
and 
modeling
(e.g., \htmladdnormallinkfoot{PyCCSM}{http://code.google.com/p/pyccsm/}).

In comparison to traditional compiled languages like Fortran,
Python's lack of a separate compile step greatly simplifies the
debugging and testing phases of development, because code snippets
can be testing as code is written.
Python's extensive suite of higher-level tools (e.g., statistics,
visualization, string and file manipulation) accessible at runtime 
enables modeling and analysis to occur concurrently.  

The \mods{qtcm} package is an implementation of the Neelin-Zeng
QTCM1 in a Python object-oriented environment.  The conversion
package
\htmladdnormallinkfoot{\mods{f2py}}{http://cens.ioc.ee/projects/f2py2e/} is
used to wrap the QTCM1 Fortran model routines and manage model
execution using Python datatypes and utilities.  The result is a
modeling package where order and choice of subroutine execution can
be altered at runtime.  Model analysis and visualization can also
be integrated with model execution at runtime.




%---------------------------------------------------------------------
\section{Conventions In This Manual}

	\subsection{Audience}

In this manual I assume you have a rudimentary knowledge of Python.
Thus, I do not describe basic Python data types (e.g., dictionaries,
lists), object framework and syntax (e.g., classes, methods,
attributes, instantiation), module and package importing.  If you
need to brush up (or learn) Python, I'd recommend the following
resources:

\begin{itemize}
\item \htmladdnormallinkfoot{Python Tutorial:}{http://docs.python.org/tut/}
	This tutorial was written by Guido van Rossum, Python's original
	author.

\item \htmladdnormallinkfoot{Instant Hacking:}%
	{http://www.hetland.org/python/instant-hacking.php}
	Learn how to program with Python.

\item \htmladdnormallinkfoot{Dive Into Python:}%
	{http://diveintopython.org/index.html}
	Reasonably complete and cohesive. The entire book is available for 
	free online.

\item \htmladdnormallinkfoot{Handbook of the Physics Computing Course:}%
	{http://www.pentangle.net/python/handbook/}
	Written for a science audience.  Recommended.

\item \latexhtml{CDAT/Python Tips for Earth Scientists:\footnote%
	{http://www.johnny-lin.com/cdat\_tips/}}%
	{\htmladdnormallink{CDAT/Python Tips for Earth Scientists:}%
		{http://www.johnny-lin.com/cdat_tips/}}
	This web site is a FAQ of sorts for people using Python and
	the Climate Data Analysis Tools (CDAT) in the earth sciences,
	and thus focuses on using Python to do science rather than
	the computer science aspects of the language.

\end{itemize}

The purpose of this package is to make the QTCM1 model easier to
use.  In order to interpret the results, however, you still need
to have a robust sense of what climate models can and cannot tell
you.  A starting point for the QTCM1 model is the brief description
of the model in Section~\ref{sec:brief_qtcm}.  After that, I would
read the original papers describing the model formulation and results
\cite{Neelin/Zeng:2000,Zeng/etal:2000}, and 
\latexhtml{papers citing the model formulation work.\footnote%
{http://scholar.google.com/scholar?hl=en\&lr=\&cites=14217886709842286738}}%
{\htmladdnormallink{papers citing the model formulation work}%
{http://scholar.google.com/scholar?hl=en&lr=&cites=14217886709842286738}.}
Being an intermediate-level model using the quasi-equilibrium assumption,
QTCM1 (and thus \mods{qtcm}) has distinct strengths and limitations; 
please be aware of them.


	\subsection{Typographic Conventions}

\begin{center}
\begin{tabular}{\tabletabulardims}
\cmd{commands} & to be typed at the command-line
	are rendered in a 
	blue, serif, fixed-width typewriter font
	(e.g., \cmd{make \_qtcm\_full\_365}). \\ \hline
\dumarg{dummy arguments} &
	coupled with code or screen display is rendered in a 
	serif, proportional, italicized font
	(e.g., \screen{Error-Value too long in} \dumarg{variable}). \\ \hline
\fn{file and directory names} & are rendered in a 
	sans-serif, italicized font
	(e.g., \fn{setbypy.F90}). \\ \hline
\screen{screen display} & is rendered in a 
	green, serif, fixed-width typewriter font. \\ \hline
\mods{module, method, and subroutine names} & are rendered in a 
	blue, serif, fixed-width typewriter font. \\ \hline
\vars{variable and attribute names} & are rendered in a 
	blue, serif, fixed-width typewriter font. \\ \hline
\class{class names} & are rendered in a 
	blue, serif, fixed-width typewriter font.
\end{tabular}
\end{center}

Blocks of code (usually commands, screen display, and module,
variable, and class names) are displayed in a blue, serif, fixed-width
typewriter font.


	\subsection{Terminology}

\begin{description}
\item[attribute and method references:]
	If there is any possibility of confusion, I will give the
	class that an attribute or method comes from when that
	attribute or method is referenced.  If no class is mentioned
	by name or context,
	assume that the attribute/method comes from the
	\class{Qtcm} class.

\item[``compiled QTCM1 model'':]
	This usually is used to denote when I'm talking about
	compiled Fortran QTCM1 routines and variables therein,
	as an extension module to the Python \mods{qtcm} package..
	Thus, ``compiled QTCM1 model \vars{u1}'' is the value
	of variable \vars{u1} in the Fortran model, not the
	value at the Python-level.  Sometimes I refer to the
	compiled QTCM1 model as just ``QTCM1'' or as
	``compiled QTCM1 Fortran model''.

\item[``pure-Fortran QTCM1'':]
	This refers to the Neelin-Zeng QTCM1 model in it's
	original Fortran form, not as an extension module to
	the Python \mods{qtcm} package.

\item[``Python-level'':]
	This usually denotes the value of a variable as an
	attribute of a \class{Qtcm} instance.  This variable
	is stored at the Python interpreter level.

\item[\class{Qtcm}:]
	This refers to the Python \class{Qtcm} class
	(note the capitalized first letter).

\item[\mods{qtcm}:]
	This refers to the Python \mods{qtcm} package.

\item[QTCM1 vs.\ QTCM:]
	Although the QTCM1 is currently the only version of a
	quasi-equilibrium tropical circulation model (QTCM), in
	principle one can construct a QTCM with any number of
	baroclinic modes.  In anticipation of this, the names of
	the Python package and class do not end in a numeral.  In
	this manual and the \mods{qtcm} docstrings, QTCM and QTCM1
	are used interchangably.
	Usually either of these phrases mean the quasi-equilibrium
	tropical circulation model in a generic sense, regardless
	of its form of implementation.
\end{description}




%---------------------------------------------------------------------
\section{Current Version Information and Acknowledgments}  \label{sec:ver}

% This file is automatically generated by
    % code_to_latex.py.

This manual describes version 0.1.2 (dated September 12, 2008), of package \mods{qtcm}.
Johnny Linis the primary author of the package.

The \mods{qtcm} package is built upon the pure-Fortran QTCM1 model,
version 2.3 (August 2002), with a few minor changes.
Those changes are described in detail in
Section~\ref{sec:f90changes}.

The homepage for the \mods{qtcm} package is
\htmladdnormallink{http://www.johnny-lin.com/py\_pkgs/qtcm}%
	{http://www.johnny-lin.com/py_pkgs/qtcm}.
All Python code in this package, 
and the Fortran files \fn{setbypy.F90} and \fn{wrapcall.F90},
are \copyright\ 2003--2008 by 
\htmladdnormallinkfoot{Johnny Lin}%
		{http://www.johnny-lin.com} 
and constitutes a
library that is covered under the GNU Lesser General Public License
(LGPL):

\begin{quotation}
	This library is free software; you can redistribute it
	and/or modify it under the terms of the 
	\htmladdnormallinkfoot{GNU Lesser General Public License}%
		{http://www.gnu.org/copyleft/lesser.html} 
	as published by
	the Free Software Foundation; either version 2.1 of the
	License, or (at your option) any later version.

	This library is distributed in the hope that it will be
	useful, but WITHOUT ANY WARRANTY; without even the implied
	warranty of MERCHANTABILITY or FITNESS FOR A PARTICULAR
	PURPOSE. See the GNU Lesser General Public License for more
	details.

	You should have received a copy of the GNU Lesser General
	Public License along with this library; if not, write to
	the Free Software Foundation, Inc., 59 Temple Place, Suite
	330, Boston, MA 02111-1307 USA.

	You can contact Johnny Lin at his email address 
	or at North Park University, Physics Department,
	3225 W. Foster Ave., Chicago, IL 60625, USA.  
\end{quotation}

All other Fortran code in this package, as well as the makefiles,
are covered by licenses (if any) chosen by their respective owners.

This manual, in all forms (e.g., HTML, PDF, \LaTeX),
is part of the documentation of the \mods{qtcm} package 
and is \copyright\ 2007--2008 by Johnny Lin.
Permission is granted to copy, distribute and/or modify 
this document under the terms of the 
GNU Free Documentation License, Version 1.2 
or any later version published by the Free Software Foundation; 
with no Invariant Sections, no Front-Cover Texts, 
and no Back-Cover Texts. 
A copy of the license can be found 
\htmladdnormallinkfoot{here}{http://www.gnu.org/licenses/fdl.html}.

Transparent copies of this document are located online in
\latexhtml{%
\htmladdnormallinkfoot{PDF}%
	{http://www.johnny-lin.com/py\_pkgs/qtcm/doc/manual.pdf}}%
{\htmladdnormallink{PDF}%
	{http://www.johnny-lin.com/py_pkgs/qtcm/doc/manual.pdf}}
and
\latexhtml{%
\htmladdnormallinkfoot{HTML}%
	{http://www.johnny-lin.com/py\_pkgs/qtcm/doc/}}%
{\htmladdnormallink{HTML}%
	{http://www.johnny-lin.com/py_pkgs/qtcm/doc/}}
formats.
The \LaTeX\ source files are distributed with the \mods{qtcm}
distribution.
While the HTML version is nearly identical to the PDF
and \LaTeX\ versions, not every feature in the manual was successfully
converted.  This is particularly true with figures, which are
unnumbered in the HTML version and may be formatted differently
than the authoritative PDF version.
Thus, please consider the \LaTeX\ version as the authoritative
version.

\vspace{\baselineskip}

\emphpara{Acknowledgements:}
Thanks to David Neelin and Ning Zeng and the Climate Systems
Interactions Group at UCLA for their encouragement and help.
On the Python side,
thanks to Alexis Zubrow, Christian Dieterich, Rodrigo Caballero,
Michael Tobis, and Ray Pierrehumbert for steering me straight.
Early versions of some of this work was carried out 
at the University of Chicago Climate Systems Center, 
funded by the National Science Foundation (NSF) 
Information Technology Research Program under grant ATM-0121028. 
Any opinions, findings and conclusions or recommendations 
expressed in this material are those of the author and 
do not necessarily reflect the views of the NSF.

Intel\textregistered\ and
   Xeon\textregistered\ are registered trademarks of Intel Corporation.
Matlab\textregistered\ is a registered trademark of The MathWorks.
UNIX\textregistered\ is a registered trademark of The Open Group.




%---------------------------------------------------------------------
\section{Summary of Release History}

\begin{itemize}
\item 2008 Sep 12:  Version 0.1.2 released.  Summary of changes:
	\begin{itemize}
	\item Create \class{Qtcm} method \mods{get\_qtcm1\_item}.
		This method is effectively an alias of method 
		\mods{get\_qtcm\_item}.
	\item Create \class{Qtcm} method \mods{set\_qtcm1\_item}.
		This method is effectively an alias of method 
		\mods{set\_qtcm\_item}.
	\item Update User's Guide to phase out references to
		the \mods{get\_qtcm\_item}
		and \mods{set\_qtcm\_item} methods.  
		Adding the ``1'' to the method names makes the purpose
		of the methods clearer.
	\item Add unit tests to cover methods \mods{get\_qtcm1\_item} and
		\mods{set\_qtcm1\_item}.
	\end{itemize}

\item 2008 Jul 30:  Updates to the User's Guide (just the online versions,
        not the copies released with the tarball).

\item 2008 Jul 15:  First publicly available distribution 
	released (v0.1.1).
\end{itemize}




%---------------------------------------------------------------------
\section{A Brief Description of The QTCM1}   \label{sec:brief_qtcm}

This description is copied from Ch.\ 3 of Lin \cite{Lin:2000}, 
with minor revisions.
Model formulation is fully described in
Neelin \& Zeng \cite{Neelin/Zeng:2000} and model
results are described in Zeng et~al.\ \cite{Zeng/etal:2000}.
Neelin \& Zeng \cite{Neelin/Zeng:2000} is based upon v2.0 of QTCM1
and Zeng et~al.\ \cite{Zeng/etal:2000} is based on QTCM1 v2.1.
The 
\latexhtml{%
\htmladdnormallinkfoot{QTCM1 manual}%
	{http://www.atmos.ucla.edu/$\sim$csi/qtcm\_man/v2.3/qtcm\_manv2.3.pdf}}%
{\htmladdnormallink{QTCM1 manual}%
	{http://www.atmos.ucla.edu/~csi/qtcm_man/v2.3/qtcm_manv2.3.pdf}}
\cite{Neelin/etal:2002}
describes the details of model implementation.

QTCM1 differs from most full-scale GCMs primarily in how the vertical
temperature, humidity, and velocity structure of the atmosphere is
represented.  First, instead of representing the vertical structure
by finite-differenced levels, the model uses a Galerkin expansion
in the vertical.  The vertical basis functions are chosen according
to analytical solutions under convective quasi-equilibrium conditions,
so only a few need be retained.
Temperature and humidity are each described by separate
vertical basis functions ($a_1$ and $b_1$, respectively).
Low-level variations in the humidity basis
are larger than in the temperature basis.
For velocity, QTCM1 uses a single baroclinic basis function ($V_1$)
defined consistently with the temperature basis function,
as well as a barotropic velocity mode ($V_0$).
The vertical profiles of $a_1$, $b_1$, and $V_1$
are given in Figure~\ref{fig:qtcm.basis}.
Currently, QTCM1 does not include a separate
vertical degree of freedom describing the PBL.
The horizontal grid spacing of the model is 
$5.625^{\circ}$ longitude by $3.75^{\circ}$ latitude.


% <QTCM1 beta version vertical structure modes>
%
% (1) LaTeX version:
%
\begin{latexonly}
\begin{figure}
   \noindent
   \begin{minipage}[b]{.49\linewidth}
      \settowidth{\enumlabel}{(a) }%
      \setlength{\remainder}{\linewidth}% 
      \addtolength{\remainder}{-\enumlabel}
      {(a)}~\makebox[\remainder]{$a_1$ and $b_1$}
      \centering\includegraphics[width=\linewidth,viewport=58 72 389 344,clip]%
                    {figs/a1b1.pdf}
   \end{minipage}\hfill
   \begin{minipage}[b]{.49\linewidth}
      \settowidth{\enumlabel}{(b) }%
      \setlength{\remainder}{\linewidth}% 
      \addtolength{\remainder}{-\enumlabel}
      {(b)}~\makebox[\remainder]{$V_1$}

      \centering\includegraphics[width=\linewidth,viewport=58 72 389 346,clip]%
                    {figs/V1.pdf}
   \end{minipage}

   \caption{Vertical profiles of basis functions for
		(a) temperature $a_1$ (solid) and humidity $b_1$ (dashed) and
		(b) baroclinic component of
		horizontal velocity $V_1$.}
   \label{fig:qtcm.basis}
\end{figure}
\end{latexonly}

% (2) HTML replacement version:
%
\begin{htmlonly}
\label{fig:qtcm.basis}
\begin{center}
\htmladdimg{../latex/figs/a1b1.png}
\htmladdimg{../latex/figs/V1.png}

\htmlfigcaption{Figure \ref{fig:qtcm.basis}:  
	Vertical profiles of basis functions for
   	(a) temperature $a_1$ (solid) and humidity $b_1$ (dashed) and
   	(b) baroclinic component of
   	horizontal velocity $V_1$.}
\end{center}
\end{htmlonly}


These modes are chosen to accurately capture deep convective regions.
Outside deep convective regions the mode
is simply a highly truncated
Galerkin representation.  The system is much more tightly constrained than
a full-scale GCM, yet hopefully retains the essential dynamics and nonlinear
feedbacks.  The result is that QTCM1 is easier to diagnose than a GCM,
and is computationally fast (about 8 minutes per year on a Sun Ultra 2
workstation).  Zeng et al.\ \cite{Zeng/etal:2000} show results indicating
this intermediate-level model does a reasonable job simulating
tropical climatology and ENSO variability.  


Below is a summary of the main model equations \cite{Neelin/Zeng:2000}:
\begin{equation}
   \partial_t \mathbf{v}_1 
      + \D_{V1} (\mathbf{v}_0 , \mathbf{v}_1)
      + f \mathbf{k} \times \mathbf{v}_1
      =
   - \kappa \nabla T_1 
      - \epsilon_1 \mathbf{v}_1 
      - \epsilon_{01} \mathbf{v}_0
   \label{eqn:barocl_wind}
\end{equation}
\begin{equation}
   \partial_t \zeta_0 
      + \mathrm{curl}_z (\D_{V0} (\mathbf{v}_0 , \mathbf{v}_1))
      + \beta v_0
      =
   - \mathrm{curl}_z (\epsilon_0 \mathbf{v}_0)
      - \mathrm{curl}_z (\epsilon_{10} \mathbf{v}_1)
   \label{eqn:barotr_wind}
\end{equation}
\begin{equation}
   \aonehat (\partial_t + \D_{T1}) T_1 
      + M_{S1} \nabla \cdot {\bf v}_1 
      =
   \langle Q_c \rangle
      + (g/p_T) (-R^\uparrow_t -R^\downarrow_s + R^\uparrow_s + S_t - S_s + H)
   \label{eqn:temperature}
\end{equation}
\begin{equation}
   \bonehat (\partial_t + \D_{q1}) q_1 
      - M_{q1} \nabla \cdot {\bf v}_1 
      =
   \langle Q_q \rangle
      + (g/p_T) E
   \label{eqn:moisture}
\end{equation}
where (\ref{eqn:barocl_wind}) describes the baroclinic wind component,
      (\ref{eqn:barotr_wind}) describes the barotropic wind component,
      (\ref{eqn:temperature}) is the temperature equation, and
      (\ref{eqn:moisture}) is the moisture equation.

In the simplest formulation, the vertically integrated
convective heating and moisture sink
are assumed to be equal and opposite, so:
\begin{equation}
  -\langle Q_q \rangle = \langle Q_c \rangle 
                              = \epsilon^\ast_c (q_1 - T_1)
\end{equation}

For its convective parameterization for $Q_c$, this model uses the
Betts-Miller \cite{Betts/Miller:1986} moist convective
adjustment scheme, a scheme that is also used in some GCMs.
In the convective parameterization, the coefficient
$\epsilon^\ast_c$ is defined as:
\begin{equation}
   \epsilon^\ast_c 
      \equiv 
   \aonehat \bonehat (\aonehat + \bonehat)^{-1} \tau_c^{-1} 
      \mathcal{H}( \mathit{C}_{\mathrm{1}} )
\end{equation}
where $\mathcal{H}( \mathit{C}_{\mathrm{1}} )$ is zero for
$C_{1} \leq 0$, and one for $C_{1} > 0$, and $C_{1}$
is a measure of the convective available potential energy (CAPE),
projected onto the model's temperature and moisture basis functions.

Sensible heat ($H$) and evaporation ($E$) are given as
bulk-aerodynamic formulations:
\begin{equation}
   H
      =
   \rho_a C_D \mathrm{V}_s (T_s - (T_{rs} + a_{1s} T_1))
\end{equation}
\begin{equation}
   E
      =
   \rho_a C_D \mathrm{V}_s (q_\mathit{sat} (T_s) 
      - (q_{rs} + b_{1s} q_1))
\end{equation}

Longwave radiation components are denoted by $R$, and net shortwave
radiation is denoted by $S$.
The terms $\D_{V1}$ and $\D_{V0}$ are the advection-diffusion operators
for the momentum equations (projected onto $V_0$ and $V_1 (p)$,
respectively).
The terms $\D_{T1}$ and $\D_{q1}$ are the
advection-diffusion operators for the temperature and moisture
equations, respectively, using a vertical average projection.
The $\langle X \rangle$ and $\widehat{X}$ operators are
equivalent and denote vertically integration over the troposphere.
Please see Neelin \& Zeng \cite{Neelin/Zeng:2000} and 
Zeng et al.\ \cite{Zeng/etal:2000}
for a more complete description of equations and coefficients.







% ====== end file ======


\chapter{Installation and Configuration}    \label{ch:install}
	\section{Summary and Conventions}      \label{sec:install.sum}
	% ==========================================================================
% Installation Summary
%
% By Johnny Lin
% ==========================================================================


% ------ BODY -----
%

This section provides a summary of the steps needed to install
\mods{qtcm}, and a description of the naming conventions used in
this chapter.  If you have had a decent amount of experience with
Python and installing software on a Unix system, this section will
probably be all you need to read.  The installation steps are:

\begin{enumerate}
\item Install a Fortran compiler (see Section~\ref{sec:fort.compilers}
	for a list of compilers known to work).
	This compiler should be in a directory
	listed in your system path (e.g., \fn{/usr/bin}, etc.).

\item Install all required packages
	(see Section~\ref{sec:py.etc.pkgs} for details):
	Python,
	\mods{matplotlib} (plus the \mods{basemap} toolkit),
	NumPy (which includes \mods{f2py}),
	Scientific Python,
	\LaTeX,
	and
	netCDF.

	Python packages are required to be installed on your
	system in a directory listed in your \vars{sys.path},
	and the other packages/libraries are required to be in 
	standard directories listed in your system path 
	(e.g., \fn{/usr/bin}, \fn{/sw/include}, etc.).

	Make sure the executable for Python can be called at the
	Unix command line by typing both \cmd{python}.
	You might need to define a Unix alias
	that maps \cmd{python2.4} (or whichever version of Python
	you are using) to \cmd{python}.

\item \latexhtml{Download\footnote{http://www.johnny-lin.com/py\_pkgs/qtcm/}}%
        {\htmladdnormallink{Download}{http://www.johnny-lin.com/py_pkgs/qtcm/}}
	the \mods{qtcm} tarball and extract the distribution
	into a temporary directory for building purposes.
	\fn{qtcm-0.1.2}is the name of
	the \mods{qtcm} distribution directory;
	the number following the hyphen is the
	version number of the distribution.  \label{list:download.qtcm.sum}

	In this manual, the path to \fn{qtcm-0.1.2}will
	be called the ``\mods{qtcm} build path'' and be given as
	\fn{/buildpath}.  When you see \fn{/buildpath}, please substitute
	the actual temporary directory you created for building purposes.

\item The \mods{qtcm} distribution directory 
	\fn{qtcm-0.1.2}contains the following 
	principal sub-directories:
	\fn{doc}, \fn{lib}, \fn{src}, \fn{test}.
	Documentation is in \fn{doc},
	all the package modules are in \fn{lib},
	building of extension modules will take place in \fn{src},
	and testing of the package is done in \fn{test}.

\item Compile \mods{qtcm} extension modules in \fn{src}:
	Go to \fn{src}, copy the makefile from
	\fn{src/Makefiles} corresponding to your
	system into \fn{src}, rename to \fn{makefile},
	make changes to the makefile as needed,
	and execute:
	\begin{codeblock}
	\codeblockfont{%
	make clean \\
	make \_qtcm\_full\_365.so \\
	make \_qtcm\_parts\_365.so}
	\end{codeblock}
	If you executed the make commands in \fn{src,},
	the extension modules will be automatically placed in
	\fn{lib} in the \fn{qtcm-0.1.2}directory.
	See Section~\ref{sec:create.so} for details.
	\label{list:compile.so.sum}

\item Copy the entire contents of \fn{lib} in
	\fn{qtcm-0.1.2}(not \fn{lib} itself) 
	to a directory named
	\fn{qtcm} that is on your \mods{sys.path}.  For instance,
	for Mac OS X using Fink,
	many Python packages are located in a directory
	named \fn{/sw/\-lib/\-python2.4/\-site-packages}, or something
	similar, and this directory is on the system \mods{sys.path}.  
	If this is the case for your system, copy the
	contents of \fn{lib} into
	\fn{/sw/lib/\-python2.4/\-site-packages/\-qtcm}.
	(For Unix systems, the equivalent directory is usually
	\fn{/usr/\-local/\-lib/\-python2.4/\-site-packages}.)

\item Test the \mods{qtcm} distribution in \fn{test}:
	This step is optional and can take a while.
	Testing requires you to first generate a suite of benchmarks
	using the pure-Fortran QTCM1 model, then running the tests of
	\mods{qtcm} by typing:
	\begin{codeblock}
	\codeblockfont{%
python test\_all.py}
	\end{codeblock}
	at the Unix command line while in \fn{test}.
	See Section~\ref{sec:test.qtcm} for details.

\end{enumerate}

At some point, I will automate the installation using Python's
\htmladdnormallinkfoot{\mods{distutils}}{http://docs.python.org/dist/dist.html}
utilities.



% ===== end of file =====

	\section{Fortran Compiler}             \label{sec:fort.compilers}
	% ==========================================================================
% Fortran compilers
%
% By Johnny Lin
% ==========================================================================


% ------ BODY -----
%

You must have a Fortran compiler installed on your system in order
to compile \mods{qtcm}.  The compiler must be able to interface with
a pre-processor, as QTCM1 makes copious use of pre-processor directives.
\mods{qtcm} is known to work with the following Fortran compilers on the
following platforms:

\begin{center}
\begin{tabular}{l|l|l}
\textbf{Compiler}  & \textbf{Compiler Web Site} & \textbf{Platform(s)} \\ 
\hline
\mods{g95} & \htmladdnormallink{http://www.g95.org/}{http://www.g95.org/}  
	& Mac OS X \\
\end{tabular}
\end{center}

It will probably work with other platforms, but I haven't been able
to test platforms besides those listed above.  Note that \mods{g95}
is not \htmladdnormallink{GNU Fortran}{http://gcc.gnu.org/fortran/}
(\mods{gfortran}), the Fortran 95 compiler included with the more
recent versions of GCC.




% ===== end of file =====

	\section{Required Packages}            \label{sec:py.etc.pkgs}
	% ==========================================================================
% Python packages
%
% By Johnny Lin
% ==========================================================================


% ------ BODY -----
%

The following Python packages are required to be installed on your
system in a directory listed in your \vars{sys.path}:
\begin{itemize}
\item \htmladdnormallinkfoot{Python}%
	{http://www.python.org/}:  The Python programming language
	and interpreter.  Make sure you have a version recent enough
	to be compatible with all the needed Python packages.
\item \htmladdnormallinkfoot{\mods{matplotlib}}%
	{http://matplotlib.sourceforge.net/}:  Scientific plotting
	package, using Matlab-like syntax.  The \mods{basemap} toolkit
	for \mods{matplotlib} must also be installed.
\item \htmladdnormallinkfoot{NumPy}%
	{http://numpy.scipy.org/}:  The standard array package for
	Python.  The module name of NumPy imported in a Python 
	session is \mods{numpy}.
\item \htmladdnormallinkfoot{Scientific Python}%
	{http://dirac.cnrs-orleans.fr/plone/software/scientificpython/}:
	Has netCDF file operators, in addition to other routines
	of use in scientific computing.  The module name of
	Scientific Python imported in a Python session is
	\mods{Scientific}.
\end{itemize}

One other required Python package, \mods{f2py}, is now a part of the
NumPy package, and so installation of NumPy is sufficient to give
you both.

The package \htmladdnormallinkfoot{SciPy}{http://www.scipy.org},
which includes several Python-accessible scientific libraries, also
includes NumPy (and thus \mods{f2py}), so if you install SciPy,
you don't have to install NumPy again.  Note that SciPy is not the
same as Scientific Python; the names are confusing.

A few non-Python packages are also required:
\begin{itemize}
\item \LaTeX: A scientific typesetting program used by the 
	\class{Qtcm} instance method \mods{plotm} to handle 
	exponents and subscripts.  The most common Unix 
	distribution of \LaTeX\ is
	\htmladdnormallinkfoot{teTeX}{http://www.tug.org/teTeX}.

\item netCDF:  This set of libraries enables one to write datasets into
	a platform independent, binary format, with metdata attached.
	The \htmladdnormallinkfoot{netCDF 3.6.2 library}%
        	{http://www.unidata.ucar.edu/software/netcdf/}
	source code can be
\latexhtml{downloaded from UCAR\footnote{http://www.unidata.ucar.edu/downloads/netcdf/netcdf-3\_6\_2/}}%
        {\htmladdnormallink{downloaded from UCAR}{http://www.unidata.ucar.edu/downloads/netcdf/netcdf-3_6_2/}}.
\end{itemize}

For most Unix installations, the easiest way to install all the
above is via a package manager, for instance \mods{apt-get} in
Debian GNU/Linux, \mods{aptitude} or \mods{synaptic} in Ubuntu
GNU/Linux, and \mods{fink} in Mac OS X.  Of course, you can also
download a package's source code and build direct and/or install
using Python's
\htmladdnormallinkfoot{\mods{distutils}}{http://docs.python.org/dist/dist.html}
utilities.




% ===== end of file =====

	\section{Compiling Extension Modules}  \label{sec:create.so}
	% ==========================================================================
% Compiling extension modules
%
% By Johnny Lin
% ==========================================================================


% ------ BODY -----
%

The extension modules (\fn{.so} files) are imported and used by
\mods{qtcm} objects, and contain the Fortran QTCM1 model that is
called by the \mods{qtcm} Python wrappers.  These extension modules
are located in the \fn{lib} directory of the \mods{qtcm} distribution,
and, in general, need to be created only when the \mods{qtcm} package
is installed.

Two extension modules are created:  \fn{\_qtcm\_full\_365.so} and
\fn{\_qtcm\_parts\_365.so}.  Both modules define QTCM1 models where:

\begin{itemize}
\item A year is 365 days long 
	(makefile macro \vars{YEAR360} is off).
\item Model output is written to netCDF files
	(makefile macro \vars{NETCDFOUT} is on).
\item The atmospheric boundary layer model is used
	(makefile macro \vars{NO\_ABL} is off).
\item A global domain is used
	(makefile macro \vars{SPONGES} is off).
\item Topography effects due to induced divergence are not included
	(makefile macro \vars{TOPO} is off).
\item Coupling between atmosphere and ocean is through mean fluxes
	(makefile macro \vars{CPLMEAN} is off).
\item The mixed layer ocean model is not used
	(makefile macros \vars{MXL\_OCEAN} and \vars{BLEND\_SST} are both off).
\end{itemize}

(All other makefile macros not listed are also turned off.)
The only difference between these two extension modules is that the
``full'' module is used by \class{Qtcm} instances where
\vars{compiled\_form} is set to \vars{'full'}, and the ``parts''
module is used by \class{Qtcm} instances where \vars{compiled\_form}
is set to \vars{'parts'}.  See Section~\ref{sec:compiledform} for
details about the \vars{compiled\_form} attribute.

The extension modules are created through the following steps:
\begin{enumerate}
\item Go to the \mods{qtcm} distribution directory
	\fn{qtcm-0.1.2}located in
	your build path \fn{/buildpath}.  Go to the \fn{src}
	sub-directory.  This is where all the building of the
	extension modules will take place.

\item Copy the makefile that corresponds to your platform to
	the \fn{src} directory, and rename it \fn{makefile}.
	The \fn{Makefiles} sub-directory of \fn{src} contains
	makefiles for various platforms.

\item In \fn{makefile}, make the following changes:
	\begin{enumerate}
	\item Change the \vars{FC} environment variable as needed, 
		if your Fortran compiler is different.
	\item Change the \vars{FFLAGSM} environment variable, if the
		compiler flags listed are not supported by your
		compiler.
	\item Change the \vars{-I} and \vars{-L} parts of the
		\vars{NCINC} and \vars{NCLIB} environment
		variables so that the paths for the netCDF library and
		include files match your system's installation:
		\begin{codeblock}
		\codeblockfont{%
NCINC=-I/yourpath/netcdf/include \\
NCLIB=-L/yourpath/netcdf/lib -lnetcdf}
		\end{codeblock}
		Set \dumarg{yourpath} to the full path to the
		\fn{netcdf} directory where the \fn{include} and
		\fn{lib} sub-directories are that hold the netCDF
		libraries and include files.
		(You shouldn't have to change the \vars{-l} part of
		\vars{NCLIB}, since it is standard to name the netCDF
		library \fn{libnetcdf.a}.  But if you have a non-standard
		installation, change the \vars{-l} part too.)
	\end{enumerate}

\item At the Unix prompt, type:
\begin{codeblock}
\codeblockfont{%
\small
make clean \&\& make \_qtcm\_full\_365.so \&\& make \_qtcm\_parts\_365.so}
\end{codeblock}
	to clean up leftover files from previous compilations, and to
	compile the extension module shared object files
	\fn{\_qtcm\_full\_365.so} and \fn{\_qtcm\_parts\_365.so}.
\end{enumerate}

The makefile will automatically move the shared object files into
\fn{../lib}, overwriting any pre-existing files of the same name.
A detailed description of the makefile and using \mods{f2py} is
given in Section~\ref{sec:create.new.so}, if you wish to create a
different extension module.




% ===== end of file =====

	\section{Testing the Installation}     \label{sec:test.qtcm}
	% ==========================================================================
% Installation Summary
%
% By Johnny Lin
% ==========================================================================


% ------ BODY -----
%

The \mods{qtcm} distribution comes with a set of tests for the
package, using Python's \mods{unittest} package.  
Just to warn you, the tests take around an hour to run.
The tests will not work if the contents of \fn{lib}
after you've finished building \mods{qtcm} have not been copied
to a directory named \fn{qtcm} that is on your \mods{sys.path} path,
so make sure you've gone through all the install steps
(summarized in Section~\ref{sec:install.sum}) before you do these
tests.

\emphpara{NB:}  For these tests to work, both \cmd{python} and
\cmd{python2.4} must refer to the executable for the Python
installation on your system that you are using for running \mods{qtcm}.

The tests require a set of benchmark output files in the
\fn{test/benchmarks} directory in the
\fn{qtcm-0.1.2}directory (the output will be in
directories whose names begin with ``aquaplanet'' or ``landon'').
These output files are not included with the \mods{qtcm} distribution,
and must be created, by doing the following:

\begin{enumerate}
\item Goto the directory \fn{test/benchmarks/create/src} in the
	\fn{qtcm-0.1.2}\mods{qtcm} distribution directory,
	and copy the makefile from sub-directory \fn{Makesfiles},
	that corresponds to your system to the
	\fn{test/benchmarks/create/src} directory.  Rename the makefile 
	in \fn{test/benchmarks/create/src} to \fn{makefile}.

\item In \fn{makefile}, make the following changes:
        \begin{enumerate}
        \item Change the \vars{FC} environment variable as needed,
                if your Fortran compiler is different.
        \item Change the \vars{FFLAGSM} environment variable, if the
                compiler flags listed are not supported by your
                compiler.
        \item Change the \vars{-I} and \vars{-L} parts of the
                \vars{NCINC} and \vars{NCLIB} environment
                variables so that the paths for the netCDF library and
                include files match your system's installation:
                \begin{codeblock}
                \codeblockfont{%
NCINC=-I/yourpath/netcdf/include \\
NCLIB=-L/yourpath/netcdf/lib -lnetcdf}
                \end{codeblock}
                Set \dumarg{yourpath} to the full path to the
                \fn{netcdf} directory where the \fn{include} and
                \fn{lib} sub-directories are that hold the netCDF
                libraries and include files.
                (You shouldn't have to change the \vars{-l} part of
                \vars{NCLIB}, since it is standard to name the netCDF
                library \fn{libnetcdf.a}.  But if you have a non-standard
                installation, change the \vars{-l} part too.)
        \end{enumerate}

\item Go to the directory \fn{test/benchmarks/create} in the
	\fn{qtcm-0.1.2}\mods{qtcm} distribution directory.

\item Type \cmd{python create\_benchmarks.py} at the Unix command line
	to run the benchmark creation script.
\end{enumerate}

The created benchmarks will be located in 
\fn{test/benchmarks}, in directories with names related to the
run that was done, as described earlier.
The benchmarks are created using the
pure-Fortran QTCM1 model code,
version 2.3 (August 2002), with an altered makefile
(described above) and the following code change:
In all \fn{.F90} files, occurrences of:
        \begin{codeblock}
        \codeblockfont{%
        Character(len=130)}
        \end{codeblock}
        are changed to:
        \begin{codeblock}
        \codeblockfont{%
        Character(len=305)}
        \end{codeblock}
This enables the model to properly deal with longer filenames.
The number ``305'' is chosen to make search and replace easier.

Once the benchmarks are created, you can test the \mods{qtcm} package
by doing the following:
\begin{enumerate}
\item Go to the \fn{test} directory in the 
	\fn{qtcm-0.1.2}directory.
\item Type \cmd{python test\_all.py} at the Unix command line.
\end{enumerate}

If at the end of the test runs you see this message (or something similar):
\begin{codeblock}
\codeblockfont{%
\footnotesize
---------------------------------------------------------------------- \\
Ran 93 tests in 1244.205s \\
 \\
OK}
\end{codeblock}
then everything worked fine!  If you get any other message, the test(s) have
failed.



% ===== end of file =====

	\section{Model Performance}
	%=====================================================================
% Model Performance
%=====================================================================


% ----- BEGIN TEXT -----
%
%---------------------------------------------------------------------

The wall-clock time values below give the mean over three
separate 365 day aquaplanet runs,
using climatological sea surface temperature for lower boundary forcing.
NetCDF output is written daily, for both instantaneous and mean values.
The time step is 1200~sec, and the version of \mods{qtcm} used
is 0.1.1.
The horizontal grid spacing of all model versions is
$5.625^{\circ}$ longitude by $3.75^{\circ}$ latitude.
Values are in seconds:
\begin{center}
\begin{tabular}{p{0.5\linewidth}|c|c|c}
\textbf{System} & \textbf{Pure} & \textbf{Full} & \textbf{Parts} \\
\hline
Mac OS X:  MacBook 1.83 GHz Intel Core Duo running Mac OS X
	10.4.10 with 1 GB RAM
	(Python 2.4.3, NumPy 1.0.3, \mods{f2py} 2\_3816).
    & 152.59 & 153.63 & 158.94 \\
\hline
Ubuntu GNU/Linux:  Dell PowerEdge 860 with 2.66 GHz Quad Core Intel
	Xeon processors (64 bit) running Ubuntu 8.04.1 LTS
	(Python 2.5.2, NumPy 1.1.0, \mods{f2py} 2\_5237).
    & 43.73 & 44.79 & 47.45
\end{tabular}
\end{center}

``Pure'' refers to the pure-Fortran version of QTCM1.
``Full'' refers to a \mods{qtcm} run session with \vars{compiled\_form}
set to \vars{'full'}.  ``Parts'' refers to a \mods{qtcm} run session
with \vars{compiled\_form} set to \vars{'parts'}.
(Section~\ref{sec:compiledform} has details about the difference
between compiled forms.)

The \vars{'parts'} version of \mods{qtcm} gives Python the maximum
flexibility in accessing compiled QTCM1 model subroutines and
variables.  The price of that flexibility is an increase in
run time of approximately 4--9\% over the pure-Fortran version.
The difference in performance between the
\vars{'full'} version of \mods{qtcm} and the pure-Fortran version of
QTCM1 is between negligible and 3\% longer.

To make a timing for the pure-Fortran model, go to
\fn{test/benchmarks/timing/work} in \fn{/buildpath} and run the
\fn{timing\_365.sh} script in that directory.  That script runs the
QTCM1 model using \cmd{/usr/bin/time}, which at the end of the
script will output the amount of time it took to make the model
run.  Run the timing script three times and average the values to
obtain a time comparable to the above.

To make a timing for the \mods{qtcm} model, type \cmd{python
timing\_365.py} while in the \fn{test} directory in \fn{/buildpath}.
Three run sessions will be made for \vars{compiled\_form} equal to
\vars{'full'} and \vars{'parts'}, the times are averaged, and the
value are output at the end of the script.




% ====== end file ======

	\section{Installing in Mac OS X}       \label{sec:install.macosx}
	% ==========================================================================
% Description of installing in Mac OS X
%
% By Johnny Lin
% ==========================================================================


% ------ BODY -----
%
%------------------------------------------------------------------------
\subsection{Introduction}

This section describes issues and a summary of the installation steps
I followed to install \mods{qtcm} on a Mac running OS X.
It is a specific realization of the general installation
instructions found in Sections~\ref{sec:install.sum}--\ref{sec:test.qtcm}.
I first worked through these installation steps during June--July 2007,
with updates during July 2008.
The best way to go through this section is to go through
the summary of the installation steps in 
Section~\ref{sec:osx.install.summary},
and looking back to other sections as needed.




%------------------------------------------------------------------------
\subsection{Platform and Unix Dependencies}

This work was done on a MacBook 1.83 GHz Intel Core Duo running Mac OS X
10.4.11.  My machine has 1 GB RAM and 64 GB of disk in its main partition.

I recommend you turn-off your antivirus software before you
do the installs.  
Problems have been
\latexhtml{reported by Fink users\footnote%
		{http://finkproject.org/faq/usage-fink.php?phpLang=en\#kernel-panics}}%
	{\htmladdnormallink{reported by Fink users}%
		{http://finkproject.org/faq/usage-fink.php?phpLang=en#kernel-panics}}
using the Fink package manager with antivirus software enabled.

There are a variety of dependencies that are required to get your Mac
up-and-running as a scientific computing platform.  The most basic is
installing Apple's 
\htmladdnormallinkfoot{XCode}{http://developer.apple.com/tools/xcode/}
developer tools.\footnote%
	{The package should work in Mac OS X 10.4 with XCode 2.4.1 and higher;
	I've tried it with both 2.4.1 and XCode 2.5.  Note that
	XCode 3.1 only works on Mac OS X 10.5.}
This set of tools contains compilers and libraries
needed to do anything further.  You have to be a member of Apple's
Developer Connection, but registration is free.

Besides XCode, there are a variety of Unix libraries and utilities that you
need.  I first tried installing them by myself, from scratch, into
\fn{/usr/local}, but it was hard to keep track of all the dependencies.
A few that did work, and that I installed from their disk images, are:
\htmladdnormallinkfoot{MacTeX}{http://www.tug.org/mactex/}, 
\htmladdnormallinkfoot{MAMP}{http://www.mamp.info/}, and 
\htmladdnormallinkfoot{Tcl/Tk Aqua BI (Batteries Included)}%
	{http://tcltkaqua.sourceforge.net/}.\footnote%
		{Theoretically you can use Fink to install the equivalent
		of these packages, but I like the specific collection 
		found in these packages.  For instance, Tcl/Tk Aqua BI
		runs natively on the Mac.}

For everything else, thankfully, there's the
\htmladdnormallinkfoot{Fink Project}{http://www.finkproject.org/} which
uses a package manager built upon Debian tools to install ports of
Unix programs onto a Mac.  I just 
\htmladdnormallinkfoot{downloaded}%
	{http://www.finkproject.org/download/index.php?phpLang=en}
a binary version of the Fink 0.8.1 installer for Intel Macs,
installed Fink, and used its package management tools to install
(almost) everything else I needed.\footnote%
	{The one drawback of Fink is that it sometimes
	has stability problems.  In those cases, Fink provides
	command line suggestions to fix the problems, which sometimes
	will work.  If not, sometimes
\latexhtml{%
	deleting Fink and everything it installed,\footnote%
	{http://www.finkproject.org/faq/usage-fink.php?phpLang=en\#removing}}{%
\htmladdnormallink{deleting Fink and everything it installed}
	{http://www.finkproject.org/faq/usage-fink.php?phpLang=en#removing},}
	and starting afresh, will do the trick.
	It also appeared to me that sometimes when I installed 
	multiple packages
	via one \cmd{fink install} call, the installation did not work
	as well as when I installed only one package per call.}

Although you do not need anything besides a Fortran compiler and
the netCDF libraries to run QTCM1 in its pure-Fortran form, in order to
manipulate the model and use this Python version \mods{qtcm}, you
need to have Python installed.  The default Python that comes
with the Mac is a little old, so I used Fink to also install
Python 2.5 and related packages, including
\htmladdnormallinkfoot{matplotlib}{http://matplotlib.sourceforge.net/},
\htmladdnormallinkfoot{ScientificPython}{http://dirac.cnrs-orleans.fr/plone/software/scientificpython/},
and
\htmladdnormallinkfoot{SciPy}{http://www.scipy.org}
(see Section~\ref{sec:osx.summary} for details).




%------------------------------------------------------------------------
\subsection{Fortran Compiler}

There are a variety of high-quality, commercial Fortran compilers.
Unfortunately, because I do not have a research budget, I am not able
to use those compilers.  The 
\htmladdnormallinkfoot{GNU Compiler Collection}{http://gcc.gnu.org/}
(GCC) provides a suite of open-source compilers, some of which are the
standards of their language.  Most of the GCC compilers are installed
on your Mac when you install XCode.

\htmladdnormallinkfoot{GNU Fortran}{http://gcc.gnu.org/fortran/}
(\mods{gfortran}), is the Fortran 95 compiler included with the more
recent versions of GCC.
Unfortunately, I was not able to get it to compile QTCM1.
There is a second open-source Fortran compiler,
\htmladdnormallinkfoot{G95}{http://www.g95.org/} (\mods{g95}),
which some feel is farther along in its development than \mods{gfortran}.
I was able to successfully compile QTCM1 with \mods{g95} on my Mac.
I used Fink to install G95
(see Section~\ref{sec:osx.summary} for details).




%------------------------------------------------------------------------
\subsection{NetCDF Libraries}   \label{sec:netcdf}

For some reason, the netCDF libraries and include files
installed by Fink didn't correspond to the files needed
by the calling routines in \mods{qtcm}.  To solve this, I compiled
my own set of 
\htmladdnormallinkfoot{netCDF 3.6.2 libraries}%
	{http://www.unidata.ucar.edu/software/netcdf/}
using the tarball 
\latexhtml{downloaded from UCAR\footnote{http://www.unidata.ucar.edu/downloads/netcdf/netcdf-3\_6\_2/}}%
        {\htmladdnormallink{downloaded from UCAR}{http://www.unidata.ucar.edu/downloads/netcdf/netcdf-3_6_2/}}.

Once I uncompressed and untarred the package, and went into 
the top-level directory of the package, I built the package by typing
the following at the Unix prompt:

\begin{codeblock}
\codeblockfont{%
./configure --prefix=/Users/jlin/extra/netcdf \\
make check \\
make install}
\end{codeblock}

This installed the netCDF binaries, libraries, and include files into
sub-directories \fn{bin}, \fn{lib}, and \fn{include} in 
the directory specified by \vars{--prefix}.
If you want to install the netCDF libraries in the default
(usually \fn{/usr/local}), just leave out the \vars{--prefix}
option.

Note:  When you build netCDF, make sure the build directory
is not in the directory tree of \vars{--prefix}
(or the default directory \fn{/usr/local}).




%------------------------------------------------------------------------
\subsection{Makefile Configuration}  \label{sec:osx.makefile}

	\subsubsection{NetCDF}

In the \fn{src} directory in the \mods{qtcm} distribution, there is a
sub-directory \fn{Makefiles} that contains the makefiles for a
variety of platforms.  Edit the file \fn{makefile.osx\_g95}
so that the lines specifying the environment variables for the
netCDF libraries and include files:

\begin{codeblock}
\codeblockfont{%
NCINC=-I/Users/jlin/extra/netcdf/include \\
NCLIB=-L/Users/jlin/extra/netcdf/lib -lnetcdf}
\end{codeblock}

are changed to the path where your \emph{manually compiled} 
netCDF libraries and include files are.

Copy \fn{makefile.osx\_g95} from the \fn{Makefiles} sub-directory
in \fn{src} into \fn{src}.  
In other words, from the \mods{qtcm} distribution directory
(i.e., \fn{/buildpath}), at the Unix prompt execute:

\begin{codeblock}
\codeblockfont{%
cp src/Makefiles/makefile.osx\_g95 src/makefile}
\end{codeblock}


	\subsubsection{Linking Order}

Compilers in the GNU Compiler Collection (GCC) search libraries
and object files in the order they are listed in the command-line, 
\latexhtml{from left-to-right\footnote%
        {http://gcc.gnu.org/onlinedocs/gcc-4.1.2/gcc/Link-Options.html\#index-l-670}}%
        {\htmladdnormallink{from left-to-right}{http://gcc.gnu.org/onlinedocs/gcc-4.1.2/gcc/Link-Options.html#index-l-670}}.
Thus, if routines in \fn{b.o} call routines in \fn{a.o}, 
you must list the files in the order \fn{a.o b.o}.

For some reason, that isn't the case for \mods{g95}.  Thus, you will
find \mods{g95} makefile rules structured like the following
(below is part of the rule to create an executable (\fn{qtcm}) for
benchmark runs):

% --- Two versions of this rule, one for display in PDF and the other
%     for display in HTML:
%
\begin{latexonly}
\begin{codeblock}
\codeblockfont{%
qtcm: main.o \\
\hspace*{8ex}\$(FC)~-O~\$(NCINC)~-o~\$@ main.o~\$(QTCMLIB)~\$(NCLIB)}
\end{codeblock}
\end{latexonly}

\begin{htmlonly}
\begin{rawhtml}
<p><code><font color="blue">qtcm: main.o<br>
&nbsp;&nbsp;&nbsp;&nbsp;&nbsp;&nbsp;&nbsp;$(FC) -O $(NCINC) -o 
$@ main.o $(QTCMLIB) $(NCLIB)</font></code></p>
\end{rawhtml}
\end{htmlonly}

even though \fn{main.o} depends on the QTCM library 
(specified in macro setting \vars{\$(QTCMLIB)}), which in turn
depends on the netCDF library (specified in macro setting \vars{\$(NCLIB)}).


%------------------------------------------------------------------------
\subsection{Summary of Steps}   \label{sec:osx.install.summary}

The following summarizes all the steps I took to install
\mods{qtcm} in Mac OS X:

\begin{enumerate}
\item Install
	\htmladdnormallinkfoot{XCode 2.5}%
		{http://developer.apple.com/tools/xcode/}.

\item Install 
	\htmladdnormallinkfoot{MacTeX}{http://www.tug.org/mactex/}, 
	\htmladdnormallinkfoot{MAMP}{http://www.mamp.info/}, and 
	\htmladdnormallinkfoot{TCL/Tk Aqua BI (Batteries Included)}%
		{http://tcltkaqua.sourceforge.net/}.

\item Install
	\htmladdnormallinkfoot{Fink 0.8.1}%
		{http://www.finkproject.org/download/index.php?phpLang=en}.
	Make sure you
	\htmladdnormallink{set up your environment}%
		{http://www.finkproject.org/doc/users-guide/install.php\#setup}
	to enable you to use the packages you install with Fink
	(e.g. \vars{PATH} settings, etc.).
	Most of the time, that just means adding the line
	\cmd{source /sw/bin/init.csh} to your \fn{.cshrc} file (or the
	equivalent in your \fn{.bashrc}).

	Note that for many of the packages needed to run \mods{qtcm},
	you need to 
	\htmladdnormallink{configure Fink to download packages 
		from the unstable trees}%
	{http://www.finkproject.org/faq/usage-fink.php?phpLang=en\#unstable}.
	To do that, add \vars{unstable/main} and \vars{unstable/crypto}
	to the \vars{Trees:} line in \fn{/sw/etc/fink.conf}, and run:

	\begin{codeblock}
	\codeblockfont{fink selfupdate} \\
	\codeblockfont{fink index} \\
	\codeblockfont{fink scanpackages} \\
	\codeblockfont{fink update-all}
	\end{codeblock}

	When \cmd{selfupdate} runs, choose \cmd{rsync} for the
	self update method.  If you do not, Fink will not look in the
	unstable trees for packages.

\item Use Fink to install the \mods{g95} Fortran compiler.
	From a Unix prompt, type:

	\begin{codeblock}
	\codeblockfont{fink -$\,\!$-use-binary-dist install g95}
	\end{codeblock}

\item Use Fink to install Python 
	and the NumPy package (which \mods{f2py} is a part of).
	From a Unix prompt, type:

	\begin{codeblock}
	\codeblockfont{%
	fink -$\,\!$-use-binary-dist install python25 \\
	fink -$\,\!$-use-binary-dist install scipy-core-py25}
	\end{codeblock}

	(Numpy used to be called SciPy Core.)  If you want to
	install Python 2.4 instead, just change the ``25'' and ``py25'' above
	(and in later occurrences) to ``24'' and ``py24'', respectively.
	Note that Fink does not have a version of epydoc for Python 2.4,
	so if you wish to create documentation using epydoc, you will
	need to install Python 2.5.

\item Install teTeX and \LaTeX{2HTML} using Fink.
	From a Unix prompt, type:

	\begin{codeblock}
	\codeblockfont{fink -$\,\!$-use-binary-dist install tetex} \\
	\codeblockfont{fink -$\,\!$-use-binary-dist install latex2html}
	\end{codeblock}

	When prompted, choose ghostscript and ghostscript-fonts to
	satistfy the dependency (which should be the default options).
	I tried choosing system-ghostscript8, but Fink looks for
	ghostscript 8.51 and didn't recognize ghostscript 8.57 that
	was already installed in \fn{/usr/local} (via my MacTeX
	install).  \LaTeX{2HTML} has a package required by the
	\mods{qtcm} manual \LaTeX\ file.

\item Install additional programming and
	scientific packages and libraries using Fink.
	From a Unix prompt, type:

	\begin{codeblock}
	\codeblockfont{%
	fink -$\,\!$-use-binary-dist install scientificpython-py25 \\
	fink -$\,\!$-use-binary-dist install matplotlib-py25 \\
	fink -$\,\!$-use-binary-dist install matplotlib-basemap-py25 \\
	fink -$\,\!$-use-binary-dist install matplotlib-basemap-data-py25 \\
	fink -$\,\!$-use-binary-dist install xaw3d \\
	fink -$\,\!$-use-binary-dist install fftw fftw3 \\
	fink -$\,\!$-use-binary-dist install epydoc-py25 \\
	fink -$\,\!$-use-binary-dist install graphviz \\
	fink -$\,\!$-use-binary-dist install scipy-py25}
	\end{codeblock}

\item Manually install netCDF 3.6.2
	(see Section \ref{sec:netcdf}).

\item From this point on, you can follow the
	general instructions given in Section~\ref{sec:install.sum},
	starting with step~\ref{list:download.qtcm.sum}.
	Please do not ignore, however, Section~\ref{sec:install.macosx}'s
	Mac-specific details.

\end{enumerate}



% ===== end of file =====

	\section{Installing in Ubuntu}         \label{sec:install.ubuntu}
	% ==========================================================================
% Description of installing in Ubuntu
%
% By Johnny Lin
% ==========================================================================


% ------ BODY -----
%
%------------------------------------------------------------------------
\subsection{Introduction}

This section describes installation issues 
I followed to install \mods{qtcm} on my
Dell PowerEdge 860 running Ubuntu GNU/Linux 8.04.1 LTS (Hardy).
The machine has 2.66 GHz Quad Core Intel Xeon processors (64 bit),
4 GB RAM, and 677 GB of disk in its main partition.
This section is a specific realization of the general installation
instructions found in Sections~\ref{sec:install.sum}--\ref{sec:test.qtcm}.
I worked through these installation steps during July 2008.
The best way to go through this section is to go through
the summary of the installation steps in 
Section~\ref{sec:ubuntu.install.summary},
and looking back to other sections as needed.



%------------------------------------------------------------------------
\subsection{Fortran Compiler}     \label{sec:ubuntu.fort.install}

The easiest Fortran compiler to install in Ubuntu 8.04.1 is
\htmladdnormallinkfoot{GNU Fortran}{http://gcc.gnu.org/fortran/}
(\mods{gfortran}), the Fortran 95 compiler included with the more
recent versions of the GNU Compiler Collection (GCC); you can
use any package manager (e.g., \mods{apt-get}, \mods{aptitude})
to install it.
Unfortunately, I was not able to get it to compile QTCM1.
I was, however, able to successfully compile QTCM1 using
the second open-source Fortran compiler,
\htmladdnormallinkfoot{G95}{http://www.g95.org/} (\mods{g95}),
which some feel is farther along in its development than \mods{gfortran}.
G95, however, is not supported as an Ubuntu package, and so I had
to manually install it.

I downloaded the binary version of G95 v0.91 
(the Linux x86\_64/EMT64 with 32 bit default integers) 
using the following
\cmd{curl} command:\footnote%
	{I use \mods{curl} because I usually access my
	Ubuntu server via a terminal session.}

\begin{codeblock}
\codeblockfont{%
\small
curl -o g95.tgz http://ftp.g95.org/v0.91/g95-x86\_64-32-linux.tgz}
\end{codeblock}

which saves the \fn{.tgz} file as the local file \fn{g95.tgz}.
After that, I followed the G95 project's standard
\latexhtml{installation instructions\footnote%
	{http://g95.sourceforge.net/docs.html\#starting}}%
	{\htmladdnormallink{installation instructions}%
		{http://g95.sourceforge.net/docs.html#starting}}
to finish the install.\footnote%
	{The G95 installation instructions say you can put
	\fn{g95-install} anywhere, and make a link to the
	executable \mods{g95} in
	\fn{$\sim$/bin}.  I put \fn{g95-install} in
	\fn{/usr/local}, and while in \fn{/usr/local/bin}, 
	I put a link to the G95 executable using the command:
	\begin{codeblock}
	\codeblockfont{%
	sudo ln -s ../g95-install\_64/bin/x86\_64-suse-linux-gnu-g95 g95.}
	\end{codeblock}}
The regular Linux x86 version of G95
(in \fn{g95-x86-linux.tgz} from the G95 website) did not work on my
machine.




%------------------------------------------------------------------------
\subsection{NetCDF Libraries}   \label{sec:ubuntu.netcdf}

%Here things were very confusing for my machine, as I needed to
%install two versions of the
%\htmladdnormallinkfoot{netCDF}%
%	{http://www.unidata.ucar.edu/software/netcdf/}
%libraries and include files, one 
%for a successful compilation of the extension modules
%(as described in Section~\ref{sec:create.so}),
%and the other 
%for a successful run of the pure-Fortran QTCM1 model
%(used to create the testing benchmarks, as described in
%Section~\ref{sec:test.qtcm}).
%
%The first set of netCDF files (for the extension modules) are
%installed from Ubuntu's package management system.
%These are automatically installed when the \mods{python-netcdf}
%package is installed via an Ubuntu package manager
%(see Section~\ref{sec:ubuntu.install.summary}).
%The include files for this netCDF installation are 
%located in \fn{/usr/include}, and the libraries for this
%netCDF installation are location in \fn{/usr/lib}.

For some reason, the netCDF libraries and include files
installed from the Ubuntu packages do not
correspond to the files needed
by the calling routines in \mods{qtcm}.  To solve this, I compiled
my own set of
\htmladdnormallinkfoot{netCDF 3.6.2 libraries}%
        {http://www.unidata.ucar.edu/software/netcdf/}
using the tarball
\latexhtml{downloaded from UCAR\footnote{http://www.unidata.ucar.edu/downloads/netcdf/netcdf-3\_6\_2/}}%
        {\htmladdnormallink{downloaded from UCAR}{http://www.unidata.ucar.edu/downloads/netcdf/netcdf-3_6_2/}}.

Once I uncompressed and untarred the package, and went into
the top-level directory of the package, I built the package by typing
the following at the Unix prompt:

\begin{codeblock}
\codeblockfont{%
export FC=g95 \\
export FFLAGS="-O -fPIC" \\
export FFLAGS="-fPIC" \\
export F90FLAGS="-fPIC" \\
export CFLAGS="-fPIC" \\
export CXXFLAGS="-fPIC" \\
./configure \\
make check \\
sudo make install}
\end{codeblock}

(The \cmd{export} commands set environment variables for the
Fortran compiler and Fortran and other compiler flags.  The
\vars{-fPIC} flag enables the compilers to create
position independent code, needed for shared libraries in
Ubuntu on a 64 bit Intel processor.)

The above installs the netCDF binaries, libraries, and include files into
sub-directories \fn{bin}, \fn{lib}, and \fn{include} in 
\fn{/usr/local}, the default.
The include files for this netCDF installation are thus
located in \fn{/usr/local/include}, and the libraries for this
netCDF installation are location in \fn{/usr/local/lib}.
(If you want to specify a different installation
location, use the \vars{--prefix} option in \cmd{configure}.)
While you don't have to have root privileges during the configuration
and check steps, you do during the installation step if you're installing
into \fn{/usr/local} (thus the \cmd{sudo} in the last step).\footnote%
	{Note that when you build netCDF, make sure the build directory
	is not in the directory tree of \vars{--prefix}
	or the default directory \fn{/usr/local}.}

%Because there are two different netCDF installations used in the
%\mods{qtcm} package, the makefiles for creating the benchmarks
%and extensions files will have different \vars{NCLIB} and \vars{NCINC}
%environment variables (see Section~\ref{sec:ubuntu.makefile}).




%------------------------------------------------------------------------
\subsection{Makefile Configuration}  \label{sec:ubuntu.makefile}

	\subsubsection{NetCDF}

In the \fn{src} directory in the \mods{qtcm} distribution, there is a
sub-directory \fn{Makefiles} that contains the makefiles for a
variety of platforms.  Edit the file \fn{makefile.ubuntu\_64\_g95}
so that the lines specifying the environment variables for the
netCDF libraries and include files:

\begin{codeblock}
\codeblockfont{%
NCINC=-I/usr/local/include \\
NCLIB=-L/usr/local/lib -lnetcdf}
\end{codeblock}

are changed to the path where your manually compiled
netCDF libraries and include files are.

Copy \fn{makefile.ubuntu\_64\_g95} from the \fn{Makefiles} sub-directory
in \fn{src} into \fn{src}.  
In other words, from the \mods{qtcm} distribution directory
(i.e., \fn{/buildpath}), at the Unix prompt execute:

\begin{codeblock}
\codeblockfont{%
cp src/Makefiles/makefile.ubuntu\_64\_g95 src/makefile}
\end{codeblock}


	\subsubsection{Linking Order}

Compilers in the GNU Compiler Collection (GCC) search libraries
and object files in the order they are listed in the command-line,
\latexhtml{from left-to-right\footnote%
	{http://gcc.gnu.org/onlinedocs/gcc-4.1.2/gcc/Link-Options.html\#index-l-670}}%
	{\htmladdnormallink{from left-to-right}{http://gcc.gnu.org/onlinedocs/gcc-4.1.2/gcc/Link-Options.html#index-l-670}}.
Thus, if routines in \fn{b.o} call routines in \fn{a.o}, 
you must list the files in the order \fn{a.o b.o}.

For some reason, that isn't the case for \mods{g95}.  Thus, you will
find \mods{g95} makefile rules structured like the following
(below is part of the rule to create an executable (\fn{qtcm}) for
benchmark runs):

% --- Two versions of this rule, one for display in PDF and the other
%     for display in HTML:
%
\begin{latexonly}
\begin{codeblock}
\codeblockfont{%
qtcm: main.o \\
\hspace*{8ex}\$(FC)~-O~\$(NCINC)~-o~\$@ main.o~\$(QTCMLIB)~\$(NCLIB)}
\end{codeblock}
\end{latexonly}

\begin{htmlonly}
\begin{rawhtml}
<p><code><font color="blue">qtcm: main.o<br>
&nbsp;&nbsp;&nbsp;&nbsp;&nbsp;&nbsp;&nbsp;$(FC) -O $(NCINC) -o 
$@ main.o $(QTCMLIB) $(NCLIB)</font></code></p>
\end{rawhtml}
\end{htmlonly}

even though \fn{main.o} depends on the QTCM library 
(specified in macro setting \vars{QTCMLIB}), which in turn
depends on the netCDF library (specified in macro setting \vars{NCLIB}).


	\subsubsection{Shared Object PIC}   \label{sec:sopic}

In order to compile the model in Ubuntu on a 64 bit Intel processor,
the model and the netCDF library it is linked to needs to be
compiled to be 
\latexhtml{position independent code (PIC).\footnote%
		{http://www.gentoo.org/proj/en/base/amd64/howtos/index.xml?part=1\&chap=3}}%
	{\htmladdnormallink{position independent code (PIC)}%
		{http://www.gentoo.org/proj/en/base/amd64/howtos/index.xml?part=1&chap=3}.}
This is accomplished with the 
\htmladdnormallinkfoot{\cmd{-fPIC} flag}%
	{http://www.fortran-2000.com/ArnaudRecipes/sharedlib.html}.

In the \mods{qtcm} makefiles, the \cmd{-fPIC} flag is introduced in the
macro \vars{FFLAGSM}, for instance:
\begin{codeblock}
\codeblockfont{%
FFLAGSM = -O -fPIC}
\end{codeblock}
For makefiles used in creating extension modules, \cmd{-fPIC} must
be passed into the \mods{f2py} call.  To do so, put the flags:
\begin{codeblock}
\codeblockfont{%
--f90flags="-fPIC" --f77flags="-fPIC"}
\end{codeblock}
after the \vars{--fcompiler} flag in the \mods{f2py} calling line.

The \cmd{-fPIC} flag must also be used when compiling the netCDF
libraries, as described in Section~\ref{sec:ubuntu.netcdf}.
Failure to create PIC libraries in 64 bit Ubuntu can result in errors 
like the following when creating the \mods{qtcm} extension modules:
\begin{codeblock}
\codeblockfont{%
ld: /usr/local/lib/libnetcdf.a(fort-attio.o): relocation R\_X86\_64\_32 against `a local symbol' can not be used when making a shared object; recompile with -fPIC /usr/local/lib/libnetcdf.a: could not read symbols: Bad value}
\end{codeblock}




%------------------------------------------------------------------------
\subsection{Summary of Steps}      \label{sec:ubuntu.install.summary}

The following summarizes all the steps I took to install
\mods{qtcm} in
Ubuntu 8.04.1 LTS (Hardy) running on a
Quad Core Intel Xeon (64 bit) machine.
Note that while I use the \mods{aptitude} package manager, you are
free to use any manager of your choice (e.g., \mods{apt-get},
\mods{synaptic}, etc.):

\begin{enumerate}
\item Install the G95 Fortran compiler from the binary distribution.
	See Section~\ref{sec:ubuntu.fort.install} for details.

\item Use an Ubuntu package manager
	to install the following packages, by typing:
	\begin{codeblock}
	\codeblockfont{%
sudo aptitude update \\
sudo aptitude install curl \\
sudo aptitude install python-epydoc \\
sudo aptitude install python-matplotlib \\
sudo aptitude install python-netcdf \\
sudo aptitude install python-scientific \\
sudo aptitude install python-scipy \\
sudo aptitude install texlive}
	\end{codeblock}

	Installing \mods{python-scipy} will also install NumPy and
	\mods{f2py}, so you don't have to install the
	\mods{python-numpy} package separately.

	Early-on as I debugged my \mods{qtcm} install on Ubuntu,
	I encountered errors that I thought came from an 
	\htmladdnormallinkfoot{old version of NumPy}%
		{http://cens.ioc.ee/pipermail/f2py-users/2008-June/001617.html},
	and thus I replaced Ubuntu's packaged NumPy with NumPy 1.1.0
	built 
	\latexhtml{directly from source.\footnote%
			{http://sourceforge.net/project/showfiles.php?group\_id=1369\&package\_id=175103}}%
		{\htmladdnormallink{directly from source}{http://sourceforge.net/project/showfiles.php?group_id=1369&package_id=175103}.}
	(Note, you shouldn't install your new NumPy in the default
	location, which may cause problems later-on with Ubuntu's
	package manager.)
	Later on, I concluded the errors I had encountered were not
	because of the NumPy version, but by then I didn't want to
	try to reinstall NumPy again.
	So strictly speaking, the version of Numpy I used is not
	the one bundled with \mods{python-scipy}, but that shouldn't
	be a problem.

\item Manually install netCDF 3.6.2 from source
	(see Section \ref{sec:ubuntu.netcdf}).

\item Manually install the \mods{basemap} package of
	\mods{matplotlib}.  
	The source for the \mods{basemap} toolkit is
	available 
	\latexhtml{from Sourceforge\footnote%
			{http://sourceforge.net/project/showfiles.php?group\_id=80706}}%
		{\htmladdnormallink{from Sourceforge}%
			{http://sourceforge.net/project/showfiles.php?group_id=80706}}
	I obtained version 0.9.9.1 using the
	following \cmd{curl} command:
	\begin{codeblock}
	\codeblockfont{%
\scriptsize
curl -o basemap.tar.gz $\backslash$ \\
http://voxel.dl.sourceforge.net/sourceforge/matplotlib/basemap-0.9.9.1.tar.gz}
	\end{codeblock}

	The \fn{README} file in the \fn{basemap-0.9.9.1} directory has
	detailed installation instructions.  Note that you have to
	install the GEOS library first (\fn{README} has detailed
	directions on how to do that too).  To be on the safe-side,
	I would set the \vars{FC} environment variable to the G95
	compiler
	(e.g., with \cmd{export FC=g95} in Bash).

\item From this point on, you can follow the
	general instructions given in Section~\ref{sec:install.sum},
	starting with step~\ref{list:download.qtcm.sum}.
	Please do not ignore, however, Section~\ref{sec:install.ubuntu}'s
	Ubuntu-specific details.

\end{enumerate}



% ===== end of file =====


\chapter{Getting Started With \mods{qtcm}}  \label{ch:getting.started}
% ==========================================================================
% Getting Started With qtcm
%
% By Johnny Lin
% ==========================================================================


% ------ BODY -----
%
%---------------------------------------------------------------------
\section{Your First Model Run}

Figure~\ref{fig:my.first.run} shows an example of a script to make
a 30 day seasonal, aquaplanet model run, with run name ``test'',
starting from November 1, Year 1.


%--- Two versions, one for PDF, one for HTML:
\begin{latexonly}
\begin{figure}[htp]
\begin{center}
\begin{codeblock}
\codeblockfont{%
from qtcm import Qtcm \\
inputs = \{\} \\
inputs['runname'] = 'test' \\
inputs['landon'] = 0 \\
inputs['year0'] = 1 \\
inputs['month0'] = 11 \\
inputs['day0'] = 1 \\
inputs['lastday'] = 30 \\
inputs['mrestart'] = 0 \\
inputs['compiled\_form'] = 'parts' \\
model = Qtcm(**inputs) \\
model.run\_session()}
\end{codeblock}
\end{center}
\caption{An example of a simple \mods{qtcm} run.}
\label{fig:my.first.run}
\end{figure}
\end{latexonly}

\begin{htmlonly}
\label{fig:my.first.run}
\begin{center}
\htmlfigcaption{%
	\codeblockfont{%
from qtcm import Qtcm \\
inputs = \{\} \\
inputs['runname'] = 'test' \\
inputs['landon'] = 0 \\
inputs['year0'] = 1 \\
inputs['month0'] = 11 \\
inputs['day0'] = 1 \\
inputs['lastday'] = 30 \\
inputs['mrestart'] = 0 \\
inputs['compiled\_form'] = 'parts' \\
model = Qtcm(**inputs) \\
model.run\_session()}
	}

\htmlfigcaption{Figure~\ref{fig:my.first.run}:
	An example of a simple \mods{qtcm} run.}
\end{center}
\end{htmlonly}



The class describing the QTCM1 model is \class{Qtcm}.  An instance
of \class{Qtcm}, in this example \vars{model}, is created the same
way you create an instance of any class.  When instantiating an
instance of \class{Qtcm}, keyword parameters can be used to override
any default settings.  In the example above, the dictionary
\vars{inputs} specifying all keyword parameters is passed in on the
instantiation of \vars{model}.

The keyword parameter settings in
Figure~\ref{fig:my.first.run} have the following meanings:
\begin{itemize}
\item \vars{runname}:  This string (``test'') is used in the
	output filename.  QTCM1 writes mean and instantaneous
	output files to the directory given in \vars{model.outdir.value},
	with filenames 
	\fn{qm\_}\dumarg{runname}\fn{.nc} for mean output and
	\fn{qi\_}\dumarg{runname}\fn{.nc} for instantaneous output.

\item \vars{landon}: When set to ``0'', the land is turned off and
	the run is an aquaplanet run.  When set to ``1'', the land
	model is turned on.

\item \vars{year0}:  The year the run starts on.

\item \vars{month0}:  The month the run starts on (11 = November).

\item \vars{day0}: The day of the month the run starts on.

\item \vars{lastday}:  The model runs from day 1 to \vars{lastday}.

\item \vars{mrestart}:  When set to ``0'', the run starts from
	default initial conditions
	(see Section~\ref{sec:initial.variables} for a table of
	those values).
	When set to ``1'', the run starts from a restart file.

\item \vars{compiled\_form}:  This keyword sets what form the
	compiled QTCM1 model has, and its value is saved to
	the instance's \vars{compiled\_form} attribute.
	It is a string and can be set either to
	``parts'' or ``full''.  Most of the time, you will want
	to set it to \vars{'parts'}.
	This keyword is the only one
	that must be specified on instantiation; the model instance
	will at least instantiate
	using only the default settings for all the other keyword
	parameters (given in Appendix~\ref{app:defaults.values}).
	See Section~\ref{sec:compiledform} for details about
	what the \vars{compiled\_form} attribute controls.
\end{itemize}

By default, the \vars{SSTmode} attribute, which controls whether the
model will use climatological sea-surface temperatures (SST) 
or real SSTs, is set to the \vars{value} ``seasonal'', thus giving a
run with seasonal forcing at the lower-boundary over the ocean.

This example assumes that the boundary condition files, sea surface
temperature files, and the model output directories are as specified
in submodule \mods{defaults}.  Those values are described in
Section~\ref{sec:defaults.scalar}.




%---------------------------------------------------------------------
\section{Managing Directories}

Most of the time, your boundary condition files and output files
will not be in the locations specified in
Section~\ref{sec:defaults.scalar}, or in the directory your
Python script resides.  The easiest way to tell your \class{Qtcm} 
instance where your input/output files are is to pass them in
as keyword parameters on instantiation.


%--- Two versions, one for PDF, one for HTML:
\begin{latexonly}
\begin{figure}[htp]
\begin{codeblock}
\codeblockfont{%
\small
from qtcm import Qtcm \\
rundirname = 'test' \\
dirbasepath = os.path.join(os.getcwd(), rundirname) \\
inputs = \{\} \\
inputs['bnddir'] = os.path.join( os.getcwd(), 'bnddir', \\
\hspace*{40ex}'r64x42' ) \\
inputs['SSTdir'] = os.path.join( os.getcwd(), 'bnddir', \\
\hspace*{40ex}'r64x42', 'SST\_Reynolds' ) \\
inputs['outdir'] = dirbasepath \\
inputs['runname'] = rundirname \\
inputs['landon'] = 0 \\
inputs['year0'] = 1 \\
inputs['month0'] = 11 \\
inputs['day0'] = 1 \\
inputs['lastday'] = 30 \\
inputs['mrestart'] = 0 \\
inputs['compiled\_form'] = 'parts' \\
model = Qtcm(**inputs) \\
model.run\_session()}
\end{codeblock}

\caption{An example \mods{qtcm} run showing detailed description of
	input and output directories.}
\label{fig:manage.dir.example}
\end{figure}
\end{latexonly}

\begin{htmlonly}
\label{fig:manage.dir.example}
\begin{center}
\htmlfigcaption{%
	\codeblockfont{%
from qtcm import Qtcm \\
rundirname = 'test' \\
dirbasepath = os.path.join(os.getcwd(), rundirname) \\
inputs = \{\} \\
inputs['bnddir'] = os.path.join( os.getcwd(), 'bnddir', \\
\hspace*{40ex}'r64x42' ) \\
inputs['SSTdir'] = os.path.join( os.getcwd(), 'bnddir', \\
\hspace*{40ex}'r64x42', 'SST\_Reynolds' ) \\
inputs['outdir'] = dirbasepath \\
inputs['runname'] = rundirname \\
inputs['landon'] = 0 \\
inputs['year0'] = 1 \\
inputs['month0'] = 11 \\
inputs['day0'] = 1 \\
inputs['lastday'] = 30 \\
inputs['mrestart'] = 0 \\
inputs['compiled\_form'] = 'parts' \\
model = Qtcm(**inputs) \\
model.run\_session()}
	}

\htmlfigcaption{Figure~\ref{fig:manage.dir.example}:
	An example \mods{qtcm} run showing detailed description of
        input and output directories.}
\end{center}
\end{htmlonly}


Figure~\ref{fig:manage.dir.example} shows an example run where those
directories are explicitly specified; in all other aspects, the run
is identical to the one in Figure~\ref{fig:my.first.run}.
In Figure~\ref{fig:manage.dir.example}, output from the model is
directed to the directory described by string variable
\vars{dirbasepath}.  \vars{dirbasepath} is created by joining the
current working directory with the run name given in string variable
\vars{rundirname}.\footnote%
	{The Python \mods{os} module enables platform-independent
	handling of files and directories.  The \mods{os.path.join}
	function resolves paths without the programmer needing to know
	all the possible directory separation characters; the function
	chooses the correct separation character at runtime.  The
	\mods{os.getcwd} function returns the current working directory.}
Setting keyword parameter \vars{outdir} to \vars{dirbasepath} sends
output to \vars{dirbasepath}.  
Keywords \vars{bnddir} and \vars{SSTdir} specify the directories
where non-SST and SST boundary condition files, respectively, are
found.

Interestingly, the default version of QTCM1 does \emph{not} send
all output from the model to \vars{outdir}.  The restart file
\fn{qtcm\_}\dumarg{yyyymmdd}\fn{.restart} (where \dumarg{yyyymmdd}
is the year, month, and day of the model date when the restart
file was written) is written into the current working directory,
not the output directory.  Thus, if you do multiple runs, you'll
have to manually deal with the restart files that will proliferate.

Neither the QTCM1 model nor the \class{Qtcm} object
create the directories specified in \mods{bnddir}, \mods{SSTdir},
and \mods{outdir}.  Failure to do so will create an error.  I use
Python's file management tools to make sure the output directory
is created, and any old output files are deleted.  Here's an example
that does that, using the \vars{dirbasepath} and \vars{rundirname}
variables from Figure~\ref{fig:manage.dir.example}:

\begin{codeblock}
\codeblockfont{%
\small
if not os.path.exists(dirbasepath):  os.makedirs(dirbasepath) \\
qi\_file = os.path.join( dirbasepath, 'qi\_'+rundirname+'.nc' ) \\
qm\_file = os.path.join( dirbasepath, 'qm\_'+rundirname+'.nc' ) \\
if os.path.exists(qi\_file):   os.remove(qi\_file) \\
if os.path.exists(qm\_file):   os.remove(qm\_file)}
\end{codeblock}




%---------------------------------------------------------------------
\section{Model Field Variables}   \label{sec:field.variables.intro}

The term ``field'' variable refers to QTCM1 model variables that 
are accessible at both the compiled Fortran QTCM1 model-level as
well as the Python \class{Qtcm} instance-level.
Field variables are all instances of the \class{Field} class,
and are stored as attributes of the \class{Qtcm} instance.\footnote%
	{Note non-field variables can also be instances of \class{Field},
	and that \class{Qtcm} instances have other attributes that are
	not equal to \class{Field} instances.}

\class{Field} class instances have the following attributes:
\begin{itemize}
\item \vars{id}:  A string naming the field (e.g., ``Qc'', ``mrestart'').
	This string should contain no whitespace.
\item \vars{value}:  The value of the field.  Can be of any type, though
	typically is either a string or numeric scalar or a numeric array.
\item \vars{units}:  A string giving the units of the field.
\item \vars{long\_name}:  A string giving a description of the field.
\end{itemize}

\class{Field} instances also have methods to return the rank 
and typecode of \vars{value}.

Remember, if you want to access the value of a \class{Field} object,
make sure you access that object's \vars{value} attribute.  
Thus, for example,
to assign a variable \vars{foo} to the
\vars{lastday} value for a given
\class{Qtcm} instance \vars{model}, type the following:
\begin{codeblock}
\codeblockfont{%
foo = model.lastday.value}
\end{codeblock}

For scalars, this assignment sets \vars{foo} by value (i.e., a copy
of the value of attribute \vars{model.lastday} is set to \vars{foo}).
In general, however, Python assigns variables by reference.  Use
the \mods{copy} module if you truly want a copy of a field variable's
value (such as an array), rather than an alias.  For more details
about field variables, see Section~\ref{sec:field.variables}.




%---------------------------------------------------------------------
\section{Run Sessions}

	\subsection{What is a Run Session?}

A run session is a unit of simulation where the model is run from
day 1 of simulation to the day specified by the \vars{lastday}
attribute of a \class{Qtcm} instance.  A run session is a
``complete'' model run, at the beginning of which all compiled QTCM1
model variables are set to the values given at the Python-level,
and at the end of which restart files are written, the values
at the Python-level are overwritten by the values in the Fortran
model, and a Python-accessible snapshot is taken of the 
model variables that were written to the restart file.


	\subsection{Changing Variables}

Between run sessions, changing any field variable is as easy
as a Python assignment.  For instance, to change the atmosphere
mixed layer depth to 100~m, just type:
\begin{codeblock}
\codeblockfont{%
model.ziml.value = 100.0}
\end{codeblock}

When changing arrays, be careful to try to match the shape of the 
array.\footnote%
	{At the very least, match the rank of the array, which is required
	for the routines in \mods{setbypy} to properly choose which
	Fortran subroutine to use in reading the Python value.
	I haven't tested if only the rank is needed, however,
	for the passing to work, for a continuation run (my hunch is
	it won't).}
You can use the NumPy \mods{shape} function on a NumPy array to
check its shape.


	\subsection{Continuing a Model Run}  \label{sec:continuation.intro}

Figure~\ref{fig:continuation.example} shows an example of two run
sessions, where the second run session is a continuation of the
first.


%--- Two versions, one for PDF, one for HTML:
\begin{latexonly}
\begin{figure}[htp]
\begin{codeblock}
\codeblockfont{%
\small
inputs['year0'] = 1 \\
inputs['month0'] = 11 \\
inputs['day0'] = 1 \\
inputs['lastday'] = 10 \\
inputs['mrestart'] = 0 \\
inputs['compiled\_form'] = 'parts' \\ \\
model = Qtcm(**inputs) \\
model.run\_session() \\
model.u1.value = model.u1.value * 2.0 \\
model.init\_with\_instance\_state = True \\
model.run\_session(cont=30)}
\end{codeblock}

\caption{An example of two \mods{qtcm} run sessions where the second
	run session is a continuation of the first.  Assume 
	\vars{inputs} is a dictionary, and that earlier in the
	script the run name and
	all input and output directory names were added
	to the dictionary.}
\label{fig:continuation.example}
\end{figure}
\end{latexonly}

\begin{htmlonly}
\label{fig:continuation.example}
\begin{center}
\htmlfigcaption{%
	\codeblockfont{%
inputs['year0'] = 1 \\
inputs['month0'] = 11 \\
inputs['day0'] = 1 \\
inputs['lastday'] = 10 \\
inputs['mrestart'] = 0 \\
inputs['compiled\_form'] = 'parts' \\ \\
model = Qtcm(**inputs) \\
model.run\_session() \\
model.u1.value = model.u1.value * 2.0 \\
model.init\_with\_instance\_state = True \\
model.run\_session(cont=30)}
	}

\htmlfigcaption{Figure~\ref{fig:continuation.example}:
	An example of two \mods{qtcm} run sessions where the second
	run session is a continuation of the first.  Assume 
	\vars{inputs} is a dictionary, and that earlier in the
	script the run name and
	all input and output directory names were added
	to the dictionary.}
\end{center}
\end{htmlonly}


The first run session runs from day 1 to day 10.  The second
run session runs the model for another 30 days.  
Setting the \vars{init\_with\_instance\_state} of
\vars{model} to \vars{True} tells the model to use the
the values of the instance attributes 
(for prognostic variables, right-hand sides, and start date) 
are currently stored \vars{model}
as the initial values for the run\_session.\footnote%
	{Unless overridden, by default, 
	\vars{init\_with\_instance\_state} is set
	to True on \class{Qtcm} instance instantiation.}
The \vars{cont}
keyword in the second \mods{run\_session} call specifies a
continuation run, and the value gives the number of additional
days to run the model.

The set of runs described above would produce the exact same
results as if you had gone into the Fortran model after 10 days,
doubled the first baroclinic mode zonal velocity, and continued
the run for another 30 days.  With the Python example above, however,
you didn't need to know you were going to do that ahead of starting
the model run (which is what a compiled model requires you to do).
Section~\ref{sec:contination.run.sessions} describes continuation
runs in detail.


	\subsection{Passing Restart Snapshots Between Run Sessions}
					\label{sec:snapshot.intro}

The pure-Fortran QTCM1 uses a restart file to enable continuation
runs.  A \class{Qtcm} instance can also make use of that option,
through setting the \vars{mrestart} attribute value
(see Section~\ref{sec:contination.run.sessions} and
Neelin et al.\ \cite{Neelin/etal:2002} for details).  
It's easier, however, instead of using a restart file, to pass 
along a ``snapshot'' dictionary.

The \class{Qtcm} instance method \mods{make\_snapshot} copies the
variables that would be written out to a restart file into a
dictionary that is saves as the instance attribute \vars{snapshot}.
This snapshot can be saved separately, for later recall.  Note that
snapshots are automatically made at the end of a run session.

The following example shows a model \mods{run\_session} call,
following which the snapshot is saved to the variable
\vars{snapshot}:\footnote%
	{Remember Python assignment defaults to assignment by
	reference, so in this example the variable \vars{mysnapshot}
	is a pointer to the \vars{model.snapshot} attribute.
	(However, note that \vars{model.snapshot} itself is not a
	reference, but a distinct copy of those variables; to do
	otherwise would result in a non-static snapshot.)
	If the \vars{model.snapshot} attribute is dereferenced,
	then \vars{mysnapshot} will become the sole pointer to the
	dictionary.}

\begin{codeblock}
\codeblockfont{%
model.run\_session() \\
mysnapshot = model.snapshot}
\end{codeblock}

After taking the snapshot, you might continue the run a while, and
then decide to return to the snapshot you saved.  To do so, use
the \mods{sync\_set\_py\_values\_to\_snapshot}
method to reset the model instance values to
\vars{mysnapshot} before your next run session:
\begin{codeblock}
\codeblockfont{%
model.sync\_set\_py\_values\_to\_snapshot(snapshot=mysnapshot) \\
model.init\_with\_instance\_state = True \\
model.run\_session()}
\end{codeblock}

See Section~\ref{sec:snapshots} for details regarding the use of
snapshots, as well as for a list of what variables are saved in
a snapshot.




%---------------------------------------------------------------------
\section{Creating Multiple Models}

	\subsection{Model Instances}

Creating a new QTCM1 model is as simple as creating another
\class{Qtcm} instance.
For instance, to instantiate two QTCM1
models, \vars{model1} and \vars{model2}, type the following:

\begin{codeblock}
\codeblockfont{%
from qtcm import Qtcm \\
model1 = Qtcm(compiled\_form='parts') \\
model2 = Qtcm(compiled\_form='parts')}
\end{codeblock}

\vars{model1} and \vars{model2} do \emph{not} share any variables
in common, including the extension modules holding the Fortran
code.  In creating the instances, a copy of the extension modules
are saved in temporary directories.


	\subsection{Passing Snapshots To Other Models}

The snapshots described in Section~\ref{sec:snapshot.intro}
can also be passed around to other model instances,
enabling you to easily branch a model run:

\begin{codeblock}
\codeblockfont{%
model.run\_session() \\
mysnapshot = model.snapshot \\
model1.sync\_set\_py\_values\_to\_snapshot(snapshot=mysnapshot) \\
model2.sync\_set\_py\_values\_to\_snapshot(snapshot=mysnapshot) \\
model1.run\_session() \\
model2.run\_session()}
\end{codeblock}

The state of \vars{model} after its run session is used to start
\vars{model1} and \vars{model2}.  This is an easy way to save time
in spinning-up multiple models.




%---------------------------------------------------------------------
\section{Run Lists}		\label{sec:runlist.intro}

This feature of \class{Qtcm} objects is what really gives 
\class{Qtcm} model instances their flexibility.
A run list is a list of strings and dictionaries that specify
what routines to run in order to execute a particular part of
the model.  Each element of the run list specifies the method
or subroutine to execute, and the order of the elements specifies
their execution order.

For instance, the standard run list for initializing the the
atmospheric portion of the model is named ``qtcminit'', and
equals the following list:

\begin{latexonly}
\begin{codeblock}
\codeblockfont{%
\parbox{46ex}{% This file is automatically generated by
% code_to_latex.py.  It lists the code of the qtcminit
% runlist.

['\_\_qtcm.wrapcall.wparinit', \\
 '\_\_qtcm.wrapcall.wbndinit', \\
 'varinit', \\
 \{'\_\_qtcm.wrapcall.wtimemanager': [1]\}, \\
 'atm\_physics1']}}
\end{codeblock}
\end{latexonly}

\begin{htmlonly}
\begin{quotation}
% This file is automatically generated by
% code_to_latex.py.  It lists the code of the qtcminit
% runlist.

['\_\_qtcm.wrapcall.wparinit', \\
 '\_\_qtcm.wrapcall.wbndinit', \\
 'varinit', \\
 \{'\_\_qtcm.wrapcall.wtimemanager': [1]\}, \\
 'atm\_physics1']
\end{quotation}
\end{htmlonly}

This list is stored as an entry in the \vars{runlists} dictionary
(with key \vars{'qtcminit'}).
\vars{runlists} is an attribute of a \class{Qtcm} instance.
Table~\ref{tab:stnd.runlists} lists all standard run lists.

When the run list element in the list is a string, the string gives the
name of the routine to execute.  The routine has no parameter
list.  The routine can be a
compiled QTCM1 model subroutine for which an interface has been
written (e.g., \mods{\_\_qtcm.wrapcall.wparinit}), 
a method of the of the Python model instance 
(e.g., \mods{varinit}), or another run list
(e.g., \vars{atm\_physics1}).

When the run list element is a 1-element dictionary, the key of
the dictionary element is the name of the routine, and the value
of the dictionary element is a list specifying input parameters
to be passed to the routine on call.  Thus, the element:
\begin{codeblock}
\codeblockfont{%
{\{'\_\_qtcm.wrapcall.wtimemanager': [1]\}}}
\end{codeblock}
calls the \mods{\_\_qtcm.wrapcall.wtimemanager} routine, passing in
one input parameter, which in this case is the value 1.

If you want to change the order of the run list, just change the
order of the list.  To add or remove routines to be executed, just
add and remove their names from the run list.
Python provides a number of methods to manipulate
lists (e.g., \mods{append}).  Since lists are dynamic data types
in Python, you do not have to do any recompiling to implement
the change.

The \vars{compiled\_form} attribute must be set to \vars{'parts'}
in the \class{Qtcm} instance in order to take advantage of the run
lists feature of the class.  Run lists are not available for
\vars{compiled\_form\thinspace=\thinspace'full'}, because subroutine
calls are hardwired in the compiled QTCM1 model Fortran code in
that case.




%---------------------------------------------------------------------
\section{Model Output}			\label{sec:output.intro}

	\subsection{NetCDF Output}

Model output is written to netCDF files in the directory
specified by the \class{Qtcm} instance attribute \vars{outdir}.
Mean values are written to an output file beginning with
\fn{qm\_}, and instantaneous values are written to an output
file beginning with \fn{qi\_}.

The frequency of mean output is controlled by \vars{ntout}, and the
frequency of instantaneous output is controlled by \vars{ntouti}.
\vars{ntout.value} gives the number of days over which to average
(and if equals \vars{-30}, monthly means are calculated).
\vars{ntouti.value} gives the frequency in days that instantaneous
values are output (monthly if it equals \vars{-30}).  (See
Section~\ref{sec:initial.variables} for a description of other
output-control variables, and see the QTCM1 manual \cite{Neelin/etal:2002}
for a detailed description of how these variables control output.)

Figure~\ref{fig:netcdf.read} gives an example of a block of code
to read netCDF output, where \vars{datafn} is the netCDF filename, and
\vars{id} is the string name of the field variable (e.g.,
\vars{'u1'}, \vars{'T1'}, etc.).
(Note that the netCDF identifier for field variables is the same as
the name in \class{Qtcm}, except for the variables given in
Table~\ref{tab:qtcm.netcdf.ids}.)

In the code in Figure~\ref{fig:netcdf.read},
the array value is read into \vars{data}, and the longitude values, 
latitude values, and time values are read into variables
\vars{lon}, \vars{lat}, and \vars{time}, respectively.
As netCDF files also hold metadata, a description and the units
of the variable given by \vars{id}, and each dimension, are read
into variables ending in \vars{\_name} and \vars{\_units},
respectively.


%--- Two versions, one for PDF, one for HTML:
\begin{latexonly}
\begin{figure}[htp]
\begin{codeblock}
\codeblockfont{%
import numpy as N \\
import Scientific as S \\ \\
fileobj = S.NetCDFFile(datafn, mode='r') \\ \\
data = N.array(fileobj.variables[id].getValue()) \\
data\_name = fileobj.variables[id].long\_name \\
data\_units = fileobj.variables[id].units \\ \\
lat = N.array(fileobj.variables['lat'].getValue()) \\
lat\_name = fileobj.variables['lat'].long\_name \\
lat\_units = fileobj.variables['lat'].units \\ \\
lon = N.array(fileobj.variables['lon'].getValue()) \\
lon\_name = fileobj.variables['lon'].long\_name \\
lon\_units = fileobj.variables['lon'].units \\ \\
time = N.array(fileobj.variables['time'].getValue()) \\
time\_name = fileobj.variables['time'].long\_name \\
time\_units = fileobj.variables['time'].units \\ \\
fileobj.close()}
\end{codeblock}

\caption{Example of Python code to read netCDF output.
	See text for description.}
\label{fig:netcdf.read}
\end{figure}
\end{latexonly}

\begin{htmlonly}
\label{fig:netcdf.read}
\begin{center}
\htmlfigcaption{%
	\codeblockfont{%
import numpy as N \\
import Scientific as S \\ \\
fileobj = S.NetCDFFile(datafn, mode='r') \\ \\
data = N.array(fileobj.variables[id].getValue()) \\
data\_name = fileobj.variables[id].long\_name \\
data\_units = fileobj.variables[id].units \\ \\
lat = N.array(fileobj.variables['lat'].getValue()) \\
lat\_name = fileobj.variables['lat'].long\_name \\
lat\_units = fileobj.variables['lat'].units \\ \\
lon = N.array(fileobj.variables['lon'].getValue()) \\
lon\_name = fileobj.variables['lon'].long\_name \\
lon\_units = fileobj.variables['lon'].units \\ \\
time = N.array(fileobj.variables['time'].getValue()) \\
time\_name = fileobj.variables['time'].long\_name \\
time\_units = fileobj.variables['time'].units \\ \\
fileobj.close()}
	}

\htmlfigcaption{Figure~\ref{fig:netcdf.read}:
	Example of Python code to read netCDF output.
	See text for description.}
\end{center}
\end{htmlonly}





\begin{table}[tp]
\begin{center}
\begin{tabular}{l|l}
\textbf{\class{Qtcm} Attribute Name} & \textbf{NetCDF Output Name} \\
\hline
\vars{'Qc'}                & \vars{'Prec'} \\
\vars{'FLWut'}             & \vars{'OLR'} \\
\vars{'STYPE'}             & \vars{'stype'}
\end{tabular}
\end{center}
\caption{NetCDF output names for \class{Qtcm} field variables that
	are different from the \class{Qtcm} and compiled QTCM1 model
	variable names.  The netCDF names are case-sensitive.}
\label{tab:qtcm.netcdf.ids}
\end{table}


\emphpara{NB:}  All netCDF array output is dimensioned (time, latitude,
longitude) when read into Python using the \mods{Scientific} package.
This differs from the way \class{Qtcm} saves field variables, which
follows Fortran convention (longitude, latitude).  Please be careful
when relating the two types of arrays.
Section~\ref{sec:field.var.shape} for a discussion of why there is
this discrepancy.


	\subsection{Visualization}	\label{sec:viz.intro}

The \mods{plotm} method of \class{Qtcm} instances creates line
plots or contour plots, as appropriate, of model output of
average fields of run session(s) associated with the instance.
Some examples, assuming \vars{model} is an instance of \class{Qtcm}
and has already executed a run session:
\begin{itemize}
\item \cmd{model.plotm('Qc', lat=1.875)}:
	A time vs.\ longitude contour
          plot is made for the full range of time and longitude,
          at the latitude 1.875 deg N, for mean precipitation.

\item \cmd{model.plotm('Qc', time=10)}:
	A latitude vs.\ longitude contour plot of precipitation
	is made for the full spatial domain at day 10 of the model run.

\item \cmd{model.plotm('Evap', lat=1.875, lon=[100,200])}:  A contour
	plot of time vs.\ longitude of evaporation is made for the
          longitude points between 100 and 200 degrees E, at the
          latitude 1.875 deg N.  

\item \cmd{model.plotm('cl1', lat=1.875, lon=[100,200], time=20)}:
          A deep cloud amount vs.\ longitude line plot is made for
          the longitude points between 100 and 200 degrees east,
          at the latitude 1.875 deg N, at day 20 of the model run.
\end{itemize}

In these examples, the number of days over which the mean is taken
equals \vars{model.ntout.value}.
Also, the \mods{plotm} method automatically takes into account the
\class{Qtcm}/netCDF variable differences described in
Table~\ref{tab:qtcm.netcdf.ids}.



%---------------------------------------------------------------------
\section{Documentation}

Section~\ref{sec:ver} gives the online locations of the
transparent copies of this manual.  
Model formulation is fully described in
Neelin \& Zeng \cite{Neelin/Zeng:2000} and model
results are described in Zeng et~al.\ \cite{Zeng/etal:2000}
(\cite{Neelin/Zeng:2000} is based upon v2.0 of QTCM1
and \cite{Zeng/etal:2000} is based on QTCM1 v2.1).
Additional documentation you'll find useful include:

\begin{itemize}
\item \latexhtml{%
\htmladdnormallinkfoot{The \mods{qtcm} Package API Documentation}%
        {http://www.johnny-lin.com/py\_pkgs/qtcm/doc/html-api/}}%
{\htmladdnormallink{The \mods{qtcm} Package API Documentation}%
        {http://www.johnny-lin.com/py_pkgs/qtcm/doc/html-api/}}

\item \latexhtml{%
\htmladdnormallinkfoot{The Pure-Fortran QTCM1 Manual}%
        {http://www.atmos.ucla.edu/$\sim$csi/qtcm\_man/v2.3/qtcm\_manv2.3.pdf}}%
{\htmladdnormallink{The Pure-Fortran QTCM1 Manual}%
        {http://www.atmos.ucla.edu/~csi/qtcm_man/v2.3/qtcm_manv2.3.pdf}}
\cite{Neelin/etal:2002}

\end{itemize}



% ===== end of file =====


\chapter{Using \mods{qtcm}}                 \label{ch:using}
% ==========================================================================
% Using QTCM
%
% By Johnny Lin
% ==========================================================================


% ------ BODY -----
%
%---------------------------------------------------------------------
\section{Introduction}

Now that you've successfully run your first model instances, in
this chapter I provide detailed explanations regarding the features
of \mods{qtcm}.  I present these explanations in a documentary
rather than didactic fashion; my goal is to document how the features
work.  More details are given in the code docstrings.  At the end
of the chapter, in Section~\ref{sec:cookbook}, I provide a few
cookbook ideas/examples of ways to use the model.




%---------------------------------------------------------------------
\section{Model Instances}  \label{sec:model.instances}

An instance of a \class{Qtcm} model is created in \mods{qtcm} the same way
you create an instance of any class.
For instance, to instantiate two \class{Qtcm}
models, \vars{model1} and \vars{model2}, I type the following:

\begin{codeblock}
\codeblockfont{%
from qtcm import Qtcm \\
model1 = Qtcm(compiled\_form\thinspace=\thinspace'full') \\
model2 = Qtcm(compiled\_form\thinspace=\thinspace'parts')}
\end{codeblock}

In the above example, \vars{model1} uses the compiled QTCM1 model
that runs the model (essentially) using the Fortran driver,
while \vars{model2} uses the compiled QTCM1 model where execution
order and content all the way down to the atmospheric timestep level
is controlled by Python run lists.  (Section~\ref{sec:compiledform}
has more details about the difference between compiled forms.)

For each instance of \class{Qtcm}, copies of all needed extension
modules (e.g., \fn{.so} files) are copied to a temporary directory
that is automatically created by Python.  The full path name of
that directory is saved in the instance attribute \vars{sodir}.
These extension modules are then associated with the specific instance 
through private instance attributes,
and thus every instance of \class{Qtcm} has its own separate variable
and name space on both the Fortran and Python sides.\footnote%
	{The private instance attribute is \vars{\_\_qtcm}.
	See Section~\ref{sec:Qtcm.private.attrib} for details about 
	private \class{Qtcm} instance attributes.}
The temporary directory and all of its contents are deleted when the 
model instance is deleted.

On instantiation, \class{Qtcm} instances set all scalar field
variables to their default values as given in the submodule
\mods{defaults} (and listed in Section~\ref{sec:defaults.scalar}),
and assign the fields as instance attributes.  The instance attribute
\vars{init\_with\_instance\_state} is set to True by default, unless
overridden on instantiation.




%---------------------------------------------------------------------
\section{Initializing a Model Run}

In the pure-Fortran QTCM1, there are three broad
classes of initialized variables:
\begin{enumerate}
\item Those that are read-in using a namelist, 
\item Those that the are read-in from a restart file, and
\item Those that are set by assignment in the Fortran code.  
\end{enumerate}
These variables are a combination of scalars and arrays.

For \mods{qtcm}, interfaces were built so that all classes of
initialized variables that could be user-controlled are accessible
and changeable at the Python-level.  For \mods{qtcm},
the set of variables that could be changed is also expanded, to
include not just the first and second classes of pure-Fortran
QTCM1 initialized variables.  This was done to make \mods{qtcm}
more flexible.  All variables that can be passed between the
compiled QTCM1 model and Python model levels are called
field variables, and are described in detail in
Section~\ref{sec:field.variables}.

As it happens, all the namelist-set variables are scalars.  In the
pure-Fortran QTCM1, those variables are given default values prior
to reading in of the namelist.  To duplicate this functionality,
on model instantiation, all scalar fields are set to their default
values as given in the submodule \mods{defaults} and listed in
Section~\ref{sec:defaults.scalar}.  Most of the default values in
\mods{defaults} are the same as in the pure-Fortran QTCM1, but
there are a few differences.\footnote%
	{One difference being \vars{mrestart}, which in \vars{qtcm} 
	will have the value of 0 in contrast to the pure-Fortran 
	QTCM1 where the default is the 1.}
This setting of scalar defaults is the same for both
\vars{compiled\_form\thinspace=\thinspace'full'} and
\vars{compiled\_form\thinspace=\thinspace'parts'} instances.
Of course, all
\mods{qtcm} fields are user-controllable, both via keyword input
parameters at model instantiation as well as through direct
manipulation of the instance attribute that stores the field variable.

The pure-Fortran QTCM1 initialized prognostic variables and
right-hand sides are set in the Fortran subroutine \mods{varinit}.
Their they are read-in from a restart file or, as default,
set by assignment.
In \mods{qtcm}, the same variables are initialized by a \class{Qtcm}
instance method of the same name, \mods{varinit}, for the case when
\vars{compiled\_form\thinspace=\thinspace'parts'}.  For the case
of \vars{compiled\_form\thinspace=\thinspace'full'}, the compiled
QTCM1 subroutine that is the same as in the pure-Fortran case is
used, and that routine is inaccessible at the Python level.
See Section~\ref{sec:snapshots}'s listing of snapshot variables,
which also includes the prognostic variables and right-hand sides that
are set in \mods{varinit} (both Fortran and Python).




%---------------------------------------------------------------------
\section{The \vars{compiled\_form} Keyword}  \label{sec:compiledform}

The \mods{qtcm} package is a Python wrap of the Fortran routines
that make up QTCM1.  The wrapping layer adds flexibility and
functionality, but at the cost of speed.  Thus, I created two
types of extension modules from the Fortran QTCM1 code, one
which permits very little control over the compiled Fortran
\emph{routines} at the Python level, and one that allows the Python-level
to control model execution in the compiled QTCM1 model
all the way down to the atmospheric timestep level.\footnote%
	{That control is via run lists, which are described in
	Section~\ref{sec:runlists}.}
The former extension module corresponds to 
\vars{compiled\_form\thinspace=\thinspace'full'} and
the latter extension module to
\vars{compiled\_form\thinspace=\thinspace'parts'}.

For \vars{compiled\_form\thinspace=\thinspace'full'},
the compiled portion of the model encompasses (nearly) the
entire QTCM1 model as a whole.  Thus, the only compiled QTCM1 model
modules or subroutines that Python should interact with is
the \mods{driver} routine (which executes the entire model) and
the \mods{setbypy} module (which enables communication between the
compiled model and the Python-level of model fields.\footnote%
	{The \mods{setbypy} Python module is the wrap of the
	Fortran QTCM1 \mods{SetByPy} module.}

For \vars{compiled\_form\thinspace=\thinspace'parts'}, the compiled
portion of the model does not encompasses the model as a whole, but
rather is broken up into separate units (as appropriate) all the
way down to an atmosphere timestep.  Thus, compiled QTCM1 model
modules/subroutines that are accessible at the Python-level include
those that are executed within an atmosphere timestep on up.

Because the difference in compiled forms fundamentally affects how
the \class{Qtcm} instance facilitates Python-Fortran communication,
this attribute must be set on instantiation via a keyword input
parameter.

In the rest of this section, to avoid being verbose, when I
write \vars{'full'}, I mean the situation where
\vars{compiled\_form\thinspace=\thinspace'full'}.
Likewise, when I
write \vars{'parts'}, I mean the situation where
\vars{compiled\_form\thinspace=\thinspace'parts'}.


	\subsection{Initialization for 
			\vars{compiled\_form\thinspace=\thinspace'full'}}
				\label{sec:init.compiledform.full}

For a model run of this case, the \class{Qtcm} instance will
initialize the model using the Fortran \mods{varinit} subroutine
in the compiled QTCM1 model.  This subroutine does the following:

\begin{itemize}
\item If \vars{mrestart\thinspace=\thinspace1}, 
	the restart file is used to initialize all prognostic
	variables.  In terms of start date, the following rules are
	used:
	\begin{enumerate}
	\item Variable \vars{dateofmodel} is read from the restart file.
	\item If \vars{day0}, \vars{month0}, and \vars{year0}
		are negative, or otherwise
		invalid (e.g., \vars{month0} greater than 12), the invalid
		value is replaced with the
		day, month, and/or year of the day \emph{after} 
		that given by \vars{dateofmodel}.
		If the value of \vars{day0}, \vars{month0}, or \vars{year0}
		is not invalid in this sense, it is not replaced.
	\end{enumerate}
	Thus, if the restart file gives 
	\vars{dateofmodel} equal to 101102
	(year 10, month 11, day 2), and 
	\vars{day0\thinspace=\thinspace-1}, 
	\vars{month0\thinspace=\thinspace-1}, 
	\vars{year0\thinspace=\thinspace-1},
	and 
	\vars{mrestart\thinspace=\thinspace1}, 
	the model will start running from year 10, month 11, day 3.
	If \vars{dateofmodel} equals to 101102, and 
	\vars{day0\thinspace=\thinspace-1}, 
	\vars{month0\thinspace=\thinspace3}, 
	\vars{year0\thinspace=\thinspace-1},
	the model will start running from year 10, month 3, day 3.

\item If \vars{mrestart\thinspace=\thinspace0}, 
	all prognostic variables and right-hand sides are set to an
	initial value (which for most of those variables is zero).
	In terms of start date, \vars{day0} is set to 1 (and thus 
	the value of \vars{day0} previously input is ignored), and
	both \vars{month0} and \vars{year0}
	are set to 1 
	if their previously input values are invalid (where
	invalid means less than
	1, or, for \vars{month0}, greater than 12).
	Otherwise, \vars{month0} and \vars{year0} are left unchanged.
	Variable \vars{dateofmodel} has the value it had when the variable
	was declared (which is determined by the compiler and usually
	is zero; \vars{dateofmodel} will not be properly set until
	subroutine \mods{TimeManager} is called.

	Thus, if 
	\vars{day0\thinspace=\thinspace-1},
	\vars{month0\thinspace=\thinspace-1}, 
	\vars{year0\thinspace=\thinspace-1} is input into the model
	(say from a namelist) and 
	\vars{mrestart\thinspace=\thinspace0},
	the model will start running from year 1, month 1, day 1,
	and \vars{dateofmodel} at the exit of subroutine 
	\mods{varinit} will equal its compiler-set default.
	If 
	\vars{day0\thinspace=\thinspace14}, 
	\vars{month0\thinspace=\thinspace3}, 
	\vars{year0\thinspace=\thinspace11}, and 
	\vars{mrestart\thinspace=\thinspace0} on input into the
	model,
	the model will start running from year 11, month 3, day 1,
	and \vars{dateofmodel} at the exit of subroutine 
	\mods{varinit} will equal its compiler-set default.

	Note that \vars{dateofmodel}
	can thus be inconsistent with 
	\vars{month0} and \vars{year0} at the
	exit of subroutine \mods{varinit}.
\end{itemize}

This behavior with respect to initializing
the start date is different than in QTCM1 versions 1.0 and 2.1.
Please see the source code from those earlier QTCM1 versions for
details.




	\subsection{Initialization for 
			\vars{compiled\_form\thinspace=\thinspace'parts'}}
				\label{sec:init.compiledform.parts}

For \vars{'parts'} model, the methodology of how initialized
prognostic variables, right-hand sides, and start date related
variables are set is controlled by the \class{Qtcm} instance
attribute/flag \vars{init\_with\_instance\_state}.  The initialization
is (mostly) executed in the Python \vars{varinit} method in the
following way:

\begin{itemize}
\item If \vars{init\_with\_instance\_state} is False:
The method as described for
initialization for the 
\vars{'full'} case is generally
followed, with the exception that dateofmodel is set
to match \vars{day0}, \vars{month0}, \vars{year0}, prior to exit of 
\mods{varinit}.

\item If \vars{init\_with\_instance\_state} is True:
the model object will initialize the model based on the current
state of the model instance.  This enables you to set a model run
session's initial conditions based upon the state of the prognostic
variables and parameters stored at the Python level, which is
accessible at runtime.
\end{itemize}


Since the \vars{init\_with\_instance\_state\thinspace=\thinspace{False}}
case is mainly described by the initialization method for the
\vars{'full'} case, I refer the
reader to Section~\ref{sec:init.compiledform.full}.
For the case of \vars{init\_with\_instance\_state} is True, however,
the task is more complicated.  Specifically, for that case,
initialization includes the following:

\begin{enumerate}
\item If not currently defined,
	variable \vars{dateofmodel} is set to a default value of 0,
	which is specified in the module defaults.

\item The \vars{mrestart} flag is ignored for variable initialization.

\item All prognostic variables and right-hand sides
        are set to an
        initial value (which for most of those variables is zero),
	unless the variable is defined at the Python level, in which
	case the inital value is set to the Python level defined value.

\item If \vars{dateofmodel} is greater than 0, 
	\vars{day0}, \vars{month0}, and \vars{year0} are overwritten
        with values derived from \vars{dateofmodel} 
	in order to set the run to start
	the day \emph{after} \vars{dateofmodel}.

\item If \vars{dateofmodel} is less than or equal to 0, \vars{day0},
	\vars{month0}, and \vars{year0} are set to their respective
	instance attribute values, if valid.  For invalid instance
	attribute values, the invalid \vars{day0}, \vars{month0},
	and/or \vars{year0} is set to 1.

\item Variable \vars{dateofmodel} is recalculated
	and overwritten to match 
	\vars{day0}, \vars{month0}, \vars{year0}, prior to exit of 
	\mods{varinit}.
\end{enumerate}

As a result, for \vars{init\_with\_instance\_state} is True, the
way you indicate to the model that a run session is a brand-new run
is by setting, before the \mods{run\_session} method call,
\vars{dateofmodel} to a value less than or equal to 0, and \vars{day0},
\vars{model0}, and \vars{year0} to the day you want the model to
begin the run session.  To indicate to the model you wish to continue
a run, set \vars{dateofmodel} to the day \emph{before} you want the
model to start running from.

Examples:
\begin{itemize}
\item If \vars{day0\thinspace=\thinspace-1}, 
	\vars{month0\thinspace=\thinspace-1}, 
	\vars{year0\thinspace=\thinspace-1}, and
	\vars{dateofmodel\thinspace=\thinspace0} is input into 
	the model the model will start running from year 1, month 1, day 1,
	and 
	variable \vars{dateofmodel} at the exit of 
	subroutine \mods{varinit}
	will equal 10101.

\item If \vars{day0\thinspace=\thinspace14},
	\vars{month0\thinspace=\thinspace3}, 
	\vars{year0\thinspace=\thinspace11},
	and \vars{dateofmodel\thinspace=\thinspace0} is input into the
	model, the model will start running from year 11, month 3, day 14,
	and 
	variable \vars{dateofmodel} at the exit of 
	subroutine \mods{varinit} will equal
	110314.

\item If \vars{day0\thinspace=\thinspace14},
	\vars{month0\thinspace=\thinspace3}, 
	\vars{year0\thinspace=\thinspace11},
	and \vars{dateofmodel\thinspace=\thinspace341023} is input into the
	model, the model will start running from year 34, month 10, day 24,
	and at the exit of subroutine 
	\mods{varinit}, \vars{dateofmodel} will equal
	341024, with \vars{day0\thinspace=\thinspace24},
	\vars{month0\thinspace=\thinspace10}, and
	\vars{year0\thinspace=\thinspace34}.
\end{itemize}


	\subsection{Communication Between Python and Fortran-Levels}
				\label{sec:comm.py.fort.compiledform}

After initialization, the second major difference between a
\vars{'full'} and \vars{'parts'} model is how and when communication
between the Python and Fortran levels can occur.  For the \vars{'full'}
case, except for the passing in and out of variables before and after
a run session, all variable passing and subroutine calling happens in
the compiled QTCM1 model, with no control at the Python level.
For the \vars{'parts'} case, variables can be passed between the
Python and Fortran-levels at all levels down to the atmospheric
timestep, and many Fortran QTCM1 subroutines can be called from the
Python-level.  


		\subsubsection{Passing Variables}

For all \vars{compiled\_form} cases, variables are passed back and
forth between the Python \class{Qtcm} instance level and the
compiled QTCM1 model Fortran-level using the \class{Qtcm}
instance methods \mods{get\_qtcm1\_item} and \mods{set\_qtcm1\_item}:\footnote%
	{All Fortran routines used to pass variables back and forth are
	defined in the \mods{setbypy} module of the \fn{.so} extension
	module stored in the \class{Qtcm} instance variable \vars{\_\_qtcm}.
	All Fortran wrappers that enable Python to call compiled QTCM1 model
	subroutines are defined in the \mods{wrapcall} module stored in
	the \class{Qtcm} instance variable \vars{\_\_qtcm}.
	These modules are described in detail in 
	Sections~\ref{sec:setbypy} and~\ref{sec:wrapcall}, respectively.}

\begin{itemize}
\item \mods{get\_qtcm1\_item}(\dumarg{key}):
	Returns the value of the field variable given by the string
	\dumarg{key}.  If the compiled QTCM1 model variable given by
	\dumarg{key} is unreadable, the
        custom exception 
	\vars{FieldNotReadableFromCompiledModel} is thrown.
	The value returned is a copy of the value on the Fortran
	side, not a reference to the variable in memory.

\item \mods{set\_qtcm1\_item}:
	Sets the value of a field variable
	in the compiled QTCM1 model \emph{and at the Python-level,}
	automatically overriding any previous value at both levels.
	Thus, calling this method will change/create the \class{Qtcm}
	instance attribute corresponding to the field variable.
        When the compiled QTCM1 model variable is set, a copy of the
        Python value is passed to the Fortran model; the
	variable is \emph{not passed by reference.}
	This value comes from the \mods{set\_qtcm1\_item} calling
	parameter list, \emph{not} from the \class{Qtcm}
        instance attribute corresponding to the field variable.
\end{itemize}

The \mods{set\_qtcm1\_item} method has two calling forms, one with
one argument and the other with two arguments:
\begin{itemize}
\item One argument:  The method is called
	as \mods{set\_qtcm1\_item}(\dumarg{arg}), where \dumarg{arg} 
	is either a string giving the name of the field variable or 
	a \class{Field} instance.

\item Two arguments:  The method is called as
	\mods{set\_qtcm1\_item}(\dumarg{key}, \dumarg{value}), where
	\dumarg{key} is the string giving the name of the field variable
	and \dumarg{value} is the value to set the model field variable to
	(note \dumarg{value} can be a \class{Field} instance).
\end{itemize}
In either calling form, if no value given, the default value as defined
in module \mods{defaults} is used.

Some compiled QTCM1 model variables are not in a state where they
can be set.  An example is a compiled QTCM1 model pointer variable,
prior to the pointer being associated with a target (an attempt
to set would yield a bus error).  In such cases, the
\mods{set\_qtcm1\_item} method will throw a
\vars{FieldNotReadableFromCompiledModel} exception, nothing will
be set in the compiled QTCM1 model, and the Python counterpart
field variable (if it previously existed) would be left unchanged.\footnote%
	{We handle this situation in this way to enable the
	\class{Qtcm} instance to store variables
	even if the compiled model is not yet ready to accept them.}

Examples, typed in at a Python prompt, and
assuming that \vars{model} is a \class{Qtcm} instance:
\begin{itemize}
\item \cmd{dtvalue\thinspace=\thinspace{model.get\_qtcm1\_item('dt')}}:
	Retrieves the value of field variable \vars{dt} (timestep)
	from the compiled QTCM1 Fortran model and sets it to the
	Python variable \vars{dtvalue}.

\item \cmd{model.set\_qtcm1\_item('dt')}:
	Sets the value of field variable \vars{dt}
	in the compiled QTCM1 Fortran model to the default
	value (as given in \mods{defaults}),
	and sets the value of Python attribute \vars{model.dt} also to 
	that default value.  
	Remember that \vars{model.dt} is a \class{Field}
	instance.

\item \cmd{model.set\_qtcm1\_item('dt', 2000.)}:
	Sets the value of field variable \vars{dt}
	in the compiled QTCM1 Fortran model to 2000 (as a real),
	and sets the value of Python attribute \vars{model.dt} also to 2000.
\end{itemize}


		\subsubsection{Calling Compiled QTCM1 Model Subroutines}

All compiled QTCM1 model subroutines that can be called
(except \mods{driver} and \mods{varptrinit}) are in the
\mods{setbypy} or \mods{wrapcall} modules
of the \class{Qtcm} instance private attribute \vars{\_\_qtcm}.
(On \class{Qtcm} instance instantiation, \vars{\_\_qtcm} is set
to the \fn{.so} extension module that is the compiled QTCM1 Fortran model.)
Thus, to call \mods{wmconvct} in \mods{wrapcall} at the Python-level,
just type \cmd{model.\_\_qtcm.wrapcall.wmconvct()} (where \vars{model}
is a \class{Qtcm} instance).
For \mods{driver} and \mods{varptrinit}, these subroutines are not
contained in a \vars{\_\_qtcm} module, and thus can be called
directly (e.g., just type \cmd{model.\_\_qtcm.driver()}).
See Sections~\ref{sec:setbypy} and~\ref{sec:wrapcall} for more information
on the \mods{setbypy} and \mods{wrapcall} modules.

For the \vars{'full'} case, the only compiled QTCM1 model
subroutine you can usefully call during a run session is \mods{driver}.
For the \vars{'parts'} case, while you can essentially call any subroutine
given in a run list, you usually will not directly call a compiled QTCM1
model subroutine but will instead call it through including it in a
run list.  For example, if you have the following run list in a
\vars{'parts'} model:
\begin{codeblock}
\codeblockfont{%
[ 'qtcminit', '\_\_qtcm.wrapcall.woutpinit' ]}
\end{codeblock}
Running this list using the \class{Qtcm} instance method
\mods{run\_list} will result in \class{Qtcm} instance method
\mods{qtcminit} first being run, 
then the compiled QTCM1 Fortran model subroutine
\mods{woutpinit} in Fortran module \mods{wrapcall} being run.
See Section~\ref{sec:runlists} and
Table~\ref{tab:stnd.runlists} for a discussion and list of the
standard run lists that control routine execution content and order
in the \vars{'parts'} case.




%---------------------------------------------------------------------
\section{Restart and Continuation Run Sessions}
				\label{sec:contination.run.sessions}


	\subsection{Restart Runs In the Pure-Fortran QTCM1}
					\label{sec:puref90.restart}

To enable restart of a model run, the pure-Fortran QTCM1 model
writes out a restart file with the state of the prognostic variables
and select right-hand sides at that point in the run (for a list
of the variables, see Section~\ref{sec:snapshots}).  This binary
file can then be read in by later model runs.  The Fortran
\vars{mrestart} flag is passed in via a namelist; if \vars{mrestart}
is 1, the run uses the restart file (named \fn{qtcm.restart}).

One of the problems with using the restart file to do a continuation
run is that the continuation run will not be perfect.  In other words,
a 15~day run followed by a 25~day run based on the restart file 
generated at the end of the 15~day run will \emph{not} give the
exact same output as a continuous 40~day run.


	\subsection{Overview of Restart/Continuation Options In \mods{qtcm}}
					\label{sec:restart.options.list}

For a \class{Qtcm} instance, in contrast to the pure-Fortran QTCM1,
more than one method of continuation is available.
Thus, for a continuation run, you need to tell the model
``continue from what?''
The \class{Qtcm} class provides three choices for restart/continuing
a run:
\begin{enumerate} 
\item From a restart file:  Move/rename a QTCM1 restart file
        to the current working directory to \fn{qtcm.restart}.
	\label{list:continue.from.restart}

\item From a snapshot from another run session
	(see Sections~\ref{sec:snapshot.intro} and~\ref{sec:snapshots}).
	\label{list:continue.from.snapshot}

\item From the values of the \class{Qtcm} instance you will be
	calling \mods{run\_session} from.
	\label{list:continue.from.instance}
\end{enumerate}

Restart/continuation methods~\ref{list:continue.from.restart} 
and~\ref{list:continue.from.snapshot} both suffer from the
same problem as the pure-Fortran QTCM1 restart process:
They do not produce perfect restarts
(see Section~{sec:puref90.restart} for details).
In this section, I discuss the restart/continuation options
for each \vars{compiled\_form} option.

Methods~\ref{list:continue.from.restart}
and~\ref{list:continue.from.snapshot} are best used when making a
run session from a newly instantiated \class{Qtcm} instance.
Method~\ref{list:continue.from.instance} is best used when executing
a run session using a \class{Qtcm} instance that has already gone
through at least one run session.  Regardless of which method you
use, however, please note that anytime you execute a run session
using a \class{Qtcm} instance that already has made a previous run
session, some variables \emph{cannot be updated} between run sessions.
This feature is most noticeable with the output filename, and occurs
because the name persists in the compiled QTCM model, and is stored
in the extension module (\fn{.so} files in \vars{sodir}) associated
with the instance.  If you wish to control all variables possible
from the Python level (including output filename), you need do the
run session from a new model instance.


	\subsection{Restart/Continuation for 
		\vars{compiled\_form\thinspace=\thinspace'full'} 
		Model Instances}

The only option for restart when using
\vars{compiled\_form\thinspace=\thinspace'full'} model instances
is method~\ref{list:continue.from.restart}, to use a QTCM1 restart
file.\footnote%
	{The \vars{cont} keyword parameter in \mods{run\_session}
	and the value of the \vars{init\_with\_instance\_state}
	attribute have no effect if
	\vars{compiled\_form\thinspace=\thinspace'full'}.  With
	\vars{'full'}, the call to initialize variables all happens
	at the Fortran level (via the Fortran \mods{varinit}, not
	the Python \mods{varinit}), with no reference to the Python field
	states (or even existing Fortran field states, if present).}
To use this option, the value of the \vars{mrestart} 
attribute must equal 1, the restart file must be named
\fn{qtcm.restart}, and the restart file must be in the 
current working directory.
As with the pure-Fortran QTCM1 restart process, this method
does not produce perfect restarts.



	\subsection{Restart/Continuation for 
		\vars{compiled\_form\thinspace=\thinspace'parts'} 
		Model Instances}

For the \vars{compiled\_form\thinspace=\thinspace'parts'} case,
all three restart/continuation methods
described in Section~\ref{sec:restart.options.list} are
available.


		\subsubsection{Method~\ref{list:continue.from.restart}:
			From a QTCM1 Restart File}

To use the QTCM1 restart file mechanism, not only must the
\vars{mrestart} attribute have a value to 1, but the
\vars{init\_with\_instance\_state} flag also has to be \vars{False},
otherwise the \vars{mrestart} attribute value will be ignored.  
As with the pure-Fortran QTCM1 restart process, this method does not
produce perfect restarts.


		\subsubsection{Method~\ref{list:continue.from.snapshot}:
			From a \class{Qtcm} Instance Snapshot}

You can take snapshots of the model state of a \class{Qtcm} instance
by the \mods{make\_snapshot} instance method.  This snapshot saves
a copy of all the variables saved to a QTCM1 restart file (see
Section~\ref{sec:snapshots} for the full list of fields), which
then can be passed to other \class{Qtcm} instances for use in other
run sessions.

The key difference between this method and 
method~\ref{list:continue.from.instance} (described below)
is that \mods{run\_session} calls using the snapshot are done
\emph{without} the \vars{cont} keyword input parameter
(by default, \vars{cont} is False).  If the \vars{cont} keyword
is not False, it says the run session is a continuation run
that uses the state of the compiled QTCM1 model for all variables
that are not specified at, and read-in from,
the Python level.  If the \vars{cont} keyword
is False, the run session initializes as if it were a new run.

See Section~\ref{sec:snapshot.intro} for details and
an example of using snapshots to initialize a run session.
Note that as with the pure-Fortran QTCM1 restart process, this method 
does not produce perfect restarts.


		\subsubsection{Method~\ref{list:continue.from.instance}:
			From the Calling \class{Qtcm} Instance}

This method is used when you want to make a run session that is a
``true'' continuation run, i.e., one that uses the current state
of the compiled QTCM1 model for all variables that are not read-in
from the Python level (remember that \class{Qtcm} instances hold a
subset of the variables defined at the Fortran level).  
The key reason to use this method for a continuation run session
is that the continuation is byte-for-byte the same (if no fields
are changed) as if the run just went straight on through.  Thus,
the continuation would be perfect: A 15~day run followed by a 25~day
run using the same \class{Qtcm} instance with the \vars{cont} keyword
will give the exact same output as a continuous 40~day run.  This
is not the case when making a new instance and passing a restart
file or a snapshot, because a separate extension module is used for
those new instances.

Control of this method is accomplished through the \vars{cont}
keyword input parameter to the \mods{run\_session} method and the
\vars{init\_with\_instance\_state} attribute of a
\class{Qtcm} instance:

\begin{itemize}
\item \vars{cont}: If set to False, the run session is not a
	continuation of the previous run, but a new run session.
	If set to True, the run session is a continuation of the
	previous run session.  If set to an integer greater than
	zero, the run session is a continuation just like
	\vars{cont\thinspace=\thinspace{True}}, but the value
	\vars{cont} is set to is used for \vars{lastday} and replaces
	\vars{lastday.value} in the \class{Qtcm} instance.

\item \vars{init\_with\_instance\_state}:
	If True, for a \mods{run\_session} call using the
	\vars{cont} keyword, whatever the field values are in the Python
	instance are used in the run session.
	If False, model variables are set and initialized as described in
	Section~\ref{sec:init.compiledform.parts}.  In that case,
	previous compiled QTCM1 model values will likely be overwritten.
	Thus, if you want a continuation run that uses the state of
	all field variables except for those you explicitly change at
	the Python-level, make sure \vars{init\_with\_instance\_state}
	is True.
\end{itemize}

(Note that the \vars{cont} keyword has no effect if \vars{compiled\_form}
is \vars{'full'}.  The default value of \vars{cont} in a
\mods{run\_session} call is False.  The value of keyword \vars{cont}
is stored as private instance attribute \vars{\_cont}, in case you
really need to access it elsewhere; see
Section~\ref{sec:Qtcm.private.attrib} for more details).

The example described in Section~\ref{sec:continuation.intro} is
an example of method~\ref{list:continue.from.instance} in the list
above: The second run session is continued from the state of
\vars{model}, with the values of \vars{model}'s instance variables
overriding any values in the compiled QTCM1 model in initializing
the second run session.

This method has a few caveats worthy of note:
\begin{itemize}
\item The \vars{init\_with\_instance\_state} attribute value
	will have no effect unless the instance prognostic variables
	are set, i.e., unless a previous run session has been done.
	Another way to put it is for an initial run session right
	after a \class{Qtcm} instance is created, \mods{varinit}
	will use the same initial values for prognostic variables
	(defined in \mods{defaults} module variable
	\vars{init\_prognostic\_dict})\footnote%
		{\vars{init\_prognostic\_dict} is the dictionary giving
		the default initial values of each prognostic variable
		and right-hand side (as defined by the restart file 
		specification).}
	as it would with for both
	\vars{init\_with\_instance\_state} set to True or False).

\item Continuation run sessions using this method have to continue
	with the next day from wherever the last run session left
	off, contiguously.\footnote%
		{For continuation run sessions, you keep the 
		same extension module (the compiled \fn{.so} library),
		and all the values that define the state where it
		left off.}
	If you want to do a non-contiguous run,
	create a new \class{Qtcm} instance initialized with a
	snapshot instead of the continuation method describe in
	this section.
	will use restart rules to run a new model.  

\item When making a continuation run session using this method,
	you cannot change some variables, for instance,
	\vars{outdir} and any of the date related
	variables.  In fact, the only thing you should change for
	your continuation run session are the prognostic and
	diagnostic variables and \vars{lastday}.  This is because
	some variables cannot be updated between run sessions.
	As noted in Section~\ref{sec:restart.options.list},
	if you wish to control all variables possible
	from the Python level (including output filename), you need 
	to execute the run session from a new model instance.
\end{itemize}


	\subsection{Snapshots of a \class{Qtcm} Instance}
				\label{sec:snapshots}

The snapshot dictionary (briefly described in
Section~\ref{sec:snapshot.intro}), saved as the \class{Qtcm} instance
attribute \vars{snapshot}, and generated by the method
\mods{make\_snapshot}, saves the current state of the following
instance field variables:

\begin{center}
% This file is automatically generated by
% code_to_latex.py.  It lists all the snapshot variables.


\begin{longtable}{l|c|c|p{0.4\linewidth}}
\textbf{Field} & \textbf{Shape} &
                                \textbf{Units} & \textbf{Description} \\
\hline
\endhead
    \vars{T1} & (64, 44) & K &  \\
\vars{Ts} & (64, 42) & K & Surface temperature \\
\vars{WD} & (64, 42) &  &  \\
\vars{dateofmodel} &      &  & Date of model coded as an integer as yyyymmdd \\
\vars{psi0} & (64, 43) &  &  \\
\vars{q1} & (64, 44) & K &  \\
\vars{rhsu0bar} & (3,) &  &  \\
\vars{rhsvort0} & (64, 42, 3) &  &  \\
\vars{title} &      &  & A descriptive title \\
\vars{u0} & (64, 44) & m/s & Barotropic zonal wind \\
\vars{u0bar} &      &  &  \\
\vars{u1} & (64, 44) & m/s & Current time step baroclinic zonal wind \\
\vars{v0} & (64, 43) & m/s & Barotropic meridional wind \\
\vars{v1} & (64, 43) & m/s &  \\
\vars{vort0} & (64, 42) &  &  \\
\end{longtable}
\end{center}

These are the same variables saved to a QTCM1 restart file, and so
a snapshot duplicates the restart functionality in the Python
environment, but with more flexibility.  Since the \vars{snapshot}
dictionary is a Python variable like any other, you can manipulate
it and alter it to fit any condition you wish.




%---------------------------------------------------------------------
\section{Creating and Using Run Lists}  \label{sec:runlists}

Section~\ref{sec:runlist.intro} provides an introduction to the
role and use of run lists.  A run list is a list of methods, Fortran
subroutines, and other run lists that can be executed by the
\class{Qtcm} instance \mods{run\_list} method.  Run lists are stored
in the \class{Qtcm} instance attribute \vars{runlists}, which is a
dictionary of run lists.  The names of run lists should not be
preceeded by two underscores (though elements of a run list may be
very private variables), nor should names of run lists be the same
as any instance attribute.  Run lists are not available for
\vars{compiled\_form\thinspace=\thinspace'full'}.

The \mods{run\_list} method takes a single input parameter, a list,
and runs through that list of elements that specify other run lists
or instance method names to execute.  Methods with private attribute
names are automatically mangled as needed to become executable by
the method.  Note that if an item in the input run list is an
instance method, it should be the entire name (not including the
instance name) of the callable method, separated by periods as
appropriate.

Elements in a run list are either strings or 1-element dictionaries.
Consider the following example, where \vars{model} is a \class{Qtcm}
instance, and \mods{run\_list} is called using \vars{mylist} as
input:

\begin{codeblock}
\codeblockfont{%
model = Qtcm(\ldots) \\
mylist = [ \{'varinit':None\}, \\
\hspace*{13ex}'init\_model', \\
\hspace*{13ex}'\_\_qtcm.driver', \\
\hspace*{13ex}\{'set\_qtcm1\_item': ['outdir', '/home/jlin']\} ]
model.run\_list(mylist)}
\end{codeblock}

The first element in \vars{mylist} refers to a method that requires
no positional input parameters be passed in (as shown by the None).
The second and third elements in \vars{mylist} also refers to methods
that require no positional input parameters be passed in.  The last
element in \vars{mylist} refers to a method with two input parameters.
Note that while I use the term ``method'' to describe the elements,
the strings/keys do not have to be only Python instance methods.
The second element, for instance, refers to another run list, and
the third element refers to a compiled QTCM1 model subroutine (note
the \vars{\_\_qtcm} attribute).

When the \mods{run\_list} method is called, the items in the input
run list are called in the order given in the list.  For each
element,  the \mods{run\_list} method first checks if the string
or dictionary key name corresponds to the key of an entry in the
\class{Qtcm} instance attribute \vars{runlists}.  If so, \mods{run\_list}
is called using that run list (i.e., it is a ``recursive'' call).
If the string or dictionary key name does not refer to another run
list, the \mods{run\_list} method checks if the string or dictionary
key name is a method of the \class{Qtcm} instance, and if so the
method is called.  Any other value throws an exception.

If input parameters for a method are of class \class{Field}, the
\mods{run\_list} method first tries to pass the parameters into the
method as is, i.e., as Field object(s).  If that fails, the
\mods{run\_list } method  passes its parameters in as the \vars{value}
attribute of the \class{Field} object.

If you want a variable that is being passed into a run list to be
continuously updated, you have to set the parameter in the run list
to a \class{Field} instance that is a \class{Qtcm} instance attribute,
not just to the value of the field variable (or to a non-\class{Field}
object).  Otherwise, subsequent calls to that run list element will
not use the updated values as input parameters.

For instance, if you had a run list element:
\begin{codeblock}
\codeblockfont{%
\{'\_\_qtcm.timemanager':[model.coupling\_day,]\}}
\end{codeblock}
and \vars{model.coupling\_day} were an integer (it's not by default,
but pretend it was), then \mods{run\_list} calling
\mods{\_\_qtcm.timemanager} will pass in a scalar integer rather
than a binding to the variable \vars{model.coupling\_day}.  In such
a situation, if the variable \vars{model.coupling\_day} were updated
in time, the \mods{run\_list} call of \mods{\_\_qtcm.timemanager}
would not be updated in time.  This happens because when the
dictionary that is the run list element is created, the value of
list element(s) attached to the dictionary element is set to the
scalar value of \vars{model.coupling\_day} at that instant.

You can get around this feature by setting \class{Qtcm} instance
attributes that will change with model execution to \class{Field}
instances, and then referring to those attributes in the parameter
list in the run list element.  In that case:
\begin{codeblock}
\codeblockfont{%
\{'\_\_qtcm.timemanager':[model.coupling\_day,]\}}
\end{codeblock}
will use the current value of \vars{model.coupling\_day} anytime
\vars{\_\_qtcm.timemanager} is called by \mods{run\_list}, if
\vars{model.coupling\_day} is a \class{Field} object.

When \mods{run\_list}, encounters a calling input parameter that
is a \class{Field} object, it will first try to pass the entire
\class{Field} object to the method/routine being called.  If that
raises an exception, it will then try to pass just the value of the
entire \class{Field} object.  This is done to enable \mods{run\_list}
to be used for both pure-Python and compiled QTCM Fortran model
routines.  Fortran cannot handle \class{Field} objects as input
parameters, only values.

Table~\ref{tab:stnd.runlists} shows all standard run lists
stored in the \vars{runlists} attribute upon instantiation
of a \class{Qtcm} instance.

\begin{htmlonly}
\begin{table}[htp]
\begin{center}
\fbox{Empty placeholder block for table that would have gone here.}
\end{center}
\caption{Standard run lists stored in the \vars{runlists} 
	attribute upon instantiation of a \class{Qtcm} instance.
	The run list and list element names are stored as strings.
	\emphpara{This table is improperly reproduced in the
	HTML conversion.  Please see the PDF version for the table.}}
\label{tab:stnd.runlists}
\end{table}
\end{htmlonly}

\begin{latexonly}
\begin{table}[htp]
% This file is automatically generated by
% code_to_latex.py.  It lists all the standard 
% runlists in the class.


\begin{longtable}{l|lc}
\textbf{Run List Name/Description} & \textbf{List Element(s) Name(s)} & 
                                                   \textbf{\# Arg(s)} \\
\hline
\endhead
\multirow{3}{*}{\parbox{0.4\linewidth}{atm\_bartr\_mode (calculate the atmospheric barotropic mode at the barotropic timestep)}} & \_\_qtcm.wrapcall.wsavebartr & None \\
         & \_\_qtcm.wrapcall.wbartr & None \\
         & \_\_qtcm.wrapcall.wgradphis & None \\
\hline
\multirow{5}{*}{\parbox{0.4\linewidth}{atm\_oc\_step (calculate the atmosphere and ocean models at a coupling timestep)}} & \_first\_method\_at\_atm\_oc\_step & None \\
         & \_\_qtcm.wrapcall.wtimemanager & 1\\
         & \_\_qtcm.wrapcall.wocean & 2\\
         & qtcm & None \\
         & \_\_qtcm.wrapcall.woutpall & None \\
\hline
\multirow{5}{*}{\parbox{0.4\linewidth}{atm\_physics1 (calculate atmospheric physics at one instant)}} & \_\_qtcm.wrapcall.wmconvct & None \\
         & \_\_qtcm.wrapcall.wcloud & None \\
         & \_\_qtcm.wrapcall.wradsw & None \\
         & \_\_qtcm.wrapcall.wradlw & None \\
         & \_\_qtcm.wrapcall.wsflux & None \\
\hline
\multirow{8}{*}{\parbox{0.4\linewidth}{atm\_step (calculate the entire atmosphere at one atmosphere timestep)}} & atm\_physics1 & None \\
         & \_\_qtcm.wrapcall.wsland1 & None \\
         & \_\_qtcm.wrapcall.wadvctuv & None \\
         & \_\_qtcm.wrapcall.wadvcttq & None \\
         & \_\_qtcm.wrapcall.wdffus & None \\
         & \_\_qtcm.wrapcall.wbarcl & None \\
         & \_bartropic\_mode\_at\_atm\_step & None \\
         & \_\_qtcm.wrapcall.wvarmean & None \\
\hline
\multirow{3}{*}{\parbox{0.4\linewidth}{init\_model (initialize the entire model, i.e., the atmosphere and ocean components and output)}} & qtcminit & None \\
         & \_\_qtcm.wrapcall.woceaninit & None \\
         & \_\_qtcm.wrapcall.woutpinit & None \\
\hline
\multirow{5}{*}{\parbox{0.4\linewidth}{qtcminit (initialize the atmosphere portion of the entire model)}} & \_\_qtcm.wrapcall.wparinit & None \\
         & \_\_qtcm.wrapcall.wbndinit & None \\
         & varinit & None \\
         & \_\_qtcm.wrapcall.wtimemanager & 1\\
         & atm\_physics1 & None \\
\end{longtable}
\caption{Standard run lists stored in the \vars{runlists} 
	attribute upon instantiation of a \class{Qtcm} instance.
	The run list and list element names are stored as strings.}
\label{tab:stnd.runlists}
\end{table}
\end{latexonly}

Of course, feel free to change the contents of any of the run lists
after instantiation, or to add additional run lists to the
\vars{runlists} attribute dictionary.  The ability to alter run
lists at runtime gives the \mods{qtcm} package much of its flexibility.




%---------------------------------------------------------------------
\section{Field Variables and the \class{Field} Class}
						\label{sec:field.variables}

The term ``field'' variable refers to QTCM1 model variables that 
are accessible at both the compiled Fortran QTCM1 model-level as
well as the Python \class{Qtcm} instance-level.
Field variables are all instances of the \class{Field} class
(though non-field variables can also be instances of \class{Field}).

Section~\ref{sec:field.variables.intro} gives a brief introduction to
the attributes and methods in a \class{Field} instance.
A nitty gritty description of the class is found in its docstrings.

	\subsection{Creating Field Variables}

To create a \class{Field} instance whose value is set to the
default, instantiate with the field id as the only positional
input argument.  Thus:

\begin{codeblock}
\codeblockfont{foo = Field('lastday')}
\end{codeblock}

will return \vars{foo} as a \class{Field} instance with \vars{foo.value}
set to the value listed in Section~\ref{sec:defaults.scalar}.
The value of all \class{Field} instances upon creation are specified
in the \mods{defaults} submodule of package \mods{qtcm}, and listed
in Sections~\ref{sec:defaults.scalar} and~\ref{sec:defaults.array}.

To create \class{Field} instances whose attributes are set different
from their defaults, you can specify the different settings in the
instantiation parameter list, or change the attributes once the
instance is created.  See the \class{Field} docstring for details.


	\subsection{Initial Field Variables}  \label{sec:initial.variables}

Field variables include both model parameters that do not change
for a \class{Qtcm} instance as well as prognostic variables that
do change during model integration.  As a result, many field variables
have values different from the default values listed in
Sections~\ref{sec:defaults.scalar} and~\ref{sec:defaults.array}.
In this section, I list the \emph{initial} values of all field
variables.  The ``initial'' values are the settings for \class{Qtcm}
field variables execution of the \mods{run\_session} method, but
prior to cycling through an atmosphere-ocean coupling timestep.
This is in contrast to ``default'' values, which the field variables
are given on instantiation, if no other value is specified.
Numerical values are rounded as per the conventions
of Python's \vars{\%g} format code.


		\subsubsection{Scalars}

For the fields that give the input/output directory names, and the
run name, the entry ``value varies'' is provided in the ``Value''
column.

% This file is automatically generated by the script
% code_to_latex.py in the doc/latex directory.  It is based
% upon the values found after model initialization, and should
% not be hand-edited if you want the values to correspond to
% the values in a Qtcm instance, for compiled_form='parts'.


\begin{longtable}{l|c|c|p{0.42\linewidth}}
\textbf{Field} & \textbf{Value} & \textbf{Units} & 
                                \textbf{Description} \\
\hline
\endhead
\vars{SSTdir} & value varies &  & Where SST files are \\
\vars{SSTmode} & seasonal &  & Decide what kind of SST to use \\
\vars{VVsmin} & 4.5 & m/s & Minimum wind speed for fluxes \\
\vars{bnddir} & value varies &  & Boundary data other than SST \\
\vars{dateofmodel} & 10101 &  & Date of model coded as an integer as yyyymmdd \\
\vars{day0} & 1 & dy & Starting day; if $<$ 0 use day in restart \\
\vars{dt} & 1200 & s & Time step \\
\vars{eps\_c} & 0.000138889 & 1/s & 1/tau\_c NZ (5.7) \\
\vars{interval} & 1 & dy & Atmosphere-ocean coupling interval \\
\vars{it} & 1 &  & Time of day in time steps \\
\vars{landon} & 1 &  & If not 1: land = ocean with fake SST \\
\vars{lastday} & 0 & dy & Last day of integration \\
\vars{month0} & 1 & mo & Starting month; if $<$ 0 use mo in restart \\
\vars{mrestart} & 0 &  & =1: restart using qtcm.restart \\
\vars{mt0} & 1 &  & Barotropic timestep every mt0 timesteps \\
\vars{nastep} & 1 &  & Number of atmosphere time steps within one air-sea coupling interval \\
\vars{noout} & 0 & dy & No output for the first noout days \\
\vars{nooutr} & 0 & dy & No restart file for the first nooutr days \\
\vars{ntout} & -30 & dy & Monthly mean output \\
\vars{ntouti} & 0 & dy & Monthly instantaneous data output \\
\vars{ntoutr} & 0 & dy & Restart file only at end of model run \\
\vars{outdir} & value varies &  & Where output goes to \\
\vars{runname} & value varies &  & String for an output filename \\
\vars{title} & value varies &  & A descriptive title \\
\vars{u0bar} & 0 &  &  \\
\vars{visc4x} & 700000 & m$^2$/s & Del 4 viscocity parameter in x \\
\vars{visc4y} & 700000 & m$^2$/s & Del 4 viscocity parameter in y \\
\vars{viscxT} & 1.2e+06 & m$^2$/s & Temperature diffusion parameter in x \\
\vars{viscxq} & 1.2e+06 & m$^2$/s & Humidity diffusion parameter in x \\
\vars{viscxu0} & 700000 & m$^2$/s & Viscocity parameter for u0 in x \\
\vars{viscxu1} & 700000 & m$^2$/s & Viscocity parameter for u1 in x \\
\vars{viscyT} & 1.2e+06 & m$^2$/s & Temperature diffusion parameter in y \\
\vars{viscyq} & 1.2e+06 & m$^2$/s & Humidity diffusion parameter in y \\
\vars{viscyu0} & 700000 & m$^2$/s & Viscocity parameter for u0 in y \\
\vars{viscyu1} & 700000 & m$^2$/s & Viscocity parameter for u1 in y \\
\vars{weml} & 0.01 & m/s & Mixed layer entrainment velocity \\
\vars{year0} & 1 & yr & Starting year; if $<$ 0 use year in restart \\
\vars{ziml} & 500 & m & Atmosphere mixed layer depth $\sim$ cloud base \\
\end{longtable}


		\subsubsection{Arrays}

% This file is automatically generated by the script
% code_to_latex.py in the doc/latex directory.  It is based
% upon the values found after model initialization, and should
% not be hand-edited if you want the values to correspond to
% the values in a Qtcm instance, for compiled_form='parts'.


\begin{longtable}{l|c|c|c|c|p{0.3\linewidth}}
\textbf{Field} & \textbf{Shape} & \textbf{Max} & \textbf{Min} &
                                \textbf{Units} & \textbf{Description} \\
\hline
\endhead
\vars{Evap} & (64, 42) & 1502.56 & 223.552 &  &  \\
\vars{FLW} & (64, 42) & 74.5136 & 74.5136 &  &  \\
\vars{FLWds} & (64, 42) & 206.424 & 206.424 &  &  \\
\vars{FLWus} & (64, 42) & 429.708 & 429.708 &  &  \\
\vars{FLWut} & (64, 42) & 148.771 & 148.771 &  &  \\
\vars{FSW} & (64, 42) & 147.767 & 0 &  &  \\
\vars{FSWds} & (64, 42) & 410.895 & -6.99713 &  &  \\
\vars{FSWus} & (64, 42) & 356.831 & -4.49983 &  &  \\
\vars{FSWut} & (64, 42) & 332.431 & 0 &  &  \\
\vars{FTs} & (64, 42) & 930.115 & 138.383 &  &  \\
\vars{Qc} & (64, 42) & 0 & 0 & K & Precipitation \\
\vars{S0} & (64, 42) & 534.264 & 0 &  &  \\
\vars{STYPE} & (64, 42) & 3 & 0 &  & Surface type; ocean or vegetation type over land \\
\vars{T1} & (64, 44) & -100 & -100 & K &  \\
\vars{Ts} & (64, 42) & 295 & 295 & K & Surface temperature \\
\vars{WD} & (64, 42) & 350 & 0 &  &  \\
\vars{WD0} & (4,) & 500 & 0 &  & Field capacity SIB2/CSU (approximately) \\
\vars{arr1} & (64, 42) & 0 & 0 &  & Auxiliary optional output array 1 \\
\vars{arr2} & (64, 42) & 0 & 0 &  & Auxiliary optional output array 2 \\
\vars{arr3} & (64, 42) & 0.138699 & 0.138699 &  & Auxiliary optional output array 3 \\
\vars{arr4} & (64, 42) & 0 & 0 &  & Auxiliary optional output array 4 \\
\vars{arr5} & (64, 42) & 0 & 0 &  & Auxiliary optional output array 5 \\
\vars{arr6} & (64, 42) & 0 & 0 &  & Auxiliary optional output array 6 \\
\vars{arr7} & (64, 42) & 0 & 0 &  & Auxiliary optional output array 7 \\
\vars{arr8} & (64, 42) & 0 & 0 &  & Auxiliary optional output array 8 \\
\vars{psi0} & (64, 43) & 0 & 0 &  &  \\
\vars{q1} & (64, 44) & -50 & -50 & K &  \\
\vars{rhsu0bar} & (3,) & 0 & 0 &  &  \\
\vars{rhsvort0} & (64, 42, 3) & 0 & 0 &  &  \\
\vars{taux} & (64, 42) & 0 & 0 &  &  \\
\vars{tauy} & (64, 42) & 0 & 0 &  &  \\
\vars{u0} & (64, 44) & 0 & 0 & m/s & Barotropic zonal wind \\
\vars{u1} & (64, 44) & 0 & 0 & m/s & Current time step baroclinic zonal wind \\
\vars{v0} & (64, 43) & 0 & 0 & m/s & Barotropic meridional wind \\
\vars{v1} & (64, 43) & 0 & 0 & m/s &  \\
\vars{vort0} & (64, 42) & 0 & 0 &  &  \\
\end{longtable}



	\subsection{Passing Fields Between the Python and Fortran-Levels}

Section~\ref{sec:comm.py.fort.compiledform} discusses the differences
between how the \vars{'full'} and \vars{'parts'} compiled forms
pass field variables between the Python and Fortran-levels.  That
discussion gives a detailed description of the methods used for
passing fields to and from the Python and Fortran-levels (i.e., the
\mods{get\_qtcm1\_item} and \mods{set\_qtcm1\_item} methods).

Please note the following regarding field variables as you pass them 
back and forth between the Python and Fortran-levels:
\begin{itemize}
\item Field variables with ghost latitudes, such as \vars{u1}, on
	the Python end are always the full variables (i.e., including
	the ghost latitudes).  On the Fortran end, variables like
	\vars{u1} also always have the ghost latitudes while in the
	model, but when stored as restart files, do not have the
	ghost latitudes; the end points are not saved in restart
	files or written to the netCDF output files.
	See the
	\latexhtml{%
\htmladdnormallinkfoot{QTCM1 manual}%
        {http://www.atmos.ucla.edu/$\sim$csi/qtcm\_man/v2.3/qtcm\_manv2.3.pdf}}%
{\htmladdnormallink{QTCM1 manual}%
        {http://www.atmos.ucla.edu/~csi/qtcm_man/v2.3/qtcm_manv2.3.pdf}}
	\cite{Neelin/etal:2002}
	for details about ghost latitudes.

\item You should assume there is only a full synchronizing between 
	compiled QTCM1 model and Python model field variables
	at the beginning and end of a run session.  

\item If you have a variable at the Python-level, but at the
	compiled QTCM1 Fortran model-level the variable is not
	readable, if you try to call \mods{set\_qtcm1\_item} on the
	variable, nothing is done, and the Python-level value is
	left alone.  If you have a compiled QTCM1 model variable,
	but no Python-level equivalent, if you call \mods{set\_qtcm1\_item}
	on the variable, the Python-level variable (as an attribute)
	is created.

\item To be precise, only compiled QTCM1 model variables can be
	passed pass back and forth between the Python and Fortran-levels;
	there are many \class{Qtcm} instance attributes that do not
	have any counterparts at the Fortran-level.\footnote%
		{I use the term ``field variables'' to refer to 
		compiled QTCM1 model variables that can be passed
		back and forth to the Python level.}

\item Although \vars{dayofmodel} is described in module \mods{setbypy}
	as an option for the \mods{get\_qtcm1\_item} and
	\mods{set\_qtcm1\_item} methods to operate on, in reality
	those methods cannot operate on \vars{dayofmodel}, but
	\vars{dayofmodel} is not defined in \mods{defaults}.\footnote%
		{All field variables must be defined in \mods{defaults} in
		order for the proper Fortran routine to be called
		according to the variable's type.}
\end{itemize}


	\subsection{Field Variable Shape}   \label{sec:field.var.shape}

Normally, Python arrays have a different dimension order than Fortran
arrays.  While Fortran arrays are dimensioned (col, row, slice),
with adjacent columns being contiguous, then rows, and then slices, Python
arrays are dimensioned (slice, row, col), with adjacent columns being
contiguous, then rows, and then slices.  Based on this, you would
think that everytime you passed an array between the Python and
Fortran-levels you would need to transpose the array.

Thankfully, we don't have to do this because \mods{f2py} handles
array dimension order transparently so we can refer to each element
the same way whether we're in Python or Fortran.  Thus, the array
\vars{Qc} in Fortran is dimensioned (longitude, latitude), (64,42)
by default, and the Python \class{Qtcm} instance attribute \vars{Qc}
has a \vars{value} attribute also dimensioned (longitude, latitude),
(64,42) by default.  And at both the Fortran and Python-levels, the
first longtude, second latitude element is referred to as \vars{Qc(1,2)}.

In contrast, however, netCDF output saved by the compiled QTCM1 model
and read into Python (using the \mods{Scientific} package) is
\emph{not} in Fortran array order.  Arrays read from netCDF output
into Python are in Python array order, and are dimensioned
(latitude, longitude) or (time, latitude, longitude).  The \class{Qtcm}
routines that manipulate netCDF data (e.g., \mods{plotm}), however,
automatically adjust for this, so you only need to be aware of this
when reading in output for your own analysis
(see Section~\ref{sec:model.output}).




%---------------------------------------------------------------------
\section{Model Output}			\label{sec:model.output}

Section~\ref{sec:output.intro} gives an overview of how to
use \mods{qtcm} model output to netCDF files.

All netCDF array output is dimensioned (time, latitude, longitude)
when read into Python using the \mods{Scientific} package.  This
differs from the way \class{Qtcm} saves field variables, which
follows Fortran convention (longitude, latitude).  Thus, the shapes
in Section~\ref{sec:initial.variables}, Appendix~\ref{app:defaults.values},
etc., are not the shapes of arrays read from the netCDF output.
See Section~\ref{sec:field.var.shape} for a discussion of why
there is this discrepancy.

Because netCDF files allow you to specify an ``unlimited'' dimension,
it is possible to close a netCDF file, reopen it, and add more
slices of data to the file.  Thus, continuous \class{Qtcm} run
sessions (i.e., those that use the \vars{cont} keyword input parameter
in the \mods{run\_session} method) will automatically append output
to the netCDF output files.

Field variables with ghost latitudes, such as \vars{u1}, on the
Python and Fortran ends are always the full variables (i.e., including
the ghost latitudes).  The ghost latitudes are not written to the
netCDF output files, however.
See the \latexhtml{%
\htmladdnormallinkfoot{QTCM1 manual}%
        {http://www.atmos.ucla.edu/$\sim$csi/qtcm\_man/v2.3/qtcm\_manv2.3.pdf}}%
{\htmladdnormallink{QTCM1 manual}%
        {http://www.atmos.ucla.edu/~csi/qtcm_man/v2.3/qtcm_manv2.3.pdf}}
	\cite{Neelin/etal:2002}
for details about ghost latitude structure.

\class{Qtcm} instances have a few built-in tools to visualization
model output.  These are briefly described in Section~\ref{sec:viz.intro}.
Note that the \mods{plotm} method is linked to a specific \class{Qtcm}
instance.  Do not use \mods{plotm} outside of the instance it is
linked to.  It must also be used only after a successful run session
(i.e., not in the middle of a run session).




%---------------------------------------------------------------------
\section{Miscellaneous}

A few miscellaneous items/issues about the model:
\begin{itemize}
\item The land model runs at same timestep as the atmosphere.

\item If the land model runs less often than 
	\mods{sflux} in \mods{physics1}, 
	the calculation of evaporation over the land 
	needs to be fixed in sflux.

\item The units of some field variables are not what you would expect.
	For instance, \vars{Qc} is in energy units, i.e., K, and not
	mm/day.
	See the
	\latexhtml{%
\htmladdnormallinkfoot{QTCM1 manual}%
        {http://www.atmos.ucla.edu/$\sim$csi/qtcm\_man/v2.3/qtcm\_manv2.3.pdf}}%
{\htmladdnormallink{QTCM1 manual}%
        {http://www.atmos.ucla.edu/~csi/qtcm_man/v2.3/qtcm_manv2.3.pdf}}
	\cite{Neelin/etal:2002}
	for details.
\end{itemize}




%---------------------------------------------------------------------
\section{Cookbook of Ways the Model Can Be Used}  \label{sec:cookbook}

This cookbook of a few ways to use the model is arranged by science
tasks, i.e., certain types of runs we want to do.  For some of the
examples below, I assume that the dictionary
\vars{inputs} is initially defined as given in
Figure~\ref{fig:defn.of.inputs}.  All examples assume that
\cmd{from qtcm import Qtcm} has already been executed.


%--- Two versions, one for PDF and the other for HTML:
\begin{latexonly}
\begin{figure}[tp]
\begin{codeblock}
\codeblockfont{%
inputs = \{\} \\
inputs['runname'] = 'test' \\
inputs['landon'] = 0 \\
inputs['year0'] = 1 \\
inputs['month0'] = 11 \\
inputs['day0'] = 1 \\
inputs['lastday'] = 30 \\
inputs['mrestart'] = 0 \\
inputs['init\_with\_instance\_state'] = True \\
inputs['compiled\_form'] = 'parts'}
\end{codeblock}

\caption{The initial definition of the \vars{inputs} dictionary for 
	examples given in Section~\ref{sec:cookbook}.  These settings
	imply that a run session will start on November 1, Year 1,
	last for 30 days, and will be an aquaplanet run.}
\label{fig:defn.of.inputs}
\end{figure}
\end{latexonly}

\begin{htmlonly}
\label{fig:defn.of.inputs}
\begin{center}
\htmlfigcaption{%
	\codeblockfont{%
inputs = \{\} \\
inputs['runname'] = 'test' \\
inputs['landon'] = 0 \\
inputs['year0'] = 1 \\
inputs['month0'] = 11 \\
inputs['day0'] = 1 \\
inputs['lastday'] = 30 \\
inputs['mrestart'] = 0 \\
inputs['init\_with\_instance\_state'] = True \\
inputs['compiled\_form'] = 'parts'}
	}

\htmlfigcaption{Figure~\ref{fig:defn.of.inputs}:
	The initial definition of the \vars{inputs} dictionary for 
	examples given in Section~\ref{sec:cookbook}.  These settings
	imply that a run session will start on November 1, Year 1,
	last for 30 days, and will be an aquaplanet run.}
\end{center}
\end{htmlonly}



\begin{description}
\item[Plain model run:]
	Here I just want to make a single model run.
	Tasks:  Instantiate a fresh model and execute a run session.
	The code to run the model is just:
	\begin{codeblock}
	\codeblockfont{%
inputs['init\_with\_instance\_state'] = False \\
model = Qtcm(**inputs) \\
model.run\_session()}
	\end{codeblock}
	where \vars{inputs} is initialized with the code in
	Figure~\ref{fig:defn.of.inputs}.


\item[Explore parameter space with a set of models:]
	Here I want to create an entire suite of separate models,
	in order to determine the sensitivity of the model to changing
	a parameter.
	To do this, I
	instantiate multiple fresh models, 
	and execute a run session for each instance, all within
	a \vars{for} loop:


%--- Two versions, because LaTeX2HTML does not correctly typeset
%    the hspace command:
\begin{latexonly}
	\begin{codeblock}
	\codeblockfont{%
import os \\
inputs['init\_with\_instance\_state'] = False \\
for i in xrange(0,1002,10): \\
\hspace*{5ex}iname = 'ziml-' + str(i) + 'm' \\
\hspace*{5ex}ipath = os.path.join('proc', iname) \\
\hspace*{5ex}os.makedirs(ipath) \\
\hspace*{5ex}model = Qtcm(**inputs) \\
\hspace*{5ex}model.ziml.value = float(i)  \\
\hspace*{5ex}model.runname.value = iname \\
\hspace*{5ex}model.outdir.value = ipath \\
\hspace*{5ex}model.run\_session() \\
\hspace*{5ex}del model}
	\end{codeblock}
\end{latexonly}

\begin{htmlonly}
\begin{center}
\htmlfigcaption{%
	\codeblockfont{%
import os \\
inputs['init\_with\_instance\_state'] = False \\
for i in xrange(0,1002,10): \\
\hspace*{5ex}iname = 'ziml-' + str(i) + 'm' \\
\hspace*{5ex}ipath = os.path.join('proc', iname) \\
\hspace*{5ex}os.makedirs(ipath) \\
\hspace*{5ex}model = Qtcm(**inputs) \\
\hspace*{5ex}model.ziml.value = float(i)  \\
\hspace*{5ex}model.runname.value = iname \\
\hspace*{5ex}model.outdir.value = ipath \\
\hspace*{5ex}model.run\_session() \\
\hspace*{5ex}del model}
	}
\end{center}
\end{htmlonly}


	The loop explores mixed-layer depth \vars{ziml} from 0~m to
        1000~m, in 10~m intervals.  I create the \vars{outdir}
	directory before every model call, since the compiled QTCM1 model
	requires the output directory exist, specifying the run name
	and output directory as the string \vars{iname}.
	The output directories are assumed to all be in the \fn{proc}
	sub-directory of the current working directory.
	\vars{inputs} is initialized with the code in
	Figure~\ref{fig:defn.of.inputs}.


\item[Conditionally explore parameter space:]
	Here I want to 
	conditionally explore the parameter space, on the basis of
	some mathematical criteria.
	To do this, I
	instantiate a model, evaluate results using
	that criteria, and run another fresh model depending on
	the results (passing the previous model state via a snapshot),
	all within a \vars{while} loop.
	Note that this type of investigation is very difficult to 
	automate if all you can use are shell scripts and
	Fortran.
	See Figure~\ref{fig:conditional.test.eg} for a detailed
	example.


\item[With interactive adjustments at run time:]
	The example in Figure~\ref{sec:continuation.intro}
	illustrates this type of run.  In this example,
	I instantiate a fresh model, execute a run session, analyze the
	output, change variables in the model instance, and then
	execute a continuation run session.


\item[Test alternative parameterizations:]
	I've already described how we can use run lists to arbitrarily
	change model execution order and content at run time.
	We can take advantage of Python's inheritance
	abilities, along with run lists, to simplify this.
	Figure~\ref{fig:alt.param.inherit.eg} provides an example of
	this use.

	Of course, you can use pre-processor directives and shell
	scripts to accomplish the same functionality seen in
	Figure~\ref{fig:alt.param.inherit.eg} using just Fortran.
	The Python solution, however, shortcuts the compile/linking
	step, and enables you to easily do run time swapping between
	subroutine choices based upon run time calculated
	tests (see Figure~\ref{fig:conditional.test.eg} for an
	example of such tests).
\end{description}




% --- Two versions of this block, one for display in PDF and the other
%     for display in HTML:
\begin{latexonly}
\begin{figure}[p]
	\begin{codeblock}
	\codeblockfont{%
\small
import os \\
import numpy as N \\
maxu1 = 0.0 \\
while maxu1 < 10.0: \\
\hspace*{5ex}iziml = 0.1 * maxu1 \\
\hspace*{5ex}iname = 'ziml-' + str(iziml) + 'm' \\
\hspace*{5ex}ipath = os.path.join('proc', iname) \\
\hspace*{5ex}os.makedirs(ipath) \\
\hspace*{5ex}model = Qtcm(**inputs) \\
\hspace*{5ex}try: \\
\hspace*{10ex}model.sync\_set\_py\_values\_to\_snapshot(snapshot=mysnapshot) \\
\hspace*{10ex}model.init\_with\_instance\_state = True \\
\hspace*{5ex}except: \\
\hspace*{10ex}model.init\_with\_instance\_state = False \\
\hspace*{5ex}model.ziml.value = iziml  \\
\hspace*{5ex}model.runname.value = iname \\
\hspace*{5ex}model.outdir.value = ipath \\
\hspace*{5ex}model.run\_session() \\
\hspace*{5ex}maxu1 = N.max(N.abs(model.u1.value)) \\
\hspace*{5ex}mysnapshot = model.snapshot \\
\hspace*{5ex}del model}
	\end{codeblock}

\caption{This code explores different values of
	mixed-layer depth \vars{ziml} for 30~day runs,
	as a function of maximum \vars{u1} magnitude,
	until it finds a case where the maximum \vars{u1} is
	greater than 10~m/s.  (The relationship between
	\vars{ziml} and the maximum of the speed of
	\vars{u1}, where 
	\vars{ziml\thinspace=\thinspace0.1\thinspace*\thinspace{maxu1}}, 
	is made up.)
	With each iteration, the new run uses the snapshot from
	a previous run to initialize (as well as the new value
	of \vars{ziml}); the \vars{try} statement is used to
	ensure the model works even if \vars{mysnapshot} is not
	defined (which is the case the first time around).
	The \vars{inputs} dictionary is initialized with the code in
	Figure~\ref{fig:defn.of.inputs}.}
\label{fig:conditional.test.eg}
\end{figure}
\end{latexonly}

\begin{htmlonly}
\label{fig:conditional.test.eg}
\begin{center}
\htmlfigcaption{%
	\codeblockfont{%
import os \\
import numpy as N \\
maxu1 = 0.0 \\
while maxu1 < 10.0: \\
\hspace*{5ex}iziml = 0.1 * maxu1 \\
\hspace*{5ex}iname = 'ziml-' + str(iziml) + 'm' \\
\hspace*{5ex}ipath = os.path.join('proc', iname) \\
\hspace*{5ex}os.makedirs(ipath) \\
\hspace*{5ex}model = Qtcm(**inputs) \\
\hspace*{5ex}try: \\
\hspace*{10ex}model.sync\_set\_py\_values\_to\_snapshot(snapshot=mysnapshot) \\
\hspace*{10ex}model.init\_with\_instance\_state = True \\
\hspace*{5ex}except: \\
\hspace*{10ex}model.init\_with\_instance\_state = False \\
\hspace*{5ex}model.ziml.value = iziml  \\
\hspace*{5ex}model.runname.value = iname \\
\hspace*{5ex}model.outdir.value = ipath \\
\hspace*{5ex}model.run\_session() \\
\hspace*{5ex}maxu1 = N.max(N.abs(model.u1.value)) \\
\hspace*{5ex}mysnapshot = model.snapshot \\
\hspace*{5ex}del model}
	}

\htmlfigcaption{Figure \ref{fig:conditional.test.eg}:
	This code explores different values of
	mixed-layer depth \vars{ziml} for 30~day runs,
	as a function of maximum \vars{u1} magnitude,
	until it finds a case where the maximum \vars{u1} is
	greater than 10~m/s.  (The relationship between
	\vars{ziml} and the maximum of the speed of
	\vars{u1}, where 
	\vars{ziml\thinspace=\thinspace0.1\thinspace*\thinspace{maxu1}}, 
	is made up.)
	With each iteration, the new run uses the snapshot from
	a previous run to initialize (as well as the new value
	of \vars{ziml}); the \vars{try} statement is used to
	ensure the model works even if \vars{mysnapshot} is not
	defined (which is the case the first time around).
	The \vars{inputs} dictionary is initialized with the code in
	Figure~\ref{fig:defn.of.inputs}.}
\end{center}
\end{htmlonly}


% --- Two versions of this block, one for display in PDF and the other
%     for display in HTML:
\begin{latexonly}
\begin{figure}[p]
\begin{center}
	\begin{codeblock}
	\codeblockfont{%
\small
import os \\
\\
class NewQtcm(Qtcm): \\
\hspace*{5ex}def cloud0(self):\\
\hspace*{10ex}[\ldots] \\
\hspace*{5ex}def cloud1(self):\\
\hspace*{10ex}[\ldots] \\
\hspace*{5ex}def cloud2(self):\\
\hspace*{10ex}[\ldots] \\
\hspace*{5ex}[\ldots] \\
\\
inputs['init\_with\_instance\_state'] = False \\
for i in xrange(10): \\
\hspace*{5ex}iname = 'cloudroutine-' + str(i)  \\
\hspace*{5ex}ipath = os.path.join('proc', iname) \\
\hspace*{5ex}os.makedirs(ipath) \\
\hspace*{5ex}model = NewQtcm(**inputs) \\
\hspace*{5ex}model.runlists['atm\_physics1'][1] = 'cloud' + str(i) \\
\hspace*{5ex}model.runname.value = iname \\
\hspace*{5ex}model.outdir.value = ipath \\
\hspace*{5ex}model.run\_session() \\
\hspace*{5ex}del model}
	\end{codeblock}
\end{center}

\caption{Let's say we have 9 different cloud physics schemes we wish
	to try out in 9 different runs.  The easiest way to do this
	is to create a new class \class{NewQtcm} that
	inherits everything from \class{Qtcm}, and to which we'll
	add the additional cloud schemes (\vars{cloud0}, \vars{cloud1},
	etc.).
	In the \vars{for} loop, I change the cloud model
	run list entry in the run list that governs
	atmospheric physics at one instant to whatever the cloud
	model is at this point in the loop.
	The \vars{inputs} dictionary is initialized with the code in
	Figure~\ref{fig:defn.of.inputs}.
	Of course, we could do the same thing by running the 9
	models separately, but this set-up makes it easy to do
	hypothesis testing with these 9 models.  For instance, we
	can create a test by which we will choose which of the 9
	models to use:  Within this framework, the selection of
	those models can be altered by changing a string.}
\label{fig:alt.param.inherit.eg}
\end{figure}
\end{latexonly}

\begin{htmlonly}
\label{fig:alt.param.inherit.eg}
\begin{center}
\htmlfigcaption{%
	\codeblockfont{%
import os \\
\\
class NewQtcm(Qtcm): \\
\hspace*{5ex}def cloud0(self):\\
\hspace*{10ex}[\ldots] \\
\hspace*{5ex}def cloud1(self):\\
\hspace*{10ex}[\ldots] \\
\hspace*{5ex}def cloud2(self):\\
\hspace*{10ex}[\ldots] \\
\hspace*{5ex}[\ldots] \\
\\
inputs['init\_with\_instance\_state'] = False \\
for i in xrange(10): \\
\hspace*{5ex}iname = 'cloudroutine-' + str(i)  \\
\hspace*{5ex}ipath = os.path.join('proc', iname) \\
\hspace*{5ex}os.makedirs(ipath) \\
\hspace*{5ex}model = NewQtcm(**inputs) \\
\hspace*{5ex}model.runlists['atm\_physics1'][1] = 'cloud' + str(i) \\
\hspace*{5ex}model.runname.value = iname \\
\hspace*{5ex}model.outdir.value = ipath \\
\hspace*{5ex}model.run\_session() \\
\hspace*{5ex}del model}
	}

\htmlfigcaption{Figure \ref{fig:alt.param.inherit.eg}:
	Let's say we have 9 different cloud physics schemes we wish
	to try out in 9 different runs.  The easiest way to do this
	is to create a new class \class{NewQtcm} that
	inherits everything from \class{Qtcm}, and to which we'll
	add the additional cloud schemes (\vars{cloud0}, \vars{cloud1},
	etc.).
	In the \vars{for} loop, I change the cloud model
	run list entry in the run list that governs
	atmospheric physics at one instant to whatever the cloud
	model is at this point in the loop.
	The \vars{inputs} dictionary is initialized with the code in
	Figure~\ref{fig:defn.of.inputs}.
	Of course, we could do the same thing by running the 9
	models separately, but this set-up makes it easy to do
	hypothesis testing with these 9 models.  For instance, we
	can create a test by which we will choose which of the 9
	models to use:  Within this framework, the selection of
	those models can be altered by changing a string.}
\end{center}
\end{htmlonly}




% ===== end of file =====


%@@@\chapter{Combining \code{qtcm} with \code{CliMT}}
%@@@% ==========================================================================
% CliMT
%
% By Johnny Lin
% ==========================================================================


% ------ BODY -----
%
\section{General Tutorial on CliMT}


General notes of things I think I may have observed about
\code{Parameters} objects:
\begin{itemize}
\item You can treat a \code{Parameters} instance as a dictionary, where
	the key is the name of the field, because \code{\_\_getitem\_\_},
	etc.\ have been defined for the instance.  However, the values,
	units, and long names of the fields are stored in dictionaries
	assigned to \code{value}, \code{units}, and \code{long\_name},
	keyed to the field name (a string).
\end{itemize}


General notes of things I think I may have observed about
\code{Components} objects:
\begin{itemize}
\item All variables and quantities, whether they be physical fields,
	filenames, or metadata,
	are stored as attributes in the \code{Components} instance.
\item \code{Components} have these special attributes:
        \code{Required},
        \code{Prognostic},
	and
        \code{Diagnostic},
	which are lists that contain the names of describe whether
\item Scalar parameters in \code{Component} objects
	are stored as an instance of the \code{Parameters}
	class, under the attribute \code{Params}.
\end{itemize}


General notes of things I think I may have observed about
\code{Federation} objects:
\begin{itemize}
\item \code{Federation} objects hold the \code{Components} instances
	in a list assigned to the attribute \code{list}.
\item \code{Federation} attributes
        \code{Required},
        and
	\code{Prognostic},
	are unions of the same attributes of the constituent
	\code{Components} objects.
\end{itemize}






% ===== end of file =====


\chapter{Troubleshooting}                   \label{ch:trouble}
% ==========================================================================
% Troubleshooting
%
% By Johnny Lin
% ==========================================================================


% ------ BODY -----
%
\section{Error Messages Produced by \mods{qtcm}}

\begin{description}
\item[\screen{Error-Value too long in SetbyPy module getitem\_str for}
	\dumarg{key}:]
	This message is produced by the Fortran
	subroutine \mods{getitem\_str}
	in the module \mods{SetbyPy} in the compiled QTCM1 Fortran code.
	The code is in the file \fn{setbypy.F90}.  This error occurs when
	the Fortran variable whose name is given by the string \dumarg{key}
	has a value that is greater than the local parameter
	\vars{maxitemlen} in \mods{getitem\_str}.  To fix this, you have
	to go into \fn{setbypy.F90} and change the value of
	\vars{maxitemlen}.

\item[\screen{Error-real\_rank1\_array should be deallocated}:]
	Fortran module \mods{SetByPy}'s subroutine
	\mods{getitem\_real\_array} generates this message
	(or a similar message for other ranks) if the Fortran
	variable for the input \dumarg{key} are allocated on entry
	to the routine.  This may indicate the user has written another
	Fortran routine to access the \mods{real\_rank1\_array} variable
	outside of the standard interfaces..

\item[\screen{Error-Bad call to SetbyPy module \ldots}:]
	Often times, this error occurs because a get or set routine
	in \mods{SetByPy} tried to act on a variable for which the
	corresponding input \dumarg{key} is not defined.  The solution
	is to add that case in the if/then construct for the get and set
	routines in \mods{SetByPy} and rebuild the extension modules.
\end{description}


\section{Other Errors}

\begin{description}
\item[Python cannot find some packages:]
	This error often happens when the version of Python in which
	you have installed all your packages is not the version that
	is called at the Unix command line by typing in \cmd{python}.
	To get around this, 
        define a Unix alias
        that maps \cmd{python2.4} (or whichever version of Python
	has all your packages installed) to \cmd{python}.  If you
	have multiple Python's installed on your system, you might
	have to use a more specific name for the Python executable.
	As a result, you may have to change the test scripts in
	\fn{test} in the \mods{qtcm} distribution directory.

\item[\mods{get\_qtcm1\_item} and compiled QTCM1 model pointer
	variables:]
	If you try to use the \mods{get\_qtcm1\_item} method on a compiled
	QTCM1 model pointer variable 
	(i.e., \vars{u1}, \vars{v1}, \vars{q1}, \vars{T1}),
	 before the compiled
	model \mods{varinit} subroutine is run, you'll get a bus error
	with no additional message.

\item[Mismatch between Python and Fortran array field variables:]
	You change an array field variable on the Python side, but
	it seems like the wrong elements are changed on the Fortran
	side.  Or you type in the same index address for accessing a
	\mods{qtcm} netCDF output array as well as its \class{Qtcm}
	instance attribute counterpart, and find you get different
	answers.  Some possible reasons and fixes:

	\begin{itemize}
	\item This will occur if you haven't accounted for the
		difference in how field variables are saved at the
		Python-level, Fortran-level, and in a netCDF file.
		All netCDF array output is dimensioned (time,
		latitude, longitude) when read into Python using
		the \mods{Scientific} package.  This differs from
		the way \class{Qtcm} saves field variables, \emph{both}
		at the Python- and Fortran-levels, which follows
		Fortran convention (longitude, latitude).

		Note that the way \class{Qtcm} saves field variables
		at the Python- and Fortran-levels is different than
		the default way Python and Fortran save arrays.
		Section~\ref{sec:field.var.shape} for more information.

	\item You may have forgotten that array indices in Python start at
		0, while indices in Fortran (generally) start at 1.
		Also, ranges in Python are exclusive at the upper-bound,
		while ranges in Fortran are inclusive at the upper-bound.
		(Both Python and Fortran array indice ranges are inclusive
		at the lower-bound.)

	\item You may have forgotten some field variables have
		ghost latitudes, and thus there are extra latitude bands
		when the array is stored as a Python or Fortran field
		variable, but there are \emph{no} extra latitude bands
		when the array is stored as netCDF output (the QTCM1
		output routines strip off the ghost latitudes when
		writing those field variables out).
	        See the
        \latexhtml{%
\htmladdnormallinkfoot{QTCM1 manual}%
        {http://www.atmos.ucla.edu/$\sim$csi/qtcm\_man/v2.3/qtcm\_manv2.3.pdf}}%
{\htmladdnormallink{QTCM1 manual}%
        {http://www.atmos.ucla.edu/~csi/qtcm_man/v2.3/qtcm_manv2.3.pdf}}
        \cite{Neelin/etal:2002}
        for details about ghost latitudes.

		The safest and easiest way to tell whether the variable has a
		ghost latitudes is to look at its shape.
		A call to the \class{Qtcm} instance
		method \mods{get\_qtcm1\_item} will give you the array,
		and the use of NumPy's \mods{shape} function will give you
		the shape.
	\end{itemize}
\end{description}




% ===== end of file =====


\chapter{Developer Notes}                   \label{ch:devnotes}
% ==========================================================================
% Using QTCM
%
% By Johnny Lin
% ==========================================================================


% ------ BODY -----
%

%---------------------------------------------------------------------
\section{Introduction}

This section describes programming practices and issues related to
the \mods{qtcm} package that might be of interest to users wishing
to add/change code in the package.
Please see the package
\latexhtml{API documentation,%
		\footnote{http://www.johnny-lin.com/py\_pkgs/qtcm/doc/html-api/}
		which includes the source code}%
        {\htmladdnormallink{API documentation}%
		{http://www.johnny-lin.com/py\_pkgs/qtcm/doc/html-api/},
		which includes the source code},
for details.




%---------------------------------------------------------------------
\section{Changes to QTCM1 Fortran Files}  \label{sec:f90changes}

The source code used to generate the shared object files used
in this Python package is unchanged
from the pure-Fortran QTCM1 model source code, except in the
following ways:

\begin{itemize}
\item The suffix of all source code files 
	has been changed from \fn{.f90} to \fn{.F90}, 
	in order to ensure the compiler preprocesses 
	the source code.  Some compilers use the capitalization to
	tell whether or not to run the code through a preprocessor.

\item In all \fn{.F90} files, occurrences of:
	\begin{codeblock}
	\codeblockfont{%
	Character(len=130)}
	\end{codeblock}
	are changed to:
	\begin{codeblock}
	\codeblockfont{%
	Character(len=305)}
	\end{codeblock}
	This enables the model to properly deal with longer filenames.
	The number ``305'' is chosen to make search and replace easier.

\item In \fn{qtcmpar.F90}, the 
	\vars{eps\_c} variable is changed from an unchangable
	parameter to a changeable real, 
	so that it can be changed in the model at runtime.

\item All occurrences of an underscore (``\_'') character in a
	subroutine or function name are removed.  The
	presence of the underscore messes up the dynamic lookup
	mechanism for the \mods{f2py} generated extension module.
	The following names are changed, both in subroutine definitions
	and calls:
	\begin{itemize}
	\item \mods{out\_restart} to \mods{outrestart},
	\item \mods{save\_bartr} to \mods{savebartr},
	\item \mods{grad\_phis} to \mods{gradphis}.
	\end{itemize}

\item \fn{driver.F90} is changed so that program
	\mods{driver} becomes a subroutine, and 
	subroutine \mods{driverinit} is deleted (along with
	all calls to it) because basic model initialization is
	handled at the Python level.

\item In \fn{clrad.F90}, subroutine \mods{cloud}, the first
	\vars{COUNTCAP} preprocessor macro, a comment line for
	that ifdef is moved to prevent a warning message during
	building with \mods{f2py}.

\item The order of subroutine \mods{qtcminit} is changed.  The original
	pure-Fortran QTCM1 \mods{qtcminit} code has the following
	calling sequence:

	\begin{codeblock}
        \codeblockfont{%
Call parinit            !Initialize model parameters \\
Call varinit            !Initialize variables \\
Call TimeManager(1)     !mm set model time \\
Call bndinit            !input boundary datasets \\
Call physics1           !diagnostic fields for initial condition}
	\end{codeblock}

	For the \mods{qtcm} package, I've altered this order so
	\mods{bndinit} comes after \mods{parinit} but before \mods{varinit}:
	\begin{codeblock}
        \codeblockfont{%
Call parinit            !Initialize model parameters \\
Call bndinit            !input boundary datasets \\
Call varinit            !Initialize variables \\
Call TimeManager(1)     !mm set model time  \\
Call physics1           !diagnostic fields for initial condition}
	\end{codeblock}

	This is done because \vars{STYPE} is not read in for the
	\vars{landon} \vars{True} case until \mods{bndinit}, but
	in \mods{varinit} \vars{STYPE} is used to calculate the
	original values of \vars{WD} for the non-restart case.  This
	also corrects the conflicting order found in the pure-Fortran
	QTCM1 manual (compare pp.\ 29 and 32).  As far as I can
	tell, \mods{bndinit} has no dependencies that require it
	to come after \mods{timemanager} or \mods{varinit}.

\end{itemize}

In addition, the Fortran files \fn{setbypy.F90}, \fn{wrapcall.F90},
and \fn{varptrinit.F90} are added.  The routines in these files, 
however, just add more flexibility and functionality to the model;
they do not automatically affect any model computations.  See
Section~\ref{sec:newf90} for details.




%---------------------------------------------------------------------
\section{New Interfaces and Fortran Functionality}  \label{sec:newf90}

As described in Section~\ref{sec:f90changes}, the Fortran files
\fn{setbypy.F90}, \fn{wrapcall.F90}, and \fn{varptrinit.F90} are
added to the QTCM1 source directory.  The first two files define the Fortran
90 modules (\mods{SetbyPy} and \mods{WrapCall}) needed to interface
the Python and Fortran levels.  The last file defines a new Fortran
subroutine \mods{varptrinit} that associates QTCM1 model pointer
variables at the Fortran level.  In a pure-Fortran run of QTCM1,
this occurs in subroutine \mods{varinit}; for a
\vars{compiled\_form\thinspace=\thinspace'parts'} run, since the
functionality of the Fortran \mods{varinit} is now in the Python
\mods{varinit} method, a separate Fortran pointer association
subroutine needed to be defined.  The Fortran subroutine \mods{varptrinit}
is called as the \mods{varptrinit} function of the 
\vars{compiled\_form\thinspace=\thinspace'parts'}
\fn{.so} extension module.


	\subsection{Fortran Module \mods{SetbyPy}}   \label{sec:setbypy}

		\subsubsection{Design Description}

This module defines functions and subroutines used to read variables
from the Fortran-level to the Python-level, and in setting Fortran-level
variables using the Python-level values.  These routines are used
by \class{Qtcm} methods \mods{get\_qtcm1\_item} and \mods{set\_qtcm1\_item}
(and dependencies thereof) to ``get'' and ``set'' the Fortran-level
variables.  Note that the Fortran module \mods{SetbyPy} is referred
to in lowercase at the Python level, i.e., as the
attribute \vars{\_\_.qtcm.setbypy} of a \class{Qtcm} instance.

Because Fortran variables are not dynamically typed, separate Fortran
functions and subroutines need to be defined to get and set variables
of different types.\footnote%
	{The \mods{interface} construct in Fortran 90 is supposed to
	provide a way to create a single interface to multiple
	routines, e.g.:
	\begin{codeblock}
	\codeblockfont{%
Interface setitem \\
\hspace*{3ex}Module Procedure setitem\_real, setitem\_int, setitem\_str \\
End Interface}
	\end{codeblock}
	This construct, however, causes a bus error
	(Mac OS X 10.4, Intel).  Thus, I put the
	same functionality in the Python code.}
The \class{Qtcm} methods \mods{get\_qtcm1\_item}
and \mods{set\_qtcm1\_item} know which one of the Fortran routines
to call on the basis of the type and rank of the value for the field
variable in the \mods{defaults} submodule.  This is why all field
variables need to have defaults defined in \mods{defaults}.  For
array variables, the field variable defaults also provide the rank
of the Fortran-level variable being gotten or set.  However, the
array default values do \emph{not} have to have the same shape as
the Fortran-level variables; on the Python-side, variable shape
adjusts to what is declared on the Fortran-side.  
Thus, if you change the resolution of
the compiled QTCM1 model, you do not have to change the dimensions
of the field variable values in \mods{defaults}.

The \class{Qtcm} method \mods{get\_qtcm1\_item} directly calls
the \mods{SetByPy} routines.
The \class{Qtcm} method \mods{set\_qtcm1\_item} makes use of
private instance methods that make the calls to the \mods{SetByPy} routines.

For scalar field variables, \mods{SetByPy} provides functions and
subroutines that provide the value of the variable on output.
For array field variables, \mods{SetByPy}
dynamic \emph{module} arrays are used to pass array
variables in and out; I could not get the 
\mods{SetByPy} Fortran routines to set
locally defined dynamic arrays (that is, locally within a function or
subroutine).\footnote%
	{I tried to implement Fortran subroutine
	\mods{getitem\_real\_array} using traditional array 
	dimension passing 
	(e.g., \code{subroutine foo(nx, ny, a)}) as well
	as declaring the allocatable array inside the subroutine, 
	but neither option worked on my \mods{f2py} (version 2\_3816) 
	and Python (version 2.4.3).}
In the \mods{SetByPy} module, these dynamic arrays
are defined as follows:

\begin{codeblock}
\codeblockfont{%
Real, allocatable, dimension(:) :: real\_rank1\_array \\
Real, allocatable, dimension(:,:) :: real\_rank2\_array \\
Real, allocatable, dimension(:,:,:) :: real\_rank3\_array}
\end{codeblock}

For all field variables, scalar or array, the \mods{SetByPy} module
has a fourth module variable defined, \vars{is\_readable}, that the
Fortran get and set routines will set to \vars{.TRUE.} if the
variable is readable and \vars{.FALSE.} if not (it's declared as a
logical variable).  This Fortran variable can be used to prevent
Python from accessing pointer variables that aren't yet associated
to targets.

In general, \mods{SetByPy} routines make use of Fortran constructs
to enable them to accomodate all possible
variables of a given type and shape.  However, 
for string scalars, the \mods{SetByPy} function \mods{getitem\_str}
has to have a return value of a predefined length, in order to
work properly.  That length is given by the parameter
\vars{maxitemlen} and is set to 505 (the value is chosen to
be larger than all filename variables described in
Section~\ref{sec:f90changes} and to be easily found in
the \fn{.F90} files).


		\subsubsection{Module Structure}

If you're a Fortran programmer, you can probably get all the information
in this section from just reading the \fn{setbypy.F90} file directly.
This description of the module structure, however, permits me to highlight
what you need to do if you want to make additional compiled QTCM1 variables
accessible to Python \class{Qtcm} objects.

\begin{itemize}
\item All \mods{Use} statements are given in the beginning of 
	the \mods{SetByPy} module.  These statements cover
	nearly all of the QTCM1 Fortran
	modules that contain variables of interest.  If the
	QTCM1 variable you're interested in isn't in a module
	listed here, you'll have to add your own
	\mods{Use} statement of that module here.

\item Next comes the definitions for the
	\vars{real\_rank1\_array},
	\vars{real\_rank2\_array}, and
	\vars{real\_rank3\_array} dynamic array variables, and
	the \vars{is\_readable} boolean variable.

\item The \mods{Contains} block of the module defines the module
	routines called by the \class{Qtcm} instance methods to
	set and get the compiled QTCM1 model variables.  The
	routines are:
	\begin{itemize}
	\item Function \mods{getitem\_real}
	\item Subroutine \mods{getitem\_real\_array}
	\item Function \mods{getitem\_int}
	\item Function \mods{getitem\_str}
	\item Subroutine \mods{setitem\_real}
	\item Subroutine \mods{setitem\_real\_array}
	\item Subroutine \mods{setitem\_int}
	\item Subroutine \mods{setitem\_str}
	\end{itemize}

\end{itemize}

Each of the routines in the module \mods{Contains} block is essentially
a list of \mods{if}/\mods{elseif} statements.  The list tests for the
name of the variable of interest (a string), and gets or sets the
compiled QTCM1 model variable corresponding to that name.  For pointer
array variables, a test is also made as to whether or not the variable
has been associated.  If not, the variable is not readable
and \vars{is\_readable} is set to \vars{.FALSE.}\ accordingly.

If you wish to add another compiled QTCM1 model variable to be
accessible to \class{Qtcm} instance methods \mods{get\_qtcm1\_item}
and \mods{set\_qtcm1\_item}, just add another \mods{if}/\mods{else\-if},
like the other \mods{if}/\mods{elseif} blocks, in the Fortran set
and get routines corresponding to the QTCM1 variable type (scalar
vs.\ array, and real, integer, or string).  On the Python side, add
an entry in \mods{defaults} corresponding to the new field variable
you've created access to.  I would strongly recommend making the
Python name of your new field variable
(given in \mods{defaults}) be the same as the compiled
QTCM1 model variable name.



	\subsection{Fortran Module \mods{WrapCall}}   \label{sec:wrapcall}

Most of the time, if you want to call a compiled QTCM1 model subroutine
from the Python level, you will use the version of the subroutine that
is found in this Fortran module.  
Note that the Fortran module \mods{WrapCall} is referred
to in lowercase at the Python level, i.e., as the
attribute \vars{\_\_.qtcm.wrapcall} of a \class{Qtcm} instance.

All the routines in this module do is wrap one of the compiled QTCM1
model routines.  For instance, \mods{WrapCall} subroutine
\mods{wadvcttq} is defined as just:

% --- Two versions of this block, one for display in PDF and the other
%     for display in HTML:
%
\begin{latexonly}
\begin{codeblock}
\codeblockfont{%
Subroutine wadvcttq \\
\hspace*{3ex}Call advcttq \\
End Subroutine wadvcttq}
\end{codeblock}
\end{latexonly}

\begin{htmlonly}
\begin{rawhtml}
<p><code><font color="blue">Subroutine wadvcttq<br>
&nbsp;&nbsp;&nbsp;Call advcttq<br>
End Subroutine wadvcttq</font></code></p>
\end{rawhtml}
\end{htmlonly}

All subroutines in this module begin with ``w'', with the rest of
the name being the Fortran QTCM1 subroutine name.  The calling
interface for the ``w'' version is the same as the Fortran QTCM1
original version.  There are no subroutines in this module that do
not have an exact counterpart in the Fortran QTCM1 code, and thus
this module's subroutines sole purpose is to call other subroutines
in the compiled QTCM1 model.

These wrapper routines are needed because \mods{f2py}, for some
reason I can't figure out, will not properly wrap Fortran routines
(that are then callable at the Python level) that create local
arrays using parameters obtained through a Fortran \mods{use}
statment.  Thus, as an example, a Fortran subroutine \mods{foo}
with the following definition:

% --- Two versions of this block, one for display in PDF and the other
%     for display in HTML:
%
\begin{latexonly}
\begin{codeblock}
\codeblockfont{%
subroutine foo \\
\hspace*{3ex}use dimensions \\
\hspace*{3ex}real a(nx,ny) \\
\hspace*{3ex}[\ldots] \\
end subroutine foo}
\end{codeblock}
\end{latexonly}

\begin{htmlonly}
\begin{rawhtml}
<p><code><font color="blue">
subroutine foo<br>
&nbsp;&nbsp;&nbsp;use dimensions<br>
&nbsp;&nbsp;&nbsp;real a(nx,ny)<br>
&nbsp;&nbsp;&nbsp;[\ldots]<br>
end subroutine foo
</font></code></p>
\end{rawhtml}
\end{htmlonly}


where \vars{nx} and \vars{ny} are defined in the module vars{dimensions},
will return an error, with the result that the extension module
will not be created, or an extension modules that yields output
that is different from running the pure-Fortran version of QTCM1.

By wrapping these calls into this file, I also avoid having to
separate out the Fortran QTCM1 subroutines into separate \fn{.F90}
files.  For Fortran subroutines that you want callable from the
Python level, \mods{f2py} seems to require each Fortran subroutine
to be in its own file of the same name (e.g., the version of
\fn{driver.F90} for this package). If several Fortran subroutines
are all found in a single \fn{.F90} files, \mods{f2py} seems unable
to create wrappers for those subroutines.




%---------------------------------------------------------------------
\section{Python \mods{qtcm} and Pure-Fortran QTCM1 Differences}

This section describes differences between how the \mods{qtcm}
package and the pure-Fortran QTCM1 assign some varables.  A list
of changes to the QTCM1 Fortran Files for use in the \mods{qtcm}
package is found in Section~\ref{sec:f90changes}.


	\subsection{QTCM1 \mods{driverinit}}   \label{sec:driverinit.diffs}

In the pure-Fortran version of QTCM1, by default, the following variables are
set by reference (as given below), not by value, in the \mods{driverinit}
routine:\footnote%
	{In the pure-Fortran version of QTCM1, this routine is found
	in \fn{driver.F90}.}
\begin{codeblock}
\codeblockfont{%
lastday\thinspace=\thinspace{daysperyear} \\
viscxu0\thinspace=\thinspace{viscU} \\
viscyu0\thinspace=\thinspace{viscU} \\
visc4x\thinspace=\thinspace{viscU} \\
visc4y\thinspace=\thinspace{viscU} \\
viscxu1\thinspace=\thinspace{viscU} \\
viscyu1\thinspace=\thinspace{viscU} \\
viscxT\thinspace=\thinspace{viscT} \\
viscyT\thinspace=\thinspace{viscT} \\
viscxq\thinspace=\thinspace{viscQ} \\
viscyq\thinspace=\thinspace{viscQ}}
\end{codeblock}

Thus, in pure-Fortran QTCM1, if you change \vars{daysperyear},
\vars{viscU}, etc.
and recompile (as needed), you will automatically change 
\vars{lastday}, \vars{viscxu0}, etc.
(Though, in the pure-Fortran QTCM1, the default values may be overwritten by
namelist input values.)

The \mods{driverinit} routine is eliminated
in the Python \code{qtcm} package.  Instead, inital values 
of field variables are specified in the \mods{defaults} submodule
and set by value to attributes of the \code{Qtcm} instance.
Thus, for instance, in a \class{Qtcm} instance, \code{lastday} 
is set to \code{365} by default, not to some variable
\vars{daysperyear}.  For the diffusion and viscosity terms,
the \class{Qtcm} instance attributes corresponding to those
terms are set to literals.\footnote%
	{Those literals are defined by \mods{defaults} private
	module variables \vars{\_\_viscT}, \vars{\_\_viscQ},
	and \vars{\_\_viscU}.}

In contrast, in the pure-Fortran QTCM1,
\mods{driverinit} declares local
variables \code{viscU}, \code{viscT}, and \code{viscQ},
and reads values into those variables via the input namelist.
Those values are then used to set
\vars{viscxu0}, \vars{viscyu0}, etc., as described above.
In pure-Fortran QTCM1, \code{viscU}, \code{viscT}, and \code{viscQ}
are not directly accessed anywhere else in the model.
Thus, \code{viscU}, \code{viscT}, and \code{viscQ} are not
defined as field variables in the \code{qtcm} package, and
\class{Qtcm} instances do not have attributes corresponding
to those names.
Additionally, if you wish to change a viscosity parameter
\vars{visc*} (given above), the parameter for each direction
must be set one-by-one even if the flow is isotropic.


	\subsection{The \mods{varinit} Routine}

One of the functions of the pure-Fortran QTCM1 \mods{varinit}
subroutine is to associate the pointer variables \vars{u1}, \vars{v1},
\vars{q1}, and \vars{T1}.  For the extension modules in the \mods{qtcm}
package, a Fortran subroutine \mods{varptrinit} is added that can
also do this association.  This subroutine is called in the
\class{Qtcm} instance method
\latexhtml{\mods{varinit}%
		\footnote{http://www.johnny-lin.com/py\_docs/qtcm/doc/html-api/qtcm.qtcm.Qtcm-class.html\#varinit}}%
	{\htmladdnormallink{\mods{varinit}}{http://www.johnny-lin.com/py_docs/qtcm/doc/html-api/qtcm.qtcm.Qtcm-class.html#varinit}}
(which duplicates and
extends the function of its pure-Fortran counterpart, enabling
alternative ways of handling restart).

The \mods{varptrinit} is not accessed via \mods{wrapcall}.  Remember
that \mods{wrapcall} contains only those routines that were in the
original pure-Fortran QTCM1 code, and that we want to have access
to at the Python level.


	\subsection{The \mods{qtcm} Method of \class{Qtcm}}

The \class{Qtcm} method \mods{qtcm} duplicates the functionality
of the \mods{qtcm} subroutine in the pure-Fortran QTCM1 model.
There are a few differences, however.  First, the \mods{qtcm} method
for \class{Qtcm} instances does not include a call to \mods{cplmean},
which uses mean surface flux for air-sea coupling.  This state is
consistent with the pure-Fortran QTCM1 pre-processor macro
\vars{CPLMEAN} being off.  Thus, if you wish to use mean surface
flux for air-sea coupling, you will have to revise the \mods{qtcm}
method of \class{Qtcm} to call \mods{cplmean}.  You'll also have to
check for any other code additions needed that are associated with
the \vars{CPLMEAN} macro.

Second, the \mods{qtcm} method for \class{Qtcm} instances does not
include the option of not using the atmospheric boundary layer
model.  This is consistent with macro \vars{NO\_ABL} being off.  If
you wish to have no atmospheric boundary layer model, change the
run list \vars{atm\_bartr\_mode} so that the \mods{wsavebartr} and
\mods{wgradphis} routines are not called.  You'll also have to check
for any other code additions needed that are associated with the
\vars{NO\_ABL} macro.



	\subsection{Miscellaneous Differences}

\begin{itemize}
\item In Python \class{Qtcm} instances,
	\vars{dateofmodel} is set to 0 by default.  
	In contrast, in the compiled QTCM1 model,
	the default (i.e., initial value) is calculated from 
	\vars{day0}, \vars{month0}, and \vars{year0}.
	See Section~\ref{sec:init.compiledform.full} for details.

\item The \class{Qtcm} instance attribute
	\vars{\_\_qtcm} is not copyable using \mods{copy.deepcopy}.

\item In general, when executing a \class{Qtcm} instance method, 
	if you change a \class{Qtcm} instance attribute 
	that has a counterpart in the compiled QTCM1 model,
	the compiled QTCM1 counterpart is not changed until the
	end of the method.  Likewise, if you call a compiled QTCM1 model
	subroutine and change a compiled QTCM1 model variable with
	a \class{Qtcm} instance counterpart, the \class{Qtcm}
	instance counterpart is not changed until the end of the
	subroutine.

\item In general, even though some of the compiled QTCM1 model
	Fortran subroutines/functions have counterparts in \class{Qtcm}
	that duplicate the former's functionality, the Fortran
	versions are kept intact so that the
	\vars{compiled\_form\thinspace=\thinspace'full'} case will work.
\end{itemize}




%---------------------------------------------------------------------
\section{Considerations When Adding Fortran Code}

In this section I describe issues to consider if you wish to add
your own compiled code to the package as separate extension modules.
(This is different from creating new standard extension modules,
which is described in Section~\ref{sec:create.new.so}.):

\begin{itemize}
\item The \class{Qtcm} class assumes that the directory path 
	to the original shared object file is the same as for the 
	\mods{package\_version} module.

\item If you want to be able to pass other Fortran variables 
	in and out to/from Python, please see the 
	Section~\ref{sec:setbypy}
	discussion of the Fotran \mods{SetByPy} module.

\item Fortran and Python routines to get and set compiled QTCM1 model
	arrays are currently written only for floating point array.

\item If you ever change 
	\class{Qtcm} instance method
	\mods{\_set\_qtcm\_array\_item\_in\_model}
	to work with non-floating point values, you will also
	have to change the array handling section in 
	\mods{set\_qtcm1\_item}.

\item The restart mechanism in the pure-Fortran QTCM1 model is 
	\emph{not} bit-for-bit correct.  Thus, if you compare the final
	output from a 40 day run with a 30 day run restarted from
	a 10 day run, the output will not be the same.
	This behavior has been duplicated in \class{Qtcm} 
	instances when the \vars{mrestart} flag is used
	and applicable.

\item When creating new extension modules using the \fn{src} makefile,
	be sure you first use the \cmd{make clean} command to clean-up
	any old files.

\end{itemize}




%---------------------------------------------------------------------
\section{Creating New Standard Extension Modules}   \label{sec:create.new.so}

The steps involved in creating the standard extension modules (e.g.,
\fn{\_qtcm\_full\_365.so}, etc.) on installation are given in
Section~\ref{sec:create.so}.  The makefile provided in \fn{/buildpath/src}
uses a Fortran compiler to create the object code, runs \mods{f2py}
to create the shared object file in \fn{src}, and moves the shared
object files into \fn{../lib}, overwriting any pre-existing files
of the same name.  In this section, I describe the makefile and
\mods{f2py} in a little more detail, in case you wish to create
standard extension modules with additions from the ones the default
makefile creates.


	\subsection{Makefile Rules}    \label{sec:makefile.rules}

This section describes the rules of the
makefile found in the \fn{src} directory
of the \mods{qtcm} distribution.  
This makefile is used by the Python package to create the extension
module (\fn{.so} files) imported and used by \mods{qtcm} objects
(as described in Section~\ref{sec:create.so}).
The makefile will, in general, be used only during \mods{qtcm}
installation, but if you wish to recompile the QTCM1 libraries
and make changes in the Python extension module,
you'll want to use/change this makefile.

\begin{description}
\item[clean] Removes old files in preparation for compiling new
	extension modules.

\item[libqtcm.a] Creates library \fn{libqtcm.a} that contains all
	QTCM1 object files in the directory \fn{src},, except
	\fn{setbypy.o}, \fn{wrapcall.o}, \fn{varptrinit.o}, and
	\fn{driver.o}.  This archive is compiled with the netCDF
	libraries.  Previous versions of \fn{libqtcm.a} are overwritten.

\item[\_qtcm\_full\_365.so] Creates the extension module
	\fn{\_qtcm\_full\_365.so}.  \mods{f2py} is run on applicable code
	in \fn{src}, and the extension module is moved to \fn{../lib}.
	Any previous versions of \fn{../lib/\_qtcm\_full\_365.so}
	are overwritten.

\item[\_qtcm\_parts\_365.so] Creates the extension module
	\fn{\_qtcm\_parts\_365.so}.  \mods{f2py} is run on applicable code
	in \fn{src}, and the extension module is moved to \fn{../lib}.
	Any previous versions of \fn{../lib/\_qtcm\_parts\_365.so}
	are overwritten.

\end{description}



	\subsection{Using \mods{f2py}}      \label{sec:using.f2py}

This section briefly describes how \mods{f2py} is used in the
makefile during the creation of the extension modules.
\htmladdnormallink{\mods{F2py}}{http://cens.ioc.ee/projects/f2py2e/} is a
program that generates shared object libraries that allow you to call
Fortran routines in Python.  \mods{F2py} comes with Python's
\htmladdnormallink{NumPy}{http://numpy.scipy.org/}
array handling package, so you do not need to install anything
extra if you have NumPy already installed.

To create the extension modules in \mods{qtcm} using
the makefile described in Section~\ref{sec:makefile.rules},
I use a method similar to the
\latexhtml{``Quick and Smart Way,''\footnote%
{http://cens.ioc.ee/projects/f2py2e/usersguide/index.html\#the-quick-and-smart-way}}%
{\htmladdnormallink{``Quick and Smart Way''}%
{http://cens.ioc.ee/projects/f2py2e/usersguide/index.html#the-quick-and-smart-way}}
described in the \mods{f2py} manual.
For the \fn{\_qtcm\_full\_365.so} extension module, the 
\mods{f2py} call is:

\begin{codeblock}
\codeblockfont{%
f2py --fcompiler=\$(FC) -c -m \_qtcm\_full\_365 driver.F90 $\backslash$ \\
\hspace*{10ex}setbypy.F90 libqtcm.a \$(NCLIB)}
\end{codeblock}

and for the \fn{\_qtcm\_parts\_365.so} extension module, the call is:

\begin{codeblock}
\codeblockfont{%
f2py --fcompiler=\$(FC) -c -m \_qtcm\_parts\_365 $\backslash$ \\
\hspace*{10ex}varptrinit.F90 wrapcall.F90 setbypy.F90 $\backslash$ \\
\hspace*{10ex}libqtcm.a \$(NCLIB)}
\end{codeblock}

For both calls, \vars{FC} and \vars{NCLIB} are the environment
variables in the makefile specifying the Fortran compiler and netCDF
libraries, respectively.  The \vars{-m} flag specifies the extension
module name (without the \fn{.so} suffix).  The \fn{.F90} files
specify the files that have modules and routines that will be
accessible at the extension module level, and the rest of the Fortran
files in QTCM1 are compiled and archived in a library \fn{libqtcm.a}.
For \mods{f2py} to work properly,
the \fn{.F90} files may define \emph{only one} module or routine.

If you add Fortran files containing new modules, and you wish those
modules to be accessible at the Python level, compile your new code
with \mods{f2py}.  If we have a file of such new code, \fn{newcode.F90},
the \mods{f2py} call to create the \fn{\_qtcm\_parts\_365.so}
extension module will become:

\begin{codeblock}
\codeblockfont{%
f2py --fcompiler=\$(FC) -c -m \_qtcm\_parts\_365 $\backslash$ \\
\hspace*{10ex}varptrinit.F90 wrapcall.F90 setbypy.F90 $\backslash$ \\
\hspace*{10ex}newcode.F90 $\backslash$ \\
\hspace*{10ex}libqtcm.a \$(NCLIB)}
\end{codeblock}

If you write new Fortran code for the compiled QTCM1 model that
will \emph{not} be accessed from the Python-level, just add the
object code filename to the variable \vars{QTCMOBJS} in the
makefile; you don't have to do anything else.  If you are adding
Fortran code to existing Fortran modules, it's even easier:  You
don't need change the makefile.  Note that for 64 bit processor
machines, you may have to use \mods{f2py} with the \cmd{-fPIC} flag;
see Section~\ref{sec:sopic} for details on how the lines above will
change.


	\subsection{Two Examples}

\emphpara{A Function:}
Let's say you have written a piece of Fortran code called
\fn{myfunction.F90} that contains one function called
\mods{myfunction}, and you want to have this function
callable from the Python level through the \class{Qtcm} 
instance method \mods{\_\_qtcm.myfunction}.  Do the following:

\begin{enumerate}
\item Move \fn{myfunction.F90} to \fn{src} in the \mods{qtcm}
	distribution directory \fn{/buildpath}.

\item Add \cmd{myfunction.o} to the end of the object file list lines
	after the target names
	\vars{\_qtcm\_full\_365.so} and
	\vars{\_qtcm\_parts\_365.so}.

\item In the
	\vars{\_qtcm\_full\_365.so} and
	\vars{\_qtcm\_parts\_365.so} target descriptions,
	add \cmd{myfunction.F90} to the 
	beginning of the list of \fn{.F90} names 
	in the \mods{f2py} lines.
\end{enumerate}


\emphpara{A Module:} 
Let's say you have written a piece of Fortran code called
\fn{mymodule.F90} that contains the Fortran module \mods{MyModule}
containing multiple routines and variables.  You want to have those
routines and variables callable from the Python level through the
\class{Qtcm} instance attribute \mods{\_\_qtcm.mymodule}.  The steps
to add \mods{MyModule} to the extension modules are exactly the
same as for a single function, with \cmd{mymodule} being
substituted in the makefile everywhere you have \cmd{myfunction}.




%---------------------------------------------------------------------
\section{Attributes and Methods in \class{Qtcm} Instances}

In this section I describe some attributes, particularly private ones,
that may be of interest to developers.
As is the convention in Python, private
attributes and methods are prepended by one or two underscores,
with two underscores being the ``more'' private attribute.
Please see the package
\latexhtml{API documentation%
		\footnote{http://www.johnny-lin.com/py\_pkgs/qtcm/doc/html-api/}}
        {\htmladdnormallink{API documentation}%
		{http://www.johnny-lin.com/py\_pkgs/qtcm/doc/html-api/}}
for details about all variables, including private variables.


	\subsection{Public \mods{num\_settings} Submodule Attributes/Methods}

\begin{itemize}
\item \vars{typecode}:  This module function returns the
	type code of the data array passed in as its argument.

\item \vars{typecodes}:  This dictionary is the same as the
	NumPy (or Numeric and \mods{numarray})
	dictionary \vars{typecodes}, except that the character
	\vars{'S'} and \vars{'c'} are added to the
	\vars{typecodes['Character']} entry, if absent.  This
	functionality is added because I found 
	\vars{typecodes['Character']} had different values in
	Mac OS X and Ubuntu GNU/Linux.
\end{itemize}


	\subsection{Private \mods{qtcm} Submodule Attributes}

This submodule of the package \mods{qtcm} is the module that defines
the \class{Qtcm} class.

\begin{itemize}
\item \vars{\_init\_prog\_dict}:  This dictionary contains
	the default values of all prognostic variables and 
	right-hand sides that can be initialized.  In the
	submodule \mods{qtcm}, it is set to
	the \vars{init\_prognostic\_dict} module variable in
	submodule \mods{defaults}.

\item \vars{\_init\_vars\_keys}:  List of all keys in
	\vars{\_init\_prog\_dict}, plus \vars{'dateofmodel'}
	and \vars{'title'}.  These names correspond to the
	field variables that are usually written out into a
	restart file.

\item \vars{\_test\_field}:  \class{Field} object instance used 
	in type tests.
\end{itemize}



	\subsection{Private \class{Qtcm} Attributes}  
					\label{sec:Qtcm.private.attrib}

\begin{itemize}
\item \vars{\_cont}:  A boolean attribute that is \vars{True}
	if the run session is a continuation run session and
	\vars{False} if not.  Set the value passed in by
	the keyword \vars{cont} when the \mods{run\_session}
	method is executed.

\item \vars{\_monlen}:  Integer array of the number of days in 
	each month, assuming a 365~day year.

\item \vars{\_\_qtcm}:  The extension module that is the
	compiled QTCM1 Fortran model for this instance.
	This attribute is unique for every instance:  The
	extension module \fn{.so} file is first copied to
	a temporary directory (given by the \vars{sodir}
	instance attribute) and then imported to the
	\class{Qtcm} instance.
	This private attribute is set on instantiation.

\item \vars{\_qtcm\_fields\_ids}:  Field ids for all default 
	field variables, set on instantiation.

\item \vars{\_runlists\_long\_names}:  Dictionary holding the
	descriptions of the standard run lists.  The keys of
	the dictionary are the names of the standard run lists.
\end{itemize}




%---------------------------------------------------------------------
\section{Creating Documentation}

The distribution of \mods{qtcm} comes with the full set of
documentation in readable form (PDF and HTML).  The documentation
consists of two kinds:  this User's Guide and the API documentation.
The User's Guide is written in \LaTeX.  The PDF version is generated
directly from \LaTeX, and the HTML version is created by
\LaTeX{2}HTML.

I use the \fn{make\_docs} shell script in \fn{doc} creates all these
documents.  Briefly, that script does the following:

\begin{itemize}
\item In the \fn{doc/latex} directory, uses \cmd{python} to
	run \fn{code\_to\_latex.py}, which generates the
	\LaTeX\ files describing the current \mods{qtcm} 
	package settings, including text in the manual which gives
	all uses of the current version number.

\item \LaTeX\ is run on the \LaTeX\ files in the \fn{doc/latex} directory.
	The PDF generated by the run is moved from \fn{doc/latex} to
	\fn{doc}.

\item \LaTeX{2}HTML is run on the \LaTeX\ files in \fn{doc/latex}.
	The HTML files generated by the run are moved to \fn{doc/html}.

\item \mods{epydoc} is run on the \mods{qtcm} package libraries.
	This is run in \fn{doc}, to make use of the \fn{epydoc}
	configuration file present there.  The syntax from the
	command line is:

\begin{codeblock}
\codeblockfont{%
epydoc -v --config epydocrc [name]}
\end{codeblock}
\vars{[name]} is either \cmd{qtcm}, if the \mods{qtcm} package is
installed in a directory listed in \vars{sys.path}, or 
\vars{[name]} is the name of the directory the \mods{qtcm} package is
located in (e.g., \fn{/usr/lib/python2.4/site-packages/qtcm}).

\end{itemize}

The \fn{make\_docs} script cannot be used without customizing it
to your system, so please \emphpara{DO NOT USE IT} if you do
not know what you are doing.  You could easily wipe out all your
documentation by mistake.





% ===== end of file =====


\chapter{Future Work}                       \label{ch:future}
% ==========================================================================
% Future
%
% By Johnny Lin
% ==========================================================================


% ------ BODY -----
%
This section describes the features and fixes I plan to work on
in this package.  The most urgent items are listed closer to the
begining of the lists.

\begin{itemize}
\item Add \code{implicit none} top setbypy.F90.

\item Check through Fortran routines that have arguments, to make sure
	f2py is properly understanding the intentions
	(i.e., in, out, inout) of the variables, since we're using the
	``quick way'' of making shared object libraries using f2py.
	The \fn{utilities.F90} file has a number of Fortran routines
	with arguments.

\item Cite:  Peterson, P. (2009) 
	F2PY: a tool for connecting Fortran and Python programs, 
	\emph{Int. J. Computational Science and Engineering,}
	Vol.\ 4, No.\ 4, pp.\ 296--305 for f2py.

\item Create a method like \mods{calc\_derived('T100')} which would
	primarily operate on a data file and provide a derived variable
	such as the temperature at 100 hPa, as given in this example.
	Figure out where to put the parameters (V1s, etc.) that are
	needed to make such a calculation.  As attributes?  Create a
	method to write the quantity out to an output file?
	Perhaps make an ability to calculate these values at heights
	at a given time each day during a run session?

\item Automate the installation using Python's
\htmladdnormallinkfoot{\mods{distutils}}{http://docs.python.org/dist/dist.html}
	utilities.

\item Describe a way of using job control (either via the operating system
	or IPython's \mods{jobctrl} module) 
	to do a quick-and-dirty parallelization of multiple
	\class{Qtcm} instance run sessions.  Or use some sort of threading
	to fire up two simulataneously running models.  Check that the
	simultaneously running models have different memory space.

\item Add capability for \fn{create\_benchmark.py} to overwrite
	existing benchmark files.

\item Make \vars{compiled\_form} set to \vars{'parts'} as the
	default instantiation.  Change documentation accordingly.

\item Currently, the \class{Qtcm} \mods{plotm} method works only on
	3-D output (time, latitude, longitude).  Some of the fields
	in the netCDF output files are 2-D.  Add the capability to
	\mods{plot\_netcdf\_output} in the \mods{plot} submodule
	to handle 2-D fields.

\item Add documentation about removing temporary files.
	Add documentation in Section~\ref{sec:model.instances}
	of details of what occurs during instantiation of 
	a \class{Qtcm} instance.

\item Add the units and long names for all field variables in the
	\mods{defaults} module.

\item Create a keyword to automatically change precipitation and
	evaporation units to mm/day (or similar).

\item Add ability to calculate and plot fields at different pressure
	levels.  Create another module like defaults that specifies
	the vertical fields and gives the equation to use to calculate
	those fields; call the module ``derivfields'' or something
	similar.

\item Throughout the \mods{qtcm} package I use the condition
	\mods{N.rank(}\dumarg{arg}\mods{)\thinspace=\thinspace0} 
	to test whether
	\dumarg{arg} is a scalar.  This works fine for \mods{numpy}
	objects, but it does not work properly for
	\mods{Numeric} and \mods{numarray} arrays.  In those
	array packages, \mods{rank('abc')} returns the value~1.
	This is not a problem, as long as everyone has \mods{numpy},
	but in order to make the package interoperable, I need to
	find a better way of testing for scalars.  The definitions
	of isscalar need to be changed in \mods{num\_settings}.

\item \mods{num\_settings} needs to be changed to truly enable me
	to test whether \mods{qtcm} works for 
	\mods{numarray} and \mods{Numeric} arrays.  The tests
	do not do this right now, because \mods{num\_settings}
	defaults to \mods{numpy}, if it exists.

\item Create makefiles for other platforms.
 
\item A few fields (e.g., \vars{u1}) have data for extra latitude bands,
	due to the use of ``ghost latitudes'' as part of the
	implementation of the numerics.  Details are found in the 
\latexhtml{%
\htmladdnormallinkfoot{QTCM1 manual}%
        {http://www.atmos.ucla.edu/$\sim$csi/qtcm\_man/v2.3/qtcm\_manv2.3.pdf}}%
{\htmladdnormallink{QTCM1 manual}%
        {http://www.atmos.ucla.edu/~csi/qtcm_man/v2.3/qtcm_manv2.3.pdf}}
\cite{Neelin/etal:2002}.

	Though adjusting to this idiosyncracy is not that difficult, 
	in the future I hope to implement a method of handing
	fields with ghost latitudes so that they have the same
	dimensions as the other gridded output variables.  In order
	to do this, I plan to write a Python method to read the
	Fortran generated binary restart file.

\item Change the \mods{set\_qtcm\_item} method so that it can 
	automatically accomodate setting Fortran real variables
	if integer values are input.

\item Currently, the \mods{get\_item\_qtcm} and 
	\mods{set\_item\_qtcm} methods will not work
	on integer and character arrays, only scalars and real arrays.
	Add that missing functionality to those methods.

\item Currently, the \mods{make\_snapshot} method duplicates the
	functionality of the pure-Fortran QTCM1 restart file mechanism.
	However, the restart file mechanism itself does not do a true
	restart.  A continuous run does not provide the same results
	as two runs over the same period, joined by the restart file.

	To see whether saving more variables would do the trick,
	I altered \mods{make\_snapshot} to store all Python level
	variables (i.e., \vars{self.\_qtcm\_fields\_ids}).  However,
	the restart failing described above still continued.  In the
	future, I hope to figure out exactly how many variables are
	needed in order to make the restart feature do a true
	restart.

\item Add a test of using the \vars{mrestart\thinspace=\thinspace1}
	restart option.  Does the \fn{qtcm.restart} file need to be
	in the current working directory or another?

\item Add a test in the unit test scripts to
	confirm that the \vars{init\_with\_instance\_state}
	attribute setting only has an effect if 
	\vars{compiled\_form\thinspace=\thinspace'parts'}.

\item Document \vars{tmppreview} keyword in \mods{plot.plot\_ncdf\_output}.

\item Confirm and document that
	for netCDF output, time is model time since dd-mm-yyyy.

\item Add to the \mods{plotm} method the ability to
	plot as text onto the figure the
	runname string and the calling line
	for the plotm method.

\item Couple with the
	\latexhtml{CliMT\footnote{http://maths.ucd.ie/$\sim$rca/climt/}}%
	{\htmladdnormallink{CliMT}{http://maths.ucd.ie/~rca/climt/}}
	climate modeling toolkit.

\item Enable Python to set \vars{arr1name}, etc., which are string
	variables at the Python level.  I haven't really thought through
	how \vars{arr1} variables work with the Python \class{Qtcm}
	instance.

\item Possible:  In the \class{Qtcm} method
	\mods{\_\_setattr\_\_}, add a test to raise an exception
	if the instance tries to set \vars{viscU}, \vars{viscT},
	or \vars{viscQ} as attributes.  Also create a method
	\code{isotropic\_visc} that will set all viscosity parameters
	non-dependent on direction.  See Section~\ref{sec:driverinit.diffs}
	for details.

\item Go through the manual and create HTML-only versions of tables
	that have table numbers (use a similar construct as in
	figure environments).

\item Go through documentation to check that
	output variable names are capitalized consistently.

\item Create way to redirect stdout.

\item Create a step method to run an arbitrary number of timesteps at
	the atmosphere level.

\end{itemize}


% ===== end of file =====





% ----- BACK MATTER OF THE DOCUMENT -----
%
\normalsize
\pagebreak
\bibliographystyle{plain}
\bibliography{/Users/jlin/work/res/bib/master}

%- Uncomment the input line below and comment out the \bibliographystyle
%  and \bibliography lines if you're running this without the master.bib 
%  BibTeX database
%% ==========================================================================
% Manual for QTCM Python Package
%
% Usage:
% - If you are running this on your own system, you will not have a copy of
%   my master.bib BibTeX database.  To run this, you'll have to comment out:
%
%      \bibliographystyle{chicago-jl}
%      \bibliography{/Users/jlin/work/res/bib/master}
%
%   and comment back in:
%
%      % ==========================================================================
% Manual for QTCM Python Package
%
% Usage:
% - If you are running this on your own system, you will not have a copy of
%   my master.bib BibTeX database.  To run this, you'll have to comment out:
%
%      \bibliographystyle{chicago-jl}
%      \bibliography{/Users/jlin/work/res/bib/master}
%
%   and comment back in:
%
%      % ==========================================================================
% Manual for QTCM Python Package
%
% Usage:
% - If you are running this on your own system, you will not have a copy of
%   my master.bib BibTeX database.  To run this, you'll have to comment out:
%
%      \bibliographystyle{chicago-jl}
%      \bibliography{/Users/jlin/work/res/bib/master}
%
%   and comment back in:
%
%      \input{manual.bbl}
%
%   in this file.  Then you can use pdflatex on this file to get the PDF of
%   the manual.  These 3 lines are in the back matter of the document.
%
% Revision Notes:
% - By Johnny Lin, North Park University, http://www.johnny-lin.com/
% - The chicago BibTeX style is unrecognized by latex2html, so I use
%   the plain style.
% ==========================================================================


% ------ DOCUMENT DEFINITIONS ------
%
\documentclass[12pt]{book}
\usepackage{color}
\usepackage{html}
\usepackage{graphicx}
\usepackage{textcomp}
%\usepackage{comment}    %- Unrecognized by latex2html; its use causes errors
%\usepackage{fancyvrb}   %- Unrecognized by latex2html; its use causes errors


%- Packages unrecognized by latex2html, but causes no error:
%
%\usepackage[letterpaper,margin=1in,includefoot]{geometry}
\usepackage[letterpaper,margin=1.25in]{geometry}
\usepackage{bibnames}
\usepackage{longtable}
\usepackage{multirow}


%+ Comment out explicity margin settings since use package geometry:
%\setlength{\topmargin}{0in}
%\setlength{\headheight}{0in}
%\setlength{\headsep}{0in}
%\setlength{\oddsidemargin}{0in}
%\setlength{\evensidemargin}{0in}
%\setlength{\textheight}{8.5in}
%\setlength{\textwidth}{6.5in}




% ------ COMMANDS AND LENGTHS ------
%
% --- Define colors:  Have to do this because for some reason LaTeX
%     sometimes looks for "BLUE" instead of "blue" and complains when
%     "BLUE" isn't found.
%
\definecolor{Blue}{rgb}{0,0,1}
\definecolor{BLUE}{rgb}{0,0,1}
\definecolor{green}{rgb}{0,0.6,0}
\definecolor{Green}{rgb}{0,0.6,0}
\definecolor{GREEN}{rgb}{0,0.6,0}


% --- Format code blocks.  Currently set to print out the code in just 
%     typewriter font with no box.  Will work the same for pdflatex 
%     and latex2html:
%
%     codeblock:  Environment for blocks of computer code or internet 
%       addresses.
%     codeblockfont:  Sets font for codeblocks.
%
\newenvironment{codeblock}%
	{\begin{quotation}\begin{minipage}[t]{0.9\textwidth}}%
	{\end{minipage}\end{quotation}}
	%{\begin{flushleft}}%
	%{\end{flushleft}}
\newcommand{\codeblockfont}[1]{\textcolor{blue}{\texttt{#1}}}
%     *** Version that only works for pdflatex that puts a box around 
%         the block and centers it (commented out).  Note that using
%         fancyvrb is the better way of creating such a boxed section
%         of code, but fancyvrb isn't recognized by latex2html:
%\newenvironment{codeblock}%
%	{\begin{center}\begin{tabular}{|c|} \hline \\ }%
%	{\\ \\ \hline \end{tabular}\end{center}}
%\newcommand{\codeblockfont}[1]{\parbox{0.8\textwidth}{\texttt{#1}}}


% --- Text titling/emphasis settings:
%
%     emphpara:  Emphasis for the first phrase or sentence of a 
%         paragraph.
%     booktitle:  Formats book titles.
%     tabletitle:  Title for an item block in the information table.
%     paratitle:  Title for a paragraph in an item block in the
%         information table.
%     emphdate:  Emphasize date in paragraph text.
%
%     cmd:  Commands
%     dumarg:  Dummy arguments
%     codearg:  Same as dumarg.
%     fn:  File and directory names
%     screen:  Screen display
%     vars:  Variable and attribute names
%     mods:  Module, subroutine, and method names
%     class:  Class names
%     code:  Generic code (avoid using this)
%
\newcommand{\emphpara}[1]{\textbf{#1}}
\newcommand{\booktitle}[1]{\textit{#1}}
%\newcommand{\tabletitle}[1]{\textsf{\textbf{#1}}}
\newcommand{\paratitle}[1]{\textit{#1}}
\newcommand{\emphdate}[1]{\textbf{#1}}

\newcommand{\code}[1]{\textcolor{blue}{\texttt{#1}}}
\newcommand{\cmd}[1]{\textcolor{blue}{\texttt{#1}}}
\newcommand{\dumarg}[1]{\textit{#1}}
\newcommand{\codearg}[1]{\textit{#1}}
\newcommand{\fn}[1]{\textsf{\textit{#1}}}
\newcommand{\screen}[1]{\textcolor{green}{\texttt{#1}}}
\newcommand{\vars}[1]{\textcolor{blue}{\texttt{#1}}}
\newcommand{\class}[1]{\textcolor{blue}{\texttt{#1}}}
\newcommand{\mods}[1]{\textcolor{blue}{\texttt{#1}}}


% --- Special table formatting:
%
%     tabletitlewidth:  Width for title field of an item block in the 
%         information table.
%     tablebodywidth:  Width for body field of an item block in the 
%         information table.
%     tabletabulardims:  Dimensions for the information table, used in
%         the tabular command.
%     tableitemlinespace:  Vertical spacing between item blocks in the
%         information table.
%     infotitle and infotext:  Used for two-column sub-information 
%         tables found in the body field of the information table.  
%         These are not global lengths but have values specific to the 
%         local context in which they're used.
%
\newlength{\tabletitlewidth}
\settowidth{\tabletitlewidth}{file and directory names}

\newlength{\tablebodywidth}
\setlength{\tablebodywidth}{0.9\textwidth}
\addtolength{\tablebodywidth}{-4ex}
\addtolength{\tablebodywidth}{-\tabletitlewidth}

\newcommand{\tabletabulardims}%
	{p{\tabletitlewidth}@{\hspace{4ex}}p{\tablebodywidth}}

\newcommand{\tableitemlinespace}{\baselineskip}
\newlength{\infotitle}
\newlength{\infotext}


% --- Lengths for formatting:
%
\newlength{\remainder}        % length to describe the residual of the
                              %   linewidth minus \enumlabel
\newlength{\enumlabel}        % length to describe figure sub-label width
                              %   (e.g. "(a)")


% --- TtH stuff:
%
%\def\tthdump#1{#1}


% --- LaTeX2HTML stuff:
%
%     htmlfigcaption:  Formatting for HTML replacement figure captions.
%
\newcommand{\htmlfigcaption}[1]{\parbox[c]{70ex}{\footnotesize{#1}}}


% --- Some book title abbreviations:
%
%     rute:  Booktitle for Rute User's.
%     linuxnut:  Booktitle for Linux in a Nutshell.
%     pynut:  Booktitle for Python in a Nutshell.
%
\newcommand{\rute}{\booktitle{Rute User's}}
\newcommand{\linuxnut}{\booktitle{Linux in a Nutshell}}
\newcommand{\pynut}{\booktitle{Python in a Nutshell}}


% --- Define special characters ---
%
\newcommand{\aonehat}{\ensuremath{\widehat{a_1}}}
\newcommand{\bonehat}{\ensuremath{\widehat{b_1}}}
\newcommand{\D}{\ensuremath{\mathcal{D}}}
\def\BibTeX{B\kern-.03em i\kern-.03em b\kern-.15em\TeX}




% ------ BEGINNING OF DOCUMENT TEXT ------
%
\begin{document}

    

    
% ------ TITLE AND TOC ------
%
\title{\mods{qtcm} User's Guide}
\author{Johnny Wei-Bing Lin\thanks{Physics Department, North Park University,
	3225 W.\ Foster Ave., Chicago, IL  60625, USA}}
\date{\today}
\maketitle
\tableofcontents




% ------ BODY ------
%
\chapter{Introduction}
\input{intro}

\chapter{Installation and Configuration}    \label{ch:install}
	\section{Summary and Conventions}      \label{sec:install.sum}
	\input{install_sum}
	\section{Fortran Compiler}             \label{sec:fort.compilers}
	\input{install_fort}
	\section{Required Packages}            \label{sec:py.etc.pkgs}
	\input{install_pkgs}
	\section{Compiling Extension Modules}  \label{sec:create.so}
	\input{compile_so}
	\section{Testing the Installation}     \label{sec:test.qtcm}
	\input{test_qtcm}
	\section{Model Performance}
	\input{perform}
	\section{Installing in Mac OS X}       \label{sec:install.macosx}
	\input{qtcm_in_macosx}
	\section{Installing in Ubuntu}         \label{sec:install.ubuntu}
	\input{qtcm_in_ubuntu}

\chapter{Getting Started With \mods{qtcm}}  \label{ch:getting.started}
\input{started}

\chapter{Using \mods{qtcm}}                 \label{ch:using}
\input{using}

%@@@\chapter{Combining \code{qtcm} with \code{CliMT}}
%@@@\input{climt}

\chapter{Troubleshooting}                   \label{ch:trouble}
\input{trouble}

\chapter{Developer Notes}                   \label{ch:devnotes}
\input{devnotes}

\chapter{Future Work}                       \label{ch:future}
\input{future}




% ----- BACK MATTER OF THE DOCUMENT -----
%
\normalsize
\pagebreak
\bibliographystyle{plain}
\bibliography{/Users/jlin/work/res/bib/master}

%- Uncomment the input line below and comment out the \bibliographystyle
%  and \bibliography lines if you're running this without the master.bib 
%  BibTeX database
%\input{manual.bbl}        

\appendix
\chapter{Field Settings in \mods{defaults}}  \label{app:defaults.values}
\input{defaults}




% ------ END OF DOCUMENT TEXT ------
%
\end{document}


% ===== end of file =====

%
%   in this file.  Then you can use pdflatex on this file to get the PDF of
%   the manual.  These 3 lines are in the back matter of the document.
%
% Revision Notes:
% - By Johnny Lin, North Park University, http://www.johnny-lin.com/
% - The chicago BibTeX style is unrecognized by latex2html, so I use
%   the plain style.
% ==========================================================================


% ------ DOCUMENT DEFINITIONS ------
%
\documentclass[12pt]{book}
\usepackage{color}
\usepackage{html}
\usepackage{graphicx}
\usepackage{textcomp}
%\usepackage{comment}    %- Unrecognized by latex2html; its use causes errors
%\usepackage{fancyvrb}   %- Unrecognized by latex2html; its use causes errors


%- Packages unrecognized by latex2html, but causes no error:
%
%\usepackage[letterpaper,margin=1in,includefoot]{geometry}
\usepackage[letterpaper,margin=1.25in]{geometry}
\usepackage{bibnames}
\usepackage{longtable}
\usepackage{multirow}


%+ Comment out explicity margin settings since use package geometry:
%\setlength{\topmargin}{0in}
%\setlength{\headheight}{0in}
%\setlength{\headsep}{0in}
%\setlength{\oddsidemargin}{0in}
%\setlength{\evensidemargin}{0in}
%\setlength{\textheight}{8.5in}
%\setlength{\textwidth}{6.5in}




% ------ COMMANDS AND LENGTHS ------
%
% --- Define colors:  Have to do this because for some reason LaTeX
%     sometimes looks for "BLUE" instead of "blue" and complains when
%     "BLUE" isn't found.
%
\definecolor{Blue}{rgb}{0,0,1}
\definecolor{BLUE}{rgb}{0,0,1}
\definecolor{green}{rgb}{0,0.6,0}
\definecolor{Green}{rgb}{0,0.6,0}
\definecolor{GREEN}{rgb}{0,0.6,0}


% --- Format code blocks.  Currently set to print out the code in just 
%     typewriter font with no box.  Will work the same for pdflatex 
%     and latex2html:
%
%     codeblock:  Environment for blocks of computer code or internet 
%       addresses.
%     codeblockfont:  Sets font for codeblocks.
%
\newenvironment{codeblock}%
	{\begin{quotation}\begin{minipage}[t]{0.9\textwidth}}%
	{\end{minipage}\end{quotation}}
	%{\begin{flushleft}}%
	%{\end{flushleft}}
\newcommand{\codeblockfont}[1]{\textcolor{blue}{\texttt{#1}}}
%     *** Version that only works for pdflatex that puts a box around 
%         the block and centers it (commented out).  Note that using
%         fancyvrb is the better way of creating such a boxed section
%         of code, but fancyvrb isn't recognized by latex2html:
%\newenvironment{codeblock}%
%	{\begin{center}\begin{tabular}{|c|} \hline \\ }%
%	{\\ \\ \hline \end{tabular}\end{center}}
%\newcommand{\codeblockfont}[1]{\parbox{0.8\textwidth}{\texttt{#1}}}


% --- Text titling/emphasis settings:
%
%     emphpara:  Emphasis for the first phrase or sentence of a 
%         paragraph.
%     booktitle:  Formats book titles.
%     tabletitle:  Title for an item block in the information table.
%     paratitle:  Title for a paragraph in an item block in the
%         information table.
%     emphdate:  Emphasize date in paragraph text.
%
%     cmd:  Commands
%     dumarg:  Dummy arguments
%     codearg:  Same as dumarg.
%     fn:  File and directory names
%     screen:  Screen display
%     vars:  Variable and attribute names
%     mods:  Module, subroutine, and method names
%     class:  Class names
%     code:  Generic code (avoid using this)
%
\newcommand{\emphpara}[1]{\textbf{#1}}
\newcommand{\booktitle}[1]{\textit{#1}}
%\newcommand{\tabletitle}[1]{\textsf{\textbf{#1}}}
\newcommand{\paratitle}[1]{\textit{#1}}
\newcommand{\emphdate}[1]{\textbf{#1}}

\newcommand{\code}[1]{\textcolor{blue}{\texttt{#1}}}
\newcommand{\cmd}[1]{\textcolor{blue}{\texttt{#1}}}
\newcommand{\dumarg}[1]{\textit{#1}}
\newcommand{\codearg}[1]{\textit{#1}}
\newcommand{\fn}[1]{\textsf{\textit{#1}}}
\newcommand{\screen}[1]{\textcolor{green}{\texttt{#1}}}
\newcommand{\vars}[1]{\textcolor{blue}{\texttt{#1}}}
\newcommand{\class}[1]{\textcolor{blue}{\texttt{#1}}}
\newcommand{\mods}[1]{\textcolor{blue}{\texttt{#1}}}


% --- Special table formatting:
%
%     tabletitlewidth:  Width for title field of an item block in the 
%         information table.
%     tablebodywidth:  Width for body field of an item block in the 
%         information table.
%     tabletabulardims:  Dimensions for the information table, used in
%         the tabular command.
%     tableitemlinespace:  Vertical spacing between item blocks in the
%         information table.
%     infotitle and infotext:  Used for two-column sub-information 
%         tables found in the body field of the information table.  
%         These are not global lengths but have values specific to the 
%         local context in which they're used.
%
\newlength{\tabletitlewidth}
\settowidth{\tabletitlewidth}{file and directory names}

\newlength{\tablebodywidth}
\setlength{\tablebodywidth}{0.9\textwidth}
\addtolength{\tablebodywidth}{-4ex}
\addtolength{\tablebodywidth}{-\tabletitlewidth}

\newcommand{\tabletabulardims}%
	{p{\tabletitlewidth}@{\hspace{4ex}}p{\tablebodywidth}}

\newcommand{\tableitemlinespace}{\baselineskip}
\newlength{\infotitle}
\newlength{\infotext}


% --- Lengths for formatting:
%
\newlength{\remainder}        % length to describe the residual of the
                              %   linewidth minus \enumlabel
\newlength{\enumlabel}        % length to describe figure sub-label width
                              %   (e.g. "(a)")


% --- TtH stuff:
%
%\def\tthdump#1{#1}


% --- LaTeX2HTML stuff:
%
%     htmlfigcaption:  Formatting for HTML replacement figure captions.
%
\newcommand{\htmlfigcaption}[1]{\parbox[c]{70ex}{\footnotesize{#1}}}


% --- Some book title abbreviations:
%
%     rute:  Booktitle for Rute User's.
%     linuxnut:  Booktitle for Linux in a Nutshell.
%     pynut:  Booktitle for Python in a Nutshell.
%
\newcommand{\rute}{\booktitle{Rute User's}}
\newcommand{\linuxnut}{\booktitle{Linux in a Nutshell}}
\newcommand{\pynut}{\booktitle{Python in a Nutshell}}


% --- Define special characters ---
%
\newcommand{\aonehat}{\ensuremath{\widehat{a_1}}}
\newcommand{\bonehat}{\ensuremath{\widehat{b_1}}}
\newcommand{\D}{\ensuremath{\mathcal{D}}}
\def\BibTeX{B\kern-.03em i\kern-.03em b\kern-.15em\TeX}




% ------ BEGINNING OF DOCUMENT TEXT ------
%
\begin{document}

    

    
% ------ TITLE AND TOC ------
%
\title{\mods{qtcm} User's Guide}
\author{Johnny Wei-Bing Lin\thanks{Physics Department, North Park University,
	3225 W.\ Foster Ave., Chicago, IL  60625, USA}}
\date{\today}
\maketitle
\tableofcontents




% ------ BODY ------
%
\chapter{Introduction}
%=====================================================================
% Introduction
%=====================================================================


% ----- BEGIN TEXT -----
%
%---------------------------------------------------------------------
\section{How to Read This Manual}

\emphpara{Most users:} 
Just read 
(1) the installation instructions in Chapter~\ref{ch:install},
(2) Chapter~\ref{ch:getting.started},
which tells you all you need to get started using \mods{qtcm}, and
(3) examples in Section~\ref{sec:cookbook} that give a feel
for how you can use the model.

\emphpara{Users having problems:}
Chapter~\ref{ch:trouble} provides troubleshooting tips for
a few problems.
The detailed description of how the package functions, 
in Chapter~\ref{ch:using}, will probably be more useful.

\emphpara{Developers:}
If you want to change the source code, please read
Chapter~\ref{ch:devnotes}.  Chapter~\ref{ch:future} describes
all the things I'd like to do to improve the package, but haven't
gotten to yet.




%---------------------------------------------------------------------
\section{About the Package}

The single-baroclinic mode
Neelin-Zeng Quasi-Equilibrium Tropical Circulation Model
\latexhtml{(QTCM1)\footnote{http://www.atmos.ucla.edu/$\sim$csi}}%
	{\htmladdnormallink{(QTCM1)}{http://www.atmos.ucla.edu/~csi}}
is a primitive equation-based intermediate-level atmospheric model
that focuses on simulating the tropical atmosphere.  Being more
complicated than a simple model, the model has full non-linearity
with a basic representation of baroclinic instability,
includes a radiative-convective feedback package, and includes a
simple land soil moisture routine (but does not include topography).
A brief, but more detailed, description of QTCM1 is given in
Section~\ref{sec:brief_qtcm}.

\htmladdnormallinkfoot{Python}{http://www.python.org}
is an interpreted, object-oriented, multi-platform,
open-source language that is used in a variety of software applications,
ranging from game development to bioinformatics.
In climate studies, Python has been used as the core language for
data analysis
(e.g., \htmladdnormallinkfoot{Climate Data Analysis Tools}{http://cdat.sf.net}),
visualization
(e.g., \htmladdnormallinkfoot{Matplotlib}{http://matplotlib.sf.net}),
and 
modeling
(e.g., \htmladdnormallinkfoot{PyCCSM}{http://code.google.com/p/pyccsm/}).

In comparison to traditional compiled languages like Fortran,
Python's lack of a separate compile step greatly simplifies the
debugging and testing phases of development, because code snippets
can be testing as code is written.
Python's extensive suite of higher-level tools (e.g., statistics,
visualization, string and file manipulation) accessible at runtime 
enables modeling and analysis to occur concurrently.  

The \mods{qtcm} package is an implementation of the Neelin-Zeng
QTCM1 in a Python object-oriented environment.  The conversion
package
\htmladdnormallinkfoot{\mods{f2py}}{http://cens.ioc.ee/projects/f2py2e/} is
used to wrap the QTCM1 Fortran model routines and manage model
execution using Python datatypes and utilities.  The result is a
modeling package where order and choice of subroutine execution can
be altered at runtime.  Model analysis and visualization can also
be integrated with model execution at runtime.




%---------------------------------------------------------------------
\section{Conventions In This Manual}

	\subsection{Audience}

In this manual I assume you have a rudimentary knowledge of Python.
Thus, I do not describe basic Python data types (e.g., dictionaries,
lists), object framework and syntax (e.g., classes, methods,
attributes, instantiation), module and package importing.  If you
need to brush up (or learn) Python, I'd recommend the following
resources:

\begin{itemize}
\item \htmladdnormallinkfoot{Python Tutorial:}{http://docs.python.org/tut/}
	This tutorial was written by Guido van Rossum, Python's original
	author.

\item \htmladdnormallinkfoot{Instant Hacking:}%
	{http://www.hetland.org/python/instant-hacking.php}
	Learn how to program with Python.

\item \htmladdnormallinkfoot{Dive Into Python:}%
	{http://diveintopython.org/index.html}
	Reasonably complete and cohesive. The entire book is available for 
	free online.

\item \htmladdnormallinkfoot{Handbook of the Physics Computing Course:}%
	{http://www.pentangle.net/python/handbook/}
	Written for a science audience.  Recommended.

\item \latexhtml{CDAT/Python Tips for Earth Scientists:\footnote%
	{http://www.johnny-lin.com/cdat\_tips/}}%
	{\htmladdnormallink{CDAT/Python Tips for Earth Scientists:}%
		{http://www.johnny-lin.com/cdat_tips/}}
	This web site is a FAQ of sorts for people using Python and
	the Climate Data Analysis Tools (CDAT) in the earth sciences,
	and thus focuses on using Python to do science rather than
	the computer science aspects of the language.

\end{itemize}

The purpose of this package is to make the QTCM1 model easier to
use.  In order to interpret the results, however, you still need
to have a robust sense of what climate models can and cannot tell
you.  A starting point for the QTCM1 model is the brief description
of the model in Section~\ref{sec:brief_qtcm}.  After that, I would
read the original papers describing the model formulation and results
\cite{Neelin/Zeng:2000,Zeng/etal:2000}, and 
\latexhtml{papers citing the model formulation work.\footnote%
{http://scholar.google.com/scholar?hl=en\&lr=\&cites=14217886709842286738}}%
{\htmladdnormallink{papers citing the model formulation work}%
{http://scholar.google.com/scholar?hl=en&lr=&cites=14217886709842286738}.}
Being an intermediate-level model using the quasi-equilibrium assumption,
QTCM1 (and thus \mods{qtcm}) has distinct strengths and limitations; 
please be aware of them.


	\subsection{Typographic Conventions}

\begin{center}
\begin{tabular}{\tabletabulardims}
\cmd{commands} & to be typed at the command-line
	are rendered in a 
	blue, serif, fixed-width typewriter font
	(e.g., \cmd{make \_qtcm\_full\_365}). \\ \hline
\dumarg{dummy arguments} &
	coupled with code or screen display is rendered in a 
	serif, proportional, italicized font
	(e.g., \screen{Error-Value too long in} \dumarg{variable}). \\ \hline
\fn{file and directory names} & are rendered in a 
	sans-serif, italicized font
	(e.g., \fn{setbypy.F90}). \\ \hline
\screen{screen display} & is rendered in a 
	green, serif, fixed-width typewriter font. \\ \hline
\mods{module, method, and subroutine names} & are rendered in a 
	blue, serif, fixed-width typewriter font. \\ \hline
\vars{variable and attribute names} & are rendered in a 
	blue, serif, fixed-width typewriter font. \\ \hline
\class{class names} & are rendered in a 
	blue, serif, fixed-width typewriter font.
\end{tabular}
\end{center}

Blocks of code (usually commands, screen display, and module,
variable, and class names) are displayed in a blue, serif, fixed-width
typewriter font.


	\subsection{Terminology}

\begin{description}
\item[attribute and method references:]
	If there is any possibility of confusion, I will give the
	class that an attribute or method comes from when that
	attribute or method is referenced.  If no class is mentioned
	by name or context,
	assume that the attribute/method comes from the
	\class{Qtcm} class.

\item[``compiled QTCM1 model'':]
	This usually is used to denote when I'm talking about
	compiled Fortran QTCM1 routines and variables therein,
	as an extension module to the Python \mods{qtcm} package..
	Thus, ``compiled QTCM1 model \vars{u1}'' is the value
	of variable \vars{u1} in the Fortran model, not the
	value at the Python-level.  Sometimes I refer to the
	compiled QTCM1 model as just ``QTCM1'' or as
	``compiled QTCM1 Fortran model''.

\item[``pure-Fortran QTCM1'':]
	This refers to the Neelin-Zeng QTCM1 model in it's
	original Fortran form, not as an extension module to
	the Python \mods{qtcm} package.

\item[``Python-level'':]
	This usually denotes the value of a variable as an
	attribute of a \class{Qtcm} instance.  This variable
	is stored at the Python interpreter level.

\item[\class{Qtcm}:]
	This refers to the Python \class{Qtcm} class
	(note the capitalized first letter).

\item[\mods{qtcm}:]
	This refers to the Python \mods{qtcm} package.

\item[QTCM1 vs.\ QTCM:]
	Although the QTCM1 is currently the only version of a
	quasi-equilibrium tropical circulation model (QTCM), in
	principle one can construct a QTCM with any number of
	baroclinic modes.  In anticipation of this, the names of
	the Python package and class do not end in a numeral.  In
	this manual and the \mods{qtcm} docstrings, QTCM and QTCM1
	are used interchangably.
	Usually either of these phrases mean the quasi-equilibrium
	tropical circulation model in a generic sense, regardless
	of its form of implementation.
\end{description}




%---------------------------------------------------------------------
\section{Current Version Information and Acknowledgments}  \label{sec:ver}

\input{pkg_version_date.tex}
\input{pkg_author.tex}is the primary author of the package.

The \mods{qtcm} package is built upon the pure-Fortran QTCM1 model,
version 2.3 (August 2002), with a few minor changes.
Those changes are described in detail in
Section~\ref{sec:f90changes}.

The homepage for the \mods{qtcm} package is
\htmladdnormallink{http://www.johnny-lin.com/py\_pkgs/qtcm}%
	{http://www.johnny-lin.com/py_pkgs/qtcm}.
All Python code in this package, 
and the Fortran files \fn{setbypy.F90} and \fn{wrapcall.F90},
are \copyright\ 2003--2008 by 
\htmladdnormallinkfoot{Johnny Lin}%
		{http://www.johnny-lin.com} 
and constitutes a
library that is covered under the GNU Lesser General Public License
(LGPL):

\begin{quotation}
	This library is free software; you can redistribute it
	and/or modify it under the terms of the 
	\htmladdnormallinkfoot{GNU Lesser General Public License}%
		{http://www.gnu.org/copyleft/lesser.html} 
	as published by
	the Free Software Foundation; either version 2.1 of the
	License, or (at your option) any later version.

	This library is distributed in the hope that it will be
	useful, but WITHOUT ANY WARRANTY; without even the implied
	warranty of MERCHANTABILITY or FITNESS FOR A PARTICULAR
	PURPOSE. See the GNU Lesser General Public License for more
	details.

	You should have received a copy of the GNU Lesser General
	Public License along with this library; if not, write to
	the Free Software Foundation, Inc., 59 Temple Place, Suite
	330, Boston, MA 02111-1307 USA.

	You can contact Johnny Lin at his email address 
	or at North Park University, Physics Department,
	3225 W. Foster Ave., Chicago, IL 60625, USA.  
\end{quotation}

All other Fortran code in this package, as well as the makefiles,
are covered by licenses (if any) chosen by their respective owners.

This manual, in all forms (e.g., HTML, PDF, \LaTeX),
is part of the documentation of the \mods{qtcm} package 
and is \copyright\ 2007--2008 by Johnny Lin.
Permission is granted to copy, distribute and/or modify 
this document under the terms of the 
GNU Free Documentation License, Version 1.2 
or any later version published by the Free Software Foundation; 
with no Invariant Sections, no Front-Cover Texts, 
and no Back-Cover Texts. 
A copy of the license can be found 
\htmladdnormallinkfoot{here}{http://www.gnu.org/licenses/fdl.html}.

Transparent copies of this document are located online in
\latexhtml{%
\htmladdnormallinkfoot{PDF}%
	{http://www.johnny-lin.com/py\_pkgs/qtcm/doc/manual.pdf}}%
{\htmladdnormallink{PDF}%
	{http://www.johnny-lin.com/py_pkgs/qtcm/doc/manual.pdf}}
and
\latexhtml{%
\htmladdnormallinkfoot{HTML}%
	{http://www.johnny-lin.com/py\_pkgs/qtcm/doc/}}%
{\htmladdnormallink{HTML}%
	{http://www.johnny-lin.com/py_pkgs/qtcm/doc/}}
formats.
The \LaTeX\ source files are distributed with the \mods{qtcm}
distribution.
While the HTML version is nearly identical to the PDF
and \LaTeX\ versions, not every feature in the manual was successfully
converted.  This is particularly true with figures, which are
unnumbered in the HTML version and may be formatted differently
than the authoritative PDF version.
Thus, please consider the \LaTeX\ version as the authoritative
version.

\vspace{\baselineskip}

\emphpara{Acknowledgements:}
Thanks to David Neelin and Ning Zeng and the Climate Systems
Interactions Group at UCLA for their encouragement and help.
On the Python side,
thanks to Alexis Zubrow, Christian Dieterich, Rodrigo Caballero,
Michael Tobis, and Ray Pierrehumbert for steering me straight.
Early versions of some of this work was carried out 
at the University of Chicago Climate Systems Center, 
funded by the National Science Foundation (NSF) 
Information Technology Research Program under grant ATM-0121028. 
Any opinions, findings and conclusions or recommendations 
expressed in this material are those of the author and 
do not necessarily reflect the views of the NSF.

Intel\textregistered\ and
   Xeon\textregistered\ are registered trademarks of Intel Corporation.
Matlab\textregistered\ is a registered trademark of The MathWorks.
UNIX\textregistered\ is a registered trademark of The Open Group.




%---------------------------------------------------------------------
\section{Summary of Release History}

\begin{itemize}
\item 2008 Sep 12:  Version 0.1.2 released.  Summary of changes:
	\begin{itemize}
	\item Create \class{Qtcm} method \mods{get\_qtcm1\_item}.
		This method is effectively an alias of method 
		\mods{get\_qtcm\_item}.
	\item Create \class{Qtcm} method \mods{set\_qtcm1\_item}.
		This method is effectively an alias of method 
		\mods{set\_qtcm\_item}.
	\item Update User's Guide to phase out references to
		the \mods{get\_qtcm\_item}
		and \mods{set\_qtcm\_item} methods.  
		Adding the ``1'' to the method names makes the purpose
		of the methods clearer.
	\item Add unit tests to cover methods \mods{get\_qtcm1\_item} and
		\mods{set\_qtcm1\_item}.
	\end{itemize}

\item 2008 Jul 30:  Updates to the User's Guide (just the online versions,
        not the copies released with the tarball).

\item 2008 Jul 15:  First publicly available distribution 
	released (v0.1.1).
\end{itemize}




%---------------------------------------------------------------------
\section{A Brief Description of The QTCM1}   \label{sec:brief_qtcm}

This description is copied from Ch.\ 3 of Lin \cite{Lin:2000}, 
with minor revisions.
Model formulation is fully described in
Neelin \& Zeng \cite{Neelin/Zeng:2000} and model
results are described in Zeng et~al.\ \cite{Zeng/etal:2000}.
Neelin \& Zeng \cite{Neelin/Zeng:2000} is based upon v2.0 of QTCM1
and Zeng et~al.\ \cite{Zeng/etal:2000} is based on QTCM1 v2.1.
The 
\latexhtml{%
\htmladdnormallinkfoot{QTCM1 manual}%
	{http://www.atmos.ucla.edu/$\sim$csi/qtcm\_man/v2.3/qtcm\_manv2.3.pdf}}%
{\htmladdnormallink{QTCM1 manual}%
	{http://www.atmos.ucla.edu/~csi/qtcm_man/v2.3/qtcm_manv2.3.pdf}}
\cite{Neelin/etal:2002}
describes the details of model implementation.

QTCM1 differs from most full-scale GCMs primarily in how the vertical
temperature, humidity, and velocity structure of the atmosphere is
represented.  First, instead of representing the vertical structure
by finite-differenced levels, the model uses a Galerkin expansion
in the vertical.  The vertical basis functions are chosen according
to analytical solutions under convective quasi-equilibrium conditions,
so only a few need be retained.
Temperature and humidity are each described by separate
vertical basis functions ($a_1$ and $b_1$, respectively).
Low-level variations in the humidity basis
are larger than in the temperature basis.
For velocity, QTCM1 uses a single baroclinic basis function ($V_1$)
defined consistently with the temperature basis function,
as well as a barotropic velocity mode ($V_0$).
The vertical profiles of $a_1$, $b_1$, and $V_1$
are given in Figure~\ref{fig:qtcm.basis}.
Currently, QTCM1 does not include a separate
vertical degree of freedom describing the PBL.
The horizontal grid spacing of the model is 
$5.625^{\circ}$ longitude by $3.75^{\circ}$ latitude.


% <QTCM1 beta version vertical structure modes>
%
% (1) LaTeX version:
%
\begin{latexonly}
\begin{figure}
   \noindent
   \begin{minipage}[b]{.49\linewidth}
      \settowidth{\enumlabel}{(a) }%
      \setlength{\remainder}{\linewidth}% 
      \addtolength{\remainder}{-\enumlabel}
      {(a)}~\makebox[\remainder]{$a_1$ and $b_1$}
      \centering\includegraphics[width=\linewidth,viewport=58 72 389 344,clip]%
                    {figs/a1b1.pdf}
   \end{minipage}\hfill
   \begin{minipage}[b]{.49\linewidth}
      \settowidth{\enumlabel}{(b) }%
      \setlength{\remainder}{\linewidth}% 
      \addtolength{\remainder}{-\enumlabel}
      {(b)}~\makebox[\remainder]{$V_1$}

      \centering\includegraphics[width=\linewidth,viewport=58 72 389 346,clip]%
                    {figs/V1.pdf}
   \end{minipage}

   \caption{Vertical profiles of basis functions for
		(a) temperature $a_1$ (solid) and humidity $b_1$ (dashed) and
		(b) baroclinic component of
		horizontal velocity $V_1$.}
   \label{fig:qtcm.basis}
\end{figure}
\end{latexonly}

% (2) HTML replacement version:
%
\begin{htmlonly}
\label{fig:qtcm.basis}
\begin{center}
\htmladdimg{../latex/figs/a1b1.png}
\htmladdimg{../latex/figs/V1.png}

\htmlfigcaption{Figure \ref{fig:qtcm.basis}:  
	Vertical profiles of basis functions for
   	(a) temperature $a_1$ (solid) and humidity $b_1$ (dashed) and
   	(b) baroclinic component of
   	horizontal velocity $V_1$.}
\end{center}
\end{htmlonly}


These modes are chosen to accurately capture deep convective regions.
Outside deep convective regions the mode
is simply a highly truncated
Galerkin representation.  The system is much more tightly constrained than
a full-scale GCM, yet hopefully retains the essential dynamics and nonlinear
feedbacks.  The result is that QTCM1 is easier to diagnose than a GCM,
and is computationally fast (about 8 minutes per year on a Sun Ultra 2
workstation).  Zeng et al.\ \cite{Zeng/etal:2000} show results indicating
this intermediate-level model does a reasonable job simulating
tropical climatology and ENSO variability.  


Below is a summary of the main model equations \cite{Neelin/Zeng:2000}:
\begin{equation}
   \partial_t \mathbf{v}_1 
      + \D_{V1} (\mathbf{v}_0 , \mathbf{v}_1)
      + f \mathbf{k} \times \mathbf{v}_1
      =
   - \kappa \nabla T_1 
      - \epsilon_1 \mathbf{v}_1 
      - \epsilon_{01} \mathbf{v}_0
   \label{eqn:barocl_wind}
\end{equation}
\begin{equation}
   \partial_t \zeta_0 
      + \mathrm{curl}_z (\D_{V0} (\mathbf{v}_0 , \mathbf{v}_1))
      + \beta v_0
      =
   - \mathrm{curl}_z (\epsilon_0 \mathbf{v}_0)
      - \mathrm{curl}_z (\epsilon_{10} \mathbf{v}_1)
   \label{eqn:barotr_wind}
\end{equation}
\begin{equation}
   \aonehat (\partial_t + \D_{T1}) T_1 
      + M_{S1} \nabla \cdot {\bf v}_1 
      =
   \langle Q_c \rangle
      + (g/p_T) (-R^\uparrow_t -R^\downarrow_s + R^\uparrow_s + S_t - S_s + H)
   \label{eqn:temperature}
\end{equation}
\begin{equation}
   \bonehat (\partial_t + \D_{q1}) q_1 
      - M_{q1} \nabla \cdot {\bf v}_1 
      =
   \langle Q_q \rangle
      + (g/p_T) E
   \label{eqn:moisture}
\end{equation}
where (\ref{eqn:barocl_wind}) describes the baroclinic wind component,
      (\ref{eqn:barotr_wind}) describes the barotropic wind component,
      (\ref{eqn:temperature}) is the temperature equation, and
      (\ref{eqn:moisture}) is the moisture equation.

In the simplest formulation, the vertically integrated
convective heating and moisture sink
are assumed to be equal and opposite, so:
\begin{equation}
  -\langle Q_q \rangle = \langle Q_c \rangle 
                              = \epsilon^\ast_c (q_1 - T_1)
\end{equation}

For its convective parameterization for $Q_c$, this model uses the
Betts-Miller \cite{Betts/Miller:1986} moist convective
adjustment scheme, a scheme that is also used in some GCMs.
In the convective parameterization, the coefficient
$\epsilon^\ast_c$ is defined as:
\begin{equation}
   \epsilon^\ast_c 
      \equiv 
   \aonehat \bonehat (\aonehat + \bonehat)^{-1} \tau_c^{-1} 
      \mathcal{H}( \mathit{C}_{\mathrm{1}} )
\end{equation}
where $\mathcal{H}( \mathit{C}_{\mathrm{1}} )$ is zero for
$C_{1} \leq 0$, and one for $C_{1} > 0$, and $C_{1}$
is a measure of the convective available potential energy (CAPE),
projected onto the model's temperature and moisture basis functions.

Sensible heat ($H$) and evaporation ($E$) are given as
bulk-aerodynamic formulations:
\begin{equation}
   H
      =
   \rho_a C_D \mathrm{V}_s (T_s - (T_{rs} + a_{1s} T_1))
\end{equation}
\begin{equation}
   E
      =
   \rho_a C_D \mathrm{V}_s (q_\mathit{sat} (T_s) 
      - (q_{rs} + b_{1s} q_1))
\end{equation}

Longwave radiation components are denoted by $R$, and net shortwave
radiation is denoted by $S$.
The terms $\D_{V1}$ and $\D_{V0}$ are the advection-diffusion operators
for the momentum equations (projected onto $V_0$ and $V_1 (p)$,
respectively).
The terms $\D_{T1}$ and $\D_{q1}$ are the
advection-diffusion operators for the temperature and moisture
equations, respectively, using a vertical average projection.
The $\langle X \rangle$ and $\widehat{X}$ operators are
equivalent and denote vertically integration over the troposphere.
Please see Neelin \& Zeng \cite{Neelin/Zeng:2000} and 
Zeng et al.\ \cite{Zeng/etal:2000}
for a more complete description of equations and coefficients.







% ====== end file ======


\chapter{Installation and Configuration}    \label{ch:install}
	\section{Summary and Conventions}      \label{sec:install.sum}
	% ==========================================================================
% Installation Summary
%
% By Johnny Lin
% ==========================================================================


% ------ BODY -----
%

This section provides a summary of the steps needed to install
\mods{qtcm}, and a description of the naming conventions used in
this chapter.  If you have had a decent amount of experience with
Python and installing software on a Unix system, this section will
probably be all you need to read.  The installation steps are:

\begin{enumerate}
\item Install a Fortran compiler (see Section~\ref{sec:fort.compilers}
	for a list of compilers known to work).
	This compiler should be in a directory
	listed in your system path (e.g., \fn{/usr/bin}, etc.).

\item Install all required packages
	(see Section~\ref{sec:py.etc.pkgs} for details):
	Python,
	\mods{matplotlib} (plus the \mods{basemap} toolkit),
	NumPy (which includes \mods{f2py}),
	Scientific Python,
	\LaTeX,
	and
	netCDF.

	Python packages are required to be installed on your
	system in a directory listed in your \vars{sys.path},
	and the other packages/libraries are required to be in 
	standard directories listed in your system path 
	(e.g., \fn{/usr/bin}, \fn{/sw/include}, etc.).

	Make sure the executable for Python can be called at the
	Unix command line by typing both \cmd{python}.
	You might need to define a Unix alias
	that maps \cmd{python2.4} (or whichever version of Python
	you are using) to \cmd{python}.

\item \latexhtml{Download\footnote{http://www.johnny-lin.com/py\_pkgs/qtcm/}}%
        {\htmladdnormallink{Download}{http://www.johnny-lin.com/py_pkgs/qtcm/}}
	the \mods{qtcm} tarball and extract the distribution
	into a temporary directory for building purposes.
	\fn{\input{pkg_distro_dirname}}is the name of
	the \mods{qtcm} distribution directory;
	the number following the hyphen is the
	version number of the distribution.  \label{list:download.qtcm.sum}

	In this manual, the path to \fn{\input{pkg_distro_dirname}}will
	be called the ``\mods{qtcm} build path'' and be given as
	\fn{/buildpath}.  When you see \fn{/buildpath}, please substitute
	the actual temporary directory you created for building purposes.

\item The \mods{qtcm} distribution directory 
	\fn{\input{pkg_distro_dirname}}contains the following 
	principal sub-directories:
	\fn{doc}, \fn{lib}, \fn{src}, \fn{test}.
	Documentation is in \fn{doc},
	all the package modules are in \fn{lib},
	building of extension modules will take place in \fn{src},
	and testing of the package is done in \fn{test}.

\item Compile \mods{qtcm} extension modules in \fn{src}:
	Go to \fn{src}, copy the makefile from
	\fn{src/Makefiles} corresponding to your
	system into \fn{src}, rename to \fn{makefile},
	make changes to the makefile as needed,
	and execute:
	\begin{codeblock}
	\codeblockfont{%
	make clean \\
	make \_qtcm\_full\_365.so \\
	make \_qtcm\_parts\_365.so}
	\end{codeblock}
	If you executed the make commands in \fn{src,},
	the extension modules will be automatically placed in
	\fn{lib} in the \fn{\input{pkg_distro_dirname}}directory.
	See Section~\ref{sec:create.so} for details.
	\label{list:compile.so.sum}

\item Copy the entire contents of \fn{lib} in
	\fn{\input{pkg_distro_dirname}}(not \fn{lib} itself) 
	to a directory named
	\fn{qtcm} that is on your \mods{sys.path}.  For instance,
	for Mac OS X using Fink,
	many Python packages are located in a directory
	named \fn{/sw/\-lib/\-python2.4/\-site-packages}, or something
	similar, and this directory is on the system \mods{sys.path}.  
	If this is the case for your system, copy the
	contents of \fn{lib} into
	\fn{/sw/lib/\-python2.4/\-site-packages/\-qtcm}.
	(For Unix systems, the equivalent directory is usually
	\fn{/usr/\-local/\-lib/\-python2.4/\-site-packages}.)

\item Test the \mods{qtcm} distribution in \fn{test}:
	This step is optional and can take a while.
	Testing requires you to first generate a suite of benchmarks
	using the pure-Fortran QTCM1 model, then running the tests of
	\mods{qtcm} by typing:
	\begin{codeblock}
	\codeblockfont{%
python test\_all.py}
	\end{codeblock}
	at the Unix command line while in \fn{test}.
	See Section~\ref{sec:test.qtcm} for details.

\end{enumerate}

At some point, I will automate the installation using Python's
\htmladdnormallinkfoot{\mods{distutils}}{http://docs.python.org/dist/dist.html}
utilities.



% ===== end of file =====

	\section{Fortran Compiler}             \label{sec:fort.compilers}
	% ==========================================================================
% Fortran compilers
%
% By Johnny Lin
% ==========================================================================


% ------ BODY -----
%

You must have a Fortran compiler installed on your system in order
to compile \mods{qtcm}.  The compiler must be able to interface with
a pre-processor, as QTCM1 makes copious use of pre-processor directives.
\mods{qtcm} is known to work with the following Fortran compilers on the
following platforms:

\begin{center}
\begin{tabular}{l|l|l}
\textbf{Compiler}  & \textbf{Compiler Web Site} & \textbf{Platform(s)} \\ 
\hline
\mods{g95} & \htmladdnormallink{http://www.g95.org/}{http://www.g95.org/}  
	& Mac OS X \\
\end{tabular}
\end{center}

It will probably work with other platforms, but I haven't been able
to test platforms besides those listed above.  Note that \mods{g95}
is not \htmladdnormallink{GNU Fortran}{http://gcc.gnu.org/fortran/}
(\mods{gfortran}), the Fortran 95 compiler included with the more
recent versions of GCC.




% ===== end of file =====

	\section{Required Packages}            \label{sec:py.etc.pkgs}
	% ==========================================================================
% Python packages
%
% By Johnny Lin
% ==========================================================================


% ------ BODY -----
%

The following Python packages are required to be installed on your
system in a directory listed in your \vars{sys.path}:
\begin{itemize}
\item \htmladdnormallinkfoot{Python}%
	{http://www.python.org/}:  The Python programming language
	and interpreter.  Make sure you have a version recent enough
	to be compatible with all the needed Python packages.
\item \htmladdnormallinkfoot{\mods{matplotlib}}%
	{http://matplotlib.sourceforge.net/}:  Scientific plotting
	package, using Matlab-like syntax.  The \mods{basemap} toolkit
	for \mods{matplotlib} must also be installed.
\item \htmladdnormallinkfoot{NumPy}%
	{http://numpy.scipy.org/}:  The standard array package for
	Python.  The module name of NumPy imported in a Python 
	session is \mods{numpy}.
\item \htmladdnormallinkfoot{Scientific Python}%
	{http://dirac.cnrs-orleans.fr/plone/software/scientificpython/}:
	Has netCDF file operators, in addition to other routines
	of use in scientific computing.  The module name of
	Scientific Python imported in a Python session is
	\mods{Scientific}.
\end{itemize}

One other required Python package, \mods{f2py}, is now a part of the
NumPy package, and so installation of NumPy is sufficient to give
you both.

The package \htmladdnormallinkfoot{SciPy}{http://www.scipy.org},
which includes several Python-accessible scientific libraries, also
includes NumPy (and thus \mods{f2py}), so if you install SciPy,
you don't have to install NumPy again.  Note that SciPy is not the
same as Scientific Python; the names are confusing.

A few non-Python packages are also required:
\begin{itemize}
\item \LaTeX: A scientific typesetting program used by the 
	\class{Qtcm} instance method \mods{plotm} to handle 
	exponents and subscripts.  The most common Unix 
	distribution of \LaTeX\ is
	\htmladdnormallinkfoot{teTeX}{http://www.tug.org/teTeX}.

\item netCDF:  This set of libraries enables one to write datasets into
	a platform independent, binary format, with metdata attached.
	The \htmladdnormallinkfoot{netCDF 3.6.2 library}%
        	{http://www.unidata.ucar.edu/software/netcdf/}
	source code can be
\latexhtml{downloaded from UCAR\footnote{http://www.unidata.ucar.edu/downloads/netcdf/netcdf-3\_6\_2/}}%
        {\htmladdnormallink{downloaded from UCAR}{http://www.unidata.ucar.edu/downloads/netcdf/netcdf-3_6_2/}}.
\end{itemize}

For most Unix installations, the easiest way to install all the
above is via a package manager, for instance \mods{apt-get} in
Debian GNU/Linux, \mods{aptitude} or \mods{synaptic} in Ubuntu
GNU/Linux, and \mods{fink} in Mac OS X.  Of course, you can also
download a package's source code and build direct and/or install
using Python's
\htmladdnormallinkfoot{\mods{distutils}}{http://docs.python.org/dist/dist.html}
utilities.




% ===== end of file =====

	\section{Compiling Extension Modules}  \label{sec:create.so}
	% ==========================================================================
% Compiling extension modules
%
% By Johnny Lin
% ==========================================================================


% ------ BODY -----
%

The extension modules (\fn{.so} files) are imported and used by
\mods{qtcm} objects, and contain the Fortran QTCM1 model that is
called by the \mods{qtcm} Python wrappers.  These extension modules
are located in the \fn{lib} directory of the \mods{qtcm} distribution,
and, in general, need to be created only when the \mods{qtcm} package
is installed.

Two extension modules are created:  \fn{\_qtcm\_full\_365.so} and
\fn{\_qtcm\_parts\_365.so}.  Both modules define QTCM1 models where:

\begin{itemize}
\item A year is 365 days long 
	(makefile macro \vars{YEAR360} is off).
\item Model output is written to netCDF files
	(makefile macro \vars{NETCDFOUT} is on).
\item The atmospheric boundary layer model is used
	(makefile macro \vars{NO\_ABL} is off).
\item A global domain is used
	(makefile macro \vars{SPONGES} is off).
\item Topography effects due to induced divergence are not included
	(makefile macro \vars{TOPO} is off).
\item Coupling between atmosphere and ocean is through mean fluxes
	(makefile macro \vars{CPLMEAN} is off).
\item The mixed layer ocean model is not used
	(makefile macros \vars{MXL\_OCEAN} and \vars{BLEND\_SST} are both off).
\end{itemize}

(All other makefile macros not listed are also turned off.)
The only difference between these two extension modules is that the
``full'' module is used by \class{Qtcm} instances where
\vars{compiled\_form} is set to \vars{'full'}, and the ``parts''
module is used by \class{Qtcm} instances where \vars{compiled\_form}
is set to \vars{'parts'}.  See Section~\ref{sec:compiledform} for
details about the \vars{compiled\_form} attribute.

The extension modules are created through the following steps:
\begin{enumerate}
\item Go to the \mods{qtcm} distribution directory
	\fn{\input{pkg_distro_dirname}}located in
	your build path \fn{/buildpath}.  Go to the \fn{src}
	sub-directory.  This is where all the building of the
	extension modules will take place.

\item Copy the makefile that corresponds to your platform to
	the \fn{src} directory, and rename it \fn{makefile}.
	The \fn{Makefiles} sub-directory of \fn{src} contains
	makefiles for various platforms.

\item In \fn{makefile}, make the following changes:
	\begin{enumerate}
	\item Change the \vars{FC} environment variable as needed, 
		if your Fortran compiler is different.
	\item Change the \vars{FFLAGSM} environment variable, if the
		compiler flags listed are not supported by your
		compiler.
	\item Change the \vars{-I} and \vars{-L} parts of the
		\vars{NCINC} and \vars{NCLIB} environment
		variables so that the paths for the netCDF library and
		include files match your system's installation:
		\begin{codeblock}
		\codeblockfont{%
NCINC=-I/yourpath/netcdf/include \\
NCLIB=-L/yourpath/netcdf/lib -lnetcdf}
		\end{codeblock}
		Set \dumarg{yourpath} to the full path to the
		\fn{netcdf} directory where the \fn{include} and
		\fn{lib} sub-directories are that hold the netCDF
		libraries and include files.
		(You shouldn't have to change the \vars{-l} part of
		\vars{NCLIB}, since it is standard to name the netCDF
		library \fn{libnetcdf.a}.  But if you have a non-standard
		installation, change the \vars{-l} part too.)
	\end{enumerate}

\item At the Unix prompt, type:
\begin{codeblock}
\codeblockfont{%
\small
make clean \&\& make \_qtcm\_full\_365.so \&\& make \_qtcm\_parts\_365.so}
\end{codeblock}
	to clean up leftover files from previous compilations, and to
	compile the extension module shared object files
	\fn{\_qtcm\_full\_365.so} and \fn{\_qtcm\_parts\_365.so}.
\end{enumerate}

The makefile will automatically move the shared object files into
\fn{../lib}, overwriting any pre-existing files of the same name.
A detailed description of the makefile and using \mods{f2py} is
given in Section~\ref{sec:create.new.so}, if you wish to create a
different extension module.




% ===== end of file =====

	\section{Testing the Installation}     \label{sec:test.qtcm}
	% ==========================================================================
% Installation Summary
%
% By Johnny Lin
% ==========================================================================


% ------ BODY -----
%

The \mods{qtcm} distribution comes with a set of tests for the
package, using Python's \mods{unittest} package.  
Just to warn you, the tests take around an hour to run.
The tests will not work if the contents of \fn{lib}
after you've finished building \mods{qtcm} have not been copied
to a directory named \fn{qtcm} that is on your \mods{sys.path} path,
so make sure you've gone through all the install steps
(summarized in Section~\ref{sec:install.sum}) before you do these
tests.

\emphpara{NB:}  For these tests to work, both \cmd{python} and
\cmd{python2.4} must refer to the executable for the Python
installation on your system that you are using for running \mods{qtcm}.

The tests require a set of benchmark output files in the
\fn{test/benchmarks} directory in the
\fn{\input{pkg_distro_dirname}}directory (the output will be in
directories whose names begin with ``aquaplanet'' or ``landon'').
These output files are not included with the \mods{qtcm} distribution,
and must be created, by doing the following:

\begin{enumerate}
\item Goto the directory \fn{test/benchmarks/create/src} in the
	\fn{\input{pkg_distro_dirname}}\mods{qtcm} distribution directory,
	and copy the makefile from sub-directory \fn{Makesfiles},
	that corresponds to your system to the
	\fn{test/benchmarks/create/src} directory.  Rename the makefile 
	in \fn{test/benchmarks/create/src} to \fn{makefile}.

\item In \fn{makefile}, make the following changes:
        \begin{enumerate}
        \item Change the \vars{FC} environment variable as needed,
                if your Fortran compiler is different.
        \item Change the \vars{FFLAGSM} environment variable, if the
                compiler flags listed are not supported by your
                compiler.
        \item Change the \vars{-I} and \vars{-L} parts of the
                \vars{NCINC} and \vars{NCLIB} environment
                variables so that the paths for the netCDF library and
                include files match your system's installation:
                \begin{codeblock}
                \codeblockfont{%
NCINC=-I/yourpath/netcdf/include \\
NCLIB=-L/yourpath/netcdf/lib -lnetcdf}
                \end{codeblock}
                Set \dumarg{yourpath} to the full path to the
                \fn{netcdf} directory where the \fn{include} and
                \fn{lib} sub-directories are that hold the netCDF
                libraries and include files.
                (You shouldn't have to change the \vars{-l} part of
                \vars{NCLIB}, since it is standard to name the netCDF
                library \fn{libnetcdf.a}.  But if you have a non-standard
                installation, change the \vars{-l} part too.)
        \end{enumerate}

\item Go to the directory \fn{test/benchmarks/create} in the
	\fn{\input{pkg_distro_dirname}}\mods{qtcm} distribution directory.

\item Type \cmd{python create\_benchmarks.py} at the Unix command line
	to run the benchmark creation script.
\end{enumerate}

The created benchmarks will be located in 
\fn{test/benchmarks}, in directories with names related to the
run that was done, as described earlier.
The benchmarks are created using the
pure-Fortran QTCM1 model code,
version 2.3 (August 2002), with an altered makefile
(described above) and the following code change:
In all \fn{.F90} files, occurrences of:
        \begin{codeblock}
        \codeblockfont{%
        Character(len=130)}
        \end{codeblock}
        are changed to:
        \begin{codeblock}
        \codeblockfont{%
        Character(len=305)}
        \end{codeblock}
This enables the model to properly deal with longer filenames.
The number ``305'' is chosen to make search and replace easier.

Once the benchmarks are created, you can test the \mods{qtcm} package
by doing the following:
\begin{enumerate}
\item Go to the \fn{test} directory in the 
	\fn{\input{pkg_distro_dirname}}directory.
\item Type \cmd{python test\_all.py} at the Unix command line.
\end{enumerate}

If at the end of the test runs you see this message (or something similar):
\begin{codeblock}
\codeblockfont{%
\footnotesize
---------------------------------------------------------------------- \\
Ran 93 tests in 1244.205s \\
 \\
OK}
\end{codeblock}
then everything worked fine!  If you get any other message, the test(s) have
failed.



% ===== end of file =====

	\section{Model Performance}
	%=====================================================================
% Model Performance
%=====================================================================


% ----- BEGIN TEXT -----
%
%---------------------------------------------------------------------

The wall-clock time values below give the mean over three
separate 365 day aquaplanet runs,
using climatological sea surface temperature for lower boundary forcing.
NetCDF output is written daily, for both instantaneous and mean values.
The time step is 1200~sec, and the version of \mods{qtcm} used
is 0.1.1.
The horizontal grid spacing of all model versions is
$5.625^{\circ}$ longitude by $3.75^{\circ}$ latitude.
Values are in seconds:
\begin{center}
\begin{tabular}{p{0.5\linewidth}|c|c|c}
\textbf{System} & \textbf{Pure} & \textbf{Full} & \textbf{Parts} \\
\hline
Mac OS X:  MacBook 1.83 GHz Intel Core Duo running Mac OS X
	10.4.10 with 1 GB RAM
	(Python 2.4.3, NumPy 1.0.3, \mods{f2py} 2\_3816).
    & 152.59 & 153.63 & 158.94 \\
\hline
Ubuntu GNU/Linux:  Dell PowerEdge 860 with 2.66 GHz Quad Core Intel
	Xeon processors (64 bit) running Ubuntu 8.04.1 LTS
	(Python 2.5.2, NumPy 1.1.0, \mods{f2py} 2\_5237).
    & 43.73 & 44.79 & 47.45
\end{tabular}
\end{center}

``Pure'' refers to the pure-Fortran version of QTCM1.
``Full'' refers to a \mods{qtcm} run session with \vars{compiled\_form}
set to \vars{'full'}.  ``Parts'' refers to a \mods{qtcm} run session
with \vars{compiled\_form} set to \vars{'parts'}.
(Section~\ref{sec:compiledform} has details about the difference
between compiled forms.)

The \vars{'parts'} version of \mods{qtcm} gives Python the maximum
flexibility in accessing compiled QTCM1 model subroutines and
variables.  The price of that flexibility is an increase in
run time of approximately 4--9\% over the pure-Fortran version.
The difference in performance between the
\vars{'full'} version of \mods{qtcm} and the pure-Fortran version of
QTCM1 is between negligible and 3\% longer.

To make a timing for the pure-Fortran model, go to
\fn{test/benchmarks/timing/work} in \fn{/buildpath} and run the
\fn{timing\_365.sh} script in that directory.  That script runs the
QTCM1 model using \cmd{/usr/bin/time}, which at the end of the
script will output the amount of time it took to make the model
run.  Run the timing script three times and average the values to
obtain a time comparable to the above.

To make a timing for the \mods{qtcm} model, type \cmd{python
timing\_365.py} while in the \fn{test} directory in \fn{/buildpath}.
Three run sessions will be made for \vars{compiled\_form} equal to
\vars{'full'} and \vars{'parts'}, the times are averaged, and the
value are output at the end of the script.




% ====== end file ======

	\section{Installing in Mac OS X}       \label{sec:install.macosx}
	% ==========================================================================
% Description of installing in Mac OS X
%
% By Johnny Lin
% ==========================================================================


% ------ BODY -----
%
%------------------------------------------------------------------------
\subsection{Introduction}

This section describes issues and a summary of the installation steps
I followed to install \mods{qtcm} on a Mac running OS X.
It is a specific realization of the general installation
instructions found in Sections~\ref{sec:install.sum}--\ref{sec:test.qtcm}.
I first worked through these installation steps during June--July 2007,
with updates during July 2008.
The best way to go through this section is to go through
the summary of the installation steps in 
Section~\ref{sec:osx.install.summary},
and looking back to other sections as needed.




%------------------------------------------------------------------------
\subsection{Platform and Unix Dependencies}

This work was done on a MacBook 1.83 GHz Intel Core Duo running Mac OS X
10.4.11.  My machine has 1 GB RAM and 64 GB of disk in its main partition.

I recommend you turn-off your antivirus software before you
do the installs.  
Problems have been
\latexhtml{reported by Fink users\footnote%
		{http://finkproject.org/faq/usage-fink.php?phpLang=en\#kernel-panics}}%
	{\htmladdnormallink{reported by Fink users}%
		{http://finkproject.org/faq/usage-fink.php?phpLang=en#kernel-panics}}
using the Fink package manager with antivirus software enabled.

There are a variety of dependencies that are required to get your Mac
up-and-running as a scientific computing platform.  The most basic is
installing Apple's 
\htmladdnormallinkfoot{XCode}{http://developer.apple.com/tools/xcode/}
developer tools.\footnote%
	{The package should work in Mac OS X 10.4 with XCode 2.4.1 and higher;
	I've tried it with both 2.4.1 and XCode 2.5.  Note that
	XCode 3.1 only works on Mac OS X 10.5.}
This set of tools contains compilers and libraries
needed to do anything further.  You have to be a member of Apple's
Developer Connection, but registration is free.

Besides XCode, there are a variety of Unix libraries and utilities that you
need.  I first tried installing them by myself, from scratch, into
\fn{/usr/local}, but it was hard to keep track of all the dependencies.
A few that did work, and that I installed from their disk images, are:
\htmladdnormallinkfoot{MacTeX}{http://www.tug.org/mactex/}, 
\htmladdnormallinkfoot{MAMP}{http://www.mamp.info/}, and 
\htmladdnormallinkfoot{Tcl/Tk Aqua BI (Batteries Included)}%
	{http://tcltkaqua.sourceforge.net/}.\footnote%
		{Theoretically you can use Fink to install the equivalent
		of these packages, but I like the specific collection 
		found in these packages.  For instance, Tcl/Tk Aqua BI
		runs natively on the Mac.}

For everything else, thankfully, there's the
\htmladdnormallinkfoot{Fink Project}{http://www.finkproject.org/} which
uses a package manager built upon Debian tools to install ports of
Unix programs onto a Mac.  I just 
\htmladdnormallinkfoot{downloaded}%
	{http://www.finkproject.org/download/index.php?phpLang=en}
a binary version of the Fink 0.8.1 installer for Intel Macs,
installed Fink, and used its package management tools to install
(almost) everything else I needed.\footnote%
	{The one drawback of Fink is that it sometimes
	has stability problems.  In those cases, Fink provides
	command line suggestions to fix the problems, which sometimes
	will work.  If not, sometimes
\latexhtml{%
	deleting Fink and everything it installed,\footnote%
	{http://www.finkproject.org/faq/usage-fink.php?phpLang=en\#removing}}{%
\htmladdnormallink{deleting Fink and everything it installed}
	{http://www.finkproject.org/faq/usage-fink.php?phpLang=en#removing},}
	and starting afresh, will do the trick.
	It also appeared to me that sometimes when I installed 
	multiple packages
	via one \cmd{fink install} call, the installation did not work
	as well as when I installed only one package per call.}

Although you do not need anything besides a Fortran compiler and
the netCDF libraries to run QTCM1 in its pure-Fortran form, in order to
manipulate the model and use this Python version \mods{qtcm}, you
need to have Python installed.  The default Python that comes
with the Mac is a little old, so I used Fink to also install
Python 2.5 and related packages, including
\htmladdnormallinkfoot{matplotlib}{http://matplotlib.sourceforge.net/},
\htmladdnormallinkfoot{ScientificPython}{http://dirac.cnrs-orleans.fr/plone/software/scientificpython/},
and
\htmladdnormallinkfoot{SciPy}{http://www.scipy.org}
(see Section~\ref{sec:osx.summary} for details).




%------------------------------------------------------------------------
\subsection{Fortran Compiler}

There are a variety of high-quality, commercial Fortran compilers.
Unfortunately, because I do not have a research budget, I am not able
to use those compilers.  The 
\htmladdnormallinkfoot{GNU Compiler Collection}{http://gcc.gnu.org/}
(GCC) provides a suite of open-source compilers, some of which are the
standards of their language.  Most of the GCC compilers are installed
on your Mac when you install XCode.

\htmladdnormallinkfoot{GNU Fortran}{http://gcc.gnu.org/fortran/}
(\mods{gfortran}), is the Fortran 95 compiler included with the more
recent versions of GCC.
Unfortunately, I was not able to get it to compile QTCM1.
There is a second open-source Fortran compiler,
\htmladdnormallinkfoot{G95}{http://www.g95.org/} (\mods{g95}),
which some feel is farther along in its development than \mods{gfortran}.
I was able to successfully compile QTCM1 with \mods{g95} on my Mac.
I used Fink to install G95
(see Section~\ref{sec:osx.summary} for details).




%------------------------------------------------------------------------
\subsection{NetCDF Libraries}   \label{sec:netcdf}

For some reason, the netCDF libraries and include files
installed by Fink didn't correspond to the files needed
by the calling routines in \mods{qtcm}.  To solve this, I compiled
my own set of 
\htmladdnormallinkfoot{netCDF 3.6.2 libraries}%
	{http://www.unidata.ucar.edu/software/netcdf/}
using the tarball 
\latexhtml{downloaded from UCAR\footnote{http://www.unidata.ucar.edu/downloads/netcdf/netcdf-3\_6\_2/}}%
        {\htmladdnormallink{downloaded from UCAR}{http://www.unidata.ucar.edu/downloads/netcdf/netcdf-3_6_2/}}.

Once I uncompressed and untarred the package, and went into 
the top-level directory of the package, I built the package by typing
the following at the Unix prompt:

\begin{codeblock}
\codeblockfont{%
./configure --prefix=/Users/jlin/extra/netcdf \\
make check \\
make install}
\end{codeblock}

This installed the netCDF binaries, libraries, and include files into
sub-directories \fn{bin}, \fn{lib}, and \fn{include} in 
the directory specified by \vars{--prefix}.
If you want to install the netCDF libraries in the default
(usually \fn{/usr/local}), just leave out the \vars{--prefix}
option.

Note:  When you build netCDF, make sure the build directory
is not in the directory tree of \vars{--prefix}
(or the default directory \fn{/usr/local}).




%------------------------------------------------------------------------
\subsection{Makefile Configuration}  \label{sec:osx.makefile}

	\subsubsection{NetCDF}

In the \fn{src} directory in the \mods{qtcm} distribution, there is a
sub-directory \fn{Makefiles} that contains the makefiles for a
variety of platforms.  Edit the file \fn{makefile.osx\_g95}
so that the lines specifying the environment variables for the
netCDF libraries and include files:

\begin{codeblock}
\codeblockfont{%
NCINC=-I/Users/jlin/extra/netcdf/include \\
NCLIB=-L/Users/jlin/extra/netcdf/lib -lnetcdf}
\end{codeblock}

are changed to the path where your \emph{manually compiled} 
netCDF libraries and include files are.

Copy \fn{makefile.osx\_g95} from the \fn{Makefiles} sub-directory
in \fn{src} into \fn{src}.  
In other words, from the \mods{qtcm} distribution directory
(i.e., \fn{/buildpath}), at the Unix prompt execute:

\begin{codeblock}
\codeblockfont{%
cp src/Makefiles/makefile.osx\_g95 src/makefile}
\end{codeblock}


	\subsubsection{Linking Order}

Compilers in the GNU Compiler Collection (GCC) search libraries
and object files in the order they are listed in the command-line, 
\latexhtml{from left-to-right\footnote%
        {http://gcc.gnu.org/onlinedocs/gcc-4.1.2/gcc/Link-Options.html\#index-l-670}}%
        {\htmladdnormallink{from left-to-right}{http://gcc.gnu.org/onlinedocs/gcc-4.1.2/gcc/Link-Options.html#index-l-670}}.
Thus, if routines in \fn{b.o} call routines in \fn{a.o}, 
you must list the files in the order \fn{a.o b.o}.

For some reason, that isn't the case for \mods{g95}.  Thus, you will
find \mods{g95} makefile rules structured like the following
(below is part of the rule to create an executable (\fn{qtcm}) for
benchmark runs):

% --- Two versions of this rule, one for display in PDF and the other
%     for display in HTML:
%
\begin{latexonly}
\begin{codeblock}
\codeblockfont{%
qtcm: main.o \\
\hspace*{8ex}\$(FC)~-O~\$(NCINC)~-o~\$@ main.o~\$(QTCMLIB)~\$(NCLIB)}
\end{codeblock}
\end{latexonly}

\begin{htmlonly}
\begin{rawhtml}
<p><code><font color="blue">qtcm: main.o<br>
&nbsp;&nbsp;&nbsp;&nbsp;&nbsp;&nbsp;&nbsp;$(FC) -O $(NCINC) -o 
$@ main.o $(QTCMLIB) $(NCLIB)</font></code></p>
\end{rawhtml}
\end{htmlonly}

even though \fn{main.o} depends on the QTCM library 
(specified in macro setting \vars{\$(QTCMLIB)}), which in turn
depends on the netCDF library (specified in macro setting \vars{\$(NCLIB)}).


%------------------------------------------------------------------------
\subsection{Summary of Steps}   \label{sec:osx.install.summary}

The following summarizes all the steps I took to install
\mods{qtcm} in Mac OS X:

\begin{enumerate}
\item Install
	\htmladdnormallinkfoot{XCode 2.5}%
		{http://developer.apple.com/tools/xcode/}.

\item Install 
	\htmladdnormallinkfoot{MacTeX}{http://www.tug.org/mactex/}, 
	\htmladdnormallinkfoot{MAMP}{http://www.mamp.info/}, and 
	\htmladdnormallinkfoot{TCL/Tk Aqua BI (Batteries Included)}%
		{http://tcltkaqua.sourceforge.net/}.

\item Install
	\htmladdnormallinkfoot{Fink 0.8.1}%
		{http://www.finkproject.org/download/index.php?phpLang=en}.
	Make sure you
	\htmladdnormallink{set up your environment}%
		{http://www.finkproject.org/doc/users-guide/install.php\#setup}
	to enable you to use the packages you install with Fink
	(e.g. \vars{PATH} settings, etc.).
	Most of the time, that just means adding the line
	\cmd{source /sw/bin/init.csh} to your \fn{.cshrc} file (or the
	equivalent in your \fn{.bashrc}).

	Note that for many of the packages needed to run \mods{qtcm},
	you need to 
	\htmladdnormallink{configure Fink to download packages 
		from the unstable trees}%
	{http://www.finkproject.org/faq/usage-fink.php?phpLang=en\#unstable}.
	To do that, add \vars{unstable/main} and \vars{unstable/crypto}
	to the \vars{Trees:} line in \fn{/sw/etc/fink.conf}, and run:

	\begin{codeblock}
	\codeblockfont{fink selfupdate} \\
	\codeblockfont{fink index} \\
	\codeblockfont{fink scanpackages} \\
	\codeblockfont{fink update-all}
	\end{codeblock}

	When \cmd{selfupdate} runs, choose \cmd{rsync} for the
	self update method.  If you do not, Fink will not look in the
	unstable trees for packages.

\item Use Fink to install the \mods{g95} Fortran compiler.
	From a Unix prompt, type:

	\begin{codeblock}
	\codeblockfont{fink -$\,\!$-use-binary-dist install g95}
	\end{codeblock}

\item Use Fink to install Python 
	and the NumPy package (which \mods{f2py} is a part of).
	From a Unix prompt, type:

	\begin{codeblock}
	\codeblockfont{%
	fink -$\,\!$-use-binary-dist install python25 \\
	fink -$\,\!$-use-binary-dist install scipy-core-py25}
	\end{codeblock}

	(Numpy used to be called SciPy Core.)  If you want to
	install Python 2.4 instead, just change the ``25'' and ``py25'' above
	(and in later occurrences) to ``24'' and ``py24'', respectively.
	Note that Fink does not have a version of epydoc for Python 2.4,
	so if you wish to create documentation using epydoc, you will
	need to install Python 2.5.

\item Install teTeX and \LaTeX{2HTML} using Fink.
	From a Unix prompt, type:

	\begin{codeblock}
	\codeblockfont{fink -$\,\!$-use-binary-dist install tetex} \\
	\codeblockfont{fink -$\,\!$-use-binary-dist install latex2html}
	\end{codeblock}

	When prompted, choose ghostscript and ghostscript-fonts to
	satistfy the dependency (which should be the default options).
	I tried choosing system-ghostscript8, but Fink looks for
	ghostscript 8.51 and didn't recognize ghostscript 8.57 that
	was already installed in \fn{/usr/local} (via my MacTeX
	install).  \LaTeX{2HTML} has a package required by the
	\mods{qtcm} manual \LaTeX\ file.

\item Install additional programming and
	scientific packages and libraries using Fink.
	From a Unix prompt, type:

	\begin{codeblock}
	\codeblockfont{%
	fink -$\,\!$-use-binary-dist install scientificpython-py25 \\
	fink -$\,\!$-use-binary-dist install matplotlib-py25 \\
	fink -$\,\!$-use-binary-dist install matplotlib-basemap-py25 \\
	fink -$\,\!$-use-binary-dist install matplotlib-basemap-data-py25 \\
	fink -$\,\!$-use-binary-dist install xaw3d \\
	fink -$\,\!$-use-binary-dist install fftw fftw3 \\
	fink -$\,\!$-use-binary-dist install epydoc-py25 \\
	fink -$\,\!$-use-binary-dist install graphviz \\
	fink -$\,\!$-use-binary-dist install scipy-py25}
	\end{codeblock}

\item Manually install netCDF 3.6.2
	(see Section \ref{sec:netcdf}).

\item From this point on, you can follow the
	general instructions given in Section~\ref{sec:install.sum},
	starting with step~\ref{list:download.qtcm.sum}.
	Please do not ignore, however, Section~\ref{sec:install.macosx}'s
	Mac-specific details.

\end{enumerate}



% ===== end of file =====

	\section{Installing in Ubuntu}         \label{sec:install.ubuntu}
	% ==========================================================================
% Description of installing in Ubuntu
%
% By Johnny Lin
% ==========================================================================


% ------ BODY -----
%
%------------------------------------------------------------------------
\subsection{Introduction}

This section describes installation issues 
I followed to install \mods{qtcm} on my
Dell PowerEdge 860 running Ubuntu GNU/Linux 8.04.1 LTS (Hardy).
The machine has 2.66 GHz Quad Core Intel Xeon processors (64 bit),
4 GB RAM, and 677 GB of disk in its main partition.
This section is a specific realization of the general installation
instructions found in Sections~\ref{sec:install.sum}--\ref{sec:test.qtcm}.
I worked through these installation steps during July 2008.
The best way to go through this section is to go through
the summary of the installation steps in 
Section~\ref{sec:ubuntu.install.summary},
and looking back to other sections as needed.



%------------------------------------------------------------------------
\subsection{Fortran Compiler}     \label{sec:ubuntu.fort.install}

The easiest Fortran compiler to install in Ubuntu 8.04.1 is
\htmladdnormallinkfoot{GNU Fortran}{http://gcc.gnu.org/fortran/}
(\mods{gfortran}), the Fortran 95 compiler included with the more
recent versions of the GNU Compiler Collection (GCC); you can
use any package manager (e.g., \mods{apt-get}, \mods{aptitude})
to install it.
Unfortunately, I was not able to get it to compile QTCM1.
I was, however, able to successfully compile QTCM1 using
the second open-source Fortran compiler,
\htmladdnormallinkfoot{G95}{http://www.g95.org/} (\mods{g95}),
which some feel is farther along in its development than \mods{gfortran}.
G95, however, is not supported as an Ubuntu package, and so I had
to manually install it.

I downloaded the binary version of G95 v0.91 
(the Linux x86\_64/EMT64 with 32 bit default integers) 
using the following
\cmd{curl} command:\footnote%
	{I use \mods{curl} because I usually access my
	Ubuntu server via a terminal session.}

\begin{codeblock}
\codeblockfont{%
\small
curl -o g95.tgz http://ftp.g95.org/v0.91/g95-x86\_64-32-linux.tgz}
\end{codeblock}

which saves the \fn{.tgz} file as the local file \fn{g95.tgz}.
After that, I followed the G95 project's standard
\latexhtml{installation instructions\footnote%
	{http://g95.sourceforge.net/docs.html\#starting}}%
	{\htmladdnormallink{installation instructions}%
		{http://g95.sourceforge.net/docs.html#starting}}
to finish the install.\footnote%
	{The G95 installation instructions say you can put
	\fn{g95-install} anywhere, and make a link to the
	executable \mods{g95} in
	\fn{$\sim$/bin}.  I put \fn{g95-install} in
	\fn{/usr/local}, and while in \fn{/usr/local/bin}, 
	I put a link to the G95 executable using the command:
	\begin{codeblock}
	\codeblockfont{%
	sudo ln -s ../g95-install\_64/bin/x86\_64-suse-linux-gnu-g95 g95.}
	\end{codeblock}}
The regular Linux x86 version of G95
(in \fn{g95-x86-linux.tgz} from the G95 website) did not work on my
machine.




%------------------------------------------------------------------------
\subsection{NetCDF Libraries}   \label{sec:ubuntu.netcdf}

%Here things were very confusing for my machine, as I needed to
%install two versions of the
%\htmladdnormallinkfoot{netCDF}%
%	{http://www.unidata.ucar.edu/software/netcdf/}
%libraries and include files, one 
%for a successful compilation of the extension modules
%(as described in Section~\ref{sec:create.so}),
%and the other 
%for a successful run of the pure-Fortran QTCM1 model
%(used to create the testing benchmarks, as described in
%Section~\ref{sec:test.qtcm}).
%
%The first set of netCDF files (for the extension modules) are
%installed from Ubuntu's package management system.
%These are automatically installed when the \mods{python-netcdf}
%package is installed via an Ubuntu package manager
%(see Section~\ref{sec:ubuntu.install.summary}).
%The include files for this netCDF installation are 
%located in \fn{/usr/include}, and the libraries for this
%netCDF installation are location in \fn{/usr/lib}.

For some reason, the netCDF libraries and include files
installed from the Ubuntu packages do not
correspond to the files needed
by the calling routines in \mods{qtcm}.  To solve this, I compiled
my own set of
\htmladdnormallinkfoot{netCDF 3.6.2 libraries}%
        {http://www.unidata.ucar.edu/software/netcdf/}
using the tarball
\latexhtml{downloaded from UCAR\footnote{http://www.unidata.ucar.edu/downloads/netcdf/netcdf-3\_6\_2/}}%
        {\htmladdnormallink{downloaded from UCAR}{http://www.unidata.ucar.edu/downloads/netcdf/netcdf-3_6_2/}}.

Once I uncompressed and untarred the package, and went into
the top-level directory of the package, I built the package by typing
the following at the Unix prompt:

\begin{codeblock}
\codeblockfont{%
export FC=g95 \\
export FFLAGS="-O -fPIC" \\
export FFLAGS="-fPIC" \\
export F90FLAGS="-fPIC" \\
export CFLAGS="-fPIC" \\
export CXXFLAGS="-fPIC" \\
./configure \\
make check \\
sudo make install}
\end{codeblock}

(The \cmd{export} commands set environment variables for the
Fortran compiler and Fortran and other compiler flags.  The
\vars{-fPIC} flag enables the compilers to create
position independent code, needed for shared libraries in
Ubuntu on a 64 bit Intel processor.)

The above installs the netCDF binaries, libraries, and include files into
sub-directories \fn{bin}, \fn{lib}, and \fn{include} in 
\fn{/usr/local}, the default.
The include files for this netCDF installation are thus
located in \fn{/usr/local/include}, and the libraries for this
netCDF installation are location in \fn{/usr/local/lib}.
(If you want to specify a different installation
location, use the \vars{--prefix} option in \cmd{configure}.)
While you don't have to have root privileges during the configuration
and check steps, you do during the installation step if you're installing
into \fn{/usr/local} (thus the \cmd{sudo} in the last step).\footnote%
	{Note that when you build netCDF, make sure the build directory
	is not in the directory tree of \vars{--prefix}
	or the default directory \fn{/usr/local}.}

%Because there are two different netCDF installations used in the
%\mods{qtcm} package, the makefiles for creating the benchmarks
%and extensions files will have different \vars{NCLIB} and \vars{NCINC}
%environment variables (see Section~\ref{sec:ubuntu.makefile}).




%------------------------------------------------------------------------
\subsection{Makefile Configuration}  \label{sec:ubuntu.makefile}

	\subsubsection{NetCDF}

In the \fn{src} directory in the \mods{qtcm} distribution, there is a
sub-directory \fn{Makefiles} that contains the makefiles for a
variety of platforms.  Edit the file \fn{makefile.ubuntu\_64\_g95}
so that the lines specifying the environment variables for the
netCDF libraries and include files:

\begin{codeblock}
\codeblockfont{%
NCINC=-I/usr/local/include \\
NCLIB=-L/usr/local/lib -lnetcdf}
\end{codeblock}

are changed to the path where your manually compiled
netCDF libraries and include files are.

Copy \fn{makefile.ubuntu\_64\_g95} from the \fn{Makefiles} sub-directory
in \fn{src} into \fn{src}.  
In other words, from the \mods{qtcm} distribution directory
(i.e., \fn{/buildpath}), at the Unix prompt execute:

\begin{codeblock}
\codeblockfont{%
cp src/Makefiles/makefile.ubuntu\_64\_g95 src/makefile}
\end{codeblock}


	\subsubsection{Linking Order}

Compilers in the GNU Compiler Collection (GCC) search libraries
and object files in the order they are listed in the command-line,
\latexhtml{from left-to-right\footnote%
	{http://gcc.gnu.org/onlinedocs/gcc-4.1.2/gcc/Link-Options.html\#index-l-670}}%
	{\htmladdnormallink{from left-to-right}{http://gcc.gnu.org/onlinedocs/gcc-4.1.2/gcc/Link-Options.html#index-l-670}}.
Thus, if routines in \fn{b.o} call routines in \fn{a.o}, 
you must list the files in the order \fn{a.o b.o}.

For some reason, that isn't the case for \mods{g95}.  Thus, you will
find \mods{g95} makefile rules structured like the following
(below is part of the rule to create an executable (\fn{qtcm}) for
benchmark runs):

% --- Two versions of this rule, one for display in PDF and the other
%     for display in HTML:
%
\begin{latexonly}
\begin{codeblock}
\codeblockfont{%
qtcm: main.o \\
\hspace*{8ex}\$(FC)~-O~\$(NCINC)~-o~\$@ main.o~\$(QTCMLIB)~\$(NCLIB)}
\end{codeblock}
\end{latexonly}

\begin{htmlonly}
\begin{rawhtml}
<p><code><font color="blue">qtcm: main.o<br>
&nbsp;&nbsp;&nbsp;&nbsp;&nbsp;&nbsp;&nbsp;$(FC) -O $(NCINC) -o 
$@ main.o $(QTCMLIB) $(NCLIB)</font></code></p>
\end{rawhtml}
\end{htmlonly}

even though \fn{main.o} depends on the QTCM library 
(specified in macro setting \vars{QTCMLIB}), which in turn
depends on the netCDF library (specified in macro setting \vars{NCLIB}).


	\subsubsection{Shared Object PIC}   \label{sec:sopic}

In order to compile the model in Ubuntu on a 64 bit Intel processor,
the model and the netCDF library it is linked to needs to be
compiled to be 
\latexhtml{position independent code (PIC).\footnote%
		{http://www.gentoo.org/proj/en/base/amd64/howtos/index.xml?part=1\&chap=3}}%
	{\htmladdnormallink{position independent code (PIC)}%
		{http://www.gentoo.org/proj/en/base/amd64/howtos/index.xml?part=1&chap=3}.}
This is accomplished with the 
\htmladdnormallinkfoot{\cmd{-fPIC} flag}%
	{http://www.fortran-2000.com/ArnaudRecipes/sharedlib.html}.

In the \mods{qtcm} makefiles, the \cmd{-fPIC} flag is introduced in the
macro \vars{FFLAGSM}, for instance:
\begin{codeblock}
\codeblockfont{%
FFLAGSM = -O -fPIC}
\end{codeblock}
For makefiles used in creating extension modules, \cmd{-fPIC} must
be passed into the \mods{f2py} call.  To do so, put the flags:
\begin{codeblock}
\codeblockfont{%
--f90flags="-fPIC" --f77flags="-fPIC"}
\end{codeblock}
after the \vars{--fcompiler} flag in the \mods{f2py} calling line.

The \cmd{-fPIC} flag must also be used when compiling the netCDF
libraries, as described in Section~\ref{sec:ubuntu.netcdf}.
Failure to create PIC libraries in 64 bit Ubuntu can result in errors 
like the following when creating the \mods{qtcm} extension modules:
\begin{codeblock}
\codeblockfont{%
ld: /usr/local/lib/libnetcdf.a(fort-attio.o): relocation R\_X86\_64\_32 against `a local symbol' can not be used when making a shared object; recompile with -fPIC /usr/local/lib/libnetcdf.a: could not read symbols: Bad value}
\end{codeblock}




%------------------------------------------------------------------------
\subsection{Summary of Steps}      \label{sec:ubuntu.install.summary}

The following summarizes all the steps I took to install
\mods{qtcm} in
Ubuntu 8.04.1 LTS (Hardy) running on a
Quad Core Intel Xeon (64 bit) machine.
Note that while I use the \mods{aptitude} package manager, you are
free to use any manager of your choice (e.g., \mods{apt-get},
\mods{synaptic}, etc.):

\begin{enumerate}
\item Install the G95 Fortran compiler from the binary distribution.
	See Section~\ref{sec:ubuntu.fort.install} for details.

\item Use an Ubuntu package manager
	to install the following packages, by typing:
	\begin{codeblock}
	\codeblockfont{%
sudo aptitude update \\
sudo aptitude install curl \\
sudo aptitude install python-epydoc \\
sudo aptitude install python-matplotlib \\
sudo aptitude install python-netcdf \\
sudo aptitude install python-scientific \\
sudo aptitude install python-scipy \\
sudo aptitude install texlive}
	\end{codeblock}

	Installing \mods{python-scipy} will also install NumPy and
	\mods{f2py}, so you don't have to install the
	\mods{python-numpy} package separately.

	Early-on as I debugged my \mods{qtcm} install on Ubuntu,
	I encountered errors that I thought came from an 
	\htmladdnormallinkfoot{old version of NumPy}%
		{http://cens.ioc.ee/pipermail/f2py-users/2008-June/001617.html},
	and thus I replaced Ubuntu's packaged NumPy with NumPy 1.1.0
	built 
	\latexhtml{directly from source.\footnote%
			{http://sourceforge.net/project/showfiles.php?group\_id=1369\&package\_id=175103}}%
		{\htmladdnormallink{directly from source}{http://sourceforge.net/project/showfiles.php?group_id=1369&package_id=175103}.}
	(Note, you shouldn't install your new NumPy in the default
	location, which may cause problems later-on with Ubuntu's
	package manager.)
	Later on, I concluded the errors I had encountered were not
	because of the NumPy version, but by then I didn't want to
	try to reinstall NumPy again.
	So strictly speaking, the version of Numpy I used is not
	the one bundled with \mods{python-scipy}, but that shouldn't
	be a problem.

\item Manually install netCDF 3.6.2 from source
	(see Section \ref{sec:ubuntu.netcdf}).

\item Manually install the \mods{basemap} package of
	\mods{matplotlib}.  
	The source for the \mods{basemap} toolkit is
	available 
	\latexhtml{from Sourceforge\footnote%
			{http://sourceforge.net/project/showfiles.php?group\_id=80706}}%
		{\htmladdnormallink{from Sourceforge}%
			{http://sourceforge.net/project/showfiles.php?group_id=80706}}
	I obtained version 0.9.9.1 using the
	following \cmd{curl} command:
	\begin{codeblock}
	\codeblockfont{%
\scriptsize
curl -o basemap.tar.gz $\backslash$ \\
http://voxel.dl.sourceforge.net/sourceforge/matplotlib/basemap-0.9.9.1.tar.gz}
	\end{codeblock}

	The \fn{README} file in the \fn{basemap-0.9.9.1} directory has
	detailed installation instructions.  Note that you have to
	install the GEOS library first (\fn{README} has detailed
	directions on how to do that too).  To be on the safe-side,
	I would set the \vars{FC} environment variable to the G95
	compiler
	(e.g., with \cmd{export FC=g95} in Bash).

\item From this point on, you can follow the
	general instructions given in Section~\ref{sec:install.sum},
	starting with step~\ref{list:download.qtcm.sum}.
	Please do not ignore, however, Section~\ref{sec:install.ubuntu}'s
	Ubuntu-specific details.

\end{enumerate}



% ===== end of file =====


\chapter{Getting Started With \mods{qtcm}}  \label{ch:getting.started}
% ==========================================================================
% Getting Started With qtcm
%
% By Johnny Lin
% ==========================================================================


% ------ BODY -----
%
%---------------------------------------------------------------------
\section{Your First Model Run}

Figure~\ref{fig:my.first.run} shows an example of a script to make
a 30 day seasonal, aquaplanet model run, with run name ``test'',
starting from November 1, Year 1.


%--- Two versions, one for PDF, one for HTML:
\begin{latexonly}
\begin{figure}[htp]
\begin{center}
\begin{codeblock}
\codeblockfont{%
from qtcm import Qtcm \\
inputs = \{\} \\
inputs['runname'] = 'test' \\
inputs['landon'] = 0 \\
inputs['year0'] = 1 \\
inputs['month0'] = 11 \\
inputs['day0'] = 1 \\
inputs['lastday'] = 30 \\
inputs['mrestart'] = 0 \\
inputs['compiled\_form'] = 'parts' \\
model = Qtcm(**inputs) \\
model.run\_session()}
\end{codeblock}
\end{center}
\caption{An example of a simple \mods{qtcm} run.}
\label{fig:my.first.run}
\end{figure}
\end{latexonly}

\begin{htmlonly}
\label{fig:my.first.run}
\begin{center}
\htmlfigcaption{%
	\codeblockfont{%
from qtcm import Qtcm \\
inputs = \{\} \\
inputs['runname'] = 'test' \\
inputs['landon'] = 0 \\
inputs['year0'] = 1 \\
inputs['month0'] = 11 \\
inputs['day0'] = 1 \\
inputs['lastday'] = 30 \\
inputs['mrestart'] = 0 \\
inputs['compiled\_form'] = 'parts' \\
model = Qtcm(**inputs) \\
model.run\_session()}
	}

\htmlfigcaption{Figure~\ref{fig:my.first.run}:
	An example of a simple \mods{qtcm} run.}
\end{center}
\end{htmlonly}



The class describing the QTCM1 model is \class{Qtcm}.  An instance
of \class{Qtcm}, in this example \vars{model}, is created the same
way you create an instance of any class.  When instantiating an
instance of \class{Qtcm}, keyword parameters can be used to override
any default settings.  In the example above, the dictionary
\vars{inputs} specifying all keyword parameters is passed in on the
instantiation of \vars{model}.

The keyword parameter settings in
Figure~\ref{fig:my.first.run} have the following meanings:
\begin{itemize}
\item \vars{runname}:  This string (``test'') is used in the
	output filename.  QTCM1 writes mean and instantaneous
	output files to the directory given in \vars{model.outdir.value},
	with filenames 
	\fn{qm\_}\dumarg{runname}\fn{.nc} for mean output and
	\fn{qi\_}\dumarg{runname}\fn{.nc} for instantaneous output.

\item \vars{landon}: When set to ``0'', the land is turned off and
	the run is an aquaplanet run.  When set to ``1'', the land
	model is turned on.

\item \vars{year0}:  The year the run starts on.

\item \vars{month0}:  The month the run starts on (11 = November).

\item \vars{day0}: The day of the month the run starts on.

\item \vars{lastday}:  The model runs from day 1 to \vars{lastday}.

\item \vars{mrestart}:  When set to ``0'', the run starts from
	default initial conditions
	(see Section~\ref{sec:initial.variables} for a table of
	those values).
	When set to ``1'', the run starts from a restart file.

\item \vars{compiled\_form}:  This keyword sets what form the
	compiled QTCM1 model has, and its value is saved to
	the instance's \vars{compiled\_form} attribute.
	It is a string and can be set either to
	``parts'' or ``full''.  Most of the time, you will want
	to set it to \vars{'parts'}.
	This keyword is the only one
	that must be specified on instantiation; the model instance
	will at least instantiate
	using only the default settings for all the other keyword
	parameters (given in Appendix~\ref{app:defaults.values}).
	See Section~\ref{sec:compiledform} for details about
	what the \vars{compiled\_form} attribute controls.
\end{itemize}

By default, the \vars{SSTmode} attribute, which controls whether the
model will use climatological sea-surface temperatures (SST) 
or real SSTs, is set to the \vars{value} ``seasonal'', thus giving a
run with seasonal forcing at the lower-boundary over the ocean.

This example assumes that the boundary condition files, sea surface
temperature files, and the model output directories are as specified
in submodule \mods{defaults}.  Those values are described in
Section~\ref{sec:defaults.scalar}.




%---------------------------------------------------------------------
\section{Managing Directories}

Most of the time, your boundary condition files and output files
will not be in the locations specified in
Section~\ref{sec:defaults.scalar}, or in the directory your
Python script resides.  The easiest way to tell your \class{Qtcm} 
instance where your input/output files are is to pass them in
as keyword parameters on instantiation.


%--- Two versions, one for PDF, one for HTML:
\begin{latexonly}
\begin{figure}[htp]
\begin{codeblock}
\codeblockfont{%
\small
from qtcm import Qtcm \\
rundirname = 'test' \\
dirbasepath = os.path.join(os.getcwd(), rundirname) \\
inputs = \{\} \\
inputs['bnddir'] = os.path.join( os.getcwd(), 'bnddir', \\
\hspace*{40ex}'r64x42' ) \\
inputs['SSTdir'] = os.path.join( os.getcwd(), 'bnddir', \\
\hspace*{40ex}'r64x42', 'SST\_Reynolds' ) \\
inputs['outdir'] = dirbasepath \\
inputs['runname'] = rundirname \\
inputs['landon'] = 0 \\
inputs['year0'] = 1 \\
inputs['month0'] = 11 \\
inputs['day0'] = 1 \\
inputs['lastday'] = 30 \\
inputs['mrestart'] = 0 \\
inputs['compiled\_form'] = 'parts' \\
model = Qtcm(**inputs) \\
model.run\_session()}
\end{codeblock}

\caption{An example \mods{qtcm} run showing detailed description of
	input and output directories.}
\label{fig:manage.dir.example}
\end{figure}
\end{latexonly}

\begin{htmlonly}
\label{fig:manage.dir.example}
\begin{center}
\htmlfigcaption{%
	\codeblockfont{%
from qtcm import Qtcm \\
rundirname = 'test' \\
dirbasepath = os.path.join(os.getcwd(), rundirname) \\
inputs = \{\} \\
inputs['bnddir'] = os.path.join( os.getcwd(), 'bnddir', \\
\hspace*{40ex}'r64x42' ) \\
inputs['SSTdir'] = os.path.join( os.getcwd(), 'bnddir', \\
\hspace*{40ex}'r64x42', 'SST\_Reynolds' ) \\
inputs['outdir'] = dirbasepath \\
inputs['runname'] = rundirname \\
inputs['landon'] = 0 \\
inputs['year0'] = 1 \\
inputs['month0'] = 11 \\
inputs['day0'] = 1 \\
inputs['lastday'] = 30 \\
inputs['mrestart'] = 0 \\
inputs['compiled\_form'] = 'parts' \\
model = Qtcm(**inputs) \\
model.run\_session()}
	}

\htmlfigcaption{Figure~\ref{fig:manage.dir.example}:
	An example \mods{qtcm} run showing detailed description of
        input and output directories.}
\end{center}
\end{htmlonly}


Figure~\ref{fig:manage.dir.example} shows an example run where those
directories are explicitly specified; in all other aspects, the run
is identical to the one in Figure~\ref{fig:my.first.run}.
In Figure~\ref{fig:manage.dir.example}, output from the model is
directed to the directory described by string variable
\vars{dirbasepath}.  \vars{dirbasepath} is created by joining the
current working directory with the run name given in string variable
\vars{rundirname}.\footnote%
	{The Python \mods{os} module enables platform-independent
	handling of files and directories.  The \mods{os.path.join}
	function resolves paths without the programmer needing to know
	all the possible directory separation characters; the function
	chooses the correct separation character at runtime.  The
	\mods{os.getcwd} function returns the current working directory.}
Setting keyword parameter \vars{outdir} to \vars{dirbasepath} sends
output to \vars{dirbasepath}.  
Keywords \vars{bnddir} and \vars{SSTdir} specify the directories
where non-SST and SST boundary condition files, respectively, are
found.

Interestingly, the default version of QTCM1 does \emph{not} send
all output from the model to \vars{outdir}.  The restart file
\fn{qtcm\_}\dumarg{yyyymmdd}\fn{.restart} (where \dumarg{yyyymmdd}
is the year, month, and day of the model date when the restart
file was written) is written into the current working directory,
not the output directory.  Thus, if you do multiple runs, you'll
have to manually deal with the restart files that will proliferate.

Neither the QTCM1 model nor the \class{Qtcm} object
create the directories specified in \mods{bnddir}, \mods{SSTdir},
and \mods{outdir}.  Failure to do so will create an error.  I use
Python's file management tools to make sure the output directory
is created, and any old output files are deleted.  Here's an example
that does that, using the \vars{dirbasepath} and \vars{rundirname}
variables from Figure~\ref{fig:manage.dir.example}:

\begin{codeblock}
\codeblockfont{%
\small
if not os.path.exists(dirbasepath):  os.makedirs(dirbasepath) \\
qi\_file = os.path.join( dirbasepath, 'qi\_'+rundirname+'.nc' ) \\
qm\_file = os.path.join( dirbasepath, 'qm\_'+rundirname+'.nc' ) \\
if os.path.exists(qi\_file):   os.remove(qi\_file) \\
if os.path.exists(qm\_file):   os.remove(qm\_file)}
\end{codeblock}




%---------------------------------------------------------------------
\section{Model Field Variables}   \label{sec:field.variables.intro}

The term ``field'' variable refers to QTCM1 model variables that 
are accessible at both the compiled Fortran QTCM1 model-level as
well as the Python \class{Qtcm} instance-level.
Field variables are all instances of the \class{Field} class,
and are stored as attributes of the \class{Qtcm} instance.\footnote%
	{Note non-field variables can also be instances of \class{Field},
	and that \class{Qtcm} instances have other attributes that are
	not equal to \class{Field} instances.}

\class{Field} class instances have the following attributes:
\begin{itemize}
\item \vars{id}:  A string naming the field (e.g., ``Qc'', ``mrestart'').
	This string should contain no whitespace.
\item \vars{value}:  The value of the field.  Can be of any type, though
	typically is either a string or numeric scalar or a numeric array.
\item \vars{units}:  A string giving the units of the field.
\item \vars{long\_name}:  A string giving a description of the field.
\end{itemize}

\class{Field} instances also have methods to return the rank 
and typecode of \vars{value}.

Remember, if you want to access the value of a \class{Field} object,
make sure you access that object's \vars{value} attribute.  
Thus, for example,
to assign a variable \vars{foo} to the
\vars{lastday} value for a given
\class{Qtcm} instance \vars{model}, type the following:
\begin{codeblock}
\codeblockfont{%
foo = model.lastday.value}
\end{codeblock}

For scalars, this assignment sets \vars{foo} by value (i.e., a copy
of the value of attribute \vars{model.lastday} is set to \vars{foo}).
In general, however, Python assigns variables by reference.  Use
the \mods{copy} module if you truly want a copy of a field variable's
value (such as an array), rather than an alias.  For more details
about field variables, see Section~\ref{sec:field.variables}.




%---------------------------------------------------------------------
\section{Run Sessions}

	\subsection{What is a Run Session?}

A run session is a unit of simulation where the model is run from
day 1 of simulation to the day specified by the \vars{lastday}
attribute of a \class{Qtcm} instance.  A run session is a
``complete'' model run, at the beginning of which all compiled QTCM1
model variables are set to the values given at the Python-level,
and at the end of which restart files are written, the values
at the Python-level are overwritten by the values in the Fortran
model, and a Python-accessible snapshot is taken of the 
model variables that were written to the restart file.


	\subsection{Changing Variables}

Between run sessions, changing any field variable is as easy
as a Python assignment.  For instance, to change the atmosphere
mixed layer depth to 100~m, just type:
\begin{codeblock}
\codeblockfont{%
model.ziml.value = 100.0}
\end{codeblock}

When changing arrays, be careful to try to match the shape of the 
array.\footnote%
	{At the very least, match the rank of the array, which is required
	for the routines in \mods{setbypy} to properly choose which
	Fortran subroutine to use in reading the Python value.
	I haven't tested if only the rank is needed, however,
	for the passing to work, for a continuation run (my hunch is
	it won't).}
You can use the NumPy \mods{shape} function on a NumPy array to
check its shape.


	\subsection{Continuing a Model Run}  \label{sec:continuation.intro}

Figure~\ref{fig:continuation.example} shows an example of two run
sessions, where the second run session is a continuation of the
first.


%--- Two versions, one for PDF, one for HTML:
\begin{latexonly}
\begin{figure}[htp]
\begin{codeblock}
\codeblockfont{%
\small
inputs['year0'] = 1 \\
inputs['month0'] = 11 \\
inputs['day0'] = 1 \\
inputs['lastday'] = 10 \\
inputs['mrestart'] = 0 \\
inputs['compiled\_form'] = 'parts' \\ \\
model = Qtcm(**inputs) \\
model.run\_session() \\
model.u1.value = model.u1.value * 2.0 \\
model.init\_with\_instance\_state = True \\
model.run\_session(cont=30)}
\end{codeblock}

\caption{An example of two \mods{qtcm} run sessions where the second
	run session is a continuation of the first.  Assume 
	\vars{inputs} is a dictionary, and that earlier in the
	script the run name and
	all input and output directory names were added
	to the dictionary.}
\label{fig:continuation.example}
\end{figure}
\end{latexonly}

\begin{htmlonly}
\label{fig:continuation.example}
\begin{center}
\htmlfigcaption{%
	\codeblockfont{%
inputs['year0'] = 1 \\
inputs['month0'] = 11 \\
inputs['day0'] = 1 \\
inputs['lastday'] = 10 \\
inputs['mrestart'] = 0 \\
inputs['compiled\_form'] = 'parts' \\ \\
model = Qtcm(**inputs) \\
model.run\_session() \\
model.u1.value = model.u1.value * 2.0 \\
model.init\_with\_instance\_state = True \\
model.run\_session(cont=30)}
	}

\htmlfigcaption{Figure~\ref{fig:continuation.example}:
	An example of two \mods{qtcm} run sessions where the second
	run session is a continuation of the first.  Assume 
	\vars{inputs} is a dictionary, and that earlier in the
	script the run name and
	all input and output directory names were added
	to the dictionary.}
\end{center}
\end{htmlonly}


The first run session runs from day 1 to day 10.  The second
run session runs the model for another 30 days.  
Setting the \vars{init\_with\_instance\_state} of
\vars{model} to \vars{True} tells the model to use the
the values of the instance attributes 
(for prognostic variables, right-hand sides, and start date) 
are currently stored \vars{model}
as the initial values for the run\_session.\footnote%
	{Unless overridden, by default, 
	\vars{init\_with\_instance\_state} is set
	to True on \class{Qtcm} instance instantiation.}
The \vars{cont}
keyword in the second \mods{run\_session} call specifies a
continuation run, and the value gives the number of additional
days to run the model.

The set of runs described above would produce the exact same
results as if you had gone into the Fortran model after 10 days,
doubled the first baroclinic mode zonal velocity, and continued
the run for another 30 days.  With the Python example above, however,
you didn't need to know you were going to do that ahead of starting
the model run (which is what a compiled model requires you to do).
Section~\ref{sec:contination.run.sessions} describes continuation
runs in detail.


	\subsection{Passing Restart Snapshots Between Run Sessions}
					\label{sec:snapshot.intro}

The pure-Fortran QTCM1 uses a restart file to enable continuation
runs.  A \class{Qtcm} instance can also make use of that option,
through setting the \vars{mrestart} attribute value
(see Section~\ref{sec:contination.run.sessions} and
Neelin et al.\ \cite{Neelin/etal:2002} for details).  
It's easier, however, instead of using a restart file, to pass 
along a ``snapshot'' dictionary.

The \class{Qtcm} instance method \mods{make\_snapshot} copies the
variables that would be written out to a restart file into a
dictionary that is saves as the instance attribute \vars{snapshot}.
This snapshot can be saved separately, for later recall.  Note that
snapshots are automatically made at the end of a run session.

The following example shows a model \mods{run\_session} call,
following which the snapshot is saved to the variable
\vars{snapshot}:\footnote%
	{Remember Python assignment defaults to assignment by
	reference, so in this example the variable \vars{mysnapshot}
	is a pointer to the \vars{model.snapshot} attribute.
	(However, note that \vars{model.snapshot} itself is not a
	reference, but a distinct copy of those variables; to do
	otherwise would result in a non-static snapshot.)
	If the \vars{model.snapshot} attribute is dereferenced,
	then \vars{mysnapshot} will become the sole pointer to the
	dictionary.}

\begin{codeblock}
\codeblockfont{%
model.run\_session() \\
mysnapshot = model.snapshot}
\end{codeblock}

After taking the snapshot, you might continue the run a while, and
then decide to return to the snapshot you saved.  To do so, use
the \mods{sync\_set\_py\_values\_to\_snapshot}
method to reset the model instance values to
\vars{mysnapshot} before your next run session:
\begin{codeblock}
\codeblockfont{%
model.sync\_set\_py\_values\_to\_snapshot(snapshot=mysnapshot) \\
model.init\_with\_instance\_state = True \\
model.run\_session()}
\end{codeblock}

See Section~\ref{sec:snapshots} for details regarding the use of
snapshots, as well as for a list of what variables are saved in
a snapshot.




%---------------------------------------------------------------------
\section{Creating Multiple Models}

	\subsection{Model Instances}

Creating a new QTCM1 model is as simple as creating another
\class{Qtcm} instance.
For instance, to instantiate two QTCM1
models, \vars{model1} and \vars{model2}, type the following:

\begin{codeblock}
\codeblockfont{%
from qtcm import Qtcm \\
model1 = Qtcm(compiled\_form='parts') \\
model2 = Qtcm(compiled\_form='parts')}
\end{codeblock}

\vars{model1} and \vars{model2} do \emph{not} share any variables
in common, including the extension modules holding the Fortran
code.  In creating the instances, a copy of the extension modules
are saved in temporary directories.


	\subsection{Passing Snapshots To Other Models}

The snapshots described in Section~\ref{sec:snapshot.intro}
can also be passed around to other model instances,
enabling you to easily branch a model run:

\begin{codeblock}
\codeblockfont{%
model.run\_session() \\
mysnapshot = model.snapshot \\
model1.sync\_set\_py\_values\_to\_snapshot(snapshot=mysnapshot) \\
model2.sync\_set\_py\_values\_to\_snapshot(snapshot=mysnapshot) \\
model1.run\_session() \\
model2.run\_session()}
\end{codeblock}

The state of \vars{model} after its run session is used to start
\vars{model1} and \vars{model2}.  This is an easy way to save time
in spinning-up multiple models.




%---------------------------------------------------------------------
\section{Run Lists}		\label{sec:runlist.intro}

This feature of \class{Qtcm} objects is what really gives 
\class{Qtcm} model instances their flexibility.
A run list is a list of strings and dictionaries that specify
what routines to run in order to execute a particular part of
the model.  Each element of the run list specifies the method
or subroutine to execute, and the order of the elements specifies
their execution order.

For instance, the standard run list for initializing the the
atmospheric portion of the model is named ``qtcminit'', and
equals the following list:

\begin{latexonly}
\begin{codeblock}
\codeblockfont{%
\parbox{46ex}{\input{qtcminit_runlist}}}
\end{codeblock}
\end{latexonly}

\begin{htmlonly}
\begin{quotation}
\input{qtcminit_runlist}
\end{quotation}
\end{htmlonly}

This list is stored as an entry in the \vars{runlists} dictionary
(with key \vars{'qtcminit'}).
\vars{runlists} is an attribute of a \class{Qtcm} instance.
Table~\ref{tab:stnd.runlists} lists all standard run lists.

When the run list element in the list is a string, the string gives the
name of the routine to execute.  The routine has no parameter
list.  The routine can be a
compiled QTCM1 model subroutine for which an interface has been
written (e.g., \mods{\_\_qtcm.wrapcall.wparinit}), 
a method of the of the Python model instance 
(e.g., \mods{varinit}), or another run list
(e.g., \vars{atm\_physics1}).

When the run list element is a 1-element dictionary, the key of
the dictionary element is the name of the routine, and the value
of the dictionary element is a list specifying input parameters
to be passed to the routine on call.  Thus, the element:
\begin{codeblock}
\codeblockfont{%
{\{'\_\_qtcm.wrapcall.wtimemanager': [1]\}}}
\end{codeblock}
calls the \mods{\_\_qtcm.wrapcall.wtimemanager} routine, passing in
one input parameter, which in this case is the value 1.

If you want to change the order of the run list, just change the
order of the list.  To add or remove routines to be executed, just
add and remove their names from the run list.
Python provides a number of methods to manipulate
lists (e.g., \mods{append}).  Since lists are dynamic data types
in Python, you do not have to do any recompiling to implement
the change.

The \vars{compiled\_form} attribute must be set to \vars{'parts'}
in the \class{Qtcm} instance in order to take advantage of the run
lists feature of the class.  Run lists are not available for
\vars{compiled\_form\thinspace=\thinspace'full'}, because subroutine
calls are hardwired in the compiled QTCM1 model Fortran code in
that case.




%---------------------------------------------------------------------
\section{Model Output}			\label{sec:output.intro}

	\subsection{NetCDF Output}

Model output is written to netCDF files in the directory
specified by the \class{Qtcm} instance attribute \vars{outdir}.
Mean values are written to an output file beginning with
\fn{qm\_}, and instantaneous values are written to an output
file beginning with \fn{qi\_}.

The frequency of mean output is controlled by \vars{ntout}, and the
frequency of instantaneous output is controlled by \vars{ntouti}.
\vars{ntout.value} gives the number of days over which to average
(and if equals \vars{-30}, monthly means are calculated).
\vars{ntouti.value} gives the frequency in days that instantaneous
values are output (monthly if it equals \vars{-30}).  (See
Section~\ref{sec:initial.variables} for a description of other
output-control variables, and see the QTCM1 manual \cite{Neelin/etal:2002}
for a detailed description of how these variables control output.)

Figure~\ref{fig:netcdf.read} gives an example of a block of code
to read netCDF output, where \vars{datafn} is the netCDF filename, and
\vars{id} is the string name of the field variable (e.g.,
\vars{'u1'}, \vars{'T1'}, etc.).
(Note that the netCDF identifier for field variables is the same as
the name in \class{Qtcm}, except for the variables given in
Table~\ref{tab:qtcm.netcdf.ids}.)

In the code in Figure~\ref{fig:netcdf.read},
the array value is read into \vars{data}, and the longitude values, 
latitude values, and time values are read into variables
\vars{lon}, \vars{lat}, and \vars{time}, respectively.
As netCDF files also hold metadata, a description and the units
of the variable given by \vars{id}, and each dimension, are read
into variables ending in \vars{\_name} and \vars{\_units},
respectively.


%--- Two versions, one for PDF, one for HTML:
\begin{latexonly}
\begin{figure}[htp]
\begin{codeblock}
\codeblockfont{%
import numpy as N \\
import Scientific as S \\ \\
fileobj = S.NetCDFFile(datafn, mode='r') \\ \\
data = N.array(fileobj.variables[id].getValue()) \\
data\_name = fileobj.variables[id].long\_name \\
data\_units = fileobj.variables[id].units \\ \\
lat = N.array(fileobj.variables['lat'].getValue()) \\
lat\_name = fileobj.variables['lat'].long\_name \\
lat\_units = fileobj.variables['lat'].units \\ \\
lon = N.array(fileobj.variables['lon'].getValue()) \\
lon\_name = fileobj.variables['lon'].long\_name \\
lon\_units = fileobj.variables['lon'].units \\ \\
time = N.array(fileobj.variables['time'].getValue()) \\
time\_name = fileobj.variables['time'].long\_name \\
time\_units = fileobj.variables['time'].units \\ \\
fileobj.close()}
\end{codeblock}

\caption{Example of Python code to read netCDF output.
	See text for description.}
\label{fig:netcdf.read}
\end{figure}
\end{latexonly}

\begin{htmlonly}
\label{fig:netcdf.read}
\begin{center}
\htmlfigcaption{%
	\codeblockfont{%
import numpy as N \\
import Scientific as S \\ \\
fileobj = S.NetCDFFile(datafn, mode='r') \\ \\
data = N.array(fileobj.variables[id].getValue()) \\
data\_name = fileobj.variables[id].long\_name \\
data\_units = fileobj.variables[id].units \\ \\
lat = N.array(fileobj.variables['lat'].getValue()) \\
lat\_name = fileobj.variables['lat'].long\_name \\
lat\_units = fileobj.variables['lat'].units \\ \\
lon = N.array(fileobj.variables['lon'].getValue()) \\
lon\_name = fileobj.variables['lon'].long\_name \\
lon\_units = fileobj.variables['lon'].units \\ \\
time = N.array(fileobj.variables['time'].getValue()) \\
time\_name = fileobj.variables['time'].long\_name \\
time\_units = fileobj.variables['time'].units \\ \\
fileobj.close()}
	}

\htmlfigcaption{Figure~\ref{fig:netcdf.read}:
	Example of Python code to read netCDF output.
	See text for description.}
\end{center}
\end{htmlonly}





\begin{table}[tp]
\begin{center}
\begin{tabular}{l|l}
\textbf{\class{Qtcm} Attribute Name} & \textbf{NetCDF Output Name} \\
\hline
\vars{'Qc'}                & \vars{'Prec'} \\
\vars{'FLWut'}             & \vars{'OLR'} \\
\vars{'STYPE'}             & \vars{'stype'}
\end{tabular}
\end{center}
\caption{NetCDF output names for \class{Qtcm} field variables that
	are different from the \class{Qtcm} and compiled QTCM1 model
	variable names.  The netCDF names are case-sensitive.}
\label{tab:qtcm.netcdf.ids}
\end{table}


\emphpara{NB:}  All netCDF array output is dimensioned (time, latitude,
longitude) when read into Python using the \mods{Scientific} package.
This differs from the way \class{Qtcm} saves field variables, which
follows Fortran convention (longitude, latitude).  Please be careful
when relating the two types of arrays.
Section~\ref{sec:field.var.shape} for a discussion of why there is
this discrepancy.


	\subsection{Visualization}	\label{sec:viz.intro}

The \mods{plotm} method of \class{Qtcm} instances creates line
plots or contour plots, as appropriate, of model output of
average fields of run session(s) associated with the instance.
Some examples, assuming \vars{model} is an instance of \class{Qtcm}
and has already executed a run session:
\begin{itemize}
\item \cmd{model.plotm('Qc', lat=1.875)}:
	A time vs.\ longitude contour
          plot is made for the full range of time and longitude,
          at the latitude 1.875 deg N, for mean precipitation.

\item \cmd{model.plotm('Qc', time=10)}:
	A latitude vs.\ longitude contour plot of precipitation
	is made for the full spatial domain at day 10 of the model run.

\item \cmd{model.plotm('Evap', lat=1.875, lon=[100,200])}:  A contour
	plot of time vs.\ longitude of evaporation is made for the
          longitude points between 100 and 200 degrees E, at the
          latitude 1.875 deg N.  

\item \cmd{model.plotm('cl1', lat=1.875, lon=[100,200], time=20)}:
          A deep cloud amount vs.\ longitude line plot is made for
          the longitude points between 100 and 200 degrees east,
          at the latitude 1.875 deg N, at day 20 of the model run.
\end{itemize}

In these examples, the number of days over which the mean is taken
equals \vars{model.ntout.value}.
Also, the \mods{plotm} method automatically takes into account the
\class{Qtcm}/netCDF variable differences described in
Table~\ref{tab:qtcm.netcdf.ids}.



%---------------------------------------------------------------------
\section{Documentation}

Section~\ref{sec:ver} gives the online locations of the
transparent copies of this manual.  
Model formulation is fully described in
Neelin \& Zeng \cite{Neelin/Zeng:2000} and model
results are described in Zeng et~al.\ \cite{Zeng/etal:2000}
(\cite{Neelin/Zeng:2000} is based upon v2.0 of QTCM1
and \cite{Zeng/etal:2000} is based on QTCM1 v2.1).
Additional documentation you'll find useful include:

\begin{itemize}
\item \latexhtml{%
\htmladdnormallinkfoot{The \mods{qtcm} Package API Documentation}%
        {http://www.johnny-lin.com/py\_pkgs/qtcm/doc/html-api/}}%
{\htmladdnormallink{The \mods{qtcm} Package API Documentation}%
        {http://www.johnny-lin.com/py_pkgs/qtcm/doc/html-api/}}

\item \latexhtml{%
\htmladdnormallinkfoot{The Pure-Fortran QTCM1 Manual}%
        {http://www.atmos.ucla.edu/$\sim$csi/qtcm\_man/v2.3/qtcm\_manv2.3.pdf}}%
{\htmladdnormallink{The Pure-Fortran QTCM1 Manual}%
        {http://www.atmos.ucla.edu/~csi/qtcm_man/v2.3/qtcm_manv2.3.pdf}}
\cite{Neelin/etal:2002}

\end{itemize}



% ===== end of file =====


\chapter{Using \mods{qtcm}}                 \label{ch:using}
% ==========================================================================
% Using QTCM
%
% By Johnny Lin
% ==========================================================================


% ------ BODY -----
%
%---------------------------------------------------------------------
\section{Introduction}

Now that you've successfully run your first model instances, in
this chapter I provide detailed explanations regarding the features
of \mods{qtcm}.  I present these explanations in a documentary
rather than didactic fashion; my goal is to document how the features
work.  More details are given in the code docstrings.  At the end
of the chapter, in Section~\ref{sec:cookbook}, I provide a few
cookbook ideas/examples of ways to use the model.




%---------------------------------------------------------------------
\section{Model Instances}  \label{sec:model.instances}

An instance of a \class{Qtcm} model is created in \mods{qtcm} the same way
you create an instance of any class.
For instance, to instantiate two \class{Qtcm}
models, \vars{model1} and \vars{model2}, I type the following:

\begin{codeblock}
\codeblockfont{%
from qtcm import Qtcm \\
model1 = Qtcm(compiled\_form\thinspace=\thinspace'full') \\
model2 = Qtcm(compiled\_form\thinspace=\thinspace'parts')}
\end{codeblock}

In the above example, \vars{model1} uses the compiled QTCM1 model
that runs the model (essentially) using the Fortran driver,
while \vars{model2} uses the compiled QTCM1 model where execution
order and content all the way down to the atmospheric timestep level
is controlled by Python run lists.  (Section~\ref{sec:compiledform}
has more details about the difference between compiled forms.)

For each instance of \class{Qtcm}, copies of all needed extension
modules (e.g., \fn{.so} files) are copied to a temporary directory
that is automatically created by Python.  The full path name of
that directory is saved in the instance attribute \vars{sodir}.
These extension modules are then associated with the specific instance 
through private instance attributes,
and thus every instance of \class{Qtcm} has its own separate variable
and name space on both the Fortran and Python sides.\footnote%
	{The private instance attribute is \vars{\_\_qtcm}.
	See Section~\ref{sec:Qtcm.private.attrib} for details about 
	private \class{Qtcm} instance attributes.}
The temporary directory and all of its contents are deleted when the 
model instance is deleted.

On instantiation, \class{Qtcm} instances set all scalar field
variables to their default values as given in the submodule
\mods{defaults} (and listed in Section~\ref{sec:defaults.scalar}),
and assign the fields as instance attributes.  The instance attribute
\vars{init\_with\_instance\_state} is set to True by default, unless
overridden on instantiation.




%---------------------------------------------------------------------
\section{Initializing a Model Run}

In the pure-Fortran QTCM1, there are three broad
classes of initialized variables:
\begin{enumerate}
\item Those that are read-in using a namelist, 
\item Those that the are read-in from a restart file, and
\item Those that are set by assignment in the Fortran code.  
\end{enumerate}
These variables are a combination of scalars and arrays.

For \mods{qtcm}, interfaces were built so that all classes of
initialized variables that could be user-controlled are accessible
and changeable at the Python-level.  For \mods{qtcm},
the set of variables that could be changed is also expanded, to
include not just the first and second classes of pure-Fortran
QTCM1 initialized variables.  This was done to make \mods{qtcm}
more flexible.  All variables that can be passed between the
compiled QTCM1 model and Python model levels are called
field variables, and are described in detail in
Section~\ref{sec:field.variables}.

As it happens, all the namelist-set variables are scalars.  In the
pure-Fortran QTCM1, those variables are given default values prior
to reading in of the namelist.  To duplicate this functionality,
on model instantiation, all scalar fields are set to their default
values as given in the submodule \mods{defaults} and listed in
Section~\ref{sec:defaults.scalar}.  Most of the default values in
\mods{defaults} are the same as in the pure-Fortran QTCM1, but
there are a few differences.\footnote%
	{One difference being \vars{mrestart}, which in \vars{qtcm} 
	will have the value of 0 in contrast to the pure-Fortran 
	QTCM1 where the default is the 1.}
This setting of scalar defaults is the same for both
\vars{compiled\_form\thinspace=\thinspace'full'} and
\vars{compiled\_form\thinspace=\thinspace'parts'} instances.
Of course, all
\mods{qtcm} fields are user-controllable, both via keyword input
parameters at model instantiation as well as through direct
manipulation of the instance attribute that stores the field variable.

The pure-Fortran QTCM1 initialized prognostic variables and
right-hand sides are set in the Fortran subroutine \mods{varinit}.
Their they are read-in from a restart file or, as default,
set by assignment.
In \mods{qtcm}, the same variables are initialized by a \class{Qtcm}
instance method of the same name, \mods{varinit}, for the case when
\vars{compiled\_form\thinspace=\thinspace'parts'}.  For the case
of \vars{compiled\_form\thinspace=\thinspace'full'}, the compiled
QTCM1 subroutine that is the same as in the pure-Fortran case is
used, and that routine is inaccessible at the Python level.
See Section~\ref{sec:snapshots}'s listing of snapshot variables,
which also includes the prognostic variables and right-hand sides that
are set in \mods{varinit} (both Fortran and Python).




%---------------------------------------------------------------------
\section{The \vars{compiled\_form} Keyword}  \label{sec:compiledform}

The \mods{qtcm} package is a Python wrap of the Fortran routines
that make up QTCM1.  The wrapping layer adds flexibility and
functionality, but at the cost of speed.  Thus, I created two
types of extension modules from the Fortran QTCM1 code, one
which permits very little control over the compiled Fortran
\emph{routines} at the Python level, and one that allows the Python-level
to control model execution in the compiled QTCM1 model
all the way down to the atmospheric timestep level.\footnote%
	{That control is via run lists, which are described in
	Section~\ref{sec:runlists}.}
The former extension module corresponds to 
\vars{compiled\_form\thinspace=\thinspace'full'} and
the latter extension module to
\vars{compiled\_form\thinspace=\thinspace'parts'}.

For \vars{compiled\_form\thinspace=\thinspace'full'},
the compiled portion of the model encompasses (nearly) the
entire QTCM1 model as a whole.  Thus, the only compiled QTCM1 model
modules or subroutines that Python should interact with is
the \mods{driver} routine (which executes the entire model) and
the \mods{setbypy} module (which enables communication between the
compiled model and the Python-level of model fields.\footnote%
	{The \mods{setbypy} Python module is the wrap of the
	Fortran QTCM1 \mods{SetByPy} module.}

For \vars{compiled\_form\thinspace=\thinspace'parts'}, the compiled
portion of the model does not encompasses the model as a whole, but
rather is broken up into separate units (as appropriate) all the
way down to an atmosphere timestep.  Thus, compiled QTCM1 model
modules/subroutines that are accessible at the Python-level include
those that are executed within an atmosphere timestep on up.

Because the difference in compiled forms fundamentally affects how
the \class{Qtcm} instance facilitates Python-Fortran communication,
this attribute must be set on instantiation via a keyword input
parameter.

In the rest of this section, to avoid being verbose, when I
write \vars{'full'}, I mean the situation where
\vars{compiled\_form\thinspace=\thinspace'full'}.
Likewise, when I
write \vars{'parts'}, I mean the situation where
\vars{compiled\_form\thinspace=\thinspace'parts'}.


	\subsection{Initialization for 
			\vars{compiled\_form\thinspace=\thinspace'full'}}
				\label{sec:init.compiledform.full}

For a model run of this case, the \class{Qtcm} instance will
initialize the model using the Fortran \mods{varinit} subroutine
in the compiled QTCM1 model.  This subroutine does the following:

\begin{itemize}
\item If \vars{mrestart\thinspace=\thinspace1}, 
	the restart file is used to initialize all prognostic
	variables.  In terms of start date, the following rules are
	used:
	\begin{enumerate}
	\item Variable \vars{dateofmodel} is read from the restart file.
	\item If \vars{day0}, \vars{month0}, and \vars{year0}
		are negative, or otherwise
		invalid (e.g., \vars{month0} greater than 12), the invalid
		value is replaced with the
		day, month, and/or year of the day \emph{after} 
		that given by \vars{dateofmodel}.
		If the value of \vars{day0}, \vars{month0}, or \vars{year0}
		is not invalid in this sense, it is not replaced.
	\end{enumerate}
	Thus, if the restart file gives 
	\vars{dateofmodel} equal to 101102
	(year 10, month 11, day 2), and 
	\vars{day0\thinspace=\thinspace-1}, 
	\vars{month0\thinspace=\thinspace-1}, 
	\vars{year0\thinspace=\thinspace-1},
	and 
	\vars{mrestart\thinspace=\thinspace1}, 
	the model will start running from year 10, month 11, day 3.
	If \vars{dateofmodel} equals to 101102, and 
	\vars{day0\thinspace=\thinspace-1}, 
	\vars{month0\thinspace=\thinspace3}, 
	\vars{year0\thinspace=\thinspace-1},
	the model will start running from year 10, month 3, day 3.

\item If \vars{mrestart\thinspace=\thinspace0}, 
	all prognostic variables and right-hand sides are set to an
	initial value (which for most of those variables is zero).
	In terms of start date, \vars{day0} is set to 1 (and thus 
	the value of \vars{day0} previously input is ignored), and
	both \vars{month0} and \vars{year0}
	are set to 1 
	if their previously input values are invalid (where
	invalid means less than
	1, or, for \vars{month0}, greater than 12).
	Otherwise, \vars{month0} and \vars{year0} are left unchanged.
	Variable \vars{dateofmodel} has the value it had when the variable
	was declared (which is determined by the compiler and usually
	is zero; \vars{dateofmodel} will not be properly set until
	subroutine \mods{TimeManager} is called.

	Thus, if 
	\vars{day0\thinspace=\thinspace-1},
	\vars{month0\thinspace=\thinspace-1}, 
	\vars{year0\thinspace=\thinspace-1} is input into the model
	(say from a namelist) and 
	\vars{mrestart\thinspace=\thinspace0},
	the model will start running from year 1, month 1, day 1,
	and \vars{dateofmodel} at the exit of subroutine 
	\mods{varinit} will equal its compiler-set default.
	If 
	\vars{day0\thinspace=\thinspace14}, 
	\vars{month0\thinspace=\thinspace3}, 
	\vars{year0\thinspace=\thinspace11}, and 
	\vars{mrestart\thinspace=\thinspace0} on input into the
	model,
	the model will start running from year 11, month 3, day 1,
	and \vars{dateofmodel} at the exit of subroutine 
	\mods{varinit} will equal its compiler-set default.

	Note that \vars{dateofmodel}
	can thus be inconsistent with 
	\vars{month0} and \vars{year0} at the
	exit of subroutine \mods{varinit}.
\end{itemize}

This behavior with respect to initializing
the start date is different than in QTCM1 versions 1.0 and 2.1.
Please see the source code from those earlier QTCM1 versions for
details.




	\subsection{Initialization for 
			\vars{compiled\_form\thinspace=\thinspace'parts'}}
				\label{sec:init.compiledform.parts}

For \vars{'parts'} model, the methodology of how initialized
prognostic variables, right-hand sides, and start date related
variables are set is controlled by the \class{Qtcm} instance
attribute/flag \vars{init\_with\_instance\_state}.  The initialization
is (mostly) executed in the Python \vars{varinit} method in the
following way:

\begin{itemize}
\item If \vars{init\_with\_instance\_state} is False:
The method as described for
initialization for the 
\vars{'full'} case is generally
followed, with the exception that dateofmodel is set
to match \vars{day0}, \vars{month0}, \vars{year0}, prior to exit of 
\mods{varinit}.

\item If \vars{init\_with\_instance\_state} is True:
the model object will initialize the model based on the current
state of the model instance.  This enables you to set a model run
session's initial conditions based upon the state of the prognostic
variables and parameters stored at the Python level, which is
accessible at runtime.
\end{itemize}


Since the \vars{init\_with\_instance\_state\thinspace=\thinspace{False}}
case is mainly described by the initialization method for the
\vars{'full'} case, I refer the
reader to Section~\ref{sec:init.compiledform.full}.
For the case of \vars{init\_with\_instance\_state} is True, however,
the task is more complicated.  Specifically, for that case,
initialization includes the following:

\begin{enumerate}
\item If not currently defined,
	variable \vars{dateofmodel} is set to a default value of 0,
	which is specified in the module defaults.

\item The \vars{mrestart} flag is ignored for variable initialization.

\item All prognostic variables and right-hand sides
        are set to an
        initial value (which for most of those variables is zero),
	unless the variable is defined at the Python level, in which
	case the inital value is set to the Python level defined value.

\item If \vars{dateofmodel} is greater than 0, 
	\vars{day0}, \vars{month0}, and \vars{year0} are overwritten
        with values derived from \vars{dateofmodel} 
	in order to set the run to start
	the day \emph{after} \vars{dateofmodel}.

\item If \vars{dateofmodel} is less than or equal to 0, \vars{day0},
	\vars{month0}, and \vars{year0} are set to their respective
	instance attribute values, if valid.  For invalid instance
	attribute values, the invalid \vars{day0}, \vars{month0},
	and/or \vars{year0} is set to 1.

\item Variable \vars{dateofmodel} is recalculated
	and overwritten to match 
	\vars{day0}, \vars{month0}, \vars{year0}, prior to exit of 
	\mods{varinit}.
\end{enumerate}

As a result, for \vars{init\_with\_instance\_state} is True, the
way you indicate to the model that a run session is a brand-new run
is by setting, before the \mods{run\_session} method call,
\vars{dateofmodel} to a value less than or equal to 0, and \vars{day0},
\vars{model0}, and \vars{year0} to the day you want the model to
begin the run session.  To indicate to the model you wish to continue
a run, set \vars{dateofmodel} to the day \emph{before} you want the
model to start running from.

Examples:
\begin{itemize}
\item If \vars{day0\thinspace=\thinspace-1}, 
	\vars{month0\thinspace=\thinspace-1}, 
	\vars{year0\thinspace=\thinspace-1}, and
	\vars{dateofmodel\thinspace=\thinspace0} is input into 
	the model the model will start running from year 1, month 1, day 1,
	and 
	variable \vars{dateofmodel} at the exit of 
	subroutine \mods{varinit}
	will equal 10101.

\item If \vars{day0\thinspace=\thinspace14},
	\vars{month0\thinspace=\thinspace3}, 
	\vars{year0\thinspace=\thinspace11},
	and \vars{dateofmodel\thinspace=\thinspace0} is input into the
	model, the model will start running from year 11, month 3, day 14,
	and 
	variable \vars{dateofmodel} at the exit of 
	subroutine \mods{varinit} will equal
	110314.

\item If \vars{day0\thinspace=\thinspace14},
	\vars{month0\thinspace=\thinspace3}, 
	\vars{year0\thinspace=\thinspace11},
	and \vars{dateofmodel\thinspace=\thinspace341023} is input into the
	model, the model will start running from year 34, month 10, day 24,
	and at the exit of subroutine 
	\mods{varinit}, \vars{dateofmodel} will equal
	341024, with \vars{day0\thinspace=\thinspace24},
	\vars{month0\thinspace=\thinspace10}, and
	\vars{year0\thinspace=\thinspace34}.
\end{itemize}


	\subsection{Communication Between Python and Fortran-Levels}
				\label{sec:comm.py.fort.compiledform}

After initialization, the second major difference between a
\vars{'full'} and \vars{'parts'} model is how and when communication
between the Python and Fortran levels can occur.  For the \vars{'full'}
case, except for the passing in and out of variables before and after
a run session, all variable passing and subroutine calling happens in
the compiled QTCM1 model, with no control at the Python level.
For the \vars{'parts'} case, variables can be passed between the
Python and Fortran-levels at all levels down to the atmospheric
timestep, and many Fortran QTCM1 subroutines can be called from the
Python-level.  


		\subsubsection{Passing Variables}

For all \vars{compiled\_form} cases, variables are passed back and
forth between the Python \class{Qtcm} instance level and the
compiled QTCM1 model Fortran-level using the \class{Qtcm}
instance methods \mods{get\_qtcm1\_item} and \mods{set\_qtcm1\_item}:\footnote%
	{All Fortran routines used to pass variables back and forth are
	defined in the \mods{setbypy} module of the \fn{.so} extension
	module stored in the \class{Qtcm} instance variable \vars{\_\_qtcm}.
	All Fortran wrappers that enable Python to call compiled QTCM1 model
	subroutines are defined in the \mods{wrapcall} module stored in
	the \class{Qtcm} instance variable \vars{\_\_qtcm}.
	These modules are described in detail in 
	Sections~\ref{sec:setbypy} and~\ref{sec:wrapcall}, respectively.}

\begin{itemize}
\item \mods{get\_qtcm1\_item}(\dumarg{key}):
	Returns the value of the field variable given by the string
	\dumarg{key}.  If the compiled QTCM1 model variable given by
	\dumarg{key} is unreadable, the
        custom exception 
	\vars{FieldNotReadableFromCompiledModel} is thrown.
	The value returned is a copy of the value on the Fortran
	side, not a reference to the variable in memory.

\item \mods{set\_qtcm1\_item}:
	Sets the value of a field variable
	in the compiled QTCM1 model \emph{and at the Python-level,}
	automatically overriding any previous value at both levels.
	Thus, calling this method will change/create the \class{Qtcm}
	instance attribute corresponding to the field variable.
        When the compiled QTCM1 model variable is set, a copy of the
        Python value is passed to the Fortran model; the
	variable is \emph{not passed by reference.}
	This value comes from the \mods{set\_qtcm1\_item} calling
	parameter list, \emph{not} from the \class{Qtcm}
        instance attribute corresponding to the field variable.
\end{itemize}

The \mods{set\_qtcm1\_item} method has two calling forms, one with
one argument and the other with two arguments:
\begin{itemize}
\item One argument:  The method is called
	as \mods{set\_qtcm1\_item}(\dumarg{arg}), where \dumarg{arg} 
	is either a string giving the name of the field variable or 
	a \class{Field} instance.

\item Two arguments:  The method is called as
	\mods{set\_qtcm1\_item}(\dumarg{key}, \dumarg{value}), where
	\dumarg{key} is the string giving the name of the field variable
	and \dumarg{value} is the value to set the model field variable to
	(note \dumarg{value} can be a \class{Field} instance).
\end{itemize}
In either calling form, if no value given, the default value as defined
in module \mods{defaults} is used.

Some compiled QTCM1 model variables are not in a state where they
can be set.  An example is a compiled QTCM1 model pointer variable,
prior to the pointer being associated with a target (an attempt
to set would yield a bus error).  In such cases, the
\mods{set\_qtcm1\_item} method will throw a
\vars{FieldNotReadableFromCompiledModel} exception, nothing will
be set in the compiled QTCM1 model, and the Python counterpart
field variable (if it previously existed) would be left unchanged.\footnote%
	{We handle this situation in this way to enable the
	\class{Qtcm} instance to store variables
	even if the compiled model is not yet ready to accept them.}

Examples, typed in at a Python prompt, and
assuming that \vars{model} is a \class{Qtcm} instance:
\begin{itemize}
\item \cmd{dtvalue\thinspace=\thinspace{model.get\_qtcm1\_item('dt')}}:
	Retrieves the value of field variable \vars{dt} (timestep)
	from the compiled QTCM1 Fortran model and sets it to the
	Python variable \vars{dtvalue}.

\item \cmd{model.set\_qtcm1\_item('dt')}:
	Sets the value of field variable \vars{dt}
	in the compiled QTCM1 Fortran model to the default
	value (as given in \mods{defaults}),
	and sets the value of Python attribute \vars{model.dt} also to 
	that default value.  
	Remember that \vars{model.dt} is a \class{Field}
	instance.

\item \cmd{model.set\_qtcm1\_item('dt', 2000.)}:
	Sets the value of field variable \vars{dt}
	in the compiled QTCM1 Fortran model to 2000 (as a real),
	and sets the value of Python attribute \vars{model.dt} also to 2000.
\end{itemize}


		\subsubsection{Calling Compiled QTCM1 Model Subroutines}

All compiled QTCM1 model subroutines that can be called
(except \mods{driver} and \mods{varptrinit}) are in the
\mods{setbypy} or \mods{wrapcall} modules
of the \class{Qtcm} instance private attribute \vars{\_\_qtcm}.
(On \class{Qtcm} instance instantiation, \vars{\_\_qtcm} is set
to the \fn{.so} extension module that is the compiled QTCM1 Fortran model.)
Thus, to call \mods{wmconvct} in \mods{wrapcall} at the Python-level,
just type \cmd{model.\_\_qtcm.wrapcall.wmconvct()} (where \vars{model}
is a \class{Qtcm} instance).
For \mods{driver} and \mods{varptrinit}, these subroutines are not
contained in a \vars{\_\_qtcm} module, and thus can be called
directly (e.g., just type \cmd{model.\_\_qtcm.driver()}).
See Sections~\ref{sec:setbypy} and~\ref{sec:wrapcall} for more information
on the \mods{setbypy} and \mods{wrapcall} modules.

For the \vars{'full'} case, the only compiled QTCM1 model
subroutine you can usefully call during a run session is \mods{driver}.
For the \vars{'parts'} case, while you can essentially call any subroutine
given in a run list, you usually will not directly call a compiled QTCM1
model subroutine but will instead call it through including it in a
run list.  For example, if you have the following run list in a
\vars{'parts'} model:
\begin{codeblock}
\codeblockfont{%
[ 'qtcminit', '\_\_qtcm.wrapcall.woutpinit' ]}
\end{codeblock}
Running this list using the \class{Qtcm} instance method
\mods{run\_list} will result in \class{Qtcm} instance method
\mods{qtcminit} first being run, 
then the compiled QTCM1 Fortran model subroutine
\mods{woutpinit} in Fortran module \mods{wrapcall} being run.
See Section~\ref{sec:runlists} and
Table~\ref{tab:stnd.runlists} for a discussion and list of the
standard run lists that control routine execution content and order
in the \vars{'parts'} case.




%---------------------------------------------------------------------
\section{Restart and Continuation Run Sessions}
				\label{sec:contination.run.sessions}


	\subsection{Restart Runs In the Pure-Fortran QTCM1}
					\label{sec:puref90.restart}

To enable restart of a model run, the pure-Fortran QTCM1 model
writes out a restart file with the state of the prognostic variables
and select right-hand sides at that point in the run (for a list
of the variables, see Section~\ref{sec:snapshots}).  This binary
file can then be read in by later model runs.  The Fortran
\vars{mrestart} flag is passed in via a namelist; if \vars{mrestart}
is 1, the run uses the restart file (named \fn{qtcm.restart}).

One of the problems with using the restart file to do a continuation
run is that the continuation run will not be perfect.  In other words,
a 15~day run followed by a 25~day run based on the restart file 
generated at the end of the 15~day run will \emph{not} give the
exact same output as a continuous 40~day run.


	\subsection{Overview of Restart/Continuation Options In \mods{qtcm}}
					\label{sec:restart.options.list}

For a \class{Qtcm} instance, in contrast to the pure-Fortran QTCM1,
more than one method of continuation is available.
Thus, for a continuation run, you need to tell the model
``continue from what?''
The \class{Qtcm} class provides three choices for restart/continuing
a run:
\begin{enumerate} 
\item From a restart file:  Move/rename a QTCM1 restart file
        to the current working directory to \fn{qtcm.restart}.
	\label{list:continue.from.restart}

\item From a snapshot from another run session
	(see Sections~\ref{sec:snapshot.intro} and~\ref{sec:snapshots}).
	\label{list:continue.from.snapshot}

\item From the values of the \class{Qtcm} instance you will be
	calling \mods{run\_session} from.
	\label{list:continue.from.instance}
\end{enumerate}

Restart/continuation methods~\ref{list:continue.from.restart} 
and~\ref{list:continue.from.snapshot} both suffer from the
same problem as the pure-Fortran QTCM1 restart process:
They do not produce perfect restarts
(see Section~{sec:puref90.restart} for details).
In this section, I discuss the restart/continuation options
for each \vars{compiled\_form} option.

Methods~\ref{list:continue.from.restart}
and~\ref{list:continue.from.snapshot} are best used when making a
run session from a newly instantiated \class{Qtcm} instance.
Method~\ref{list:continue.from.instance} is best used when executing
a run session using a \class{Qtcm} instance that has already gone
through at least one run session.  Regardless of which method you
use, however, please note that anytime you execute a run session
using a \class{Qtcm} instance that already has made a previous run
session, some variables \emph{cannot be updated} between run sessions.
This feature is most noticeable with the output filename, and occurs
because the name persists in the compiled QTCM model, and is stored
in the extension module (\fn{.so} files in \vars{sodir}) associated
with the instance.  If you wish to control all variables possible
from the Python level (including output filename), you need do the
run session from a new model instance.


	\subsection{Restart/Continuation for 
		\vars{compiled\_form\thinspace=\thinspace'full'} 
		Model Instances}

The only option for restart when using
\vars{compiled\_form\thinspace=\thinspace'full'} model instances
is method~\ref{list:continue.from.restart}, to use a QTCM1 restart
file.\footnote%
	{The \vars{cont} keyword parameter in \mods{run\_session}
	and the value of the \vars{init\_with\_instance\_state}
	attribute have no effect if
	\vars{compiled\_form\thinspace=\thinspace'full'}.  With
	\vars{'full'}, the call to initialize variables all happens
	at the Fortran level (via the Fortran \mods{varinit}, not
	the Python \mods{varinit}), with no reference to the Python field
	states (or even existing Fortran field states, if present).}
To use this option, the value of the \vars{mrestart} 
attribute must equal 1, the restart file must be named
\fn{qtcm.restart}, and the restart file must be in the 
current working directory.
As with the pure-Fortran QTCM1 restart process, this method
does not produce perfect restarts.



	\subsection{Restart/Continuation for 
		\vars{compiled\_form\thinspace=\thinspace'parts'} 
		Model Instances}

For the \vars{compiled\_form\thinspace=\thinspace'parts'} case,
all three restart/continuation methods
described in Section~\ref{sec:restart.options.list} are
available.


		\subsubsection{Method~\ref{list:continue.from.restart}:
			From a QTCM1 Restart File}

To use the QTCM1 restart file mechanism, not only must the
\vars{mrestart} attribute have a value to 1, but the
\vars{init\_with\_instance\_state} flag also has to be \vars{False},
otherwise the \vars{mrestart} attribute value will be ignored.  
As with the pure-Fortran QTCM1 restart process, this method does not
produce perfect restarts.


		\subsubsection{Method~\ref{list:continue.from.snapshot}:
			From a \class{Qtcm} Instance Snapshot}

You can take snapshots of the model state of a \class{Qtcm} instance
by the \mods{make\_snapshot} instance method.  This snapshot saves
a copy of all the variables saved to a QTCM1 restart file (see
Section~\ref{sec:snapshots} for the full list of fields), which
then can be passed to other \class{Qtcm} instances for use in other
run sessions.

The key difference between this method and 
method~\ref{list:continue.from.instance} (described below)
is that \mods{run\_session} calls using the snapshot are done
\emph{without} the \vars{cont} keyword input parameter
(by default, \vars{cont} is False).  If the \vars{cont} keyword
is not False, it says the run session is a continuation run
that uses the state of the compiled QTCM1 model for all variables
that are not specified at, and read-in from,
the Python level.  If the \vars{cont} keyword
is False, the run session initializes as if it were a new run.

See Section~\ref{sec:snapshot.intro} for details and
an example of using snapshots to initialize a run session.
Note that as with the pure-Fortran QTCM1 restart process, this method 
does not produce perfect restarts.


		\subsubsection{Method~\ref{list:continue.from.instance}:
			From the Calling \class{Qtcm} Instance}

This method is used when you want to make a run session that is a
``true'' continuation run, i.e., one that uses the current state
of the compiled QTCM1 model for all variables that are not read-in
from the Python level (remember that \class{Qtcm} instances hold a
subset of the variables defined at the Fortran level).  
The key reason to use this method for a continuation run session
is that the continuation is byte-for-byte the same (if no fields
are changed) as if the run just went straight on through.  Thus,
the continuation would be perfect: A 15~day run followed by a 25~day
run using the same \class{Qtcm} instance with the \vars{cont} keyword
will give the exact same output as a continuous 40~day run.  This
is not the case when making a new instance and passing a restart
file or a snapshot, because a separate extension module is used for
those new instances.

Control of this method is accomplished through the \vars{cont}
keyword input parameter to the \mods{run\_session} method and the
\vars{init\_with\_instance\_state} attribute of a
\class{Qtcm} instance:

\begin{itemize}
\item \vars{cont}: If set to False, the run session is not a
	continuation of the previous run, but a new run session.
	If set to True, the run session is a continuation of the
	previous run session.  If set to an integer greater than
	zero, the run session is a continuation just like
	\vars{cont\thinspace=\thinspace{True}}, but the value
	\vars{cont} is set to is used for \vars{lastday} and replaces
	\vars{lastday.value} in the \class{Qtcm} instance.

\item \vars{init\_with\_instance\_state}:
	If True, for a \mods{run\_session} call using the
	\vars{cont} keyword, whatever the field values are in the Python
	instance are used in the run session.
	If False, model variables are set and initialized as described in
	Section~\ref{sec:init.compiledform.parts}.  In that case,
	previous compiled QTCM1 model values will likely be overwritten.
	Thus, if you want a continuation run that uses the state of
	all field variables except for those you explicitly change at
	the Python-level, make sure \vars{init\_with\_instance\_state}
	is True.
\end{itemize}

(Note that the \vars{cont} keyword has no effect if \vars{compiled\_form}
is \vars{'full'}.  The default value of \vars{cont} in a
\mods{run\_session} call is False.  The value of keyword \vars{cont}
is stored as private instance attribute \vars{\_cont}, in case you
really need to access it elsewhere; see
Section~\ref{sec:Qtcm.private.attrib} for more details).

The example described in Section~\ref{sec:continuation.intro} is
an example of method~\ref{list:continue.from.instance} in the list
above: The second run session is continued from the state of
\vars{model}, with the values of \vars{model}'s instance variables
overriding any values in the compiled QTCM1 model in initializing
the second run session.

This method has a few caveats worthy of note:
\begin{itemize}
\item The \vars{init\_with\_instance\_state} attribute value
	will have no effect unless the instance prognostic variables
	are set, i.e., unless a previous run session has been done.
	Another way to put it is for an initial run session right
	after a \class{Qtcm} instance is created, \mods{varinit}
	will use the same initial values for prognostic variables
	(defined in \mods{defaults} module variable
	\vars{init\_prognostic\_dict})\footnote%
		{\vars{init\_prognostic\_dict} is the dictionary giving
		the default initial values of each prognostic variable
		and right-hand side (as defined by the restart file 
		specification).}
	as it would with for both
	\vars{init\_with\_instance\_state} set to True or False).

\item Continuation run sessions using this method have to continue
	with the next day from wherever the last run session left
	off, contiguously.\footnote%
		{For continuation run sessions, you keep the 
		same extension module (the compiled \fn{.so} library),
		and all the values that define the state where it
		left off.}
	If you want to do a non-contiguous run,
	create a new \class{Qtcm} instance initialized with a
	snapshot instead of the continuation method describe in
	this section.
	will use restart rules to run a new model.  

\item When making a continuation run session using this method,
	you cannot change some variables, for instance,
	\vars{outdir} and any of the date related
	variables.  In fact, the only thing you should change for
	your continuation run session are the prognostic and
	diagnostic variables and \vars{lastday}.  This is because
	some variables cannot be updated between run sessions.
	As noted in Section~\ref{sec:restart.options.list},
	if you wish to control all variables possible
	from the Python level (including output filename), you need 
	to execute the run session from a new model instance.
\end{itemize}


	\subsection{Snapshots of a \class{Qtcm} Instance}
				\label{sec:snapshots}

The snapshot dictionary (briefly described in
Section~\ref{sec:snapshot.intro}), saved as the \class{Qtcm} instance
attribute \vars{snapshot}, and generated by the method
\mods{make\_snapshot}, saves the current state of the following
instance field variables:

\begin{center}
\input{snapshot_vars.tex}
\end{center}

These are the same variables saved to a QTCM1 restart file, and so
a snapshot duplicates the restart functionality in the Python
environment, but with more flexibility.  Since the \vars{snapshot}
dictionary is a Python variable like any other, you can manipulate
it and alter it to fit any condition you wish.




%---------------------------------------------------------------------
\section{Creating and Using Run Lists}  \label{sec:runlists}

Section~\ref{sec:runlist.intro} provides an introduction to the
role and use of run lists.  A run list is a list of methods, Fortran
subroutines, and other run lists that can be executed by the
\class{Qtcm} instance \mods{run\_list} method.  Run lists are stored
in the \class{Qtcm} instance attribute \vars{runlists}, which is a
dictionary of run lists.  The names of run lists should not be
preceeded by two underscores (though elements of a run list may be
very private variables), nor should names of run lists be the same
as any instance attribute.  Run lists are not available for
\vars{compiled\_form\thinspace=\thinspace'full'}.

The \mods{run\_list} method takes a single input parameter, a list,
and runs through that list of elements that specify other run lists
or instance method names to execute.  Methods with private attribute
names are automatically mangled as needed to become executable by
the method.  Note that if an item in the input run list is an
instance method, it should be the entire name (not including the
instance name) of the callable method, separated by periods as
appropriate.

Elements in a run list are either strings or 1-element dictionaries.
Consider the following example, where \vars{model} is a \class{Qtcm}
instance, and \mods{run\_list} is called using \vars{mylist} as
input:

\begin{codeblock}
\codeblockfont{%
model = Qtcm(\ldots) \\
mylist = [ \{'varinit':None\}, \\
\hspace*{13ex}'init\_model', \\
\hspace*{13ex}'\_\_qtcm.driver', \\
\hspace*{13ex}\{'set\_qtcm1\_item': ['outdir', '/home/jlin']\} ]
model.run\_list(mylist)}
\end{codeblock}

The first element in \vars{mylist} refers to a method that requires
no positional input parameters be passed in (as shown by the None).
The second and third elements in \vars{mylist} also refers to methods
that require no positional input parameters be passed in.  The last
element in \vars{mylist} refers to a method with two input parameters.
Note that while I use the term ``method'' to describe the elements,
the strings/keys do not have to be only Python instance methods.
The second element, for instance, refers to another run list, and
the third element refers to a compiled QTCM1 model subroutine (note
the \vars{\_\_qtcm} attribute).

When the \mods{run\_list} method is called, the items in the input
run list are called in the order given in the list.  For each
element,  the \mods{run\_list} method first checks if the string
or dictionary key name corresponds to the key of an entry in the
\class{Qtcm} instance attribute \vars{runlists}.  If so, \mods{run\_list}
is called using that run list (i.e., it is a ``recursive'' call).
If the string or dictionary key name does not refer to another run
list, the \mods{run\_list} method checks if the string or dictionary
key name is a method of the \class{Qtcm} instance, and if so the
method is called.  Any other value throws an exception.

If input parameters for a method are of class \class{Field}, the
\mods{run\_list} method first tries to pass the parameters into the
method as is, i.e., as Field object(s).  If that fails, the
\mods{run\_list } method  passes its parameters in as the \vars{value}
attribute of the \class{Field} object.

If you want a variable that is being passed into a run list to be
continuously updated, you have to set the parameter in the run list
to a \class{Field} instance that is a \class{Qtcm} instance attribute,
not just to the value of the field variable (or to a non-\class{Field}
object).  Otherwise, subsequent calls to that run list element will
not use the updated values as input parameters.

For instance, if you had a run list element:
\begin{codeblock}
\codeblockfont{%
\{'\_\_qtcm.timemanager':[model.coupling\_day,]\}}
\end{codeblock}
and \vars{model.coupling\_day} were an integer (it's not by default,
but pretend it was), then \mods{run\_list} calling
\mods{\_\_qtcm.timemanager} will pass in a scalar integer rather
than a binding to the variable \vars{model.coupling\_day}.  In such
a situation, if the variable \vars{model.coupling\_day} were updated
in time, the \mods{run\_list} call of \mods{\_\_qtcm.timemanager}
would not be updated in time.  This happens because when the
dictionary that is the run list element is created, the value of
list element(s) attached to the dictionary element is set to the
scalar value of \vars{model.coupling\_day} at that instant.

You can get around this feature by setting \class{Qtcm} instance
attributes that will change with model execution to \class{Field}
instances, and then referring to those attributes in the parameter
list in the run list element.  In that case:
\begin{codeblock}
\codeblockfont{%
\{'\_\_qtcm.timemanager':[model.coupling\_day,]\}}
\end{codeblock}
will use the current value of \vars{model.coupling\_day} anytime
\vars{\_\_qtcm.timemanager} is called by \mods{run\_list}, if
\vars{model.coupling\_day} is a \class{Field} object.

When \mods{run\_list}, encounters a calling input parameter that
is a \class{Field} object, it will first try to pass the entire
\class{Field} object to the method/routine being called.  If that
raises an exception, it will then try to pass just the value of the
entire \class{Field} object.  This is done to enable \mods{run\_list}
to be used for both pure-Python and compiled QTCM Fortran model
routines.  Fortran cannot handle \class{Field} objects as input
parameters, only values.

Table~\ref{tab:stnd.runlists} shows all standard run lists
stored in the \vars{runlists} attribute upon instantiation
of a \class{Qtcm} instance.

\begin{htmlonly}
\begin{table}[htp]
\begin{center}
\fbox{Empty placeholder block for table that would have gone here.}
\end{center}
\caption{Standard run lists stored in the \vars{runlists} 
	attribute upon instantiation of a \class{Qtcm} instance.
	The run list and list element names are stored as strings.
	\emphpara{This table is improperly reproduced in the
	HTML conversion.  Please see the PDF version for the table.}}
\label{tab:stnd.runlists}
\end{table}
\end{htmlonly}

\begin{latexonly}
\begin{table}[htp]
\input{runlists}
\caption{Standard run lists stored in the \vars{runlists} 
	attribute upon instantiation of a \class{Qtcm} instance.
	The run list and list element names are stored as strings.}
\label{tab:stnd.runlists}
\end{table}
\end{latexonly}

Of course, feel free to change the contents of any of the run lists
after instantiation, or to add additional run lists to the
\vars{runlists} attribute dictionary.  The ability to alter run
lists at runtime gives the \mods{qtcm} package much of its flexibility.




%---------------------------------------------------------------------
\section{Field Variables and the \class{Field} Class}
						\label{sec:field.variables}

The term ``field'' variable refers to QTCM1 model variables that 
are accessible at both the compiled Fortran QTCM1 model-level as
well as the Python \class{Qtcm} instance-level.
Field variables are all instances of the \class{Field} class
(though non-field variables can also be instances of \class{Field}).

Section~\ref{sec:field.variables.intro} gives a brief introduction to
the attributes and methods in a \class{Field} instance.
A nitty gritty description of the class is found in its docstrings.

	\subsection{Creating Field Variables}

To create a \class{Field} instance whose value is set to the
default, instantiate with the field id as the only positional
input argument.  Thus:

\begin{codeblock}
\codeblockfont{foo = Field('lastday')}
\end{codeblock}

will return \vars{foo} as a \class{Field} instance with \vars{foo.value}
set to the value listed in Section~\ref{sec:defaults.scalar}.
The value of all \class{Field} instances upon creation are specified
in the \mods{defaults} submodule of package \mods{qtcm}, and listed
in Sections~\ref{sec:defaults.scalar} and~\ref{sec:defaults.array}.

To create \class{Field} instances whose attributes are set different
from their defaults, you can specify the different settings in the
instantiation parameter list, or change the attributes once the
instance is created.  See the \class{Field} docstring for details.


	\subsection{Initial Field Variables}  \label{sec:initial.variables}

Field variables include both model parameters that do not change
for a \class{Qtcm} instance as well as prognostic variables that
do change during model integration.  As a result, many field variables
have values different from the default values listed in
Sections~\ref{sec:defaults.scalar} and~\ref{sec:defaults.array}.
In this section, I list the \emph{initial} values of all field
variables.  The ``initial'' values are the settings for \class{Qtcm}
field variables execution of the \mods{run\_session} method, but
prior to cycling through an atmosphere-ocean coupling timestep.
This is in contrast to ``default'' values, which the field variables
are given on instantiation, if no other value is specified.
Numerical values are rounded as per the conventions
of Python's \vars{\%g} format code.


		\subsubsection{Scalars}

For the fields that give the input/output directory names, and the
run name, the entry ``value varies'' is provided in the ``Value''
column.

\input{init_scalars}

		\subsubsection{Arrays}

\input{init_arrays}


	\subsection{Passing Fields Between the Python and Fortran-Levels}

Section~\ref{sec:comm.py.fort.compiledform} discusses the differences
between how the \vars{'full'} and \vars{'parts'} compiled forms
pass field variables between the Python and Fortran-levels.  That
discussion gives a detailed description of the methods used for
passing fields to and from the Python and Fortran-levels (i.e., the
\mods{get\_qtcm1\_item} and \mods{set\_qtcm1\_item} methods).

Please note the following regarding field variables as you pass them 
back and forth between the Python and Fortran-levels:
\begin{itemize}
\item Field variables with ghost latitudes, such as \vars{u1}, on
	the Python end are always the full variables (i.e., including
	the ghost latitudes).  On the Fortran end, variables like
	\vars{u1} also always have the ghost latitudes while in the
	model, but when stored as restart files, do not have the
	ghost latitudes; the end points are not saved in restart
	files or written to the netCDF output files.
	See the
	\latexhtml{%
\htmladdnormallinkfoot{QTCM1 manual}%
        {http://www.atmos.ucla.edu/$\sim$csi/qtcm\_man/v2.3/qtcm\_manv2.3.pdf}}%
{\htmladdnormallink{QTCM1 manual}%
        {http://www.atmos.ucla.edu/~csi/qtcm_man/v2.3/qtcm_manv2.3.pdf}}
	\cite{Neelin/etal:2002}
	for details about ghost latitudes.

\item You should assume there is only a full synchronizing between 
	compiled QTCM1 model and Python model field variables
	at the beginning and end of a run session.  

\item If you have a variable at the Python-level, but at the
	compiled QTCM1 Fortran model-level the variable is not
	readable, if you try to call \mods{set\_qtcm1\_item} on the
	variable, nothing is done, and the Python-level value is
	left alone.  If you have a compiled QTCM1 model variable,
	but no Python-level equivalent, if you call \mods{set\_qtcm1\_item}
	on the variable, the Python-level variable (as an attribute)
	is created.

\item To be precise, only compiled QTCM1 model variables can be
	passed pass back and forth between the Python and Fortran-levels;
	there are many \class{Qtcm} instance attributes that do not
	have any counterparts at the Fortran-level.\footnote%
		{I use the term ``field variables'' to refer to 
		compiled QTCM1 model variables that can be passed
		back and forth to the Python level.}

\item Although \vars{dayofmodel} is described in module \mods{setbypy}
	as an option for the \mods{get\_qtcm1\_item} and
	\mods{set\_qtcm1\_item} methods to operate on, in reality
	those methods cannot operate on \vars{dayofmodel}, but
	\vars{dayofmodel} is not defined in \mods{defaults}.\footnote%
		{All field variables must be defined in \mods{defaults} in
		order for the proper Fortran routine to be called
		according to the variable's type.}
\end{itemize}


	\subsection{Field Variable Shape}   \label{sec:field.var.shape}

Normally, Python arrays have a different dimension order than Fortran
arrays.  While Fortran arrays are dimensioned (col, row, slice),
with adjacent columns being contiguous, then rows, and then slices, Python
arrays are dimensioned (slice, row, col), with adjacent columns being
contiguous, then rows, and then slices.  Based on this, you would
think that everytime you passed an array between the Python and
Fortran-levels you would need to transpose the array.

Thankfully, we don't have to do this because \mods{f2py} handles
array dimension order transparently so we can refer to each element
the same way whether we're in Python or Fortran.  Thus, the array
\vars{Qc} in Fortran is dimensioned (longitude, latitude), (64,42)
by default, and the Python \class{Qtcm} instance attribute \vars{Qc}
has a \vars{value} attribute also dimensioned (longitude, latitude),
(64,42) by default.  And at both the Fortran and Python-levels, the
first longtude, second latitude element is referred to as \vars{Qc(1,2)}.

In contrast, however, netCDF output saved by the compiled QTCM1 model
and read into Python (using the \mods{Scientific} package) is
\emph{not} in Fortran array order.  Arrays read from netCDF output
into Python are in Python array order, and are dimensioned
(latitude, longitude) or (time, latitude, longitude).  The \class{Qtcm}
routines that manipulate netCDF data (e.g., \mods{plotm}), however,
automatically adjust for this, so you only need to be aware of this
when reading in output for your own analysis
(see Section~\ref{sec:model.output}).




%---------------------------------------------------------------------
\section{Model Output}			\label{sec:model.output}

Section~\ref{sec:output.intro} gives an overview of how to
use \mods{qtcm} model output to netCDF files.

All netCDF array output is dimensioned (time, latitude, longitude)
when read into Python using the \mods{Scientific} package.  This
differs from the way \class{Qtcm} saves field variables, which
follows Fortran convention (longitude, latitude).  Thus, the shapes
in Section~\ref{sec:initial.variables}, Appendix~\ref{app:defaults.values},
etc., are not the shapes of arrays read from the netCDF output.
See Section~\ref{sec:field.var.shape} for a discussion of why
there is this discrepancy.

Because netCDF files allow you to specify an ``unlimited'' dimension,
it is possible to close a netCDF file, reopen it, and add more
slices of data to the file.  Thus, continuous \class{Qtcm} run
sessions (i.e., those that use the \vars{cont} keyword input parameter
in the \mods{run\_session} method) will automatically append output
to the netCDF output files.

Field variables with ghost latitudes, such as \vars{u1}, on the
Python and Fortran ends are always the full variables (i.e., including
the ghost latitudes).  The ghost latitudes are not written to the
netCDF output files, however.
See the \latexhtml{%
\htmladdnormallinkfoot{QTCM1 manual}%
        {http://www.atmos.ucla.edu/$\sim$csi/qtcm\_man/v2.3/qtcm\_manv2.3.pdf}}%
{\htmladdnormallink{QTCM1 manual}%
        {http://www.atmos.ucla.edu/~csi/qtcm_man/v2.3/qtcm_manv2.3.pdf}}
	\cite{Neelin/etal:2002}
for details about ghost latitude structure.

\class{Qtcm} instances have a few built-in tools to visualization
model output.  These are briefly described in Section~\ref{sec:viz.intro}.
Note that the \mods{plotm} method is linked to a specific \class{Qtcm}
instance.  Do not use \mods{plotm} outside of the instance it is
linked to.  It must also be used only after a successful run session
(i.e., not in the middle of a run session).




%---------------------------------------------------------------------
\section{Miscellaneous}

A few miscellaneous items/issues about the model:
\begin{itemize}
\item The land model runs at same timestep as the atmosphere.

\item If the land model runs less often than 
	\mods{sflux} in \mods{physics1}, 
	the calculation of evaporation over the land 
	needs to be fixed in sflux.

\item The units of some field variables are not what you would expect.
	For instance, \vars{Qc} is in energy units, i.e., K, and not
	mm/day.
	See the
	\latexhtml{%
\htmladdnormallinkfoot{QTCM1 manual}%
        {http://www.atmos.ucla.edu/$\sim$csi/qtcm\_man/v2.3/qtcm\_manv2.3.pdf}}%
{\htmladdnormallink{QTCM1 manual}%
        {http://www.atmos.ucla.edu/~csi/qtcm_man/v2.3/qtcm_manv2.3.pdf}}
	\cite{Neelin/etal:2002}
	for details.
\end{itemize}




%---------------------------------------------------------------------
\section{Cookbook of Ways the Model Can Be Used}  \label{sec:cookbook}

This cookbook of a few ways to use the model is arranged by science
tasks, i.e., certain types of runs we want to do.  For some of the
examples below, I assume that the dictionary
\vars{inputs} is initially defined as given in
Figure~\ref{fig:defn.of.inputs}.  All examples assume that
\cmd{from qtcm import Qtcm} has already been executed.


%--- Two versions, one for PDF and the other for HTML:
\begin{latexonly}
\begin{figure}[tp]
\begin{codeblock}
\codeblockfont{%
inputs = \{\} \\
inputs['runname'] = 'test' \\
inputs['landon'] = 0 \\
inputs['year0'] = 1 \\
inputs['month0'] = 11 \\
inputs['day0'] = 1 \\
inputs['lastday'] = 30 \\
inputs['mrestart'] = 0 \\
inputs['init\_with\_instance\_state'] = True \\
inputs['compiled\_form'] = 'parts'}
\end{codeblock}

\caption{The initial definition of the \vars{inputs} dictionary for 
	examples given in Section~\ref{sec:cookbook}.  These settings
	imply that a run session will start on November 1, Year 1,
	last for 30 days, and will be an aquaplanet run.}
\label{fig:defn.of.inputs}
\end{figure}
\end{latexonly}

\begin{htmlonly}
\label{fig:defn.of.inputs}
\begin{center}
\htmlfigcaption{%
	\codeblockfont{%
inputs = \{\} \\
inputs['runname'] = 'test' \\
inputs['landon'] = 0 \\
inputs['year0'] = 1 \\
inputs['month0'] = 11 \\
inputs['day0'] = 1 \\
inputs['lastday'] = 30 \\
inputs['mrestart'] = 0 \\
inputs['init\_with\_instance\_state'] = True \\
inputs['compiled\_form'] = 'parts'}
	}

\htmlfigcaption{Figure~\ref{fig:defn.of.inputs}:
	The initial definition of the \vars{inputs} dictionary for 
	examples given in Section~\ref{sec:cookbook}.  These settings
	imply that a run session will start on November 1, Year 1,
	last for 30 days, and will be an aquaplanet run.}
\end{center}
\end{htmlonly}



\begin{description}
\item[Plain model run:]
	Here I just want to make a single model run.
	Tasks:  Instantiate a fresh model and execute a run session.
	The code to run the model is just:
	\begin{codeblock}
	\codeblockfont{%
inputs['init\_with\_instance\_state'] = False \\
model = Qtcm(**inputs) \\
model.run\_session()}
	\end{codeblock}
	where \vars{inputs} is initialized with the code in
	Figure~\ref{fig:defn.of.inputs}.


\item[Explore parameter space with a set of models:]
	Here I want to create an entire suite of separate models,
	in order to determine the sensitivity of the model to changing
	a parameter.
	To do this, I
	instantiate multiple fresh models, 
	and execute a run session for each instance, all within
	a \vars{for} loop:


%--- Two versions, because LaTeX2HTML does not correctly typeset
%    the hspace command:
\begin{latexonly}
	\begin{codeblock}
	\codeblockfont{%
import os \\
inputs['init\_with\_instance\_state'] = False \\
for i in xrange(0,1002,10): \\
\hspace*{5ex}iname = 'ziml-' + str(i) + 'm' \\
\hspace*{5ex}ipath = os.path.join('proc', iname) \\
\hspace*{5ex}os.makedirs(ipath) \\
\hspace*{5ex}model = Qtcm(**inputs) \\
\hspace*{5ex}model.ziml.value = float(i)  \\
\hspace*{5ex}model.runname.value = iname \\
\hspace*{5ex}model.outdir.value = ipath \\
\hspace*{5ex}model.run\_session() \\
\hspace*{5ex}del model}
	\end{codeblock}
\end{latexonly}

\begin{htmlonly}
\begin{center}
\htmlfigcaption{%
	\codeblockfont{%
import os \\
inputs['init\_with\_instance\_state'] = False \\
for i in xrange(0,1002,10): \\
\hspace*{5ex}iname = 'ziml-' + str(i) + 'm' \\
\hspace*{5ex}ipath = os.path.join('proc', iname) \\
\hspace*{5ex}os.makedirs(ipath) \\
\hspace*{5ex}model = Qtcm(**inputs) \\
\hspace*{5ex}model.ziml.value = float(i)  \\
\hspace*{5ex}model.runname.value = iname \\
\hspace*{5ex}model.outdir.value = ipath \\
\hspace*{5ex}model.run\_session() \\
\hspace*{5ex}del model}
	}
\end{center}
\end{htmlonly}


	The loop explores mixed-layer depth \vars{ziml} from 0~m to
        1000~m, in 10~m intervals.  I create the \vars{outdir}
	directory before every model call, since the compiled QTCM1 model
	requires the output directory exist, specifying the run name
	and output directory as the string \vars{iname}.
	The output directories are assumed to all be in the \fn{proc}
	sub-directory of the current working directory.
	\vars{inputs} is initialized with the code in
	Figure~\ref{fig:defn.of.inputs}.


\item[Conditionally explore parameter space:]
	Here I want to 
	conditionally explore the parameter space, on the basis of
	some mathematical criteria.
	To do this, I
	instantiate a model, evaluate results using
	that criteria, and run another fresh model depending on
	the results (passing the previous model state via a snapshot),
	all within a \vars{while} loop.
	Note that this type of investigation is very difficult to 
	automate if all you can use are shell scripts and
	Fortran.
	See Figure~\ref{fig:conditional.test.eg} for a detailed
	example.


\item[With interactive adjustments at run time:]
	The example in Figure~\ref{sec:continuation.intro}
	illustrates this type of run.  In this example,
	I instantiate a fresh model, execute a run session, analyze the
	output, change variables in the model instance, and then
	execute a continuation run session.


\item[Test alternative parameterizations:]
	I've already described how we can use run lists to arbitrarily
	change model execution order and content at run time.
	We can take advantage of Python's inheritance
	abilities, along with run lists, to simplify this.
	Figure~\ref{fig:alt.param.inherit.eg} provides an example of
	this use.

	Of course, you can use pre-processor directives and shell
	scripts to accomplish the same functionality seen in
	Figure~\ref{fig:alt.param.inherit.eg} using just Fortran.
	The Python solution, however, shortcuts the compile/linking
	step, and enables you to easily do run time swapping between
	subroutine choices based upon run time calculated
	tests (see Figure~\ref{fig:conditional.test.eg} for an
	example of such tests).
\end{description}




% --- Two versions of this block, one for display in PDF and the other
%     for display in HTML:
\begin{latexonly}
\begin{figure}[p]
	\begin{codeblock}
	\codeblockfont{%
\small
import os \\
import numpy as N \\
maxu1 = 0.0 \\
while maxu1 < 10.0: \\
\hspace*{5ex}iziml = 0.1 * maxu1 \\
\hspace*{5ex}iname = 'ziml-' + str(iziml) + 'm' \\
\hspace*{5ex}ipath = os.path.join('proc', iname) \\
\hspace*{5ex}os.makedirs(ipath) \\
\hspace*{5ex}model = Qtcm(**inputs) \\
\hspace*{5ex}try: \\
\hspace*{10ex}model.sync\_set\_py\_values\_to\_snapshot(snapshot=mysnapshot) \\
\hspace*{10ex}model.init\_with\_instance\_state = True \\
\hspace*{5ex}except: \\
\hspace*{10ex}model.init\_with\_instance\_state = False \\
\hspace*{5ex}model.ziml.value = iziml  \\
\hspace*{5ex}model.runname.value = iname \\
\hspace*{5ex}model.outdir.value = ipath \\
\hspace*{5ex}model.run\_session() \\
\hspace*{5ex}maxu1 = N.max(N.abs(model.u1.value)) \\
\hspace*{5ex}mysnapshot = model.snapshot \\
\hspace*{5ex}del model}
	\end{codeblock}

\caption{This code explores different values of
	mixed-layer depth \vars{ziml} for 30~day runs,
	as a function of maximum \vars{u1} magnitude,
	until it finds a case where the maximum \vars{u1} is
	greater than 10~m/s.  (The relationship between
	\vars{ziml} and the maximum of the speed of
	\vars{u1}, where 
	\vars{ziml\thinspace=\thinspace0.1\thinspace*\thinspace{maxu1}}, 
	is made up.)
	With each iteration, the new run uses the snapshot from
	a previous run to initialize (as well as the new value
	of \vars{ziml}); the \vars{try} statement is used to
	ensure the model works even if \vars{mysnapshot} is not
	defined (which is the case the first time around).
	The \vars{inputs} dictionary is initialized with the code in
	Figure~\ref{fig:defn.of.inputs}.}
\label{fig:conditional.test.eg}
\end{figure}
\end{latexonly}

\begin{htmlonly}
\label{fig:conditional.test.eg}
\begin{center}
\htmlfigcaption{%
	\codeblockfont{%
import os \\
import numpy as N \\
maxu1 = 0.0 \\
while maxu1 < 10.0: \\
\hspace*{5ex}iziml = 0.1 * maxu1 \\
\hspace*{5ex}iname = 'ziml-' + str(iziml) + 'm' \\
\hspace*{5ex}ipath = os.path.join('proc', iname) \\
\hspace*{5ex}os.makedirs(ipath) \\
\hspace*{5ex}model = Qtcm(**inputs) \\
\hspace*{5ex}try: \\
\hspace*{10ex}model.sync\_set\_py\_values\_to\_snapshot(snapshot=mysnapshot) \\
\hspace*{10ex}model.init\_with\_instance\_state = True \\
\hspace*{5ex}except: \\
\hspace*{10ex}model.init\_with\_instance\_state = False \\
\hspace*{5ex}model.ziml.value = iziml  \\
\hspace*{5ex}model.runname.value = iname \\
\hspace*{5ex}model.outdir.value = ipath \\
\hspace*{5ex}model.run\_session() \\
\hspace*{5ex}maxu1 = N.max(N.abs(model.u1.value)) \\
\hspace*{5ex}mysnapshot = model.snapshot \\
\hspace*{5ex}del model}
	}

\htmlfigcaption{Figure \ref{fig:conditional.test.eg}:
	This code explores different values of
	mixed-layer depth \vars{ziml} for 30~day runs,
	as a function of maximum \vars{u1} magnitude,
	until it finds a case where the maximum \vars{u1} is
	greater than 10~m/s.  (The relationship between
	\vars{ziml} and the maximum of the speed of
	\vars{u1}, where 
	\vars{ziml\thinspace=\thinspace0.1\thinspace*\thinspace{maxu1}}, 
	is made up.)
	With each iteration, the new run uses the snapshot from
	a previous run to initialize (as well as the new value
	of \vars{ziml}); the \vars{try} statement is used to
	ensure the model works even if \vars{mysnapshot} is not
	defined (which is the case the first time around).
	The \vars{inputs} dictionary is initialized with the code in
	Figure~\ref{fig:defn.of.inputs}.}
\end{center}
\end{htmlonly}


% --- Two versions of this block, one for display in PDF and the other
%     for display in HTML:
\begin{latexonly}
\begin{figure}[p]
\begin{center}
	\begin{codeblock}
	\codeblockfont{%
\small
import os \\
\\
class NewQtcm(Qtcm): \\
\hspace*{5ex}def cloud0(self):\\
\hspace*{10ex}[\ldots] \\
\hspace*{5ex}def cloud1(self):\\
\hspace*{10ex}[\ldots] \\
\hspace*{5ex}def cloud2(self):\\
\hspace*{10ex}[\ldots] \\
\hspace*{5ex}[\ldots] \\
\\
inputs['init\_with\_instance\_state'] = False \\
for i in xrange(10): \\
\hspace*{5ex}iname = 'cloudroutine-' + str(i)  \\
\hspace*{5ex}ipath = os.path.join('proc', iname) \\
\hspace*{5ex}os.makedirs(ipath) \\
\hspace*{5ex}model = NewQtcm(**inputs) \\
\hspace*{5ex}model.runlists['atm\_physics1'][1] = 'cloud' + str(i) \\
\hspace*{5ex}model.runname.value = iname \\
\hspace*{5ex}model.outdir.value = ipath \\
\hspace*{5ex}model.run\_session() \\
\hspace*{5ex}del model}
	\end{codeblock}
\end{center}

\caption{Let's say we have 9 different cloud physics schemes we wish
	to try out in 9 different runs.  The easiest way to do this
	is to create a new class \class{NewQtcm} that
	inherits everything from \class{Qtcm}, and to which we'll
	add the additional cloud schemes (\vars{cloud0}, \vars{cloud1},
	etc.).
	In the \vars{for} loop, I change the cloud model
	run list entry in the run list that governs
	atmospheric physics at one instant to whatever the cloud
	model is at this point in the loop.
	The \vars{inputs} dictionary is initialized with the code in
	Figure~\ref{fig:defn.of.inputs}.
	Of course, we could do the same thing by running the 9
	models separately, but this set-up makes it easy to do
	hypothesis testing with these 9 models.  For instance, we
	can create a test by which we will choose which of the 9
	models to use:  Within this framework, the selection of
	those models can be altered by changing a string.}
\label{fig:alt.param.inherit.eg}
\end{figure}
\end{latexonly}

\begin{htmlonly}
\label{fig:alt.param.inherit.eg}
\begin{center}
\htmlfigcaption{%
	\codeblockfont{%
import os \\
\\
class NewQtcm(Qtcm): \\
\hspace*{5ex}def cloud0(self):\\
\hspace*{10ex}[\ldots] \\
\hspace*{5ex}def cloud1(self):\\
\hspace*{10ex}[\ldots] \\
\hspace*{5ex}def cloud2(self):\\
\hspace*{10ex}[\ldots] \\
\hspace*{5ex}[\ldots] \\
\\
inputs['init\_with\_instance\_state'] = False \\
for i in xrange(10): \\
\hspace*{5ex}iname = 'cloudroutine-' + str(i)  \\
\hspace*{5ex}ipath = os.path.join('proc', iname) \\
\hspace*{5ex}os.makedirs(ipath) \\
\hspace*{5ex}model = NewQtcm(**inputs) \\
\hspace*{5ex}model.runlists['atm\_physics1'][1] = 'cloud' + str(i) \\
\hspace*{5ex}model.runname.value = iname \\
\hspace*{5ex}model.outdir.value = ipath \\
\hspace*{5ex}model.run\_session() \\
\hspace*{5ex}del model}
	}

\htmlfigcaption{Figure \ref{fig:alt.param.inherit.eg}:
	Let's say we have 9 different cloud physics schemes we wish
	to try out in 9 different runs.  The easiest way to do this
	is to create a new class \class{NewQtcm} that
	inherits everything from \class{Qtcm}, and to which we'll
	add the additional cloud schemes (\vars{cloud0}, \vars{cloud1},
	etc.).
	In the \vars{for} loop, I change the cloud model
	run list entry in the run list that governs
	atmospheric physics at one instant to whatever the cloud
	model is at this point in the loop.
	The \vars{inputs} dictionary is initialized with the code in
	Figure~\ref{fig:defn.of.inputs}.
	Of course, we could do the same thing by running the 9
	models separately, but this set-up makes it easy to do
	hypothesis testing with these 9 models.  For instance, we
	can create a test by which we will choose which of the 9
	models to use:  Within this framework, the selection of
	those models can be altered by changing a string.}
\end{center}
\end{htmlonly}




% ===== end of file =====


%@@@\chapter{Combining \code{qtcm} with \code{CliMT}}
%@@@% ==========================================================================
% CliMT
%
% By Johnny Lin
% ==========================================================================


% ------ BODY -----
%
\section{General Tutorial on CliMT}


General notes of things I think I may have observed about
\code{Parameters} objects:
\begin{itemize}
\item You can treat a \code{Parameters} instance as a dictionary, where
	the key is the name of the field, because \code{\_\_getitem\_\_},
	etc.\ have been defined for the instance.  However, the values,
	units, and long names of the fields are stored in dictionaries
	assigned to \code{value}, \code{units}, and \code{long\_name},
	keyed to the field name (a string).
\end{itemize}


General notes of things I think I may have observed about
\code{Components} objects:
\begin{itemize}
\item All variables and quantities, whether they be physical fields,
	filenames, or metadata,
	are stored as attributes in the \code{Components} instance.
\item \code{Components} have these special attributes:
        \code{Required},
        \code{Prognostic},
	and
        \code{Diagnostic},
	which are lists that contain the names of describe whether
\item Scalar parameters in \code{Component} objects
	are stored as an instance of the \code{Parameters}
	class, under the attribute \code{Params}.
\end{itemize}


General notes of things I think I may have observed about
\code{Federation} objects:
\begin{itemize}
\item \code{Federation} objects hold the \code{Components} instances
	in a list assigned to the attribute \code{list}.
\item \code{Federation} attributes
        \code{Required},
        and
	\code{Prognostic},
	are unions of the same attributes of the constituent
	\code{Components} objects.
\end{itemize}






% ===== end of file =====


\chapter{Troubleshooting}                   \label{ch:trouble}
% ==========================================================================
% Troubleshooting
%
% By Johnny Lin
% ==========================================================================


% ------ BODY -----
%
\section{Error Messages Produced by \mods{qtcm}}

\begin{description}
\item[\screen{Error-Value too long in SetbyPy module getitem\_str for}
	\dumarg{key}:]
	This message is produced by the Fortran
	subroutine \mods{getitem\_str}
	in the module \mods{SetbyPy} in the compiled QTCM1 Fortran code.
	The code is in the file \fn{setbypy.F90}.  This error occurs when
	the Fortran variable whose name is given by the string \dumarg{key}
	has a value that is greater than the local parameter
	\vars{maxitemlen} in \mods{getitem\_str}.  To fix this, you have
	to go into \fn{setbypy.F90} and change the value of
	\vars{maxitemlen}.

\item[\screen{Error-real\_rank1\_array should be deallocated}:]
	Fortran module \mods{SetByPy}'s subroutine
	\mods{getitem\_real\_array} generates this message
	(or a similar message for other ranks) if the Fortran
	variable for the input \dumarg{key} are allocated on entry
	to the routine.  This may indicate the user has written another
	Fortran routine to access the \mods{real\_rank1\_array} variable
	outside of the standard interfaces..

\item[\screen{Error-Bad call to SetbyPy module \ldots}:]
	Often times, this error occurs because a get or set routine
	in \mods{SetByPy} tried to act on a variable for which the
	corresponding input \dumarg{key} is not defined.  The solution
	is to add that case in the if/then construct for the get and set
	routines in \mods{SetByPy} and rebuild the extension modules.
\end{description}


\section{Other Errors}

\begin{description}
\item[Python cannot find some packages:]
	This error often happens when the version of Python in which
	you have installed all your packages is not the version that
	is called at the Unix command line by typing in \cmd{python}.
	To get around this, 
        define a Unix alias
        that maps \cmd{python2.4} (or whichever version of Python
	has all your packages installed) to \cmd{python}.  If you
	have multiple Python's installed on your system, you might
	have to use a more specific name for the Python executable.
	As a result, you may have to change the test scripts in
	\fn{test} in the \mods{qtcm} distribution directory.

\item[\mods{get\_qtcm1\_item} and compiled QTCM1 model pointer
	variables:]
	If you try to use the \mods{get\_qtcm1\_item} method on a compiled
	QTCM1 model pointer variable 
	(i.e., \vars{u1}, \vars{v1}, \vars{q1}, \vars{T1}),
	 before the compiled
	model \mods{varinit} subroutine is run, you'll get a bus error
	with no additional message.

\item[Mismatch between Python and Fortran array field variables:]
	You change an array field variable on the Python side, but
	it seems like the wrong elements are changed on the Fortran
	side.  Or you type in the same index address for accessing a
	\mods{qtcm} netCDF output array as well as its \class{Qtcm}
	instance attribute counterpart, and find you get different
	answers.  Some possible reasons and fixes:

	\begin{itemize}
	\item This will occur if you haven't accounted for the
		difference in how field variables are saved at the
		Python-level, Fortran-level, and in a netCDF file.
		All netCDF array output is dimensioned (time,
		latitude, longitude) when read into Python using
		the \mods{Scientific} package.  This differs from
		the way \class{Qtcm} saves field variables, \emph{both}
		at the Python- and Fortran-levels, which follows
		Fortran convention (longitude, latitude).

		Note that the way \class{Qtcm} saves field variables
		at the Python- and Fortran-levels is different than
		the default way Python and Fortran save arrays.
		Section~\ref{sec:field.var.shape} for more information.

	\item You may have forgotten that array indices in Python start at
		0, while indices in Fortran (generally) start at 1.
		Also, ranges in Python are exclusive at the upper-bound,
		while ranges in Fortran are inclusive at the upper-bound.
		(Both Python and Fortran array indice ranges are inclusive
		at the lower-bound.)

	\item You may have forgotten some field variables have
		ghost latitudes, and thus there are extra latitude bands
		when the array is stored as a Python or Fortran field
		variable, but there are \emph{no} extra latitude bands
		when the array is stored as netCDF output (the QTCM1
		output routines strip off the ghost latitudes when
		writing those field variables out).
	        See the
        \latexhtml{%
\htmladdnormallinkfoot{QTCM1 manual}%
        {http://www.atmos.ucla.edu/$\sim$csi/qtcm\_man/v2.3/qtcm\_manv2.3.pdf}}%
{\htmladdnormallink{QTCM1 manual}%
        {http://www.atmos.ucla.edu/~csi/qtcm_man/v2.3/qtcm_manv2.3.pdf}}
        \cite{Neelin/etal:2002}
        for details about ghost latitudes.

		The safest and easiest way to tell whether the variable has a
		ghost latitudes is to look at its shape.
		A call to the \class{Qtcm} instance
		method \mods{get\_qtcm1\_item} will give you the array,
		and the use of NumPy's \mods{shape} function will give you
		the shape.
	\end{itemize}
\end{description}




% ===== end of file =====


\chapter{Developer Notes}                   \label{ch:devnotes}
% ==========================================================================
% Using QTCM
%
% By Johnny Lin
% ==========================================================================


% ------ BODY -----
%

%---------------------------------------------------------------------
\section{Introduction}

This section describes programming practices and issues related to
the \mods{qtcm} package that might be of interest to users wishing
to add/change code in the package.
Please see the package
\latexhtml{API documentation,%
		\footnote{http://www.johnny-lin.com/py\_pkgs/qtcm/doc/html-api/}
		which includes the source code}%
        {\htmladdnormallink{API documentation}%
		{http://www.johnny-lin.com/py\_pkgs/qtcm/doc/html-api/},
		which includes the source code},
for details.




%---------------------------------------------------------------------
\section{Changes to QTCM1 Fortran Files}  \label{sec:f90changes}

The source code used to generate the shared object files used
in this Python package is unchanged
from the pure-Fortran QTCM1 model source code, except in the
following ways:

\begin{itemize}
\item The suffix of all source code files 
	has been changed from \fn{.f90} to \fn{.F90}, 
	in order to ensure the compiler preprocesses 
	the source code.  Some compilers use the capitalization to
	tell whether or not to run the code through a preprocessor.

\item In all \fn{.F90} files, occurrences of:
	\begin{codeblock}
	\codeblockfont{%
	Character(len=130)}
	\end{codeblock}
	are changed to:
	\begin{codeblock}
	\codeblockfont{%
	Character(len=305)}
	\end{codeblock}
	This enables the model to properly deal with longer filenames.
	The number ``305'' is chosen to make search and replace easier.

\item In \fn{qtcmpar.F90}, the 
	\vars{eps\_c} variable is changed from an unchangable
	parameter to a changeable real, 
	so that it can be changed in the model at runtime.

\item All occurrences of an underscore (``\_'') character in a
	subroutine or function name are removed.  The
	presence of the underscore messes up the dynamic lookup
	mechanism for the \mods{f2py} generated extension module.
	The following names are changed, both in subroutine definitions
	and calls:
	\begin{itemize}
	\item \mods{out\_restart} to \mods{outrestart},
	\item \mods{save\_bartr} to \mods{savebartr},
	\item \mods{grad\_phis} to \mods{gradphis}.
	\end{itemize}

\item \fn{driver.F90} is changed so that program
	\mods{driver} becomes a subroutine, and 
	subroutine \mods{driverinit} is deleted (along with
	all calls to it) because basic model initialization is
	handled at the Python level.

\item In \fn{clrad.F90}, subroutine \mods{cloud}, the first
	\vars{COUNTCAP} preprocessor macro, a comment line for
	that ifdef is moved to prevent a warning message during
	building with \mods{f2py}.

\item The order of subroutine \mods{qtcminit} is changed.  The original
	pure-Fortran QTCM1 \mods{qtcminit} code has the following
	calling sequence:

	\begin{codeblock}
        \codeblockfont{%
Call parinit            !Initialize model parameters \\
Call varinit            !Initialize variables \\
Call TimeManager(1)     !mm set model time \\
Call bndinit            !input boundary datasets \\
Call physics1           !diagnostic fields for initial condition}
	\end{codeblock}

	For the \mods{qtcm} package, I've altered this order so
	\mods{bndinit} comes after \mods{parinit} but before \mods{varinit}:
	\begin{codeblock}
        \codeblockfont{%
Call parinit            !Initialize model parameters \\
Call bndinit            !input boundary datasets \\
Call varinit            !Initialize variables \\
Call TimeManager(1)     !mm set model time  \\
Call physics1           !diagnostic fields for initial condition}
	\end{codeblock}

	This is done because \vars{STYPE} is not read in for the
	\vars{landon} \vars{True} case until \mods{bndinit}, but
	in \mods{varinit} \vars{STYPE} is used to calculate the
	original values of \vars{WD} for the non-restart case.  This
	also corrects the conflicting order found in the pure-Fortran
	QTCM1 manual (compare pp.\ 29 and 32).  As far as I can
	tell, \mods{bndinit} has no dependencies that require it
	to come after \mods{timemanager} or \mods{varinit}.

\end{itemize}

In addition, the Fortran files \fn{setbypy.F90}, \fn{wrapcall.F90},
and \fn{varptrinit.F90} are added.  The routines in these files, 
however, just add more flexibility and functionality to the model;
they do not automatically affect any model computations.  See
Section~\ref{sec:newf90} for details.




%---------------------------------------------------------------------
\section{New Interfaces and Fortran Functionality}  \label{sec:newf90}

As described in Section~\ref{sec:f90changes}, the Fortran files
\fn{setbypy.F90}, \fn{wrapcall.F90}, and \fn{varptrinit.F90} are
added to the QTCM1 source directory.  The first two files define the Fortran
90 modules (\mods{SetbyPy} and \mods{WrapCall}) needed to interface
the Python and Fortran levels.  The last file defines a new Fortran
subroutine \mods{varptrinit} that associates QTCM1 model pointer
variables at the Fortran level.  In a pure-Fortran run of QTCM1,
this occurs in subroutine \mods{varinit}; for a
\vars{compiled\_form\thinspace=\thinspace'parts'} run, since the
functionality of the Fortran \mods{varinit} is now in the Python
\mods{varinit} method, a separate Fortran pointer association
subroutine needed to be defined.  The Fortran subroutine \mods{varptrinit}
is called as the \mods{varptrinit} function of the 
\vars{compiled\_form\thinspace=\thinspace'parts'}
\fn{.so} extension module.


	\subsection{Fortran Module \mods{SetbyPy}}   \label{sec:setbypy}

		\subsubsection{Design Description}

This module defines functions and subroutines used to read variables
from the Fortran-level to the Python-level, and in setting Fortran-level
variables using the Python-level values.  These routines are used
by \class{Qtcm} methods \mods{get\_qtcm1\_item} and \mods{set\_qtcm1\_item}
(and dependencies thereof) to ``get'' and ``set'' the Fortran-level
variables.  Note that the Fortran module \mods{SetbyPy} is referred
to in lowercase at the Python level, i.e., as the
attribute \vars{\_\_.qtcm.setbypy} of a \class{Qtcm} instance.

Because Fortran variables are not dynamically typed, separate Fortran
functions and subroutines need to be defined to get and set variables
of different types.\footnote%
	{The \mods{interface} construct in Fortran 90 is supposed to
	provide a way to create a single interface to multiple
	routines, e.g.:
	\begin{codeblock}
	\codeblockfont{%
Interface setitem \\
\hspace*{3ex}Module Procedure setitem\_real, setitem\_int, setitem\_str \\
End Interface}
	\end{codeblock}
	This construct, however, causes a bus error
	(Mac OS X 10.4, Intel).  Thus, I put the
	same functionality in the Python code.}
The \class{Qtcm} methods \mods{get\_qtcm1\_item}
and \mods{set\_qtcm1\_item} know which one of the Fortran routines
to call on the basis of the type and rank of the value for the field
variable in the \mods{defaults} submodule.  This is why all field
variables need to have defaults defined in \mods{defaults}.  For
array variables, the field variable defaults also provide the rank
of the Fortran-level variable being gotten or set.  However, the
array default values do \emph{not} have to have the same shape as
the Fortran-level variables; on the Python-side, variable shape
adjusts to what is declared on the Fortran-side.  
Thus, if you change the resolution of
the compiled QTCM1 model, you do not have to change the dimensions
of the field variable values in \mods{defaults}.

The \class{Qtcm} method \mods{get\_qtcm1\_item} directly calls
the \mods{SetByPy} routines.
The \class{Qtcm} method \mods{set\_qtcm1\_item} makes use of
private instance methods that make the calls to the \mods{SetByPy} routines.

For scalar field variables, \mods{SetByPy} provides functions and
subroutines that provide the value of the variable on output.
For array field variables, \mods{SetByPy}
dynamic \emph{module} arrays are used to pass array
variables in and out; I could not get the 
\mods{SetByPy} Fortran routines to set
locally defined dynamic arrays (that is, locally within a function or
subroutine).\footnote%
	{I tried to implement Fortran subroutine
	\mods{getitem\_real\_array} using traditional array 
	dimension passing 
	(e.g., \code{subroutine foo(nx, ny, a)}) as well
	as declaring the allocatable array inside the subroutine, 
	but neither option worked on my \mods{f2py} (version 2\_3816) 
	and Python (version 2.4.3).}
In the \mods{SetByPy} module, these dynamic arrays
are defined as follows:

\begin{codeblock}
\codeblockfont{%
Real, allocatable, dimension(:) :: real\_rank1\_array \\
Real, allocatable, dimension(:,:) :: real\_rank2\_array \\
Real, allocatable, dimension(:,:,:) :: real\_rank3\_array}
\end{codeblock}

For all field variables, scalar or array, the \mods{SetByPy} module
has a fourth module variable defined, \vars{is\_readable}, that the
Fortran get and set routines will set to \vars{.TRUE.} if the
variable is readable and \vars{.FALSE.} if not (it's declared as a
logical variable).  This Fortran variable can be used to prevent
Python from accessing pointer variables that aren't yet associated
to targets.

In general, \mods{SetByPy} routines make use of Fortran constructs
to enable them to accomodate all possible
variables of a given type and shape.  However, 
for string scalars, the \mods{SetByPy} function \mods{getitem\_str}
has to have a return value of a predefined length, in order to
work properly.  That length is given by the parameter
\vars{maxitemlen} and is set to 505 (the value is chosen to
be larger than all filename variables described in
Section~\ref{sec:f90changes} and to be easily found in
the \fn{.F90} files).


		\subsubsection{Module Structure}

If you're a Fortran programmer, you can probably get all the information
in this section from just reading the \fn{setbypy.F90} file directly.
This description of the module structure, however, permits me to highlight
what you need to do if you want to make additional compiled QTCM1 variables
accessible to Python \class{Qtcm} objects.

\begin{itemize}
\item All \mods{Use} statements are given in the beginning of 
	the \mods{SetByPy} module.  These statements cover
	nearly all of the QTCM1 Fortran
	modules that contain variables of interest.  If the
	QTCM1 variable you're interested in isn't in a module
	listed here, you'll have to add your own
	\mods{Use} statement of that module here.

\item Next comes the definitions for the
	\vars{real\_rank1\_array},
	\vars{real\_rank2\_array}, and
	\vars{real\_rank3\_array} dynamic array variables, and
	the \vars{is\_readable} boolean variable.

\item The \mods{Contains} block of the module defines the module
	routines called by the \class{Qtcm} instance methods to
	set and get the compiled QTCM1 model variables.  The
	routines are:
	\begin{itemize}
	\item Function \mods{getitem\_real}
	\item Subroutine \mods{getitem\_real\_array}
	\item Function \mods{getitem\_int}
	\item Function \mods{getitem\_str}
	\item Subroutine \mods{setitem\_real}
	\item Subroutine \mods{setitem\_real\_array}
	\item Subroutine \mods{setitem\_int}
	\item Subroutine \mods{setitem\_str}
	\end{itemize}

\end{itemize}

Each of the routines in the module \mods{Contains} block is essentially
a list of \mods{if}/\mods{elseif} statements.  The list tests for the
name of the variable of interest (a string), and gets or sets the
compiled QTCM1 model variable corresponding to that name.  For pointer
array variables, a test is also made as to whether or not the variable
has been associated.  If not, the variable is not readable
and \vars{is\_readable} is set to \vars{.FALSE.}\ accordingly.

If you wish to add another compiled QTCM1 model variable to be
accessible to \class{Qtcm} instance methods \mods{get\_qtcm1\_item}
and \mods{set\_qtcm1\_item}, just add another \mods{if}/\mods{else\-if},
like the other \mods{if}/\mods{elseif} blocks, in the Fortran set
and get routines corresponding to the QTCM1 variable type (scalar
vs.\ array, and real, integer, or string).  On the Python side, add
an entry in \mods{defaults} corresponding to the new field variable
you've created access to.  I would strongly recommend making the
Python name of your new field variable
(given in \mods{defaults}) be the same as the compiled
QTCM1 model variable name.



	\subsection{Fortran Module \mods{WrapCall}}   \label{sec:wrapcall}

Most of the time, if you want to call a compiled QTCM1 model subroutine
from the Python level, you will use the version of the subroutine that
is found in this Fortran module.  
Note that the Fortran module \mods{WrapCall} is referred
to in lowercase at the Python level, i.e., as the
attribute \vars{\_\_.qtcm.wrapcall} of a \class{Qtcm} instance.

All the routines in this module do is wrap one of the compiled QTCM1
model routines.  For instance, \mods{WrapCall} subroutine
\mods{wadvcttq} is defined as just:

% --- Two versions of this block, one for display in PDF and the other
%     for display in HTML:
%
\begin{latexonly}
\begin{codeblock}
\codeblockfont{%
Subroutine wadvcttq \\
\hspace*{3ex}Call advcttq \\
End Subroutine wadvcttq}
\end{codeblock}
\end{latexonly}

\begin{htmlonly}
\begin{rawhtml}
<p><code><font color="blue">Subroutine wadvcttq<br>
&nbsp;&nbsp;&nbsp;Call advcttq<br>
End Subroutine wadvcttq</font></code></p>
\end{rawhtml}
\end{htmlonly}

All subroutines in this module begin with ``w'', with the rest of
the name being the Fortran QTCM1 subroutine name.  The calling
interface for the ``w'' version is the same as the Fortran QTCM1
original version.  There are no subroutines in this module that do
not have an exact counterpart in the Fortran QTCM1 code, and thus
this module's subroutines sole purpose is to call other subroutines
in the compiled QTCM1 model.

These wrapper routines are needed because \mods{f2py}, for some
reason I can't figure out, will not properly wrap Fortran routines
(that are then callable at the Python level) that create local
arrays using parameters obtained through a Fortran \mods{use}
statment.  Thus, as an example, a Fortran subroutine \mods{foo}
with the following definition:

% --- Two versions of this block, one for display in PDF and the other
%     for display in HTML:
%
\begin{latexonly}
\begin{codeblock}
\codeblockfont{%
subroutine foo \\
\hspace*{3ex}use dimensions \\
\hspace*{3ex}real a(nx,ny) \\
\hspace*{3ex}[\ldots] \\
end subroutine foo}
\end{codeblock}
\end{latexonly}

\begin{htmlonly}
\begin{rawhtml}
<p><code><font color="blue">
subroutine foo<br>
&nbsp;&nbsp;&nbsp;use dimensions<br>
&nbsp;&nbsp;&nbsp;real a(nx,ny)<br>
&nbsp;&nbsp;&nbsp;[\ldots]<br>
end subroutine foo
</font></code></p>
\end{rawhtml}
\end{htmlonly}


where \vars{nx} and \vars{ny} are defined in the module vars{dimensions},
will return an error, with the result that the extension module
will not be created, or an extension modules that yields output
that is different from running the pure-Fortran version of QTCM1.

By wrapping these calls into this file, I also avoid having to
separate out the Fortran QTCM1 subroutines into separate \fn{.F90}
files.  For Fortran subroutines that you want callable from the
Python level, \mods{f2py} seems to require each Fortran subroutine
to be in its own file of the same name (e.g., the version of
\fn{driver.F90} for this package). If several Fortran subroutines
are all found in a single \fn{.F90} files, \mods{f2py} seems unable
to create wrappers for those subroutines.




%---------------------------------------------------------------------
\section{Python \mods{qtcm} and Pure-Fortran QTCM1 Differences}

This section describes differences between how the \mods{qtcm}
package and the pure-Fortran QTCM1 assign some varables.  A list
of changes to the QTCM1 Fortran Files for use in the \mods{qtcm}
package is found in Section~\ref{sec:f90changes}.


	\subsection{QTCM1 \mods{driverinit}}   \label{sec:driverinit.diffs}

In the pure-Fortran version of QTCM1, by default, the following variables are
set by reference (as given below), not by value, in the \mods{driverinit}
routine:\footnote%
	{In the pure-Fortran version of QTCM1, this routine is found
	in \fn{driver.F90}.}
\begin{codeblock}
\codeblockfont{%
lastday\thinspace=\thinspace{daysperyear} \\
viscxu0\thinspace=\thinspace{viscU} \\
viscyu0\thinspace=\thinspace{viscU} \\
visc4x\thinspace=\thinspace{viscU} \\
visc4y\thinspace=\thinspace{viscU} \\
viscxu1\thinspace=\thinspace{viscU} \\
viscyu1\thinspace=\thinspace{viscU} \\
viscxT\thinspace=\thinspace{viscT} \\
viscyT\thinspace=\thinspace{viscT} \\
viscxq\thinspace=\thinspace{viscQ} \\
viscyq\thinspace=\thinspace{viscQ}}
\end{codeblock}

Thus, in pure-Fortran QTCM1, if you change \vars{daysperyear},
\vars{viscU}, etc.
and recompile (as needed), you will automatically change 
\vars{lastday}, \vars{viscxu0}, etc.
(Though, in the pure-Fortran QTCM1, the default values may be overwritten by
namelist input values.)

The \mods{driverinit} routine is eliminated
in the Python \code{qtcm} package.  Instead, inital values 
of field variables are specified in the \mods{defaults} submodule
and set by value to attributes of the \code{Qtcm} instance.
Thus, for instance, in a \class{Qtcm} instance, \code{lastday} 
is set to \code{365} by default, not to some variable
\vars{daysperyear}.  For the diffusion and viscosity terms,
the \class{Qtcm} instance attributes corresponding to those
terms are set to literals.\footnote%
	{Those literals are defined by \mods{defaults} private
	module variables \vars{\_\_viscT}, \vars{\_\_viscQ},
	and \vars{\_\_viscU}.}

In contrast, in the pure-Fortran QTCM1,
\mods{driverinit} declares local
variables \code{viscU}, \code{viscT}, and \code{viscQ},
and reads values into those variables via the input namelist.
Those values are then used to set
\vars{viscxu0}, \vars{viscyu0}, etc., as described above.
In pure-Fortran QTCM1, \code{viscU}, \code{viscT}, and \code{viscQ}
are not directly accessed anywhere else in the model.
Thus, \code{viscU}, \code{viscT}, and \code{viscQ} are not
defined as field variables in the \code{qtcm} package, and
\class{Qtcm} instances do not have attributes corresponding
to those names.
Additionally, if you wish to change a viscosity parameter
\vars{visc*} (given above), the parameter for each direction
must be set one-by-one even if the flow is isotropic.


	\subsection{The \mods{varinit} Routine}

One of the functions of the pure-Fortran QTCM1 \mods{varinit}
subroutine is to associate the pointer variables \vars{u1}, \vars{v1},
\vars{q1}, and \vars{T1}.  For the extension modules in the \mods{qtcm}
package, a Fortran subroutine \mods{varptrinit} is added that can
also do this association.  This subroutine is called in the
\class{Qtcm} instance method
\latexhtml{\mods{varinit}%
		\footnote{http://www.johnny-lin.com/py\_docs/qtcm/doc/html-api/qtcm.qtcm.Qtcm-class.html\#varinit}}%
	{\htmladdnormallink{\mods{varinit}}{http://www.johnny-lin.com/py_docs/qtcm/doc/html-api/qtcm.qtcm.Qtcm-class.html#varinit}}
(which duplicates and
extends the function of its pure-Fortran counterpart, enabling
alternative ways of handling restart).

The \mods{varptrinit} is not accessed via \mods{wrapcall}.  Remember
that \mods{wrapcall} contains only those routines that were in the
original pure-Fortran QTCM1 code, and that we want to have access
to at the Python level.


	\subsection{The \mods{qtcm} Method of \class{Qtcm}}

The \class{Qtcm} method \mods{qtcm} duplicates the functionality
of the \mods{qtcm} subroutine in the pure-Fortran QTCM1 model.
There are a few differences, however.  First, the \mods{qtcm} method
for \class{Qtcm} instances does not include a call to \mods{cplmean},
which uses mean surface flux for air-sea coupling.  This state is
consistent with the pure-Fortran QTCM1 pre-processor macro
\vars{CPLMEAN} being off.  Thus, if you wish to use mean surface
flux for air-sea coupling, you will have to revise the \mods{qtcm}
method of \class{Qtcm} to call \mods{cplmean}.  You'll also have to
check for any other code additions needed that are associated with
the \vars{CPLMEAN} macro.

Second, the \mods{qtcm} method for \class{Qtcm} instances does not
include the option of not using the atmospheric boundary layer
model.  This is consistent with macro \vars{NO\_ABL} being off.  If
you wish to have no atmospheric boundary layer model, change the
run list \vars{atm\_bartr\_mode} so that the \mods{wsavebartr} and
\mods{wgradphis} routines are not called.  You'll also have to check
for any other code additions needed that are associated with the
\vars{NO\_ABL} macro.



	\subsection{Miscellaneous Differences}

\begin{itemize}
\item In Python \class{Qtcm} instances,
	\vars{dateofmodel} is set to 0 by default.  
	In contrast, in the compiled QTCM1 model,
	the default (i.e., initial value) is calculated from 
	\vars{day0}, \vars{month0}, and \vars{year0}.
	See Section~\ref{sec:init.compiledform.full} for details.

\item The \class{Qtcm} instance attribute
	\vars{\_\_qtcm} is not copyable using \mods{copy.deepcopy}.

\item In general, when executing a \class{Qtcm} instance method, 
	if you change a \class{Qtcm} instance attribute 
	that has a counterpart in the compiled QTCM1 model,
	the compiled QTCM1 counterpart is not changed until the
	end of the method.  Likewise, if you call a compiled QTCM1 model
	subroutine and change a compiled QTCM1 model variable with
	a \class{Qtcm} instance counterpart, the \class{Qtcm}
	instance counterpart is not changed until the end of the
	subroutine.

\item In general, even though some of the compiled QTCM1 model
	Fortran subroutines/functions have counterparts in \class{Qtcm}
	that duplicate the former's functionality, the Fortran
	versions are kept intact so that the
	\vars{compiled\_form\thinspace=\thinspace'full'} case will work.
\end{itemize}




%---------------------------------------------------------------------
\section{Considerations When Adding Fortran Code}

In this section I describe issues to consider if you wish to add
your own compiled code to the package as separate extension modules.
(This is different from creating new standard extension modules,
which is described in Section~\ref{sec:create.new.so}.):

\begin{itemize}
\item The \class{Qtcm} class assumes that the directory path 
	to the original shared object file is the same as for the 
	\mods{package\_version} module.

\item If you want to be able to pass other Fortran variables 
	in and out to/from Python, please see the 
	Section~\ref{sec:setbypy}
	discussion of the Fotran \mods{SetByPy} module.

\item Fortran and Python routines to get and set compiled QTCM1 model
	arrays are currently written only for floating point array.

\item If you ever change 
	\class{Qtcm} instance method
	\mods{\_set\_qtcm\_array\_item\_in\_model}
	to work with non-floating point values, you will also
	have to change the array handling section in 
	\mods{set\_qtcm1\_item}.

\item The restart mechanism in the pure-Fortran QTCM1 model is 
	\emph{not} bit-for-bit correct.  Thus, if you compare the final
	output from a 40 day run with a 30 day run restarted from
	a 10 day run, the output will not be the same.
	This behavior has been duplicated in \class{Qtcm} 
	instances when the \vars{mrestart} flag is used
	and applicable.

\item When creating new extension modules using the \fn{src} makefile,
	be sure you first use the \cmd{make clean} command to clean-up
	any old files.

\end{itemize}




%---------------------------------------------------------------------
\section{Creating New Standard Extension Modules}   \label{sec:create.new.so}

The steps involved in creating the standard extension modules (e.g.,
\fn{\_qtcm\_full\_365.so}, etc.) on installation are given in
Section~\ref{sec:create.so}.  The makefile provided in \fn{/buildpath/src}
uses a Fortran compiler to create the object code, runs \mods{f2py}
to create the shared object file in \fn{src}, and moves the shared
object files into \fn{../lib}, overwriting any pre-existing files
of the same name.  In this section, I describe the makefile and
\mods{f2py} in a little more detail, in case you wish to create
standard extension modules with additions from the ones the default
makefile creates.


	\subsection{Makefile Rules}    \label{sec:makefile.rules}

This section describes the rules of the
makefile found in the \fn{src} directory
of the \mods{qtcm} distribution.  
This makefile is used by the Python package to create the extension
module (\fn{.so} files) imported and used by \mods{qtcm} objects
(as described in Section~\ref{sec:create.so}).
The makefile will, in general, be used only during \mods{qtcm}
installation, but if you wish to recompile the QTCM1 libraries
and make changes in the Python extension module,
you'll want to use/change this makefile.

\begin{description}
\item[clean] Removes old files in preparation for compiling new
	extension modules.

\item[libqtcm.a] Creates library \fn{libqtcm.a} that contains all
	QTCM1 object files in the directory \fn{src},, except
	\fn{setbypy.o}, \fn{wrapcall.o}, \fn{varptrinit.o}, and
	\fn{driver.o}.  This archive is compiled with the netCDF
	libraries.  Previous versions of \fn{libqtcm.a} are overwritten.

\item[\_qtcm\_full\_365.so] Creates the extension module
	\fn{\_qtcm\_full\_365.so}.  \mods{f2py} is run on applicable code
	in \fn{src}, and the extension module is moved to \fn{../lib}.
	Any previous versions of \fn{../lib/\_qtcm\_full\_365.so}
	are overwritten.

\item[\_qtcm\_parts\_365.so] Creates the extension module
	\fn{\_qtcm\_parts\_365.so}.  \mods{f2py} is run on applicable code
	in \fn{src}, and the extension module is moved to \fn{../lib}.
	Any previous versions of \fn{../lib/\_qtcm\_parts\_365.so}
	are overwritten.

\end{description}



	\subsection{Using \mods{f2py}}      \label{sec:using.f2py}

This section briefly describes how \mods{f2py} is used in the
makefile during the creation of the extension modules.
\htmladdnormallink{\mods{F2py}}{http://cens.ioc.ee/projects/f2py2e/} is a
program that generates shared object libraries that allow you to call
Fortran routines in Python.  \mods{F2py} comes with Python's
\htmladdnormallink{NumPy}{http://numpy.scipy.org/}
array handling package, so you do not need to install anything
extra if you have NumPy already installed.

To create the extension modules in \mods{qtcm} using
the makefile described in Section~\ref{sec:makefile.rules},
I use a method similar to the
\latexhtml{``Quick and Smart Way,''\footnote%
{http://cens.ioc.ee/projects/f2py2e/usersguide/index.html\#the-quick-and-smart-way}}%
{\htmladdnormallink{``Quick and Smart Way''}%
{http://cens.ioc.ee/projects/f2py2e/usersguide/index.html#the-quick-and-smart-way}}
described in the \mods{f2py} manual.
For the \fn{\_qtcm\_full\_365.so} extension module, the 
\mods{f2py} call is:

\begin{codeblock}
\codeblockfont{%
f2py --fcompiler=\$(FC) -c -m \_qtcm\_full\_365 driver.F90 $\backslash$ \\
\hspace*{10ex}setbypy.F90 libqtcm.a \$(NCLIB)}
\end{codeblock}

and for the \fn{\_qtcm\_parts\_365.so} extension module, the call is:

\begin{codeblock}
\codeblockfont{%
f2py --fcompiler=\$(FC) -c -m \_qtcm\_parts\_365 $\backslash$ \\
\hspace*{10ex}varptrinit.F90 wrapcall.F90 setbypy.F90 $\backslash$ \\
\hspace*{10ex}libqtcm.a \$(NCLIB)}
\end{codeblock}

For both calls, \vars{FC} and \vars{NCLIB} are the environment
variables in the makefile specifying the Fortran compiler and netCDF
libraries, respectively.  The \vars{-m} flag specifies the extension
module name (without the \fn{.so} suffix).  The \fn{.F90} files
specify the files that have modules and routines that will be
accessible at the extension module level, and the rest of the Fortran
files in QTCM1 are compiled and archived in a library \fn{libqtcm.a}.
For \mods{f2py} to work properly,
the \fn{.F90} files may define \emph{only one} module or routine.

If you add Fortran files containing new modules, and you wish those
modules to be accessible at the Python level, compile your new code
with \mods{f2py}.  If we have a file of such new code, \fn{newcode.F90},
the \mods{f2py} call to create the \fn{\_qtcm\_parts\_365.so}
extension module will become:

\begin{codeblock}
\codeblockfont{%
f2py --fcompiler=\$(FC) -c -m \_qtcm\_parts\_365 $\backslash$ \\
\hspace*{10ex}varptrinit.F90 wrapcall.F90 setbypy.F90 $\backslash$ \\
\hspace*{10ex}newcode.F90 $\backslash$ \\
\hspace*{10ex}libqtcm.a \$(NCLIB)}
\end{codeblock}

If you write new Fortran code for the compiled QTCM1 model that
will \emph{not} be accessed from the Python-level, just add the
object code filename to the variable \vars{QTCMOBJS} in the
makefile; you don't have to do anything else.  If you are adding
Fortran code to existing Fortran modules, it's even easier:  You
don't need change the makefile.  Note that for 64 bit processor
machines, you may have to use \mods{f2py} with the \cmd{-fPIC} flag;
see Section~\ref{sec:sopic} for details on how the lines above will
change.


	\subsection{Two Examples}

\emphpara{A Function:}
Let's say you have written a piece of Fortran code called
\fn{myfunction.F90} that contains one function called
\mods{myfunction}, and you want to have this function
callable from the Python level through the \class{Qtcm} 
instance method \mods{\_\_qtcm.myfunction}.  Do the following:

\begin{enumerate}
\item Move \fn{myfunction.F90} to \fn{src} in the \mods{qtcm}
	distribution directory \fn{/buildpath}.

\item Add \cmd{myfunction.o} to the end of the object file list lines
	after the target names
	\vars{\_qtcm\_full\_365.so} and
	\vars{\_qtcm\_parts\_365.so}.

\item In the
	\vars{\_qtcm\_full\_365.so} and
	\vars{\_qtcm\_parts\_365.so} target descriptions,
	add \cmd{myfunction.F90} to the 
	beginning of the list of \fn{.F90} names 
	in the \mods{f2py} lines.
\end{enumerate}


\emphpara{A Module:} 
Let's say you have written a piece of Fortran code called
\fn{mymodule.F90} that contains the Fortran module \mods{MyModule}
containing multiple routines and variables.  You want to have those
routines and variables callable from the Python level through the
\class{Qtcm} instance attribute \mods{\_\_qtcm.mymodule}.  The steps
to add \mods{MyModule} to the extension modules are exactly the
same as for a single function, with \cmd{mymodule} being
substituted in the makefile everywhere you have \cmd{myfunction}.




%---------------------------------------------------------------------
\section{Attributes and Methods in \class{Qtcm} Instances}

In this section I describe some attributes, particularly private ones,
that may be of interest to developers.
As is the convention in Python, private
attributes and methods are prepended by one or two underscores,
with two underscores being the ``more'' private attribute.
Please see the package
\latexhtml{API documentation%
		\footnote{http://www.johnny-lin.com/py\_pkgs/qtcm/doc/html-api/}}
        {\htmladdnormallink{API documentation}%
		{http://www.johnny-lin.com/py\_pkgs/qtcm/doc/html-api/}}
for details about all variables, including private variables.


	\subsection{Public \mods{num\_settings} Submodule Attributes/Methods}

\begin{itemize}
\item \vars{typecode}:  This module function returns the
	type code of the data array passed in as its argument.

\item \vars{typecodes}:  This dictionary is the same as the
	NumPy (or Numeric and \mods{numarray})
	dictionary \vars{typecodes}, except that the character
	\vars{'S'} and \vars{'c'} are added to the
	\vars{typecodes['Character']} entry, if absent.  This
	functionality is added because I found 
	\vars{typecodes['Character']} had different values in
	Mac OS X and Ubuntu GNU/Linux.
\end{itemize}


	\subsection{Private \mods{qtcm} Submodule Attributes}

This submodule of the package \mods{qtcm} is the module that defines
the \class{Qtcm} class.

\begin{itemize}
\item \vars{\_init\_prog\_dict}:  This dictionary contains
	the default values of all prognostic variables and 
	right-hand sides that can be initialized.  In the
	submodule \mods{qtcm}, it is set to
	the \vars{init\_prognostic\_dict} module variable in
	submodule \mods{defaults}.

\item \vars{\_init\_vars\_keys}:  List of all keys in
	\vars{\_init\_prog\_dict}, plus \vars{'dateofmodel'}
	and \vars{'title'}.  These names correspond to the
	field variables that are usually written out into a
	restart file.

\item \vars{\_test\_field}:  \class{Field} object instance used 
	in type tests.
\end{itemize}



	\subsection{Private \class{Qtcm} Attributes}  
					\label{sec:Qtcm.private.attrib}

\begin{itemize}
\item \vars{\_cont}:  A boolean attribute that is \vars{True}
	if the run session is a continuation run session and
	\vars{False} if not.  Set the value passed in by
	the keyword \vars{cont} when the \mods{run\_session}
	method is executed.

\item \vars{\_monlen}:  Integer array of the number of days in 
	each month, assuming a 365~day year.

\item \vars{\_\_qtcm}:  The extension module that is the
	compiled QTCM1 Fortran model for this instance.
	This attribute is unique for every instance:  The
	extension module \fn{.so} file is first copied to
	a temporary directory (given by the \vars{sodir}
	instance attribute) and then imported to the
	\class{Qtcm} instance.
	This private attribute is set on instantiation.

\item \vars{\_qtcm\_fields\_ids}:  Field ids for all default 
	field variables, set on instantiation.

\item \vars{\_runlists\_long\_names}:  Dictionary holding the
	descriptions of the standard run lists.  The keys of
	the dictionary are the names of the standard run lists.
\end{itemize}




%---------------------------------------------------------------------
\section{Creating Documentation}

The distribution of \mods{qtcm} comes with the full set of
documentation in readable form (PDF and HTML).  The documentation
consists of two kinds:  this User's Guide and the API documentation.
The User's Guide is written in \LaTeX.  The PDF version is generated
directly from \LaTeX, and the HTML version is created by
\LaTeX{2}HTML.

I use the \fn{make\_docs} shell script in \fn{doc} creates all these
documents.  Briefly, that script does the following:

\begin{itemize}
\item In the \fn{doc/latex} directory, uses \cmd{python} to
	run \fn{code\_to\_latex.py}, which generates the
	\LaTeX\ files describing the current \mods{qtcm} 
	package settings, including text in the manual which gives
	all uses of the current version number.

\item \LaTeX\ is run on the \LaTeX\ files in the \fn{doc/latex} directory.
	The PDF generated by the run is moved from \fn{doc/latex} to
	\fn{doc}.

\item \LaTeX{2}HTML is run on the \LaTeX\ files in \fn{doc/latex}.
	The HTML files generated by the run are moved to \fn{doc/html}.

\item \mods{epydoc} is run on the \mods{qtcm} package libraries.
	This is run in \fn{doc}, to make use of the \fn{epydoc}
	configuration file present there.  The syntax from the
	command line is:

\begin{codeblock}
\codeblockfont{%
epydoc -v --config epydocrc [name]}
\end{codeblock}
\vars{[name]} is either \cmd{qtcm}, if the \mods{qtcm} package is
installed in a directory listed in \vars{sys.path}, or 
\vars{[name]} is the name of the directory the \mods{qtcm} package is
located in (e.g., \fn{/usr/lib/python2.4/site-packages/qtcm}).

\end{itemize}

The \fn{make\_docs} script cannot be used without customizing it
to your system, so please \emphpara{DO NOT USE IT} if you do
not know what you are doing.  You could easily wipe out all your
documentation by mistake.





% ===== end of file =====


\chapter{Future Work}                       \label{ch:future}
% ==========================================================================
% Future
%
% By Johnny Lin
% ==========================================================================


% ------ BODY -----
%
This section describes the features and fixes I plan to work on
in this package.  The most urgent items are listed closer to the
begining of the lists.

\begin{itemize}
\item Add \code{implicit none} top setbypy.F90.

\item Check through Fortran routines that have arguments, to make sure
	f2py is properly understanding the intentions
	(i.e., in, out, inout) of the variables, since we're using the
	``quick way'' of making shared object libraries using f2py.
	The \fn{utilities.F90} file has a number of Fortran routines
	with arguments.

\item Cite:  Peterson, P. (2009) 
	F2PY: a tool for connecting Fortran and Python programs, 
	\emph{Int. J. Computational Science and Engineering,}
	Vol.\ 4, No.\ 4, pp.\ 296--305 for f2py.

\item Create a method like \mods{calc\_derived('T100')} which would
	primarily operate on a data file and provide a derived variable
	such as the temperature at 100 hPa, as given in this example.
	Figure out where to put the parameters (V1s, etc.) that are
	needed to make such a calculation.  As attributes?  Create a
	method to write the quantity out to an output file?
	Perhaps make an ability to calculate these values at heights
	at a given time each day during a run session?

\item Automate the installation using Python's
\htmladdnormallinkfoot{\mods{distutils}}{http://docs.python.org/dist/dist.html}
	utilities.

\item Describe a way of using job control (either via the operating system
	or IPython's \mods{jobctrl} module) 
	to do a quick-and-dirty parallelization of multiple
	\class{Qtcm} instance run sessions.  Or use some sort of threading
	to fire up two simulataneously running models.  Check that the
	simultaneously running models have different memory space.

\item Add capability for \fn{create\_benchmark.py} to overwrite
	existing benchmark files.

\item Make \vars{compiled\_form} set to \vars{'parts'} as the
	default instantiation.  Change documentation accordingly.

\item Currently, the \class{Qtcm} \mods{plotm} method works only on
	3-D output (time, latitude, longitude).  Some of the fields
	in the netCDF output files are 2-D.  Add the capability to
	\mods{plot\_netcdf\_output} in the \mods{plot} submodule
	to handle 2-D fields.

\item Add documentation about removing temporary files.
	Add documentation in Section~\ref{sec:model.instances}
	of details of what occurs during instantiation of 
	a \class{Qtcm} instance.

\item Add the units and long names for all field variables in the
	\mods{defaults} module.

\item Create a keyword to automatically change precipitation and
	evaporation units to mm/day (or similar).

\item Add ability to calculate and plot fields at different pressure
	levels.  Create another module like defaults that specifies
	the vertical fields and gives the equation to use to calculate
	those fields; call the module ``derivfields'' or something
	similar.

\item Throughout the \mods{qtcm} package I use the condition
	\mods{N.rank(}\dumarg{arg}\mods{)\thinspace=\thinspace0} 
	to test whether
	\dumarg{arg} is a scalar.  This works fine for \mods{numpy}
	objects, but it does not work properly for
	\mods{Numeric} and \mods{numarray} arrays.  In those
	array packages, \mods{rank('abc')} returns the value~1.
	This is not a problem, as long as everyone has \mods{numpy},
	but in order to make the package interoperable, I need to
	find a better way of testing for scalars.  The definitions
	of isscalar need to be changed in \mods{num\_settings}.

\item \mods{num\_settings} needs to be changed to truly enable me
	to test whether \mods{qtcm} works for 
	\mods{numarray} and \mods{Numeric} arrays.  The tests
	do not do this right now, because \mods{num\_settings}
	defaults to \mods{numpy}, if it exists.

\item Create makefiles for other platforms.
 
\item A few fields (e.g., \vars{u1}) have data for extra latitude bands,
	due to the use of ``ghost latitudes'' as part of the
	implementation of the numerics.  Details are found in the 
\latexhtml{%
\htmladdnormallinkfoot{QTCM1 manual}%
        {http://www.atmos.ucla.edu/$\sim$csi/qtcm\_man/v2.3/qtcm\_manv2.3.pdf}}%
{\htmladdnormallink{QTCM1 manual}%
        {http://www.atmos.ucla.edu/~csi/qtcm_man/v2.3/qtcm_manv2.3.pdf}}
\cite{Neelin/etal:2002}.

	Though adjusting to this idiosyncracy is not that difficult, 
	in the future I hope to implement a method of handing
	fields with ghost latitudes so that they have the same
	dimensions as the other gridded output variables.  In order
	to do this, I plan to write a Python method to read the
	Fortran generated binary restart file.

\item Change the \mods{set\_qtcm\_item} method so that it can 
	automatically accomodate setting Fortran real variables
	if integer values are input.

\item Currently, the \mods{get\_item\_qtcm} and 
	\mods{set\_item\_qtcm} methods will not work
	on integer and character arrays, only scalars and real arrays.
	Add that missing functionality to those methods.

\item Currently, the \mods{make\_snapshot} method duplicates the
	functionality of the pure-Fortran QTCM1 restart file mechanism.
	However, the restart file mechanism itself does not do a true
	restart.  A continuous run does not provide the same results
	as two runs over the same period, joined by the restart file.

	To see whether saving more variables would do the trick,
	I altered \mods{make\_snapshot} to store all Python level
	variables (i.e., \vars{self.\_qtcm\_fields\_ids}).  However,
	the restart failing described above still continued.  In the
	future, I hope to figure out exactly how many variables are
	needed in order to make the restart feature do a true
	restart.

\item Add a test of using the \vars{mrestart\thinspace=\thinspace1}
	restart option.  Does the \fn{qtcm.restart} file need to be
	in the current working directory or another?

\item Add a test in the unit test scripts to
	confirm that the \vars{init\_with\_instance\_state}
	attribute setting only has an effect if 
	\vars{compiled\_form\thinspace=\thinspace'parts'}.

\item Document \vars{tmppreview} keyword in \mods{plot.plot\_ncdf\_output}.

\item Confirm and document that
	for netCDF output, time is model time since dd-mm-yyyy.

\item Add to the \mods{plotm} method the ability to
	plot as text onto the figure the
	runname string and the calling line
	for the plotm method.

\item Couple with the
	\latexhtml{CliMT\footnote{http://maths.ucd.ie/$\sim$rca/climt/}}%
	{\htmladdnormallink{CliMT}{http://maths.ucd.ie/~rca/climt/}}
	climate modeling toolkit.

\item Enable Python to set \vars{arr1name}, etc., which are string
	variables at the Python level.  I haven't really thought through
	how \vars{arr1} variables work with the Python \class{Qtcm}
	instance.

\item Possible:  In the \class{Qtcm} method
	\mods{\_\_setattr\_\_}, add a test to raise an exception
	if the instance tries to set \vars{viscU}, \vars{viscT},
	or \vars{viscQ} as attributes.  Also create a method
	\code{isotropic\_visc} that will set all viscosity parameters
	non-dependent on direction.  See Section~\ref{sec:driverinit.diffs}
	for details.

\item Go through the manual and create HTML-only versions of tables
	that have table numbers (use a similar construct as in
	figure environments).

\item Go through documentation to check that
	output variable names are capitalized consistently.

\item Create way to redirect stdout.

\item Create a step method to run an arbitrary number of timesteps at
	the atmosphere level.

\end{itemize}


% ===== end of file =====





% ----- BACK MATTER OF THE DOCUMENT -----
%
\normalsize
\pagebreak
\bibliographystyle{plain}
\bibliography{/Users/jlin/work/res/bib/master}

%- Uncomment the input line below and comment out the \bibliographystyle
%  and \bibliography lines if you're running this without the master.bib 
%  BibTeX database
%% ==========================================================================
% Manual for QTCM Python Package
%
% Usage:
% - If you are running this on your own system, you will not have a copy of
%   my master.bib BibTeX database.  To run this, you'll have to comment out:
%
%      \bibliographystyle{chicago-jl}
%      \bibliography{/Users/jlin/work/res/bib/master}
%
%   and comment back in:
%
%      \input{manual.bbl}
%
%   in this file.  Then you can use pdflatex on this file to get the PDF of
%   the manual.  These 3 lines are in the back matter of the document.
%
% Revision Notes:
% - By Johnny Lin, North Park University, http://www.johnny-lin.com/
% - The chicago BibTeX style is unrecognized by latex2html, so I use
%   the plain style.
% ==========================================================================


% ------ DOCUMENT DEFINITIONS ------
%
\documentclass[12pt]{book}
\usepackage{color}
\usepackage{html}
\usepackage{graphicx}
\usepackage{textcomp}
%\usepackage{comment}    %- Unrecognized by latex2html; its use causes errors
%\usepackage{fancyvrb}   %- Unrecognized by latex2html; its use causes errors


%- Packages unrecognized by latex2html, but causes no error:
%
%\usepackage[letterpaper,margin=1in,includefoot]{geometry}
\usepackage[letterpaper,margin=1.25in]{geometry}
\usepackage{bibnames}
\usepackage{longtable}
\usepackage{multirow}


%+ Comment out explicity margin settings since use package geometry:
%\setlength{\topmargin}{0in}
%\setlength{\headheight}{0in}
%\setlength{\headsep}{0in}
%\setlength{\oddsidemargin}{0in}
%\setlength{\evensidemargin}{0in}
%\setlength{\textheight}{8.5in}
%\setlength{\textwidth}{6.5in}




% ------ COMMANDS AND LENGTHS ------
%
% --- Define colors:  Have to do this because for some reason LaTeX
%     sometimes looks for "BLUE" instead of "blue" and complains when
%     "BLUE" isn't found.
%
\definecolor{Blue}{rgb}{0,0,1}
\definecolor{BLUE}{rgb}{0,0,1}
\definecolor{green}{rgb}{0,0.6,0}
\definecolor{Green}{rgb}{0,0.6,0}
\definecolor{GREEN}{rgb}{0,0.6,0}


% --- Format code blocks.  Currently set to print out the code in just 
%     typewriter font with no box.  Will work the same for pdflatex 
%     and latex2html:
%
%     codeblock:  Environment for blocks of computer code or internet 
%       addresses.
%     codeblockfont:  Sets font for codeblocks.
%
\newenvironment{codeblock}%
	{\begin{quotation}\begin{minipage}[t]{0.9\textwidth}}%
	{\end{minipage}\end{quotation}}
	%{\begin{flushleft}}%
	%{\end{flushleft}}
\newcommand{\codeblockfont}[1]{\textcolor{blue}{\texttt{#1}}}
%     *** Version that only works for pdflatex that puts a box around 
%         the block and centers it (commented out).  Note that using
%         fancyvrb is the better way of creating such a boxed section
%         of code, but fancyvrb isn't recognized by latex2html:
%\newenvironment{codeblock}%
%	{\begin{center}\begin{tabular}{|c|} \hline \\ }%
%	{\\ \\ \hline \end{tabular}\end{center}}
%\newcommand{\codeblockfont}[1]{\parbox{0.8\textwidth}{\texttt{#1}}}


% --- Text titling/emphasis settings:
%
%     emphpara:  Emphasis for the first phrase or sentence of a 
%         paragraph.
%     booktitle:  Formats book titles.
%     tabletitle:  Title for an item block in the information table.
%     paratitle:  Title for a paragraph in an item block in the
%         information table.
%     emphdate:  Emphasize date in paragraph text.
%
%     cmd:  Commands
%     dumarg:  Dummy arguments
%     codearg:  Same as dumarg.
%     fn:  File and directory names
%     screen:  Screen display
%     vars:  Variable and attribute names
%     mods:  Module, subroutine, and method names
%     class:  Class names
%     code:  Generic code (avoid using this)
%
\newcommand{\emphpara}[1]{\textbf{#1}}
\newcommand{\booktitle}[1]{\textit{#1}}
%\newcommand{\tabletitle}[1]{\textsf{\textbf{#1}}}
\newcommand{\paratitle}[1]{\textit{#1}}
\newcommand{\emphdate}[1]{\textbf{#1}}

\newcommand{\code}[1]{\textcolor{blue}{\texttt{#1}}}
\newcommand{\cmd}[1]{\textcolor{blue}{\texttt{#1}}}
\newcommand{\dumarg}[1]{\textit{#1}}
\newcommand{\codearg}[1]{\textit{#1}}
\newcommand{\fn}[1]{\textsf{\textit{#1}}}
\newcommand{\screen}[1]{\textcolor{green}{\texttt{#1}}}
\newcommand{\vars}[1]{\textcolor{blue}{\texttt{#1}}}
\newcommand{\class}[1]{\textcolor{blue}{\texttt{#1}}}
\newcommand{\mods}[1]{\textcolor{blue}{\texttt{#1}}}


% --- Special table formatting:
%
%     tabletitlewidth:  Width for title field of an item block in the 
%         information table.
%     tablebodywidth:  Width for body field of an item block in the 
%         information table.
%     tabletabulardims:  Dimensions for the information table, used in
%         the tabular command.
%     tableitemlinespace:  Vertical spacing between item blocks in the
%         information table.
%     infotitle and infotext:  Used for two-column sub-information 
%         tables found in the body field of the information table.  
%         These are not global lengths but have values specific to the 
%         local context in which they're used.
%
\newlength{\tabletitlewidth}
\settowidth{\tabletitlewidth}{file and directory names}

\newlength{\tablebodywidth}
\setlength{\tablebodywidth}{0.9\textwidth}
\addtolength{\tablebodywidth}{-4ex}
\addtolength{\tablebodywidth}{-\tabletitlewidth}

\newcommand{\tabletabulardims}%
	{p{\tabletitlewidth}@{\hspace{4ex}}p{\tablebodywidth}}

\newcommand{\tableitemlinespace}{\baselineskip}
\newlength{\infotitle}
\newlength{\infotext}


% --- Lengths for formatting:
%
\newlength{\remainder}        % length to describe the residual of the
                              %   linewidth minus \enumlabel
\newlength{\enumlabel}        % length to describe figure sub-label width
                              %   (e.g. "(a)")


% --- TtH stuff:
%
%\def\tthdump#1{#1}


% --- LaTeX2HTML stuff:
%
%     htmlfigcaption:  Formatting for HTML replacement figure captions.
%
\newcommand{\htmlfigcaption}[1]{\parbox[c]{70ex}{\footnotesize{#1}}}


% --- Some book title abbreviations:
%
%     rute:  Booktitle for Rute User's.
%     linuxnut:  Booktitle for Linux in a Nutshell.
%     pynut:  Booktitle for Python in a Nutshell.
%
\newcommand{\rute}{\booktitle{Rute User's}}
\newcommand{\linuxnut}{\booktitle{Linux in a Nutshell}}
\newcommand{\pynut}{\booktitle{Python in a Nutshell}}


% --- Define special characters ---
%
\newcommand{\aonehat}{\ensuremath{\widehat{a_1}}}
\newcommand{\bonehat}{\ensuremath{\widehat{b_1}}}
\newcommand{\D}{\ensuremath{\mathcal{D}}}
\def\BibTeX{B\kern-.03em i\kern-.03em b\kern-.15em\TeX}




% ------ BEGINNING OF DOCUMENT TEXT ------
%
\begin{document}

    

    
% ------ TITLE AND TOC ------
%
\title{\mods{qtcm} User's Guide}
\author{Johnny Wei-Bing Lin\thanks{Physics Department, North Park University,
	3225 W.\ Foster Ave., Chicago, IL  60625, USA}}
\date{\today}
\maketitle
\tableofcontents




% ------ BODY ------
%
\chapter{Introduction}
\input{intro}

\chapter{Installation and Configuration}    \label{ch:install}
	\section{Summary and Conventions}      \label{sec:install.sum}
	\input{install_sum}
	\section{Fortran Compiler}             \label{sec:fort.compilers}
	\input{install_fort}
	\section{Required Packages}            \label{sec:py.etc.pkgs}
	\input{install_pkgs}
	\section{Compiling Extension Modules}  \label{sec:create.so}
	\input{compile_so}
	\section{Testing the Installation}     \label{sec:test.qtcm}
	\input{test_qtcm}
	\section{Model Performance}
	\input{perform}
	\section{Installing in Mac OS X}       \label{sec:install.macosx}
	\input{qtcm_in_macosx}
	\section{Installing in Ubuntu}         \label{sec:install.ubuntu}
	\input{qtcm_in_ubuntu}

\chapter{Getting Started With \mods{qtcm}}  \label{ch:getting.started}
\input{started}

\chapter{Using \mods{qtcm}}                 \label{ch:using}
\input{using}

%@@@\chapter{Combining \code{qtcm} with \code{CliMT}}
%@@@\input{climt}

\chapter{Troubleshooting}                   \label{ch:trouble}
\input{trouble}

\chapter{Developer Notes}                   \label{ch:devnotes}
\input{devnotes}

\chapter{Future Work}                       \label{ch:future}
\input{future}




% ----- BACK MATTER OF THE DOCUMENT -----
%
\normalsize
\pagebreak
\bibliographystyle{plain}
\bibliography{/Users/jlin/work/res/bib/master}

%- Uncomment the input line below and comment out the \bibliographystyle
%  and \bibliography lines if you're running this without the master.bib 
%  BibTeX database
%\input{manual.bbl}        

\appendix
\chapter{Field Settings in \mods{defaults}}  \label{app:defaults.values}
\input{defaults}




% ------ END OF DOCUMENT TEXT ------
%
\end{document}


% ===== end of file =====
        

\appendix
\chapter{Field Settings in \mods{defaults}}  \label{app:defaults.values}
% ==========================================================================
% Appendix:  Defaults from the submodule defaults
%
% By Johnny Lin
% ==========================================================================


% ------ BODY -----
%
%---------------------------------------------------------------------------
\section{Scalar Field Variables}  \label{sec:defaults.scalar}

This table lists the default settings for scalar \mods{qtcm} fields
as set by the \mods{defaults} submodule.  All fields are of class
\class{Field}.  Numerical values are rounded as per the conventions
of Python's \vars{\%g} format code.
To create a \class{Field} instance whose value is set to the
default, instantiate with the field id as the argument

\input{defaults_scalars}




%---------------------------------------------------------------------------
\section{Array Field Variables}   \label{sec:defaults.array}

This table lists the default settings for array \mods{qtcm} fields
as set by the \mods{defaults} submodule.  All fields are of class
\class{Field}.  Numerical values are rounded as per the conventions
of Python's \vars{\%g} format code.

\input{defaults_arrays}




% ===== end of file =====





% ------ END OF DOCUMENT TEXT ------
%
\end{document}


% ===== end of file =====

%
%   in this file.  Then you can use pdflatex on this file to get the PDF of
%   the manual.  These 3 lines are in the back matter of the document.
%
% Revision Notes:
% - By Johnny Lin, North Park University, http://www.johnny-lin.com/
% - The chicago BibTeX style is unrecognized by latex2html, so I use
%   the plain style.
% ==========================================================================


% ------ DOCUMENT DEFINITIONS ------
%
\documentclass[12pt]{book}
\usepackage{color}
\usepackage{html}
\usepackage{graphicx}
\usepackage{textcomp}
%\usepackage{comment}    %- Unrecognized by latex2html; its use causes errors
%\usepackage{fancyvrb}   %- Unrecognized by latex2html; its use causes errors


%- Packages unrecognized by latex2html, but causes no error:
%
%\usepackage[letterpaper,margin=1in,includefoot]{geometry}
\usepackage[letterpaper,margin=1.25in]{geometry}
\usepackage{bibnames}
\usepackage{longtable}
\usepackage{multirow}


%+ Comment out explicity margin settings since use package geometry:
%\setlength{\topmargin}{0in}
%\setlength{\headheight}{0in}
%\setlength{\headsep}{0in}
%\setlength{\oddsidemargin}{0in}
%\setlength{\evensidemargin}{0in}
%\setlength{\textheight}{8.5in}
%\setlength{\textwidth}{6.5in}




% ------ COMMANDS AND LENGTHS ------
%
% --- Define colors:  Have to do this because for some reason LaTeX
%     sometimes looks for "BLUE" instead of "blue" and complains when
%     "BLUE" isn't found.
%
\definecolor{Blue}{rgb}{0,0,1}
\definecolor{BLUE}{rgb}{0,0,1}
\definecolor{green}{rgb}{0,0.6,0}
\definecolor{Green}{rgb}{0,0.6,0}
\definecolor{GREEN}{rgb}{0,0.6,0}


% --- Format code blocks.  Currently set to print out the code in just 
%     typewriter font with no box.  Will work the same for pdflatex 
%     and latex2html:
%
%     codeblock:  Environment for blocks of computer code or internet 
%       addresses.
%     codeblockfont:  Sets font for codeblocks.
%
\newenvironment{codeblock}%
	{\begin{quotation}\begin{minipage}[t]{0.9\textwidth}}%
	{\end{minipage}\end{quotation}}
	%{\begin{flushleft}}%
	%{\end{flushleft}}
\newcommand{\codeblockfont}[1]{\textcolor{blue}{\texttt{#1}}}
%     *** Version that only works for pdflatex that puts a box around 
%         the block and centers it (commented out).  Note that using
%         fancyvrb is the better way of creating such a boxed section
%         of code, but fancyvrb isn't recognized by latex2html:
%\newenvironment{codeblock}%
%	{\begin{center}\begin{tabular}{|c|} \hline \\ }%
%	{\\ \\ \hline \end{tabular}\end{center}}
%\newcommand{\codeblockfont}[1]{\parbox{0.8\textwidth}{\texttt{#1}}}


% --- Text titling/emphasis settings:
%
%     emphpara:  Emphasis for the first phrase or sentence of a 
%         paragraph.
%     booktitle:  Formats book titles.
%     tabletitle:  Title for an item block in the information table.
%     paratitle:  Title for a paragraph in an item block in the
%         information table.
%     emphdate:  Emphasize date in paragraph text.
%
%     cmd:  Commands
%     dumarg:  Dummy arguments
%     codearg:  Same as dumarg.
%     fn:  File and directory names
%     screen:  Screen display
%     vars:  Variable and attribute names
%     mods:  Module, subroutine, and method names
%     class:  Class names
%     code:  Generic code (avoid using this)
%
\newcommand{\emphpara}[1]{\textbf{#1}}
\newcommand{\booktitle}[1]{\textit{#1}}
%\newcommand{\tabletitle}[1]{\textsf{\textbf{#1}}}
\newcommand{\paratitle}[1]{\textit{#1}}
\newcommand{\emphdate}[1]{\textbf{#1}}

\newcommand{\code}[1]{\textcolor{blue}{\texttt{#1}}}
\newcommand{\cmd}[1]{\textcolor{blue}{\texttt{#1}}}
\newcommand{\dumarg}[1]{\textit{#1}}
\newcommand{\codearg}[1]{\textit{#1}}
\newcommand{\fn}[1]{\textsf{\textit{#1}}}
\newcommand{\screen}[1]{\textcolor{green}{\texttt{#1}}}
\newcommand{\vars}[1]{\textcolor{blue}{\texttt{#1}}}
\newcommand{\class}[1]{\textcolor{blue}{\texttt{#1}}}
\newcommand{\mods}[1]{\textcolor{blue}{\texttt{#1}}}


% --- Special table formatting:
%
%     tabletitlewidth:  Width for title field of an item block in the 
%         information table.
%     tablebodywidth:  Width for body field of an item block in the 
%         information table.
%     tabletabulardims:  Dimensions for the information table, used in
%         the tabular command.
%     tableitemlinespace:  Vertical spacing between item blocks in the
%         information table.
%     infotitle and infotext:  Used for two-column sub-information 
%         tables found in the body field of the information table.  
%         These are not global lengths but have values specific to the 
%         local context in which they're used.
%
\newlength{\tabletitlewidth}
\settowidth{\tabletitlewidth}{file and directory names}

\newlength{\tablebodywidth}
\setlength{\tablebodywidth}{0.9\textwidth}
\addtolength{\tablebodywidth}{-4ex}
\addtolength{\tablebodywidth}{-\tabletitlewidth}

\newcommand{\tabletabulardims}%
	{p{\tabletitlewidth}@{\hspace{4ex}}p{\tablebodywidth}}

\newcommand{\tableitemlinespace}{\baselineskip}
\newlength{\infotitle}
\newlength{\infotext}


% --- Lengths for formatting:
%
\newlength{\remainder}        % length to describe the residual of the
                              %   linewidth minus \enumlabel
\newlength{\enumlabel}        % length to describe figure sub-label width
                              %   (e.g. "(a)")


% --- TtH stuff:
%
%\def\tthdump#1{#1}


% --- LaTeX2HTML stuff:
%
%     htmlfigcaption:  Formatting for HTML replacement figure captions.
%
\newcommand{\htmlfigcaption}[1]{\parbox[c]{70ex}{\footnotesize{#1}}}


% --- Some book title abbreviations:
%
%     rute:  Booktitle for Rute User's.
%     linuxnut:  Booktitle for Linux in a Nutshell.
%     pynut:  Booktitle for Python in a Nutshell.
%
\newcommand{\rute}{\booktitle{Rute User's}}
\newcommand{\linuxnut}{\booktitle{Linux in a Nutshell}}
\newcommand{\pynut}{\booktitle{Python in a Nutshell}}


% --- Define special characters ---
%
\newcommand{\aonehat}{\ensuremath{\widehat{a_1}}}
\newcommand{\bonehat}{\ensuremath{\widehat{b_1}}}
\newcommand{\D}{\ensuremath{\mathcal{D}}}
\def\BibTeX{B\kern-.03em i\kern-.03em b\kern-.15em\TeX}




% ------ BEGINNING OF DOCUMENT TEXT ------
%
\begin{document}

    

    
% ------ TITLE AND TOC ------
%
\title{\mods{qtcm} User's Guide}
\author{Johnny Wei-Bing Lin\thanks{Physics Department, North Park University,
	3225 W.\ Foster Ave., Chicago, IL  60625, USA}}
\date{\today}
\maketitle
\tableofcontents




% ------ BODY ------
%
\chapter{Introduction}
%=====================================================================
% Introduction
%=====================================================================


% ----- BEGIN TEXT -----
%
%---------------------------------------------------------------------
\section{How to Read This Manual}

\emphpara{Most users:} 
Just read 
(1) the installation instructions in Chapter~\ref{ch:install},
(2) Chapter~\ref{ch:getting.started},
which tells you all you need to get started using \mods{qtcm}, and
(3) examples in Section~\ref{sec:cookbook} that give a feel
for how you can use the model.

\emphpara{Users having problems:}
Chapter~\ref{ch:trouble} provides troubleshooting tips for
a few problems.
The detailed description of how the package functions, 
in Chapter~\ref{ch:using}, will probably be more useful.

\emphpara{Developers:}
If you want to change the source code, please read
Chapter~\ref{ch:devnotes}.  Chapter~\ref{ch:future} describes
all the things I'd like to do to improve the package, but haven't
gotten to yet.




%---------------------------------------------------------------------
\section{About the Package}

The single-baroclinic mode
Neelin-Zeng Quasi-Equilibrium Tropical Circulation Model
\latexhtml{(QTCM1)\footnote{http://www.atmos.ucla.edu/$\sim$csi}}%
	{\htmladdnormallink{(QTCM1)}{http://www.atmos.ucla.edu/~csi}}
is a primitive equation-based intermediate-level atmospheric model
that focuses on simulating the tropical atmosphere.  Being more
complicated than a simple model, the model has full non-linearity
with a basic representation of baroclinic instability,
includes a radiative-convective feedback package, and includes a
simple land soil moisture routine (but does not include topography).
A brief, but more detailed, description of QTCM1 is given in
Section~\ref{sec:brief_qtcm}.

\htmladdnormallinkfoot{Python}{http://www.python.org}
is an interpreted, object-oriented, multi-platform,
open-source language that is used in a variety of software applications,
ranging from game development to bioinformatics.
In climate studies, Python has been used as the core language for
data analysis
(e.g., \htmladdnormallinkfoot{Climate Data Analysis Tools}{http://cdat.sf.net}),
visualization
(e.g., \htmladdnormallinkfoot{Matplotlib}{http://matplotlib.sf.net}),
and 
modeling
(e.g., \htmladdnormallinkfoot{PyCCSM}{http://code.google.com/p/pyccsm/}).

In comparison to traditional compiled languages like Fortran,
Python's lack of a separate compile step greatly simplifies the
debugging and testing phases of development, because code snippets
can be testing as code is written.
Python's extensive suite of higher-level tools (e.g., statistics,
visualization, string and file manipulation) accessible at runtime 
enables modeling and analysis to occur concurrently.  

The \mods{qtcm} package is an implementation of the Neelin-Zeng
QTCM1 in a Python object-oriented environment.  The conversion
package
\htmladdnormallinkfoot{\mods{f2py}}{http://cens.ioc.ee/projects/f2py2e/} is
used to wrap the QTCM1 Fortran model routines and manage model
execution using Python datatypes and utilities.  The result is a
modeling package where order and choice of subroutine execution can
be altered at runtime.  Model analysis and visualization can also
be integrated with model execution at runtime.




%---------------------------------------------------------------------
\section{Conventions In This Manual}

	\subsection{Audience}

In this manual I assume you have a rudimentary knowledge of Python.
Thus, I do not describe basic Python data types (e.g., dictionaries,
lists), object framework and syntax (e.g., classes, methods,
attributes, instantiation), module and package importing.  If you
need to brush up (or learn) Python, I'd recommend the following
resources:

\begin{itemize}
\item \htmladdnormallinkfoot{Python Tutorial:}{http://docs.python.org/tut/}
	This tutorial was written by Guido van Rossum, Python's original
	author.

\item \htmladdnormallinkfoot{Instant Hacking:}%
	{http://www.hetland.org/python/instant-hacking.php}
	Learn how to program with Python.

\item \htmladdnormallinkfoot{Dive Into Python:}%
	{http://diveintopython.org/index.html}
	Reasonably complete and cohesive. The entire book is available for 
	free online.

\item \htmladdnormallinkfoot{Handbook of the Physics Computing Course:}%
	{http://www.pentangle.net/python/handbook/}
	Written for a science audience.  Recommended.

\item \latexhtml{CDAT/Python Tips for Earth Scientists:\footnote%
	{http://www.johnny-lin.com/cdat\_tips/}}%
	{\htmladdnormallink{CDAT/Python Tips for Earth Scientists:}%
		{http://www.johnny-lin.com/cdat_tips/}}
	This web site is a FAQ of sorts for people using Python and
	the Climate Data Analysis Tools (CDAT) in the earth sciences,
	and thus focuses on using Python to do science rather than
	the computer science aspects of the language.

\end{itemize}

The purpose of this package is to make the QTCM1 model easier to
use.  In order to interpret the results, however, you still need
to have a robust sense of what climate models can and cannot tell
you.  A starting point for the QTCM1 model is the brief description
of the model in Section~\ref{sec:brief_qtcm}.  After that, I would
read the original papers describing the model formulation and results
\cite{Neelin/Zeng:2000,Zeng/etal:2000}, and 
\latexhtml{papers citing the model formulation work.\footnote%
{http://scholar.google.com/scholar?hl=en\&lr=\&cites=14217886709842286738}}%
{\htmladdnormallink{papers citing the model formulation work}%
{http://scholar.google.com/scholar?hl=en&lr=&cites=14217886709842286738}.}
Being an intermediate-level model using the quasi-equilibrium assumption,
QTCM1 (and thus \mods{qtcm}) has distinct strengths and limitations; 
please be aware of them.


	\subsection{Typographic Conventions}

\begin{center}
\begin{tabular}{\tabletabulardims}
\cmd{commands} & to be typed at the command-line
	are rendered in a 
	blue, serif, fixed-width typewriter font
	(e.g., \cmd{make \_qtcm\_full\_365}). \\ \hline
\dumarg{dummy arguments} &
	coupled with code or screen display is rendered in a 
	serif, proportional, italicized font
	(e.g., \screen{Error-Value too long in} \dumarg{variable}). \\ \hline
\fn{file and directory names} & are rendered in a 
	sans-serif, italicized font
	(e.g., \fn{setbypy.F90}). \\ \hline
\screen{screen display} & is rendered in a 
	green, serif, fixed-width typewriter font. \\ \hline
\mods{module, method, and subroutine names} & are rendered in a 
	blue, serif, fixed-width typewriter font. \\ \hline
\vars{variable and attribute names} & are rendered in a 
	blue, serif, fixed-width typewriter font. \\ \hline
\class{class names} & are rendered in a 
	blue, serif, fixed-width typewriter font.
\end{tabular}
\end{center}

Blocks of code (usually commands, screen display, and module,
variable, and class names) are displayed in a blue, serif, fixed-width
typewriter font.


	\subsection{Terminology}

\begin{description}
\item[attribute and method references:]
	If there is any possibility of confusion, I will give the
	class that an attribute or method comes from when that
	attribute or method is referenced.  If no class is mentioned
	by name or context,
	assume that the attribute/method comes from the
	\class{Qtcm} class.

\item[``compiled QTCM1 model'':]
	This usually is used to denote when I'm talking about
	compiled Fortran QTCM1 routines and variables therein,
	as an extension module to the Python \mods{qtcm} package..
	Thus, ``compiled QTCM1 model \vars{u1}'' is the value
	of variable \vars{u1} in the Fortran model, not the
	value at the Python-level.  Sometimes I refer to the
	compiled QTCM1 model as just ``QTCM1'' or as
	``compiled QTCM1 Fortran model''.

\item[``pure-Fortran QTCM1'':]
	This refers to the Neelin-Zeng QTCM1 model in it's
	original Fortran form, not as an extension module to
	the Python \mods{qtcm} package.

\item[``Python-level'':]
	This usually denotes the value of a variable as an
	attribute of a \class{Qtcm} instance.  This variable
	is stored at the Python interpreter level.

\item[\class{Qtcm}:]
	This refers to the Python \class{Qtcm} class
	(note the capitalized first letter).

\item[\mods{qtcm}:]
	This refers to the Python \mods{qtcm} package.

\item[QTCM1 vs.\ QTCM:]
	Although the QTCM1 is currently the only version of a
	quasi-equilibrium tropical circulation model (QTCM), in
	principle one can construct a QTCM with any number of
	baroclinic modes.  In anticipation of this, the names of
	the Python package and class do not end in a numeral.  In
	this manual and the \mods{qtcm} docstrings, QTCM and QTCM1
	are used interchangably.
	Usually either of these phrases mean the quasi-equilibrium
	tropical circulation model in a generic sense, regardless
	of its form of implementation.
\end{description}




%---------------------------------------------------------------------
\section{Current Version Information and Acknowledgments}  \label{sec:ver}

% This file is automatically generated by
    % code_to_latex.py.

This manual describes version 0.1.2 (dated September 12, 2008), of package \mods{qtcm}.
Johnny Linis the primary author of the package.

The \mods{qtcm} package is built upon the pure-Fortran QTCM1 model,
version 2.3 (August 2002), with a few minor changes.
Those changes are described in detail in
Section~\ref{sec:f90changes}.

The homepage for the \mods{qtcm} package is
\htmladdnormallink{http://www.johnny-lin.com/py\_pkgs/qtcm}%
	{http://www.johnny-lin.com/py_pkgs/qtcm}.
All Python code in this package, 
and the Fortran files \fn{setbypy.F90} and \fn{wrapcall.F90},
are \copyright\ 2003--2008 by 
\htmladdnormallinkfoot{Johnny Lin}%
		{http://www.johnny-lin.com} 
and constitutes a
library that is covered under the GNU Lesser General Public License
(LGPL):

\begin{quotation}
	This library is free software; you can redistribute it
	and/or modify it under the terms of the 
	\htmladdnormallinkfoot{GNU Lesser General Public License}%
		{http://www.gnu.org/copyleft/lesser.html} 
	as published by
	the Free Software Foundation; either version 2.1 of the
	License, or (at your option) any later version.

	This library is distributed in the hope that it will be
	useful, but WITHOUT ANY WARRANTY; without even the implied
	warranty of MERCHANTABILITY or FITNESS FOR A PARTICULAR
	PURPOSE. See the GNU Lesser General Public License for more
	details.

	You should have received a copy of the GNU Lesser General
	Public License along with this library; if not, write to
	the Free Software Foundation, Inc., 59 Temple Place, Suite
	330, Boston, MA 02111-1307 USA.

	You can contact Johnny Lin at his email address 
	or at North Park University, Physics Department,
	3225 W. Foster Ave., Chicago, IL 60625, USA.  
\end{quotation}

All other Fortran code in this package, as well as the makefiles,
are covered by licenses (if any) chosen by their respective owners.

This manual, in all forms (e.g., HTML, PDF, \LaTeX),
is part of the documentation of the \mods{qtcm} package 
and is \copyright\ 2007--2008 by Johnny Lin.
Permission is granted to copy, distribute and/or modify 
this document under the terms of the 
GNU Free Documentation License, Version 1.2 
or any later version published by the Free Software Foundation; 
with no Invariant Sections, no Front-Cover Texts, 
and no Back-Cover Texts. 
A copy of the license can be found 
\htmladdnormallinkfoot{here}{http://www.gnu.org/licenses/fdl.html}.

Transparent copies of this document are located online in
\latexhtml{%
\htmladdnormallinkfoot{PDF}%
	{http://www.johnny-lin.com/py\_pkgs/qtcm/doc/manual.pdf}}%
{\htmladdnormallink{PDF}%
	{http://www.johnny-lin.com/py_pkgs/qtcm/doc/manual.pdf}}
and
\latexhtml{%
\htmladdnormallinkfoot{HTML}%
	{http://www.johnny-lin.com/py\_pkgs/qtcm/doc/}}%
{\htmladdnormallink{HTML}%
	{http://www.johnny-lin.com/py_pkgs/qtcm/doc/}}
formats.
The \LaTeX\ source files are distributed with the \mods{qtcm}
distribution.
While the HTML version is nearly identical to the PDF
and \LaTeX\ versions, not every feature in the manual was successfully
converted.  This is particularly true with figures, which are
unnumbered in the HTML version and may be formatted differently
than the authoritative PDF version.
Thus, please consider the \LaTeX\ version as the authoritative
version.

\vspace{\baselineskip}

\emphpara{Acknowledgements:}
Thanks to David Neelin and Ning Zeng and the Climate Systems
Interactions Group at UCLA for their encouragement and help.
On the Python side,
thanks to Alexis Zubrow, Christian Dieterich, Rodrigo Caballero,
Michael Tobis, and Ray Pierrehumbert for steering me straight.
Early versions of some of this work was carried out 
at the University of Chicago Climate Systems Center, 
funded by the National Science Foundation (NSF) 
Information Technology Research Program under grant ATM-0121028. 
Any opinions, findings and conclusions or recommendations 
expressed in this material are those of the author and 
do not necessarily reflect the views of the NSF.

Intel\textregistered\ and
   Xeon\textregistered\ are registered trademarks of Intel Corporation.
Matlab\textregistered\ is a registered trademark of The MathWorks.
UNIX\textregistered\ is a registered trademark of The Open Group.




%---------------------------------------------------------------------
\section{Summary of Release History}

\begin{itemize}
\item 2008 Sep 12:  Version 0.1.2 released.  Summary of changes:
	\begin{itemize}
	\item Create \class{Qtcm} method \mods{get\_qtcm1\_item}.
		This method is effectively an alias of method 
		\mods{get\_qtcm\_item}.
	\item Create \class{Qtcm} method \mods{set\_qtcm1\_item}.
		This method is effectively an alias of method 
		\mods{set\_qtcm\_item}.
	\item Update User's Guide to phase out references to
		the \mods{get\_qtcm\_item}
		and \mods{set\_qtcm\_item} methods.  
		Adding the ``1'' to the method names makes the purpose
		of the methods clearer.
	\item Add unit tests to cover methods \mods{get\_qtcm1\_item} and
		\mods{set\_qtcm1\_item}.
	\end{itemize}

\item 2008 Jul 30:  Updates to the User's Guide (just the online versions,
        not the copies released with the tarball).

\item 2008 Jul 15:  First publicly available distribution 
	released (v0.1.1).
\end{itemize}




%---------------------------------------------------------------------
\section{A Brief Description of The QTCM1}   \label{sec:brief_qtcm}

This description is copied from Ch.\ 3 of Lin \cite{Lin:2000}, 
with minor revisions.
Model formulation is fully described in
Neelin \& Zeng \cite{Neelin/Zeng:2000} and model
results are described in Zeng et~al.\ \cite{Zeng/etal:2000}.
Neelin \& Zeng \cite{Neelin/Zeng:2000} is based upon v2.0 of QTCM1
and Zeng et~al.\ \cite{Zeng/etal:2000} is based on QTCM1 v2.1.
The 
\latexhtml{%
\htmladdnormallinkfoot{QTCM1 manual}%
	{http://www.atmos.ucla.edu/$\sim$csi/qtcm\_man/v2.3/qtcm\_manv2.3.pdf}}%
{\htmladdnormallink{QTCM1 manual}%
	{http://www.atmos.ucla.edu/~csi/qtcm_man/v2.3/qtcm_manv2.3.pdf}}
\cite{Neelin/etal:2002}
describes the details of model implementation.

QTCM1 differs from most full-scale GCMs primarily in how the vertical
temperature, humidity, and velocity structure of the atmosphere is
represented.  First, instead of representing the vertical structure
by finite-differenced levels, the model uses a Galerkin expansion
in the vertical.  The vertical basis functions are chosen according
to analytical solutions under convective quasi-equilibrium conditions,
so only a few need be retained.
Temperature and humidity are each described by separate
vertical basis functions ($a_1$ and $b_1$, respectively).
Low-level variations in the humidity basis
are larger than in the temperature basis.
For velocity, QTCM1 uses a single baroclinic basis function ($V_1$)
defined consistently with the temperature basis function,
as well as a barotropic velocity mode ($V_0$).
The vertical profiles of $a_1$, $b_1$, and $V_1$
are given in Figure~\ref{fig:qtcm.basis}.
Currently, QTCM1 does not include a separate
vertical degree of freedom describing the PBL.
The horizontal grid spacing of the model is 
$5.625^{\circ}$ longitude by $3.75^{\circ}$ latitude.


% <QTCM1 beta version vertical structure modes>
%
% (1) LaTeX version:
%
\begin{latexonly}
\begin{figure}
   \noindent
   \begin{minipage}[b]{.49\linewidth}
      \settowidth{\enumlabel}{(a) }%
      \setlength{\remainder}{\linewidth}% 
      \addtolength{\remainder}{-\enumlabel}
      {(a)}~\makebox[\remainder]{$a_1$ and $b_1$}
      \centering\includegraphics[width=\linewidth,viewport=58 72 389 344,clip]%
                    {figs/a1b1.pdf}
   \end{minipage}\hfill
   \begin{minipage}[b]{.49\linewidth}
      \settowidth{\enumlabel}{(b) }%
      \setlength{\remainder}{\linewidth}% 
      \addtolength{\remainder}{-\enumlabel}
      {(b)}~\makebox[\remainder]{$V_1$}

      \centering\includegraphics[width=\linewidth,viewport=58 72 389 346,clip]%
                    {figs/V1.pdf}
   \end{minipage}

   \caption{Vertical profiles of basis functions for
		(a) temperature $a_1$ (solid) and humidity $b_1$ (dashed) and
		(b) baroclinic component of
		horizontal velocity $V_1$.}
   \label{fig:qtcm.basis}
\end{figure}
\end{latexonly}

% (2) HTML replacement version:
%
\begin{htmlonly}
\label{fig:qtcm.basis}
\begin{center}
\htmladdimg{../latex/figs/a1b1.png}
\htmladdimg{../latex/figs/V1.png}

\htmlfigcaption{Figure \ref{fig:qtcm.basis}:  
	Vertical profiles of basis functions for
   	(a) temperature $a_1$ (solid) and humidity $b_1$ (dashed) and
   	(b) baroclinic component of
   	horizontal velocity $V_1$.}
\end{center}
\end{htmlonly}


These modes are chosen to accurately capture deep convective regions.
Outside deep convective regions the mode
is simply a highly truncated
Galerkin representation.  The system is much more tightly constrained than
a full-scale GCM, yet hopefully retains the essential dynamics and nonlinear
feedbacks.  The result is that QTCM1 is easier to diagnose than a GCM,
and is computationally fast (about 8 minutes per year on a Sun Ultra 2
workstation).  Zeng et al.\ \cite{Zeng/etal:2000} show results indicating
this intermediate-level model does a reasonable job simulating
tropical climatology and ENSO variability.  


Below is a summary of the main model equations \cite{Neelin/Zeng:2000}:
\begin{equation}
   \partial_t \mathbf{v}_1 
      + \D_{V1} (\mathbf{v}_0 , \mathbf{v}_1)
      + f \mathbf{k} \times \mathbf{v}_1
      =
   - \kappa \nabla T_1 
      - \epsilon_1 \mathbf{v}_1 
      - \epsilon_{01} \mathbf{v}_0
   \label{eqn:barocl_wind}
\end{equation}
\begin{equation}
   \partial_t \zeta_0 
      + \mathrm{curl}_z (\D_{V0} (\mathbf{v}_0 , \mathbf{v}_1))
      + \beta v_0
      =
   - \mathrm{curl}_z (\epsilon_0 \mathbf{v}_0)
      - \mathrm{curl}_z (\epsilon_{10} \mathbf{v}_1)
   \label{eqn:barotr_wind}
\end{equation}
\begin{equation}
   \aonehat (\partial_t + \D_{T1}) T_1 
      + M_{S1} \nabla \cdot {\bf v}_1 
      =
   \langle Q_c \rangle
      + (g/p_T) (-R^\uparrow_t -R^\downarrow_s + R^\uparrow_s + S_t - S_s + H)
   \label{eqn:temperature}
\end{equation}
\begin{equation}
   \bonehat (\partial_t + \D_{q1}) q_1 
      - M_{q1} \nabla \cdot {\bf v}_1 
      =
   \langle Q_q \rangle
      + (g/p_T) E
   \label{eqn:moisture}
\end{equation}
where (\ref{eqn:barocl_wind}) describes the baroclinic wind component,
      (\ref{eqn:barotr_wind}) describes the barotropic wind component,
      (\ref{eqn:temperature}) is the temperature equation, and
      (\ref{eqn:moisture}) is the moisture equation.

In the simplest formulation, the vertically integrated
convective heating and moisture sink
are assumed to be equal and opposite, so:
\begin{equation}
  -\langle Q_q \rangle = \langle Q_c \rangle 
                              = \epsilon^\ast_c (q_1 - T_1)
\end{equation}

For its convective parameterization for $Q_c$, this model uses the
Betts-Miller \cite{Betts/Miller:1986} moist convective
adjustment scheme, a scheme that is also used in some GCMs.
In the convective parameterization, the coefficient
$\epsilon^\ast_c$ is defined as:
\begin{equation}
   \epsilon^\ast_c 
      \equiv 
   \aonehat \bonehat (\aonehat + \bonehat)^{-1} \tau_c^{-1} 
      \mathcal{H}( \mathit{C}_{\mathrm{1}} )
\end{equation}
where $\mathcal{H}( \mathit{C}_{\mathrm{1}} )$ is zero for
$C_{1} \leq 0$, and one for $C_{1} > 0$, and $C_{1}$
is a measure of the convective available potential energy (CAPE),
projected onto the model's temperature and moisture basis functions.

Sensible heat ($H$) and evaporation ($E$) are given as
bulk-aerodynamic formulations:
\begin{equation}
   H
      =
   \rho_a C_D \mathrm{V}_s (T_s - (T_{rs} + a_{1s} T_1))
\end{equation}
\begin{equation}
   E
      =
   \rho_a C_D \mathrm{V}_s (q_\mathit{sat} (T_s) 
      - (q_{rs} + b_{1s} q_1))
\end{equation}

Longwave radiation components are denoted by $R$, and net shortwave
radiation is denoted by $S$.
The terms $\D_{V1}$ and $\D_{V0}$ are the advection-diffusion operators
for the momentum equations (projected onto $V_0$ and $V_1 (p)$,
respectively).
The terms $\D_{T1}$ and $\D_{q1}$ are the
advection-diffusion operators for the temperature and moisture
equations, respectively, using a vertical average projection.
The $\langle X \rangle$ and $\widehat{X}$ operators are
equivalent and denote vertically integration over the troposphere.
Please see Neelin \& Zeng \cite{Neelin/Zeng:2000} and 
Zeng et al.\ \cite{Zeng/etal:2000}
for a more complete description of equations and coefficients.







% ====== end file ======


\chapter{Installation and Configuration}    \label{ch:install}
	\section{Summary and Conventions}      \label{sec:install.sum}
	% ==========================================================================
% Installation Summary
%
% By Johnny Lin
% ==========================================================================


% ------ BODY -----
%

This section provides a summary of the steps needed to install
\mods{qtcm}, and a description of the naming conventions used in
this chapter.  If you have had a decent amount of experience with
Python and installing software on a Unix system, this section will
probably be all you need to read.  The installation steps are:

\begin{enumerate}
\item Install a Fortran compiler (see Section~\ref{sec:fort.compilers}
	for a list of compilers known to work).
	This compiler should be in a directory
	listed in your system path (e.g., \fn{/usr/bin}, etc.).

\item Install all required packages
	(see Section~\ref{sec:py.etc.pkgs} for details):
	Python,
	\mods{matplotlib} (plus the \mods{basemap} toolkit),
	NumPy (which includes \mods{f2py}),
	Scientific Python,
	\LaTeX,
	and
	netCDF.

	Python packages are required to be installed on your
	system in a directory listed in your \vars{sys.path},
	and the other packages/libraries are required to be in 
	standard directories listed in your system path 
	(e.g., \fn{/usr/bin}, \fn{/sw/include}, etc.).

	Make sure the executable for Python can be called at the
	Unix command line by typing both \cmd{python}.
	You might need to define a Unix alias
	that maps \cmd{python2.4} (or whichever version of Python
	you are using) to \cmd{python}.

\item \latexhtml{Download\footnote{http://www.johnny-lin.com/py\_pkgs/qtcm/}}%
        {\htmladdnormallink{Download}{http://www.johnny-lin.com/py_pkgs/qtcm/}}
	the \mods{qtcm} tarball and extract the distribution
	into a temporary directory for building purposes.
	\fn{qtcm-0.1.2}is the name of
	the \mods{qtcm} distribution directory;
	the number following the hyphen is the
	version number of the distribution.  \label{list:download.qtcm.sum}

	In this manual, the path to \fn{qtcm-0.1.2}will
	be called the ``\mods{qtcm} build path'' and be given as
	\fn{/buildpath}.  When you see \fn{/buildpath}, please substitute
	the actual temporary directory you created for building purposes.

\item The \mods{qtcm} distribution directory 
	\fn{qtcm-0.1.2}contains the following 
	principal sub-directories:
	\fn{doc}, \fn{lib}, \fn{src}, \fn{test}.
	Documentation is in \fn{doc},
	all the package modules are in \fn{lib},
	building of extension modules will take place in \fn{src},
	and testing of the package is done in \fn{test}.

\item Compile \mods{qtcm} extension modules in \fn{src}:
	Go to \fn{src}, copy the makefile from
	\fn{src/Makefiles} corresponding to your
	system into \fn{src}, rename to \fn{makefile},
	make changes to the makefile as needed,
	and execute:
	\begin{codeblock}
	\codeblockfont{%
	make clean \\
	make \_qtcm\_full\_365.so \\
	make \_qtcm\_parts\_365.so}
	\end{codeblock}
	If you executed the make commands in \fn{src,},
	the extension modules will be automatically placed in
	\fn{lib} in the \fn{qtcm-0.1.2}directory.
	See Section~\ref{sec:create.so} for details.
	\label{list:compile.so.sum}

\item Copy the entire contents of \fn{lib} in
	\fn{qtcm-0.1.2}(not \fn{lib} itself) 
	to a directory named
	\fn{qtcm} that is on your \mods{sys.path}.  For instance,
	for Mac OS X using Fink,
	many Python packages are located in a directory
	named \fn{/sw/\-lib/\-python2.4/\-site-packages}, or something
	similar, and this directory is on the system \mods{sys.path}.  
	If this is the case for your system, copy the
	contents of \fn{lib} into
	\fn{/sw/lib/\-python2.4/\-site-packages/\-qtcm}.
	(For Unix systems, the equivalent directory is usually
	\fn{/usr/\-local/\-lib/\-python2.4/\-site-packages}.)

\item Test the \mods{qtcm} distribution in \fn{test}:
	This step is optional and can take a while.
	Testing requires you to first generate a suite of benchmarks
	using the pure-Fortran QTCM1 model, then running the tests of
	\mods{qtcm} by typing:
	\begin{codeblock}
	\codeblockfont{%
python test\_all.py}
	\end{codeblock}
	at the Unix command line while in \fn{test}.
	See Section~\ref{sec:test.qtcm} for details.

\end{enumerate}

At some point, I will automate the installation using Python's
\htmladdnormallinkfoot{\mods{distutils}}{http://docs.python.org/dist/dist.html}
utilities.



% ===== end of file =====

	\section{Fortran Compiler}             \label{sec:fort.compilers}
	% ==========================================================================
% Fortran compilers
%
% By Johnny Lin
% ==========================================================================


% ------ BODY -----
%

You must have a Fortran compiler installed on your system in order
to compile \mods{qtcm}.  The compiler must be able to interface with
a pre-processor, as QTCM1 makes copious use of pre-processor directives.
\mods{qtcm} is known to work with the following Fortran compilers on the
following platforms:

\begin{center}
\begin{tabular}{l|l|l}
\textbf{Compiler}  & \textbf{Compiler Web Site} & \textbf{Platform(s)} \\ 
\hline
\mods{g95} & \htmladdnormallink{http://www.g95.org/}{http://www.g95.org/}  
	& Mac OS X \\
\end{tabular}
\end{center}

It will probably work with other platforms, but I haven't been able
to test platforms besides those listed above.  Note that \mods{g95}
is not \htmladdnormallink{GNU Fortran}{http://gcc.gnu.org/fortran/}
(\mods{gfortran}), the Fortran 95 compiler included with the more
recent versions of GCC.




% ===== end of file =====

	\section{Required Packages}            \label{sec:py.etc.pkgs}
	% ==========================================================================
% Python packages
%
% By Johnny Lin
% ==========================================================================


% ------ BODY -----
%

The following Python packages are required to be installed on your
system in a directory listed in your \vars{sys.path}:
\begin{itemize}
\item \htmladdnormallinkfoot{Python}%
	{http://www.python.org/}:  The Python programming language
	and interpreter.  Make sure you have a version recent enough
	to be compatible with all the needed Python packages.
\item \htmladdnormallinkfoot{\mods{matplotlib}}%
	{http://matplotlib.sourceforge.net/}:  Scientific plotting
	package, using Matlab-like syntax.  The \mods{basemap} toolkit
	for \mods{matplotlib} must also be installed.
\item \htmladdnormallinkfoot{NumPy}%
	{http://numpy.scipy.org/}:  The standard array package for
	Python.  The module name of NumPy imported in a Python 
	session is \mods{numpy}.
\item \htmladdnormallinkfoot{Scientific Python}%
	{http://dirac.cnrs-orleans.fr/plone/software/scientificpython/}:
	Has netCDF file operators, in addition to other routines
	of use in scientific computing.  The module name of
	Scientific Python imported in a Python session is
	\mods{Scientific}.
\end{itemize}

One other required Python package, \mods{f2py}, is now a part of the
NumPy package, and so installation of NumPy is sufficient to give
you both.

The package \htmladdnormallinkfoot{SciPy}{http://www.scipy.org},
which includes several Python-accessible scientific libraries, also
includes NumPy (and thus \mods{f2py}), so if you install SciPy,
you don't have to install NumPy again.  Note that SciPy is not the
same as Scientific Python; the names are confusing.

A few non-Python packages are also required:
\begin{itemize}
\item \LaTeX: A scientific typesetting program used by the 
	\class{Qtcm} instance method \mods{plotm} to handle 
	exponents and subscripts.  The most common Unix 
	distribution of \LaTeX\ is
	\htmladdnormallinkfoot{teTeX}{http://www.tug.org/teTeX}.

\item netCDF:  This set of libraries enables one to write datasets into
	a platform independent, binary format, with metdata attached.
	The \htmladdnormallinkfoot{netCDF 3.6.2 library}%
        	{http://www.unidata.ucar.edu/software/netcdf/}
	source code can be
\latexhtml{downloaded from UCAR\footnote{http://www.unidata.ucar.edu/downloads/netcdf/netcdf-3\_6\_2/}}%
        {\htmladdnormallink{downloaded from UCAR}{http://www.unidata.ucar.edu/downloads/netcdf/netcdf-3_6_2/}}.
\end{itemize}

For most Unix installations, the easiest way to install all the
above is via a package manager, for instance \mods{apt-get} in
Debian GNU/Linux, \mods{aptitude} or \mods{synaptic} in Ubuntu
GNU/Linux, and \mods{fink} in Mac OS X.  Of course, you can also
download a package's source code and build direct and/or install
using Python's
\htmladdnormallinkfoot{\mods{distutils}}{http://docs.python.org/dist/dist.html}
utilities.




% ===== end of file =====

	\section{Compiling Extension Modules}  \label{sec:create.so}
	% ==========================================================================
% Compiling extension modules
%
% By Johnny Lin
% ==========================================================================


% ------ BODY -----
%

The extension modules (\fn{.so} files) are imported and used by
\mods{qtcm} objects, and contain the Fortran QTCM1 model that is
called by the \mods{qtcm} Python wrappers.  These extension modules
are located in the \fn{lib} directory of the \mods{qtcm} distribution,
and, in general, need to be created only when the \mods{qtcm} package
is installed.

Two extension modules are created:  \fn{\_qtcm\_full\_365.so} and
\fn{\_qtcm\_parts\_365.so}.  Both modules define QTCM1 models where:

\begin{itemize}
\item A year is 365 days long 
	(makefile macro \vars{YEAR360} is off).
\item Model output is written to netCDF files
	(makefile macro \vars{NETCDFOUT} is on).
\item The atmospheric boundary layer model is used
	(makefile macro \vars{NO\_ABL} is off).
\item A global domain is used
	(makefile macro \vars{SPONGES} is off).
\item Topography effects due to induced divergence are not included
	(makefile macro \vars{TOPO} is off).
\item Coupling between atmosphere and ocean is through mean fluxes
	(makefile macro \vars{CPLMEAN} is off).
\item The mixed layer ocean model is not used
	(makefile macros \vars{MXL\_OCEAN} and \vars{BLEND\_SST} are both off).
\end{itemize}

(All other makefile macros not listed are also turned off.)
The only difference between these two extension modules is that the
``full'' module is used by \class{Qtcm} instances where
\vars{compiled\_form} is set to \vars{'full'}, and the ``parts''
module is used by \class{Qtcm} instances where \vars{compiled\_form}
is set to \vars{'parts'}.  See Section~\ref{sec:compiledform} for
details about the \vars{compiled\_form} attribute.

The extension modules are created through the following steps:
\begin{enumerate}
\item Go to the \mods{qtcm} distribution directory
	\fn{qtcm-0.1.2}located in
	your build path \fn{/buildpath}.  Go to the \fn{src}
	sub-directory.  This is where all the building of the
	extension modules will take place.

\item Copy the makefile that corresponds to your platform to
	the \fn{src} directory, and rename it \fn{makefile}.
	The \fn{Makefiles} sub-directory of \fn{src} contains
	makefiles for various platforms.

\item In \fn{makefile}, make the following changes:
	\begin{enumerate}
	\item Change the \vars{FC} environment variable as needed, 
		if your Fortran compiler is different.
	\item Change the \vars{FFLAGSM} environment variable, if the
		compiler flags listed are not supported by your
		compiler.
	\item Change the \vars{-I} and \vars{-L} parts of the
		\vars{NCINC} and \vars{NCLIB} environment
		variables so that the paths for the netCDF library and
		include files match your system's installation:
		\begin{codeblock}
		\codeblockfont{%
NCINC=-I/yourpath/netcdf/include \\
NCLIB=-L/yourpath/netcdf/lib -lnetcdf}
		\end{codeblock}
		Set \dumarg{yourpath} to the full path to the
		\fn{netcdf} directory where the \fn{include} and
		\fn{lib} sub-directories are that hold the netCDF
		libraries and include files.
		(You shouldn't have to change the \vars{-l} part of
		\vars{NCLIB}, since it is standard to name the netCDF
		library \fn{libnetcdf.a}.  But if you have a non-standard
		installation, change the \vars{-l} part too.)
	\end{enumerate}

\item At the Unix prompt, type:
\begin{codeblock}
\codeblockfont{%
\small
make clean \&\& make \_qtcm\_full\_365.so \&\& make \_qtcm\_parts\_365.so}
\end{codeblock}
	to clean up leftover files from previous compilations, and to
	compile the extension module shared object files
	\fn{\_qtcm\_full\_365.so} and \fn{\_qtcm\_parts\_365.so}.
\end{enumerate}

The makefile will automatically move the shared object files into
\fn{../lib}, overwriting any pre-existing files of the same name.
A detailed description of the makefile and using \mods{f2py} is
given in Section~\ref{sec:create.new.so}, if you wish to create a
different extension module.




% ===== end of file =====

	\section{Testing the Installation}     \label{sec:test.qtcm}
	% ==========================================================================
% Installation Summary
%
% By Johnny Lin
% ==========================================================================


% ------ BODY -----
%

The \mods{qtcm} distribution comes with a set of tests for the
package, using Python's \mods{unittest} package.  
Just to warn you, the tests take around an hour to run.
The tests will not work if the contents of \fn{lib}
after you've finished building \mods{qtcm} have not been copied
to a directory named \fn{qtcm} that is on your \mods{sys.path} path,
so make sure you've gone through all the install steps
(summarized in Section~\ref{sec:install.sum}) before you do these
tests.

\emphpara{NB:}  For these tests to work, both \cmd{python} and
\cmd{python2.4} must refer to the executable for the Python
installation on your system that you are using for running \mods{qtcm}.

The tests require a set of benchmark output files in the
\fn{test/benchmarks} directory in the
\fn{qtcm-0.1.2}directory (the output will be in
directories whose names begin with ``aquaplanet'' or ``landon'').
These output files are not included with the \mods{qtcm} distribution,
and must be created, by doing the following:

\begin{enumerate}
\item Goto the directory \fn{test/benchmarks/create/src} in the
	\fn{qtcm-0.1.2}\mods{qtcm} distribution directory,
	and copy the makefile from sub-directory \fn{Makesfiles},
	that corresponds to your system to the
	\fn{test/benchmarks/create/src} directory.  Rename the makefile 
	in \fn{test/benchmarks/create/src} to \fn{makefile}.

\item In \fn{makefile}, make the following changes:
        \begin{enumerate}
        \item Change the \vars{FC} environment variable as needed,
                if your Fortran compiler is different.
        \item Change the \vars{FFLAGSM} environment variable, if the
                compiler flags listed are not supported by your
                compiler.
        \item Change the \vars{-I} and \vars{-L} parts of the
                \vars{NCINC} and \vars{NCLIB} environment
                variables so that the paths for the netCDF library and
                include files match your system's installation:
                \begin{codeblock}
                \codeblockfont{%
NCINC=-I/yourpath/netcdf/include \\
NCLIB=-L/yourpath/netcdf/lib -lnetcdf}
                \end{codeblock}
                Set \dumarg{yourpath} to the full path to the
                \fn{netcdf} directory where the \fn{include} and
                \fn{lib} sub-directories are that hold the netCDF
                libraries and include files.
                (You shouldn't have to change the \vars{-l} part of
                \vars{NCLIB}, since it is standard to name the netCDF
                library \fn{libnetcdf.a}.  But if you have a non-standard
                installation, change the \vars{-l} part too.)
        \end{enumerate}

\item Go to the directory \fn{test/benchmarks/create} in the
	\fn{qtcm-0.1.2}\mods{qtcm} distribution directory.

\item Type \cmd{python create\_benchmarks.py} at the Unix command line
	to run the benchmark creation script.
\end{enumerate}

The created benchmarks will be located in 
\fn{test/benchmarks}, in directories with names related to the
run that was done, as described earlier.
The benchmarks are created using the
pure-Fortran QTCM1 model code,
version 2.3 (August 2002), with an altered makefile
(described above) and the following code change:
In all \fn{.F90} files, occurrences of:
        \begin{codeblock}
        \codeblockfont{%
        Character(len=130)}
        \end{codeblock}
        are changed to:
        \begin{codeblock}
        \codeblockfont{%
        Character(len=305)}
        \end{codeblock}
This enables the model to properly deal with longer filenames.
The number ``305'' is chosen to make search and replace easier.

Once the benchmarks are created, you can test the \mods{qtcm} package
by doing the following:
\begin{enumerate}
\item Go to the \fn{test} directory in the 
	\fn{qtcm-0.1.2}directory.
\item Type \cmd{python test\_all.py} at the Unix command line.
\end{enumerate}

If at the end of the test runs you see this message (or something similar):
\begin{codeblock}
\codeblockfont{%
\footnotesize
---------------------------------------------------------------------- \\
Ran 93 tests in 1244.205s \\
 \\
OK}
\end{codeblock}
then everything worked fine!  If you get any other message, the test(s) have
failed.



% ===== end of file =====

	\section{Model Performance}
	%=====================================================================
% Model Performance
%=====================================================================


% ----- BEGIN TEXT -----
%
%---------------------------------------------------------------------

The wall-clock time values below give the mean over three
separate 365 day aquaplanet runs,
using climatological sea surface temperature for lower boundary forcing.
NetCDF output is written daily, for both instantaneous and mean values.
The time step is 1200~sec, and the version of \mods{qtcm} used
is 0.1.1.
The horizontal grid spacing of all model versions is
$5.625^{\circ}$ longitude by $3.75^{\circ}$ latitude.
Values are in seconds:
\begin{center}
\begin{tabular}{p{0.5\linewidth}|c|c|c}
\textbf{System} & \textbf{Pure} & \textbf{Full} & \textbf{Parts} \\
\hline
Mac OS X:  MacBook 1.83 GHz Intel Core Duo running Mac OS X
	10.4.10 with 1 GB RAM
	(Python 2.4.3, NumPy 1.0.3, \mods{f2py} 2\_3816).
    & 152.59 & 153.63 & 158.94 \\
\hline
Ubuntu GNU/Linux:  Dell PowerEdge 860 with 2.66 GHz Quad Core Intel
	Xeon processors (64 bit) running Ubuntu 8.04.1 LTS
	(Python 2.5.2, NumPy 1.1.0, \mods{f2py} 2\_5237).
    & 43.73 & 44.79 & 47.45
\end{tabular}
\end{center}

``Pure'' refers to the pure-Fortran version of QTCM1.
``Full'' refers to a \mods{qtcm} run session with \vars{compiled\_form}
set to \vars{'full'}.  ``Parts'' refers to a \mods{qtcm} run session
with \vars{compiled\_form} set to \vars{'parts'}.
(Section~\ref{sec:compiledform} has details about the difference
between compiled forms.)

The \vars{'parts'} version of \mods{qtcm} gives Python the maximum
flexibility in accessing compiled QTCM1 model subroutines and
variables.  The price of that flexibility is an increase in
run time of approximately 4--9\% over the pure-Fortran version.
The difference in performance between the
\vars{'full'} version of \mods{qtcm} and the pure-Fortran version of
QTCM1 is between negligible and 3\% longer.

To make a timing for the pure-Fortran model, go to
\fn{test/benchmarks/timing/work} in \fn{/buildpath} and run the
\fn{timing\_365.sh} script in that directory.  That script runs the
QTCM1 model using \cmd{/usr/bin/time}, which at the end of the
script will output the amount of time it took to make the model
run.  Run the timing script three times and average the values to
obtain a time comparable to the above.

To make a timing for the \mods{qtcm} model, type \cmd{python
timing\_365.py} while in the \fn{test} directory in \fn{/buildpath}.
Three run sessions will be made for \vars{compiled\_form} equal to
\vars{'full'} and \vars{'parts'}, the times are averaged, and the
value are output at the end of the script.




% ====== end file ======

	\section{Installing in Mac OS X}       \label{sec:install.macosx}
	% ==========================================================================
% Description of installing in Mac OS X
%
% By Johnny Lin
% ==========================================================================


% ------ BODY -----
%
%------------------------------------------------------------------------
\subsection{Introduction}

This section describes issues and a summary of the installation steps
I followed to install \mods{qtcm} on a Mac running OS X.
It is a specific realization of the general installation
instructions found in Sections~\ref{sec:install.sum}--\ref{sec:test.qtcm}.
I first worked through these installation steps during June--July 2007,
with updates during July 2008.
The best way to go through this section is to go through
the summary of the installation steps in 
Section~\ref{sec:osx.install.summary},
and looking back to other sections as needed.




%------------------------------------------------------------------------
\subsection{Platform and Unix Dependencies}

This work was done on a MacBook 1.83 GHz Intel Core Duo running Mac OS X
10.4.11.  My machine has 1 GB RAM and 64 GB of disk in its main partition.

I recommend you turn-off your antivirus software before you
do the installs.  
Problems have been
\latexhtml{reported by Fink users\footnote%
		{http://finkproject.org/faq/usage-fink.php?phpLang=en\#kernel-panics}}%
	{\htmladdnormallink{reported by Fink users}%
		{http://finkproject.org/faq/usage-fink.php?phpLang=en#kernel-panics}}
using the Fink package manager with antivirus software enabled.

There are a variety of dependencies that are required to get your Mac
up-and-running as a scientific computing platform.  The most basic is
installing Apple's 
\htmladdnormallinkfoot{XCode}{http://developer.apple.com/tools/xcode/}
developer tools.\footnote%
	{The package should work in Mac OS X 10.4 with XCode 2.4.1 and higher;
	I've tried it with both 2.4.1 and XCode 2.5.  Note that
	XCode 3.1 only works on Mac OS X 10.5.}
This set of tools contains compilers and libraries
needed to do anything further.  You have to be a member of Apple's
Developer Connection, but registration is free.

Besides XCode, there are a variety of Unix libraries and utilities that you
need.  I first tried installing them by myself, from scratch, into
\fn{/usr/local}, but it was hard to keep track of all the dependencies.
A few that did work, and that I installed from their disk images, are:
\htmladdnormallinkfoot{MacTeX}{http://www.tug.org/mactex/}, 
\htmladdnormallinkfoot{MAMP}{http://www.mamp.info/}, and 
\htmladdnormallinkfoot{Tcl/Tk Aqua BI (Batteries Included)}%
	{http://tcltkaqua.sourceforge.net/}.\footnote%
		{Theoretically you can use Fink to install the equivalent
		of these packages, but I like the specific collection 
		found in these packages.  For instance, Tcl/Tk Aqua BI
		runs natively on the Mac.}

For everything else, thankfully, there's the
\htmladdnormallinkfoot{Fink Project}{http://www.finkproject.org/} which
uses a package manager built upon Debian tools to install ports of
Unix programs onto a Mac.  I just 
\htmladdnormallinkfoot{downloaded}%
	{http://www.finkproject.org/download/index.php?phpLang=en}
a binary version of the Fink 0.8.1 installer for Intel Macs,
installed Fink, and used its package management tools to install
(almost) everything else I needed.\footnote%
	{The one drawback of Fink is that it sometimes
	has stability problems.  In those cases, Fink provides
	command line suggestions to fix the problems, which sometimes
	will work.  If not, sometimes
\latexhtml{%
	deleting Fink and everything it installed,\footnote%
	{http://www.finkproject.org/faq/usage-fink.php?phpLang=en\#removing}}{%
\htmladdnormallink{deleting Fink and everything it installed}
	{http://www.finkproject.org/faq/usage-fink.php?phpLang=en#removing},}
	and starting afresh, will do the trick.
	It also appeared to me that sometimes when I installed 
	multiple packages
	via one \cmd{fink install} call, the installation did not work
	as well as when I installed only one package per call.}

Although you do not need anything besides a Fortran compiler and
the netCDF libraries to run QTCM1 in its pure-Fortran form, in order to
manipulate the model and use this Python version \mods{qtcm}, you
need to have Python installed.  The default Python that comes
with the Mac is a little old, so I used Fink to also install
Python 2.5 and related packages, including
\htmladdnormallinkfoot{matplotlib}{http://matplotlib.sourceforge.net/},
\htmladdnormallinkfoot{ScientificPython}{http://dirac.cnrs-orleans.fr/plone/software/scientificpython/},
and
\htmladdnormallinkfoot{SciPy}{http://www.scipy.org}
(see Section~\ref{sec:osx.summary} for details).




%------------------------------------------------------------------------
\subsection{Fortran Compiler}

There are a variety of high-quality, commercial Fortran compilers.
Unfortunately, because I do not have a research budget, I am not able
to use those compilers.  The 
\htmladdnormallinkfoot{GNU Compiler Collection}{http://gcc.gnu.org/}
(GCC) provides a suite of open-source compilers, some of which are the
standards of their language.  Most of the GCC compilers are installed
on your Mac when you install XCode.

\htmladdnormallinkfoot{GNU Fortran}{http://gcc.gnu.org/fortran/}
(\mods{gfortran}), is the Fortran 95 compiler included with the more
recent versions of GCC.
Unfortunately, I was not able to get it to compile QTCM1.
There is a second open-source Fortran compiler,
\htmladdnormallinkfoot{G95}{http://www.g95.org/} (\mods{g95}),
which some feel is farther along in its development than \mods{gfortran}.
I was able to successfully compile QTCM1 with \mods{g95} on my Mac.
I used Fink to install G95
(see Section~\ref{sec:osx.summary} for details).




%------------------------------------------------------------------------
\subsection{NetCDF Libraries}   \label{sec:netcdf}

For some reason, the netCDF libraries and include files
installed by Fink didn't correspond to the files needed
by the calling routines in \mods{qtcm}.  To solve this, I compiled
my own set of 
\htmladdnormallinkfoot{netCDF 3.6.2 libraries}%
	{http://www.unidata.ucar.edu/software/netcdf/}
using the tarball 
\latexhtml{downloaded from UCAR\footnote{http://www.unidata.ucar.edu/downloads/netcdf/netcdf-3\_6\_2/}}%
        {\htmladdnormallink{downloaded from UCAR}{http://www.unidata.ucar.edu/downloads/netcdf/netcdf-3_6_2/}}.

Once I uncompressed and untarred the package, and went into 
the top-level directory of the package, I built the package by typing
the following at the Unix prompt:

\begin{codeblock}
\codeblockfont{%
./configure --prefix=/Users/jlin/extra/netcdf \\
make check \\
make install}
\end{codeblock}

This installed the netCDF binaries, libraries, and include files into
sub-directories \fn{bin}, \fn{lib}, and \fn{include} in 
the directory specified by \vars{--prefix}.
If you want to install the netCDF libraries in the default
(usually \fn{/usr/local}), just leave out the \vars{--prefix}
option.

Note:  When you build netCDF, make sure the build directory
is not in the directory tree of \vars{--prefix}
(or the default directory \fn{/usr/local}).




%------------------------------------------------------------------------
\subsection{Makefile Configuration}  \label{sec:osx.makefile}

	\subsubsection{NetCDF}

In the \fn{src} directory in the \mods{qtcm} distribution, there is a
sub-directory \fn{Makefiles} that contains the makefiles for a
variety of platforms.  Edit the file \fn{makefile.osx\_g95}
so that the lines specifying the environment variables for the
netCDF libraries and include files:

\begin{codeblock}
\codeblockfont{%
NCINC=-I/Users/jlin/extra/netcdf/include \\
NCLIB=-L/Users/jlin/extra/netcdf/lib -lnetcdf}
\end{codeblock}

are changed to the path where your \emph{manually compiled} 
netCDF libraries and include files are.

Copy \fn{makefile.osx\_g95} from the \fn{Makefiles} sub-directory
in \fn{src} into \fn{src}.  
In other words, from the \mods{qtcm} distribution directory
(i.e., \fn{/buildpath}), at the Unix prompt execute:

\begin{codeblock}
\codeblockfont{%
cp src/Makefiles/makefile.osx\_g95 src/makefile}
\end{codeblock}


	\subsubsection{Linking Order}

Compilers in the GNU Compiler Collection (GCC) search libraries
and object files in the order they are listed in the command-line, 
\latexhtml{from left-to-right\footnote%
        {http://gcc.gnu.org/onlinedocs/gcc-4.1.2/gcc/Link-Options.html\#index-l-670}}%
        {\htmladdnormallink{from left-to-right}{http://gcc.gnu.org/onlinedocs/gcc-4.1.2/gcc/Link-Options.html#index-l-670}}.
Thus, if routines in \fn{b.o} call routines in \fn{a.o}, 
you must list the files in the order \fn{a.o b.o}.

For some reason, that isn't the case for \mods{g95}.  Thus, you will
find \mods{g95} makefile rules structured like the following
(below is part of the rule to create an executable (\fn{qtcm}) for
benchmark runs):

% --- Two versions of this rule, one for display in PDF and the other
%     for display in HTML:
%
\begin{latexonly}
\begin{codeblock}
\codeblockfont{%
qtcm: main.o \\
\hspace*{8ex}\$(FC)~-O~\$(NCINC)~-o~\$@ main.o~\$(QTCMLIB)~\$(NCLIB)}
\end{codeblock}
\end{latexonly}

\begin{htmlonly}
\begin{rawhtml}
<p><code><font color="blue">qtcm: main.o<br>
&nbsp;&nbsp;&nbsp;&nbsp;&nbsp;&nbsp;&nbsp;$(FC) -O $(NCINC) -o 
$@ main.o $(QTCMLIB) $(NCLIB)</font></code></p>
\end{rawhtml}
\end{htmlonly}

even though \fn{main.o} depends on the QTCM library 
(specified in macro setting \vars{\$(QTCMLIB)}), which in turn
depends on the netCDF library (specified in macro setting \vars{\$(NCLIB)}).


%------------------------------------------------------------------------
\subsection{Summary of Steps}   \label{sec:osx.install.summary}

The following summarizes all the steps I took to install
\mods{qtcm} in Mac OS X:

\begin{enumerate}
\item Install
	\htmladdnormallinkfoot{XCode 2.5}%
		{http://developer.apple.com/tools/xcode/}.

\item Install 
	\htmladdnormallinkfoot{MacTeX}{http://www.tug.org/mactex/}, 
	\htmladdnormallinkfoot{MAMP}{http://www.mamp.info/}, and 
	\htmladdnormallinkfoot{TCL/Tk Aqua BI (Batteries Included)}%
		{http://tcltkaqua.sourceforge.net/}.

\item Install
	\htmladdnormallinkfoot{Fink 0.8.1}%
		{http://www.finkproject.org/download/index.php?phpLang=en}.
	Make sure you
	\htmladdnormallink{set up your environment}%
		{http://www.finkproject.org/doc/users-guide/install.php\#setup}
	to enable you to use the packages you install with Fink
	(e.g. \vars{PATH} settings, etc.).
	Most of the time, that just means adding the line
	\cmd{source /sw/bin/init.csh} to your \fn{.cshrc} file (or the
	equivalent in your \fn{.bashrc}).

	Note that for many of the packages needed to run \mods{qtcm},
	you need to 
	\htmladdnormallink{configure Fink to download packages 
		from the unstable trees}%
	{http://www.finkproject.org/faq/usage-fink.php?phpLang=en\#unstable}.
	To do that, add \vars{unstable/main} and \vars{unstable/crypto}
	to the \vars{Trees:} line in \fn{/sw/etc/fink.conf}, and run:

	\begin{codeblock}
	\codeblockfont{fink selfupdate} \\
	\codeblockfont{fink index} \\
	\codeblockfont{fink scanpackages} \\
	\codeblockfont{fink update-all}
	\end{codeblock}

	When \cmd{selfupdate} runs, choose \cmd{rsync} for the
	self update method.  If you do not, Fink will not look in the
	unstable trees for packages.

\item Use Fink to install the \mods{g95} Fortran compiler.
	From a Unix prompt, type:

	\begin{codeblock}
	\codeblockfont{fink -$\,\!$-use-binary-dist install g95}
	\end{codeblock}

\item Use Fink to install Python 
	and the NumPy package (which \mods{f2py} is a part of).
	From a Unix prompt, type:

	\begin{codeblock}
	\codeblockfont{%
	fink -$\,\!$-use-binary-dist install python25 \\
	fink -$\,\!$-use-binary-dist install scipy-core-py25}
	\end{codeblock}

	(Numpy used to be called SciPy Core.)  If you want to
	install Python 2.4 instead, just change the ``25'' and ``py25'' above
	(and in later occurrences) to ``24'' and ``py24'', respectively.
	Note that Fink does not have a version of epydoc for Python 2.4,
	so if you wish to create documentation using epydoc, you will
	need to install Python 2.5.

\item Install teTeX and \LaTeX{2HTML} using Fink.
	From a Unix prompt, type:

	\begin{codeblock}
	\codeblockfont{fink -$\,\!$-use-binary-dist install tetex} \\
	\codeblockfont{fink -$\,\!$-use-binary-dist install latex2html}
	\end{codeblock}

	When prompted, choose ghostscript and ghostscript-fonts to
	satistfy the dependency (which should be the default options).
	I tried choosing system-ghostscript8, but Fink looks for
	ghostscript 8.51 and didn't recognize ghostscript 8.57 that
	was already installed in \fn{/usr/local} (via my MacTeX
	install).  \LaTeX{2HTML} has a package required by the
	\mods{qtcm} manual \LaTeX\ file.

\item Install additional programming and
	scientific packages and libraries using Fink.
	From a Unix prompt, type:

	\begin{codeblock}
	\codeblockfont{%
	fink -$\,\!$-use-binary-dist install scientificpython-py25 \\
	fink -$\,\!$-use-binary-dist install matplotlib-py25 \\
	fink -$\,\!$-use-binary-dist install matplotlib-basemap-py25 \\
	fink -$\,\!$-use-binary-dist install matplotlib-basemap-data-py25 \\
	fink -$\,\!$-use-binary-dist install xaw3d \\
	fink -$\,\!$-use-binary-dist install fftw fftw3 \\
	fink -$\,\!$-use-binary-dist install epydoc-py25 \\
	fink -$\,\!$-use-binary-dist install graphviz \\
	fink -$\,\!$-use-binary-dist install scipy-py25}
	\end{codeblock}

\item Manually install netCDF 3.6.2
	(see Section \ref{sec:netcdf}).

\item From this point on, you can follow the
	general instructions given in Section~\ref{sec:install.sum},
	starting with step~\ref{list:download.qtcm.sum}.
	Please do not ignore, however, Section~\ref{sec:install.macosx}'s
	Mac-specific details.

\end{enumerate}



% ===== end of file =====

	\section{Installing in Ubuntu}         \label{sec:install.ubuntu}
	% ==========================================================================
% Description of installing in Ubuntu
%
% By Johnny Lin
% ==========================================================================


% ------ BODY -----
%
%------------------------------------------------------------------------
\subsection{Introduction}

This section describes installation issues 
I followed to install \mods{qtcm} on my
Dell PowerEdge 860 running Ubuntu GNU/Linux 8.04.1 LTS (Hardy).
The machine has 2.66 GHz Quad Core Intel Xeon processors (64 bit),
4 GB RAM, and 677 GB of disk in its main partition.
This section is a specific realization of the general installation
instructions found in Sections~\ref{sec:install.sum}--\ref{sec:test.qtcm}.
I worked through these installation steps during July 2008.
The best way to go through this section is to go through
the summary of the installation steps in 
Section~\ref{sec:ubuntu.install.summary},
and looking back to other sections as needed.



%------------------------------------------------------------------------
\subsection{Fortran Compiler}     \label{sec:ubuntu.fort.install}

The easiest Fortran compiler to install in Ubuntu 8.04.1 is
\htmladdnormallinkfoot{GNU Fortran}{http://gcc.gnu.org/fortran/}
(\mods{gfortran}), the Fortran 95 compiler included with the more
recent versions of the GNU Compiler Collection (GCC); you can
use any package manager (e.g., \mods{apt-get}, \mods{aptitude})
to install it.
Unfortunately, I was not able to get it to compile QTCM1.
I was, however, able to successfully compile QTCM1 using
the second open-source Fortran compiler,
\htmladdnormallinkfoot{G95}{http://www.g95.org/} (\mods{g95}),
which some feel is farther along in its development than \mods{gfortran}.
G95, however, is not supported as an Ubuntu package, and so I had
to manually install it.

I downloaded the binary version of G95 v0.91 
(the Linux x86\_64/EMT64 with 32 bit default integers) 
using the following
\cmd{curl} command:\footnote%
	{I use \mods{curl} because I usually access my
	Ubuntu server via a terminal session.}

\begin{codeblock}
\codeblockfont{%
\small
curl -o g95.tgz http://ftp.g95.org/v0.91/g95-x86\_64-32-linux.tgz}
\end{codeblock}

which saves the \fn{.tgz} file as the local file \fn{g95.tgz}.
After that, I followed the G95 project's standard
\latexhtml{installation instructions\footnote%
	{http://g95.sourceforge.net/docs.html\#starting}}%
	{\htmladdnormallink{installation instructions}%
		{http://g95.sourceforge.net/docs.html#starting}}
to finish the install.\footnote%
	{The G95 installation instructions say you can put
	\fn{g95-install} anywhere, and make a link to the
	executable \mods{g95} in
	\fn{$\sim$/bin}.  I put \fn{g95-install} in
	\fn{/usr/local}, and while in \fn{/usr/local/bin}, 
	I put a link to the G95 executable using the command:
	\begin{codeblock}
	\codeblockfont{%
	sudo ln -s ../g95-install\_64/bin/x86\_64-suse-linux-gnu-g95 g95.}
	\end{codeblock}}
The regular Linux x86 version of G95
(in \fn{g95-x86-linux.tgz} from the G95 website) did not work on my
machine.




%------------------------------------------------------------------------
\subsection{NetCDF Libraries}   \label{sec:ubuntu.netcdf}

%Here things were very confusing for my machine, as I needed to
%install two versions of the
%\htmladdnormallinkfoot{netCDF}%
%	{http://www.unidata.ucar.edu/software/netcdf/}
%libraries and include files, one 
%for a successful compilation of the extension modules
%(as described in Section~\ref{sec:create.so}),
%and the other 
%for a successful run of the pure-Fortran QTCM1 model
%(used to create the testing benchmarks, as described in
%Section~\ref{sec:test.qtcm}).
%
%The first set of netCDF files (for the extension modules) are
%installed from Ubuntu's package management system.
%These are automatically installed when the \mods{python-netcdf}
%package is installed via an Ubuntu package manager
%(see Section~\ref{sec:ubuntu.install.summary}).
%The include files for this netCDF installation are 
%located in \fn{/usr/include}, and the libraries for this
%netCDF installation are location in \fn{/usr/lib}.

For some reason, the netCDF libraries and include files
installed from the Ubuntu packages do not
correspond to the files needed
by the calling routines in \mods{qtcm}.  To solve this, I compiled
my own set of
\htmladdnormallinkfoot{netCDF 3.6.2 libraries}%
        {http://www.unidata.ucar.edu/software/netcdf/}
using the tarball
\latexhtml{downloaded from UCAR\footnote{http://www.unidata.ucar.edu/downloads/netcdf/netcdf-3\_6\_2/}}%
        {\htmladdnormallink{downloaded from UCAR}{http://www.unidata.ucar.edu/downloads/netcdf/netcdf-3_6_2/}}.

Once I uncompressed and untarred the package, and went into
the top-level directory of the package, I built the package by typing
the following at the Unix prompt:

\begin{codeblock}
\codeblockfont{%
export FC=g95 \\
export FFLAGS="-O -fPIC" \\
export FFLAGS="-fPIC" \\
export F90FLAGS="-fPIC" \\
export CFLAGS="-fPIC" \\
export CXXFLAGS="-fPIC" \\
./configure \\
make check \\
sudo make install}
\end{codeblock}

(The \cmd{export} commands set environment variables for the
Fortran compiler and Fortran and other compiler flags.  The
\vars{-fPIC} flag enables the compilers to create
position independent code, needed for shared libraries in
Ubuntu on a 64 bit Intel processor.)

The above installs the netCDF binaries, libraries, and include files into
sub-directories \fn{bin}, \fn{lib}, and \fn{include} in 
\fn{/usr/local}, the default.
The include files for this netCDF installation are thus
located in \fn{/usr/local/include}, and the libraries for this
netCDF installation are location in \fn{/usr/local/lib}.
(If you want to specify a different installation
location, use the \vars{--prefix} option in \cmd{configure}.)
While you don't have to have root privileges during the configuration
and check steps, you do during the installation step if you're installing
into \fn{/usr/local} (thus the \cmd{sudo} in the last step).\footnote%
	{Note that when you build netCDF, make sure the build directory
	is not in the directory tree of \vars{--prefix}
	or the default directory \fn{/usr/local}.}

%Because there are two different netCDF installations used in the
%\mods{qtcm} package, the makefiles for creating the benchmarks
%and extensions files will have different \vars{NCLIB} and \vars{NCINC}
%environment variables (see Section~\ref{sec:ubuntu.makefile}).




%------------------------------------------------------------------------
\subsection{Makefile Configuration}  \label{sec:ubuntu.makefile}

	\subsubsection{NetCDF}

In the \fn{src} directory in the \mods{qtcm} distribution, there is a
sub-directory \fn{Makefiles} that contains the makefiles for a
variety of platforms.  Edit the file \fn{makefile.ubuntu\_64\_g95}
so that the lines specifying the environment variables for the
netCDF libraries and include files:

\begin{codeblock}
\codeblockfont{%
NCINC=-I/usr/local/include \\
NCLIB=-L/usr/local/lib -lnetcdf}
\end{codeblock}

are changed to the path where your manually compiled
netCDF libraries and include files are.

Copy \fn{makefile.ubuntu\_64\_g95} from the \fn{Makefiles} sub-directory
in \fn{src} into \fn{src}.  
In other words, from the \mods{qtcm} distribution directory
(i.e., \fn{/buildpath}), at the Unix prompt execute:

\begin{codeblock}
\codeblockfont{%
cp src/Makefiles/makefile.ubuntu\_64\_g95 src/makefile}
\end{codeblock}


	\subsubsection{Linking Order}

Compilers in the GNU Compiler Collection (GCC) search libraries
and object files in the order they are listed in the command-line,
\latexhtml{from left-to-right\footnote%
	{http://gcc.gnu.org/onlinedocs/gcc-4.1.2/gcc/Link-Options.html\#index-l-670}}%
	{\htmladdnormallink{from left-to-right}{http://gcc.gnu.org/onlinedocs/gcc-4.1.2/gcc/Link-Options.html#index-l-670}}.
Thus, if routines in \fn{b.o} call routines in \fn{a.o}, 
you must list the files in the order \fn{a.o b.o}.

For some reason, that isn't the case for \mods{g95}.  Thus, you will
find \mods{g95} makefile rules structured like the following
(below is part of the rule to create an executable (\fn{qtcm}) for
benchmark runs):

% --- Two versions of this rule, one for display in PDF and the other
%     for display in HTML:
%
\begin{latexonly}
\begin{codeblock}
\codeblockfont{%
qtcm: main.o \\
\hspace*{8ex}\$(FC)~-O~\$(NCINC)~-o~\$@ main.o~\$(QTCMLIB)~\$(NCLIB)}
\end{codeblock}
\end{latexonly}

\begin{htmlonly}
\begin{rawhtml}
<p><code><font color="blue">qtcm: main.o<br>
&nbsp;&nbsp;&nbsp;&nbsp;&nbsp;&nbsp;&nbsp;$(FC) -O $(NCINC) -o 
$@ main.o $(QTCMLIB) $(NCLIB)</font></code></p>
\end{rawhtml}
\end{htmlonly}

even though \fn{main.o} depends on the QTCM library 
(specified in macro setting \vars{QTCMLIB}), which in turn
depends on the netCDF library (specified in macro setting \vars{NCLIB}).


	\subsubsection{Shared Object PIC}   \label{sec:sopic}

In order to compile the model in Ubuntu on a 64 bit Intel processor,
the model and the netCDF library it is linked to needs to be
compiled to be 
\latexhtml{position independent code (PIC).\footnote%
		{http://www.gentoo.org/proj/en/base/amd64/howtos/index.xml?part=1\&chap=3}}%
	{\htmladdnormallink{position independent code (PIC)}%
		{http://www.gentoo.org/proj/en/base/amd64/howtos/index.xml?part=1&chap=3}.}
This is accomplished with the 
\htmladdnormallinkfoot{\cmd{-fPIC} flag}%
	{http://www.fortran-2000.com/ArnaudRecipes/sharedlib.html}.

In the \mods{qtcm} makefiles, the \cmd{-fPIC} flag is introduced in the
macro \vars{FFLAGSM}, for instance:
\begin{codeblock}
\codeblockfont{%
FFLAGSM = -O -fPIC}
\end{codeblock}
For makefiles used in creating extension modules, \cmd{-fPIC} must
be passed into the \mods{f2py} call.  To do so, put the flags:
\begin{codeblock}
\codeblockfont{%
--f90flags="-fPIC" --f77flags="-fPIC"}
\end{codeblock}
after the \vars{--fcompiler} flag in the \mods{f2py} calling line.

The \cmd{-fPIC} flag must also be used when compiling the netCDF
libraries, as described in Section~\ref{sec:ubuntu.netcdf}.
Failure to create PIC libraries in 64 bit Ubuntu can result in errors 
like the following when creating the \mods{qtcm} extension modules:
\begin{codeblock}
\codeblockfont{%
ld: /usr/local/lib/libnetcdf.a(fort-attio.o): relocation R\_X86\_64\_32 against `a local symbol' can not be used when making a shared object; recompile with -fPIC /usr/local/lib/libnetcdf.a: could not read symbols: Bad value}
\end{codeblock}




%------------------------------------------------------------------------
\subsection{Summary of Steps}      \label{sec:ubuntu.install.summary}

The following summarizes all the steps I took to install
\mods{qtcm} in
Ubuntu 8.04.1 LTS (Hardy) running on a
Quad Core Intel Xeon (64 bit) machine.
Note that while I use the \mods{aptitude} package manager, you are
free to use any manager of your choice (e.g., \mods{apt-get},
\mods{synaptic}, etc.):

\begin{enumerate}
\item Install the G95 Fortran compiler from the binary distribution.
	See Section~\ref{sec:ubuntu.fort.install} for details.

\item Use an Ubuntu package manager
	to install the following packages, by typing:
	\begin{codeblock}
	\codeblockfont{%
sudo aptitude update \\
sudo aptitude install curl \\
sudo aptitude install python-epydoc \\
sudo aptitude install python-matplotlib \\
sudo aptitude install python-netcdf \\
sudo aptitude install python-scientific \\
sudo aptitude install python-scipy \\
sudo aptitude install texlive}
	\end{codeblock}

	Installing \mods{python-scipy} will also install NumPy and
	\mods{f2py}, so you don't have to install the
	\mods{python-numpy} package separately.

	Early-on as I debugged my \mods{qtcm} install on Ubuntu,
	I encountered errors that I thought came from an 
	\htmladdnormallinkfoot{old version of NumPy}%
		{http://cens.ioc.ee/pipermail/f2py-users/2008-June/001617.html},
	and thus I replaced Ubuntu's packaged NumPy with NumPy 1.1.0
	built 
	\latexhtml{directly from source.\footnote%
			{http://sourceforge.net/project/showfiles.php?group\_id=1369\&package\_id=175103}}%
		{\htmladdnormallink{directly from source}{http://sourceforge.net/project/showfiles.php?group_id=1369&package_id=175103}.}
	(Note, you shouldn't install your new NumPy in the default
	location, which may cause problems later-on with Ubuntu's
	package manager.)
	Later on, I concluded the errors I had encountered were not
	because of the NumPy version, but by then I didn't want to
	try to reinstall NumPy again.
	So strictly speaking, the version of Numpy I used is not
	the one bundled with \mods{python-scipy}, but that shouldn't
	be a problem.

\item Manually install netCDF 3.6.2 from source
	(see Section \ref{sec:ubuntu.netcdf}).

\item Manually install the \mods{basemap} package of
	\mods{matplotlib}.  
	The source for the \mods{basemap} toolkit is
	available 
	\latexhtml{from Sourceforge\footnote%
			{http://sourceforge.net/project/showfiles.php?group\_id=80706}}%
		{\htmladdnormallink{from Sourceforge}%
			{http://sourceforge.net/project/showfiles.php?group_id=80706}}
	I obtained version 0.9.9.1 using the
	following \cmd{curl} command:
	\begin{codeblock}
	\codeblockfont{%
\scriptsize
curl -o basemap.tar.gz $\backslash$ \\
http://voxel.dl.sourceforge.net/sourceforge/matplotlib/basemap-0.9.9.1.tar.gz}
	\end{codeblock}

	The \fn{README} file in the \fn{basemap-0.9.9.1} directory has
	detailed installation instructions.  Note that you have to
	install the GEOS library first (\fn{README} has detailed
	directions on how to do that too).  To be on the safe-side,
	I would set the \vars{FC} environment variable to the G95
	compiler
	(e.g., with \cmd{export FC=g95} in Bash).

\item From this point on, you can follow the
	general instructions given in Section~\ref{sec:install.sum},
	starting with step~\ref{list:download.qtcm.sum}.
	Please do not ignore, however, Section~\ref{sec:install.ubuntu}'s
	Ubuntu-specific details.

\end{enumerate}



% ===== end of file =====


\chapter{Getting Started With \mods{qtcm}}  \label{ch:getting.started}
% ==========================================================================
% Getting Started With qtcm
%
% By Johnny Lin
% ==========================================================================


% ------ BODY -----
%
%---------------------------------------------------------------------
\section{Your First Model Run}

Figure~\ref{fig:my.first.run} shows an example of a script to make
a 30 day seasonal, aquaplanet model run, with run name ``test'',
starting from November 1, Year 1.


%--- Two versions, one for PDF, one for HTML:
\begin{latexonly}
\begin{figure}[htp]
\begin{center}
\begin{codeblock}
\codeblockfont{%
from qtcm import Qtcm \\
inputs = \{\} \\
inputs['runname'] = 'test' \\
inputs['landon'] = 0 \\
inputs['year0'] = 1 \\
inputs['month0'] = 11 \\
inputs['day0'] = 1 \\
inputs['lastday'] = 30 \\
inputs['mrestart'] = 0 \\
inputs['compiled\_form'] = 'parts' \\
model = Qtcm(**inputs) \\
model.run\_session()}
\end{codeblock}
\end{center}
\caption{An example of a simple \mods{qtcm} run.}
\label{fig:my.first.run}
\end{figure}
\end{latexonly}

\begin{htmlonly}
\label{fig:my.first.run}
\begin{center}
\htmlfigcaption{%
	\codeblockfont{%
from qtcm import Qtcm \\
inputs = \{\} \\
inputs['runname'] = 'test' \\
inputs['landon'] = 0 \\
inputs['year0'] = 1 \\
inputs['month0'] = 11 \\
inputs['day0'] = 1 \\
inputs['lastday'] = 30 \\
inputs['mrestart'] = 0 \\
inputs['compiled\_form'] = 'parts' \\
model = Qtcm(**inputs) \\
model.run\_session()}
	}

\htmlfigcaption{Figure~\ref{fig:my.first.run}:
	An example of a simple \mods{qtcm} run.}
\end{center}
\end{htmlonly}



The class describing the QTCM1 model is \class{Qtcm}.  An instance
of \class{Qtcm}, in this example \vars{model}, is created the same
way you create an instance of any class.  When instantiating an
instance of \class{Qtcm}, keyword parameters can be used to override
any default settings.  In the example above, the dictionary
\vars{inputs} specifying all keyword parameters is passed in on the
instantiation of \vars{model}.

The keyword parameter settings in
Figure~\ref{fig:my.first.run} have the following meanings:
\begin{itemize}
\item \vars{runname}:  This string (``test'') is used in the
	output filename.  QTCM1 writes mean and instantaneous
	output files to the directory given in \vars{model.outdir.value},
	with filenames 
	\fn{qm\_}\dumarg{runname}\fn{.nc} for mean output and
	\fn{qi\_}\dumarg{runname}\fn{.nc} for instantaneous output.

\item \vars{landon}: When set to ``0'', the land is turned off and
	the run is an aquaplanet run.  When set to ``1'', the land
	model is turned on.

\item \vars{year0}:  The year the run starts on.

\item \vars{month0}:  The month the run starts on (11 = November).

\item \vars{day0}: The day of the month the run starts on.

\item \vars{lastday}:  The model runs from day 1 to \vars{lastday}.

\item \vars{mrestart}:  When set to ``0'', the run starts from
	default initial conditions
	(see Section~\ref{sec:initial.variables} for a table of
	those values).
	When set to ``1'', the run starts from a restart file.

\item \vars{compiled\_form}:  This keyword sets what form the
	compiled QTCM1 model has, and its value is saved to
	the instance's \vars{compiled\_form} attribute.
	It is a string and can be set either to
	``parts'' or ``full''.  Most of the time, you will want
	to set it to \vars{'parts'}.
	This keyword is the only one
	that must be specified on instantiation; the model instance
	will at least instantiate
	using only the default settings for all the other keyword
	parameters (given in Appendix~\ref{app:defaults.values}).
	See Section~\ref{sec:compiledform} for details about
	what the \vars{compiled\_form} attribute controls.
\end{itemize}

By default, the \vars{SSTmode} attribute, which controls whether the
model will use climatological sea-surface temperatures (SST) 
or real SSTs, is set to the \vars{value} ``seasonal'', thus giving a
run with seasonal forcing at the lower-boundary over the ocean.

This example assumes that the boundary condition files, sea surface
temperature files, and the model output directories are as specified
in submodule \mods{defaults}.  Those values are described in
Section~\ref{sec:defaults.scalar}.




%---------------------------------------------------------------------
\section{Managing Directories}

Most of the time, your boundary condition files and output files
will not be in the locations specified in
Section~\ref{sec:defaults.scalar}, or in the directory your
Python script resides.  The easiest way to tell your \class{Qtcm} 
instance where your input/output files are is to pass them in
as keyword parameters on instantiation.


%--- Two versions, one for PDF, one for HTML:
\begin{latexonly}
\begin{figure}[htp]
\begin{codeblock}
\codeblockfont{%
\small
from qtcm import Qtcm \\
rundirname = 'test' \\
dirbasepath = os.path.join(os.getcwd(), rundirname) \\
inputs = \{\} \\
inputs['bnddir'] = os.path.join( os.getcwd(), 'bnddir', \\
\hspace*{40ex}'r64x42' ) \\
inputs['SSTdir'] = os.path.join( os.getcwd(), 'bnddir', \\
\hspace*{40ex}'r64x42', 'SST\_Reynolds' ) \\
inputs['outdir'] = dirbasepath \\
inputs['runname'] = rundirname \\
inputs['landon'] = 0 \\
inputs['year0'] = 1 \\
inputs['month0'] = 11 \\
inputs['day0'] = 1 \\
inputs['lastday'] = 30 \\
inputs['mrestart'] = 0 \\
inputs['compiled\_form'] = 'parts' \\
model = Qtcm(**inputs) \\
model.run\_session()}
\end{codeblock}

\caption{An example \mods{qtcm} run showing detailed description of
	input and output directories.}
\label{fig:manage.dir.example}
\end{figure}
\end{latexonly}

\begin{htmlonly}
\label{fig:manage.dir.example}
\begin{center}
\htmlfigcaption{%
	\codeblockfont{%
from qtcm import Qtcm \\
rundirname = 'test' \\
dirbasepath = os.path.join(os.getcwd(), rundirname) \\
inputs = \{\} \\
inputs['bnddir'] = os.path.join( os.getcwd(), 'bnddir', \\
\hspace*{40ex}'r64x42' ) \\
inputs['SSTdir'] = os.path.join( os.getcwd(), 'bnddir', \\
\hspace*{40ex}'r64x42', 'SST\_Reynolds' ) \\
inputs['outdir'] = dirbasepath \\
inputs['runname'] = rundirname \\
inputs['landon'] = 0 \\
inputs['year0'] = 1 \\
inputs['month0'] = 11 \\
inputs['day0'] = 1 \\
inputs['lastday'] = 30 \\
inputs['mrestart'] = 0 \\
inputs['compiled\_form'] = 'parts' \\
model = Qtcm(**inputs) \\
model.run\_session()}
	}

\htmlfigcaption{Figure~\ref{fig:manage.dir.example}:
	An example \mods{qtcm} run showing detailed description of
        input and output directories.}
\end{center}
\end{htmlonly}


Figure~\ref{fig:manage.dir.example} shows an example run where those
directories are explicitly specified; in all other aspects, the run
is identical to the one in Figure~\ref{fig:my.first.run}.
In Figure~\ref{fig:manage.dir.example}, output from the model is
directed to the directory described by string variable
\vars{dirbasepath}.  \vars{dirbasepath} is created by joining the
current working directory with the run name given in string variable
\vars{rundirname}.\footnote%
	{The Python \mods{os} module enables platform-independent
	handling of files and directories.  The \mods{os.path.join}
	function resolves paths without the programmer needing to know
	all the possible directory separation characters; the function
	chooses the correct separation character at runtime.  The
	\mods{os.getcwd} function returns the current working directory.}
Setting keyword parameter \vars{outdir} to \vars{dirbasepath} sends
output to \vars{dirbasepath}.  
Keywords \vars{bnddir} and \vars{SSTdir} specify the directories
where non-SST and SST boundary condition files, respectively, are
found.

Interestingly, the default version of QTCM1 does \emph{not} send
all output from the model to \vars{outdir}.  The restart file
\fn{qtcm\_}\dumarg{yyyymmdd}\fn{.restart} (where \dumarg{yyyymmdd}
is the year, month, and day of the model date when the restart
file was written) is written into the current working directory,
not the output directory.  Thus, if you do multiple runs, you'll
have to manually deal with the restart files that will proliferate.

Neither the QTCM1 model nor the \class{Qtcm} object
create the directories specified in \mods{bnddir}, \mods{SSTdir},
and \mods{outdir}.  Failure to do so will create an error.  I use
Python's file management tools to make sure the output directory
is created, and any old output files are deleted.  Here's an example
that does that, using the \vars{dirbasepath} and \vars{rundirname}
variables from Figure~\ref{fig:manage.dir.example}:

\begin{codeblock}
\codeblockfont{%
\small
if not os.path.exists(dirbasepath):  os.makedirs(dirbasepath) \\
qi\_file = os.path.join( dirbasepath, 'qi\_'+rundirname+'.nc' ) \\
qm\_file = os.path.join( dirbasepath, 'qm\_'+rundirname+'.nc' ) \\
if os.path.exists(qi\_file):   os.remove(qi\_file) \\
if os.path.exists(qm\_file):   os.remove(qm\_file)}
\end{codeblock}




%---------------------------------------------------------------------
\section{Model Field Variables}   \label{sec:field.variables.intro}

The term ``field'' variable refers to QTCM1 model variables that 
are accessible at both the compiled Fortran QTCM1 model-level as
well as the Python \class{Qtcm} instance-level.
Field variables are all instances of the \class{Field} class,
and are stored as attributes of the \class{Qtcm} instance.\footnote%
	{Note non-field variables can also be instances of \class{Field},
	and that \class{Qtcm} instances have other attributes that are
	not equal to \class{Field} instances.}

\class{Field} class instances have the following attributes:
\begin{itemize}
\item \vars{id}:  A string naming the field (e.g., ``Qc'', ``mrestart'').
	This string should contain no whitespace.
\item \vars{value}:  The value of the field.  Can be of any type, though
	typically is either a string or numeric scalar or a numeric array.
\item \vars{units}:  A string giving the units of the field.
\item \vars{long\_name}:  A string giving a description of the field.
\end{itemize}

\class{Field} instances also have methods to return the rank 
and typecode of \vars{value}.

Remember, if you want to access the value of a \class{Field} object,
make sure you access that object's \vars{value} attribute.  
Thus, for example,
to assign a variable \vars{foo} to the
\vars{lastday} value for a given
\class{Qtcm} instance \vars{model}, type the following:
\begin{codeblock}
\codeblockfont{%
foo = model.lastday.value}
\end{codeblock}

For scalars, this assignment sets \vars{foo} by value (i.e., a copy
of the value of attribute \vars{model.lastday} is set to \vars{foo}).
In general, however, Python assigns variables by reference.  Use
the \mods{copy} module if you truly want a copy of a field variable's
value (such as an array), rather than an alias.  For more details
about field variables, see Section~\ref{sec:field.variables}.




%---------------------------------------------------------------------
\section{Run Sessions}

	\subsection{What is a Run Session?}

A run session is a unit of simulation where the model is run from
day 1 of simulation to the day specified by the \vars{lastday}
attribute of a \class{Qtcm} instance.  A run session is a
``complete'' model run, at the beginning of which all compiled QTCM1
model variables are set to the values given at the Python-level,
and at the end of which restart files are written, the values
at the Python-level are overwritten by the values in the Fortran
model, and a Python-accessible snapshot is taken of the 
model variables that were written to the restart file.


	\subsection{Changing Variables}

Between run sessions, changing any field variable is as easy
as a Python assignment.  For instance, to change the atmosphere
mixed layer depth to 100~m, just type:
\begin{codeblock}
\codeblockfont{%
model.ziml.value = 100.0}
\end{codeblock}

When changing arrays, be careful to try to match the shape of the 
array.\footnote%
	{At the very least, match the rank of the array, which is required
	for the routines in \mods{setbypy} to properly choose which
	Fortran subroutine to use in reading the Python value.
	I haven't tested if only the rank is needed, however,
	for the passing to work, for a continuation run (my hunch is
	it won't).}
You can use the NumPy \mods{shape} function on a NumPy array to
check its shape.


	\subsection{Continuing a Model Run}  \label{sec:continuation.intro}

Figure~\ref{fig:continuation.example} shows an example of two run
sessions, where the second run session is a continuation of the
first.


%--- Two versions, one for PDF, one for HTML:
\begin{latexonly}
\begin{figure}[htp]
\begin{codeblock}
\codeblockfont{%
\small
inputs['year0'] = 1 \\
inputs['month0'] = 11 \\
inputs['day0'] = 1 \\
inputs['lastday'] = 10 \\
inputs['mrestart'] = 0 \\
inputs['compiled\_form'] = 'parts' \\ \\
model = Qtcm(**inputs) \\
model.run\_session() \\
model.u1.value = model.u1.value * 2.0 \\
model.init\_with\_instance\_state = True \\
model.run\_session(cont=30)}
\end{codeblock}

\caption{An example of two \mods{qtcm} run sessions where the second
	run session is a continuation of the first.  Assume 
	\vars{inputs} is a dictionary, and that earlier in the
	script the run name and
	all input and output directory names were added
	to the dictionary.}
\label{fig:continuation.example}
\end{figure}
\end{latexonly}

\begin{htmlonly}
\label{fig:continuation.example}
\begin{center}
\htmlfigcaption{%
	\codeblockfont{%
inputs['year0'] = 1 \\
inputs['month0'] = 11 \\
inputs['day0'] = 1 \\
inputs['lastday'] = 10 \\
inputs['mrestart'] = 0 \\
inputs['compiled\_form'] = 'parts' \\ \\
model = Qtcm(**inputs) \\
model.run\_session() \\
model.u1.value = model.u1.value * 2.0 \\
model.init\_with\_instance\_state = True \\
model.run\_session(cont=30)}
	}

\htmlfigcaption{Figure~\ref{fig:continuation.example}:
	An example of two \mods{qtcm} run sessions where the second
	run session is a continuation of the first.  Assume 
	\vars{inputs} is a dictionary, and that earlier in the
	script the run name and
	all input and output directory names were added
	to the dictionary.}
\end{center}
\end{htmlonly}


The first run session runs from day 1 to day 10.  The second
run session runs the model for another 30 days.  
Setting the \vars{init\_with\_instance\_state} of
\vars{model} to \vars{True} tells the model to use the
the values of the instance attributes 
(for prognostic variables, right-hand sides, and start date) 
are currently stored \vars{model}
as the initial values for the run\_session.\footnote%
	{Unless overridden, by default, 
	\vars{init\_with\_instance\_state} is set
	to True on \class{Qtcm} instance instantiation.}
The \vars{cont}
keyword in the second \mods{run\_session} call specifies a
continuation run, and the value gives the number of additional
days to run the model.

The set of runs described above would produce the exact same
results as if you had gone into the Fortran model after 10 days,
doubled the first baroclinic mode zonal velocity, and continued
the run for another 30 days.  With the Python example above, however,
you didn't need to know you were going to do that ahead of starting
the model run (which is what a compiled model requires you to do).
Section~\ref{sec:contination.run.sessions} describes continuation
runs in detail.


	\subsection{Passing Restart Snapshots Between Run Sessions}
					\label{sec:snapshot.intro}

The pure-Fortran QTCM1 uses a restart file to enable continuation
runs.  A \class{Qtcm} instance can also make use of that option,
through setting the \vars{mrestart} attribute value
(see Section~\ref{sec:contination.run.sessions} and
Neelin et al.\ \cite{Neelin/etal:2002} for details).  
It's easier, however, instead of using a restart file, to pass 
along a ``snapshot'' dictionary.

The \class{Qtcm} instance method \mods{make\_snapshot} copies the
variables that would be written out to a restart file into a
dictionary that is saves as the instance attribute \vars{snapshot}.
This snapshot can be saved separately, for later recall.  Note that
snapshots are automatically made at the end of a run session.

The following example shows a model \mods{run\_session} call,
following which the snapshot is saved to the variable
\vars{snapshot}:\footnote%
	{Remember Python assignment defaults to assignment by
	reference, so in this example the variable \vars{mysnapshot}
	is a pointer to the \vars{model.snapshot} attribute.
	(However, note that \vars{model.snapshot} itself is not a
	reference, but a distinct copy of those variables; to do
	otherwise would result in a non-static snapshot.)
	If the \vars{model.snapshot} attribute is dereferenced,
	then \vars{mysnapshot} will become the sole pointer to the
	dictionary.}

\begin{codeblock}
\codeblockfont{%
model.run\_session() \\
mysnapshot = model.snapshot}
\end{codeblock}

After taking the snapshot, you might continue the run a while, and
then decide to return to the snapshot you saved.  To do so, use
the \mods{sync\_set\_py\_values\_to\_snapshot}
method to reset the model instance values to
\vars{mysnapshot} before your next run session:
\begin{codeblock}
\codeblockfont{%
model.sync\_set\_py\_values\_to\_snapshot(snapshot=mysnapshot) \\
model.init\_with\_instance\_state = True \\
model.run\_session()}
\end{codeblock}

See Section~\ref{sec:snapshots} for details regarding the use of
snapshots, as well as for a list of what variables are saved in
a snapshot.




%---------------------------------------------------------------------
\section{Creating Multiple Models}

	\subsection{Model Instances}

Creating a new QTCM1 model is as simple as creating another
\class{Qtcm} instance.
For instance, to instantiate two QTCM1
models, \vars{model1} and \vars{model2}, type the following:

\begin{codeblock}
\codeblockfont{%
from qtcm import Qtcm \\
model1 = Qtcm(compiled\_form='parts') \\
model2 = Qtcm(compiled\_form='parts')}
\end{codeblock}

\vars{model1} and \vars{model2} do \emph{not} share any variables
in common, including the extension modules holding the Fortran
code.  In creating the instances, a copy of the extension modules
are saved in temporary directories.


	\subsection{Passing Snapshots To Other Models}

The snapshots described in Section~\ref{sec:snapshot.intro}
can also be passed around to other model instances,
enabling you to easily branch a model run:

\begin{codeblock}
\codeblockfont{%
model.run\_session() \\
mysnapshot = model.snapshot \\
model1.sync\_set\_py\_values\_to\_snapshot(snapshot=mysnapshot) \\
model2.sync\_set\_py\_values\_to\_snapshot(snapshot=mysnapshot) \\
model1.run\_session() \\
model2.run\_session()}
\end{codeblock}

The state of \vars{model} after its run session is used to start
\vars{model1} and \vars{model2}.  This is an easy way to save time
in spinning-up multiple models.




%---------------------------------------------------------------------
\section{Run Lists}		\label{sec:runlist.intro}

This feature of \class{Qtcm} objects is what really gives 
\class{Qtcm} model instances their flexibility.
A run list is a list of strings and dictionaries that specify
what routines to run in order to execute a particular part of
the model.  Each element of the run list specifies the method
or subroutine to execute, and the order of the elements specifies
their execution order.

For instance, the standard run list for initializing the the
atmospheric portion of the model is named ``qtcminit'', and
equals the following list:

\begin{latexonly}
\begin{codeblock}
\codeblockfont{%
\parbox{46ex}{% This file is automatically generated by
% code_to_latex.py.  It lists the code of the qtcminit
% runlist.

['\_\_qtcm.wrapcall.wparinit', \\
 '\_\_qtcm.wrapcall.wbndinit', \\
 'varinit', \\
 \{'\_\_qtcm.wrapcall.wtimemanager': [1]\}, \\
 'atm\_physics1']}}
\end{codeblock}
\end{latexonly}

\begin{htmlonly}
\begin{quotation}
% This file is automatically generated by
% code_to_latex.py.  It lists the code of the qtcminit
% runlist.

['\_\_qtcm.wrapcall.wparinit', \\
 '\_\_qtcm.wrapcall.wbndinit', \\
 'varinit', \\
 \{'\_\_qtcm.wrapcall.wtimemanager': [1]\}, \\
 'atm\_physics1']
\end{quotation}
\end{htmlonly}

This list is stored as an entry in the \vars{runlists} dictionary
(with key \vars{'qtcminit'}).
\vars{runlists} is an attribute of a \class{Qtcm} instance.
Table~\ref{tab:stnd.runlists} lists all standard run lists.

When the run list element in the list is a string, the string gives the
name of the routine to execute.  The routine has no parameter
list.  The routine can be a
compiled QTCM1 model subroutine for which an interface has been
written (e.g., \mods{\_\_qtcm.wrapcall.wparinit}), 
a method of the of the Python model instance 
(e.g., \mods{varinit}), or another run list
(e.g., \vars{atm\_physics1}).

When the run list element is a 1-element dictionary, the key of
the dictionary element is the name of the routine, and the value
of the dictionary element is a list specifying input parameters
to be passed to the routine on call.  Thus, the element:
\begin{codeblock}
\codeblockfont{%
{\{'\_\_qtcm.wrapcall.wtimemanager': [1]\}}}
\end{codeblock}
calls the \mods{\_\_qtcm.wrapcall.wtimemanager} routine, passing in
one input parameter, which in this case is the value 1.

If you want to change the order of the run list, just change the
order of the list.  To add or remove routines to be executed, just
add and remove their names from the run list.
Python provides a number of methods to manipulate
lists (e.g., \mods{append}).  Since lists are dynamic data types
in Python, you do not have to do any recompiling to implement
the change.

The \vars{compiled\_form} attribute must be set to \vars{'parts'}
in the \class{Qtcm} instance in order to take advantage of the run
lists feature of the class.  Run lists are not available for
\vars{compiled\_form\thinspace=\thinspace'full'}, because subroutine
calls are hardwired in the compiled QTCM1 model Fortran code in
that case.




%---------------------------------------------------------------------
\section{Model Output}			\label{sec:output.intro}

	\subsection{NetCDF Output}

Model output is written to netCDF files in the directory
specified by the \class{Qtcm} instance attribute \vars{outdir}.
Mean values are written to an output file beginning with
\fn{qm\_}, and instantaneous values are written to an output
file beginning with \fn{qi\_}.

The frequency of mean output is controlled by \vars{ntout}, and the
frequency of instantaneous output is controlled by \vars{ntouti}.
\vars{ntout.value} gives the number of days over which to average
(and if equals \vars{-30}, monthly means are calculated).
\vars{ntouti.value} gives the frequency in days that instantaneous
values are output (monthly if it equals \vars{-30}).  (See
Section~\ref{sec:initial.variables} for a description of other
output-control variables, and see the QTCM1 manual \cite{Neelin/etal:2002}
for a detailed description of how these variables control output.)

Figure~\ref{fig:netcdf.read} gives an example of a block of code
to read netCDF output, where \vars{datafn} is the netCDF filename, and
\vars{id} is the string name of the field variable (e.g.,
\vars{'u1'}, \vars{'T1'}, etc.).
(Note that the netCDF identifier for field variables is the same as
the name in \class{Qtcm}, except for the variables given in
Table~\ref{tab:qtcm.netcdf.ids}.)

In the code in Figure~\ref{fig:netcdf.read},
the array value is read into \vars{data}, and the longitude values, 
latitude values, and time values are read into variables
\vars{lon}, \vars{lat}, and \vars{time}, respectively.
As netCDF files also hold metadata, a description and the units
of the variable given by \vars{id}, and each dimension, are read
into variables ending in \vars{\_name} and \vars{\_units},
respectively.


%--- Two versions, one for PDF, one for HTML:
\begin{latexonly}
\begin{figure}[htp]
\begin{codeblock}
\codeblockfont{%
import numpy as N \\
import Scientific as S \\ \\
fileobj = S.NetCDFFile(datafn, mode='r') \\ \\
data = N.array(fileobj.variables[id].getValue()) \\
data\_name = fileobj.variables[id].long\_name \\
data\_units = fileobj.variables[id].units \\ \\
lat = N.array(fileobj.variables['lat'].getValue()) \\
lat\_name = fileobj.variables['lat'].long\_name \\
lat\_units = fileobj.variables['lat'].units \\ \\
lon = N.array(fileobj.variables['lon'].getValue()) \\
lon\_name = fileobj.variables['lon'].long\_name \\
lon\_units = fileobj.variables['lon'].units \\ \\
time = N.array(fileobj.variables['time'].getValue()) \\
time\_name = fileobj.variables['time'].long\_name \\
time\_units = fileobj.variables['time'].units \\ \\
fileobj.close()}
\end{codeblock}

\caption{Example of Python code to read netCDF output.
	See text for description.}
\label{fig:netcdf.read}
\end{figure}
\end{latexonly}

\begin{htmlonly}
\label{fig:netcdf.read}
\begin{center}
\htmlfigcaption{%
	\codeblockfont{%
import numpy as N \\
import Scientific as S \\ \\
fileobj = S.NetCDFFile(datafn, mode='r') \\ \\
data = N.array(fileobj.variables[id].getValue()) \\
data\_name = fileobj.variables[id].long\_name \\
data\_units = fileobj.variables[id].units \\ \\
lat = N.array(fileobj.variables['lat'].getValue()) \\
lat\_name = fileobj.variables['lat'].long\_name \\
lat\_units = fileobj.variables['lat'].units \\ \\
lon = N.array(fileobj.variables['lon'].getValue()) \\
lon\_name = fileobj.variables['lon'].long\_name \\
lon\_units = fileobj.variables['lon'].units \\ \\
time = N.array(fileobj.variables['time'].getValue()) \\
time\_name = fileobj.variables['time'].long\_name \\
time\_units = fileobj.variables['time'].units \\ \\
fileobj.close()}
	}

\htmlfigcaption{Figure~\ref{fig:netcdf.read}:
	Example of Python code to read netCDF output.
	See text for description.}
\end{center}
\end{htmlonly}





\begin{table}[tp]
\begin{center}
\begin{tabular}{l|l}
\textbf{\class{Qtcm} Attribute Name} & \textbf{NetCDF Output Name} \\
\hline
\vars{'Qc'}                & \vars{'Prec'} \\
\vars{'FLWut'}             & \vars{'OLR'} \\
\vars{'STYPE'}             & \vars{'stype'}
\end{tabular}
\end{center}
\caption{NetCDF output names for \class{Qtcm} field variables that
	are different from the \class{Qtcm} and compiled QTCM1 model
	variable names.  The netCDF names are case-sensitive.}
\label{tab:qtcm.netcdf.ids}
\end{table}


\emphpara{NB:}  All netCDF array output is dimensioned (time, latitude,
longitude) when read into Python using the \mods{Scientific} package.
This differs from the way \class{Qtcm} saves field variables, which
follows Fortran convention (longitude, latitude).  Please be careful
when relating the two types of arrays.
Section~\ref{sec:field.var.shape} for a discussion of why there is
this discrepancy.


	\subsection{Visualization}	\label{sec:viz.intro}

The \mods{plotm} method of \class{Qtcm} instances creates line
plots or contour plots, as appropriate, of model output of
average fields of run session(s) associated with the instance.
Some examples, assuming \vars{model} is an instance of \class{Qtcm}
and has already executed a run session:
\begin{itemize}
\item \cmd{model.plotm('Qc', lat=1.875)}:
	A time vs.\ longitude contour
          plot is made for the full range of time and longitude,
          at the latitude 1.875 deg N, for mean precipitation.

\item \cmd{model.plotm('Qc', time=10)}:
	A latitude vs.\ longitude contour plot of precipitation
	is made for the full spatial domain at day 10 of the model run.

\item \cmd{model.plotm('Evap', lat=1.875, lon=[100,200])}:  A contour
	plot of time vs.\ longitude of evaporation is made for the
          longitude points between 100 and 200 degrees E, at the
          latitude 1.875 deg N.  

\item \cmd{model.plotm('cl1', lat=1.875, lon=[100,200], time=20)}:
          A deep cloud amount vs.\ longitude line plot is made for
          the longitude points between 100 and 200 degrees east,
          at the latitude 1.875 deg N, at day 20 of the model run.
\end{itemize}

In these examples, the number of days over which the mean is taken
equals \vars{model.ntout.value}.
Also, the \mods{plotm} method automatically takes into account the
\class{Qtcm}/netCDF variable differences described in
Table~\ref{tab:qtcm.netcdf.ids}.



%---------------------------------------------------------------------
\section{Documentation}

Section~\ref{sec:ver} gives the online locations of the
transparent copies of this manual.  
Model formulation is fully described in
Neelin \& Zeng \cite{Neelin/Zeng:2000} and model
results are described in Zeng et~al.\ \cite{Zeng/etal:2000}
(\cite{Neelin/Zeng:2000} is based upon v2.0 of QTCM1
and \cite{Zeng/etal:2000} is based on QTCM1 v2.1).
Additional documentation you'll find useful include:

\begin{itemize}
\item \latexhtml{%
\htmladdnormallinkfoot{The \mods{qtcm} Package API Documentation}%
        {http://www.johnny-lin.com/py\_pkgs/qtcm/doc/html-api/}}%
{\htmladdnormallink{The \mods{qtcm} Package API Documentation}%
        {http://www.johnny-lin.com/py_pkgs/qtcm/doc/html-api/}}

\item \latexhtml{%
\htmladdnormallinkfoot{The Pure-Fortran QTCM1 Manual}%
        {http://www.atmos.ucla.edu/$\sim$csi/qtcm\_man/v2.3/qtcm\_manv2.3.pdf}}%
{\htmladdnormallink{The Pure-Fortran QTCM1 Manual}%
        {http://www.atmos.ucla.edu/~csi/qtcm_man/v2.3/qtcm_manv2.3.pdf}}
\cite{Neelin/etal:2002}

\end{itemize}



% ===== end of file =====


\chapter{Using \mods{qtcm}}                 \label{ch:using}
% ==========================================================================
% Using QTCM
%
% By Johnny Lin
% ==========================================================================


% ------ BODY -----
%
%---------------------------------------------------------------------
\section{Introduction}

Now that you've successfully run your first model instances, in
this chapter I provide detailed explanations regarding the features
of \mods{qtcm}.  I present these explanations in a documentary
rather than didactic fashion; my goal is to document how the features
work.  More details are given in the code docstrings.  At the end
of the chapter, in Section~\ref{sec:cookbook}, I provide a few
cookbook ideas/examples of ways to use the model.




%---------------------------------------------------------------------
\section{Model Instances}  \label{sec:model.instances}

An instance of a \class{Qtcm} model is created in \mods{qtcm} the same way
you create an instance of any class.
For instance, to instantiate two \class{Qtcm}
models, \vars{model1} and \vars{model2}, I type the following:

\begin{codeblock}
\codeblockfont{%
from qtcm import Qtcm \\
model1 = Qtcm(compiled\_form\thinspace=\thinspace'full') \\
model2 = Qtcm(compiled\_form\thinspace=\thinspace'parts')}
\end{codeblock}

In the above example, \vars{model1} uses the compiled QTCM1 model
that runs the model (essentially) using the Fortran driver,
while \vars{model2} uses the compiled QTCM1 model where execution
order and content all the way down to the atmospheric timestep level
is controlled by Python run lists.  (Section~\ref{sec:compiledform}
has more details about the difference between compiled forms.)

For each instance of \class{Qtcm}, copies of all needed extension
modules (e.g., \fn{.so} files) are copied to a temporary directory
that is automatically created by Python.  The full path name of
that directory is saved in the instance attribute \vars{sodir}.
These extension modules are then associated with the specific instance 
through private instance attributes,
and thus every instance of \class{Qtcm} has its own separate variable
and name space on both the Fortran and Python sides.\footnote%
	{The private instance attribute is \vars{\_\_qtcm}.
	See Section~\ref{sec:Qtcm.private.attrib} for details about 
	private \class{Qtcm} instance attributes.}
The temporary directory and all of its contents are deleted when the 
model instance is deleted.

On instantiation, \class{Qtcm} instances set all scalar field
variables to their default values as given in the submodule
\mods{defaults} (and listed in Section~\ref{sec:defaults.scalar}),
and assign the fields as instance attributes.  The instance attribute
\vars{init\_with\_instance\_state} is set to True by default, unless
overridden on instantiation.




%---------------------------------------------------------------------
\section{Initializing a Model Run}

In the pure-Fortran QTCM1, there are three broad
classes of initialized variables:
\begin{enumerate}
\item Those that are read-in using a namelist, 
\item Those that the are read-in from a restart file, and
\item Those that are set by assignment in the Fortran code.  
\end{enumerate}
These variables are a combination of scalars and arrays.

For \mods{qtcm}, interfaces were built so that all classes of
initialized variables that could be user-controlled are accessible
and changeable at the Python-level.  For \mods{qtcm},
the set of variables that could be changed is also expanded, to
include not just the first and second classes of pure-Fortran
QTCM1 initialized variables.  This was done to make \mods{qtcm}
more flexible.  All variables that can be passed between the
compiled QTCM1 model and Python model levels are called
field variables, and are described in detail in
Section~\ref{sec:field.variables}.

As it happens, all the namelist-set variables are scalars.  In the
pure-Fortran QTCM1, those variables are given default values prior
to reading in of the namelist.  To duplicate this functionality,
on model instantiation, all scalar fields are set to their default
values as given in the submodule \mods{defaults} and listed in
Section~\ref{sec:defaults.scalar}.  Most of the default values in
\mods{defaults} are the same as in the pure-Fortran QTCM1, but
there are a few differences.\footnote%
	{One difference being \vars{mrestart}, which in \vars{qtcm} 
	will have the value of 0 in contrast to the pure-Fortran 
	QTCM1 where the default is the 1.}
This setting of scalar defaults is the same for both
\vars{compiled\_form\thinspace=\thinspace'full'} and
\vars{compiled\_form\thinspace=\thinspace'parts'} instances.
Of course, all
\mods{qtcm} fields are user-controllable, both via keyword input
parameters at model instantiation as well as through direct
manipulation of the instance attribute that stores the field variable.

The pure-Fortran QTCM1 initialized prognostic variables and
right-hand sides are set in the Fortran subroutine \mods{varinit}.
Their they are read-in from a restart file or, as default,
set by assignment.
In \mods{qtcm}, the same variables are initialized by a \class{Qtcm}
instance method of the same name, \mods{varinit}, for the case when
\vars{compiled\_form\thinspace=\thinspace'parts'}.  For the case
of \vars{compiled\_form\thinspace=\thinspace'full'}, the compiled
QTCM1 subroutine that is the same as in the pure-Fortran case is
used, and that routine is inaccessible at the Python level.
See Section~\ref{sec:snapshots}'s listing of snapshot variables,
which also includes the prognostic variables and right-hand sides that
are set in \mods{varinit} (both Fortran and Python).




%---------------------------------------------------------------------
\section{The \vars{compiled\_form} Keyword}  \label{sec:compiledform}

The \mods{qtcm} package is a Python wrap of the Fortran routines
that make up QTCM1.  The wrapping layer adds flexibility and
functionality, but at the cost of speed.  Thus, I created two
types of extension modules from the Fortran QTCM1 code, one
which permits very little control over the compiled Fortran
\emph{routines} at the Python level, and one that allows the Python-level
to control model execution in the compiled QTCM1 model
all the way down to the atmospheric timestep level.\footnote%
	{That control is via run lists, which are described in
	Section~\ref{sec:runlists}.}
The former extension module corresponds to 
\vars{compiled\_form\thinspace=\thinspace'full'} and
the latter extension module to
\vars{compiled\_form\thinspace=\thinspace'parts'}.

For \vars{compiled\_form\thinspace=\thinspace'full'},
the compiled portion of the model encompasses (nearly) the
entire QTCM1 model as a whole.  Thus, the only compiled QTCM1 model
modules or subroutines that Python should interact with is
the \mods{driver} routine (which executes the entire model) and
the \mods{setbypy} module (which enables communication between the
compiled model and the Python-level of model fields.\footnote%
	{The \mods{setbypy} Python module is the wrap of the
	Fortran QTCM1 \mods{SetByPy} module.}

For \vars{compiled\_form\thinspace=\thinspace'parts'}, the compiled
portion of the model does not encompasses the model as a whole, but
rather is broken up into separate units (as appropriate) all the
way down to an atmosphere timestep.  Thus, compiled QTCM1 model
modules/subroutines that are accessible at the Python-level include
those that are executed within an atmosphere timestep on up.

Because the difference in compiled forms fundamentally affects how
the \class{Qtcm} instance facilitates Python-Fortran communication,
this attribute must be set on instantiation via a keyword input
parameter.

In the rest of this section, to avoid being verbose, when I
write \vars{'full'}, I mean the situation where
\vars{compiled\_form\thinspace=\thinspace'full'}.
Likewise, when I
write \vars{'parts'}, I mean the situation where
\vars{compiled\_form\thinspace=\thinspace'parts'}.


	\subsection{Initialization for 
			\vars{compiled\_form\thinspace=\thinspace'full'}}
				\label{sec:init.compiledform.full}

For a model run of this case, the \class{Qtcm} instance will
initialize the model using the Fortran \mods{varinit} subroutine
in the compiled QTCM1 model.  This subroutine does the following:

\begin{itemize}
\item If \vars{mrestart\thinspace=\thinspace1}, 
	the restart file is used to initialize all prognostic
	variables.  In terms of start date, the following rules are
	used:
	\begin{enumerate}
	\item Variable \vars{dateofmodel} is read from the restart file.
	\item If \vars{day0}, \vars{month0}, and \vars{year0}
		are negative, or otherwise
		invalid (e.g., \vars{month0} greater than 12), the invalid
		value is replaced with the
		day, month, and/or year of the day \emph{after} 
		that given by \vars{dateofmodel}.
		If the value of \vars{day0}, \vars{month0}, or \vars{year0}
		is not invalid in this sense, it is not replaced.
	\end{enumerate}
	Thus, if the restart file gives 
	\vars{dateofmodel} equal to 101102
	(year 10, month 11, day 2), and 
	\vars{day0\thinspace=\thinspace-1}, 
	\vars{month0\thinspace=\thinspace-1}, 
	\vars{year0\thinspace=\thinspace-1},
	and 
	\vars{mrestart\thinspace=\thinspace1}, 
	the model will start running from year 10, month 11, day 3.
	If \vars{dateofmodel} equals to 101102, and 
	\vars{day0\thinspace=\thinspace-1}, 
	\vars{month0\thinspace=\thinspace3}, 
	\vars{year0\thinspace=\thinspace-1},
	the model will start running from year 10, month 3, day 3.

\item If \vars{mrestart\thinspace=\thinspace0}, 
	all prognostic variables and right-hand sides are set to an
	initial value (which for most of those variables is zero).
	In terms of start date, \vars{day0} is set to 1 (and thus 
	the value of \vars{day0} previously input is ignored), and
	both \vars{month0} and \vars{year0}
	are set to 1 
	if their previously input values are invalid (where
	invalid means less than
	1, or, for \vars{month0}, greater than 12).
	Otherwise, \vars{month0} and \vars{year0} are left unchanged.
	Variable \vars{dateofmodel} has the value it had when the variable
	was declared (which is determined by the compiler and usually
	is zero; \vars{dateofmodel} will not be properly set until
	subroutine \mods{TimeManager} is called.

	Thus, if 
	\vars{day0\thinspace=\thinspace-1},
	\vars{month0\thinspace=\thinspace-1}, 
	\vars{year0\thinspace=\thinspace-1} is input into the model
	(say from a namelist) and 
	\vars{mrestart\thinspace=\thinspace0},
	the model will start running from year 1, month 1, day 1,
	and \vars{dateofmodel} at the exit of subroutine 
	\mods{varinit} will equal its compiler-set default.
	If 
	\vars{day0\thinspace=\thinspace14}, 
	\vars{month0\thinspace=\thinspace3}, 
	\vars{year0\thinspace=\thinspace11}, and 
	\vars{mrestart\thinspace=\thinspace0} on input into the
	model,
	the model will start running from year 11, month 3, day 1,
	and \vars{dateofmodel} at the exit of subroutine 
	\mods{varinit} will equal its compiler-set default.

	Note that \vars{dateofmodel}
	can thus be inconsistent with 
	\vars{month0} and \vars{year0} at the
	exit of subroutine \mods{varinit}.
\end{itemize}

This behavior with respect to initializing
the start date is different than in QTCM1 versions 1.0 and 2.1.
Please see the source code from those earlier QTCM1 versions for
details.




	\subsection{Initialization for 
			\vars{compiled\_form\thinspace=\thinspace'parts'}}
				\label{sec:init.compiledform.parts}

For \vars{'parts'} model, the methodology of how initialized
prognostic variables, right-hand sides, and start date related
variables are set is controlled by the \class{Qtcm} instance
attribute/flag \vars{init\_with\_instance\_state}.  The initialization
is (mostly) executed in the Python \vars{varinit} method in the
following way:

\begin{itemize}
\item If \vars{init\_with\_instance\_state} is False:
The method as described for
initialization for the 
\vars{'full'} case is generally
followed, with the exception that dateofmodel is set
to match \vars{day0}, \vars{month0}, \vars{year0}, prior to exit of 
\mods{varinit}.

\item If \vars{init\_with\_instance\_state} is True:
the model object will initialize the model based on the current
state of the model instance.  This enables you to set a model run
session's initial conditions based upon the state of the prognostic
variables and parameters stored at the Python level, which is
accessible at runtime.
\end{itemize}


Since the \vars{init\_with\_instance\_state\thinspace=\thinspace{False}}
case is mainly described by the initialization method for the
\vars{'full'} case, I refer the
reader to Section~\ref{sec:init.compiledform.full}.
For the case of \vars{init\_with\_instance\_state} is True, however,
the task is more complicated.  Specifically, for that case,
initialization includes the following:

\begin{enumerate}
\item If not currently defined,
	variable \vars{dateofmodel} is set to a default value of 0,
	which is specified in the module defaults.

\item The \vars{mrestart} flag is ignored for variable initialization.

\item All prognostic variables and right-hand sides
        are set to an
        initial value (which for most of those variables is zero),
	unless the variable is defined at the Python level, in which
	case the inital value is set to the Python level defined value.

\item If \vars{dateofmodel} is greater than 0, 
	\vars{day0}, \vars{month0}, and \vars{year0} are overwritten
        with values derived from \vars{dateofmodel} 
	in order to set the run to start
	the day \emph{after} \vars{dateofmodel}.

\item If \vars{dateofmodel} is less than or equal to 0, \vars{day0},
	\vars{month0}, and \vars{year0} are set to their respective
	instance attribute values, if valid.  For invalid instance
	attribute values, the invalid \vars{day0}, \vars{month0},
	and/or \vars{year0} is set to 1.

\item Variable \vars{dateofmodel} is recalculated
	and overwritten to match 
	\vars{day0}, \vars{month0}, \vars{year0}, prior to exit of 
	\mods{varinit}.
\end{enumerate}

As a result, for \vars{init\_with\_instance\_state} is True, the
way you indicate to the model that a run session is a brand-new run
is by setting, before the \mods{run\_session} method call,
\vars{dateofmodel} to a value less than or equal to 0, and \vars{day0},
\vars{model0}, and \vars{year0} to the day you want the model to
begin the run session.  To indicate to the model you wish to continue
a run, set \vars{dateofmodel} to the day \emph{before} you want the
model to start running from.

Examples:
\begin{itemize}
\item If \vars{day0\thinspace=\thinspace-1}, 
	\vars{month0\thinspace=\thinspace-1}, 
	\vars{year0\thinspace=\thinspace-1}, and
	\vars{dateofmodel\thinspace=\thinspace0} is input into 
	the model the model will start running from year 1, month 1, day 1,
	and 
	variable \vars{dateofmodel} at the exit of 
	subroutine \mods{varinit}
	will equal 10101.

\item If \vars{day0\thinspace=\thinspace14},
	\vars{month0\thinspace=\thinspace3}, 
	\vars{year0\thinspace=\thinspace11},
	and \vars{dateofmodel\thinspace=\thinspace0} is input into the
	model, the model will start running from year 11, month 3, day 14,
	and 
	variable \vars{dateofmodel} at the exit of 
	subroutine \mods{varinit} will equal
	110314.

\item If \vars{day0\thinspace=\thinspace14},
	\vars{month0\thinspace=\thinspace3}, 
	\vars{year0\thinspace=\thinspace11},
	and \vars{dateofmodel\thinspace=\thinspace341023} is input into the
	model, the model will start running from year 34, month 10, day 24,
	and at the exit of subroutine 
	\mods{varinit}, \vars{dateofmodel} will equal
	341024, with \vars{day0\thinspace=\thinspace24},
	\vars{month0\thinspace=\thinspace10}, and
	\vars{year0\thinspace=\thinspace34}.
\end{itemize}


	\subsection{Communication Between Python and Fortran-Levels}
				\label{sec:comm.py.fort.compiledform}

After initialization, the second major difference between a
\vars{'full'} and \vars{'parts'} model is how and when communication
between the Python and Fortran levels can occur.  For the \vars{'full'}
case, except for the passing in and out of variables before and after
a run session, all variable passing and subroutine calling happens in
the compiled QTCM1 model, with no control at the Python level.
For the \vars{'parts'} case, variables can be passed between the
Python and Fortran-levels at all levels down to the atmospheric
timestep, and many Fortran QTCM1 subroutines can be called from the
Python-level.  


		\subsubsection{Passing Variables}

For all \vars{compiled\_form} cases, variables are passed back and
forth between the Python \class{Qtcm} instance level and the
compiled QTCM1 model Fortran-level using the \class{Qtcm}
instance methods \mods{get\_qtcm1\_item} and \mods{set\_qtcm1\_item}:\footnote%
	{All Fortran routines used to pass variables back and forth are
	defined in the \mods{setbypy} module of the \fn{.so} extension
	module stored in the \class{Qtcm} instance variable \vars{\_\_qtcm}.
	All Fortran wrappers that enable Python to call compiled QTCM1 model
	subroutines are defined in the \mods{wrapcall} module stored in
	the \class{Qtcm} instance variable \vars{\_\_qtcm}.
	These modules are described in detail in 
	Sections~\ref{sec:setbypy} and~\ref{sec:wrapcall}, respectively.}

\begin{itemize}
\item \mods{get\_qtcm1\_item}(\dumarg{key}):
	Returns the value of the field variable given by the string
	\dumarg{key}.  If the compiled QTCM1 model variable given by
	\dumarg{key} is unreadable, the
        custom exception 
	\vars{FieldNotReadableFromCompiledModel} is thrown.
	The value returned is a copy of the value on the Fortran
	side, not a reference to the variable in memory.

\item \mods{set\_qtcm1\_item}:
	Sets the value of a field variable
	in the compiled QTCM1 model \emph{and at the Python-level,}
	automatically overriding any previous value at both levels.
	Thus, calling this method will change/create the \class{Qtcm}
	instance attribute corresponding to the field variable.
        When the compiled QTCM1 model variable is set, a copy of the
        Python value is passed to the Fortran model; the
	variable is \emph{not passed by reference.}
	This value comes from the \mods{set\_qtcm1\_item} calling
	parameter list, \emph{not} from the \class{Qtcm}
        instance attribute corresponding to the field variable.
\end{itemize}

The \mods{set\_qtcm1\_item} method has two calling forms, one with
one argument and the other with two arguments:
\begin{itemize}
\item One argument:  The method is called
	as \mods{set\_qtcm1\_item}(\dumarg{arg}), where \dumarg{arg} 
	is either a string giving the name of the field variable or 
	a \class{Field} instance.

\item Two arguments:  The method is called as
	\mods{set\_qtcm1\_item}(\dumarg{key}, \dumarg{value}), where
	\dumarg{key} is the string giving the name of the field variable
	and \dumarg{value} is the value to set the model field variable to
	(note \dumarg{value} can be a \class{Field} instance).
\end{itemize}
In either calling form, if no value given, the default value as defined
in module \mods{defaults} is used.

Some compiled QTCM1 model variables are not in a state where they
can be set.  An example is a compiled QTCM1 model pointer variable,
prior to the pointer being associated with a target (an attempt
to set would yield a bus error).  In such cases, the
\mods{set\_qtcm1\_item} method will throw a
\vars{FieldNotReadableFromCompiledModel} exception, nothing will
be set in the compiled QTCM1 model, and the Python counterpart
field variable (if it previously existed) would be left unchanged.\footnote%
	{We handle this situation in this way to enable the
	\class{Qtcm} instance to store variables
	even if the compiled model is not yet ready to accept them.}

Examples, typed in at a Python prompt, and
assuming that \vars{model} is a \class{Qtcm} instance:
\begin{itemize}
\item \cmd{dtvalue\thinspace=\thinspace{model.get\_qtcm1\_item('dt')}}:
	Retrieves the value of field variable \vars{dt} (timestep)
	from the compiled QTCM1 Fortran model and sets it to the
	Python variable \vars{dtvalue}.

\item \cmd{model.set\_qtcm1\_item('dt')}:
	Sets the value of field variable \vars{dt}
	in the compiled QTCM1 Fortran model to the default
	value (as given in \mods{defaults}),
	and sets the value of Python attribute \vars{model.dt} also to 
	that default value.  
	Remember that \vars{model.dt} is a \class{Field}
	instance.

\item \cmd{model.set\_qtcm1\_item('dt', 2000.)}:
	Sets the value of field variable \vars{dt}
	in the compiled QTCM1 Fortran model to 2000 (as a real),
	and sets the value of Python attribute \vars{model.dt} also to 2000.
\end{itemize}


		\subsubsection{Calling Compiled QTCM1 Model Subroutines}

All compiled QTCM1 model subroutines that can be called
(except \mods{driver} and \mods{varptrinit}) are in the
\mods{setbypy} or \mods{wrapcall} modules
of the \class{Qtcm} instance private attribute \vars{\_\_qtcm}.
(On \class{Qtcm} instance instantiation, \vars{\_\_qtcm} is set
to the \fn{.so} extension module that is the compiled QTCM1 Fortran model.)
Thus, to call \mods{wmconvct} in \mods{wrapcall} at the Python-level,
just type \cmd{model.\_\_qtcm.wrapcall.wmconvct()} (where \vars{model}
is a \class{Qtcm} instance).
For \mods{driver} and \mods{varptrinit}, these subroutines are not
contained in a \vars{\_\_qtcm} module, and thus can be called
directly (e.g., just type \cmd{model.\_\_qtcm.driver()}).
See Sections~\ref{sec:setbypy} and~\ref{sec:wrapcall} for more information
on the \mods{setbypy} and \mods{wrapcall} modules.

For the \vars{'full'} case, the only compiled QTCM1 model
subroutine you can usefully call during a run session is \mods{driver}.
For the \vars{'parts'} case, while you can essentially call any subroutine
given in a run list, you usually will not directly call a compiled QTCM1
model subroutine but will instead call it through including it in a
run list.  For example, if you have the following run list in a
\vars{'parts'} model:
\begin{codeblock}
\codeblockfont{%
[ 'qtcminit', '\_\_qtcm.wrapcall.woutpinit' ]}
\end{codeblock}
Running this list using the \class{Qtcm} instance method
\mods{run\_list} will result in \class{Qtcm} instance method
\mods{qtcminit} first being run, 
then the compiled QTCM1 Fortran model subroutine
\mods{woutpinit} in Fortran module \mods{wrapcall} being run.
See Section~\ref{sec:runlists} and
Table~\ref{tab:stnd.runlists} for a discussion and list of the
standard run lists that control routine execution content and order
in the \vars{'parts'} case.




%---------------------------------------------------------------------
\section{Restart and Continuation Run Sessions}
				\label{sec:contination.run.sessions}


	\subsection{Restart Runs In the Pure-Fortran QTCM1}
					\label{sec:puref90.restart}

To enable restart of a model run, the pure-Fortran QTCM1 model
writes out a restart file with the state of the prognostic variables
and select right-hand sides at that point in the run (for a list
of the variables, see Section~\ref{sec:snapshots}).  This binary
file can then be read in by later model runs.  The Fortran
\vars{mrestart} flag is passed in via a namelist; if \vars{mrestart}
is 1, the run uses the restart file (named \fn{qtcm.restart}).

One of the problems with using the restart file to do a continuation
run is that the continuation run will not be perfect.  In other words,
a 15~day run followed by a 25~day run based on the restart file 
generated at the end of the 15~day run will \emph{not} give the
exact same output as a continuous 40~day run.


	\subsection{Overview of Restart/Continuation Options In \mods{qtcm}}
					\label{sec:restart.options.list}

For a \class{Qtcm} instance, in contrast to the pure-Fortran QTCM1,
more than one method of continuation is available.
Thus, for a continuation run, you need to tell the model
``continue from what?''
The \class{Qtcm} class provides three choices for restart/continuing
a run:
\begin{enumerate} 
\item From a restart file:  Move/rename a QTCM1 restart file
        to the current working directory to \fn{qtcm.restart}.
	\label{list:continue.from.restart}

\item From a snapshot from another run session
	(see Sections~\ref{sec:snapshot.intro} and~\ref{sec:snapshots}).
	\label{list:continue.from.snapshot}

\item From the values of the \class{Qtcm} instance you will be
	calling \mods{run\_session} from.
	\label{list:continue.from.instance}
\end{enumerate}

Restart/continuation methods~\ref{list:continue.from.restart} 
and~\ref{list:continue.from.snapshot} both suffer from the
same problem as the pure-Fortran QTCM1 restart process:
They do not produce perfect restarts
(see Section~{sec:puref90.restart} for details).
In this section, I discuss the restart/continuation options
for each \vars{compiled\_form} option.

Methods~\ref{list:continue.from.restart}
and~\ref{list:continue.from.snapshot} are best used when making a
run session from a newly instantiated \class{Qtcm} instance.
Method~\ref{list:continue.from.instance} is best used when executing
a run session using a \class{Qtcm} instance that has already gone
through at least one run session.  Regardless of which method you
use, however, please note that anytime you execute a run session
using a \class{Qtcm} instance that already has made a previous run
session, some variables \emph{cannot be updated} between run sessions.
This feature is most noticeable with the output filename, and occurs
because the name persists in the compiled QTCM model, and is stored
in the extension module (\fn{.so} files in \vars{sodir}) associated
with the instance.  If you wish to control all variables possible
from the Python level (including output filename), you need do the
run session from a new model instance.


	\subsection{Restart/Continuation for 
		\vars{compiled\_form\thinspace=\thinspace'full'} 
		Model Instances}

The only option for restart when using
\vars{compiled\_form\thinspace=\thinspace'full'} model instances
is method~\ref{list:continue.from.restart}, to use a QTCM1 restart
file.\footnote%
	{The \vars{cont} keyword parameter in \mods{run\_session}
	and the value of the \vars{init\_with\_instance\_state}
	attribute have no effect if
	\vars{compiled\_form\thinspace=\thinspace'full'}.  With
	\vars{'full'}, the call to initialize variables all happens
	at the Fortran level (via the Fortran \mods{varinit}, not
	the Python \mods{varinit}), with no reference to the Python field
	states (or even existing Fortran field states, if present).}
To use this option, the value of the \vars{mrestart} 
attribute must equal 1, the restart file must be named
\fn{qtcm.restart}, and the restart file must be in the 
current working directory.
As with the pure-Fortran QTCM1 restart process, this method
does not produce perfect restarts.



	\subsection{Restart/Continuation for 
		\vars{compiled\_form\thinspace=\thinspace'parts'} 
		Model Instances}

For the \vars{compiled\_form\thinspace=\thinspace'parts'} case,
all three restart/continuation methods
described in Section~\ref{sec:restart.options.list} are
available.


		\subsubsection{Method~\ref{list:continue.from.restart}:
			From a QTCM1 Restart File}

To use the QTCM1 restart file mechanism, not only must the
\vars{mrestart} attribute have a value to 1, but the
\vars{init\_with\_instance\_state} flag also has to be \vars{False},
otherwise the \vars{mrestart} attribute value will be ignored.  
As with the pure-Fortran QTCM1 restart process, this method does not
produce perfect restarts.


		\subsubsection{Method~\ref{list:continue.from.snapshot}:
			From a \class{Qtcm} Instance Snapshot}

You can take snapshots of the model state of a \class{Qtcm} instance
by the \mods{make\_snapshot} instance method.  This snapshot saves
a copy of all the variables saved to a QTCM1 restart file (see
Section~\ref{sec:snapshots} for the full list of fields), which
then can be passed to other \class{Qtcm} instances for use in other
run sessions.

The key difference between this method and 
method~\ref{list:continue.from.instance} (described below)
is that \mods{run\_session} calls using the snapshot are done
\emph{without} the \vars{cont} keyword input parameter
(by default, \vars{cont} is False).  If the \vars{cont} keyword
is not False, it says the run session is a continuation run
that uses the state of the compiled QTCM1 model for all variables
that are not specified at, and read-in from,
the Python level.  If the \vars{cont} keyword
is False, the run session initializes as if it were a new run.

See Section~\ref{sec:snapshot.intro} for details and
an example of using snapshots to initialize a run session.
Note that as with the pure-Fortran QTCM1 restart process, this method 
does not produce perfect restarts.


		\subsubsection{Method~\ref{list:continue.from.instance}:
			From the Calling \class{Qtcm} Instance}

This method is used when you want to make a run session that is a
``true'' continuation run, i.e., one that uses the current state
of the compiled QTCM1 model for all variables that are not read-in
from the Python level (remember that \class{Qtcm} instances hold a
subset of the variables defined at the Fortran level).  
The key reason to use this method for a continuation run session
is that the continuation is byte-for-byte the same (if no fields
are changed) as if the run just went straight on through.  Thus,
the continuation would be perfect: A 15~day run followed by a 25~day
run using the same \class{Qtcm} instance with the \vars{cont} keyword
will give the exact same output as a continuous 40~day run.  This
is not the case when making a new instance and passing a restart
file or a snapshot, because a separate extension module is used for
those new instances.

Control of this method is accomplished through the \vars{cont}
keyword input parameter to the \mods{run\_session} method and the
\vars{init\_with\_instance\_state} attribute of a
\class{Qtcm} instance:

\begin{itemize}
\item \vars{cont}: If set to False, the run session is not a
	continuation of the previous run, but a new run session.
	If set to True, the run session is a continuation of the
	previous run session.  If set to an integer greater than
	zero, the run session is a continuation just like
	\vars{cont\thinspace=\thinspace{True}}, but the value
	\vars{cont} is set to is used for \vars{lastday} and replaces
	\vars{lastday.value} in the \class{Qtcm} instance.

\item \vars{init\_with\_instance\_state}:
	If True, for a \mods{run\_session} call using the
	\vars{cont} keyword, whatever the field values are in the Python
	instance are used in the run session.
	If False, model variables are set and initialized as described in
	Section~\ref{sec:init.compiledform.parts}.  In that case,
	previous compiled QTCM1 model values will likely be overwritten.
	Thus, if you want a continuation run that uses the state of
	all field variables except for those you explicitly change at
	the Python-level, make sure \vars{init\_with\_instance\_state}
	is True.
\end{itemize}

(Note that the \vars{cont} keyword has no effect if \vars{compiled\_form}
is \vars{'full'}.  The default value of \vars{cont} in a
\mods{run\_session} call is False.  The value of keyword \vars{cont}
is stored as private instance attribute \vars{\_cont}, in case you
really need to access it elsewhere; see
Section~\ref{sec:Qtcm.private.attrib} for more details).

The example described in Section~\ref{sec:continuation.intro} is
an example of method~\ref{list:continue.from.instance} in the list
above: The second run session is continued from the state of
\vars{model}, with the values of \vars{model}'s instance variables
overriding any values in the compiled QTCM1 model in initializing
the second run session.

This method has a few caveats worthy of note:
\begin{itemize}
\item The \vars{init\_with\_instance\_state} attribute value
	will have no effect unless the instance prognostic variables
	are set, i.e., unless a previous run session has been done.
	Another way to put it is for an initial run session right
	after a \class{Qtcm} instance is created, \mods{varinit}
	will use the same initial values for prognostic variables
	(defined in \mods{defaults} module variable
	\vars{init\_prognostic\_dict})\footnote%
		{\vars{init\_prognostic\_dict} is the dictionary giving
		the default initial values of each prognostic variable
		and right-hand side (as defined by the restart file 
		specification).}
	as it would with for both
	\vars{init\_with\_instance\_state} set to True or False).

\item Continuation run sessions using this method have to continue
	with the next day from wherever the last run session left
	off, contiguously.\footnote%
		{For continuation run sessions, you keep the 
		same extension module (the compiled \fn{.so} library),
		and all the values that define the state where it
		left off.}
	If you want to do a non-contiguous run,
	create a new \class{Qtcm} instance initialized with a
	snapshot instead of the continuation method describe in
	this section.
	will use restart rules to run a new model.  

\item When making a continuation run session using this method,
	you cannot change some variables, for instance,
	\vars{outdir} and any of the date related
	variables.  In fact, the only thing you should change for
	your continuation run session are the prognostic and
	diagnostic variables and \vars{lastday}.  This is because
	some variables cannot be updated between run sessions.
	As noted in Section~\ref{sec:restart.options.list},
	if you wish to control all variables possible
	from the Python level (including output filename), you need 
	to execute the run session from a new model instance.
\end{itemize}


	\subsection{Snapshots of a \class{Qtcm} Instance}
				\label{sec:snapshots}

The snapshot dictionary (briefly described in
Section~\ref{sec:snapshot.intro}), saved as the \class{Qtcm} instance
attribute \vars{snapshot}, and generated by the method
\mods{make\_snapshot}, saves the current state of the following
instance field variables:

\begin{center}
% This file is automatically generated by
% code_to_latex.py.  It lists all the snapshot variables.


\begin{longtable}{l|c|c|p{0.4\linewidth}}
\textbf{Field} & \textbf{Shape} &
                                \textbf{Units} & \textbf{Description} \\
\hline
\endhead
    \vars{T1} & (64, 44) & K &  \\
\vars{Ts} & (64, 42) & K & Surface temperature \\
\vars{WD} & (64, 42) &  &  \\
\vars{dateofmodel} &      &  & Date of model coded as an integer as yyyymmdd \\
\vars{psi0} & (64, 43) &  &  \\
\vars{q1} & (64, 44) & K &  \\
\vars{rhsu0bar} & (3,) &  &  \\
\vars{rhsvort0} & (64, 42, 3) &  &  \\
\vars{title} &      &  & A descriptive title \\
\vars{u0} & (64, 44) & m/s & Barotropic zonal wind \\
\vars{u0bar} &      &  &  \\
\vars{u1} & (64, 44) & m/s & Current time step baroclinic zonal wind \\
\vars{v0} & (64, 43) & m/s & Barotropic meridional wind \\
\vars{v1} & (64, 43) & m/s &  \\
\vars{vort0} & (64, 42) &  &  \\
\end{longtable}
\end{center}

These are the same variables saved to a QTCM1 restart file, and so
a snapshot duplicates the restart functionality in the Python
environment, but with more flexibility.  Since the \vars{snapshot}
dictionary is a Python variable like any other, you can manipulate
it and alter it to fit any condition you wish.




%---------------------------------------------------------------------
\section{Creating and Using Run Lists}  \label{sec:runlists}

Section~\ref{sec:runlist.intro} provides an introduction to the
role and use of run lists.  A run list is a list of methods, Fortran
subroutines, and other run lists that can be executed by the
\class{Qtcm} instance \mods{run\_list} method.  Run lists are stored
in the \class{Qtcm} instance attribute \vars{runlists}, which is a
dictionary of run lists.  The names of run lists should not be
preceeded by two underscores (though elements of a run list may be
very private variables), nor should names of run lists be the same
as any instance attribute.  Run lists are not available for
\vars{compiled\_form\thinspace=\thinspace'full'}.

The \mods{run\_list} method takes a single input parameter, a list,
and runs through that list of elements that specify other run lists
or instance method names to execute.  Methods with private attribute
names are automatically mangled as needed to become executable by
the method.  Note that if an item in the input run list is an
instance method, it should be the entire name (not including the
instance name) of the callable method, separated by periods as
appropriate.

Elements in a run list are either strings or 1-element dictionaries.
Consider the following example, where \vars{model} is a \class{Qtcm}
instance, and \mods{run\_list} is called using \vars{mylist} as
input:

\begin{codeblock}
\codeblockfont{%
model = Qtcm(\ldots) \\
mylist = [ \{'varinit':None\}, \\
\hspace*{13ex}'init\_model', \\
\hspace*{13ex}'\_\_qtcm.driver', \\
\hspace*{13ex}\{'set\_qtcm1\_item': ['outdir', '/home/jlin']\} ]
model.run\_list(mylist)}
\end{codeblock}

The first element in \vars{mylist} refers to a method that requires
no positional input parameters be passed in (as shown by the None).
The second and third elements in \vars{mylist} also refers to methods
that require no positional input parameters be passed in.  The last
element in \vars{mylist} refers to a method with two input parameters.
Note that while I use the term ``method'' to describe the elements,
the strings/keys do not have to be only Python instance methods.
The second element, for instance, refers to another run list, and
the third element refers to a compiled QTCM1 model subroutine (note
the \vars{\_\_qtcm} attribute).

When the \mods{run\_list} method is called, the items in the input
run list are called in the order given in the list.  For each
element,  the \mods{run\_list} method first checks if the string
or dictionary key name corresponds to the key of an entry in the
\class{Qtcm} instance attribute \vars{runlists}.  If so, \mods{run\_list}
is called using that run list (i.e., it is a ``recursive'' call).
If the string or dictionary key name does not refer to another run
list, the \mods{run\_list} method checks if the string or dictionary
key name is a method of the \class{Qtcm} instance, and if so the
method is called.  Any other value throws an exception.

If input parameters for a method are of class \class{Field}, the
\mods{run\_list} method first tries to pass the parameters into the
method as is, i.e., as Field object(s).  If that fails, the
\mods{run\_list } method  passes its parameters in as the \vars{value}
attribute of the \class{Field} object.

If you want a variable that is being passed into a run list to be
continuously updated, you have to set the parameter in the run list
to a \class{Field} instance that is a \class{Qtcm} instance attribute,
not just to the value of the field variable (or to a non-\class{Field}
object).  Otherwise, subsequent calls to that run list element will
not use the updated values as input parameters.

For instance, if you had a run list element:
\begin{codeblock}
\codeblockfont{%
\{'\_\_qtcm.timemanager':[model.coupling\_day,]\}}
\end{codeblock}
and \vars{model.coupling\_day} were an integer (it's not by default,
but pretend it was), then \mods{run\_list} calling
\mods{\_\_qtcm.timemanager} will pass in a scalar integer rather
than a binding to the variable \vars{model.coupling\_day}.  In such
a situation, if the variable \vars{model.coupling\_day} were updated
in time, the \mods{run\_list} call of \mods{\_\_qtcm.timemanager}
would not be updated in time.  This happens because when the
dictionary that is the run list element is created, the value of
list element(s) attached to the dictionary element is set to the
scalar value of \vars{model.coupling\_day} at that instant.

You can get around this feature by setting \class{Qtcm} instance
attributes that will change with model execution to \class{Field}
instances, and then referring to those attributes in the parameter
list in the run list element.  In that case:
\begin{codeblock}
\codeblockfont{%
\{'\_\_qtcm.timemanager':[model.coupling\_day,]\}}
\end{codeblock}
will use the current value of \vars{model.coupling\_day} anytime
\vars{\_\_qtcm.timemanager} is called by \mods{run\_list}, if
\vars{model.coupling\_day} is a \class{Field} object.

When \mods{run\_list}, encounters a calling input parameter that
is a \class{Field} object, it will first try to pass the entire
\class{Field} object to the method/routine being called.  If that
raises an exception, it will then try to pass just the value of the
entire \class{Field} object.  This is done to enable \mods{run\_list}
to be used for both pure-Python and compiled QTCM Fortran model
routines.  Fortran cannot handle \class{Field} objects as input
parameters, only values.

Table~\ref{tab:stnd.runlists} shows all standard run lists
stored in the \vars{runlists} attribute upon instantiation
of a \class{Qtcm} instance.

\begin{htmlonly}
\begin{table}[htp]
\begin{center}
\fbox{Empty placeholder block for table that would have gone here.}
\end{center}
\caption{Standard run lists stored in the \vars{runlists} 
	attribute upon instantiation of a \class{Qtcm} instance.
	The run list and list element names are stored as strings.
	\emphpara{This table is improperly reproduced in the
	HTML conversion.  Please see the PDF version for the table.}}
\label{tab:stnd.runlists}
\end{table}
\end{htmlonly}

\begin{latexonly}
\begin{table}[htp]
% This file is automatically generated by
% code_to_latex.py.  It lists all the standard 
% runlists in the class.


\begin{longtable}{l|lc}
\textbf{Run List Name/Description} & \textbf{List Element(s) Name(s)} & 
                                                   \textbf{\# Arg(s)} \\
\hline
\endhead
\multirow{3}{*}{\parbox{0.4\linewidth}{atm\_bartr\_mode (calculate the atmospheric barotropic mode at the barotropic timestep)}} & \_\_qtcm.wrapcall.wsavebartr & None \\
         & \_\_qtcm.wrapcall.wbartr & None \\
         & \_\_qtcm.wrapcall.wgradphis & None \\
\hline
\multirow{5}{*}{\parbox{0.4\linewidth}{atm\_oc\_step (calculate the atmosphere and ocean models at a coupling timestep)}} & \_first\_method\_at\_atm\_oc\_step & None \\
         & \_\_qtcm.wrapcall.wtimemanager & 1\\
         & \_\_qtcm.wrapcall.wocean & 2\\
         & qtcm & None \\
         & \_\_qtcm.wrapcall.woutpall & None \\
\hline
\multirow{5}{*}{\parbox{0.4\linewidth}{atm\_physics1 (calculate atmospheric physics at one instant)}} & \_\_qtcm.wrapcall.wmconvct & None \\
         & \_\_qtcm.wrapcall.wcloud & None \\
         & \_\_qtcm.wrapcall.wradsw & None \\
         & \_\_qtcm.wrapcall.wradlw & None \\
         & \_\_qtcm.wrapcall.wsflux & None \\
\hline
\multirow{8}{*}{\parbox{0.4\linewidth}{atm\_step (calculate the entire atmosphere at one atmosphere timestep)}} & atm\_physics1 & None \\
         & \_\_qtcm.wrapcall.wsland1 & None \\
         & \_\_qtcm.wrapcall.wadvctuv & None \\
         & \_\_qtcm.wrapcall.wadvcttq & None \\
         & \_\_qtcm.wrapcall.wdffus & None \\
         & \_\_qtcm.wrapcall.wbarcl & None \\
         & \_bartropic\_mode\_at\_atm\_step & None \\
         & \_\_qtcm.wrapcall.wvarmean & None \\
\hline
\multirow{3}{*}{\parbox{0.4\linewidth}{init\_model (initialize the entire model, i.e., the atmosphere and ocean components and output)}} & qtcminit & None \\
         & \_\_qtcm.wrapcall.woceaninit & None \\
         & \_\_qtcm.wrapcall.woutpinit & None \\
\hline
\multirow{5}{*}{\parbox{0.4\linewidth}{qtcminit (initialize the atmosphere portion of the entire model)}} & \_\_qtcm.wrapcall.wparinit & None \\
         & \_\_qtcm.wrapcall.wbndinit & None \\
         & varinit & None \\
         & \_\_qtcm.wrapcall.wtimemanager & 1\\
         & atm\_physics1 & None \\
\end{longtable}
\caption{Standard run lists stored in the \vars{runlists} 
	attribute upon instantiation of a \class{Qtcm} instance.
	The run list and list element names are stored as strings.}
\label{tab:stnd.runlists}
\end{table}
\end{latexonly}

Of course, feel free to change the contents of any of the run lists
after instantiation, or to add additional run lists to the
\vars{runlists} attribute dictionary.  The ability to alter run
lists at runtime gives the \mods{qtcm} package much of its flexibility.




%---------------------------------------------------------------------
\section{Field Variables and the \class{Field} Class}
						\label{sec:field.variables}

The term ``field'' variable refers to QTCM1 model variables that 
are accessible at both the compiled Fortran QTCM1 model-level as
well as the Python \class{Qtcm} instance-level.
Field variables are all instances of the \class{Field} class
(though non-field variables can also be instances of \class{Field}).

Section~\ref{sec:field.variables.intro} gives a brief introduction to
the attributes and methods in a \class{Field} instance.
A nitty gritty description of the class is found in its docstrings.

	\subsection{Creating Field Variables}

To create a \class{Field} instance whose value is set to the
default, instantiate with the field id as the only positional
input argument.  Thus:

\begin{codeblock}
\codeblockfont{foo = Field('lastday')}
\end{codeblock}

will return \vars{foo} as a \class{Field} instance with \vars{foo.value}
set to the value listed in Section~\ref{sec:defaults.scalar}.
The value of all \class{Field} instances upon creation are specified
in the \mods{defaults} submodule of package \mods{qtcm}, and listed
in Sections~\ref{sec:defaults.scalar} and~\ref{sec:defaults.array}.

To create \class{Field} instances whose attributes are set different
from their defaults, you can specify the different settings in the
instantiation parameter list, or change the attributes once the
instance is created.  See the \class{Field} docstring for details.


	\subsection{Initial Field Variables}  \label{sec:initial.variables}

Field variables include both model parameters that do not change
for a \class{Qtcm} instance as well as prognostic variables that
do change during model integration.  As a result, many field variables
have values different from the default values listed in
Sections~\ref{sec:defaults.scalar} and~\ref{sec:defaults.array}.
In this section, I list the \emph{initial} values of all field
variables.  The ``initial'' values are the settings for \class{Qtcm}
field variables execution of the \mods{run\_session} method, but
prior to cycling through an atmosphere-ocean coupling timestep.
This is in contrast to ``default'' values, which the field variables
are given on instantiation, if no other value is specified.
Numerical values are rounded as per the conventions
of Python's \vars{\%g} format code.


		\subsubsection{Scalars}

For the fields that give the input/output directory names, and the
run name, the entry ``value varies'' is provided in the ``Value''
column.

% This file is automatically generated by the script
% code_to_latex.py in the doc/latex directory.  It is based
% upon the values found after model initialization, and should
% not be hand-edited if you want the values to correspond to
% the values in a Qtcm instance, for compiled_form='parts'.


\begin{longtable}{l|c|c|p{0.42\linewidth}}
\textbf{Field} & \textbf{Value} & \textbf{Units} & 
                                \textbf{Description} \\
\hline
\endhead
\vars{SSTdir} & value varies &  & Where SST files are \\
\vars{SSTmode} & seasonal &  & Decide what kind of SST to use \\
\vars{VVsmin} & 4.5 & m/s & Minimum wind speed for fluxes \\
\vars{bnddir} & value varies &  & Boundary data other than SST \\
\vars{dateofmodel} & 10101 &  & Date of model coded as an integer as yyyymmdd \\
\vars{day0} & 1 & dy & Starting day; if $<$ 0 use day in restart \\
\vars{dt} & 1200 & s & Time step \\
\vars{eps\_c} & 0.000138889 & 1/s & 1/tau\_c NZ (5.7) \\
\vars{interval} & 1 & dy & Atmosphere-ocean coupling interval \\
\vars{it} & 1 &  & Time of day in time steps \\
\vars{landon} & 1 &  & If not 1: land = ocean with fake SST \\
\vars{lastday} & 0 & dy & Last day of integration \\
\vars{month0} & 1 & mo & Starting month; if $<$ 0 use mo in restart \\
\vars{mrestart} & 0 &  & =1: restart using qtcm.restart \\
\vars{mt0} & 1 &  & Barotropic timestep every mt0 timesteps \\
\vars{nastep} & 1 &  & Number of atmosphere time steps within one air-sea coupling interval \\
\vars{noout} & 0 & dy & No output for the first noout days \\
\vars{nooutr} & 0 & dy & No restart file for the first nooutr days \\
\vars{ntout} & -30 & dy & Monthly mean output \\
\vars{ntouti} & 0 & dy & Monthly instantaneous data output \\
\vars{ntoutr} & 0 & dy & Restart file only at end of model run \\
\vars{outdir} & value varies &  & Where output goes to \\
\vars{runname} & value varies &  & String for an output filename \\
\vars{title} & value varies &  & A descriptive title \\
\vars{u0bar} & 0 &  &  \\
\vars{visc4x} & 700000 & m$^2$/s & Del 4 viscocity parameter in x \\
\vars{visc4y} & 700000 & m$^2$/s & Del 4 viscocity parameter in y \\
\vars{viscxT} & 1.2e+06 & m$^2$/s & Temperature diffusion parameter in x \\
\vars{viscxq} & 1.2e+06 & m$^2$/s & Humidity diffusion parameter in x \\
\vars{viscxu0} & 700000 & m$^2$/s & Viscocity parameter for u0 in x \\
\vars{viscxu1} & 700000 & m$^2$/s & Viscocity parameter for u1 in x \\
\vars{viscyT} & 1.2e+06 & m$^2$/s & Temperature diffusion parameter in y \\
\vars{viscyq} & 1.2e+06 & m$^2$/s & Humidity diffusion parameter in y \\
\vars{viscyu0} & 700000 & m$^2$/s & Viscocity parameter for u0 in y \\
\vars{viscyu1} & 700000 & m$^2$/s & Viscocity parameter for u1 in y \\
\vars{weml} & 0.01 & m/s & Mixed layer entrainment velocity \\
\vars{year0} & 1 & yr & Starting year; if $<$ 0 use year in restart \\
\vars{ziml} & 500 & m & Atmosphere mixed layer depth $\sim$ cloud base \\
\end{longtable}


		\subsubsection{Arrays}

% This file is automatically generated by the script
% code_to_latex.py in the doc/latex directory.  It is based
% upon the values found after model initialization, and should
% not be hand-edited if you want the values to correspond to
% the values in a Qtcm instance, for compiled_form='parts'.


\begin{longtable}{l|c|c|c|c|p{0.3\linewidth}}
\textbf{Field} & \textbf{Shape} & \textbf{Max} & \textbf{Min} &
                                \textbf{Units} & \textbf{Description} \\
\hline
\endhead
\vars{Evap} & (64, 42) & 1502.56 & 223.552 &  &  \\
\vars{FLW} & (64, 42) & 74.5136 & 74.5136 &  &  \\
\vars{FLWds} & (64, 42) & 206.424 & 206.424 &  &  \\
\vars{FLWus} & (64, 42) & 429.708 & 429.708 &  &  \\
\vars{FLWut} & (64, 42) & 148.771 & 148.771 &  &  \\
\vars{FSW} & (64, 42) & 147.767 & 0 &  &  \\
\vars{FSWds} & (64, 42) & 410.895 & -6.99713 &  &  \\
\vars{FSWus} & (64, 42) & 356.831 & -4.49983 &  &  \\
\vars{FSWut} & (64, 42) & 332.431 & 0 &  &  \\
\vars{FTs} & (64, 42) & 930.115 & 138.383 &  &  \\
\vars{Qc} & (64, 42) & 0 & 0 & K & Precipitation \\
\vars{S0} & (64, 42) & 534.264 & 0 &  &  \\
\vars{STYPE} & (64, 42) & 3 & 0 &  & Surface type; ocean or vegetation type over land \\
\vars{T1} & (64, 44) & -100 & -100 & K &  \\
\vars{Ts} & (64, 42) & 295 & 295 & K & Surface temperature \\
\vars{WD} & (64, 42) & 350 & 0 &  &  \\
\vars{WD0} & (4,) & 500 & 0 &  & Field capacity SIB2/CSU (approximately) \\
\vars{arr1} & (64, 42) & 0 & 0 &  & Auxiliary optional output array 1 \\
\vars{arr2} & (64, 42) & 0 & 0 &  & Auxiliary optional output array 2 \\
\vars{arr3} & (64, 42) & 0.138699 & 0.138699 &  & Auxiliary optional output array 3 \\
\vars{arr4} & (64, 42) & 0 & 0 &  & Auxiliary optional output array 4 \\
\vars{arr5} & (64, 42) & 0 & 0 &  & Auxiliary optional output array 5 \\
\vars{arr6} & (64, 42) & 0 & 0 &  & Auxiliary optional output array 6 \\
\vars{arr7} & (64, 42) & 0 & 0 &  & Auxiliary optional output array 7 \\
\vars{arr8} & (64, 42) & 0 & 0 &  & Auxiliary optional output array 8 \\
\vars{psi0} & (64, 43) & 0 & 0 &  &  \\
\vars{q1} & (64, 44) & -50 & -50 & K &  \\
\vars{rhsu0bar} & (3,) & 0 & 0 &  &  \\
\vars{rhsvort0} & (64, 42, 3) & 0 & 0 &  &  \\
\vars{taux} & (64, 42) & 0 & 0 &  &  \\
\vars{tauy} & (64, 42) & 0 & 0 &  &  \\
\vars{u0} & (64, 44) & 0 & 0 & m/s & Barotropic zonal wind \\
\vars{u1} & (64, 44) & 0 & 0 & m/s & Current time step baroclinic zonal wind \\
\vars{v0} & (64, 43) & 0 & 0 & m/s & Barotropic meridional wind \\
\vars{v1} & (64, 43) & 0 & 0 & m/s &  \\
\vars{vort0} & (64, 42) & 0 & 0 &  &  \\
\end{longtable}



	\subsection{Passing Fields Between the Python and Fortran-Levels}

Section~\ref{sec:comm.py.fort.compiledform} discusses the differences
between how the \vars{'full'} and \vars{'parts'} compiled forms
pass field variables between the Python and Fortran-levels.  That
discussion gives a detailed description of the methods used for
passing fields to and from the Python and Fortran-levels (i.e., the
\mods{get\_qtcm1\_item} and \mods{set\_qtcm1\_item} methods).

Please note the following regarding field variables as you pass them 
back and forth between the Python and Fortran-levels:
\begin{itemize}
\item Field variables with ghost latitudes, such as \vars{u1}, on
	the Python end are always the full variables (i.e., including
	the ghost latitudes).  On the Fortran end, variables like
	\vars{u1} also always have the ghost latitudes while in the
	model, but when stored as restart files, do not have the
	ghost latitudes; the end points are not saved in restart
	files or written to the netCDF output files.
	See the
	\latexhtml{%
\htmladdnormallinkfoot{QTCM1 manual}%
        {http://www.atmos.ucla.edu/$\sim$csi/qtcm\_man/v2.3/qtcm\_manv2.3.pdf}}%
{\htmladdnormallink{QTCM1 manual}%
        {http://www.atmos.ucla.edu/~csi/qtcm_man/v2.3/qtcm_manv2.3.pdf}}
	\cite{Neelin/etal:2002}
	for details about ghost latitudes.

\item You should assume there is only a full synchronizing between 
	compiled QTCM1 model and Python model field variables
	at the beginning and end of a run session.  

\item If you have a variable at the Python-level, but at the
	compiled QTCM1 Fortran model-level the variable is not
	readable, if you try to call \mods{set\_qtcm1\_item} on the
	variable, nothing is done, and the Python-level value is
	left alone.  If you have a compiled QTCM1 model variable,
	but no Python-level equivalent, if you call \mods{set\_qtcm1\_item}
	on the variable, the Python-level variable (as an attribute)
	is created.

\item To be precise, only compiled QTCM1 model variables can be
	passed pass back and forth between the Python and Fortran-levels;
	there are many \class{Qtcm} instance attributes that do not
	have any counterparts at the Fortran-level.\footnote%
		{I use the term ``field variables'' to refer to 
		compiled QTCM1 model variables that can be passed
		back and forth to the Python level.}

\item Although \vars{dayofmodel} is described in module \mods{setbypy}
	as an option for the \mods{get\_qtcm1\_item} and
	\mods{set\_qtcm1\_item} methods to operate on, in reality
	those methods cannot operate on \vars{dayofmodel}, but
	\vars{dayofmodel} is not defined in \mods{defaults}.\footnote%
		{All field variables must be defined in \mods{defaults} in
		order for the proper Fortran routine to be called
		according to the variable's type.}
\end{itemize}


	\subsection{Field Variable Shape}   \label{sec:field.var.shape}

Normally, Python arrays have a different dimension order than Fortran
arrays.  While Fortran arrays are dimensioned (col, row, slice),
with adjacent columns being contiguous, then rows, and then slices, Python
arrays are dimensioned (slice, row, col), with adjacent columns being
contiguous, then rows, and then slices.  Based on this, you would
think that everytime you passed an array between the Python and
Fortran-levels you would need to transpose the array.

Thankfully, we don't have to do this because \mods{f2py} handles
array dimension order transparently so we can refer to each element
the same way whether we're in Python or Fortran.  Thus, the array
\vars{Qc} in Fortran is dimensioned (longitude, latitude), (64,42)
by default, and the Python \class{Qtcm} instance attribute \vars{Qc}
has a \vars{value} attribute also dimensioned (longitude, latitude),
(64,42) by default.  And at both the Fortran and Python-levels, the
first longtude, second latitude element is referred to as \vars{Qc(1,2)}.

In contrast, however, netCDF output saved by the compiled QTCM1 model
and read into Python (using the \mods{Scientific} package) is
\emph{not} in Fortran array order.  Arrays read from netCDF output
into Python are in Python array order, and are dimensioned
(latitude, longitude) or (time, latitude, longitude).  The \class{Qtcm}
routines that manipulate netCDF data (e.g., \mods{plotm}), however,
automatically adjust for this, so you only need to be aware of this
when reading in output for your own analysis
(see Section~\ref{sec:model.output}).




%---------------------------------------------------------------------
\section{Model Output}			\label{sec:model.output}

Section~\ref{sec:output.intro} gives an overview of how to
use \mods{qtcm} model output to netCDF files.

All netCDF array output is dimensioned (time, latitude, longitude)
when read into Python using the \mods{Scientific} package.  This
differs from the way \class{Qtcm} saves field variables, which
follows Fortran convention (longitude, latitude).  Thus, the shapes
in Section~\ref{sec:initial.variables}, Appendix~\ref{app:defaults.values},
etc., are not the shapes of arrays read from the netCDF output.
See Section~\ref{sec:field.var.shape} for a discussion of why
there is this discrepancy.

Because netCDF files allow you to specify an ``unlimited'' dimension,
it is possible to close a netCDF file, reopen it, and add more
slices of data to the file.  Thus, continuous \class{Qtcm} run
sessions (i.e., those that use the \vars{cont} keyword input parameter
in the \mods{run\_session} method) will automatically append output
to the netCDF output files.

Field variables with ghost latitudes, such as \vars{u1}, on the
Python and Fortran ends are always the full variables (i.e., including
the ghost latitudes).  The ghost latitudes are not written to the
netCDF output files, however.
See the \latexhtml{%
\htmladdnormallinkfoot{QTCM1 manual}%
        {http://www.atmos.ucla.edu/$\sim$csi/qtcm\_man/v2.3/qtcm\_manv2.3.pdf}}%
{\htmladdnormallink{QTCM1 manual}%
        {http://www.atmos.ucla.edu/~csi/qtcm_man/v2.3/qtcm_manv2.3.pdf}}
	\cite{Neelin/etal:2002}
for details about ghost latitude structure.

\class{Qtcm} instances have a few built-in tools to visualization
model output.  These are briefly described in Section~\ref{sec:viz.intro}.
Note that the \mods{plotm} method is linked to a specific \class{Qtcm}
instance.  Do not use \mods{plotm} outside of the instance it is
linked to.  It must also be used only after a successful run session
(i.e., not in the middle of a run session).




%---------------------------------------------------------------------
\section{Miscellaneous}

A few miscellaneous items/issues about the model:
\begin{itemize}
\item The land model runs at same timestep as the atmosphere.

\item If the land model runs less often than 
	\mods{sflux} in \mods{physics1}, 
	the calculation of evaporation over the land 
	needs to be fixed in sflux.

\item The units of some field variables are not what you would expect.
	For instance, \vars{Qc} is in energy units, i.e., K, and not
	mm/day.
	See the
	\latexhtml{%
\htmladdnormallinkfoot{QTCM1 manual}%
        {http://www.atmos.ucla.edu/$\sim$csi/qtcm\_man/v2.3/qtcm\_manv2.3.pdf}}%
{\htmladdnormallink{QTCM1 manual}%
        {http://www.atmos.ucla.edu/~csi/qtcm_man/v2.3/qtcm_manv2.3.pdf}}
	\cite{Neelin/etal:2002}
	for details.
\end{itemize}




%---------------------------------------------------------------------
\section{Cookbook of Ways the Model Can Be Used}  \label{sec:cookbook}

This cookbook of a few ways to use the model is arranged by science
tasks, i.e., certain types of runs we want to do.  For some of the
examples below, I assume that the dictionary
\vars{inputs} is initially defined as given in
Figure~\ref{fig:defn.of.inputs}.  All examples assume that
\cmd{from qtcm import Qtcm} has already been executed.


%--- Two versions, one for PDF and the other for HTML:
\begin{latexonly}
\begin{figure}[tp]
\begin{codeblock}
\codeblockfont{%
inputs = \{\} \\
inputs['runname'] = 'test' \\
inputs['landon'] = 0 \\
inputs['year0'] = 1 \\
inputs['month0'] = 11 \\
inputs['day0'] = 1 \\
inputs['lastday'] = 30 \\
inputs['mrestart'] = 0 \\
inputs['init\_with\_instance\_state'] = True \\
inputs['compiled\_form'] = 'parts'}
\end{codeblock}

\caption{The initial definition of the \vars{inputs} dictionary for 
	examples given in Section~\ref{sec:cookbook}.  These settings
	imply that a run session will start on November 1, Year 1,
	last for 30 days, and will be an aquaplanet run.}
\label{fig:defn.of.inputs}
\end{figure}
\end{latexonly}

\begin{htmlonly}
\label{fig:defn.of.inputs}
\begin{center}
\htmlfigcaption{%
	\codeblockfont{%
inputs = \{\} \\
inputs['runname'] = 'test' \\
inputs['landon'] = 0 \\
inputs['year0'] = 1 \\
inputs['month0'] = 11 \\
inputs['day0'] = 1 \\
inputs['lastday'] = 30 \\
inputs['mrestart'] = 0 \\
inputs['init\_with\_instance\_state'] = True \\
inputs['compiled\_form'] = 'parts'}
	}

\htmlfigcaption{Figure~\ref{fig:defn.of.inputs}:
	The initial definition of the \vars{inputs} dictionary for 
	examples given in Section~\ref{sec:cookbook}.  These settings
	imply that a run session will start on November 1, Year 1,
	last for 30 days, and will be an aquaplanet run.}
\end{center}
\end{htmlonly}



\begin{description}
\item[Plain model run:]
	Here I just want to make a single model run.
	Tasks:  Instantiate a fresh model and execute a run session.
	The code to run the model is just:
	\begin{codeblock}
	\codeblockfont{%
inputs['init\_with\_instance\_state'] = False \\
model = Qtcm(**inputs) \\
model.run\_session()}
	\end{codeblock}
	where \vars{inputs} is initialized with the code in
	Figure~\ref{fig:defn.of.inputs}.


\item[Explore parameter space with a set of models:]
	Here I want to create an entire suite of separate models,
	in order to determine the sensitivity of the model to changing
	a parameter.
	To do this, I
	instantiate multiple fresh models, 
	and execute a run session for each instance, all within
	a \vars{for} loop:


%--- Two versions, because LaTeX2HTML does not correctly typeset
%    the hspace command:
\begin{latexonly}
	\begin{codeblock}
	\codeblockfont{%
import os \\
inputs['init\_with\_instance\_state'] = False \\
for i in xrange(0,1002,10): \\
\hspace*{5ex}iname = 'ziml-' + str(i) + 'm' \\
\hspace*{5ex}ipath = os.path.join('proc', iname) \\
\hspace*{5ex}os.makedirs(ipath) \\
\hspace*{5ex}model = Qtcm(**inputs) \\
\hspace*{5ex}model.ziml.value = float(i)  \\
\hspace*{5ex}model.runname.value = iname \\
\hspace*{5ex}model.outdir.value = ipath \\
\hspace*{5ex}model.run\_session() \\
\hspace*{5ex}del model}
	\end{codeblock}
\end{latexonly}

\begin{htmlonly}
\begin{center}
\htmlfigcaption{%
	\codeblockfont{%
import os \\
inputs['init\_with\_instance\_state'] = False \\
for i in xrange(0,1002,10): \\
\hspace*{5ex}iname = 'ziml-' + str(i) + 'm' \\
\hspace*{5ex}ipath = os.path.join('proc', iname) \\
\hspace*{5ex}os.makedirs(ipath) \\
\hspace*{5ex}model = Qtcm(**inputs) \\
\hspace*{5ex}model.ziml.value = float(i)  \\
\hspace*{5ex}model.runname.value = iname \\
\hspace*{5ex}model.outdir.value = ipath \\
\hspace*{5ex}model.run\_session() \\
\hspace*{5ex}del model}
	}
\end{center}
\end{htmlonly}


	The loop explores mixed-layer depth \vars{ziml} from 0~m to
        1000~m, in 10~m intervals.  I create the \vars{outdir}
	directory before every model call, since the compiled QTCM1 model
	requires the output directory exist, specifying the run name
	and output directory as the string \vars{iname}.
	The output directories are assumed to all be in the \fn{proc}
	sub-directory of the current working directory.
	\vars{inputs} is initialized with the code in
	Figure~\ref{fig:defn.of.inputs}.


\item[Conditionally explore parameter space:]
	Here I want to 
	conditionally explore the parameter space, on the basis of
	some mathematical criteria.
	To do this, I
	instantiate a model, evaluate results using
	that criteria, and run another fresh model depending on
	the results (passing the previous model state via a snapshot),
	all within a \vars{while} loop.
	Note that this type of investigation is very difficult to 
	automate if all you can use are shell scripts and
	Fortran.
	See Figure~\ref{fig:conditional.test.eg} for a detailed
	example.


\item[With interactive adjustments at run time:]
	The example in Figure~\ref{sec:continuation.intro}
	illustrates this type of run.  In this example,
	I instantiate a fresh model, execute a run session, analyze the
	output, change variables in the model instance, and then
	execute a continuation run session.


\item[Test alternative parameterizations:]
	I've already described how we can use run lists to arbitrarily
	change model execution order and content at run time.
	We can take advantage of Python's inheritance
	abilities, along with run lists, to simplify this.
	Figure~\ref{fig:alt.param.inherit.eg} provides an example of
	this use.

	Of course, you can use pre-processor directives and shell
	scripts to accomplish the same functionality seen in
	Figure~\ref{fig:alt.param.inherit.eg} using just Fortran.
	The Python solution, however, shortcuts the compile/linking
	step, and enables you to easily do run time swapping between
	subroutine choices based upon run time calculated
	tests (see Figure~\ref{fig:conditional.test.eg} for an
	example of such tests).
\end{description}




% --- Two versions of this block, one for display in PDF and the other
%     for display in HTML:
\begin{latexonly}
\begin{figure}[p]
	\begin{codeblock}
	\codeblockfont{%
\small
import os \\
import numpy as N \\
maxu1 = 0.0 \\
while maxu1 < 10.0: \\
\hspace*{5ex}iziml = 0.1 * maxu1 \\
\hspace*{5ex}iname = 'ziml-' + str(iziml) + 'm' \\
\hspace*{5ex}ipath = os.path.join('proc', iname) \\
\hspace*{5ex}os.makedirs(ipath) \\
\hspace*{5ex}model = Qtcm(**inputs) \\
\hspace*{5ex}try: \\
\hspace*{10ex}model.sync\_set\_py\_values\_to\_snapshot(snapshot=mysnapshot) \\
\hspace*{10ex}model.init\_with\_instance\_state = True \\
\hspace*{5ex}except: \\
\hspace*{10ex}model.init\_with\_instance\_state = False \\
\hspace*{5ex}model.ziml.value = iziml  \\
\hspace*{5ex}model.runname.value = iname \\
\hspace*{5ex}model.outdir.value = ipath \\
\hspace*{5ex}model.run\_session() \\
\hspace*{5ex}maxu1 = N.max(N.abs(model.u1.value)) \\
\hspace*{5ex}mysnapshot = model.snapshot \\
\hspace*{5ex}del model}
	\end{codeblock}

\caption{This code explores different values of
	mixed-layer depth \vars{ziml} for 30~day runs,
	as a function of maximum \vars{u1} magnitude,
	until it finds a case where the maximum \vars{u1} is
	greater than 10~m/s.  (The relationship between
	\vars{ziml} and the maximum of the speed of
	\vars{u1}, where 
	\vars{ziml\thinspace=\thinspace0.1\thinspace*\thinspace{maxu1}}, 
	is made up.)
	With each iteration, the new run uses the snapshot from
	a previous run to initialize (as well as the new value
	of \vars{ziml}); the \vars{try} statement is used to
	ensure the model works even if \vars{mysnapshot} is not
	defined (which is the case the first time around).
	The \vars{inputs} dictionary is initialized with the code in
	Figure~\ref{fig:defn.of.inputs}.}
\label{fig:conditional.test.eg}
\end{figure}
\end{latexonly}

\begin{htmlonly}
\label{fig:conditional.test.eg}
\begin{center}
\htmlfigcaption{%
	\codeblockfont{%
import os \\
import numpy as N \\
maxu1 = 0.0 \\
while maxu1 < 10.0: \\
\hspace*{5ex}iziml = 0.1 * maxu1 \\
\hspace*{5ex}iname = 'ziml-' + str(iziml) + 'm' \\
\hspace*{5ex}ipath = os.path.join('proc', iname) \\
\hspace*{5ex}os.makedirs(ipath) \\
\hspace*{5ex}model = Qtcm(**inputs) \\
\hspace*{5ex}try: \\
\hspace*{10ex}model.sync\_set\_py\_values\_to\_snapshot(snapshot=mysnapshot) \\
\hspace*{10ex}model.init\_with\_instance\_state = True \\
\hspace*{5ex}except: \\
\hspace*{10ex}model.init\_with\_instance\_state = False \\
\hspace*{5ex}model.ziml.value = iziml  \\
\hspace*{5ex}model.runname.value = iname \\
\hspace*{5ex}model.outdir.value = ipath \\
\hspace*{5ex}model.run\_session() \\
\hspace*{5ex}maxu1 = N.max(N.abs(model.u1.value)) \\
\hspace*{5ex}mysnapshot = model.snapshot \\
\hspace*{5ex}del model}
	}

\htmlfigcaption{Figure \ref{fig:conditional.test.eg}:
	This code explores different values of
	mixed-layer depth \vars{ziml} for 30~day runs,
	as a function of maximum \vars{u1} magnitude,
	until it finds a case where the maximum \vars{u1} is
	greater than 10~m/s.  (The relationship between
	\vars{ziml} and the maximum of the speed of
	\vars{u1}, where 
	\vars{ziml\thinspace=\thinspace0.1\thinspace*\thinspace{maxu1}}, 
	is made up.)
	With each iteration, the new run uses the snapshot from
	a previous run to initialize (as well as the new value
	of \vars{ziml}); the \vars{try} statement is used to
	ensure the model works even if \vars{mysnapshot} is not
	defined (which is the case the first time around).
	The \vars{inputs} dictionary is initialized with the code in
	Figure~\ref{fig:defn.of.inputs}.}
\end{center}
\end{htmlonly}


% --- Two versions of this block, one for display in PDF and the other
%     for display in HTML:
\begin{latexonly}
\begin{figure}[p]
\begin{center}
	\begin{codeblock}
	\codeblockfont{%
\small
import os \\
\\
class NewQtcm(Qtcm): \\
\hspace*{5ex}def cloud0(self):\\
\hspace*{10ex}[\ldots] \\
\hspace*{5ex}def cloud1(self):\\
\hspace*{10ex}[\ldots] \\
\hspace*{5ex}def cloud2(self):\\
\hspace*{10ex}[\ldots] \\
\hspace*{5ex}[\ldots] \\
\\
inputs['init\_with\_instance\_state'] = False \\
for i in xrange(10): \\
\hspace*{5ex}iname = 'cloudroutine-' + str(i)  \\
\hspace*{5ex}ipath = os.path.join('proc', iname) \\
\hspace*{5ex}os.makedirs(ipath) \\
\hspace*{5ex}model = NewQtcm(**inputs) \\
\hspace*{5ex}model.runlists['atm\_physics1'][1] = 'cloud' + str(i) \\
\hspace*{5ex}model.runname.value = iname \\
\hspace*{5ex}model.outdir.value = ipath \\
\hspace*{5ex}model.run\_session() \\
\hspace*{5ex}del model}
	\end{codeblock}
\end{center}

\caption{Let's say we have 9 different cloud physics schemes we wish
	to try out in 9 different runs.  The easiest way to do this
	is to create a new class \class{NewQtcm} that
	inherits everything from \class{Qtcm}, and to which we'll
	add the additional cloud schemes (\vars{cloud0}, \vars{cloud1},
	etc.).
	In the \vars{for} loop, I change the cloud model
	run list entry in the run list that governs
	atmospheric physics at one instant to whatever the cloud
	model is at this point in the loop.
	The \vars{inputs} dictionary is initialized with the code in
	Figure~\ref{fig:defn.of.inputs}.
	Of course, we could do the same thing by running the 9
	models separately, but this set-up makes it easy to do
	hypothesis testing with these 9 models.  For instance, we
	can create a test by which we will choose which of the 9
	models to use:  Within this framework, the selection of
	those models can be altered by changing a string.}
\label{fig:alt.param.inherit.eg}
\end{figure}
\end{latexonly}

\begin{htmlonly}
\label{fig:alt.param.inherit.eg}
\begin{center}
\htmlfigcaption{%
	\codeblockfont{%
import os \\
\\
class NewQtcm(Qtcm): \\
\hspace*{5ex}def cloud0(self):\\
\hspace*{10ex}[\ldots] \\
\hspace*{5ex}def cloud1(self):\\
\hspace*{10ex}[\ldots] \\
\hspace*{5ex}def cloud2(self):\\
\hspace*{10ex}[\ldots] \\
\hspace*{5ex}[\ldots] \\
\\
inputs['init\_with\_instance\_state'] = False \\
for i in xrange(10): \\
\hspace*{5ex}iname = 'cloudroutine-' + str(i)  \\
\hspace*{5ex}ipath = os.path.join('proc', iname) \\
\hspace*{5ex}os.makedirs(ipath) \\
\hspace*{5ex}model = NewQtcm(**inputs) \\
\hspace*{5ex}model.runlists['atm\_physics1'][1] = 'cloud' + str(i) \\
\hspace*{5ex}model.runname.value = iname \\
\hspace*{5ex}model.outdir.value = ipath \\
\hspace*{5ex}model.run\_session() \\
\hspace*{5ex}del model}
	}

\htmlfigcaption{Figure \ref{fig:alt.param.inherit.eg}:
	Let's say we have 9 different cloud physics schemes we wish
	to try out in 9 different runs.  The easiest way to do this
	is to create a new class \class{NewQtcm} that
	inherits everything from \class{Qtcm}, and to which we'll
	add the additional cloud schemes (\vars{cloud0}, \vars{cloud1},
	etc.).
	In the \vars{for} loop, I change the cloud model
	run list entry in the run list that governs
	atmospheric physics at one instant to whatever the cloud
	model is at this point in the loop.
	The \vars{inputs} dictionary is initialized with the code in
	Figure~\ref{fig:defn.of.inputs}.
	Of course, we could do the same thing by running the 9
	models separately, but this set-up makes it easy to do
	hypothesis testing with these 9 models.  For instance, we
	can create a test by which we will choose which of the 9
	models to use:  Within this framework, the selection of
	those models can be altered by changing a string.}
\end{center}
\end{htmlonly}




% ===== end of file =====


%@@@\chapter{Combining \code{qtcm} with \code{CliMT}}
%@@@% ==========================================================================
% CliMT
%
% By Johnny Lin
% ==========================================================================


% ------ BODY -----
%
\section{General Tutorial on CliMT}


General notes of things I think I may have observed about
\code{Parameters} objects:
\begin{itemize}
\item You can treat a \code{Parameters} instance as a dictionary, where
	the key is the name of the field, because \code{\_\_getitem\_\_},
	etc.\ have been defined for the instance.  However, the values,
	units, and long names of the fields are stored in dictionaries
	assigned to \code{value}, \code{units}, and \code{long\_name},
	keyed to the field name (a string).
\end{itemize}


General notes of things I think I may have observed about
\code{Components} objects:
\begin{itemize}
\item All variables and quantities, whether they be physical fields,
	filenames, or metadata,
	are stored as attributes in the \code{Components} instance.
\item \code{Components} have these special attributes:
        \code{Required},
        \code{Prognostic},
	and
        \code{Diagnostic},
	which are lists that contain the names of describe whether
\item Scalar parameters in \code{Component} objects
	are stored as an instance of the \code{Parameters}
	class, under the attribute \code{Params}.
\end{itemize}


General notes of things I think I may have observed about
\code{Federation} objects:
\begin{itemize}
\item \code{Federation} objects hold the \code{Components} instances
	in a list assigned to the attribute \code{list}.
\item \code{Federation} attributes
        \code{Required},
        and
	\code{Prognostic},
	are unions of the same attributes of the constituent
	\code{Components} objects.
\end{itemize}






% ===== end of file =====


\chapter{Troubleshooting}                   \label{ch:trouble}
% ==========================================================================
% Troubleshooting
%
% By Johnny Lin
% ==========================================================================


% ------ BODY -----
%
\section{Error Messages Produced by \mods{qtcm}}

\begin{description}
\item[\screen{Error-Value too long in SetbyPy module getitem\_str for}
	\dumarg{key}:]
	This message is produced by the Fortran
	subroutine \mods{getitem\_str}
	in the module \mods{SetbyPy} in the compiled QTCM1 Fortran code.
	The code is in the file \fn{setbypy.F90}.  This error occurs when
	the Fortran variable whose name is given by the string \dumarg{key}
	has a value that is greater than the local parameter
	\vars{maxitemlen} in \mods{getitem\_str}.  To fix this, you have
	to go into \fn{setbypy.F90} and change the value of
	\vars{maxitemlen}.

\item[\screen{Error-real\_rank1\_array should be deallocated}:]
	Fortran module \mods{SetByPy}'s subroutine
	\mods{getitem\_real\_array} generates this message
	(or a similar message for other ranks) if the Fortran
	variable for the input \dumarg{key} are allocated on entry
	to the routine.  This may indicate the user has written another
	Fortran routine to access the \mods{real\_rank1\_array} variable
	outside of the standard interfaces..

\item[\screen{Error-Bad call to SetbyPy module \ldots}:]
	Often times, this error occurs because a get or set routine
	in \mods{SetByPy} tried to act on a variable for which the
	corresponding input \dumarg{key} is not defined.  The solution
	is to add that case in the if/then construct for the get and set
	routines in \mods{SetByPy} and rebuild the extension modules.
\end{description}


\section{Other Errors}

\begin{description}
\item[Python cannot find some packages:]
	This error often happens when the version of Python in which
	you have installed all your packages is not the version that
	is called at the Unix command line by typing in \cmd{python}.
	To get around this, 
        define a Unix alias
        that maps \cmd{python2.4} (or whichever version of Python
	has all your packages installed) to \cmd{python}.  If you
	have multiple Python's installed on your system, you might
	have to use a more specific name for the Python executable.
	As a result, you may have to change the test scripts in
	\fn{test} in the \mods{qtcm} distribution directory.

\item[\mods{get\_qtcm1\_item} and compiled QTCM1 model pointer
	variables:]
	If you try to use the \mods{get\_qtcm1\_item} method on a compiled
	QTCM1 model pointer variable 
	(i.e., \vars{u1}, \vars{v1}, \vars{q1}, \vars{T1}),
	 before the compiled
	model \mods{varinit} subroutine is run, you'll get a bus error
	with no additional message.

\item[Mismatch between Python and Fortran array field variables:]
	You change an array field variable on the Python side, but
	it seems like the wrong elements are changed on the Fortran
	side.  Or you type in the same index address for accessing a
	\mods{qtcm} netCDF output array as well as its \class{Qtcm}
	instance attribute counterpart, and find you get different
	answers.  Some possible reasons and fixes:

	\begin{itemize}
	\item This will occur if you haven't accounted for the
		difference in how field variables are saved at the
		Python-level, Fortran-level, and in a netCDF file.
		All netCDF array output is dimensioned (time,
		latitude, longitude) when read into Python using
		the \mods{Scientific} package.  This differs from
		the way \class{Qtcm} saves field variables, \emph{both}
		at the Python- and Fortran-levels, which follows
		Fortran convention (longitude, latitude).

		Note that the way \class{Qtcm} saves field variables
		at the Python- and Fortran-levels is different than
		the default way Python and Fortran save arrays.
		Section~\ref{sec:field.var.shape} for more information.

	\item You may have forgotten that array indices in Python start at
		0, while indices in Fortran (generally) start at 1.
		Also, ranges in Python are exclusive at the upper-bound,
		while ranges in Fortran are inclusive at the upper-bound.
		(Both Python and Fortran array indice ranges are inclusive
		at the lower-bound.)

	\item You may have forgotten some field variables have
		ghost latitudes, and thus there are extra latitude bands
		when the array is stored as a Python or Fortran field
		variable, but there are \emph{no} extra latitude bands
		when the array is stored as netCDF output (the QTCM1
		output routines strip off the ghost latitudes when
		writing those field variables out).
	        See the
        \latexhtml{%
\htmladdnormallinkfoot{QTCM1 manual}%
        {http://www.atmos.ucla.edu/$\sim$csi/qtcm\_man/v2.3/qtcm\_manv2.3.pdf}}%
{\htmladdnormallink{QTCM1 manual}%
        {http://www.atmos.ucla.edu/~csi/qtcm_man/v2.3/qtcm_manv2.3.pdf}}
        \cite{Neelin/etal:2002}
        for details about ghost latitudes.

		The safest and easiest way to tell whether the variable has a
		ghost latitudes is to look at its shape.
		A call to the \class{Qtcm} instance
		method \mods{get\_qtcm1\_item} will give you the array,
		and the use of NumPy's \mods{shape} function will give you
		the shape.
	\end{itemize}
\end{description}




% ===== end of file =====


\chapter{Developer Notes}                   \label{ch:devnotes}
% ==========================================================================
% Using QTCM
%
% By Johnny Lin
% ==========================================================================


% ------ BODY -----
%

%---------------------------------------------------------------------
\section{Introduction}

This section describes programming practices and issues related to
the \mods{qtcm} package that might be of interest to users wishing
to add/change code in the package.
Please see the package
\latexhtml{API documentation,%
		\footnote{http://www.johnny-lin.com/py\_pkgs/qtcm/doc/html-api/}
		which includes the source code}%
        {\htmladdnormallink{API documentation}%
		{http://www.johnny-lin.com/py\_pkgs/qtcm/doc/html-api/},
		which includes the source code},
for details.




%---------------------------------------------------------------------
\section{Changes to QTCM1 Fortran Files}  \label{sec:f90changes}

The source code used to generate the shared object files used
in this Python package is unchanged
from the pure-Fortran QTCM1 model source code, except in the
following ways:

\begin{itemize}
\item The suffix of all source code files 
	has been changed from \fn{.f90} to \fn{.F90}, 
	in order to ensure the compiler preprocesses 
	the source code.  Some compilers use the capitalization to
	tell whether or not to run the code through a preprocessor.

\item In all \fn{.F90} files, occurrences of:
	\begin{codeblock}
	\codeblockfont{%
	Character(len=130)}
	\end{codeblock}
	are changed to:
	\begin{codeblock}
	\codeblockfont{%
	Character(len=305)}
	\end{codeblock}
	This enables the model to properly deal with longer filenames.
	The number ``305'' is chosen to make search and replace easier.

\item In \fn{qtcmpar.F90}, the 
	\vars{eps\_c} variable is changed from an unchangable
	parameter to a changeable real, 
	so that it can be changed in the model at runtime.

\item All occurrences of an underscore (``\_'') character in a
	subroutine or function name are removed.  The
	presence of the underscore messes up the dynamic lookup
	mechanism for the \mods{f2py} generated extension module.
	The following names are changed, both in subroutine definitions
	and calls:
	\begin{itemize}
	\item \mods{out\_restart} to \mods{outrestart},
	\item \mods{save\_bartr} to \mods{savebartr},
	\item \mods{grad\_phis} to \mods{gradphis}.
	\end{itemize}

\item \fn{driver.F90} is changed so that program
	\mods{driver} becomes a subroutine, and 
	subroutine \mods{driverinit} is deleted (along with
	all calls to it) because basic model initialization is
	handled at the Python level.

\item In \fn{clrad.F90}, subroutine \mods{cloud}, the first
	\vars{COUNTCAP} preprocessor macro, a comment line for
	that ifdef is moved to prevent a warning message during
	building with \mods{f2py}.

\item The order of subroutine \mods{qtcminit} is changed.  The original
	pure-Fortran QTCM1 \mods{qtcminit} code has the following
	calling sequence:

	\begin{codeblock}
        \codeblockfont{%
Call parinit            !Initialize model parameters \\
Call varinit            !Initialize variables \\
Call TimeManager(1)     !mm set model time \\
Call bndinit            !input boundary datasets \\
Call physics1           !diagnostic fields for initial condition}
	\end{codeblock}

	For the \mods{qtcm} package, I've altered this order so
	\mods{bndinit} comes after \mods{parinit} but before \mods{varinit}:
	\begin{codeblock}
        \codeblockfont{%
Call parinit            !Initialize model parameters \\
Call bndinit            !input boundary datasets \\
Call varinit            !Initialize variables \\
Call TimeManager(1)     !mm set model time  \\
Call physics1           !diagnostic fields for initial condition}
	\end{codeblock}

	This is done because \vars{STYPE} is not read in for the
	\vars{landon} \vars{True} case until \mods{bndinit}, but
	in \mods{varinit} \vars{STYPE} is used to calculate the
	original values of \vars{WD} for the non-restart case.  This
	also corrects the conflicting order found in the pure-Fortran
	QTCM1 manual (compare pp.\ 29 and 32).  As far as I can
	tell, \mods{bndinit} has no dependencies that require it
	to come after \mods{timemanager} or \mods{varinit}.

\end{itemize}

In addition, the Fortran files \fn{setbypy.F90}, \fn{wrapcall.F90},
and \fn{varptrinit.F90} are added.  The routines in these files, 
however, just add more flexibility and functionality to the model;
they do not automatically affect any model computations.  See
Section~\ref{sec:newf90} for details.




%---------------------------------------------------------------------
\section{New Interfaces and Fortran Functionality}  \label{sec:newf90}

As described in Section~\ref{sec:f90changes}, the Fortran files
\fn{setbypy.F90}, \fn{wrapcall.F90}, and \fn{varptrinit.F90} are
added to the QTCM1 source directory.  The first two files define the Fortran
90 modules (\mods{SetbyPy} and \mods{WrapCall}) needed to interface
the Python and Fortran levels.  The last file defines a new Fortran
subroutine \mods{varptrinit} that associates QTCM1 model pointer
variables at the Fortran level.  In a pure-Fortran run of QTCM1,
this occurs in subroutine \mods{varinit}; for a
\vars{compiled\_form\thinspace=\thinspace'parts'} run, since the
functionality of the Fortran \mods{varinit} is now in the Python
\mods{varinit} method, a separate Fortran pointer association
subroutine needed to be defined.  The Fortran subroutine \mods{varptrinit}
is called as the \mods{varptrinit} function of the 
\vars{compiled\_form\thinspace=\thinspace'parts'}
\fn{.so} extension module.


	\subsection{Fortran Module \mods{SetbyPy}}   \label{sec:setbypy}

		\subsubsection{Design Description}

This module defines functions and subroutines used to read variables
from the Fortran-level to the Python-level, and in setting Fortran-level
variables using the Python-level values.  These routines are used
by \class{Qtcm} methods \mods{get\_qtcm1\_item} and \mods{set\_qtcm1\_item}
(and dependencies thereof) to ``get'' and ``set'' the Fortran-level
variables.  Note that the Fortran module \mods{SetbyPy} is referred
to in lowercase at the Python level, i.e., as the
attribute \vars{\_\_.qtcm.setbypy} of a \class{Qtcm} instance.

Because Fortran variables are not dynamically typed, separate Fortran
functions and subroutines need to be defined to get and set variables
of different types.\footnote%
	{The \mods{interface} construct in Fortran 90 is supposed to
	provide a way to create a single interface to multiple
	routines, e.g.:
	\begin{codeblock}
	\codeblockfont{%
Interface setitem \\
\hspace*{3ex}Module Procedure setitem\_real, setitem\_int, setitem\_str \\
End Interface}
	\end{codeblock}
	This construct, however, causes a bus error
	(Mac OS X 10.4, Intel).  Thus, I put the
	same functionality in the Python code.}
The \class{Qtcm} methods \mods{get\_qtcm1\_item}
and \mods{set\_qtcm1\_item} know which one of the Fortran routines
to call on the basis of the type and rank of the value for the field
variable in the \mods{defaults} submodule.  This is why all field
variables need to have defaults defined in \mods{defaults}.  For
array variables, the field variable defaults also provide the rank
of the Fortran-level variable being gotten or set.  However, the
array default values do \emph{not} have to have the same shape as
the Fortran-level variables; on the Python-side, variable shape
adjusts to what is declared on the Fortran-side.  
Thus, if you change the resolution of
the compiled QTCM1 model, you do not have to change the dimensions
of the field variable values in \mods{defaults}.

The \class{Qtcm} method \mods{get\_qtcm1\_item} directly calls
the \mods{SetByPy} routines.
The \class{Qtcm} method \mods{set\_qtcm1\_item} makes use of
private instance methods that make the calls to the \mods{SetByPy} routines.

For scalar field variables, \mods{SetByPy} provides functions and
subroutines that provide the value of the variable on output.
For array field variables, \mods{SetByPy}
dynamic \emph{module} arrays are used to pass array
variables in and out; I could not get the 
\mods{SetByPy} Fortran routines to set
locally defined dynamic arrays (that is, locally within a function or
subroutine).\footnote%
	{I tried to implement Fortran subroutine
	\mods{getitem\_real\_array} using traditional array 
	dimension passing 
	(e.g., \code{subroutine foo(nx, ny, a)}) as well
	as declaring the allocatable array inside the subroutine, 
	but neither option worked on my \mods{f2py} (version 2\_3816) 
	and Python (version 2.4.3).}
In the \mods{SetByPy} module, these dynamic arrays
are defined as follows:

\begin{codeblock}
\codeblockfont{%
Real, allocatable, dimension(:) :: real\_rank1\_array \\
Real, allocatable, dimension(:,:) :: real\_rank2\_array \\
Real, allocatable, dimension(:,:,:) :: real\_rank3\_array}
\end{codeblock}

For all field variables, scalar or array, the \mods{SetByPy} module
has a fourth module variable defined, \vars{is\_readable}, that the
Fortran get and set routines will set to \vars{.TRUE.} if the
variable is readable and \vars{.FALSE.} if not (it's declared as a
logical variable).  This Fortran variable can be used to prevent
Python from accessing pointer variables that aren't yet associated
to targets.

In general, \mods{SetByPy} routines make use of Fortran constructs
to enable them to accomodate all possible
variables of a given type and shape.  However, 
for string scalars, the \mods{SetByPy} function \mods{getitem\_str}
has to have a return value of a predefined length, in order to
work properly.  That length is given by the parameter
\vars{maxitemlen} and is set to 505 (the value is chosen to
be larger than all filename variables described in
Section~\ref{sec:f90changes} and to be easily found in
the \fn{.F90} files).


		\subsubsection{Module Structure}

If you're a Fortran programmer, you can probably get all the information
in this section from just reading the \fn{setbypy.F90} file directly.
This description of the module structure, however, permits me to highlight
what you need to do if you want to make additional compiled QTCM1 variables
accessible to Python \class{Qtcm} objects.

\begin{itemize}
\item All \mods{Use} statements are given in the beginning of 
	the \mods{SetByPy} module.  These statements cover
	nearly all of the QTCM1 Fortran
	modules that contain variables of interest.  If the
	QTCM1 variable you're interested in isn't in a module
	listed here, you'll have to add your own
	\mods{Use} statement of that module here.

\item Next comes the definitions for the
	\vars{real\_rank1\_array},
	\vars{real\_rank2\_array}, and
	\vars{real\_rank3\_array} dynamic array variables, and
	the \vars{is\_readable} boolean variable.

\item The \mods{Contains} block of the module defines the module
	routines called by the \class{Qtcm} instance methods to
	set and get the compiled QTCM1 model variables.  The
	routines are:
	\begin{itemize}
	\item Function \mods{getitem\_real}
	\item Subroutine \mods{getitem\_real\_array}
	\item Function \mods{getitem\_int}
	\item Function \mods{getitem\_str}
	\item Subroutine \mods{setitem\_real}
	\item Subroutine \mods{setitem\_real\_array}
	\item Subroutine \mods{setitem\_int}
	\item Subroutine \mods{setitem\_str}
	\end{itemize}

\end{itemize}

Each of the routines in the module \mods{Contains} block is essentially
a list of \mods{if}/\mods{elseif} statements.  The list tests for the
name of the variable of interest (a string), and gets or sets the
compiled QTCM1 model variable corresponding to that name.  For pointer
array variables, a test is also made as to whether or not the variable
has been associated.  If not, the variable is not readable
and \vars{is\_readable} is set to \vars{.FALSE.}\ accordingly.

If you wish to add another compiled QTCM1 model variable to be
accessible to \class{Qtcm} instance methods \mods{get\_qtcm1\_item}
and \mods{set\_qtcm1\_item}, just add another \mods{if}/\mods{else\-if},
like the other \mods{if}/\mods{elseif} blocks, in the Fortran set
and get routines corresponding to the QTCM1 variable type (scalar
vs.\ array, and real, integer, or string).  On the Python side, add
an entry in \mods{defaults} corresponding to the new field variable
you've created access to.  I would strongly recommend making the
Python name of your new field variable
(given in \mods{defaults}) be the same as the compiled
QTCM1 model variable name.



	\subsection{Fortran Module \mods{WrapCall}}   \label{sec:wrapcall}

Most of the time, if you want to call a compiled QTCM1 model subroutine
from the Python level, you will use the version of the subroutine that
is found in this Fortran module.  
Note that the Fortran module \mods{WrapCall} is referred
to in lowercase at the Python level, i.e., as the
attribute \vars{\_\_.qtcm.wrapcall} of a \class{Qtcm} instance.

All the routines in this module do is wrap one of the compiled QTCM1
model routines.  For instance, \mods{WrapCall} subroutine
\mods{wadvcttq} is defined as just:

% --- Two versions of this block, one for display in PDF and the other
%     for display in HTML:
%
\begin{latexonly}
\begin{codeblock}
\codeblockfont{%
Subroutine wadvcttq \\
\hspace*{3ex}Call advcttq \\
End Subroutine wadvcttq}
\end{codeblock}
\end{latexonly}

\begin{htmlonly}
\begin{rawhtml}
<p><code><font color="blue">Subroutine wadvcttq<br>
&nbsp;&nbsp;&nbsp;Call advcttq<br>
End Subroutine wadvcttq</font></code></p>
\end{rawhtml}
\end{htmlonly}

All subroutines in this module begin with ``w'', with the rest of
the name being the Fortran QTCM1 subroutine name.  The calling
interface for the ``w'' version is the same as the Fortran QTCM1
original version.  There are no subroutines in this module that do
not have an exact counterpart in the Fortran QTCM1 code, and thus
this module's subroutines sole purpose is to call other subroutines
in the compiled QTCM1 model.

These wrapper routines are needed because \mods{f2py}, for some
reason I can't figure out, will not properly wrap Fortran routines
(that are then callable at the Python level) that create local
arrays using parameters obtained through a Fortran \mods{use}
statment.  Thus, as an example, a Fortran subroutine \mods{foo}
with the following definition:

% --- Two versions of this block, one for display in PDF and the other
%     for display in HTML:
%
\begin{latexonly}
\begin{codeblock}
\codeblockfont{%
subroutine foo \\
\hspace*{3ex}use dimensions \\
\hspace*{3ex}real a(nx,ny) \\
\hspace*{3ex}[\ldots] \\
end subroutine foo}
\end{codeblock}
\end{latexonly}

\begin{htmlonly}
\begin{rawhtml}
<p><code><font color="blue">
subroutine foo<br>
&nbsp;&nbsp;&nbsp;use dimensions<br>
&nbsp;&nbsp;&nbsp;real a(nx,ny)<br>
&nbsp;&nbsp;&nbsp;[\ldots]<br>
end subroutine foo
</font></code></p>
\end{rawhtml}
\end{htmlonly}


where \vars{nx} and \vars{ny} are defined in the module vars{dimensions},
will return an error, with the result that the extension module
will not be created, or an extension modules that yields output
that is different from running the pure-Fortran version of QTCM1.

By wrapping these calls into this file, I also avoid having to
separate out the Fortran QTCM1 subroutines into separate \fn{.F90}
files.  For Fortran subroutines that you want callable from the
Python level, \mods{f2py} seems to require each Fortran subroutine
to be in its own file of the same name (e.g., the version of
\fn{driver.F90} for this package). If several Fortran subroutines
are all found in a single \fn{.F90} files, \mods{f2py} seems unable
to create wrappers for those subroutines.




%---------------------------------------------------------------------
\section{Python \mods{qtcm} and Pure-Fortran QTCM1 Differences}

This section describes differences between how the \mods{qtcm}
package and the pure-Fortran QTCM1 assign some varables.  A list
of changes to the QTCM1 Fortran Files for use in the \mods{qtcm}
package is found in Section~\ref{sec:f90changes}.


	\subsection{QTCM1 \mods{driverinit}}   \label{sec:driverinit.diffs}

In the pure-Fortran version of QTCM1, by default, the following variables are
set by reference (as given below), not by value, in the \mods{driverinit}
routine:\footnote%
	{In the pure-Fortran version of QTCM1, this routine is found
	in \fn{driver.F90}.}
\begin{codeblock}
\codeblockfont{%
lastday\thinspace=\thinspace{daysperyear} \\
viscxu0\thinspace=\thinspace{viscU} \\
viscyu0\thinspace=\thinspace{viscU} \\
visc4x\thinspace=\thinspace{viscU} \\
visc4y\thinspace=\thinspace{viscU} \\
viscxu1\thinspace=\thinspace{viscU} \\
viscyu1\thinspace=\thinspace{viscU} \\
viscxT\thinspace=\thinspace{viscT} \\
viscyT\thinspace=\thinspace{viscT} \\
viscxq\thinspace=\thinspace{viscQ} \\
viscyq\thinspace=\thinspace{viscQ}}
\end{codeblock}

Thus, in pure-Fortran QTCM1, if you change \vars{daysperyear},
\vars{viscU}, etc.
and recompile (as needed), you will automatically change 
\vars{lastday}, \vars{viscxu0}, etc.
(Though, in the pure-Fortran QTCM1, the default values may be overwritten by
namelist input values.)

The \mods{driverinit} routine is eliminated
in the Python \code{qtcm} package.  Instead, inital values 
of field variables are specified in the \mods{defaults} submodule
and set by value to attributes of the \code{Qtcm} instance.
Thus, for instance, in a \class{Qtcm} instance, \code{lastday} 
is set to \code{365} by default, not to some variable
\vars{daysperyear}.  For the diffusion and viscosity terms,
the \class{Qtcm} instance attributes corresponding to those
terms are set to literals.\footnote%
	{Those literals are defined by \mods{defaults} private
	module variables \vars{\_\_viscT}, \vars{\_\_viscQ},
	and \vars{\_\_viscU}.}

In contrast, in the pure-Fortran QTCM1,
\mods{driverinit} declares local
variables \code{viscU}, \code{viscT}, and \code{viscQ},
and reads values into those variables via the input namelist.
Those values are then used to set
\vars{viscxu0}, \vars{viscyu0}, etc., as described above.
In pure-Fortran QTCM1, \code{viscU}, \code{viscT}, and \code{viscQ}
are not directly accessed anywhere else in the model.
Thus, \code{viscU}, \code{viscT}, and \code{viscQ} are not
defined as field variables in the \code{qtcm} package, and
\class{Qtcm} instances do not have attributes corresponding
to those names.
Additionally, if you wish to change a viscosity parameter
\vars{visc*} (given above), the parameter for each direction
must be set one-by-one even if the flow is isotropic.


	\subsection{The \mods{varinit} Routine}

One of the functions of the pure-Fortran QTCM1 \mods{varinit}
subroutine is to associate the pointer variables \vars{u1}, \vars{v1},
\vars{q1}, and \vars{T1}.  For the extension modules in the \mods{qtcm}
package, a Fortran subroutine \mods{varptrinit} is added that can
also do this association.  This subroutine is called in the
\class{Qtcm} instance method
\latexhtml{\mods{varinit}%
		\footnote{http://www.johnny-lin.com/py\_docs/qtcm/doc/html-api/qtcm.qtcm.Qtcm-class.html\#varinit}}%
	{\htmladdnormallink{\mods{varinit}}{http://www.johnny-lin.com/py_docs/qtcm/doc/html-api/qtcm.qtcm.Qtcm-class.html#varinit}}
(which duplicates and
extends the function of its pure-Fortran counterpart, enabling
alternative ways of handling restart).

The \mods{varptrinit} is not accessed via \mods{wrapcall}.  Remember
that \mods{wrapcall} contains only those routines that were in the
original pure-Fortran QTCM1 code, and that we want to have access
to at the Python level.


	\subsection{The \mods{qtcm} Method of \class{Qtcm}}

The \class{Qtcm} method \mods{qtcm} duplicates the functionality
of the \mods{qtcm} subroutine in the pure-Fortran QTCM1 model.
There are a few differences, however.  First, the \mods{qtcm} method
for \class{Qtcm} instances does not include a call to \mods{cplmean},
which uses mean surface flux for air-sea coupling.  This state is
consistent with the pure-Fortran QTCM1 pre-processor macro
\vars{CPLMEAN} being off.  Thus, if you wish to use mean surface
flux for air-sea coupling, you will have to revise the \mods{qtcm}
method of \class{Qtcm} to call \mods{cplmean}.  You'll also have to
check for any other code additions needed that are associated with
the \vars{CPLMEAN} macro.

Second, the \mods{qtcm} method for \class{Qtcm} instances does not
include the option of not using the atmospheric boundary layer
model.  This is consistent with macro \vars{NO\_ABL} being off.  If
you wish to have no atmospheric boundary layer model, change the
run list \vars{atm\_bartr\_mode} so that the \mods{wsavebartr} and
\mods{wgradphis} routines are not called.  You'll also have to check
for any other code additions needed that are associated with the
\vars{NO\_ABL} macro.



	\subsection{Miscellaneous Differences}

\begin{itemize}
\item In Python \class{Qtcm} instances,
	\vars{dateofmodel} is set to 0 by default.  
	In contrast, in the compiled QTCM1 model,
	the default (i.e., initial value) is calculated from 
	\vars{day0}, \vars{month0}, and \vars{year0}.
	See Section~\ref{sec:init.compiledform.full} for details.

\item The \class{Qtcm} instance attribute
	\vars{\_\_qtcm} is not copyable using \mods{copy.deepcopy}.

\item In general, when executing a \class{Qtcm} instance method, 
	if you change a \class{Qtcm} instance attribute 
	that has a counterpart in the compiled QTCM1 model,
	the compiled QTCM1 counterpart is not changed until the
	end of the method.  Likewise, if you call a compiled QTCM1 model
	subroutine and change a compiled QTCM1 model variable with
	a \class{Qtcm} instance counterpart, the \class{Qtcm}
	instance counterpart is not changed until the end of the
	subroutine.

\item In general, even though some of the compiled QTCM1 model
	Fortran subroutines/functions have counterparts in \class{Qtcm}
	that duplicate the former's functionality, the Fortran
	versions are kept intact so that the
	\vars{compiled\_form\thinspace=\thinspace'full'} case will work.
\end{itemize}




%---------------------------------------------------------------------
\section{Considerations When Adding Fortran Code}

In this section I describe issues to consider if you wish to add
your own compiled code to the package as separate extension modules.
(This is different from creating new standard extension modules,
which is described in Section~\ref{sec:create.new.so}.):

\begin{itemize}
\item The \class{Qtcm} class assumes that the directory path 
	to the original shared object file is the same as for the 
	\mods{package\_version} module.

\item If you want to be able to pass other Fortran variables 
	in and out to/from Python, please see the 
	Section~\ref{sec:setbypy}
	discussion of the Fotran \mods{SetByPy} module.

\item Fortran and Python routines to get and set compiled QTCM1 model
	arrays are currently written only for floating point array.

\item If you ever change 
	\class{Qtcm} instance method
	\mods{\_set\_qtcm\_array\_item\_in\_model}
	to work with non-floating point values, you will also
	have to change the array handling section in 
	\mods{set\_qtcm1\_item}.

\item The restart mechanism in the pure-Fortran QTCM1 model is 
	\emph{not} bit-for-bit correct.  Thus, if you compare the final
	output from a 40 day run with a 30 day run restarted from
	a 10 day run, the output will not be the same.
	This behavior has been duplicated in \class{Qtcm} 
	instances when the \vars{mrestart} flag is used
	and applicable.

\item When creating new extension modules using the \fn{src} makefile,
	be sure you first use the \cmd{make clean} command to clean-up
	any old files.

\end{itemize}




%---------------------------------------------------------------------
\section{Creating New Standard Extension Modules}   \label{sec:create.new.so}

The steps involved in creating the standard extension modules (e.g.,
\fn{\_qtcm\_full\_365.so}, etc.) on installation are given in
Section~\ref{sec:create.so}.  The makefile provided in \fn{/buildpath/src}
uses a Fortran compiler to create the object code, runs \mods{f2py}
to create the shared object file in \fn{src}, and moves the shared
object files into \fn{../lib}, overwriting any pre-existing files
of the same name.  In this section, I describe the makefile and
\mods{f2py} in a little more detail, in case you wish to create
standard extension modules with additions from the ones the default
makefile creates.


	\subsection{Makefile Rules}    \label{sec:makefile.rules}

This section describes the rules of the
makefile found in the \fn{src} directory
of the \mods{qtcm} distribution.  
This makefile is used by the Python package to create the extension
module (\fn{.so} files) imported and used by \mods{qtcm} objects
(as described in Section~\ref{sec:create.so}).
The makefile will, in general, be used only during \mods{qtcm}
installation, but if you wish to recompile the QTCM1 libraries
and make changes in the Python extension module,
you'll want to use/change this makefile.

\begin{description}
\item[clean] Removes old files in preparation for compiling new
	extension modules.

\item[libqtcm.a] Creates library \fn{libqtcm.a} that contains all
	QTCM1 object files in the directory \fn{src},, except
	\fn{setbypy.o}, \fn{wrapcall.o}, \fn{varptrinit.o}, and
	\fn{driver.o}.  This archive is compiled with the netCDF
	libraries.  Previous versions of \fn{libqtcm.a} are overwritten.

\item[\_qtcm\_full\_365.so] Creates the extension module
	\fn{\_qtcm\_full\_365.so}.  \mods{f2py} is run on applicable code
	in \fn{src}, and the extension module is moved to \fn{../lib}.
	Any previous versions of \fn{../lib/\_qtcm\_full\_365.so}
	are overwritten.

\item[\_qtcm\_parts\_365.so] Creates the extension module
	\fn{\_qtcm\_parts\_365.so}.  \mods{f2py} is run on applicable code
	in \fn{src}, and the extension module is moved to \fn{../lib}.
	Any previous versions of \fn{../lib/\_qtcm\_parts\_365.so}
	are overwritten.

\end{description}



	\subsection{Using \mods{f2py}}      \label{sec:using.f2py}

This section briefly describes how \mods{f2py} is used in the
makefile during the creation of the extension modules.
\htmladdnormallink{\mods{F2py}}{http://cens.ioc.ee/projects/f2py2e/} is a
program that generates shared object libraries that allow you to call
Fortran routines in Python.  \mods{F2py} comes with Python's
\htmladdnormallink{NumPy}{http://numpy.scipy.org/}
array handling package, so you do not need to install anything
extra if you have NumPy already installed.

To create the extension modules in \mods{qtcm} using
the makefile described in Section~\ref{sec:makefile.rules},
I use a method similar to the
\latexhtml{``Quick and Smart Way,''\footnote%
{http://cens.ioc.ee/projects/f2py2e/usersguide/index.html\#the-quick-and-smart-way}}%
{\htmladdnormallink{``Quick and Smart Way''}%
{http://cens.ioc.ee/projects/f2py2e/usersguide/index.html#the-quick-and-smart-way}}
described in the \mods{f2py} manual.
For the \fn{\_qtcm\_full\_365.so} extension module, the 
\mods{f2py} call is:

\begin{codeblock}
\codeblockfont{%
f2py --fcompiler=\$(FC) -c -m \_qtcm\_full\_365 driver.F90 $\backslash$ \\
\hspace*{10ex}setbypy.F90 libqtcm.a \$(NCLIB)}
\end{codeblock}

and for the \fn{\_qtcm\_parts\_365.so} extension module, the call is:

\begin{codeblock}
\codeblockfont{%
f2py --fcompiler=\$(FC) -c -m \_qtcm\_parts\_365 $\backslash$ \\
\hspace*{10ex}varptrinit.F90 wrapcall.F90 setbypy.F90 $\backslash$ \\
\hspace*{10ex}libqtcm.a \$(NCLIB)}
\end{codeblock}

For both calls, \vars{FC} and \vars{NCLIB} are the environment
variables in the makefile specifying the Fortran compiler and netCDF
libraries, respectively.  The \vars{-m} flag specifies the extension
module name (without the \fn{.so} suffix).  The \fn{.F90} files
specify the files that have modules and routines that will be
accessible at the extension module level, and the rest of the Fortran
files in QTCM1 are compiled and archived in a library \fn{libqtcm.a}.
For \mods{f2py} to work properly,
the \fn{.F90} files may define \emph{only one} module or routine.

If you add Fortran files containing new modules, and you wish those
modules to be accessible at the Python level, compile your new code
with \mods{f2py}.  If we have a file of such new code, \fn{newcode.F90},
the \mods{f2py} call to create the \fn{\_qtcm\_parts\_365.so}
extension module will become:

\begin{codeblock}
\codeblockfont{%
f2py --fcompiler=\$(FC) -c -m \_qtcm\_parts\_365 $\backslash$ \\
\hspace*{10ex}varptrinit.F90 wrapcall.F90 setbypy.F90 $\backslash$ \\
\hspace*{10ex}newcode.F90 $\backslash$ \\
\hspace*{10ex}libqtcm.a \$(NCLIB)}
\end{codeblock}

If you write new Fortran code for the compiled QTCM1 model that
will \emph{not} be accessed from the Python-level, just add the
object code filename to the variable \vars{QTCMOBJS} in the
makefile; you don't have to do anything else.  If you are adding
Fortran code to existing Fortran modules, it's even easier:  You
don't need change the makefile.  Note that for 64 bit processor
machines, you may have to use \mods{f2py} with the \cmd{-fPIC} flag;
see Section~\ref{sec:sopic} for details on how the lines above will
change.


	\subsection{Two Examples}

\emphpara{A Function:}
Let's say you have written a piece of Fortran code called
\fn{myfunction.F90} that contains one function called
\mods{myfunction}, and you want to have this function
callable from the Python level through the \class{Qtcm} 
instance method \mods{\_\_qtcm.myfunction}.  Do the following:

\begin{enumerate}
\item Move \fn{myfunction.F90} to \fn{src} in the \mods{qtcm}
	distribution directory \fn{/buildpath}.

\item Add \cmd{myfunction.o} to the end of the object file list lines
	after the target names
	\vars{\_qtcm\_full\_365.so} and
	\vars{\_qtcm\_parts\_365.so}.

\item In the
	\vars{\_qtcm\_full\_365.so} and
	\vars{\_qtcm\_parts\_365.so} target descriptions,
	add \cmd{myfunction.F90} to the 
	beginning of the list of \fn{.F90} names 
	in the \mods{f2py} lines.
\end{enumerate}


\emphpara{A Module:} 
Let's say you have written a piece of Fortran code called
\fn{mymodule.F90} that contains the Fortran module \mods{MyModule}
containing multiple routines and variables.  You want to have those
routines and variables callable from the Python level through the
\class{Qtcm} instance attribute \mods{\_\_qtcm.mymodule}.  The steps
to add \mods{MyModule} to the extension modules are exactly the
same as for a single function, with \cmd{mymodule} being
substituted in the makefile everywhere you have \cmd{myfunction}.




%---------------------------------------------------------------------
\section{Attributes and Methods in \class{Qtcm} Instances}

In this section I describe some attributes, particularly private ones,
that may be of interest to developers.
As is the convention in Python, private
attributes and methods are prepended by one or two underscores,
with two underscores being the ``more'' private attribute.
Please see the package
\latexhtml{API documentation%
		\footnote{http://www.johnny-lin.com/py\_pkgs/qtcm/doc/html-api/}}
        {\htmladdnormallink{API documentation}%
		{http://www.johnny-lin.com/py\_pkgs/qtcm/doc/html-api/}}
for details about all variables, including private variables.


	\subsection{Public \mods{num\_settings} Submodule Attributes/Methods}

\begin{itemize}
\item \vars{typecode}:  This module function returns the
	type code of the data array passed in as its argument.

\item \vars{typecodes}:  This dictionary is the same as the
	NumPy (or Numeric and \mods{numarray})
	dictionary \vars{typecodes}, except that the character
	\vars{'S'} and \vars{'c'} are added to the
	\vars{typecodes['Character']} entry, if absent.  This
	functionality is added because I found 
	\vars{typecodes['Character']} had different values in
	Mac OS X and Ubuntu GNU/Linux.
\end{itemize}


	\subsection{Private \mods{qtcm} Submodule Attributes}

This submodule of the package \mods{qtcm} is the module that defines
the \class{Qtcm} class.

\begin{itemize}
\item \vars{\_init\_prog\_dict}:  This dictionary contains
	the default values of all prognostic variables and 
	right-hand sides that can be initialized.  In the
	submodule \mods{qtcm}, it is set to
	the \vars{init\_prognostic\_dict} module variable in
	submodule \mods{defaults}.

\item \vars{\_init\_vars\_keys}:  List of all keys in
	\vars{\_init\_prog\_dict}, plus \vars{'dateofmodel'}
	and \vars{'title'}.  These names correspond to the
	field variables that are usually written out into a
	restart file.

\item \vars{\_test\_field}:  \class{Field} object instance used 
	in type tests.
\end{itemize}



	\subsection{Private \class{Qtcm} Attributes}  
					\label{sec:Qtcm.private.attrib}

\begin{itemize}
\item \vars{\_cont}:  A boolean attribute that is \vars{True}
	if the run session is a continuation run session and
	\vars{False} if not.  Set the value passed in by
	the keyword \vars{cont} when the \mods{run\_session}
	method is executed.

\item \vars{\_monlen}:  Integer array of the number of days in 
	each month, assuming a 365~day year.

\item \vars{\_\_qtcm}:  The extension module that is the
	compiled QTCM1 Fortran model for this instance.
	This attribute is unique for every instance:  The
	extension module \fn{.so} file is first copied to
	a temporary directory (given by the \vars{sodir}
	instance attribute) and then imported to the
	\class{Qtcm} instance.
	This private attribute is set on instantiation.

\item \vars{\_qtcm\_fields\_ids}:  Field ids for all default 
	field variables, set on instantiation.

\item \vars{\_runlists\_long\_names}:  Dictionary holding the
	descriptions of the standard run lists.  The keys of
	the dictionary are the names of the standard run lists.
\end{itemize}




%---------------------------------------------------------------------
\section{Creating Documentation}

The distribution of \mods{qtcm} comes with the full set of
documentation in readable form (PDF and HTML).  The documentation
consists of two kinds:  this User's Guide and the API documentation.
The User's Guide is written in \LaTeX.  The PDF version is generated
directly from \LaTeX, and the HTML version is created by
\LaTeX{2}HTML.

I use the \fn{make\_docs} shell script in \fn{doc} creates all these
documents.  Briefly, that script does the following:

\begin{itemize}
\item In the \fn{doc/latex} directory, uses \cmd{python} to
	run \fn{code\_to\_latex.py}, which generates the
	\LaTeX\ files describing the current \mods{qtcm} 
	package settings, including text in the manual which gives
	all uses of the current version number.

\item \LaTeX\ is run on the \LaTeX\ files in the \fn{doc/latex} directory.
	The PDF generated by the run is moved from \fn{doc/latex} to
	\fn{doc}.

\item \LaTeX{2}HTML is run on the \LaTeX\ files in \fn{doc/latex}.
	The HTML files generated by the run are moved to \fn{doc/html}.

\item \mods{epydoc} is run on the \mods{qtcm} package libraries.
	This is run in \fn{doc}, to make use of the \fn{epydoc}
	configuration file present there.  The syntax from the
	command line is:

\begin{codeblock}
\codeblockfont{%
epydoc -v --config epydocrc [name]}
\end{codeblock}
\vars{[name]} is either \cmd{qtcm}, if the \mods{qtcm} package is
installed in a directory listed in \vars{sys.path}, or 
\vars{[name]} is the name of the directory the \mods{qtcm} package is
located in (e.g., \fn{/usr/lib/python2.4/site-packages/qtcm}).

\end{itemize}

The \fn{make\_docs} script cannot be used without customizing it
to your system, so please \emphpara{DO NOT USE IT} if you do
not know what you are doing.  You could easily wipe out all your
documentation by mistake.





% ===== end of file =====


\chapter{Future Work}                       \label{ch:future}
% ==========================================================================
% Future
%
% By Johnny Lin
% ==========================================================================


% ------ BODY -----
%
This section describes the features and fixes I plan to work on
in this package.  The most urgent items are listed closer to the
begining of the lists.

\begin{itemize}
\item Add \code{implicit none} top setbypy.F90.

\item Check through Fortran routines that have arguments, to make sure
	f2py is properly understanding the intentions
	(i.e., in, out, inout) of the variables, since we're using the
	``quick way'' of making shared object libraries using f2py.
	The \fn{utilities.F90} file has a number of Fortran routines
	with arguments.

\item Cite:  Peterson, P. (2009) 
	F2PY: a tool for connecting Fortran and Python programs, 
	\emph{Int. J. Computational Science and Engineering,}
	Vol.\ 4, No.\ 4, pp.\ 296--305 for f2py.

\item Create a method like \mods{calc\_derived('T100')} which would
	primarily operate on a data file and provide a derived variable
	such as the temperature at 100 hPa, as given in this example.
	Figure out where to put the parameters (V1s, etc.) that are
	needed to make such a calculation.  As attributes?  Create a
	method to write the quantity out to an output file?
	Perhaps make an ability to calculate these values at heights
	at a given time each day during a run session?

\item Automate the installation using Python's
\htmladdnormallinkfoot{\mods{distutils}}{http://docs.python.org/dist/dist.html}
	utilities.

\item Describe a way of using job control (either via the operating system
	or IPython's \mods{jobctrl} module) 
	to do a quick-and-dirty parallelization of multiple
	\class{Qtcm} instance run sessions.  Or use some sort of threading
	to fire up two simulataneously running models.  Check that the
	simultaneously running models have different memory space.

\item Add capability for \fn{create\_benchmark.py} to overwrite
	existing benchmark files.

\item Make \vars{compiled\_form} set to \vars{'parts'} as the
	default instantiation.  Change documentation accordingly.

\item Currently, the \class{Qtcm} \mods{plotm} method works only on
	3-D output (time, latitude, longitude).  Some of the fields
	in the netCDF output files are 2-D.  Add the capability to
	\mods{plot\_netcdf\_output} in the \mods{plot} submodule
	to handle 2-D fields.

\item Add documentation about removing temporary files.
	Add documentation in Section~\ref{sec:model.instances}
	of details of what occurs during instantiation of 
	a \class{Qtcm} instance.

\item Add the units and long names for all field variables in the
	\mods{defaults} module.

\item Create a keyword to automatically change precipitation and
	evaporation units to mm/day (or similar).

\item Add ability to calculate and plot fields at different pressure
	levels.  Create another module like defaults that specifies
	the vertical fields and gives the equation to use to calculate
	those fields; call the module ``derivfields'' or something
	similar.

\item Throughout the \mods{qtcm} package I use the condition
	\mods{N.rank(}\dumarg{arg}\mods{)\thinspace=\thinspace0} 
	to test whether
	\dumarg{arg} is a scalar.  This works fine for \mods{numpy}
	objects, but it does not work properly for
	\mods{Numeric} and \mods{numarray} arrays.  In those
	array packages, \mods{rank('abc')} returns the value~1.
	This is not a problem, as long as everyone has \mods{numpy},
	but in order to make the package interoperable, I need to
	find a better way of testing for scalars.  The definitions
	of isscalar need to be changed in \mods{num\_settings}.

\item \mods{num\_settings} needs to be changed to truly enable me
	to test whether \mods{qtcm} works for 
	\mods{numarray} and \mods{Numeric} arrays.  The tests
	do not do this right now, because \mods{num\_settings}
	defaults to \mods{numpy}, if it exists.

\item Create makefiles for other platforms.
 
\item A few fields (e.g., \vars{u1}) have data for extra latitude bands,
	due to the use of ``ghost latitudes'' as part of the
	implementation of the numerics.  Details are found in the 
\latexhtml{%
\htmladdnormallinkfoot{QTCM1 manual}%
        {http://www.atmos.ucla.edu/$\sim$csi/qtcm\_man/v2.3/qtcm\_manv2.3.pdf}}%
{\htmladdnormallink{QTCM1 manual}%
        {http://www.atmos.ucla.edu/~csi/qtcm_man/v2.3/qtcm_manv2.3.pdf}}
\cite{Neelin/etal:2002}.

	Though adjusting to this idiosyncracy is not that difficult, 
	in the future I hope to implement a method of handing
	fields with ghost latitudes so that they have the same
	dimensions as the other gridded output variables.  In order
	to do this, I plan to write a Python method to read the
	Fortran generated binary restart file.

\item Change the \mods{set\_qtcm\_item} method so that it can 
	automatically accomodate setting Fortran real variables
	if integer values are input.

\item Currently, the \mods{get\_item\_qtcm} and 
	\mods{set\_item\_qtcm} methods will not work
	on integer and character arrays, only scalars and real arrays.
	Add that missing functionality to those methods.

\item Currently, the \mods{make\_snapshot} method duplicates the
	functionality of the pure-Fortran QTCM1 restart file mechanism.
	However, the restart file mechanism itself does not do a true
	restart.  A continuous run does not provide the same results
	as two runs over the same period, joined by the restart file.

	To see whether saving more variables would do the trick,
	I altered \mods{make\_snapshot} to store all Python level
	variables (i.e., \vars{self.\_qtcm\_fields\_ids}).  However,
	the restart failing described above still continued.  In the
	future, I hope to figure out exactly how many variables are
	needed in order to make the restart feature do a true
	restart.

\item Add a test of using the \vars{mrestart\thinspace=\thinspace1}
	restart option.  Does the \fn{qtcm.restart} file need to be
	in the current working directory or another?

\item Add a test in the unit test scripts to
	confirm that the \vars{init\_with\_instance\_state}
	attribute setting only has an effect if 
	\vars{compiled\_form\thinspace=\thinspace'parts'}.

\item Document \vars{tmppreview} keyword in \mods{plot.plot\_ncdf\_output}.

\item Confirm and document that
	for netCDF output, time is model time since dd-mm-yyyy.

\item Add to the \mods{plotm} method the ability to
	plot as text onto the figure the
	runname string and the calling line
	for the plotm method.

\item Couple with the
	\latexhtml{CliMT\footnote{http://maths.ucd.ie/$\sim$rca/climt/}}%
	{\htmladdnormallink{CliMT}{http://maths.ucd.ie/~rca/climt/}}
	climate modeling toolkit.

\item Enable Python to set \vars{arr1name}, etc., which are string
	variables at the Python level.  I haven't really thought through
	how \vars{arr1} variables work with the Python \class{Qtcm}
	instance.

\item Possible:  In the \class{Qtcm} method
	\mods{\_\_setattr\_\_}, add a test to raise an exception
	if the instance tries to set \vars{viscU}, \vars{viscT},
	or \vars{viscQ} as attributes.  Also create a method
	\code{isotropic\_visc} that will set all viscosity parameters
	non-dependent on direction.  See Section~\ref{sec:driverinit.diffs}
	for details.

\item Go through the manual and create HTML-only versions of tables
	that have table numbers (use a similar construct as in
	figure environments).

\item Go through documentation to check that
	output variable names are capitalized consistently.

\item Create way to redirect stdout.

\item Create a step method to run an arbitrary number of timesteps at
	the atmosphere level.

\end{itemize}


% ===== end of file =====





% ----- BACK MATTER OF THE DOCUMENT -----
%
\normalsize
\pagebreak
\bibliographystyle{plain}
\bibliography{/Users/jlin/work/res/bib/master}

%- Uncomment the input line below and comment out the \bibliographystyle
%  and \bibliography lines if you're running this without the master.bib 
%  BibTeX database
%% ==========================================================================
% Manual for QTCM Python Package
%
% Usage:
% - If you are running this on your own system, you will not have a copy of
%   my master.bib BibTeX database.  To run this, you'll have to comment out:
%
%      \bibliographystyle{chicago-jl}
%      \bibliography{/Users/jlin/work/res/bib/master}
%
%   and comment back in:
%
%      % ==========================================================================
% Manual for QTCM Python Package
%
% Usage:
% - If you are running this on your own system, you will not have a copy of
%   my master.bib BibTeX database.  To run this, you'll have to comment out:
%
%      \bibliographystyle{chicago-jl}
%      \bibliography{/Users/jlin/work/res/bib/master}
%
%   and comment back in:
%
%      \input{manual.bbl}
%
%   in this file.  Then you can use pdflatex on this file to get the PDF of
%   the manual.  These 3 lines are in the back matter of the document.
%
% Revision Notes:
% - By Johnny Lin, North Park University, http://www.johnny-lin.com/
% - The chicago BibTeX style is unrecognized by latex2html, so I use
%   the plain style.
% ==========================================================================


% ------ DOCUMENT DEFINITIONS ------
%
\documentclass[12pt]{book}
\usepackage{color}
\usepackage{html}
\usepackage{graphicx}
\usepackage{textcomp}
%\usepackage{comment}    %- Unrecognized by latex2html; its use causes errors
%\usepackage{fancyvrb}   %- Unrecognized by latex2html; its use causes errors


%- Packages unrecognized by latex2html, but causes no error:
%
%\usepackage[letterpaper,margin=1in,includefoot]{geometry}
\usepackage[letterpaper,margin=1.25in]{geometry}
\usepackage{bibnames}
\usepackage{longtable}
\usepackage{multirow}


%+ Comment out explicity margin settings since use package geometry:
%\setlength{\topmargin}{0in}
%\setlength{\headheight}{0in}
%\setlength{\headsep}{0in}
%\setlength{\oddsidemargin}{0in}
%\setlength{\evensidemargin}{0in}
%\setlength{\textheight}{8.5in}
%\setlength{\textwidth}{6.5in}




% ------ COMMANDS AND LENGTHS ------
%
% --- Define colors:  Have to do this because for some reason LaTeX
%     sometimes looks for "BLUE" instead of "blue" and complains when
%     "BLUE" isn't found.
%
\definecolor{Blue}{rgb}{0,0,1}
\definecolor{BLUE}{rgb}{0,0,1}
\definecolor{green}{rgb}{0,0.6,0}
\definecolor{Green}{rgb}{0,0.6,0}
\definecolor{GREEN}{rgb}{0,0.6,0}


% --- Format code blocks.  Currently set to print out the code in just 
%     typewriter font with no box.  Will work the same for pdflatex 
%     and latex2html:
%
%     codeblock:  Environment for blocks of computer code or internet 
%       addresses.
%     codeblockfont:  Sets font for codeblocks.
%
\newenvironment{codeblock}%
	{\begin{quotation}\begin{minipage}[t]{0.9\textwidth}}%
	{\end{minipage}\end{quotation}}
	%{\begin{flushleft}}%
	%{\end{flushleft}}
\newcommand{\codeblockfont}[1]{\textcolor{blue}{\texttt{#1}}}
%     *** Version that only works for pdflatex that puts a box around 
%         the block and centers it (commented out).  Note that using
%         fancyvrb is the better way of creating such a boxed section
%         of code, but fancyvrb isn't recognized by latex2html:
%\newenvironment{codeblock}%
%	{\begin{center}\begin{tabular}{|c|} \hline \\ }%
%	{\\ \\ \hline \end{tabular}\end{center}}
%\newcommand{\codeblockfont}[1]{\parbox{0.8\textwidth}{\texttt{#1}}}


% --- Text titling/emphasis settings:
%
%     emphpara:  Emphasis for the first phrase or sentence of a 
%         paragraph.
%     booktitle:  Formats book titles.
%     tabletitle:  Title for an item block in the information table.
%     paratitle:  Title for a paragraph in an item block in the
%         information table.
%     emphdate:  Emphasize date in paragraph text.
%
%     cmd:  Commands
%     dumarg:  Dummy arguments
%     codearg:  Same as dumarg.
%     fn:  File and directory names
%     screen:  Screen display
%     vars:  Variable and attribute names
%     mods:  Module, subroutine, and method names
%     class:  Class names
%     code:  Generic code (avoid using this)
%
\newcommand{\emphpara}[1]{\textbf{#1}}
\newcommand{\booktitle}[1]{\textit{#1}}
%\newcommand{\tabletitle}[1]{\textsf{\textbf{#1}}}
\newcommand{\paratitle}[1]{\textit{#1}}
\newcommand{\emphdate}[1]{\textbf{#1}}

\newcommand{\code}[1]{\textcolor{blue}{\texttt{#1}}}
\newcommand{\cmd}[1]{\textcolor{blue}{\texttt{#1}}}
\newcommand{\dumarg}[1]{\textit{#1}}
\newcommand{\codearg}[1]{\textit{#1}}
\newcommand{\fn}[1]{\textsf{\textit{#1}}}
\newcommand{\screen}[1]{\textcolor{green}{\texttt{#1}}}
\newcommand{\vars}[1]{\textcolor{blue}{\texttt{#1}}}
\newcommand{\class}[1]{\textcolor{blue}{\texttt{#1}}}
\newcommand{\mods}[1]{\textcolor{blue}{\texttt{#1}}}


% --- Special table formatting:
%
%     tabletitlewidth:  Width for title field of an item block in the 
%         information table.
%     tablebodywidth:  Width for body field of an item block in the 
%         information table.
%     tabletabulardims:  Dimensions for the information table, used in
%         the tabular command.
%     tableitemlinespace:  Vertical spacing between item blocks in the
%         information table.
%     infotitle and infotext:  Used for two-column sub-information 
%         tables found in the body field of the information table.  
%         These are not global lengths but have values specific to the 
%         local context in which they're used.
%
\newlength{\tabletitlewidth}
\settowidth{\tabletitlewidth}{file and directory names}

\newlength{\tablebodywidth}
\setlength{\tablebodywidth}{0.9\textwidth}
\addtolength{\tablebodywidth}{-4ex}
\addtolength{\tablebodywidth}{-\tabletitlewidth}

\newcommand{\tabletabulardims}%
	{p{\tabletitlewidth}@{\hspace{4ex}}p{\tablebodywidth}}

\newcommand{\tableitemlinespace}{\baselineskip}
\newlength{\infotitle}
\newlength{\infotext}


% --- Lengths for formatting:
%
\newlength{\remainder}        % length to describe the residual of the
                              %   linewidth minus \enumlabel
\newlength{\enumlabel}        % length to describe figure sub-label width
                              %   (e.g. "(a)")


% --- TtH stuff:
%
%\def\tthdump#1{#1}


% --- LaTeX2HTML stuff:
%
%     htmlfigcaption:  Formatting for HTML replacement figure captions.
%
\newcommand{\htmlfigcaption}[1]{\parbox[c]{70ex}{\footnotesize{#1}}}


% --- Some book title abbreviations:
%
%     rute:  Booktitle for Rute User's.
%     linuxnut:  Booktitle for Linux in a Nutshell.
%     pynut:  Booktitle for Python in a Nutshell.
%
\newcommand{\rute}{\booktitle{Rute User's}}
\newcommand{\linuxnut}{\booktitle{Linux in a Nutshell}}
\newcommand{\pynut}{\booktitle{Python in a Nutshell}}


% --- Define special characters ---
%
\newcommand{\aonehat}{\ensuremath{\widehat{a_1}}}
\newcommand{\bonehat}{\ensuremath{\widehat{b_1}}}
\newcommand{\D}{\ensuremath{\mathcal{D}}}
\def\BibTeX{B\kern-.03em i\kern-.03em b\kern-.15em\TeX}




% ------ BEGINNING OF DOCUMENT TEXT ------
%
\begin{document}

    

    
% ------ TITLE AND TOC ------
%
\title{\mods{qtcm} User's Guide}
\author{Johnny Wei-Bing Lin\thanks{Physics Department, North Park University,
	3225 W.\ Foster Ave., Chicago, IL  60625, USA}}
\date{\today}
\maketitle
\tableofcontents




% ------ BODY ------
%
\chapter{Introduction}
\input{intro}

\chapter{Installation and Configuration}    \label{ch:install}
	\section{Summary and Conventions}      \label{sec:install.sum}
	\input{install_sum}
	\section{Fortran Compiler}             \label{sec:fort.compilers}
	\input{install_fort}
	\section{Required Packages}            \label{sec:py.etc.pkgs}
	\input{install_pkgs}
	\section{Compiling Extension Modules}  \label{sec:create.so}
	\input{compile_so}
	\section{Testing the Installation}     \label{sec:test.qtcm}
	\input{test_qtcm}
	\section{Model Performance}
	\input{perform}
	\section{Installing in Mac OS X}       \label{sec:install.macosx}
	\input{qtcm_in_macosx}
	\section{Installing in Ubuntu}         \label{sec:install.ubuntu}
	\input{qtcm_in_ubuntu}

\chapter{Getting Started With \mods{qtcm}}  \label{ch:getting.started}
\input{started}

\chapter{Using \mods{qtcm}}                 \label{ch:using}
\input{using}

%@@@\chapter{Combining \code{qtcm} with \code{CliMT}}
%@@@\input{climt}

\chapter{Troubleshooting}                   \label{ch:trouble}
\input{trouble}

\chapter{Developer Notes}                   \label{ch:devnotes}
\input{devnotes}

\chapter{Future Work}                       \label{ch:future}
\input{future}




% ----- BACK MATTER OF THE DOCUMENT -----
%
\normalsize
\pagebreak
\bibliographystyle{plain}
\bibliography{/Users/jlin/work/res/bib/master}

%- Uncomment the input line below and comment out the \bibliographystyle
%  and \bibliography lines if you're running this without the master.bib 
%  BibTeX database
%\input{manual.bbl}        

\appendix
\chapter{Field Settings in \mods{defaults}}  \label{app:defaults.values}
\input{defaults}




% ------ END OF DOCUMENT TEXT ------
%
\end{document}


% ===== end of file =====

%
%   in this file.  Then you can use pdflatex on this file to get the PDF of
%   the manual.  These 3 lines are in the back matter of the document.
%
% Revision Notes:
% - By Johnny Lin, North Park University, http://www.johnny-lin.com/
% - The chicago BibTeX style is unrecognized by latex2html, so I use
%   the plain style.
% ==========================================================================


% ------ DOCUMENT DEFINITIONS ------
%
\documentclass[12pt]{book}
\usepackage{color}
\usepackage{html}
\usepackage{graphicx}
\usepackage{textcomp}
%\usepackage{comment}    %- Unrecognized by latex2html; its use causes errors
%\usepackage{fancyvrb}   %- Unrecognized by latex2html; its use causes errors


%- Packages unrecognized by latex2html, but causes no error:
%
%\usepackage[letterpaper,margin=1in,includefoot]{geometry}
\usepackage[letterpaper,margin=1.25in]{geometry}
\usepackage{bibnames}
\usepackage{longtable}
\usepackage{multirow}


%+ Comment out explicity margin settings since use package geometry:
%\setlength{\topmargin}{0in}
%\setlength{\headheight}{0in}
%\setlength{\headsep}{0in}
%\setlength{\oddsidemargin}{0in}
%\setlength{\evensidemargin}{0in}
%\setlength{\textheight}{8.5in}
%\setlength{\textwidth}{6.5in}




% ------ COMMANDS AND LENGTHS ------
%
% --- Define colors:  Have to do this because for some reason LaTeX
%     sometimes looks for "BLUE" instead of "blue" and complains when
%     "BLUE" isn't found.
%
\definecolor{Blue}{rgb}{0,0,1}
\definecolor{BLUE}{rgb}{0,0,1}
\definecolor{green}{rgb}{0,0.6,0}
\definecolor{Green}{rgb}{0,0.6,0}
\definecolor{GREEN}{rgb}{0,0.6,0}


% --- Format code blocks.  Currently set to print out the code in just 
%     typewriter font with no box.  Will work the same for pdflatex 
%     and latex2html:
%
%     codeblock:  Environment for blocks of computer code or internet 
%       addresses.
%     codeblockfont:  Sets font for codeblocks.
%
\newenvironment{codeblock}%
	{\begin{quotation}\begin{minipage}[t]{0.9\textwidth}}%
	{\end{minipage}\end{quotation}}
	%{\begin{flushleft}}%
	%{\end{flushleft}}
\newcommand{\codeblockfont}[1]{\textcolor{blue}{\texttt{#1}}}
%     *** Version that only works for pdflatex that puts a box around 
%         the block and centers it (commented out).  Note that using
%         fancyvrb is the better way of creating such a boxed section
%         of code, but fancyvrb isn't recognized by latex2html:
%\newenvironment{codeblock}%
%	{\begin{center}\begin{tabular}{|c|} \hline \\ }%
%	{\\ \\ \hline \end{tabular}\end{center}}
%\newcommand{\codeblockfont}[1]{\parbox{0.8\textwidth}{\texttt{#1}}}


% --- Text titling/emphasis settings:
%
%     emphpara:  Emphasis for the first phrase or sentence of a 
%         paragraph.
%     booktitle:  Formats book titles.
%     tabletitle:  Title for an item block in the information table.
%     paratitle:  Title for a paragraph in an item block in the
%         information table.
%     emphdate:  Emphasize date in paragraph text.
%
%     cmd:  Commands
%     dumarg:  Dummy arguments
%     codearg:  Same as dumarg.
%     fn:  File and directory names
%     screen:  Screen display
%     vars:  Variable and attribute names
%     mods:  Module, subroutine, and method names
%     class:  Class names
%     code:  Generic code (avoid using this)
%
\newcommand{\emphpara}[1]{\textbf{#1}}
\newcommand{\booktitle}[1]{\textit{#1}}
%\newcommand{\tabletitle}[1]{\textsf{\textbf{#1}}}
\newcommand{\paratitle}[1]{\textit{#1}}
\newcommand{\emphdate}[1]{\textbf{#1}}

\newcommand{\code}[1]{\textcolor{blue}{\texttt{#1}}}
\newcommand{\cmd}[1]{\textcolor{blue}{\texttt{#1}}}
\newcommand{\dumarg}[1]{\textit{#1}}
\newcommand{\codearg}[1]{\textit{#1}}
\newcommand{\fn}[1]{\textsf{\textit{#1}}}
\newcommand{\screen}[1]{\textcolor{green}{\texttt{#1}}}
\newcommand{\vars}[1]{\textcolor{blue}{\texttt{#1}}}
\newcommand{\class}[1]{\textcolor{blue}{\texttt{#1}}}
\newcommand{\mods}[1]{\textcolor{blue}{\texttt{#1}}}


% --- Special table formatting:
%
%     tabletitlewidth:  Width for title field of an item block in the 
%         information table.
%     tablebodywidth:  Width for body field of an item block in the 
%         information table.
%     tabletabulardims:  Dimensions for the information table, used in
%         the tabular command.
%     tableitemlinespace:  Vertical spacing between item blocks in the
%         information table.
%     infotitle and infotext:  Used for two-column sub-information 
%         tables found in the body field of the information table.  
%         These are not global lengths but have values specific to the 
%         local context in which they're used.
%
\newlength{\tabletitlewidth}
\settowidth{\tabletitlewidth}{file and directory names}

\newlength{\tablebodywidth}
\setlength{\tablebodywidth}{0.9\textwidth}
\addtolength{\tablebodywidth}{-4ex}
\addtolength{\tablebodywidth}{-\tabletitlewidth}

\newcommand{\tabletabulardims}%
	{p{\tabletitlewidth}@{\hspace{4ex}}p{\tablebodywidth}}

\newcommand{\tableitemlinespace}{\baselineskip}
\newlength{\infotitle}
\newlength{\infotext}


% --- Lengths for formatting:
%
\newlength{\remainder}        % length to describe the residual of the
                              %   linewidth minus \enumlabel
\newlength{\enumlabel}        % length to describe figure sub-label width
                              %   (e.g. "(a)")


% --- TtH stuff:
%
%\def\tthdump#1{#1}


% --- LaTeX2HTML stuff:
%
%     htmlfigcaption:  Formatting for HTML replacement figure captions.
%
\newcommand{\htmlfigcaption}[1]{\parbox[c]{70ex}{\footnotesize{#1}}}


% --- Some book title abbreviations:
%
%     rute:  Booktitle for Rute User's.
%     linuxnut:  Booktitle for Linux in a Nutshell.
%     pynut:  Booktitle for Python in a Nutshell.
%
\newcommand{\rute}{\booktitle{Rute User's}}
\newcommand{\linuxnut}{\booktitle{Linux in a Nutshell}}
\newcommand{\pynut}{\booktitle{Python in a Nutshell}}


% --- Define special characters ---
%
\newcommand{\aonehat}{\ensuremath{\widehat{a_1}}}
\newcommand{\bonehat}{\ensuremath{\widehat{b_1}}}
\newcommand{\D}{\ensuremath{\mathcal{D}}}
\def\BibTeX{B\kern-.03em i\kern-.03em b\kern-.15em\TeX}




% ------ BEGINNING OF DOCUMENT TEXT ------
%
\begin{document}

    

    
% ------ TITLE AND TOC ------
%
\title{\mods{qtcm} User's Guide}
\author{Johnny Wei-Bing Lin\thanks{Physics Department, North Park University,
	3225 W.\ Foster Ave., Chicago, IL  60625, USA}}
\date{\today}
\maketitle
\tableofcontents




% ------ BODY ------
%
\chapter{Introduction}
%=====================================================================
% Introduction
%=====================================================================


% ----- BEGIN TEXT -----
%
%---------------------------------------------------------------------
\section{How to Read This Manual}

\emphpara{Most users:} 
Just read 
(1) the installation instructions in Chapter~\ref{ch:install},
(2) Chapter~\ref{ch:getting.started},
which tells you all you need to get started using \mods{qtcm}, and
(3) examples in Section~\ref{sec:cookbook} that give a feel
for how you can use the model.

\emphpara{Users having problems:}
Chapter~\ref{ch:trouble} provides troubleshooting tips for
a few problems.
The detailed description of how the package functions, 
in Chapter~\ref{ch:using}, will probably be more useful.

\emphpara{Developers:}
If you want to change the source code, please read
Chapter~\ref{ch:devnotes}.  Chapter~\ref{ch:future} describes
all the things I'd like to do to improve the package, but haven't
gotten to yet.




%---------------------------------------------------------------------
\section{About the Package}

The single-baroclinic mode
Neelin-Zeng Quasi-Equilibrium Tropical Circulation Model
\latexhtml{(QTCM1)\footnote{http://www.atmos.ucla.edu/$\sim$csi}}%
	{\htmladdnormallink{(QTCM1)}{http://www.atmos.ucla.edu/~csi}}
is a primitive equation-based intermediate-level atmospheric model
that focuses on simulating the tropical atmosphere.  Being more
complicated than a simple model, the model has full non-linearity
with a basic representation of baroclinic instability,
includes a radiative-convective feedback package, and includes a
simple land soil moisture routine (but does not include topography).
A brief, but more detailed, description of QTCM1 is given in
Section~\ref{sec:brief_qtcm}.

\htmladdnormallinkfoot{Python}{http://www.python.org}
is an interpreted, object-oriented, multi-platform,
open-source language that is used in a variety of software applications,
ranging from game development to bioinformatics.
In climate studies, Python has been used as the core language for
data analysis
(e.g., \htmladdnormallinkfoot{Climate Data Analysis Tools}{http://cdat.sf.net}),
visualization
(e.g., \htmladdnormallinkfoot{Matplotlib}{http://matplotlib.sf.net}),
and 
modeling
(e.g., \htmladdnormallinkfoot{PyCCSM}{http://code.google.com/p/pyccsm/}).

In comparison to traditional compiled languages like Fortran,
Python's lack of a separate compile step greatly simplifies the
debugging and testing phases of development, because code snippets
can be testing as code is written.
Python's extensive suite of higher-level tools (e.g., statistics,
visualization, string and file manipulation) accessible at runtime 
enables modeling and analysis to occur concurrently.  

The \mods{qtcm} package is an implementation of the Neelin-Zeng
QTCM1 in a Python object-oriented environment.  The conversion
package
\htmladdnormallinkfoot{\mods{f2py}}{http://cens.ioc.ee/projects/f2py2e/} is
used to wrap the QTCM1 Fortran model routines and manage model
execution using Python datatypes and utilities.  The result is a
modeling package where order and choice of subroutine execution can
be altered at runtime.  Model analysis and visualization can also
be integrated with model execution at runtime.




%---------------------------------------------------------------------
\section{Conventions In This Manual}

	\subsection{Audience}

In this manual I assume you have a rudimentary knowledge of Python.
Thus, I do not describe basic Python data types (e.g., dictionaries,
lists), object framework and syntax (e.g., classes, methods,
attributes, instantiation), module and package importing.  If you
need to brush up (or learn) Python, I'd recommend the following
resources:

\begin{itemize}
\item \htmladdnormallinkfoot{Python Tutorial:}{http://docs.python.org/tut/}
	This tutorial was written by Guido van Rossum, Python's original
	author.

\item \htmladdnormallinkfoot{Instant Hacking:}%
	{http://www.hetland.org/python/instant-hacking.php}
	Learn how to program with Python.

\item \htmladdnormallinkfoot{Dive Into Python:}%
	{http://diveintopython.org/index.html}
	Reasonably complete and cohesive. The entire book is available for 
	free online.

\item \htmladdnormallinkfoot{Handbook of the Physics Computing Course:}%
	{http://www.pentangle.net/python/handbook/}
	Written for a science audience.  Recommended.

\item \latexhtml{CDAT/Python Tips for Earth Scientists:\footnote%
	{http://www.johnny-lin.com/cdat\_tips/}}%
	{\htmladdnormallink{CDAT/Python Tips for Earth Scientists:}%
		{http://www.johnny-lin.com/cdat_tips/}}
	This web site is a FAQ of sorts for people using Python and
	the Climate Data Analysis Tools (CDAT) in the earth sciences,
	and thus focuses on using Python to do science rather than
	the computer science aspects of the language.

\end{itemize}

The purpose of this package is to make the QTCM1 model easier to
use.  In order to interpret the results, however, you still need
to have a robust sense of what climate models can and cannot tell
you.  A starting point for the QTCM1 model is the brief description
of the model in Section~\ref{sec:brief_qtcm}.  After that, I would
read the original papers describing the model formulation and results
\cite{Neelin/Zeng:2000,Zeng/etal:2000}, and 
\latexhtml{papers citing the model formulation work.\footnote%
{http://scholar.google.com/scholar?hl=en\&lr=\&cites=14217886709842286738}}%
{\htmladdnormallink{papers citing the model formulation work}%
{http://scholar.google.com/scholar?hl=en&lr=&cites=14217886709842286738}.}
Being an intermediate-level model using the quasi-equilibrium assumption,
QTCM1 (and thus \mods{qtcm}) has distinct strengths and limitations; 
please be aware of them.


	\subsection{Typographic Conventions}

\begin{center}
\begin{tabular}{\tabletabulardims}
\cmd{commands} & to be typed at the command-line
	are rendered in a 
	blue, serif, fixed-width typewriter font
	(e.g., \cmd{make \_qtcm\_full\_365}). \\ \hline
\dumarg{dummy arguments} &
	coupled with code or screen display is rendered in a 
	serif, proportional, italicized font
	(e.g., \screen{Error-Value too long in} \dumarg{variable}). \\ \hline
\fn{file and directory names} & are rendered in a 
	sans-serif, italicized font
	(e.g., \fn{setbypy.F90}). \\ \hline
\screen{screen display} & is rendered in a 
	green, serif, fixed-width typewriter font. \\ \hline
\mods{module, method, and subroutine names} & are rendered in a 
	blue, serif, fixed-width typewriter font. \\ \hline
\vars{variable and attribute names} & are rendered in a 
	blue, serif, fixed-width typewriter font. \\ \hline
\class{class names} & are rendered in a 
	blue, serif, fixed-width typewriter font.
\end{tabular}
\end{center}

Blocks of code (usually commands, screen display, and module,
variable, and class names) are displayed in a blue, serif, fixed-width
typewriter font.


	\subsection{Terminology}

\begin{description}
\item[attribute and method references:]
	If there is any possibility of confusion, I will give the
	class that an attribute or method comes from when that
	attribute or method is referenced.  If no class is mentioned
	by name or context,
	assume that the attribute/method comes from the
	\class{Qtcm} class.

\item[``compiled QTCM1 model'':]
	This usually is used to denote when I'm talking about
	compiled Fortran QTCM1 routines and variables therein,
	as an extension module to the Python \mods{qtcm} package..
	Thus, ``compiled QTCM1 model \vars{u1}'' is the value
	of variable \vars{u1} in the Fortran model, not the
	value at the Python-level.  Sometimes I refer to the
	compiled QTCM1 model as just ``QTCM1'' or as
	``compiled QTCM1 Fortran model''.

\item[``pure-Fortran QTCM1'':]
	This refers to the Neelin-Zeng QTCM1 model in it's
	original Fortran form, not as an extension module to
	the Python \mods{qtcm} package.

\item[``Python-level'':]
	This usually denotes the value of a variable as an
	attribute of a \class{Qtcm} instance.  This variable
	is stored at the Python interpreter level.

\item[\class{Qtcm}:]
	This refers to the Python \class{Qtcm} class
	(note the capitalized first letter).

\item[\mods{qtcm}:]
	This refers to the Python \mods{qtcm} package.

\item[QTCM1 vs.\ QTCM:]
	Although the QTCM1 is currently the only version of a
	quasi-equilibrium tropical circulation model (QTCM), in
	principle one can construct a QTCM with any number of
	baroclinic modes.  In anticipation of this, the names of
	the Python package and class do not end in a numeral.  In
	this manual and the \mods{qtcm} docstrings, QTCM and QTCM1
	are used interchangably.
	Usually either of these phrases mean the quasi-equilibrium
	tropical circulation model in a generic sense, regardless
	of its form of implementation.
\end{description}




%---------------------------------------------------------------------
\section{Current Version Information and Acknowledgments}  \label{sec:ver}

\input{pkg_version_date.tex}
\input{pkg_author.tex}is the primary author of the package.

The \mods{qtcm} package is built upon the pure-Fortran QTCM1 model,
version 2.3 (August 2002), with a few minor changes.
Those changes are described in detail in
Section~\ref{sec:f90changes}.

The homepage for the \mods{qtcm} package is
\htmladdnormallink{http://www.johnny-lin.com/py\_pkgs/qtcm}%
	{http://www.johnny-lin.com/py_pkgs/qtcm}.
All Python code in this package, 
and the Fortran files \fn{setbypy.F90} and \fn{wrapcall.F90},
are \copyright\ 2003--2008 by 
\htmladdnormallinkfoot{Johnny Lin}%
		{http://www.johnny-lin.com} 
and constitutes a
library that is covered under the GNU Lesser General Public License
(LGPL):

\begin{quotation}
	This library is free software; you can redistribute it
	and/or modify it under the terms of the 
	\htmladdnormallinkfoot{GNU Lesser General Public License}%
		{http://www.gnu.org/copyleft/lesser.html} 
	as published by
	the Free Software Foundation; either version 2.1 of the
	License, or (at your option) any later version.

	This library is distributed in the hope that it will be
	useful, but WITHOUT ANY WARRANTY; without even the implied
	warranty of MERCHANTABILITY or FITNESS FOR A PARTICULAR
	PURPOSE. See the GNU Lesser General Public License for more
	details.

	You should have received a copy of the GNU Lesser General
	Public License along with this library; if not, write to
	the Free Software Foundation, Inc., 59 Temple Place, Suite
	330, Boston, MA 02111-1307 USA.

	You can contact Johnny Lin at his email address 
	or at North Park University, Physics Department,
	3225 W. Foster Ave., Chicago, IL 60625, USA.  
\end{quotation}

All other Fortran code in this package, as well as the makefiles,
are covered by licenses (if any) chosen by their respective owners.

This manual, in all forms (e.g., HTML, PDF, \LaTeX),
is part of the documentation of the \mods{qtcm} package 
and is \copyright\ 2007--2008 by Johnny Lin.
Permission is granted to copy, distribute and/or modify 
this document under the terms of the 
GNU Free Documentation License, Version 1.2 
or any later version published by the Free Software Foundation; 
with no Invariant Sections, no Front-Cover Texts, 
and no Back-Cover Texts. 
A copy of the license can be found 
\htmladdnormallinkfoot{here}{http://www.gnu.org/licenses/fdl.html}.

Transparent copies of this document are located online in
\latexhtml{%
\htmladdnormallinkfoot{PDF}%
	{http://www.johnny-lin.com/py\_pkgs/qtcm/doc/manual.pdf}}%
{\htmladdnormallink{PDF}%
	{http://www.johnny-lin.com/py_pkgs/qtcm/doc/manual.pdf}}
and
\latexhtml{%
\htmladdnormallinkfoot{HTML}%
	{http://www.johnny-lin.com/py\_pkgs/qtcm/doc/}}%
{\htmladdnormallink{HTML}%
	{http://www.johnny-lin.com/py_pkgs/qtcm/doc/}}
formats.
The \LaTeX\ source files are distributed with the \mods{qtcm}
distribution.
While the HTML version is nearly identical to the PDF
and \LaTeX\ versions, not every feature in the manual was successfully
converted.  This is particularly true with figures, which are
unnumbered in the HTML version and may be formatted differently
than the authoritative PDF version.
Thus, please consider the \LaTeX\ version as the authoritative
version.

\vspace{\baselineskip}

\emphpara{Acknowledgements:}
Thanks to David Neelin and Ning Zeng and the Climate Systems
Interactions Group at UCLA for their encouragement and help.
On the Python side,
thanks to Alexis Zubrow, Christian Dieterich, Rodrigo Caballero,
Michael Tobis, and Ray Pierrehumbert for steering me straight.
Early versions of some of this work was carried out 
at the University of Chicago Climate Systems Center, 
funded by the National Science Foundation (NSF) 
Information Technology Research Program under grant ATM-0121028. 
Any opinions, findings and conclusions or recommendations 
expressed in this material are those of the author and 
do not necessarily reflect the views of the NSF.

Intel\textregistered\ and
   Xeon\textregistered\ are registered trademarks of Intel Corporation.
Matlab\textregistered\ is a registered trademark of The MathWorks.
UNIX\textregistered\ is a registered trademark of The Open Group.




%---------------------------------------------------------------------
\section{Summary of Release History}

\begin{itemize}
\item 2008 Sep 12:  Version 0.1.2 released.  Summary of changes:
	\begin{itemize}
	\item Create \class{Qtcm} method \mods{get\_qtcm1\_item}.
		This method is effectively an alias of method 
		\mods{get\_qtcm\_item}.
	\item Create \class{Qtcm} method \mods{set\_qtcm1\_item}.
		This method is effectively an alias of method 
		\mods{set\_qtcm\_item}.
	\item Update User's Guide to phase out references to
		the \mods{get\_qtcm\_item}
		and \mods{set\_qtcm\_item} methods.  
		Adding the ``1'' to the method names makes the purpose
		of the methods clearer.
	\item Add unit tests to cover methods \mods{get\_qtcm1\_item} and
		\mods{set\_qtcm1\_item}.
	\end{itemize}

\item 2008 Jul 30:  Updates to the User's Guide (just the online versions,
        not the copies released with the tarball).

\item 2008 Jul 15:  First publicly available distribution 
	released (v0.1.1).
\end{itemize}




%---------------------------------------------------------------------
\section{A Brief Description of The QTCM1}   \label{sec:brief_qtcm}

This description is copied from Ch.\ 3 of Lin \cite{Lin:2000}, 
with minor revisions.
Model formulation is fully described in
Neelin \& Zeng \cite{Neelin/Zeng:2000} and model
results are described in Zeng et~al.\ \cite{Zeng/etal:2000}.
Neelin \& Zeng \cite{Neelin/Zeng:2000} is based upon v2.0 of QTCM1
and Zeng et~al.\ \cite{Zeng/etal:2000} is based on QTCM1 v2.1.
The 
\latexhtml{%
\htmladdnormallinkfoot{QTCM1 manual}%
	{http://www.atmos.ucla.edu/$\sim$csi/qtcm\_man/v2.3/qtcm\_manv2.3.pdf}}%
{\htmladdnormallink{QTCM1 manual}%
	{http://www.atmos.ucla.edu/~csi/qtcm_man/v2.3/qtcm_manv2.3.pdf}}
\cite{Neelin/etal:2002}
describes the details of model implementation.

QTCM1 differs from most full-scale GCMs primarily in how the vertical
temperature, humidity, and velocity structure of the atmosphere is
represented.  First, instead of representing the vertical structure
by finite-differenced levels, the model uses a Galerkin expansion
in the vertical.  The vertical basis functions are chosen according
to analytical solutions under convective quasi-equilibrium conditions,
so only a few need be retained.
Temperature and humidity are each described by separate
vertical basis functions ($a_1$ and $b_1$, respectively).
Low-level variations in the humidity basis
are larger than in the temperature basis.
For velocity, QTCM1 uses a single baroclinic basis function ($V_1$)
defined consistently with the temperature basis function,
as well as a barotropic velocity mode ($V_0$).
The vertical profiles of $a_1$, $b_1$, and $V_1$
are given in Figure~\ref{fig:qtcm.basis}.
Currently, QTCM1 does not include a separate
vertical degree of freedom describing the PBL.
The horizontal grid spacing of the model is 
$5.625^{\circ}$ longitude by $3.75^{\circ}$ latitude.


% <QTCM1 beta version vertical structure modes>
%
% (1) LaTeX version:
%
\begin{latexonly}
\begin{figure}
   \noindent
   \begin{minipage}[b]{.49\linewidth}
      \settowidth{\enumlabel}{(a) }%
      \setlength{\remainder}{\linewidth}% 
      \addtolength{\remainder}{-\enumlabel}
      {(a)}~\makebox[\remainder]{$a_1$ and $b_1$}
      \centering\includegraphics[width=\linewidth,viewport=58 72 389 344,clip]%
                    {figs/a1b1.pdf}
   \end{minipage}\hfill
   \begin{minipage}[b]{.49\linewidth}
      \settowidth{\enumlabel}{(b) }%
      \setlength{\remainder}{\linewidth}% 
      \addtolength{\remainder}{-\enumlabel}
      {(b)}~\makebox[\remainder]{$V_1$}

      \centering\includegraphics[width=\linewidth,viewport=58 72 389 346,clip]%
                    {figs/V1.pdf}
   \end{minipage}

   \caption{Vertical profiles of basis functions for
		(a) temperature $a_1$ (solid) and humidity $b_1$ (dashed) and
		(b) baroclinic component of
		horizontal velocity $V_1$.}
   \label{fig:qtcm.basis}
\end{figure}
\end{latexonly}

% (2) HTML replacement version:
%
\begin{htmlonly}
\label{fig:qtcm.basis}
\begin{center}
\htmladdimg{../latex/figs/a1b1.png}
\htmladdimg{../latex/figs/V1.png}

\htmlfigcaption{Figure \ref{fig:qtcm.basis}:  
	Vertical profiles of basis functions for
   	(a) temperature $a_1$ (solid) and humidity $b_1$ (dashed) and
   	(b) baroclinic component of
   	horizontal velocity $V_1$.}
\end{center}
\end{htmlonly}


These modes are chosen to accurately capture deep convective regions.
Outside deep convective regions the mode
is simply a highly truncated
Galerkin representation.  The system is much more tightly constrained than
a full-scale GCM, yet hopefully retains the essential dynamics and nonlinear
feedbacks.  The result is that QTCM1 is easier to diagnose than a GCM,
and is computationally fast (about 8 minutes per year on a Sun Ultra 2
workstation).  Zeng et al.\ \cite{Zeng/etal:2000} show results indicating
this intermediate-level model does a reasonable job simulating
tropical climatology and ENSO variability.  


Below is a summary of the main model equations \cite{Neelin/Zeng:2000}:
\begin{equation}
   \partial_t \mathbf{v}_1 
      + \D_{V1} (\mathbf{v}_0 , \mathbf{v}_1)
      + f \mathbf{k} \times \mathbf{v}_1
      =
   - \kappa \nabla T_1 
      - \epsilon_1 \mathbf{v}_1 
      - \epsilon_{01} \mathbf{v}_0
   \label{eqn:barocl_wind}
\end{equation}
\begin{equation}
   \partial_t \zeta_0 
      + \mathrm{curl}_z (\D_{V0} (\mathbf{v}_0 , \mathbf{v}_1))
      + \beta v_0
      =
   - \mathrm{curl}_z (\epsilon_0 \mathbf{v}_0)
      - \mathrm{curl}_z (\epsilon_{10} \mathbf{v}_1)
   \label{eqn:barotr_wind}
\end{equation}
\begin{equation}
   \aonehat (\partial_t + \D_{T1}) T_1 
      + M_{S1} \nabla \cdot {\bf v}_1 
      =
   \langle Q_c \rangle
      + (g/p_T) (-R^\uparrow_t -R^\downarrow_s + R^\uparrow_s + S_t - S_s + H)
   \label{eqn:temperature}
\end{equation}
\begin{equation}
   \bonehat (\partial_t + \D_{q1}) q_1 
      - M_{q1} \nabla \cdot {\bf v}_1 
      =
   \langle Q_q \rangle
      + (g/p_T) E
   \label{eqn:moisture}
\end{equation}
where (\ref{eqn:barocl_wind}) describes the baroclinic wind component,
      (\ref{eqn:barotr_wind}) describes the barotropic wind component,
      (\ref{eqn:temperature}) is the temperature equation, and
      (\ref{eqn:moisture}) is the moisture equation.

In the simplest formulation, the vertically integrated
convective heating and moisture sink
are assumed to be equal and opposite, so:
\begin{equation}
  -\langle Q_q \rangle = \langle Q_c \rangle 
                              = \epsilon^\ast_c (q_1 - T_1)
\end{equation}

For its convective parameterization for $Q_c$, this model uses the
Betts-Miller \cite{Betts/Miller:1986} moist convective
adjustment scheme, a scheme that is also used in some GCMs.
In the convective parameterization, the coefficient
$\epsilon^\ast_c$ is defined as:
\begin{equation}
   \epsilon^\ast_c 
      \equiv 
   \aonehat \bonehat (\aonehat + \bonehat)^{-1} \tau_c^{-1} 
      \mathcal{H}( \mathit{C}_{\mathrm{1}} )
\end{equation}
where $\mathcal{H}( \mathit{C}_{\mathrm{1}} )$ is zero for
$C_{1} \leq 0$, and one for $C_{1} > 0$, and $C_{1}$
is a measure of the convective available potential energy (CAPE),
projected onto the model's temperature and moisture basis functions.

Sensible heat ($H$) and evaporation ($E$) are given as
bulk-aerodynamic formulations:
\begin{equation}
   H
      =
   \rho_a C_D \mathrm{V}_s (T_s - (T_{rs} + a_{1s} T_1))
\end{equation}
\begin{equation}
   E
      =
   \rho_a C_D \mathrm{V}_s (q_\mathit{sat} (T_s) 
      - (q_{rs} + b_{1s} q_1))
\end{equation}

Longwave radiation components are denoted by $R$, and net shortwave
radiation is denoted by $S$.
The terms $\D_{V1}$ and $\D_{V0}$ are the advection-diffusion operators
for the momentum equations (projected onto $V_0$ and $V_1 (p)$,
respectively).
The terms $\D_{T1}$ and $\D_{q1}$ are the
advection-diffusion operators for the temperature and moisture
equations, respectively, using a vertical average projection.
The $\langle X \rangle$ and $\widehat{X}$ operators are
equivalent and denote vertically integration over the troposphere.
Please see Neelin \& Zeng \cite{Neelin/Zeng:2000} and 
Zeng et al.\ \cite{Zeng/etal:2000}
for a more complete description of equations and coefficients.







% ====== end file ======


\chapter{Installation and Configuration}    \label{ch:install}
	\section{Summary and Conventions}      \label{sec:install.sum}
	% ==========================================================================
% Installation Summary
%
% By Johnny Lin
% ==========================================================================


% ------ BODY -----
%

This section provides a summary of the steps needed to install
\mods{qtcm}, and a description of the naming conventions used in
this chapter.  If you have had a decent amount of experience with
Python and installing software on a Unix system, this section will
probably be all you need to read.  The installation steps are:

\begin{enumerate}
\item Install a Fortran compiler (see Section~\ref{sec:fort.compilers}
	for a list of compilers known to work).
	This compiler should be in a directory
	listed in your system path (e.g., \fn{/usr/bin}, etc.).

\item Install all required packages
	(see Section~\ref{sec:py.etc.pkgs} for details):
	Python,
	\mods{matplotlib} (plus the \mods{basemap} toolkit),
	NumPy (which includes \mods{f2py}),
	Scientific Python,
	\LaTeX,
	and
	netCDF.

	Python packages are required to be installed on your
	system in a directory listed in your \vars{sys.path},
	and the other packages/libraries are required to be in 
	standard directories listed in your system path 
	(e.g., \fn{/usr/bin}, \fn{/sw/include}, etc.).

	Make sure the executable for Python can be called at the
	Unix command line by typing both \cmd{python}.
	You might need to define a Unix alias
	that maps \cmd{python2.4} (or whichever version of Python
	you are using) to \cmd{python}.

\item \latexhtml{Download\footnote{http://www.johnny-lin.com/py\_pkgs/qtcm/}}%
        {\htmladdnormallink{Download}{http://www.johnny-lin.com/py_pkgs/qtcm/}}
	the \mods{qtcm} tarball and extract the distribution
	into a temporary directory for building purposes.
	\fn{\input{pkg_distro_dirname}}is the name of
	the \mods{qtcm} distribution directory;
	the number following the hyphen is the
	version number of the distribution.  \label{list:download.qtcm.sum}

	In this manual, the path to \fn{\input{pkg_distro_dirname}}will
	be called the ``\mods{qtcm} build path'' and be given as
	\fn{/buildpath}.  When you see \fn{/buildpath}, please substitute
	the actual temporary directory you created for building purposes.

\item The \mods{qtcm} distribution directory 
	\fn{\input{pkg_distro_dirname}}contains the following 
	principal sub-directories:
	\fn{doc}, \fn{lib}, \fn{src}, \fn{test}.
	Documentation is in \fn{doc},
	all the package modules are in \fn{lib},
	building of extension modules will take place in \fn{src},
	and testing of the package is done in \fn{test}.

\item Compile \mods{qtcm} extension modules in \fn{src}:
	Go to \fn{src}, copy the makefile from
	\fn{src/Makefiles} corresponding to your
	system into \fn{src}, rename to \fn{makefile},
	make changes to the makefile as needed,
	and execute:
	\begin{codeblock}
	\codeblockfont{%
	make clean \\
	make \_qtcm\_full\_365.so \\
	make \_qtcm\_parts\_365.so}
	\end{codeblock}
	If you executed the make commands in \fn{src,},
	the extension modules will be automatically placed in
	\fn{lib} in the \fn{\input{pkg_distro_dirname}}directory.
	See Section~\ref{sec:create.so} for details.
	\label{list:compile.so.sum}

\item Copy the entire contents of \fn{lib} in
	\fn{\input{pkg_distro_dirname}}(not \fn{lib} itself) 
	to a directory named
	\fn{qtcm} that is on your \mods{sys.path}.  For instance,
	for Mac OS X using Fink,
	many Python packages are located in a directory
	named \fn{/sw/\-lib/\-python2.4/\-site-packages}, or something
	similar, and this directory is on the system \mods{sys.path}.  
	If this is the case for your system, copy the
	contents of \fn{lib} into
	\fn{/sw/lib/\-python2.4/\-site-packages/\-qtcm}.
	(For Unix systems, the equivalent directory is usually
	\fn{/usr/\-local/\-lib/\-python2.4/\-site-packages}.)

\item Test the \mods{qtcm} distribution in \fn{test}:
	This step is optional and can take a while.
	Testing requires you to first generate a suite of benchmarks
	using the pure-Fortran QTCM1 model, then running the tests of
	\mods{qtcm} by typing:
	\begin{codeblock}
	\codeblockfont{%
python test\_all.py}
	\end{codeblock}
	at the Unix command line while in \fn{test}.
	See Section~\ref{sec:test.qtcm} for details.

\end{enumerate}

At some point, I will automate the installation using Python's
\htmladdnormallinkfoot{\mods{distutils}}{http://docs.python.org/dist/dist.html}
utilities.



% ===== end of file =====

	\section{Fortran Compiler}             \label{sec:fort.compilers}
	% ==========================================================================
% Fortran compilers
%
% By Johnny Lin
% ==========================================================================


% ------ BODY -----
%

You must have a Fortran compiler installed on your system in order
to compile \mods{qtcm}.  The compiler must be able to interface with
a pre-processor, as QTCM1 makes copious use of pre-processor directives.
\mods{qtcm} is known to work with the following Fortran compilers on the
following platforms:

\begin{center}
\begin{tabular}{l|l|l}
\textbf{Compiler}  & \textbf{Compiler Web Site} & \textbf{Platform(s)} \\ 
\hline
\mods{g95} & \htmladdnormallink{http://www.g95.org/}{http://www.g95.org/}  
	& Mac OS X \\
\end{tabular}
\end{center}

It will probably work with other platforms, but I haven't been able
to test platforms besides those listed above.  Note that \mods{g95}
is not \htmladdnormallink{GNU Fortran}{http://gcc.gnu.org/fortran/}
(\mods{gfortran}), the Fortran 95 compiler included with the more
recent versions of GCC.




% ===== end of file =====

	\section{Required Packages}            \label{sec:py.etc.pkgs}
	% ==========================================================================
% Python packages
%
% By Johnny Lin
% ==========================================================================


% ------ BODY -----
%

The following Python packages are required to be installed on your
system in a directory listed in your \vars{sys.path}:
\begin{itemize}
\item \htmladdnormallinkfoot{Python}%
	{http://www.python.org/}:  The Python programming language
	and interpreter.  Make sure you have a version recent enough
	to be compatible with all the needed Python packages.
\item \htmladdnormallinkfoot{\mods{matplotlib}}%
	{http://matplotlib.sourceforge.net/}:  Scientific plotting
	package, using Matlab-like syntax.  The \mods{basemap} toolkit
	for \mods{matplotlib} must also be installed.
\item \htmladdnormallinkfoot{NumPy}%
	{http://numpy.scipy.org/}:  The standard array package for
	Python.  The module name of NumPy imported in a Python 
	session is \mods{numpy}.
\item \htmladdnormallinkfoot{Scientific Python}%
	{http://dirac.cnrs-orleans.fr/plone/software/scientificpython/}:
	Has netCDF file operators, in addition to other routines
	of use in scientific computing.  The module name of
	Scientific Python imported in a Python session is
	\mods{Scientific}.
\end{itemize}

One other required Python package, \mods{f2py}, is now a part of the
NumPy package, and so installation of NumPy is sufficient to give
you both.

The package \htmladdnormallinkfoot{SciPy}{http://www.scipy.org},
which includes several Python-accessible scientific libraries, also
includes NumPy (and thus \mods{f2py}), so if you install SciPy,
you don't have to install NumPy again.  Note that SciPy is not the
same as Scientific Python; the names are confusing.

A few non-Python packages are also required:
\begin{itemize}
\item \LaTeX: A scientific typesetting program used by the 
	\class{Qtcm} instance method \mods{plotm} to handle 
	exponents and subscripts.  The most common Unix 
	distribution of \LaTeX\ is
	\htmladdnormallinkfoot{teTeX}{http://www.tug.org/teTeX}.

\item netCDF:  This set of libraries enables one to write datasets into
	a platform independent, binary format, with metdata attached.
	The \htmladdnormallinkfoot{netCDF 3.6.2 library}%
        	{http://www.unidata.ucar.edu/software/netcdf/}
	source code can be
\latexhtml{downloaded from UCAR\footnote{http://www.unidata.ucar.edu/downloads/netcdf/netcdf-3\_6\_2/}}%
        {\htmladdnormallink{downloaded from UCAR}{http://www.unidata.ucar.edu/downloads/netcdf/netcdf-3_6_2/}}.
\end{itemize}

For most Unix installations, the easiest way to install all the
above is via a package manager, for instance \mods{apt-get} in
Debian GNU/Linux, \mods{aptitude} or \mods{synaptic} in Ubuntu
GNU/Linux, and \mods{fink} in Mac OS X.  Of course, you can also
download a package's source code and build direct and/or install
using Python's
\htmladdnormallinkfoot{\mods{distutils}}{http://docs.python.org/dist/dist.html}
utilities.




% ===== end of file =====

	\section{Compiling Extension Modules}  \label{sec:create.so}
	% ==========================================================================
% Compiling extension modules
%
% By Johnny Lin
% ==========================================================================


% ------ BODY -----
%

The extension modules (\fn{.so} files) are imported and used by
\mods{qtcm} objects, and contain the Fortran QTCM1 model that is
called by the \mods{qtcm} Python wrappers.  These extension modules
are located in the \fn{lib} directory of the \mods{qtcm} distribution,
and, in general, need to be created only when the \mods{qtcm} package
is installed.

Two extension modules are created:  \fn{\_qtcm\_full\_365.so} and
\fn{\_qtcm\_parts\_365.so}.  Both modules define QTCM1 models where:

\begin{itemize}
\item A year is 365 days long 
	(makefile macro \vars{YEAR360} is off).
\item Model output is written to netCDF files
	(makefile macro \vars{NETCDFOUT} is on).
\item The atmospheric boundary layer model is used
	(makefile macro \vars{NO\_ABL} is off).
\item A global domain is used
	(makefile macro \vars{SPONGES} is off).
\item Topography effects due to induced divergence are not included
	(makefile macro \vars{TOPO} is off).
\item Coupling between atmosphere and ocean is through mean fluxes
	(makefile macro \vars{CPLMEAN} is off).
\item The mixed layer ocean model is not used
	(makefile macros \vars{MXL\_OCEAN} and \vars{BLEND\_SST} are both off).
\end{itemize}

(All other makefile macros not listed are also turned off.)
The only difference between these two extension modules is that the
``full'' module is used by \class{Qtcm} instances where
\vars{compiled\_form} is set to \vars{'full'}, and the ``parts''
module is used by \class{Qtcm} instances where \vars{compiled\_form}
is set to \vars{'parts'}.  See Section~\ref{sec:compiledform} for
details about the \vars{compiled\_form} attribute.

The extension modules are created through the following steps:
\begin{enumerate}
\item Go to the \mods{qtcm} distribution directory
	\fn{\input{pkg_distro_dirname}}located in
	your build path \fn{/buildpath}.  Go to the \fn{src}
	sub-directory.  This is where all the building of the
	extension modules will take place.

\item Copy the makefile that corresponds to your platform to
	the \fn{src} directory, and rename it \fn{makefile}.
	The \fn{Makefiles} sub-directory of \fn{src} contains
	makefiles for various platforms.

\item In \fn{makefile}, make the following changes:
	\begin{enumerate}
	\item Change the \vars{FC} environment variable as needed, 
		if your Fortran compiler is different.
	\item Change the \vars{FFLAGSM} environment variable, if the
		compiler flags listed are not supported by your
		compiler.
	\item Change the \vars{-I} and \vars{-L} parts of the
		\vars{NCINC} and \vars{NCLIB} environment
		variables so that the paths for the netCDF library and
		include files match your system's installation:
		\begin{codeblock}
		\codeblockfont{%
NCINC=-I/yourpath/netcdf/include \\
NCLIB=-L/yourpath/netcdf/lib -lnetcdf}
		\end{codeblock}
		Set \dumarg{yourpath} to the full path to the
		\fn{netcdf} directory where the \fn{include} and
		\fn{lib} sub-directories are that hold the netCDF
		libraries and include files.
		(You shouldn't have to change the \vars{-l} part of
		\vars{NCLIB}, since it is standard to name the netCDF
		library \fn{libnetcdf.a}.  But if you have a non-standard
		installation, change the \vars{-l} part too.)
	\end{enumerate}

\item At the Unix prompt, type:
\begin{codeblock}
\codeblockfont{%
\small
make clean \&\& make \_qtcm\_full\_365.so \&\& make \_qtcm\_parts\_365.so}
\end{codeblock}
	to clean up leftover files from previous compilations, and to
	compile the extension module shared object files
	\fn{\_qtcm\_full\_365.so} and \fn{\_qtcm\_parts\_365.so}.
\end{enumerate}

The makefile will automatically move the shared object files into
\fn{../lib}, overwriting any pre-existing files of the same name.
A detailed description of the makefile and using \mods{f2py} is
given in Section~\ref{sec:create.new.so}, if you wish to create a
different extension module.




% ===== end of file =====

	\section{Testing the Installation}     \label{sec:test.qtcm}
	% ==========================================================================
% Installation Summary
%
% By Johnny Lin
% ==========================================================================


% ------ BODY -----
%

The \mods{qtcm} distribution comes with a set of tests for the
package, using Python's \mods{unittest} package.  
Just to warn you, the tests take around an hour to run.
The tests will not work if the contents of \fn{lib}
after you've finished building \mods{qtcm} have not been copied
to a directory named \fn{qtcm} that is on your \mods{sys.path} path,
so make sure you've gone through all the install steps
(summarized in Section~\ref{sec:install.sum}) before you do these
tests.

\emphpara{NB:}  For these tests to work, both \cmd{python} and
\cmd{python2.4} must refer to the executable for the Python
installation on your system that you are using for running \mods{qtcm}.

The tests require a set of benchmark output files in the
\fn{test/benchmarks} directory in the
\fn{\input{pkg_distro_dirname}}directory (the output will be in
directories whose names begin with ``aquaplanet'' or ``landon'').
These output files are not included with the \mods{qtcm} distribution,
and must be created, by doing the following:

\begin{enumerate}
\item Goto the directory \fn{test/benchmarks/create/src} in the
	\fn{\input{pkg_distro_dirname}}\mods{qtcm} distribution directory,
	and copy the makefile from sub-directory \fn{Makesfiles},
	that corresponds to your system to the
	\fn{test/benchmarks/create/src} directory.  Rename the makefile 
	in \fn{test/benchmarks/create/src} to \fn{makefile}.

\item In \fn{makefile}, make the following changes:
        \begin{enumerate}
        \item Change the \vars{FC} environment variable as needed,
                if your Fortran compiler is different.
        \item Change the \vars{FFLAGSM} environment variable, if the
                compiler flags listed are not supported by your
                compiler.
        \item Change the \vars{-I} and \vars{-L} parts of the
                \vars{NCINC} and \vars{NCLIB} environment
                variables so that the paths for the netCDF library and
                include files match your system's installation:
                \begin{codeblock}
                \codeblockfont{%
NCINC=-I/yourpath/netcdf/include \\
NCLIB=-L/yourpath/netcdf/lib -lnetcdf}
                \end{codeblock}
                Set \dumarg{yourpath} to the full path to the
                \fn{netcdf} directory where the \fn{include} and
                \fn{lib} sub-directories are that hold the netCDF
                libraries and include files.
                (You shouldn't have to change the \vars{-l} part of
                \vars{NCLIB}, since it is standard to name the netCDF
                library \fn{libnetcdf.a}.  But if you have a non-standard
                installation, change the \vars{-l} part too.)
        \end{enumerate}

\item Go to the directory \fn{test/benchmarks/create} in the
	\fn{\input{pkg_distro_dirname}}\mods{qtcm} distribution directory.

\item Type \cmd{python create\_benchmarks.py} at the Unix command line
	to run the benchmark creation script.
\end{enumerate}

The created benchmarks will be located in 
\fn{test/benchmarks}, in directories with names related to the
run that was done, as described earlier.
The benchmarks are created using the
pure-Fortran QTCM1 model code,
version 2.3 (August 2002), with an altered makefile
(described above) and the following code change:
In all \fn{.F90} files, occurrences of:
        \begin{codeblock}
        \codeblockfont{%
        Character(len=130)}
        \end{codeblock}
        are changed to:
        \begin{codeblock}
        \codeblockfont{%
        Character(len=305)}
        \end{codeblock}
This enables the model to properly deal with longer filenames.
The number ``305'' is chosen to make search and replace easier.

Once the benchmarks are created, you can test the \mods{qtcm} package
by doing the following:
\begin{enumerate}
\item Go to the \fn{test} directory in the 
	\fn{\input{pkg_distro_dirname}}directory.
\item Type \cmd{python test\_all.py} at the Unix command line.
\end{enumerate}

If at the end of the test runs you see this message (or something similar):
\begin{codeblock}
\codeblockfont{%
\footnotesize
---------------------------------------------------------------------- \\
Ran 93 tests in 1244.205s \\
 \\
OK}
\end{codeblock}
then everything worked fine!  If you get any other message, the test(s) have
failed.



% ===== end of file =====

	\section{Model Performance}
	%=====================================================================
% Model Performance
%=====================================================================


% ----- BEGIN TEXT -----
%
%---------------------------------------------------------------------

The wall-clock time values below give the mean over three
separate 365 day aquaplanet runs,
using climatological sea surface temperature for lower boundary forcing.
NetCDF output is written daily, for both instantaneous and mean values.
The time step is 1200~sec, and the version of \mods{qtcm} used
is 0.1.1.
The horizontal grid spacing of all model versions is
$5.625^{\circ}$ longitude by $3.75^{\circ}$ latitude.
Values are in seconds:
\begin{center}
\begin{tabular}{p{0.5\linewidth}|c|c|c}
\textbf{System} & \textbf{Pure} & \textbf{Full} & \textbf{Parts} \\
\hline
Mac OS X:  MacBook 1.83 GHz Intel Core Duo running Mac OS X
	10.4.10 with 1 GB RAM
	(Python 2.4.3, NumPy 1.0.3, \mods{f2py} 2\_3816).
    & 152.59 & 153.63 & 158.94 \\
\hline
Ubuntu GNU/Linux:  Dell PowerEdge 860 with 2.66 GHz Quad Core Intel
	Xeon processors (64 bit) running Ubuntu 8.04.1 LTS
	(Python 2.5.2, NumPy 1.1.0, \mods{f2py} 2\_5237).
    & 43.73 & 44.79 & 47.45
\end{tabular}
\end{center}

``Pure'' refers to the pure-Fortran version of QTCM1.
``Full'' refers to a \mods{qtcm} run session with \vars{compiled\_form}
set to \vars{'full'}.  ``Parts'' refers to a \mods{qtcm} run session
with \vars{compiled\_form} set to \vars{'parts'}.
(Section~\ref{sec:compiledform} has details about the difference
between compiled forms.)

The \vars{'parts'} version of \mods{qtcm} gives Python the maximum
flexibility in accessing compiled QTCM1 model subroutines and
variables.  The price of that flexibility is an increase in
run time of approximately 4--9\% over the pure-Fortran version.
The difference in performance between the
\vars{'full'} version of \mods{qtcm} and the pure-Fortran version of
QTCM1 is between negligible and 3\% longer.

To make a timing for the pure-Fortran model, go to
\fn{test/benchmarks/timing/work} in \fn{/buildpath} and run the
\fn{timing\_365.sh} script in that directory.  That script runs the
QTCM1 model using \cmd{/usr/bin/time}, which at the end of the
script will output the amount of time it took to make the model
run.  Run the timing script three times and average the values to
obtain a time comparable to the above.

To make a timing for the \mods{qtcm} model, type \cmd{python
timing\_365.py} while in the \fn{test} directory in \fn{/buildpath}.
Three run sessions will be made for \vars{compiled\_form} equal to
\vars{'full'} and \vars{'parts'}, the times are averaged, and the
value are output at the end of the script.




% ====== end file ======

	\section{Installing in Mac OS X}       \label{sec:install.macosx}
	% ==========================================================================
% Description of installing in Mac OS X
%
% By Johnny Lin
% ==========================================================================


% ------ BODY -----
%
%------------------------------------------------------------------------
\subsection{Introduction}

This section describes issues and a summary of the installation steps
I followed to install \mods{qtcm} on a Mac running OS X.
It is a specific realization of the general installation
instructions found in Sections~\ref{sec:install.sum}--\ref{sec:test.qtcm}.
I first worked through these installation steps during June--July 2007,
with updates during July 2008.
The best way to go through this section is to go through
the summary of the installation steps in 
Section~\ref{sec:osx.install.summary},
and looking back to other sections as needed.




%------------------------------------------------------------------------
\subsection{Platform and Unix Dependencies}

This work was done on a MacBook 1.83 GHz Intel Core Duo running Mac OS X
10.4.11.  My machine has 1 GB RAM and 64 GB of disk in its main partition.

I recommend you turn-off your antivirus software before you
do the installs.  
Problems have been
\latexhtml{reported by Fink users\footnote%
		{http://finkproject.org/faq/usage-fink.php?phpLang=en\#kernel-panics}}%
	{\htmladdnormallink{reported by Fink users}%
		{http://finkproject.org/faq/usage-fink.php?phpLang=en#kernel-panics}}
using the Fink package manager with antivirus software enabled.

There are a variety of dependencies that are required to get your Mac
up-and-running as a scientific computing platform.  The most basic is
installing Apple's 
\htmladdnormallinkfoot{XCode}{http://developer.apple.com/tools/xcode/}
developer tools.\footnote%
	{The package should work in Mac OS X 10.4 with XCode 2.4.1 and higher;
	I've tried it with both 2.4.1 and XCode 2.5.  Note that
	XCode 3.1 only works on Mac OS X 10.5.}
This set of tools contains compilers and libraries
needed to do anything further.  You have to be a member of Apple's
Developer Connection, but registration is free.

Besides XCode, there are a variety of Unix libraries and utilities that you
need.  I first tried installing them by myself, from scratch, into
\fn{/usr/local}, but it was hard to keep track of all the dependencies.
A few that did work, and that I installed from their disk images, are:
\htmladdnormallinkfoot{MacTeX}{http://www.tug.org/mactex/}, 
\htmladdnormallinkfoot{MAMP}{http://www.mamp.info/}, and 
\htmladdnormallinkfoot{Tcl/Tk Aqua BI (Batteries Included)}%
	{http://tcltkaqua.sourceforge.net/}.\footnote%
		{Theoretically you can use Fink to install the equivalent
		of these packages, but I like the specific collection 
		found in these packages.  For instance, Tcl/Tk Aqua BI
		runs natively on the Mac.}

For everything else, thankfully, there's the
\htmladdnormallinkfoot{Fink Project}{http://www.finkproject.org/} which
uses a package manager built upon Debian tools to install ports of
Unix programs onto a Mac.  I just 
\htmladdnormallinkfoot{downloaded}%
	{http://www.finkproject.org/download/index.php?phpLang=en}
a binary version of the Fink 0.8.1 installer for Intel Macs,
installed Fink, and used its package management tools to install
(almost) everything else I needed.\footnote%
	{The one drawback of Fink is that it sometimes
	has stability problems.  In those cases, Fink provides
	command line suggestions to fix the problems, which sometimes
	will work.  If not, sometimes
\latexhtml{%
	deleting Fink and everything it installed,\footnote%
	{http://www.finkproject.org/faq/usage-fink.php?phpLang=en\#removing}}{%
\htmladdnormallink{deleting Fink and everything it installed}
	{http://www.finkproject.org/faq/usage-fink.php?phpLang=en#removing},}
	and starting afresh, will do the trick.
	It also appeared to me that sometimes when I installed 
	multiple packages
	via one \cmd{fink install} call, the installation did not work
	as well as when I installed only one package per call.}

Although you do not need anything besides a Fortran compiler and
the netCDF libraries to run QTCM1 in its pure-Fortran form, in order to
manipulate the model and use this Python version \mods{qtcm}, you
need to have Python installed.  The default Python that comes
with the Mac is a little old, so I used Fink to also install
Python 2.5 and related packages, including
\htmladdnormallinkfoot{matplotlib}{http://matplotlib.sourceforge.net/},
\htmladdnormallinkfoot{ScientificPython}{http://dirac.cnrs-orleans.fr/plone/software/scientificpython/},
and
\htmladdnormallinkfoot{SciPy}{http://www.scipy.org}
(see Section~\ref{sec:osx.summary} for details).




%------------------------------------------------------------------------
\subsection{Fortran Compiler}

There are a variety of high-quality, commercial Fortran compilers.
Unfortunately, because I do not have a research budget, I am not able
to use those compilers.  The 
\htmladdnormallinkfoot{GNU Compiler Collection}{http://gcc.gnu.org/}
(GCC) provides a suite of open-source compilers, some of which are the
standards of their language.  Most of the GCC compilers are installed
on your Mac when you install XCode.

\htmladdnormallinkfoot{GNU Fortran}{http://gcc.gnu.org/fortran/}
(\mods{gfortran}), is the Fortran 95 compiler included with the more
recent versions of GCC.
Unfortunately, I was not able to get it to compile QTCM1.
There is a second open-source Fortran compiler,
\htmladdnormallinkfoot{G95}{http://www.g95.org/} (\mods{g95}),
which some feel is farther along in its development than \mods{gfortran}.
I was able to successfully compile QTCM1 with \mods{g95} on my Mac.
I used Fink to install G95
(see Section~\ref{sec:osx.summary} for details).




%------------------------------------------------------------------------
\subsection{NetCDF Libraries}   \label{sec:netcdf}

For some reason, the netCDF libraries and include files
installed by Fink didn't correspond to the files needed
by the calling routines in \mods{qtcm}.  To solve this, I compiled
my own set of 
\htmladdnormallinkfoot{netCDF 3.6.2 libraries}%
	{http://www.unidata.ucar.edu/software/netcdf/}
using the tarball 
\latexhtml{downloaded from UCAR\footnote{http://www.unidata.ucar.edu/downloads/netcdf/netcdf-3\_6\_2/}}%
        {\htmladdnormallink{downloaded from UCAR}{http://www.unidata.ucar.edu/downloads/netcdf/netcdf-3_6_2/}}.

Once I uncompressed and untarred the package, and went into 
the top-level directory of the package, I built the package by typing
the following at the Unix prompt:

\begin{codeblock}
\codeblockfont{%
./configure --prefix=/Users/jlin/extra/netcdf \\
make check \\
make install}
\end{codeblock}

This installed the netCDF binaries, libraries, and include files into
sub-directories \fn{bin}, \fn{lib}, and \fn{include} in 
the directory specified by \vars{--prefix}.
If you want to install the netCDF libraries in the default
(usually \fn{/usr/local}), just leave out the \vars{--prefix}
option.

Note:  When you build netCDF, make sure the build directory
is not in the directory tree of \vars{--prefix}
(or the default directory \fn{/usr/local}).




%------------------------------------------------------------------------
\subsection{Makefile Configuration}  \label{sec:osx.makefile}

	\subsubsection{NetCDF}

In the \fn{src} directory in the \mods{qtcm} distribution, there is a
sub-directory \fn{Makefiles} that contains the makefiles for a
variety of platforms.  Edit the file \fn{makefile.osx\_g95}
so that the lines specifying the environment variables for the
netCDF libraries and include files:

\begin{codeblock}
\codeblockfont{%
NCINC=-I/Users/jlin/extra/netcdf/include \\
NCLIB=-L/Users/jlin/extra/netcdf/lib -lnetcdf}
\end{codeblock}

are changed to the path where your \emph{manually compiled} 
netCDF libraries and include files are.

Copy \fn{makefile.osx\_g95} from the \fn{Makefiles} sub-directory
in \fn{src} into \fn{src}.  
In other words, from the \mods{qtcm} distribution directory
(i.e., \fn{/buildpath}), at the Unix prompt execute:

\begin{codeblock}
\codeblockfont{%
cp src/Makefiles/makefile.osx\_g95 src/makefile}
\end{codeblock}


	\subsubsection{Linking Order}

Compilers in the GNU Compiler Collection (GCC) search libraries
and object files in the order they are listed in the command-line, 
\latexhtml{from left-to-right\footnote%
        {http://gcc.gnu.org/onlinedocs/gcc-4.1.2/gcc/Link-Options.html\#index-l-670}}%
        {\htmladdnormallink{from left-to-right}{http://gcc.gnu.org/onlinedocs/gcc-4.1.2/gcc/Link-Options.html#index-l-670}}.
Thus, if routines in \fn{b.o} call routines in \fn{a.o}, 
you must list the files in the order \fn{a.o b.o}.

For some reason, that isn't the case for \mods{g95}.  Thus, you will
find \mods{g95} makefile rules structured like the following
(below is part of the rule to create an executable (\fn{qtcm}) for
benchmark runs):

% --- Two versions of this rule, one for display in PDF and the other
%     for display in HTML:
%
\begin{latexonly}
\begin{codeblock}
\codeblockfont{%
qtcm: main.o \\
\hspace*{8ex}\$(FC)~-O~\$(NCINC)~-o~\$@ main.o~\$(QTCMLIB)~\$(NCLIB)}
\end{codeblock}
\end{latexonly}

\begin{htmlonly}
\begin{rawhtml}
<p><code><font color="blue">qtcm: main.o<br>
&nbsp;&nbsp;&nbsp;&nbsp;&nbsp;&nbsp;&nbsp;$(FC) -O $(NCINC) -o 
$@ main.o $(QTCMLIB) $(NCLIB)</font></code></p>
\end{rawhtml}
\end{htmlonly}

even though \fn{main.o} depends on the QTCM library 
(specified in macro setting \vars{\$(QTCMLIB)}), which in turn
depends on the netCDF library (specified in macro setting \vars{\$(NCLIB)}).


%------------------------------------------------------------------------
\subsection{Summary of Steps}   \label{sec:osx.install.summary}

The following summarizes all the steps I took to install
\mods{qtcm} in Mac OS X:

\begin{enumerate}
\item Install
	\htmladdnormallinkfoot{XCode 2.5}%
		{http://developer.apple.com/tools/xcode/}.

\item Install 
	\htmladdnormallinkfoot{MacTeX}{http://www.tug.org/mactex/}, 
	\htmladdnormallinkfoot{MAMP}{http://www.mamp.info/}, and 
	\htmladdnormallinkfoot{TCL/Tk Aqua BI (Batteries Included)}%
		{http://tcltkaqua.sourceforge.net/}.

\item Install
	\htmladdnormallinkfoot{Fink 0.8.1}%
		{http://www.finkproject.org/download/index.php?phpLang=en}.
	Make sure you
	\htmladdnormallink{set up your environment}%
		{http://www.finkproject.org/doc/users-guide/install.php\#setup}
	to enable you to use the packages you install with Fink
	(e.g. \vars{PATH} settings, etc.).
	Most of the time, that just means adding the line
	\cmd{source /sw/bin/init.csh} to your \fn{.cshrc} file (or the
	equivalent in your \fn{.bashrc}).

	Note that for many of the packages needed to run \mods{qtcm},
	you need to 
	\htmladdnormallink{configure Fink to download packages 
		from the unstable trees}%
	{http://www.finkproject.org/faq/usage-fink.php?phpLang=en\#unstable}.
	To do that, add \vars{unstable/main} and \vars{unstable/crypto}
	to the \vars{Trees:} line in \fn{/sw/etc/fink.conf}, and run:

	\begin{codeblock}
	\codeblockfont{fink selfupdate} \\
	\codeblockfont{fink index} \\
	\codeblockfont{fink scanpackages} \\
	\codeblockfont{fink update-all}
	\end{codeblock}

	When \cmd{selfupdate} runs, choose \cmd{rsync} for the
	self update method.  If you do not, Fink will not look in the
	unstable trees for packages.

\item Use Fink to install the \mods{g95} Fortran compiler.
	From a Unix prompt, type:

	\begin{codeblock}
	\codeblockfont{fink -$\,\!$-use-binary-dist install g95}
	\end{codeblock}

\item Use Fink to install Python 
	and the NumPy package (which \mods{f2py} is a part of).
	From a Unix prompt, type:

	\begin{codeblock}
	\codeblockfont{%
	fink -$\,\!$-use-binary-dist install python25 \\
	fink -$\,\!$-use-binary-dist install scipy-core-py25}
	\end{codeblock}

	(Numpy used to be called SciPy Core.)  If you want to
	install Python 2.4 instead, just change the ``25'' and ``py25'' above
	(and in later occurrences) to ``24'' and ``py24'', respectively.
	Note that Fink does not have a version of epydoc for Python 2.4,
	so if you wish to create documentation using epydoc, you will
	need to install Python 2.5.

\item Install teTeX and \LaTeX{2HTML} using Fink.
	From a Unix prompt, type:

	\begin{codeblock}
	\codeblockfont{fink -$\,\!$-use-binary-dist install tetex} \\
	\codeblockfont{fink -$\,\!$-use-binary-dist install latex2html}
	\end{codeblock}

	When prompted, choose ghostscript and ghostscript-fonts to
	satistfy the dependency (which should be the default options).
	I tried choosing system-ghostscript8, but Fink looks for
	ghostscript 8.51 and didn't recognize ghostscript 8.57 that
	was already installed in \fn{/usr/local} (via my MacTeX
	install).  \LaTeX{2HTML} has a package required by the
	\mods{qtcm} manual \LaTeX\ file.

\item Install additional programming and
	scientific packages and libraries using Fink.
	From a Unix prompt, type:

	\begin{codeblock}
	\codeblockfont{%
	fink -$\,\!$-use-binary-dist install scientificpython-py25 \\
	fink -$\,\!$-use-binary-dist install matplotlib-py25 \\
	fink -$\,\!$-use-binary-dist install matplotlib-basemap-py25 \\
	fink -$\,\!$-use-binary-dist install matplotlib-basemap-data-py25 \\
	fink -$\,\!$-use-binary-dist install xaw3d \\
	fink -$\,\!$-use-binary-dist install fftw fftw3 \\
	fink -$\,\!$-use-binary-dist install epydoc-py25 \\
	fink -$\,\!$-use-binary-dist install graphviz \\
	fink -$\,\!$-use-binary-dist install scipy-py25}
	\end{codeblock}

\item Manually install netCDF 3.6.2
	(see Section \ref{sec:netcdf}).

\item From this point on, you can follow the
	general instructions given in Section~\ref{sec:install.sum},
	starting with step~\ref{list:download.qtcm.sum}.
	Please do not ignore, however, Section~\ref{sec:install.macosx}'s
	Mac-specific details.

\end{enumerate}



% ===== end of file =====

	\section{Installing in Ubuntu}         \label{sec:install.ubuntu}
	% ==========================================================================
% Description of installing in Ubuntu
%
% By Johnny Lin
% ==========================================================================


% ------ BODY -----
%
%------------------------------------------------------------------------
\subsection{Introduction}

This section describes installation issues 
I followed to install \mods{qtcm} on my
Dell PowerEdge 860 running Ubuntu GNU/Linux 8.04.1 LTS (Hardy).
The machine has 2.66 GHz Quad Core Intel Xeon processors (64 bit),
4 GB RAM, and 677 GB of disk in its main partition.
This section is a specific realization of the general installation
instructions found in Sections~\ref{sec:install.sum}--\ref{sec:test.qtcm}.
I worked through these installation steps during July 2008.
The best way to go through this section is to go through
the summary of the installation steps in 
Section~\ref{sec:ubuntu.install.summary},
and looking back to other sections as needed.



%------------------------------------------------------------------------
\subsection{Fortran Compiler}     \label{sec:ubuntu.fort.install}

The easiest Fortran compiler to install in Ubuntu 8.04.1 is
\htmladdnormallinkfoot{GNU Fortran}{http://gcc.gnu.org/fortran/}
(\mods{gfortran}), the Fortran 95 compiler included with the more
recent versions of the GNU Compiler Collection (GCC); you can
use any package manager (e.g., \mods{apt-get}, \mods{aptitude})
to install it.
Unfortunately, I was not able to get it to compile QTCM1.
I was, however, able to successfully compile QTCM1 using
the second open-source Fortran compiler,
\htmladdnormallinkfoot{G95}{http://www.g95.org/} (\mods{g95}),
which some feel is farther along in its development than \mods{gfortran}.
G95, however, is not supported as an Ubuntu package, and so I had
to manually install it.

I downloaded the binary version of G95 v0.91 
(the Linux x86\_64/EMT64 with 32 bit default integers) 
using the following
\cmd{curl} command:\footnote%
	{I use \mods{curl} because I usually access my
	Ubuntu server via a terminal session.}

\begin{codeblock}
\codeblockfont{%
\small
curl -o g95.tgz http://ftp.g95.org/v0.91/g95-x86\_64-32-linux.tgz}
\end{codeblock}

which saves the \fn{.tgz} file as the local file \fn{g95.tgz}.
After that, I followed the G95 project's standard
\latexhtml{installation instructions\footnote%
	{http://g95.sourceforge.net/docs.html\#starting}}%
	{\htmladdnormallink{installation instructions}%
		{http://g95.sourceforge.net/docs.html#starting}}
to finish the install.\footnote%
	{The G95 installation instructions say you can put
	\fn{g95-install} anywhere, and make a link to the
	executable \mods{g95} in
	\fn{$\sim$/bin}.  I put \fn{g95-install} in
	\fn{/usr/local}, and while in \fn{/usr/local/bin}, 
	I put a link to the G95 executable using the command:
	\begin{codeblock}
	\codeblockfont{%
	sudo ln -s ../g95-install\_64/bin/x86\_64-suse-linux-gnu-g95 g95.}
	\end{codeblock}}
The regular Linux x86 version of G95
(in \fn{g95-x86-linux.tgz} from the G95 website) did not work on my
machine.




%------------------------------------------------------------------------
\subsection{NetCDF Libraries}   \label{sec:ubuntu.netcdf}

%Here things were very confusing for my machine, as I needed to
%install two versions of the
%\htmladdnormallinkfoot{netCDF}%
%	{http://www.unidata.ucar.edu/software/netcdf/}
%libraries and include files, one 
%for a successful compilation of the extension modules
%(as described in Section~\ref{sec:create.so}),
%and the other 
%for a successful run of the pure-Fortran QTCM1 model
%(used to create the testing benchmarks, as described in
%Section~\ref{sec:test.qtcm}).
%
%The first set of netCDF files (for the extension modules) are
%installed from Ubuntu's package management system.
%These are automatically installed when the \mods{python-netcdf}
%package is installed via an Ubuntu package manager
%(see Section~\ref{sec:ubuntu.install.summary}).
%The include files for this netCDF installation are 
%located in \fn{/usr/include}, and the libraries for this
%netCDF installation are location in \fn{/usr/lib}.

For some reason, the netCDF libraries and include files
installed from the Ubuntu packages do not
correspond to the files needed
by the calling routines in \mods{qtcm}.  To solve this, I compiled
my own set of
\htmladdnormallinkfoot{netCDF 3.6.2 libraries}%
        {http://www.unidata.ucar.edu/software/netcdf/}
using the tarball
\latexhtml{downloaded from UCAR\footnote{http://www.unidata.ucar.edu/downloads/netcdf/netcdf-3\_6\_2/}}%
        {\htmladdnormallink{downloaded from UCAR}{http://www.unidata.ucar.edu/downloads/netcdf/netcdf-3_6_2/}}.

Once I uncompressed and untarred the package, and went into
the top-level directory of the package, I built the package by typing
the following at the Unix prompt:

\begin{codeblock}
\codeblockfont{%
export FC=g95 \\
export FFLAGS="-O -fPIC" \\
export FFLAGS="-fPIC" \\
export F90FLAGS="-fPIC" \\
export CFLAGS="-fPIC" \\
export CXXFLAGS="-fPIC" \\
./configure \\
make check \\
sudo make install}
\end{codeblock}

(The \cmd{export} commands set environment variables for the
Fortran compiler and Fortran and other compiler flags.  The
\vars{-fPIC} flag enables the compilers to create
position independent code, needed for shared libraries in
Ubuntu on a 64 bit Intel processor.)

The above installs the netCDF binaries, libraries, and include files into
sub-directories \fn{bin}, \fn{lib}, and \fn{include} in 
\fn{/usr/local}, the default.
The include files for this netCDF installation are thus
located in \fn{/usr/local/include}, and the libraries for this
netCDF installation are location in \fn{/usr/local/lib}.
(If you want to specify a different installation
location, use the \vars{--prefix} option in \cmd{configure}.)
While you don't have to have root privileges during the configuration
and check steps, you do during the installation step if you're installing
into \fn{/usr/local} (thus the \cmd{sudo} in the last step).\footnote%
	{Note that when you build netCDF, make sure the build directory
	is not in the directory tree of \vars{--prefix}
	or the default directory \fn{/usr/local}.}

%Because there are two different netCDF installations used in the
%\mods{qtcm} package, the makefiles for creating the benchmarks
%and extensions files will have different \vars{NCLIB} and \vars{NCINC}
%environment variables (see Section~\ref{sec:ubuntu.makefile}).




%------------------------------------------------------------------------
\subsection{Makefile Configuration}  \label{sec:ubuntu.makefile}

	\subsubsection{NetCDF}

In the \fn{src} directory in the \mods{qtcm} distribution, there is a
sub-directory \fn{Makefiles} that contains the makefiles for a
variety of platforms.  Edit the file \fn{makefile.ubuntu\_64\_g95}
so that the lines specifying the environment variables for the
netCDF libraries and include files:

\begin{codeblock}
\codeblockfont{%
NCINC=-I/usr/local/include \\
NCLIB=-L/usr/local/lib -lnetcdf}
\end{codeblock}

are changed to the path where your manually compiled
netCDF libraries and include files are.

Copy \fn{makefile.ubuntu\_64\_g95} from the \fn{Makefiles} sub-directory
in \fn{src} into \fn{src}.  
In other words, from the \mods{qtcm} distribution directory
(i.e., \fn{/buildpath}), at the Unix prompt execute:

\begin{codeblock}
\codeblockfont{%
cp src/Makefiles/makefile.ubuntu\_64\_g95 src/makefile}
\end{codeblock}


	\subsubsection{Linking Order}

Compilers in the GNU Compiler Collection (GCC) search libraries
and object files in the order they are listed in the command-line,
\latexhtml{from left-to-right\footnote%
	{http://gcc.gnu.org/onlinedocs/gcc-4.1.2/gcc/Link-Options.html\#index-l-670}}%
	{\htmladdnormallink{from left-to-right}{http://gcc.gnu.org/onlinedocs/gcc-4.1.2/gcc/Link-Options.html#index-l-670}}.
Thus, if routines in \fn{b.o} call routines in \fn{a.o}, 
you must list the files in the order \fn{a.o b.o}.

For some reason, that isn't the case for \mods{g95}.  Thus, you will
find \mods{g95} makefile rules structured like the following
(below is part of the rule to create an executable (\fn{qtcm}) for
benchmark runs):

% --- Two versions of this rule, one for display in PDF and the other
%     for display in HTML:
%
\begin{latexonly}
\begin{codeblock}
\codeblockfont{%
qtcm: main.o \\
\hspace*{8ex}\$(FC)~-O~\$(NCINC)~-o~\$@ main.o~\$(QTCMLIB)~\$(NCLIB)}
\end{codeblock}
\end{latexonly}

\begin{htmlonly}
\begin{rawhtml}
<p><code><font color="blue">qtcm: main.o<br>
&nbsp;&nbsp;&nbsp;&nbsp;&nbsp;&nbsp;&nbsp;$(FC) -O $(NCINC) -o 
$@ main.o $(QTCMLIB) $(NCLIB)</font></code></p>
\end{rawhtml}
\end{htmlonly}

even though \fn{main.o} depends on the QTCM library 
(specified in macro setting \vars{QTCMLIB}), which in turn
depends on the netCDF library (specified in macro setting \vars{NCLIB}).


	\subsubsection{Shared Object PIC}   \label{sec:sopic}

In order to compile the model in Ubuntu on a 64 bit Intel processor,
the model and the netCDF library it is linked to needs to be
compiled to be 
\latexhtml{position independent code (PIC).\footnote%
		{http://www.gentoo.org/proj/en/base/amd64/howtos/index.xml?part=1\&chap=3}}%
	{\htmladdnormallink{position independent code (PIC)}%
		{http://www.gentoo.org/proj/en/base/amd64/howtos/index.xml?part=1&chap=3}.}
This is accomplished with the 
\htmladdnormallinkfoot{\cmd{-fPIC} flag}%
	{http://www.fortran-2000.com/ArnaudRecipes/sharedlib.html}.

In the \mods{qtcm} makefiles, the \cmd{-fPIC} flag is introduced in the
macro \vars{FFLAGSM}, for instance:
\begin{codeblock}
\codeblockfont{%
FFLAGSM = -O -fPIC}
\end{codeblock}
For makefiles used in creating extension modules, \cmd{-fPIC} must
be passed into the \mods{f2py} call.  To do so, put the flags:
\begin{codeblock}
\codeblockfont{%
--f90flags="-fPIC" --f77flags="-fPIC"}
\end{codeblock}
after the \vars{--fcompiler} flag in the \mods{f2py} calling line.

The \cmd{-fPIC} flag must also be used when compiling the netCDF
libraries, as described in Section~\ref{sec:ubuntu.netcdf}.
Failure to create PIC libraries in 64 bit Ubuntu can result in errors 
like the following when creating the \mods{qtcm} extension modules:
\begin{codeblock}
\codeblockfont{%
ld: /usr/local/lib/libnetcdf.a(fort-attio.o): relocation R\_X86\_64\_32 against `a local symbol' can not be used when making a shared object; recompile with -fPIC /usr/local/lib/libnetcdf.a: could not read symbols: Bad value}
\end{codeblock}




%------------------------------------------------------------------------
\subsection{Summary of Steps}      \label{sec:ubuntu.install.summary}

The following summarizes all the steps I took to install
\mods{qtcm} in
Ubuntu 8.04.1 LTS (Hardy) running on a
Quad Core Intel Xeon (64 bit) machine.
Note that while I use the \mods{aptitude} package manager, you are
free to use any manager of your choice (e.g., \mods{apt-get},
\mods{synaptic}, etc.):

\begin{enumerate}
\item Install the G95 Fortran compiler from the binary distribution.
	See Section~\ref{sec:ubuntu.fort.install} for details.

\item Use an Ubuntu package manager
	to install the following packages, by typing:
	\begin{codeblock}
	\codeblockfont{%
sudo aptitude update \\
sudo aptitude install curl \\
sudo aptitude install python-epydoc \\
sudo aptitude install python-matplotlib \\
sudo aptitude install python-netcdf \\
sudo aptitude install python-scientific \\
sudo aptitude install python-scipy \\
sudo aptitude install texlive}
	\end{codeblock}

	Installing \mods{python-scipy} will also install NumPy and
	\mods{f2py}, so you don't have to install the
	\mods{python-numpy} package separately.

	Early-on as I debugged my \mods{qtcm} install on Ubuntu,
	I encountered errors that I thought came from an 
	\htmladdnormallinkfoot{old version of NumPy}%
		{http://cens.ioc.ee/pipermail/f2py-users/2008-June/001617.html},
	and thus I replaced Ubuntu's packaged NumPy with NumPy 1.1.0
	built 
	\latexhtml{directly from source.\footnote%
			{http://sourceforge.net/project/showfiles.php?group\_id=1369\&package\_id=175103}}%
		{\htmladdnormallink{directly from source}{http://sourceforge.net/project/showfiles.php?group_id=1369&package_id=175103}.}
	(Note, you shouldn't install your new NumPy in the default
	location, which may cause problems later-on with Ubuntu's
	package manager.)
	Later on, I concluded the errors I had encountered were not
	because of the NumPy version, but by then I didn't want to
	try to reinstall NumPy again.
	So strictly speaking, the version of Numpy I used is not
	the one bundled with \mods{python-scipy}, but that shouldn't
	be a problem.

\item Manually install netCDF 3.6.2 from source
	(see Section \ref{sec:ubuntu.netcdf}).

\item Manually install the \mods{basemap} package of
	\mods{matplotlib}.  
	The source for the \mods{basemap} toolkit is
	available 
	\latexhtml{from Sourceforge\footnote%
			{http://sourceforge.net/project/showfiles.php?group\_id=80706}}%
		{\htmladdnormallink{from Sourceforge}%
			{http://sourceforge.net/project/showfiles.php?group_id=80706}}
	I obtained version 0.9.9.1 using the
	following \cmd{curl} command:
	\begin{codeblock}
	\codeblockfont{%
\scriptsize
curl -o basemap.tar.gz $\backslash$ \\
http://voxel.dl.sourceforge.net/sourceforge/matplotlib/basemap-0.9.9.1.tar.gz}
	\end{codeblock}

	The \fn{README} file in the \fn{basemap-0.9.9.1} directory has
	detailed installation instructions.  Note that you have to
	install the GEOS library first (\fn{README} has detailed
	directions on how to do that too).  To be on the safe-side,
	I would set the \vars{FC} environment variable to the G95
	compiler
	(e.g., with \cmd{export FC=g95} in Bash).

\item From this point on, you can follow the
	general instructions given in Section~\ref{sec:install.sum},
	starting with step~\ref{list:download.qtcm.sum}.
	Please do not ignore, however, Section~\ref{sec:install.ubuntu}'s
	Ubuntu-specific details.

\end{enumerate}



% ===== end of file =====


\chapter{Getting Started With \mods{qtcm}}  \label{ch:getting.started}
% ==========================================================================
% Getting Started With qtcm
%
% By Johnny Lin
% ==========================================================================


% ------ BODY -----
%
%---------------------------------------------------------------------
\section{Your First Model Run}

Figure~\ref{fig:my.first.run} shows an example of a script to make
a 30 day seasonal, aquaplanet model run, with run name ``test'',
starting from November 1, Year 1.


%--- Two versions, one for PDF, one for HTML:
\begin{latexonly}
\begin{figure}[htp]
\begin{center}
\begin{codeblock}
\codeblockfont{%
from qtcm import Qtcm \\
inputs = \{\} \\
inputs['runname'] = 'test' \\
inputs['landon'] = 0 \\
inputs['year0'] = 1 \\
inputs['month0'] = 11 \\
inputs['day0'] = 1 \\
inputs['lastday'] = 30 \\
inputs['mrestart'] = 0 \\
inputs['compiled\_form'] = 'parts' \\
model = Qtcm(**inputs) \\
model.run\_session()}
\end{codeblock}
\end{center}
\caption{An example of a simple \mods{qtcm} run.}
\label{fig:my.first.run}
\end{figure}
\end{latexonly}

\begin{htmlonly}
\label{fig:my.first.run}
\begin{center}
\htmlfigcaption{%
	\codeblockfont{%
from qtcm import Qtcm \\
inputs = \{\} \\
inputs['runname'] = 'test' \\
inputs['landon'] = 0 \\
inputs['year0'] = 1 \\
inputs['month0'] = 11 \\
inputs['day0'] = 1 \\
inputs['lastday'] = 30 \\
inputs['mrestart'] = 0 \\
inputs['compiled\_form'] = 'parts' \\
model = Qtcm(**inputs) \\
model.run\_session()}
	}

\htmlfigcaption{Figure~\ref{fig:my.first.run}:
	An example of a simple \mods{qtcm} run.}
\end{center}
\end{htmlonly}



The class describing the QTCM1 model is \class{Qtcm}.  An instance
of \class{Qtcm}, in this example \vars{model}, is created the same
way you create an instance of any class.  When instantiating an
instance of \class{Qtcm}, keyword parameters can be used to override
any default settings.  In the example above, the dictionary
\vars{inputs} specifying all keyword parameters is passed in on the
instantiation of \vars{model}.

The keyword parameter settings in
Figure~\ref{fig:my.first.run} have the following meanings:
\begin{itemize}
\item \vars{runname}:  This string (``test'') is used in the
	output filename.  QTCM1 writes mean and instantaneous
	output files to the directory given in \vars{model.outdir.value},
	with filenames 
	\fn{qm\_}\dumarg{runname}\fn{.nc} for mean output and
	\fn{qi\_}\dumarg{runname}\fn{.nc} for instantaneous output.

\item \vars{landon}: When set to ``0'', the land is turned off and
	the run is an aquaplanet run.  When set to ``1'', the land
	model is turned on.

\item \vars{year0}:  The year the run starts on.

\item \vars{month0}:  The month the run starts on (11 = November).

\item \vars{day0}: The day of the month the run starts on.

\item \vars{lastday}:  The model runs from day 1 to \vars{lastday}.

\item \vars{mrestart}:  When set to ``0'', the run starts from
	default initial conditions
	(see Section~\ref{sec:initial.variables} for a table of
	those values).
	When set to ``1'', the run starts from a restart file.

\item \vars{compiled\_form}:  This keyword sets what form the
	compiled QTCM1 model has, and its value is saved to
	the instance's \vars{compiled\_form} attribute.
	It is a string and can be set either to
	``parts'' or ``full''.  Most of the time, you will want
	to set it to \vars{'parts'}.
	This keyword is the only one
	that must be specified on instantiation; the model instance
	will at least instantiate
	using only the default settings for all the other keyword
	parameters (given in Appendix~\ref{app:defaults.values}).
	See Section~\ref{sec:compiledform} for details about
	what the \vars{compiled\_form} attribute controls.
\end{itemize}

By default, the \vars{SSTmode} attribute, which controls whether the
model will use climatological sea-surface temperatures (SST) 
or real SSTs, is set to the \vars{value} ``seasonal'', thus giving a
run with seasonal forcing at the lower-boundary over the ocean.

This example assumes that the boundary condition files, sea surface
temperature files, and the model output directories are as specified
in submodule \mods{defaults}.  Those values are described in
Section~\ref{sec:defaults.scalar}.




%---------------------------------------------------------------------
\section{Managing Directories}

Most of the time, your boundary condition files and output files
will not be in the locations specified in
Section~\ref{sec:defaults.scalar}, or in the directory your
Python script resides.  The easiest way to tell your \class{Qtcm} 
instance where your input/output files are is to pass them in
as keyword parameters on instantiation.


%--- Two versions, one for PDF, one for HTML:
\begin{latexonly}
\begin{figure}[htp]
\begin{codeblock}
\codeblockfont{%
\small
from qtcm import Qtcm \\
rundirname = 'test' \\
dirbasepath = os.path.join(os.getcwd(), rundirname) \\
inputs = \{\} \\
inputs['bnddir'] = os.path.join( os.getcwd(), 'bnddir', \\
\hspace*{40ex}'r64x42' ) \\
inputs['SSTdir'] = os.path.join( os.getcwd(), 'bnddir', \\
\hspace*{40ex}'r64x42', 'SST\_Reynolds' ) \\
inputs['outdir'] = dirbasepath \\
inputs['runname'] = rundirname \\
inputs['landon'] = 0 \\
inputs['year0'] = 1 \\
inputs['month0'] = 11 \\
inputs['day0'] = 1 \\
inputs['lastday'] = 30 \\
inputs['mrestart'] = 0 \\
inputs['compiled\_form'] = 'parts' \\
model = Qtcm(**inputs) \\
model.run\_session()}
\end{codeblock}

\caption{An example \mods{qtcm} run showing detailed description of
	input and output directories.}
\label{fig:manage.dir.example}
\end{figure}
\end{latexonly}

\begin{htmlonly}
\label{fig:manage.dir.example}
\begin{center}
\htmlfigcaption{%
	\codeblockfont{%
from qtcm import Qtcm \\
rundirname = 'test' \\
dirbasepath = os.path.join(os.getcwd(), rundirname) \\
inputs = \{\} \\
inputs['bnddir'] = os.path.join( os.getcwd(), 'bnddir', \\
\hspace*{40ex}'r64x42' ) \\
inputs['SSTdir'] = os.path.join( os.getcwd(), 'bnddir', \\
\hspace*{40ex}'r64x42', 'SST\_Reynolds' ) \\
inputs['outdir'] = dirbasepath \\
inputs['runname'] = rundirname \\
inputs['landon'] = 0 \\
inputs['year0'] = 1 \\
inputs['month0'] = 11 \\
inputs['day0'] = 1 \\
inputs['lastday'] = 30 \\
inputs['mrestart'] = 0 \\
inputs['compiled\_form'] = 'parts' \\
model = Qtcm(**inputs) \\
model.run\_session()}
	}

\htmlfigcaption{Figure~\ref{fig:manage.dir.example}:
	An example \mods{qtcm} run showing detailed description of
        input and output directories.}
\end{center}
\end{htmlonly}


Figure~\ref{fig:manage.dir.example} shows an example run where those
directories are explicitly specified; in all other aspects, the run
is identical to the one in Figure~\ref{fig:my.first.run}.
In Figure~\ref{fig:manage.dir.example}, output from the model is
directed to the directory described by string variable
\vars{dirbasepath}.  \vars{dirbasepath} is created by joining the
current working directory with the run name given in string variable
\vars{rundirname}.\footnote%
	{The Python \mods{os} module enables platform-independent
	handling of files and directories.  The \mods{os.path.join}
	function resolves paths without the programmer needing to know
	all the possible directory separation characters; the function
	chooses the correct separation character at runtime.  The
	\mods{os.getcwd} function returns the current working directory.}
Setting keyword parameter \vars{outdir} to \vars{dirbasepath} sends
output to \vars{dirbasepath}.  
Keywords \vars{bnddir} and \vars{SSTdir} specify the directories
where non-SST and SST boundary condition files, respectively, are
found.

Interestingly, the default version of QTCM1 does \emph{not} send
all output from the model to \vars{outdir}.  The restart file
\fn{qtcm\_}\dumarg{yyyymmdd}\fn{.restart} (where \dumarg{yyyymmdd}
is the year, month, and day of the model date when the restart
file was written) is written into the current working directory,
not the output directory.  Thus, if you do multiple runs, you'll
have to manually deal with the restart files that will proliferate.

Neither the QTCM1 model nor the \class{Qtcm} object
create the directories specified in \mods{bnddir}, \mods{SSTdir},
and \mods{outdir}.  Failure to do so will create an error.  I use
Python's file management tools to make sure the output directory
is created, and any old output files are deleted.  Here's an example
that does that, using the \vars{dirbasepath} and \vars{rundirname}
variables from Figure~\ref{fig:manage.dir.example}:

\begin{codeblock}
\codeblockfont{%
\small
if not os.path.exists(dirbasepath):  os.makedirs(dirbasepath) \\
qi\_file = os.path.join( dirbasepath, 'qi\_'+rundirname+'.nc' ) \\
qm\_file = os.path.join( dirbasepath, 'qm\_'+rundirname+'.nc' ) \\
if os.path.exists(qi\_file):   os.remove(qi\_file) \\
if os.path.exists(qm\_file):   os.remove(qm\_file)}
\end{codeblock}




%---------------------------------------------------------------------
\section{Model Field Variables}   \label{sec:field.variables.intro}

The term ``field'' variable refers to QTCM1 model variables that 
are accessible at both the compiled Fortran QTCM1 model-level as
well as the Python \class{Qtcm} instance-level.
Field variables are all instances of the \class{Field} class,
and are stored as attributes of the \class{Qtcm} instance.\footnote%
	{Note non-field variables can also be instances of \class{Field},
	and that \class{Qtcm} instances have other attributes that are
	not equal to \class{Field} instances.}

\class{Field} class instances have the following attributes:
\begin{itemize}
\item \vars{id}:  A string naming the field (e.g., ``Qc'', ``mrestart'').
	This string should contain no whitespace.
\item \vars{value}:  The value of the field.  Can be of any type, though
	typically is either a string or numeric scalar or a numeric array.
\item \vars{units}:  A string giving the units of the field.
\item \vars{long\_name}:  A string giving a description of the field.
\end{itemize}

\class{Field} instances also have methods to return the rank 
and typecode of \vars{value}.

Remember, if you want to access the value of a \class{Field} object,
make sure you access that object's \vars{value} attribute.  
Thus, for example,
to assign a variable \vars{foo} to the
\vars{lastday} value for a given
\class{Qtcm} instance \vars{model}, type the following:
\begin{codeblock}
\codeblockfont{%
foo = model.lastday.value}
\end{codeblock}

For scalars, this assignment sets \vars{foo} by value (i.e., a copy
of the value of attribute \vars{model.lastday} is set to \vars{foo}).
In general, however, Python assigns variables by reference.  Use
the \mods{copy} module if you truly want a copy of a field variable's
value (such as an array), rather than an alias.  For more details
about field variables, see Section~\ref{sec:field.variables}.




%---------------------------------------------------------------------
\section{Run Sessions}

	\subsection{What is a Run Session?}

A run session is a unit of simulation where the model is run from
day 1 of simulation to the day specified by the \vars{lastday}
attribute of a \class{Qtcm} instance.  A run session is a
``complete'' model run, at the beginning of which all compiled QTCM1
model variables are set to the values given at the Python-level,
and at the end of which restart files are written, the values
at the Python-level are overwritten by the values in the Fortran
model, and a Python-accessible snapshot is taken of the 
model variables that were written to the restart file.


	\subsection{Changing Variables}

Between run sessions, changing any field variable is as easy
as a Python assignment.  For instance, to change the atmosphere
mixed layer depth to 100~m, just type:
\begin{codeblock}
\codeblockfont{%
model.ziml.value = 100.0}
\end{codeblock}

When changing arrays, be careful to try to match the shape of the 
array.\footnote%
	{At the very least, match the rank of the array, which is required
	for the routines in \mods{setbypy} to properly choose which
	Fortran subroutine to use in reading the Python value.
	I haven't tested if only the rank is needed, however,
	for the passing to work, for a continuation run (my hunch is
	it won't).}
You can use the NumPy \mods{shape} function on a NumPy array to
check its shape.


	\subsection{Continuing a Model Run}  \label{sec:continuation.intro}

Figure~\ref{fig:continuation.example} shows an example of two run
sessions, where the second run session is a continuation of the
first.


%--- Two versions, one for PDF, one for HTML:
\begin{latexonly}
\begin{figure}[htp]
\begin{codeblock}
\codeblockfont{%
\small
inputs['year0'] = 1 \\
inputs['month0'] = 11 \\
inputs['day0'] = 1 \\
inputs['lastday'] = 10 \\
inputs['mrestart'] = 0 \\
inputs['compiled\_form'] = 'parts' \\ \\
model = Qtcm(**inputs) \\
model.run\_session() \\
model.u1.value = model.u1.value * 2.0 \\
model.init\_with\_instance\_state = True \\
model.run\_session(cont=30)}
\end{codeblock}

\caption{An example of two \mods{qtcm} run sessions where the second
	run session is a continuation of the first.  Assume 
	\vars{inputs} is a dictionary, and that earlier in the
	script the run name and
	all input and output directory names were added
	to the dictionary.}
\label{fig:continuation.example}
\end{figure}
\end{latexonly}

\begin{htmlonly}
\label{fig:continuation.example}
\begin{center}
\htmlfigcaption{%
	\codeblockfont{%
inputs['year0'] = 1 \\
inputs['month0'] = 11 \\
inputs['day0'] = 1 \\
inputs['lastday'] = 10 \\
inputs['mrestart'] = 0 \\
inputs['compiled\_form'] = 'parts' \\ \\
model = Qtcm(**inputs) \\
model.run\_session() \\
model.u1.value = model.u1.value * 2.0 \\
model.init\_with\_instance\_state = True \\
model.run\_session(cont=30)}
	}

\htmlfigcaption{Figure~\ref{fig:continuation.example}:
	An example of two \mods{qtcm} run sessions where the second
	run session is a continuation of the first.  Assume 
	\vars{inputs} is a dictionary, and that earlier in the
	script the run name and
	all input and output directory names were added
	to the dictionary.}
\end{center}
\end{htmlonly}


The first run session runs from day 1 to day 10.  The second
run session runs the model for another 30 days.  
Setting the \vars{init\_with\_instance\_state} of
\vars{model} to \vars{True} tells the model to use the
the values of the instance attributes 
(for prognostic variables, right-hand sides, and start date) 
are currently stored \vars{model}
as the initial values for the run\_session.\footnote%
	{Unless overridden, by default, 
	\vars{init\_with\_instance\_state} is set
	to True on \class{Qtcm} instance instantiation.}
The \vars{cont}
keyword in the second \mods{run\_session} call specifies a
continuation run, and the value gives the number of additional
days to run the model.

The set of runs described above would produce the exact same
results as if you had gone into the Fortran model after 10 days,
doubled the first baroclinic mode zonal velocity, and continued
the run for another 30 days.  With the Python example above, however,
you didn't need to know you were going to do that ahead of starting
the model run (which is what a compiled model requires you to do).
Section~\ref{sec:contination.run.sessions} describes continuation
runs in detail.


	\subsection{Passing Restart Snapshots Between Run Sessions}
					\label{sec:snapshot.intro}

The pure-Fortran QTCM1 uses a restart file to enable continuation
runs.  A \class{Qtcm} instance can also make use of that option,
through setting the \vars{mrestart} attribute value
(see Section~\ref{sec:contination.run.sessions} and
Neelin et al.\ \cite{Neelin/etal:2002} for details).  
It's easier, however, instead of using a restart file, to pass 
along a ``snapshot'' dictionary.

The \class{Qtcm} instance method \mods{make\_snapshot} copies the
variables that would be written out to a restart file into a
dictionary that is saves as the instance attribute \vars{snapshot}.
This snapshot can be saved separately, for later recall.  Note that
snapshots are automatically made at the end of a run session.

The following example shows a model \mods{run\_session} call,
following which the snapshot is saved to the variable
\vars{snapshot}:\footnote%
	{Remember Python assignment defaults to assignment by
	reference, so in this example the variable \vars{mysnapshot}
	is a pointer to the \vars{model.snapshot} attribute.
	(However, note that \vars{model.snapshot} itself is not a
	reference, but a distinct copy of those variables; to do
	otherwise would result in a non-static snapshot.)
	If the \vars{model.snapshot} attribute is dereferenced,
	then \vars{mysnapshot} will become the sole pointer to the
	dictionary.}

\begin{codeblock}
\codeblockfont{%
model.run\_session() \\
mysnapshot = model.snapshot}
\end{codeblock}

After taking the snapshot, you might continue the run a while, and
then decide to return to the snapshot you saved.  To do so, use
the \mods{sync\_set\_py\_values\_to\_snapshot}
method to reset the model instance values to
\vars{mysnapshot} before your next run session:
\begin{codeblock}
\codeblockfont{%
model.sync\_set\_py\_values\_to\_snapshot(snapshot=mysnapshot) \\
model.init\_with\_instance\_state = True \\
model.run\_session()}
\end{codeblock}

See Section~\ref{sec:snapshots} for details regarding the use of
snapshots, as well as for a list of what variables are saved in
a snapshot.




%---------------------------------------------------------------------
\section{Creating Multiple Models}

	\subsection{Model Instances}

Creating a new QTCM1 model is as simple as creating another
\class{Qtcm} instance.
For instance, to instantiate two QTCM1
models, \vars{model1} and \vars{model2}, type the following:

\begin{codeblock}
\codeblockfont{%
from qtcm import Qtcm \\
model1 = Qtcm(compiled\_form='parts') \\
model2 = Qtcm(compiled\_form='parts')}
\end{codeblock}

\vars{model1} and \vars{model2} do \emph{not} share any variables
in common, including the extension modules holding the Fortran
code.  In creating the instances, a copy of the extension modules
are saved in temporary directories.


	\subsection{Passing Snapshots To Other Models}

The snapshots described in Section~\ref{sec:snapshot.intro}
can also be passed around to other model instances,
enabling you to easily branch a model run:

\begin{codeblock}
\codeblockfont{%
model.run\_session() \\
mysnapshot = model.snapshot \\
model1.sync\_set\_py\_values\_to\_snapshot(snapshot=mysnapshot) \\
model2.sync\_set\_py\_values\_to\_snapshot(snapshot=mysnapshot) \\
model1.run\_session() \\
model2.run\_session()}
\end{codeblock}

The state of \vars{model} after its run session is used to start
\vars{model1} and \vars{model2}.  This is an easy way to save time
in spinning-up multiple models.




%---------------------------------------------------------------------
\section{Run Lists}		\label{sec:runlist.intro}

This feature of \class{Qtcm} objects is what really gives 
\class{Qtcm} model instances their flexibility.
A run list is a list of strings and dictionaries that specify
what routines to run in order to execute a particular part of
the model.  Each element of the run list specifies the method
or subroutine to execute, and the order of the elements specifies
their execution order.

For instance, the standard run list for initializing the the
atmospheric portion of the model is named ``qtcminit'', and
equals the following list:

\begin{latexonly}
\begin{codeblock}
\codeblockfont{%
\parbox{46ex}{\input{qtcminit_runlist}}}
\end{codeblock}
\end{latexonly}

\begin{htmlonly}
\begin{quotation}
\input{qtcminit_runlist}
\end{quotation}
\end{htmlonly}

This list is stored as an entry in the \vars{runlists} dictionary
(with key \vars{'qtcminit'}).
\vars{runlists} is an attribute of a \class{Qtcm} instance.
Table~\ref{tab:stnd.runlists} lists all standard run lists.

When the run list element in the list is a string, the string gives the
name of the routine to execute.  The routine has no parameter
list.  The routine can be a
compiled QTCM1 model subroutine for which an interface has been
written (e.g., \mods{\_\_qtcm.wrapcall.wparinit}), 
a method of the of the Python model instance 
(e.g., \mods{varinit}), or another run list
(e.g., \vars{atm\_physics1}).

When the run list element is a 1-element dictionary, the key of
the dictionary element is the name of the routine, and the value
of the dictionary element is a list specifying input parameters
to be passed to the routine on call.  Thus, the element:
\begin{codeblock}
\codeblockfont{%
{\{'\_\_qtcm.wrapcall.wtimemanager': [1]\}}}
\end{codeblock}
calls the \mods{\_\_qtcm.wrapcall.wtimemanager} routine, passing in
one input parameter, which in this case is the value 1.

If you want to change the order of the run list, just change the
order of the list.  To add or remove routines to be executed, just
add and remove their names from the run list.
Python provides a number of methods to manipulate
lists (e.g., \mods{append}).  Since lists are dynamic data types
in Python, you do not have to do any recompiling to implement
the change.

The \vars{compiled\_form} attribute must be set to \vars{'parts'}
in the \class{Qtcm} instance in order to take advantage of the run
lists feature of the class.  Run lists are not available for
\vars{compiled\_form\thinspace=\thinspace'full'}, because subroutine
calls are hardwired in the compiled QTCM1 model Fortran code in
that case.




%---------------------------------------------------------------------
\section{Model Output}			\label{sec:output.intro}

	\subsection{NetCDF Output}

Model output is written to netCDF files in the directory
specified by the \class{Qtcm} instance attribute \vars{outdir}.
Mean values are written to an output file beginning with
\fn{qm\_}, and instantaneous values are written to an output
file beginning with \fn{qi\_}.

The frequency of mean output is controlled by \vars{ntout}, and the
frequency of instantaneous output is controlled by \vars{ntouti}.
\vars{ntout.value} gives the number of days over which to average
(and if equals \vars{-30}, monthly means are calculated).
\vars{ntouti.value} gives the frequency in days that instantaneous
values are output (monthly if it equals \vars{-30}).  (See
Section~\ref{sec:initial.variables} for a description of other
output-control variables, and see the QTCM1 manual \cite{Neelin/etal:2002}
for a detailed description of how these variables control output.)

Figure~\ref{fig:netcdf.read} gives an example of a block of code
to read netCDF output, where \vars{datafn} is the netCDF filename, and
\vars{id} is the string name of the field variable (e.g.,
\vars{'u1'}, \vars{'T1'}, etc.).
(Note that the netCDF identifier for field variables is the same as
the name in \class{Qtcm}, except for the variables given in
Table~\ref{tab:qtcm.netcdf.ids}.)

In the code in Figure~\ref{fig:netcdf.read},
the array value is read into \vars{data}, and the longitude values, 
latitude values, and time values are read into variables
\vars{lon}, \vars{lat}, and \vars{time}, respectively.
As netCDF files also hold metadata, a description and the units
of the variable given by \vars{id}, and each dimension, are read
into variables ending in \vars{\_name} and \vars{\_units},
respectively.


%--- Two versions, one for PDF, one for HTML:
\begin{latexonly}
\begin{figure}[htp]
\begin{codeblock}
\codeblockfont{%
import numpy as N \\
import Scientific as S \\ \\
fileobj = S.NetCDFFile(datafn, mode='r') \\ \\
data = N.array(fileobj.variables[id].getValue()) \\
data\_name = fileobj.variables[id].long\_name \\
data\_units = fileobj.variables[id].units \\ \\
lat = N.array(fileobj.variables['lat'].getValue()) \\
lat\_name = fileobj.variables['lat'].long\_name \\
lat\_units = fileobj.variables['lat'].units \\ \\
lon = N.array(fileobj.variables['lon'].getValue()) \\
lon\_name = fileobj.variables['lon'].long\_name \\
lon\_units = fileobj.variables['lon'].units \\ \\
time = N.array(fileobj.variables['time'].getValue()) \\
time\_name = fileobj.variables['time'].long\_name \\
time\_units = fileobj.variables['time'].units \\ \\
fileobj.close()}
\end{codeblock}

\caption{Example of Python code to read netCDF output.
	See text for description.}
\label{fig:netcdf.read}
\end{figure}
\end{latexonly}

\begin{htmlonly}
\label{fig:netcdf.read}
\begin{center}
\htmlfigcaption{%
	\codeblockfont{%
import numpy as N \\
import Scientific as S \\ \\
fileobj = S.NetCDFFile(datafn, mode='r') \\ \\
data = N.array(fileobj.variables[id].getValue()) \\
data\_name = fileobj.variables[id].long\_name \\
data\_units = fileobj.variables[id].units \\ \\
lat = N.array(fileobj.variables['lat'].getValue()) \\
lat\_name = fileobj.variables['lat'].long\_name \\
lat\_units = fileobj.variables['lat'].units \\ \\
lon = N.array(fileobj.variables['lon'].getValue()) \\
lon\_name = fileobj.variables['lon'].long\_name \\
lon\_units = fileobj.variables['lon'].units \\ \\
time = N.array(fileobj.variables['time'].getValue()) \\
time\_name = fileobj.variables['time'].long\_name \\
time\_units = fileobj.variables['time'].units \\ \\
fileobj.close()}
	}

\htmlfigcaption{Figure~\ref{fig:netcdf.read}:
	Example of Python code to read netCDF output.
	See text for description.}
\end{center}
\end{htmlonly}





\begin{table}[tp]
\begin{center}
\begin{tabular}{l|l}
\textbf{\class{Qtcm} Attribute Name} & \textbf{NetCDF Output Name} \\
\hline
\vars{'Qc'}                & \vars{'Prec'} \\
\vars{'FLWut'}             & \vars{'OLR'} \\
\vars{'STYPE'}             & \vars{'stype'}
\end{tabular}
\end{center}
\caption{NetCDF output names for \class{Qtcm} field variables that
	are different from the \class{Qtcm} and compiled QTCM1 model
	variable names.  The netCDF names are case-sensitive.}
\label{tab:qtcm.netcdf.ids}
\end{table}


\emphpara{NB:}  All netCDF array output is dimensioned (time, latitude,
longitude) when read into Python using the \mods{Scientific} package.
This differs from the way \class{Qtcm} saves field variables, which
follows Fortran convention (longitude, latitude).  Please be careful
when relating the two types of arrays.
Section~\ref{sec:field.var.shape} for a discussion of why there is
this discrepancy.


	\subsection{Visualization}	\label{sec:viz.intro}

The \mods{plotm} method of \class{Qtcm} instances creates line
plots or contour plots, as appropriate, of model output of
average fields of run session(s) associated with the instance.
Some examples, assuming \vars{model} is an instance of \class{Qtcm}
and has already executed a run session:
\begin{itemize}
\item \cmd{model.plotm('Qc', lat=1.875)}:
	A time vs.\ longitude contour
          plot is made for the full range of time and longitude,
          at the latitude 1.875 deg N, for mean precipitation.

\item \cmd{model.plotm('Qc', time=10)}:
	A latitude vs.\ longitude contour plot of precipitation
	is made for the full spatial domain at day 10 of the model run.

\item \cmd{model.plotm('Evap', lat=1.875, lon=[100,200])}:  A contour
	plot of time vs.\ longitude of evaporation is made for the
          longitude points between 100 and 200 degrees E, at the
          latitude 1.875 deg N.  

\item \cmd{model.plotm('cl1', lat=1.875, lon=[100,200], time=20)}:
          A deep cloud amount vs.\ longitude line plot is made for
          the longitude points between 100 and 200 degrees east,
          at the latitude 1.875 deg N, at day 20 of the model run.
\end{itemize}

In these examples, the number of days over which the mean is taken
equals \vars{model.ntout.value}.
Also, the \mods{plotm} method automatically takes into account the
\class{Qtcm}/netCDF variable differences described in
Table~\ref{tab:qtcm.netcdf.ids}.



%---------------------------------------------------------------------
\section{Documentation}

Section~\ref{sec:ver} gives the online locations of the
transparent copies of this manual.  
Model formulation is fully described in
Neelin \& Zeng \cite{Neelin/Zeng:2000} and model
results are described in Zeng et~al.\ \cite{Zeng/etal:2000}
(\cite{Neelin/Zeng:2000} is based upon v2.0 of QTCM1
and \cite{Zeng/etal:2000} is based on QTCM1 v2.1).
Additional documentation you'll find useful include:

\begin{itemize}
\item \latexhtml{%
\htmladdnormallinkfoot{The \mods{qtcm} Package API Documentation}%
        {http://www.johnny-lin.com/py\_pkgs/qtcm/doc/html-api/}}%
{\htmladdnormallink{The \mods{qtcm} Package API Documentation}%
        {http://www.johnny-lin.com/py_pkgs/qtcm/doc/html-api/}}

\item \latexhtml{%
\htmladdnormallinkfoot{The Pure-Fortran QTCM1 Manual}%
        {http://www.atmos.ucla.edu/$\sim$csi/qtcm\_man/v2.3/qtcm\_manv2.3.pdf}}%
{\htmladdnormallink{The Pure-Fortran QTCM1 Manual}%
        {http://www.atmos.ucla.edu/~csi/qtcm_man/v2.3/qtcm_manv2.3.pdf}}
\cite{Neelin/etal:2002}

\end{itemize}



% ===== end of file =====


\chapter{Using \mods{qtcm}}                 \label{ch:using}
% ==========================================================================
% Using QTCM
%
% By Johnny Lin
% ==========================================================================


% ------ BODY -----
%
%---------------------------------------------------------------------
\section{Introduction}

Now that you've successfully run your first model instances, in
this chapter I provide detailed explanations regarding the features
of \mods{qtcm}.  I present these explanations in a documentary
rather than didactic fashion; my goal is to document how the features
work.  More details are given in the code docstrings.  At the end
of the chapter, in Section~\ref{sec:cookbook}, I provide a few
cookbook ideas/examples of ways to use the model.




%---------------------------------------------------------------------
\section{Model Instances}  \label{sec:model.instances}

An instance of a \class{Qtcm} model is created in \mods{qtcm} the same way
you create an instance of any class.
For instance, to instantiate two \class{Qtcm}
models, \vars{model1} and \vars{model2}, I type the following:

\begin{codeblock}
\codeblockfont{%
from qtcm import Qtcm \\
model1 = Qtcm(compiled\_form\thinspace=\thinspace'full') \\
model2 = Qtcm(compiled\_form\thinspace=\thinspace'parts')}
\end{codeblock}

In the above example, \vars{model1} uses the compiled QTCM1 model
that runs the model (essentially) using the Fortran driver,
while \vars{model2} uses the compiled QTCM1 model where execution
order and content all the way down to the atmospheric timestep level
is controlled by Python run lists.  (Section~\ref{sec:compiledform}
has more details about the difference between compiled forms.)

For each instance of \class{Qtcm}, copies of all needed extension
modules (e.g., \fn{.so} files) are copied to a temporary directory
that is automatically created by Python.  The full path name of
that directory is saved in the instance attribute \vars{sodir}.
These extension modules are then associated with the specific instance 
through private instance attributes,
and thus every instance of \class{Qtcm} has its own separate variable
and name space on both the Fortran and Python sides.\footnote%
	{The private instance attribute is \vars{\_\_qtcm}.
	See Section~\ref{sec:Qtcm.private.attrib} for details about 
	private \class{Qtcm} instance attributes.}
The temporary directory and all of its contents are deleted when the 
model instance is deleted.

On instantiation, \class{Qtcm} instances set all scalar field
variables to their default values as given in the submodule
\mods{defaults} (and listed in Section~\ref{sec:defaults.scalar}),
and assign the fields as instance attributes.  The instance attribute
\vars{init\_with\_instance\_state} is set to True by default, unless
overridden on instantiation.




%---------------------------------------------------------------------
\section{Initializing a Model Run}

In the pure-Fortran QTCM1, there are three broad
classes of initialized variables:
\begin{enumerate}
\item Those that are read-in using a namelist, 
\item Those that the are read-in from a restart file, and
\item Those that are set by assignment in the Fortran code.  
\end{enumerate}
These variables are a combination of scalars and arrays.

For \mods{qtcm}, interfaces were built so that all classes of
initialized variables that could be user-controlled are accessible
and changeable at the Python-level.  For \mods{qtcm},
the set of variables that could be changed is also expanded, to
include not just the first and second classes of pure-Fortran
QTCM1 initialized variables.  This was done to make \mods{qtcm}
more flexible.  All variables that can be passed between the
compiled QTCM1 model and Python model levels are called
field variables, and are described in detail in
Section~\ref{sec:field.variables}.

As it happens, all the namelist-set variables are scalars.  In the
pure-Fortran QTCM1, those variables are given default values prior
to reading in of the namelist.  To duplicate this functionality,
on model instantiation, all scalar fields are set to their default
values as given in the submodule \mods{defaults} and listed in
Section~\ref{sec:defaults.scalar}.  Most of the default values in
\mods{defaults} are the same as in the pure-Fortran QTCM1, but
there are a few differences.\footnote%
	{One difference being \vars{mrestart}, which in \vars{qtcm} 
	will have the value of 0 in contrast to the pure-Fortran 
	QTCM1 where the default is the 1.}
This setting of scalar defaults is the same for both
\vars{compiled\_form\thinspace=\thinspace'full'} and
\vars{compiled\_form\thinspace=\thinspace'parts'} instances.
Of course, all
\mods{qtcm} fields are user-controllable, both via keyword input
parameters at model instantiation as well as through direct
manipulation of the instance attribute that stores the field variable.

The pure-Fortran QTCM1 initialized prognostic variables and
right-hand sides are set in the Fortran subroutine \mods{varinit}.
Their they are read-in from a restart file or, as default,
set by assignment.
In \mods{qtcm}, the same variables are initialized by a \class{Qtcm}
instance method of the same name, \mods{varinit}, for the case when
\vars{compiled\_form\thinspace=\thinspace'parts'}.  For the case
of \vars{compiled\_form\thinspace=\thinspace'full'}, the compiled
QTCM1 subroutine that is the same as in the pure-Fortran case is
used, and that routine is inaccessible at the Python level.
See Section~\ref{sec:snapshots}'s listing of snapshot variables,
which also includes the prognostic variables and right-hand sides that
are set in \mods{varinit} (both Fortran and Python).




%---------------------------------------------------------------------
\section{The \vars{compiled\_form} Keyword}  \label{sec:compiledform}

The \mods{qtcm} package is a Python wrap of the Fortran routines
that make up QTCM1.  The wrapping layer adds flexibility and
functionality, but at the cost of speed.  Thus, I created two
types of extension modules from the Fortran QTCM1 code, one
which permits very little control over the compiled Fortran
\emph{routines} at the Python level, and one that allows the Python-level
to control model execution in the compiled QTCM1 model
all the way down to the atmospheric timestep level.\footnote%
	{That control is via run lists, which are described in
	Section~\ref{sec:runlists}.}
The former extension module corresponds to 
\vars{compiled\_form\thinspace=\thinspace'full'} and
the latter extension module to
\vars{compiled\_form\thinspace=\thinspace'parts'}.

For \vars{compiled\_form\thinspace=\thinspace'full'},
the compiled portion of the model encompasses (nearly) the
entire QTCM1 model as a whole.  Thus, the only compiled QTCM1 model
modules or subroutines that Python should interact with is
the \mods{driver} routine (which executes the entire model) and
the \mods{setbypy} module (which enables communication between the
compiled model and the Python-level of model fields.\footnote%
	{The \mods{setbypy} Python module is the wrap of the
	Fortran QTCM1 \mods{SetByPy} module.}

For \vars{compiled\_form\thinspace=\thinspace'parts'}, the compiled
portion of the model does not encompasses the model as a whole, but
rather is broken up into separate units (as appropriate) all the
way down to an atmosphere timestep.  Thus, compiled QTCM1 model
modules/subroutines that are accessible at the Python-level include
those that are executed within an atmosphere timestep on up.

Because the difference in compiled forms fundamentally affects how
the \class{Qtcm} instance facilitates Python-Fortran communication,
this attribute must be set on instantiation via a keyword input
parameter.

In the rest of this section, to avoid being verbose, when I
write \vars{'full'}, I mean the situation where
\vars{compiled\_form\thinspace=\thinspace'full'}.
Likewise, when I
write \vars{'parts'}, I mean the situation where
\vars{compiled\_form\thinspace=\thinspace'parts'}.


	\subsection{Initialization for 
			\vars{compiled\_form\thinspace=\thinspace'full'}}
				\label{sec:init.compiledform.full}

For a model run of this case, the \class{Qtcm} instance will
initialize the model using the Fortran \mods{varinit} subroutine
in the compiled QTCM1 model.  This subroutine does the following:

\begin{itemize}
\item If \vars{mrestart\thinspace=\thinspace1}, 
	the restart file is used to initialize all prognostic
	variables.  In terms of start date, the following rules are
	used:
	\begin{enumerate}
	\item Variable \vars{dateofmodel} is read from the restart file.
	\item If \vars{day0}, \vars{month0}, and \vars{year0}
		are negative, or otherwise
		invalid (e.g., \vars{month0} greater than 12), the invalid
		value is replaced with the
		day, month, and/or year of the day \emph{after} 
		that given by \vars{dateofmodel}.
		If the value of \vars{day0}, \vars{month0}, or \vars{year0}
		is not invalid in this sense, it is not replaced.
	\end{enumerate}
	Thus, if the restart file gives 
	\vars{dateofmodel} equal to 101102
	(year 10, month 11, day 2), and 
	\vars{day0\thinspace=\thinspace-1}, 
	\vars{month0\thinspace=\thinspace-1}, 
	\vars{year0\thinspace=\thinspace-1},
	and 
	\vars{mrestart\thinspace=\thinspace1}, 
	the model will start running from year 10, month 11, day 3.
	If \vars{dateofmodel} equals to 101102, and 
	\vars{day0\thinspace=\thinspace-1}, 
	\vars{month0\thinspace=\thinspace3}, 
	\vars{year0\thinspace=\thinspace-1},
	the model will start running from year 10, month 3, day 3.

\item If \vars{mrestart\thinspace=\thinspace0}, 
	all prognostic variables and right-hand sides are set to an
	initial value (which for most of those variables is zero).
	In terms of start date, \vars{day0} is set to 1 (and thus 
	the value of \vars{day0} previously input is ignored), and
	both \vars{month0} and \vars{year0}
	are set to 1 
	if their previously input values are invalid (where
	invalid means less than
	1, or, for \vars{month0}, greater than 12).
	Otherwise, \vars{month0} and \vars{year0} are left unchanged.
	Variable \vars{dateofmodel} has the value it had when the variable
	was declared (which is determined by the compiler and usually
	is zero; \vars{dateofmodel} will not be properly set until
	subroutine \mods{TimeManager} is called.

	Thus, if 
	\vars{day0\thinspace=\thinspace-1},
	\vars{month0\thinspace=\thinspace-1}, 
	\vars{year0\thinspace=\thinspace-1} is input into the model
	(say from a namelist) and 
	\vars{mrestart\thinspace=\thinspace0},
	the model will start running from year 1, month 1, day 1,
	and \vars{dateofmodel} at the exit of subroutine 
	\mods{varinit} will equal its compiler-set default.
	If 
	\vars{day0\thinspace=\thinspace14}, 
	\vars{month0\thinspace=\thinspace3}, 
	\vars{year0\thinspace=\thinspace11}, and 
	\vars{mrestart\thinspace=\thinspace0} on input into the
	model,
	the model will start running from year 11, month 3, day 1,
	and \vars{dateofmodel} at the exit of subroutine 
	\mods{varinit} will equal its compiler-set default.

	Note that \vars{dateofmodel}
	can thus be inconsistent with 
	\vars{month0} and \vars{year0} at the
	exit of subroutine \mods{varinit}.
\end{itemize}

This behavior with respect to initializing
the start date is different than in QTCM1 versions 1.0 and 2.1.
Please see the source code from those earlier QTCM1 versions for
details.




	\subsection{Initialization for 
			\vars{compiled\_form\thinspace=\thinspace'parts'}}
				\label{sec:init.compiledform.parts}

For \vars{'parts'} model, the methodology of how initialized
prognostic variables, right-hand sides, and start date related
variables are set is controlled by the \class{Qtcm} instance
attribute/flag \vars{init\_with\_instance\_state}.  The initialization
is (mostly) executed in the Python \vars{varinit} method in the
following way:

\begin{itemize}
\item If \vars{init\_with\_instance\_state} is False:
The method as described for
initialization for the 
\vars{'full'} case is generally
followed, with the exception that dateofmodel is set
to match \vars{day0}, \vars{month0}, \vars{year0}, prior to exit of 
\mods{varinit}.

\item If \vars{init\_with\_instance\_state} is True:
the model object will initialize the model based on the current
state of the model instance.  This enables you to set a model run
session's initial conditions based upon the state of the prognostic
variables and parameters stored at the Python level, which is
accessible at runtime.
\end{itemize}


Since the \vars{init\_with\_instance\_state\thinspace=\thinspace{False}}
case is mainly described by the initialization method for the
\vars{'full'} case, I refer the
reader to Section~\ref{sec:init.compiledform.full}.
For the case of \vars{init\_with\_instance\_state} is True, however,
the task is more complicated.  Specifically, for that case,
initialization includes the following:

\begin{enumerate}
\item If not currently defined,
	variable \vars{dateofmodel} is set to a default value of 0,
	which is specified in the module defaults.

\item The \vars{mrestart} flag is ignored for variable initialization.

\item All prognostic variables and right-hand sides
        are set to an
        initial value (which for most of those variables is zero),
	unless the variable is defined at the Python level, in which
	case the inital value is set to the Python level defined value.

\item If \vars{dateofmodel} is greater than 0, 
	\vars{day0}, \vars{month0}, and \vars{year0} are overwritten
        with values derived from \vars{dateofmodel} 
	in order to set the run to start
	the day \emph{after} \vars{dateofmodel}.

\item If \vars{dateofmodel} is less than or equal to 0, \vars{day0},
	\vars{month0}, and \vars{year0} are set to their respective
	instance attribute values, if valid.  For invalid instance
	attribute values, the invalid \vars{day0}, \vars{month0},
	and/or \vars{year0} is set to 1.

\item Variable \vars{dateofmodel} is recalculated
	and overwritten to match 
	\vars{day0}, \vars{month0}, \vars{year0}, prior to exit of 
	\mods{varinit}.
\end{enumerate}

As a result, for \vars{init\_with\_instance\_state} is True, the
way you indicate to the model that a run session is a brand-new run
is by setting, before the \mods{run\_session} method call,
\vars{dateofmodel} to a value less than or equal to 0, and \vars{day0},
\vars{model0}, and \vars{year0} to the day you want the model to
begin the run session.  To indicate to the model you wish to continue
a run, set \vars{dateofmodel} to the day \emph{before} you want the
model to start running from.

Examples:
\begin{itemize}
\item If \vars{day0\thinspace=\thinspace-1}, 
	\vars{month0\thinspace=\thinspace-1}, 
	\vars{year0\thinspace=\thinspace-1}, and
	\vars{dateofmodel\thinspace=\thinspace0} is input into 
	the model the model will start running from year 1, month 1, day 1,
	and 
	variable \vars{dateofmodel} at the exit of 
	subroutine \mods{varinit}
	will equal 10101.

\item If \vars{day0\thinspace=\thinspace14},
	\vars{month0\thinspace=\thinspace3}, 
	\vars{year0\thinspace=\thinspace11},
	and \vars{dateofmodel\thinspace=\thinspace0} is input into the
	model, the model will start running from year 11, month 3, day 14,
	and 
	variable \vars{dateofmodel} at the exit of 
	subroutine \mods{varinit} will equal
	110314.

\item If \vars{day0\thinspace=\thinspace14},
	\vars{month0\thinspace=\thinspace3}, 
	\vars{year0\thinspace=\thinspace11},
	and \vars{dateofmodel\thinspace=\thinspace341023} is input into the
	model, the model will start running from year 34, month 10, day 24,
	and at the exit of subroutine 
	\mods{varinit}, \vars{dateofmodel} will equal
	341024, with \vars{day0\thinspace=\thinspace24},
	\vars{month0\thinspace=\thinspace10}, and
	\vars{year0\thinspace=\thinspace34}.
\end{itemize}


	\subsection{Communication Between Python and Fortran-Levels}
				\label{sec:comm.py.fort.compiledform}

After initialization, the second major difference between a
\vars{'full'} and \vars{'parts'} model is how and when communication
between the Python and Fortran levels can occur.  For the \vars{'full'}
case, except for the passing in and out of variables before and after
a run session, all variable passing and subroutine calling happens in
the compiled QTCM1 model, with no control at the Python level.
For the \vars{'parts'} case, variables can be passed between the
Python and Fortran-levels at all levels down to the atmospheric
timestep, and many Fortran QTCM1 subroutines can be called from the
Python-level.  


		\subsubsection{Passing Variables}

For all \vars{compiled\_form} cases, variables are passed back and
forth between the Python \class{Qtcm} instance level and the
compiled QTCM1 model Fortran-level using the \class{Qtcm}
instance methods \mods{get\_qtcm1\_item} and \mods{set\_qtcm1\_item}:\footnote%
	{All Fortran routines used to pass variables back and forth are
	defined in the \mods{setbypy} module of the \fn{.so} extension
	module stored in the \class{Qtcm} instance variable \vars{\_\_qtcm}.
	All Fortran wrappers that enable Python to call compiled QTCM1 model
	subroutines are defined in the \mods{wrapcall} module stored in
	the \class{Qtcm} instance variable \vars{\_\_qtcm}.
	These modules are described in detail in 
	Sections~\ref{sec:setbypy} and~\ref{sec:wrapcall}, respectively.}

\begin{itemize}
\item \mods{get\_qtcm1\_item}(\dumarg{key}):
	Returns the value of the field variable given by the string
	\dumarg{key}.  If the compiled QTCM1 model variable given by
	\dumarg{key} is unreadable, the
        custom exception 
	\vars{FieldNotReadableFromCompiledModel} is thrown.
	The value returned is a copy of the value on the Fortran
	side, not a reference to the variable in memory.

\item \mods{set\_qtcm1\_item}:
	Sets the value of a field variable
	in the compiled QTCM1 model \emph{and at the Python-level,}
	automatically overriding any previous value at both levels.
	Thus, calling this method will change/create the \class{Qtcm}
	instance attribute corresponding to the field variable.
        When the compiled QTCM1 model variable is set, a copy of the
        Python value is passed to the Fortran model; the
	variable is \emph{not passed by reference.}
	This value comes from the \mods{set\_qtcm1\_item} calling
	parameter list, \emph{not} from the \class{Qtcm}
        instance attribute corresponding to the field variable.
\end{itemize}

The \mods{set\_qtcm1\_item} method has two calling forms, one with
one argument and the other with two arguments:
\begin{itemize}
\item One argument:  The method is called
	as \mods{set\_qtcm1\_item}(\dumarg{arg}), where \dumarg{arg} 
	is either a string giving the name of the field variable or 
	a \class{Field} instance.

\item Two arguments:  The method is called as
	\mods{set\_qtcm1\_item}(\dumarg{key}, \dumarg{value}), where
	\dumarg{key} is the string giving the name of the field variable
	and \dumarg{value} is the value to set the model field variable to
	(note \dumarg{value} can be a \class{Field} instance).
\end{itemize}
In either calling form, if no value given, the default value as defined
in module \mods{defaults} is used.

Some compiled QTCM1 model variables are not in a state where they
can be set.  An example is a compiled QTCM1 model pointer variable,
prior to the pointer being associated with a target (an attempt
to set would yield a bus error).  In such cases, the
\mods{set\_qtcm1\_item} method will throw a
\vars{FieldNotReadableFromCompiledModel} exception, nothing will
be set in the compiled QTCM1 model, and the Python counterpart
field variable (if it previously existed) would be left unchanged.\footnote%
	{We handle this situation in this way to enable the
	\class{Qtcm} instance to store variables
	even if the compiled model is not yet ready to accept them.}

Examples, typed in at a Python prompt, and
assuming that \vars{model} is a \class{Qtcm} instance:
\begin{itemize}
\item \cmd{dtvalue\thinspace=\thinspace{model.get\_qtcm1\_item('dt')}}:
	Retrieves the value of field variable \vars{dt} (timestep)
	from the compiled QTCM1 Fortran model and sets it to the
	Python variable \vars{dtvalue}.

\item \cmd{model.set\_qtcm1\_item('dt')}:
	Sets the value of field variable \vars{dt}
	in the compiled QTCM1 Fortran model to the default
	value (as given in \mods{defaults}),
	and sets the value of Python attribute \vars{model.dt} also to 
	that default value.  
	Remember that \vars{model.dt} is a \class{Field}
	instance.

\item \cmd{model.set\_qtcm1\_item('dt', 2000.)}:
	Sets the value of field variable \vars{dt}
	in the compiled QTCM1 Fortran model to 2000 (as a real),
	and sets the value of Python attribute \vars{model.dt} also to 2000.
\end{itemize}


		\subsubsection{Calling Compiled QTCM1 Model Subroutines}

All compiled QTCM1 model subroutines that can be called
(except \mods{driver} and \mods{varptrinit}) are in the
\mods{setbypy} or \mods{wrapcall} modules
of the \class{Qtcm} instance private attribute \vars{\_\_qtcm}.
(On \class{Qtcm} instance instantiation, \vars{\_\_qtcm} is set
to the \fn{.so} extension module that is the compiled QTCM1 Fortran model.)
Thus, to call \mods{wmconvct} in \mods{wrapcall} at the Python-level,
just type \cmd{model.\_\_qtcm.wrapcall.wmconvct()} (where \vars{model}
is a \class{Qtcm} instance).
For \mods{driver} and \mods{varptrinit}, these subroutines are not
contained in a \vars{\_\_qtcm} module, and thus can be called
directly (e.g., just type \cmd{model.\_\_qtcm.driver()}).
See Sections~\ref{sec:setbypy} and~\ref{sec:wrapcall} for more information
on the \mods{setbypy} and \mods{wrapcall} modules.

For the \vars{'full'} case, the only compiled QTCM1 model
subroutine you can usefully call during a run session is \mods{driver}.
For the \vars{'parts'} case, while you can essentially call any subroutine
given in a run list, you usually will not directly call a compiled QTCM1
model subroutine but will instead call it through including it in a
run list.  For example, if you have the following run list in a
\vars{'parts'} model:
\begin{codeblock}
\codeblockfont{%
[ 'qtcminit', '\_\_qtcm.wrapcall.woutpinit' ]}
\end{codeblock}
Running this list using the \class{Qtcm} instance method
\mods{run\_list} will result in \class{Qtcm} instance method
\mods{qtcminit} first being run, 
then the compiled QTCM1 Fortran model subroutine
\mods{woutpinit} in Fortran module \mods{wrapcall} being run.
See Section~\ref{sec:runlists} and
Table~\ref{tab:stnd.runlists} for a discussion and list of the
standard run lists that control routine execution content and order
in the \vars{'parts'} case.




%---------------------------------------------------------------------
\section{Restart and Continuation Run Sessions}
				\label{sec:contination.run.sessions}


	\subsection{Restart Runs In the Pure-Fortran QTCM1}
					\label{sec:puref90.restart}

To enable restart of a model run, the pure-Fortran QTCM1 model
writes out a restart file with the state of the prognostic variables
and select right-hand sides at that point in the run (for a list
of the variables, see Section~\ref{sec:snapshots}).  This binary
file can then be read in by later model runs.  The Fortran
\vars{mrestart} flag is passed in via a namelist; if \vars{mrestart}
is 1, the run uses the restart file (named \fn{qtcm.restart}).

One of the problems with using the restart file to do a continuation
run is that the continuation run will not be perfect.  In other words,
a 15~day run followed by a 25~day run based on the restart file 
generated at the end of the 15~day run will \emph{not} give the
exact same output as a continuous 40~day run.


	\subsection{Overview of Restart/Continuation Options In \mods{qtcm}}
					\label{sec:restart.options.list}

For a \class{Qtcm} instance, in contrast to the pure-Fortran QTCM1,
more than one method of continuation is available.
Thus, for a continuation run, you need to tell the model
``continue from what?''
The \class{Qtcm} class provides three choices for restart/continuing
a run:
\begin{enumerate} 
\item From a restart file:  Move/rename a QTCM1 restart file
        to the current working directory to \fn{qtcm.restart}.
	\label{list:continue.from.restart}

\item From a snapshot from another run session
	(see Sections~\ref{sec:snapshot.intro} and~\ref{sec:snapshots}).
	\label{list:continue.from.snapshot}

\item From the values of the \class{Qtcm} instance you will be
	calling \mods{run\_session} from.
	\label{list:continue.from.instance}
\end{enumerate}

Restart/continuation methods~\ref{list:continue.from.restart} 
and~\ref{list:continue.from.snapshot} both suffer from the
same problem as the pure-Fortran QTCM1 restart process:
They do not produce perfect restarts
(see Section~{sec:puref90.restart} for details).
In this section, I discuss the restart/continuation options
for each \vars{compiled\_form} option.

Methods~\ref{list:continue.from.restart}
and~\ref{list:continue.from.snapshot} are best used when making a
run session from a newly instantiated \class{Qtcm} instance.
Method~\ref{list:continue.from.instance} is best used when executing
a run session using a \class{Qtcm} instance that has already gone
through at least one run session.  Regardless of which method you
use, however, please note that anytime you execute a run session
using a \class{Qtcm} instance that already has made a previous run
session, some variables \emph{cannot be updated} between run sessions.
This feature is most noticeable with the output filename, and occurs
because the name persists in the compiled QTCM model, and is stored
in the extension module (\fn{.so} files in \vars{sodir}) associated
with the instance.  If you wish to control all variables possible
from the Python level (including output filename), you need do the
run session from a new model instance.


	\subsection{Restart/Continuation for 
		\vars{compiled\_form\thinspace=\thinspace'full'} 
		Model Instances}

The only option for restart when using
\vars{compiled\_form\thinspace=\thinspace'full'} model instances
is method~\ref{list:continue.from.restart}, to use a QTCM1 restart
file.\footnote%
	{The \vars{cont} keyword parameter in \mods{run\_session}
	and the value of the \vars{init\_with\_instance\_state}
	attribute have no effect if
	\vars{compiled\_form\thinspace=\thinspace'full'}.  With
	\vars{'full'}, the call to initialize variables all happens
	at the Fortran level (via the Fortran \mods{varinit}, not
	the Python \mods{varinit}), with no reference to the Python field
	states (or even existing Fortran field states, if present).}
To use this option, the value of the \vars{mrestart} 
attribute must equal 1, the restart file must be named
\fn{qtcm.restart}, and the restart file must be in the 
current working directory.
As with the pure-Fortran QTCM1 restart process, this method
does not produce perfect restarts.



	\subsection{Restart/Continuation for 
		\vars{compiled\_form\thinspace=\thinspace'parts'} 
		Model Instances}

For the \vars{compiled\_form\thinspace=\thinspace'parts'} case,
all three restart/continuation methods
described in Section~\ref{sec:restart.options.list} are
available.


		\subsubsection{Method~\ref{list:continue.from.restart}:
			From a QTCM1 Restart File}

To use the QTCM1 restart file mechanism, not only must the
\vars{mrestart} attribute have a value to 1, but the
\vars{init\_with\_instance\_state} flag also has to be \vars{False},
otherwise the \vars{mrestart} attribute value will be ignored.  
As with the pure-Fortran QTCM1 restart process, this method does not
produce perfect restarts.


		\subsubsection{Method~\ref{list:continue.from.snapshot}:
			From a \class{Qtcm} Instance Snapshot}

You can take snapshots of the model state of a \class{Qtcm} instance
by the \mods{make\_snapshot} instance method.  This snapshot saves
a copy of all the variables saved to a QTCM1 restart file (see
Section~\ref{sec:snapshots} for the full list of fields), which
then can be passed to other \class{Qtcm} instances for use in other
run sessions.

The key difference between this method and 
method~\ref{list:continue.from.instance} (described below)
is that \mods{run\_session} calls using the snapshot are done
\emph{without} the \vars{cont} keyword input parameter
(by default, \vars{cont} is False).  If the \vars{cont} keyword
is not False, it says the run session is a continuation run
that uses the state of the compiled QTCM1 model for all variables
that are not specified at, and read-in from,
the Python level.  If the \vars{cont} keyword
is False, the run session initializes as if it were a new run.

See Section~\ref{sec:snapshot.intro} for details and
an example of using snapshots to initialize a run session.
Note that as with the pure-Fortran QTCM1 restart process, this method 
does not produce perfect restarts.


		\subsubsection{Method~\ref{list:continue.from.instance}:
			From the Calling \class{Qtcm} Instance}

This method is used when you want to make a run session that is a
``true'' continuation run, i.e., one that uses the current state
of the compiled QTCM1 model for all variables that are not read-in
from the Python level (remember that \class{Qtcm} instances hold a
subset of the variables defined at the Fortran level).  
The key reason to use this method for a continuation run session
is that the continuation is byte-for-byte the same (if no fields
are changed) as if the run just went straight on through.  Thus,
the continuation would be perfect: A 15~day run followed by a 25~day
run using the same \class{Qtcm} instance with the \vars{cont} keyword
will give the exact same output as a continuous 40~day run.  This
is not the case when making a new instance and passing a restart
file or a snapshot, because a separate extension module is used for
those new instances.

Control of this method is accomplished through the \vars{cont}
keyword input parameter to the \mods{run\_session} method and the
\vars{init\_with\_instance\_state} attribute of a
\class{Qtcm} instance:

\begin{itemize}
\item \vars{cont}: If set to False, the run session is not a
	continuation of the previous run, but a new run session.
	If set to True, the run session is a continuation of the
	previous run session.  If set to an integer greater than
	zero, the run session is a continuation just like
	\vars{cont\thinspace=\thinspace{True}}, but the value
	\vars{cont} is set to is used for \vars{lastday} and replaces
	\vars{lastday.value} in the \class{Qtcm} instance.

\item \vars{init\_with\_instance\_state}:
	If True, for a \mods{run\_session} call using the
	\vars{cont} keyword, whatever the field values are in the Python
	instance are used in the run session.
	If False, model variables are set and initialized as described in
	Section~\ref{sec:init.compiledform.parts}.  In that case,
	previous compiled QTCM1 model values will likely be overwritten.
	Thus, if you want a continuation run that uses the state of
	all field variables except for those you explicitly change at
	the Python-level, make sure \vars{init\_with\_instance\_state}
	is True.
\end{itemize}

(Note that the \vars{cont} keyword has no effect if \vars{compiled\_form}
is \vars{'full'}.  The default value of \vars{cont} in a
\mods{run\_session} call is False.  The value of keyword \vars{cont}
is stored as private instance attribute \vars{\_cont}, in case you
really need to access it elsewhere; see
Section~\ref{sec:Qtcm.private.attrib} for more details).

The example described in Section~\ref{sec:continuation.intro} is
an example of method~\ref{list:continue.from.instance} in the list
above: The second run session is continued from the state of
\vars{model}, with the values of \vars{model}'s instance variables
overriding any values in the compiled QTCM1 model in initializing
the second run session.

This method has a few caveats worthy of note:
\begin{itemize}
\item The \vars{init\_with\_instance\_state} attribute value
	will have no effect unless the instance prognostic variables
	are set, i.e., unless a previous run session has been done.
	Another way to put it is for an initial run session right
	after a \class{Qtcm} instance is created, \mods{varinit}
	will use the same initial values for prognostic variables
	(defined in \mods{defaults} module variable
	\vars{init\_prognostic\_dict})\footnote%
		{\vars{init\_prognostic\_dict} is the dictionary giving
		the default initial values of each prognostic variable
		and right-hand side (as defined by the restart file 
		specification).}
	as it would with for both
	\vars{init\_with\_instance\_state} set to True or False).

\item Continuation run sessions using this method have to continue
	with the next day from wherever the last run session left
	off, contiguously.\footnote%
		{For continuation run sessions, you keep the 
		same extension module (the compiled \fn{.so} library),
		and all the values that define the state where it
		left off.}
	If you want to do a non-contiguous run,
	create a new \class{Qtcm} instance initialized with a
	snapshot instead of the continuation method describe in
	this section.
	will use restart rules to run a new model.  

\item When making a continuation run session using this method,
	you cannot change some variables, for instance,
	\vars{outdir} and any of the date related
	variables.  In fact, the only thing you should change for
	your continuation run session are the prognostic and
	diagnostic variables and \vars{lastday}.  This is because
	some variables cannot be updated between run sessions.
	As noted in Section~\ref{sec:restart.options.list},
	if you wish to control all variables possible
	from the Python level (including output filename), you need 
	to execute the run session from a new model instance.
\end{itemize}


	\subsection{Snapshots of a \class{Qtcm} Instance}
				\label{sec:snapshots}

The snapshot dictionary (briefly described in
Section~\ref{sec:snapshot.intro}), saved as the \class{Qtcm} instance
attribute \vars{snapshot}, and generated by the method
\mods{make\_snapshot}, saves the current state of the following
instance field variables:

\begin{center}
\input{snapshot_vars.tex}
\end{center}

These are the same variables saved to a QTCM1 restart file, and so
a snapshot duplicates the restart functionality in the Python
environment, but with more flexibility.  Since the \vars{snapshot}
dictionary is a Python variable like any other, you can manipulate
it and alter it to fit any condition you wish.




%---------------------------------------------------------------------
\section{Creating and Using Run Lists}  \label{sec:runlists}

Section~\ref{sec:runlist.intro} provides an introduction to the
role and use of run lists.  A run list is a list of methods, Fortran
subroutines, and other run lists that can be executed by the
\class{Qtcm} instance \mods{run\_list} method.  Run lists are stored
in the \class{Qtcm} instance attribute \vars{runlists}, which is a
dictionary of run lists.  The names of run lists should not be
preceeded by two underscores (though elements of a run list may be
very private variables), nor should names of run lists be the same
as any instance attribute.  Run lists are not available for
\vars{compiled\_form\thinspace=\thinspace'full'}.

The \mods{run\_list} method takes a single input parameter, a list,
and runs through that list of elements that specify other run lists
or instance method names to execute.  Methods with private attribute
names are automatically mangled as needed to become executable by
the method.  Note that if an item in the input run list is an
instance method, it should be the entire name (not including the
instance name) of the callable method, separated by periods as
appropriate.

Elements in a run list are either strings or 1-element dictionaries.
Consider the following example, where \vars{model} is a \class{Qtcm}
instance, and \mods{run\_list} is called using \vars{mylist} as
input:

\begin{codeblock}
\codeblockfont{%
model = Qtcm(\ldots) \\
mylist = [ \{'varinit':None\}, \\
\hspace*{13ex}'init\_model', \\
\hspace*{13ex}'\_\_qtcm.driver', \\
\hspace*{13ex}\{'set\_qtcm1\_item': ['outdir', '/home/jlin']\} ]
model.run\_list(mylist)}
\end{codeblock}

The first element in \vars{mylist} refers to a method that requires
no positional input parameters be passed in (as shown by the None).
The second and third elements in \vars{mylist} also refers to methods
that require no positional input parameters be passed in.  The last
element in \vars{mylist} refers to a method with two input parameters.
Note that while I use the term ``method'' to describe the elements,
the strings/keys do not have to be only Python instance methods.
The second element, for instance, refers to another run list, and
the third element refers to a compiled QTCM1 model subroutine (note
the \vars{\_\_qtcm} attribute).

When the \mods{run\_list} method is called, the items in the input
run list are called in the order given in the list.  For each
element,  the \mods{run\_list} method first checks if the string
or dictionary key name corresponds to the key of an entry in the
\class{Qtcm} instance attribute \vars{runlists}.  If so, \mods{run\_list}
is called using that run list (i.e., it is a ``recursive'' call).
If the string or dictionary key name does not refer to another run
list, the \mods{run\_list} method checks if the string or dictionary
key name is a method of the \class{Qtcm} instance, and if so the
method is called.  Any other value throws an exception.

If input parameters for a method are of class \class{Field}, the
\mods{run\_list} method first tries to pass the parameters into the
method as is, i.e., as Field object(s).  If that fails, the
\mods{run\_list } method  passes its parameters in as the \vars{value}
attribute of the \class{Field} object.

If you want a variable that is being passed into a run list to be
continuously updated, you have to set the parameter in the run list
to a \class{Field} instance that is a \class{Qtcm} instance attribute,
not just to the value of the field variable (or to a non-\class{Field}
object).  Otherwise, subsequent calls to that run list element will
not use the updated values as input parameters.

For instance, if you had a run list element:
\begin{codeblock}
\codeblockfont{%
\{'\_\_qtcm.timemanager':[model.coupling\_day,]\}}
\end{codeblock}
and \vars{model.coupling\_day} were an integer (it's not by default,
but pretend it was), then \mods{run\_list} calling
\mods{\_\_qtcm.timemanager} will pass in a scalar integer rather
than a binding to the variable \vars{model.coupling\_day}.  In such
a situation, if the variable \vars{model.coupling\_day} were updated
in time, the \mods{run\_list} call of \mods{\_\_qtcm.timemanager}
would not be updated in time.  This happens because when the
dictionary that is the run list element is created, the value of
list element(s) attached to the dictionary element is set to the
scalar value of \vars{model.coupling\_day} at that instant.

You can get around this feature by setting \class{Qtcm} instance
attributes that will change with model execution to \class{Field}
instances, and then referring to those attributes in the parameter
list in the run list element.  In that case:
\begin{codeblock}
\codeblockfont{%
\{'\_\_qtcm.timemanager':[model.coupling\_day,]\}}
\end{codeblock}
will use the current value of \vars{model.coupling\_day} anytime
\vars{\_\_qtcm.timemanager} is called by \mods{run\_list}, if
\vars{model.coupling\_day} is a \class{Field} object.

When \mods{run\_list}, encounters a calling input parameter that
is a \class{Field} object, it will first try to pass the entire
\class{Field} object to the method/routine being called.  If that
raises an exception, it will then try to pass just the value of the
entire \class{Field} object.  This is done to enable \mods{run\_list}
to be used for both pure-Python and compiled QTCM Fortran model
routines.  Fortran cannot handle \class{Field} objects as input
parameters, only values.

Table~\ref{tab:stnd.runlists} shows all standard run lists
stored in the \vars{runlists} attribute upon instantiation
of a \class{Qtcm} instance.

\begin{htmlonly}
\begin{table}[htp]
\begin{center}
\fbox{Empty placeholder block for table that would have gone here.}
\end{center}
\caption{Standard run lists stored in the \vars{runlists} 
	attribute upon instantiation of a \class{Qtcm} instance.
	The run list and list element names are stored as strings.
	\emphpara{This table is improperly reproduced in the
	HTML conversion.  Please see the PDF version for the table.}}
\label{tab:stnd.runlists}
\end{table}
\end{htmlonly}

\begin{latexonly}
\begin{table}[htp]
\input{runlists}
\caption{Standard run lists stored in the \vars{runlists} 
	attribute upon instantiation of a \class{Qtcm} instance.
	The run list and list element names are stored as strings.}
\label{tab:stnd.runlists}
\end{table}
\end{latexonly}

Of course, feel free to change the contents of any of the run lists
after instantiation, or to add additional run lists to the
\vars{runlists} attribute dictionary.  The ability to alter run
lists at runtime gives the \mods{qtcm} package much of its flexibility.




%---------------------------------------------------------------------
\section{Field Variables and the \class{Field} Class}
						\label{sec:field.variables}

The term ``field'' variable refers to QTCM1 model variables that 
are accessible at both the compiled Fortran QTCM1 model-level as
well as the Python \class{Qtcm} instance-level.
Field variables are all instances of the \class{Field} class
(though non-field variables can also be instances of \class{Field}).

Section~\ref{sec:field.variables.intro} gives a brief introduction to
the attributes and methods in a \class{Field} instance.
A nitty gritty description of the class is found in its docstrings.

	\subsection{Creating Field Variables}

To create a \class{Field} instance whose value is set to the
default, instantiate with the field id as the only positional
input argument.  Thus:

\begin{codeblock}
\codeblockfont{foo = Field('lastday')}
\end{codeblock}

will return \vars{foo} as a \class{Field} instance with \vars{foo.value}
set to the value listed in Section~\ref{sec:defaults.scalar}.
The value of all \class{Field} instances upon creation are specified
in the \mods{defaults} submodule of package \mods{qtcm}, and listed
in Sections~\ref{sec:defaults.scalar} and~\ref{sec:defaults.array}.

To create \class{Field} instances whose attributes are set different
from their defaults, you can specify the different settings in the
instantiation parameter list, or change the attributes once the
instance is created.  See the \class{Field} docstring for details.


	\subsection{Initial Field Variables}  \label{sec:initial.variables}

Field variables include both model parameters that do not change
for a \class{Qtcm} instance as well as prognostic variables that
do change during model integration.  As a result, many field variables
have values different from the default values listed in
Sections~\ref{sec:defaults.scalar} and~\ref{sec:defaults.array}.
In this section, I list the \emph{initial} values of all field
variables.  The ``initial'' values are the settings for \class{Qtcm}
field variables execution of the \mods{run\_session} method, but
prior to cycling through an atmosphere-ocean coupling timestep.
This is in contrast to ``default'' values, which the field variables
are given on instantiation, if no other value is specified.
Numerical values are rounded as per the conventions
of Python's \vars{\%g} format code.


		\subsubsection{Scalars}

For the fields that give the input/output directory names, and the
run name, the entry ``value varies'' is provided in the ``Value''
column.

\input{init_scalars}

		\subsubsection{Arrays}

\input{init_arrays}


	\subsection{Passing Fields Between the Python and Fortran-Levels}

Section~\ref{sec:comm.py.fort.compiledform} discusses the differences
between how the \vars{'full'} and \vars{'parts'} compiled forms
pass field variables between the Python and Fortran-levels.  That
discussion gives a detailed description of the methods used for
passing fields to and from the Python and Fortran-levels (i.e., the
\mods{get\_qtcm1\_item} and \mods{set\_qtcm1\_item} methods).

Please note the following regarding field variables as you pass them 
back and forth between the Python and Fortran-levels:
\begin{itemize}
\item Field variables with ghost latitudes, such as \vars{u1}, on
	the Python end are always the full variables (i.e., including
	the ghost latitudes).  On the Fortran end, variables like
	\vars{u1} also always have the ghost latitudes while in the
	model, but when stored as restart files, do not have the
	ghost latitudes; the end points are not saved in restart
	files or written to the netCDF output files.
	See the
	\latexhtml{%
\htmladdnormallinkfoot{QTCM1 manual}%
        {http://www.atmos.ucla.edu/$\sim$csi/qtcm\_man/v2.3/qtcm\_manv2.3.pdf}}%
{\htmladdnormallink{QTCM1 manual}%
        {http://www.atmos.ucla.edu/~csi/qtcm_man/v2.3/qtcm_manv2.3.pdf}}
	\cite{Neelin/etal:2002}
	for details about ghost latitudes.

\item You should assume there is only a full synchronizing between 
	compiled QTCM1 model and Python model field variables
	at the beginning and end of a run session.  

\item If you have a variable at the Python-level, but at the
	compiled QTCM1 Fortran model-level the variable is not
	readable, if you try to call \mods{set\_qtcm1\_item} on the
	variable, nothing is done, and the Python-level value is
	left alone.  If you have a compiled QTCM1 model variable,
	but no Python-level equivalent, if you call \mods{set\_qtcm1\_item}
	on the variable, the Python-level variable (as an attribute)
	is created.

\item To be precise, only compiled QTCM1 model variables can be
	passed pass back and forth between the Python and Fortran-levels;
	there are many \class{Qtcm} instance attributes that do not
	have any counterparts at the Fortran-level.\footnote%
		{I use the term ``field variables'' to refer to 
		compiled QTCM1 model variables that can be passed
		back and forth to the Python level.}

\item Although \vars{dayofmodel} is described in module \mods{setbypy}
	as an option for the \mods{get\_qtcm1\_item} and
	\mods{set\_qtcm1\_item} methods to operate on, in reality
	those methods cannot operate on \vars{dayofmodel}, but
	\vars{dayofmodel} is not defined in \mods{defaults}.\footnote%
		{All field variables must be defined in \mods{defaults} in
		order for the proper Fortran routine to be called
		according to the variable's type.}
\end{itemize}


	\subsection{Field Variable Shape}   \label{sec:field.var.shape}

Normally, Python arrays have a different dimension order than Fortran
arrays.  While Fortran arrays are dimensioned (col, row, slice),
with adjacent columns being contiguous, then rows, and then slices, Python
arrays are dimensioned (slice, row, col), with adjacent columns being
contiguous, then rows, and then slices.  Based on this, you would
think that everytime you passed an array between the Python and
Fortran-levels you would need to transpose the array.

Thankfully, we don't have to do this because \mods{f2py} handles
array dimension order transparently so we can refer to each element
the same way whether we're in Python or Fortran.  Thus, the array
\vars{Qc} in Fortran is dimensioned (longitude, latitude), (64,42)
by default, and the Python \class{Qtcm} instance attribute \vars{Qc}
has a \vars{value} attribute also dimensioned (longitude, latitude),
(64,42) by default.  And at both the Fortran and Python-levels, the
first longtude, second latitude element is referred to as \vars{Qc(1,2)}.

In contrast, however, netCDF output saved by the compiled QTCM1 model
and read into Python (using the \mods{Scientific} package) is
\emph{not} in Fortran array order.  Arrays read from netCDF output
into Python are in Python array order, and are dimensioned
(latitude, longitude) or (time, latitude, longitude).  The \class{Qtcm}
routines that manipulate netCDF data (e.g., \mods{plotm}), however,
automatically adjust for this, so you only need to be aware of this
when reading in output for your own analysis
(see Section~\ref{sec:model.output}).




%---------------------------------------------------------------------
\section{Model Output}			\label{sec:model.output}

Section~\ref{sec:output.intro} gives an overview of how to
use \mods{qtcm} model output to netCDF files.

All netCDF array output is dimensioned (time, latitude, longitude)
when read into Python using the \mods{Scientific} package.  This
differs from the way \class{Qtcm} saves field variables, which
follows Fortran convention (longitude, latitude).  Thus, the shapes
in Section~\ref{sec:initial.variables}, Appendix~\ref{app:defaults.values},
etc., are not the shapes of arrays read from the netCDF output.
See Section~\ref{sec:field.var.shape} for a discussion of why
there is this discrepancy.

Because netCDF files allow you to specify an ``unlimited'' dimension,
it is possible to close a netCDF file, reopen it, and add more
slices of data to the file.  Thus, continuous \class{Qtcm} run
sessions (i.e., those that use the \vars{cont} keyword input parameter
in the \mods{run\_session} method) will automatically append output
to the netCDF output files.

Field variables with ghost latitudes, such as \vars{u1}, on the
Python and Fortran ends are always the full variables (i.e., including
the ghost latitudes).  The ghost latitudes are not written to the
netCDF output files, however.
See the \latexhtml{%
\htmladdnormallinkfoot{QTCM1 manual}%
        {http://www.atmos.ucla.edu/$\sim$csi/qtcm\_man/v2.3/qtcm\_manv2.3.pdf}}%
{\htmladdnormallink{QTCM1 manual}%
        {http://www.atmos.ucla.edu/~csi/qtcm_man/v2.3/qtcm_manv2.3.pdf}}
	\cite{Neelin/etal:2002}
for details about ghost latitude structure.

\class{Qtcm} instances have a few built-in tools to visualization
model output.  These are briefly described in Section~\ref{sec:viz.intro}.
Note that the \mods{plotm} method is linked to a specific \class{Qtcm}
instance.  Do not use \mods{plotm} outside of the instance it is
linked to.  It must also be used only after a successful run session
(i.e., not in the middle of a run session).




%---------------------------------------------------------------------
\section{Miscellaneous}

A few miscellaneous items/issues about the model:
\begin{itemize}
\item The land model runs at same timestep as the atmosphere.

\item If the land model runs less often than 
	\mods{sflux} in \mods{physics1}, 
	the calculation of evaporation over the land 
	needs to be fixed in sflux.

\item The units of some field variables are not what you would expect.
	For instance, \vars{Qc} is in energy units, i.e., K, and not
	mm/day.
	See the
	\latexhtml{%
\htmladdnormallinkfoot{QTCM1 manual}%
        {http://www.atmos.ucla.edu/$\sim$csi/qtcm\_man/v2.3/qtcm\_manv2.3.pdf}}%
{\htmladdnormallink{QTCM1 manual}%
        {http://www.atmos.ucla.edu/~csi/qtcm_man/v2.3/qtcm_manv2.3.pdf}}
	\cite{Neelin/etal:2002}
	for details.
\end{itemize}




%---------------------------------------------------------------------
\section{Cookbook of Ways the Model Can Be Used}  \label{sec:cookbook}

This cookbook of a few ways to use the model is arranged by science
tasks, i.e., certain types of runs we want to do.  For some of the
examples below, I assume that the dictionary
\vars{inputs} is initially defined as given in
Figure~\ref{fig:defn.of.inputs}.  All examples assume that
\cmd{from qtcm import Qtcm} has already been executed.


%--- Two versions, one for PDF and the other for HTML:
\begin{latexonly}
\begin{figure}[tp]
\begin{codeblock}
\codeblockfont{%
inputs = \{\} \\
inputs['runname'] = 'test' \\
inputs['landon'] = 0 \\
inputs['year0'] = 1 \\
inputs['month0'] = 11 \\
inputs['day0'] = 1 \\
inputs['lastday'] = 30 \\
inputs['mrestart'] = 0 \\
inputs['init\_with\_instance\_state'] = True \\
inputs['compiled\_form'] = 'parts'}
\end{codeblock}

\caption{The initial definition of the \vars{inputs} dictionary for 
	examples given in Section~\ref{sec:cookbook}.  These settings
	imply that a run session will start on November 1, Year 1,
	last for 30 days, and will be an aquaplanet run.}
\label{fig:defn.of.inputs}
\end{figure}
\end{latexonly}

\begin{htmlonly}
\label{fig:defn.of.inputs}
\begin{center}
\htmlfigcaption{%
	\codeblockfont{%
inputs = \{\} \\
inputs['runname'] = 'test' \\
inputs['landon'] = 0 \\
inputs['year0'] = 1 \\
inputs['month0'] = 11 \\
inputs['day0'] = 1 \\
inputs['lastday'] = 30 \\
inputs['mrestart'] = 0 \\
inputs['init\_with\_instance\_state'] = True \\
inputs['compiled\_form'] = 'parts'}
	}

\htmlfigcaption{Figure~\ref{fig:defn.of.inputs}:
	The initial definition of the \vars{inputs} dictionary for 
	examples given in Section~\ref{sec:cookbook}.  These settings
	imply that a run session will start on November 1, Year 1,
	last for 30 days, and will be an aquaplanet run.}
\end{center}
\end{htmlonly}



\begin{description}
\item[Plain model run:]
	Here I just want to make a single model run.
	Tasks:  Instantiate a fresh model and execute a run session.
	The code to run the model is just:
	\begin{codeblock}
	\codeblockfont{%
inputs['init\_with\_instance\_state'] = False \\
model = Qtcm(**inputs) \\
model.run\_session()}
	\end{codeblock}
	where \vars{inputs} is initialized with the code in
	Figure~\ref{fig:defn.of.inputs}.


\item[Explore parameter space with a set of models:]
	Here I want to create an entire suite of separate models,
	in order to determine the sensitivity of the model to changing
	a parameter.
	To do this, I
	instantiate multiple fresh models, 
	and execute a run session for each instance, all within
	a \vars{for} loop:


%--- Two versions, because LaTeX2HTML does not correctly typeset
%    the hspace command:
\begin{latexonly}
	\begin{codeblock}
	\codeblockfont{%
import os \\
inputs['init\_with\_instance\_state'] = False \\
for i in xrange(0,1002,10): \\
\hspace*{5ex}iname = 'ziml-' + str(i) + 'm' \\
\hspace*{5ex}ipath = os.path.join('proc', iname) \\
\hspace*{5ex}os.makedirs(ipath) \\
\hspace*{5ex}model = Qtcm(**inputs) \\
\hspace*{5ex}model.ziml.value = float(i)  \\
\hspace*{5ex}model.runname.value = iname \\
\hspace*{5ex}model.outdir.value = ipath \\
\hspace*{5ex}model.run\_session() \\
\hspace*{5ex}del model}
	\end{codeblock}
\end{latexonly}

\begin{htmlonly}
\begin{center}
\htmlfigcaption{%
	\codeblockfont{%
import os \\
inputs['init\_with\_instance\_state'] = False \\
for i in xrange(0,1002,10): \\
\hspace*{5ex}iname = 'ziml-' + str(i) + 'm' \\
\hspace*{5ex}ipath = os.path.join('proc', iname) \\
\hspace*{5ex}os.makedirs(ipath) \\
\hspace*{5ex}model = Qtcm(**inputs) \\
\hspace*{5ex}model.ziml.value = float(i)  \\
\hspace*{5ex}model.runname.value = iname \\
\hspace*{5ex}model.outdir.value = ipath \\
\hspace*{5ex}model.run\_session() \\
\hspace*{5ex}del model}
	}
\end{center}
\end{htmlonly}


	The loop explores mixed-layer depth \vars{ziml} from 0~m to
        1000~m, in 10~m intervals.  I create the \vars{outdir}
	directory before every model call, since the compiled QTCM1 model
	requires the output directory exist, specifying the run name
	and output directory as the string \vars{iname}.
	The output directories are assumed to all be in the \fn{proc}
	sub-directory of the current working directory.
	\vars{inputs} is initialized with the code in
	Figure~\ref{fig:defn.of.inputs}.


\item[Conditionally explore parameter space:]
	Here I want to 
	conditionally explore the parameter space, on the basis of
	some mathematical criteria.
	To do this, I
	instantiate a model, evaluate results using
	that criteria, and run another fresh model depending on
	the results (passing the previous model state via a snapshot),
	all within a \vars{while} loop.
	Note that this type of investigation is very difficult to 
	automate if all you can use are shell scripts and
	Fortran.
	See Figure~\ref{fig:conditional.test.eg} for a detailed
	example.


\item[With interactive adjustments at run time:]
	The example in Figure~\ref{sec:continuation.intro}
	illustrates this type of run.  In this example,
	I instantiate a fresh model, execute a run session, analyze the
	output, change variables in the model instance, and then
	execute a continuation run session.


\item[Test alternative parameterizations:]
	I've already described how we can use run lists to arbitrarily
	change model execution order and content at run time.
	We can take advantage of Python's inheritance
	abilities, along with run lists, to simplify this.
	Figure~\ref{fig:alt.param.inherit.eg} provides an example of
	this use.

	Of course, you can use pre-processor directives and shell
	scripts to accomplish the same functionality seen in
	Figure~\ref{fig:alt.param.inherit.eg} using just Fortran.
	The Python solution, however, shortcuts the compile/linking
	step, and enables you to easily do run time swapping between
	subroutine choices based upon run time calculated
	tests (see Figure~\ref{fig:conditional.test.eg} for an
	example of such tests).
\end{description}




% --- Two versions of this block, one for display in PDF and the other
%     for display in HTML:
\begin{latexonly}
\begin{figure}[p]
	\begin{codeblock}
	\codeblockfont{%
\small
import os \\
import numpy as N \\
maxu1 = 0.0 \\
while maxu1 < 10.0: \\
\hspace*{5ex}iziml = 0.1 * maxu1 \\
\hspace*{5ex}iname = 'ziml-' + str(iziml) + 'm' \\
\hspace*{5ex}ipath = os.path.join('proc', iname) \\
\hspace*{5ex}os.makedirs(ipath) \\
\hspace*{5ex}model = Qtcm(**inputs) \\
\hspace*{5ex}try: \\
\hspace*{10ex}model.sync\_set\_py\_values\_to\_snapshot(snapshot=mysnapshot) \\
\hspace*{10ex}model.init\_with\_instance\_state = True \\
\hspace*{5ex}except: \\
\hspace*{10ex}model.init\_with\_instance\_state = False \\
\hspace*{5ex}model.ziml.value = iziml  \\
\hspace*{5ex}model.runname.value = iname \\
\hspace*{5ex}model.outdir.value = ipath \\
\hspace*{5ex}model.run\_session() \\
\hspace*{5ex}maxu1 = N.max(N.abs(model.u1.value)) \\
\hspace*{5ex}mysnapshot = model.snapshot \\
\hspace*{5ex}del model}
	\end{codeblock}

\caption{This code explores different values of
	mixed-layer depth \vars{ziml} for 30~day runs,
	as a function of maximum \vars{u1} magnitude,
	until it finds a case where the maximum \vars{u1} is
	greater than 10~m/s.  (The relationship between
	\vars{ziml} and the maximum of the speed of
	\vars{u1}, where 
	\vars{ziml\thinspace=\thinspace0.1\thinspace*\thinspace{maxu1}}, 
	is made up.)
	With each iteration, the new run uses the snapshot from
	a previous run to initialize (as well as the new value
	of \vars{ziml}); the \vars{try} statement is used to
	ensure the model works even if \vars{mysnapshot} is not
	defined (which is the case the first time around).
	The \vars{inputs} dictionary is initialized with the code in
	Figure~\ref{fig:defn.of.inputs}.}
\label{fig:conditional.test.eg}
\end{figure}
\end{latexonly}

\begin{htmlonly}
\label{fig:conditional.test.eg}
\begin{center}
\htmlfigcaption{%
	\codeblockfont{%
import os \\
import numpy as N \\
maxu1 = 0.0 \\
while maxu1 < 10.0: \\
\hspace*{5ex}iziml = 0.1 * maxu1 \\
\hspace*{5ex}iname = 'ziml-' + str(iziml) + 'm' \\
\hspace*{5ex}ipath = os.path.join('proc', iname) \\
\hspace*{5ex}os.makedirs(ipath) \\
\hspace*{5ex}model = Qtcm(**inputs) \\
\hspace*{5ex}try: \\
\hspace*{10ex}model.sync\_set\_py\_values\_to\_snapshot(snapshot=mysnapshot) \\
\hspace*{10ex}model.init\_with\_instance\_state = True \\
\hspace*{5ex}except: \\
\hspace*{10ex}model.init\_with\_instance\_state = False \\
\hspace*{5ex}model.ziml.value = iziml  \\
\hspace*{5ex}model.runname.value = iname \\
\hspace*{5ex}model.outdir.value = ipath \\
\hspace*{5ex}model.run\_session() \\
\hspace*{5ex}maxu1 = N.max(N.abs(model.u1.value)) \\
\hspace*{5ex}mysnapshot = model.snapshot \\
\hspace*{5ex}del model}
	}

\htmlfigcaption{Figure \ref{fig:conditional.test.eg}:
	This code explores different values of
	mixed-layer depth \vars{ziml} for 30~day runs,
	as a function of maximum \vars{u1} magnitude,
	until it finds a case where the maximum \vars{u1} is
	greater than 10~m/s.  (The relationship between
	\vars{ziml} and the maximum of the speed of
	\vars{u1}, where 
	\vars{ziml\thinspace=\thinspace0.1\thinspace*\thinspace{maxu1}}, 
	is made up.)
	With each iteration, the new run uses the snapshot from
	a previous run to initialize (as well as the new value
	of \vars{ziml}); the \vars{try} statement is used to
	ensure the model works even if \vars{mysnapshot} is not
	defined (which is the case the first time around).
	The \vars{inputs} dictionary is initialized with the code in
	Figure~\ref{fig:defn.of.inputs}.}
\end{center}
\end{htmlonly}


% --- Two versions of this block, one for display in PDF and the other
%     for display in HTML:
\begin{latexonly}
\begin{figure}[p]
\begin{center}
	\begin{codeblock}
	\codeblockfont{%
\small
import os \\
\\
class NewQtcm(Qtcm): \\
\hspace*{5ex}def cloud0(self):\\
\hspace*{10ex}[\ldots] \\
\hspace*{5ex}def cloud1(self):\\
\hspace*{10ex}[\ldots] \\
\hspace*{5ex}def cloud2(self):\\
\hspace*{10ex}[\ldots] \\
\hspace*{5ex}[\ldots] \\
\\
inputs['init\_with\_instance\_state'] = False \\
for i in xrange(10): \\
\hspace*{5ex}iname = 'cloudroutine-' + str(i)  \\
\hspace*{5ex}ipath = os.path.join('proc', iname) \\
\hspace*{5ex}os.makedirs(ipath) \\
\hspace*{5ex}model = NewQtcm(**inputs) \\
\hspace*{5ex}model.runlists['atm\_physics1'][1] = 'cloud' + str(i) \\
\hspace*{5ex}model.runname.value = iname \\
\hspace*{5ex}model.outdir.value = ipath \\
\hspace*{5ex}model.run\_session() \\
\hspace*{5ex}del model}
	\end{codeblock}
\end{center}

\caption{Let's say we have 9 different cloud physics schemes we wish
	to try out in 9 different runs.  The easiest way to do this
	is to create a new class \class{NewQtcm} that
	inherits everything from \class{Qtcm}, and to which we'll
	add the additional cloud schemes (\vars{cloud0}, \vars{cloud1},
	etc.).
	In the \vars{for} loop, I change the cloud model
	run list entry in the run list that governs
	atmospheric physics at one instant to whatever the cloud
	model is at this point in the loop.
	The \vars{inputs} dictionary is initialized with the code in
	Figure~\ref{fig:defn.of.inputs}.
	Of course, we could do the same thing by running the 9
	models separately, but this set-up makes it easy to do
	hypothesis testing with these 9 models.  For instance, we
	can create a test by which we will choose which of the 9
	models to use:  Within this framework, the selection of
	those models can be altered by changing a string.}
\label{fig:alt.param.inherit.eg}
\end{figure}
\end{latexonly}

\begin{htmlonly}
\label{fig:alt.param.inherit.eg}
\begin{center}
\htmlfigcaption{%
	\codeblockfont{%
import os \\
\\
class NewQtcm(Qtcm): \\
\hspace*{5ex}def cloud0(self):\\
\hspace*{10ex}[\ldots] \\
\hspace*{5ex}def cloud1(self):\\
\hspace*{10ex}[\ldots] \\
\hspace*{5ex}def cloud2(self):\\
\hspace*{10ex}[\ldots] \\
\hspace*{5ex}[\ldots] \\
\\
inputs['init\_with\_instance\_state'] = False \\
for i in xrange(10): \\
\hspace*{5ex}iname = 'cloudroutine-' + str(i)  \\
\hspace*{5ex}ipath = os.path.join('proc', iname) \\
\hspace*{5ex}os.makedirs(ipath) \\
\hspace*{5ex}model = NewQtcm(**inputs) \\
\hspace*{5ex}model.runlists['atm\_physics1'][1] = 'cloud' + str(i) \\
\hspace*{5ex}model.runname.value = iname \\
\hspace*{5ex}model.outdir.value = ipath \\
\hspace*{5ex}model.run\_session() \\
\hspace*{5ex}del model}
	}

\htmlfigcaption{Figure \ref{fig:alt.param.inherit.eg}:
	Let's say we have 9 different cloud physics schemes we wish
	to try out in 9 different runs.  The easiest way to do this
	is to create a new class \class{NewQtcm} that
	inherits everything from \class{Qtcm}, and to which we'll
	add the additional cloud schemes (\vars{cloud0}, \vars{cloud1},
	etc.).
	In the \vars{for} loop, I change the cloud model
	run list entry in the run list that governs
	atmospheric physics at one instant to whatever the cloud
	model is at this point in the loop.
	The \vars{inputs} dictionary is initialized with the code in
	Figure~\ref{fig:defn.of.inputs}.
	Of course, we could do the same thing by running the 9
	models separately, but this set-up makes it easy to do
	hypothesis testing with these 9 models.  For instance, we
	can create a test by which we will choose which of the 9
	models to use:  Within this framework, the selection of
	those models can be altered by changing a string.}
\end{center}
\end{htmlonly}




% ===== end of file =====


%@@@\chapter{Combining \code{qtcm} with \code{CliMT}}
%@@@% ==========================================================================
% CliMT
%
% By Johnny Lin
% ==========================================================================


% ------ BODY -----
%
\section{General Tutorial on CliMT}


General notes of things I think I may have observed about
\code{Parameters} objects:
\begin{itemize}
\item You can treat a \code{Parameters} instance as a dictionary, where
	the key is the name of the field, because \code{\_\_getitem\_\_},
	etc.\ have been defined for the instance.  However, the values,
	units, and long names of the fields are stored in dictionaries
	assigned to \code{value}, \code{units}, and \code{long\_name},
	keyed to the field name (a string).
\end{itemize}


General notes of things I think I may have observed about
\code{Components} objects:
\begin{itemize}
\item All variables and quantities, whether they be physical fields,
	filenames, or metadata,
	are stored as attributes in the \code{Components} instance.
\item \code{Components} have these special attributes:
        \code{Required},
        \code{Prognostic},
	and
        \code{Diagnostic},
	which are lists that contain the names of describe whether
\item Scalar parameters in \code{Component} objects
	are stored as an instance of the \code{Parameters}
	class, under the attribute \code{Params}.
\end{itemize}


General notes of things I think I may have observed about
\code{Federation} objects:
\begin{itemize}
\item \code{Federation} objects hold the \code{Components} instances
	in a list assigned to the attribute \code{list}.
\item \code{Federation} attributes
        \code{Required},
        and
	\code{Prognostic},
	are unions of the same attributes of the constituent
	\code{Components} objects.
\end{itemize}






% ===== end of file =====


\chapter{Troubleshooting}                   \label{ch:trouble}
% ==========================================================================
% Troubleshooting
%
% By Johnny Lin
% ==========================================================================


% ------ BODY -----
%
\section{Error Messages Produced by \mods{qtcm}}

\begin{description}
\item[\screen{Error-Value too long in SetbyPy module getitem\_str for}
	\dumarg{key}:]
	This message is produced by the Fortran
	subroutine \mods{getitem\_str}
	in the module \mods{SetbyPy} in the compiled QTCM1 Fortran code.
	The code is in the file \fn{setbypy.F90}.  This error occurs when
	the Fortran variable whose name is given by the string \dumarg{key}
	has a value that is greater than the local parameter
	\vars{maxitemlen} in \mods{getitem\_str}.  To fix this, you have
	to go into \fn{setbypy.F90} and change the value of
	\vars{maxitemlen}.

\item[\screen{Error-real\_rank1\_array should be deallocated}:]
	Fortran module \mods{SetByPy}'s subroutine
	\mods{getitem\_real\_array} generates this message
	(or a similar message for other ranks) if the Fortran
	variable for the input \dumarg{key} are allocated on entry
	to the routine.  This may indicate the user has written another
	Fortran routine to access the \mods{real\_rank1\_array} variable
	outside of the standard interfaces..

\item[\screen{Error-Bad call to SetbyPy module \ldots}:]
	Often times, this error occurs because a get or set routine
	in \mods{SetByPy} tried to act on a variable for which the
	corresponding input \dumarg{key} is not defined.  The solution
	is to add that case in the if/then construct for the get and set
	routines in \mods{SetByPy} and rebuild the extension modules.
\end{description}


\section{Other Errors}

\begin{description}
\item[Python cannot find some packages:]
	This error often happens when the version of Python in which
	you have installed all your packages is not the version that
	is called at the Unix command line by typing in \cmd{python}.
	To get around this, 
        define a Unix alias
        that maps \cmd{python2.4} (or whichever version of Python
	has all your packages installed) to \cmd{python}.  If you
	have multiple Python's installed on your system, you might
	have to use a more specific name for the Python executable.
	As a result, you may have to change the test scripts in
	\fn{test} in the \mods{qtcm} distribution directory.

\item[\mods{get\_qtcm1\_item} and compiled QTCM1 model pointer
	variables:]
	If you try to use the \mods{get\_qtcm1\_item} method on a compiled
	QTCM1 model pointer variable 
	(i.e., \vars{u1}, \vars{v1}, \vars{q1}, \vars{T1}),
	 before the compiled
	model \mods{varinit} subroutine is run, you'll get a bus error
	with no additional message.

\item[Mismatch between Python and Fortran array field variables:]
	You change an array field variable on the Python side, but
	it seems like the wrong elements are changed on the Fortran
	side.  Or you type in the same index address for accessing a
	\mods{qtcm} netCDF output array as well as its \class{Qtcm}
	instance attribute counterpart, and find you get different
	answers.  Some possible reasons and fixes:

	\begin{itemize}
	\item This will occur if you haven't accounted for the
		difference in how field variables are saved at the
		Python-level, Fortran-level, and in a netCDF file.
		All netCDF array output is dimensioned (time,
		latitude, longitude) when read into Python using
		the \mods{Scientific} package.  This differs from
		the way \class{Qtcm} saves field variables, \emph{both}
		at the Python- and Fortran-levels, which follows
		Fortran convention (longitude, latitude).

		Note that the way \class{Qtcm} saves field variables
		at the Python- and Fortran-levels is different than
		the default way Python and Fortran save arrays.
		Section~\ref{sec:field.var.shape} for more information.

	\item You may have forgotten that array indices in Python start at
		0, while indices in Fortran (generally) start at 1.
		Also, ranges in Python are exclusive at the upper-bound,
		while ranges in Fortran are inclusive at the upper-bound.
		(Both Python and Fortran array indice ranges are inclusive
		at the lower-bound.)

	\item You may have forgotten some field variables have
		ghost latitudes, and thus there are extra latitude bands
		when the array is stored as a Python or Fortran field
		variable, but there are \emph{no} extra latitude bands
		when the array is stored as netCDF output (the QTCM1
		output routines strip off the ghost latitudes when
		writing those field variables out).
	        See the
        \latexhtml{%
\htmladdnormallinkfoot{QTCM1 manual}%
        {http://www.atmos.ucla.edu/$\sim$csi/qtcm\_man/v2.3/qtcm\_manv2.3.pdf}}%
{\htmladdnormallink{QTCM1 manual}%
        {http://www.atmos.ucla.edu/~csi/qtcm_man/v2.3/qtcm_manv2.3.pdf}}
        \cite{Neelin/etal:2002}
        for details about ghost latitudes.

		The safest and easiest way to tell whether the variable has a
		ghost latitudes is to look at its shape.
		A call to the \class{Qtcm} instance
		method \mods{get\_qtcm1\_item} will give you the array,
		and the use of NumPy's \mods{shape} function will give you
		the shape.
	\end{itemize}
\end{description}




% ===== end of file =====


\chapter{Developer Notes}                   \label{ch:devnotes}
% ==========================================================================
% Using QTCM
%
% By Johnny Lin
% ==========================================================================


% ------ BODY -----
%

%---------------------------------------------------------------------
\section{Introduction}

This section describes programming practices and issues related to
the \mods{qtcm} package that might be of interest to users wishing
to add/change code in the package.
Please see the package
\latexhtml{API documentation,%
		\footnote{http://www.johnny-lin.com/py\_pkgs/qtcm/doc/html-api/}
		which includes the source code}%
        {\htmladdnormallink{API documentation}%
		{http://www.johnny-lin.com/py\_pkgs/qtcm/doc/html-api/},
		which includes the source code},
for details.




%---------------------------------------------------------------------
\section{Changes to QTCM1 Fortran Files}  \label{sec:f90changes}

The source code used to generate the shared object files used
in this Python package is unchanged
from the pure-Fortran QTCM1 model source code, except in the
following ways:

\begin{itemize}
\item The suffix of all source code files 
	has been changed from \fn{.f90} to \fn{.F90}, 
	in order to ensure the compiler preprocesses 
	the source code.  Some compilers use the capitalization to
	tell whether or not to run the code through a preprocessor.

\item In all \fn{.F90} files, occurrences of:
	\begin{codeblock}
	\codeblockfont{%
	Character(len=130)}
	\end{codeblock}
	are changed to:
	\begin{codeblock}
	\codeblockfont{%
	Character(len=305)}
	\end{codeblock}
	This enables the model to properly deal with longer filenames.
	The number ``305'' is chosen to make search and replace easier.

\item In \fn{qtcmpar.F90}, the 
	\vars{eps\_c} variable is changed from an unchangable
	parameter to a changeable real, 
	so that it can be changed in the model at runtime.

\item All occurrences of an underscore (``\_'') character in a
	subroutine or function name are removed.  The
	presence of the underscore messes up the dynamic lookup
	mechanism for the \mods{f2py} generated extension module.
	The following names are changed, both in subroutine definitions
	and calls:
	\begin{itemize}
	\item \mods{out\_restart} to \mods{outrestart},
	\item \mods{save\_bartr} to \mods{savebartr},
	\item \mods{grad\_phis} to \mods{gradphis}.
	\end{itemize}

\item \fn{driver.F90} is changed so that program
	\mods{driver} becomes a subroutine, and 
	subroutine \mods{driverinit} is deleted (along with
	all calls to it) because basic model initialization is
	handled at the Python level.

\item In \fn{clrad.F90}, subroutine \mods{cloud}, the first
	\vars{COUNTCAP} preprocessor macro, a comment line for
	that ifdef is moved to prevent a warning message during
	building with \mods{f2py}.

\item The order of subroutine \mods{qtcminit} is changed.  The original
	pure-Fortran QTCM1 \mods{qtcminit} code has the following
	calling sequence:

	\begin{codeblock}
        \codeblockfont{%
Call parinit            !Initialize model parameters \\
Call varinit            !Initialize variables \\
Call TimeManager(1)     !mm set model time \\
Call bndinit            !input boundary datasets \\
Call physics1           !diagnostic fields for initial condition}
	\end{codeblock}

	For the \mods{qtcm} package, I've altered this order so
	\mods{bndinit} comes after \mods{parinit} but before \mods{varinit}:
	\begin{codeblock}
        \codeblockfont{%
Call parinit            !Initialize model parameters \\
Call bndinit            !input boundary datasets \\
Call varinit            !Initialize variables \\
Call TimeManager(1)     !mm set model time  \\
Call physics1           !diagnostic fields for initial condition}
	\end{codeblock}

	This is done because \vars{STYPE} is not read in for the
	\vars{landon} \vars{True} case until \mods{bndinit}, but
	in \mods{varinit} \vars{STYPE} is used to calculate the
	original values of \vars{WD} for the non-restart case.  This
	also corrects the conflicting order found in the pure-Fortran
	QTCM1 manual (compare pp.\ 29 and 32).  As far as I can
	tell, \mods{bndinit} has no dependencies that require it
	to come after \mods{timemanager} or \mods{varinit}.

\end{itemize}

In addition, the Fortran files \fn{setbypy.F90}, \fn{wrapcall.F90},
and \fn{varptrinit.F90} are added.  The routines in these files, 
however, just add more flexibility and functionality to the model;
they do not automatically affect any model computations.  See
Section~\ref{sec:newf90} for details.




%---------------------------------------------------------------------
\section{New Interfaces and Fortran Functionality}  \label{sec:newf90}

As described in Section~\ref{sec:f90changes}, the Fortran files
\fn{setbypy.F90}, \fn{wrapcall.F90}, and \fn{varptrinit.F90} are
added to the QTCM1 source directory.  The first two files define the Fortran
90 modules (\mods{SetbyPy} and \mods{WrapCall}) needed to interface
the Python and Fortran levels.  The last file defines a new Fortran
subroutine \mods{varptrinit} that associates QTCM1 model pointer
variables at the Fortran level.  In a pure-Fortran run of QTCM1,
this occurs in subroutine \mods{varinit}; for a
\vars{compiled\_form\thinspace=\thinspace'parts'} run, since the
functionality of the Fortran \mods{varinit} is now in the Python
\mods{varinit} method, a separate Fortran pointer association
subroutine needed to be defined.  The Fortran subroutine \mods{varptrinit}
is called as the \mods{varptrinit} function of the 
\vars{compiled\_form\thinspace=\thinspace'parts'}
\fn{.so} extension module.


	\subsection{Fortran Module \mods{SetbyPy}}   \label{sec:setbypy}

		\subsubsection{Design Description}

This module defines functions and subroutines used to read variables
from the Fortran-level to the Python-level, and in setting Fortran-level
variables using the Python-level values.  These routines are used
by \class{Qtcm} methods \mods{get\_qtcm1\_item} and \mods{set\_qtcm1\_item}
(and dependencies thereof) to ``get'' and ``set'' the Fortran-level
variables.  Note that the Fortran module \mods{SetbyPy} is referred
to in lowercase at the Python level, i.e., as the
attribute \vars{\_\_.qtcm.setbypy} of a \class{Qtcm} instance.

Because Fortran variables are not dynamically typed, separate Fortran
functions and subroutines need to be defined to get and set variables
of different types.\footnote%
	{The \mods{interface} construct in Fortran 90 is supposed to
	provide a way to create a single interface to multiple
	routines, e.g.:
	\begin{codeblock}
	\codeblockfont{%
Interface setitem \\
\hspace*{3ex}Module Procedure setitem\_real, setitem\_int, setitem\_str \\
End Interface}
	\end{codeblock}
	This construct, however, causes a bus error
	(Mac OS X 10.4, Intel).  Thus, I put the
	same functionality in the Python code.}
The \class{Qtcm} methods \mods{get\_qtcm1\_item}
and \mods{set\_qtcm1\_item} know which one of the Fortran routines
to call on the basis of the type and rank of the value for the field
variable in the \mods{defaults} submodule.  This is why all field
variables need to have defaults defined in \mods{defaults}.  For
array variables, the field variable defaults also provide the rank
of the Fortran-level variable being gotten or set.  However, the
array default values do \emph{not} have to have the same shape as
the Fortran-level variables; on the Python-side, variable shape
adjusts to what is declared on the Fortran-side.  
Thus, if you change the resolution of
the compiled QTCM1 model, you do not have to change the dimensions
of the field variable values in \mods{defaults}.

The \class{Qtcm} method \mods{get\_qtcm1\_item} directly calls
the \mods{SetByPy} routines.
The \class{Qtcm} method \mods{set\_qtcm1\_item} makes use of
private instance methods that make the calls to the \mods{SetByPy} routines.

For scalar field variables, \mods{SetByPy} provides functions and
subroutines that provide the value of the variable on output.
For array field variables, \mods{SetByPy}
dynamic \emph{module} arrays are used to pass array
variables in and out; I could not get the 
\mods{SetByPy} Fortran routines to set
locally defined dynamic arrays (that is, locally within a function or
subroutine).\footnote%
	{I tried to implement Fortran subroutine
	\mods{getitem\_real\_array} using traditional array 
	dimension passing 
	(e.g., \code{subroutine foo(nx, ny, a)}) as well
	as declaring the allocatable array inside the subroutine, 
	but neither option worked on my \mods{f2py} (version 2\_3816) 
	and Python (version 2.4.3).}
In the \mods{SetByPy} module, these dynamic arrays
are defined as follows:

\begin{codeblock}
\codeblockfont{%
Real, allocatable, dimension(:) :: real\_rank1\_array \\
Real, allocatable, dimension(:,:) :: real\_rank2\_array \\
Real, allocatable, dimension(:,:,:) :: real\_rank3\_array}
\end{codeblock}

For all field variables, scalar or array, the \mods{SetByPy} module
has a fourth module variable defined, \vars{is\_readable}, that the
Fortran get and set routines will set to \vars{.TRUE.} if the
variable is readable and \vars{.FALSE.} if not (it's declared as a
logical variable).  This Fortran variable can be used to prevent
Python from accessing pointer variables that aren't yet associated
to targets.

In general, \mods{SetByPy} routines make use of Fortran constructs
to enable them to accomodate all possible
variables of a given type and shape.  However, 
for string scalars, the \mods{SetByPy} function \mods{getitem\_str}
has to have a return value of a predefined length, in order to
work properly.  That length is given by the parameter
\vars{maxitemlen} and is set to 505 (the value is chosen to
be larger than all filename variables described in
Section~\ref{sec:f90changes} and to be easily found in
the \fn{.F90} files).


		\subsubsection{Module Structure}

If you're a Fortran programmer, you can probably get all the information
in this section from just reading the \fn{setbypy.F90} file directly.
This description of the module structure, however, permits me to highlight
what you need to do if you want to make additional compiled QTCM1 variables
accessible to Python \class{Qtcm} objects.

\begin{itemize}
\item All \mods{Use} statements are given in the beginning of 
	the \mods{SetByPy} module.  These statements cover
	nearly all of the QTCM1 Fortran
	modules that contain variables of interest.  If the
	QTCM1 variable you're interested in isn't in a module
	listed here, you'll have to add your own
	\mods{Use} statement of that module here.

\item Next comes the definitions for the
	\vars{real\_rank1\_array},
	\vars{real\_rank2\_array}, and
	\vars{real\_rank3\_array} dynamic array variables, and
	the \vars{is\_readable} boolean variable.

\item The \mods{Contains} block of the module defines the module
	routines called by the \class{Qtcm} instance methods to
	set and get the compiled QTCM1 model variables.  The
	routines are:
	\begin{itemize}
	\item Function \mods{getitem\_real}
	\item Subroutine \mods{getitem\_real\_array}
	\item Function \mods{getitem\_int}
	\item Function \mods{getitem\_str}
	\item Subroutine \mods{setitem\_real}
	\item Subroutine \mods{setitem\_real\_array}
	\item Subroutine \mods{setitem\_int}
	\item Subroutine \mods{setitem\_str}
	\end{itemize}

\end{itemize}

Each of the routines in the module \mods{Contains} block is essentially
a list of \mods{if}/\mods{elseif} statements.  The list tests for the
name of the variable of interest (a string), and gets or sets the
compiled QTCM1 model variable corresponding to that name.  For pointer
array variables, a test is also made as to whether or not the variable
has been associated.  If not, the variable is not readable
and \vars{is\_readable} is set to \vars{.FALSE.}\ accordingly.

If you wish to add another compiled QTCM1 model variable to be
accessible to \class{Qtcm} instance methods \mods{get\_qtcm1\_item}
and \mods{set\_qtcm1\_item}, just add another \mods{if}/\mods{else\-if},
like the other \mods{if}/\mods{elseif} blocks, in the Fortran set
and get routines corresponding to the QTCM1 variable type (scalar
vs.\ array, and real, integer, or string).  On the Python side, add
an entry in \mods{defaults} corresponding to the new field variable
you've created access to.  I would strongly recommend making the
Python name of your new field variable
(given in \mods{defaults}) be the same as the compiled
QTCM1 model variable name.



	\subsection{Fortran Module \mods{WrapCall}}   \label{sec:wrapcall}

Most of the time, if you want to call a compiled QTCM1 model subroutine
from the Python level, you will use the version of the subroutine that
is found in this Fortran module.  
Note that the Fortran module \mods{WrapCall} is referred
to in lowercase at the Python level, i.e., as the
attribute \vars{\_\_.qtcm.wrapcall} of a \class{Qtcm} instance.

All the routines in this module do is wrap one of the compiled QTCM1
model routines.  For instance, \mods{WrapCall} subroutine
\mods{wadvcttq} is defined as just:

% --- Two versions of this block, one for display in PDF and the other
%     for display in HTML:
%
\begin{latexonly}
\begin{codeblock}
\codeblockfont{%
Subroutine wadvcttq \\
\hspace*{3ex}Call advcttq \\
End Subroutine wadvcttq}
\end{codeblock}
\end{latexonly}

\begin{htmlonly}
\begin{rawhtml}
<p><code><font color="blue">Subroutine wadvcttq<br>
&nbsp;&nbsp;&nbsp;Call advcttq<br>
End Subroutine wadvcttq</font></code></p>
\end{rawhtml}
\end{htmlonly}

All subroutines in this module begin with ``w'', with the rest of
the name being the Fortran QTCM1 subroutine name.  The calling
interface for the ``w'' version is the same as the Fortran QTCM1
original version.  There are no subroutines in this module that do
not have an exact counterpart in the Fortran QTCM1 code, and thus
this module's subroutines sole purpose is to call other subroutines
in the compiled QTCM1 model.

These wrapper routines are needed because \mods{f2py}, for some
reason I can't figure out, will not properly wrap Fortran routines
(that are then callable at the Python level) that create local
arrays using parameters obtained through a Fortran \mods{use}
statment.  Thus, as an example, a Fortran subroutine \mods{foo}
with the following definition:

% --- Two versions of this block, one for display in PDF and the other
%     for display in HTML:
%
\begin{latexonly}
\begin{codeblock}
\codeblockfont{%
subroutine foo \\
\hspace*{3ex}use dimensions \\
\hspace*{3ex}real a(nx,ny) \\
\hspace*{3ex}[\ldots] \\
end subroutine foo}
\end{codeblock}
\end{latexonly}

\begin{htmlonly}
\begin{rawhtml}
<p><code><font color="blue">
subroutine foo<br>
&nbsp;&nbsp;&nbsp;use dimensions<br>
&nbsp;&nbsp;&nbsp;real a(nx,ny)<br>
&nbsp;&nbsp;&nbsp;[\ldots]<br>
end subroutine foo
</font></code></p>
\end{rawhtml}
\end{htmlonly}


where \vars{nx} and \vars{ny} are defined in the module vars{dimensions},
will return an error, with the result that the extension module
will not be created, or an extension modules that yields output
that is different from running the pure-Fortran version of QTCM1.

By wrapping these calls into this file, I also avoid having to
separate out the Fortran QTCM1 subroutines into separate \fn{.F90}
files.  For Fortran subroutines that you want callable from the
Python level, \mods{f2py} seems to require each Fortran subroutine
to be in its own file of the same name (e.g., the version of
\fn{driver.F90} for this package). If several Fortran subroutines
are all found in a single \fn{.F90} files, \mods{f2py} seems unable
to create wrappers for those subroutines.




%---------------------------------------------------------------------
\section{Python \mods{qtcm} and Pure-Fortran QTCM1 Differences}

This section describes differences between how the \mods{qtcm}
package and the pure-Fortran QTCM1 assign some varables.  A list
of changes to the QTCM1 Fortran Files for use in the \mods{qtcm}
package is found in Section~\ref{sec:f90changes}.


	\subsection{QTCM1 \mods{driverinit}}   \label{sec:driverinit.diffs}

In the pure-Fortran version of QTCM1, by default, the following variables are
set by reference (as given below), not by value, in the \mods{driverinit}
routine:\footnote%
	{In the pure-Fortran version of QTCM1, this routine is found
	in \fn{driver.F90}.}
\begin{codeblock}
\codeblockfont{%
lastday\thinspace=\thinspace{daysperyear} \\
viscxu0\thinspace=\thinspace{viscU} \\
viscyu0\thinspace=\thinspace{viscU} \\
visc4x\thinspace=\thinspace{viscU} \\
visc4y\thinspace=\thinspace{viscU} \\
viscxu1\thinspace=\thinspace{viscU} \\
viscyu1\thinspace=\thinspace{viscU} \\
viscxT\thinspace=\thinspace{viscT} \\
viscyT\thinspace=\thinspace{viscT} \\
viscxq\thinspace=\thinspace{viscQ} \\
viscyq\thinspace=\thinspace{viscQ}}
\end{codeblock}

Thus, in pure-Fortran QTCM1, if you change \vars{daysperyear},
\vars{viscU}, etc.
and recompile (as needed), you will automatically change 
\vars{lastday}, \vars{viscxu0}, etc.
(Though, in the pure-Fortran QTCM1, the default values may be overwritten by
namelist input values.)

The \mods{driverinit} routine is eliminated
in the Python \code{qtcm} package.  Instead, inital values 
of field variables are specified in the \mods{defaults} submodule
and set by value to attributes of the \code{Qtcm} instance.
Thus, for instance, in a \class{Qtcm} instance, \code{lastday} 
is set to \code{365} by default, not to some variable
\vars{daysperyear}.  For the diffusion and viscosity terms,
the \class{Qtcm} instance attributes corresponding to those
terms are set to literals.\footnote%
	{Those literals are defined by \mods{defaults} private
	module variables \vars{\_\_viscT}, \vars{\_\_viscQ},
	and \vars{\_\_viscU}.}

In contrast, in the pure-Fortran QTCM1,
\mods{driverinit} declares local
variables \code{viscU}, \code{viscT}, and \code{viscQ},
and reads values into those variables via the input namelist.
Those values are then used to set
\vars{viscxu0}, \vars{viscyu0}, etc., as described above.
In pure-Fortran QTCM1, \code{viscU}, \code{viscT}, and \code{viscQ}
are not directly accessed anywhere else in the model.
Thus, \code{viscU}, \code{viscT}, and \code{viscQ} are not
defined as field variables in the \code{qtcm} package, and
\class{Qtcm} instances do not have attributes corresponding
to those names.
Additionally, if you wish to change a viscosity parameter
\vars{visc*} (given above), the parameter for each direction
must be set one-by-one even if the flow is isotropic.


	\subsection{The \mods{varinit} Routine}

One of the functions of the pure-Fortran QTCM1 \mods{varinit}
subroutine is to associate the pointer variables \vars{u1}, \vars{v1},
\vars{q1}, and \vars{T1}.  For the extension modules in the \mods{qtcm}
package, a Fortran subroutine \mods{varptrinit} is added that can
also do this association.  This subroutine is called in the
\class{Qtcm} instance method
\latexhtml{\mods{varinit}%
		\footnote{http://www.johnny-lin.com/py\_docs/qtcm/doc/html-api/qtcm.qtcm.Qtcm-class.html\#varinit}}%
	{\htmladdnormallink{\mods{varinit}}{http://www.johnny-lin.com/py_docs/qtcm/doc/html-api/qtcm.qtcm.Qtcm-class.html#varinit}}
(which duplicates and
extends the function of its pure-Fortran counterpart, enabling
alternative ways of handling restart).

The \mods{varptrinit} is not accessed via \mods{wrapcall}.  Remember
that \mods{wrapcall} contains only those routines that were in the
original pure-Fortran QTCM1 code, and that we want to have access
to at the Python level.


	\subsection{The \mods{qtcm} Method of \class{Qtcm}}

The \class{Qtcm} method \mods{qtcm} duplicates the functionality
of the \mods{qtcm} subroutine in the pure-Fortran QTCM1 model.
There are a few differences, however.  First, the \mods{qtcm} method
for \class{Qtcm} instances does not include a call to \mods{cplmean},
which uses mean surface flux for air-sea coupling.  This state is
consistent with the pure-Fortran QTCM1 pre-processor macro
\vars{CPLMEAN} being off.  Thus, if you wish to use mean surface
flux for air-sea coupling, you will have to revise the \mods{qtcm}
method of \class{Qtcm} to call \mods{cplmean}.  You'll also have to
check for any other code additions needed that are associated with
the \vars{CPLMEAN} macro.

Second, the \mods{qtcm} method for \class{Qtcm} instances does not
include the option of not using the atmospheric boundary layer
model.  This is consistent with macro \vars{NO\_ABL} being off.  If
you wish to have no atmospheric boundary layer model, change the
run list \vars{atm\_bartr\_mode} so that the \mods{wsavebartr} and
\mods{wgradphis} routines are not called.  You'll also have to check
for any other code additions needed that are associated with the
\vars{NO\_ABL} macro.



	\subsection{Miscellaneous Differences}

\begin{itemize}
\item In Python \class{Qtcm} instances,
	\vars{dateofmodel} is set to 0 by default.  
	In contrast, in the compiled QTCM1 model,
	the default (i.e., initial value) is calculated from 
	\vars{day0}, \vars{month0}, and \vars{year0}.
	See Section~\ref{sec:init.compiledform.full} for details.

\item The \class{Qtcm} instance attribute
	\vars{\_\_qtcm} is not copyable using \mods{copy.deepcopy}.

\item In general, when executing a \class{Qtcm} instance method, 
	if you change a \class{Qtcm} instance attribute 
	that has a counterpart in the compiled QTCM1 model,
	the compiled QTCM1 counterpart is not changed until the
	end of the method.  Likewise, if you call a compiled QTCM1 model
	subroutine and change a compiled QTCM1 model variable with
	a \class{Qtcm} instance counterpart, the \class{Qtcm}
	instance counterpart is not changed until the end of the
	subroutine.

\item In general, even though some of the compiled QTCM1 model
	Fortran subroutines/functions have counterparts in \class{Qtcm}
	that duplicate the former's functionality, the Fortran
	versions are kept intact so that the
	\vars{compiled\_form\thinspace=\thinspace'full'} case will work.
\end{itemize}




%---------------------------------------------------------------------
\section{Considerations When Adding Fortran Code}

In this section I describe issues to consider if you wish to add
your own compiled code to the package as separate extension modules.
(This is different from creating new standard extension modules,
which is described in Section~\ref{sec:create.new.so}.):

\begin{itemize}
\item The \class{Qtcm} class assumes that the directory path 
	to the original shared object file is the same as for the 
	\mods{package\_version} module.

\item If you want to be able to pass other Fortran variables 
	in and out to/from Python, please see the 
	Section~\ref{sec:setbypy}
	discussion of the Fotran \mods{SetByPy} module.

\item Fortran and Python routines to get and set compiled QTCM1 model
	arrays are currently written only for floating point array.

\item If you ever change 
	\class{Qtcm} instance method
	\mods{\_set\_qtcm\_array\_item\_in\_model}
	to work with non-floating point values, you will also
	have to change the array handling section in 
	\mods{set\_qtcm1\_item}.

\item The restart mechanism in the pure-Fortran QTCM1 model is 
	\emph{not} bit-for-bit correct.  Thus, if you compare the final
	output from a 40 day run with a 30 day run restarted from
	a 10 day run, the output will not be the same.
	This behavior has been duplicated in \class{Qtcm} 
	instances when the \vars{mrestart} flag is used
	and applicable.

\item When creating new extension modules using the \fn{src} makefile,
	be sure you first use the \cmd{make clean} command to clean-up
	any old files.

\end{itemize}




%---------------------------------------------------------------------
\section{Creating New Standard Extension Modules}   \label{sec:create.new.so}

The steps involved in creating the standard extension modules (e.g.,
\fn{\_qtcm\_full\_365.so}, etc.) on installation are given in
Section~\ref{sec:create.so}.  The makefile provided in \fn{/buildpath/src}
uses a Fortran compiler to create the object code, runs \mods{f2py}
to create the shared object file in \fn{src}, and moves the shared
object files into \fn{../lib}, overwriting any pre-existing files
of the same name.  In this section, I describe the makefile and
\mods{f2py} in a little more detail, in case you wish to create
standard extension modules with additions from the ones the default
makefile creates.


	\subsection{Makefile Rules}    \label{sec:makefile.rules}

This section describes the rules of the
makefile found in the \fn{src} directory
of the \mods{qtcm} distribution.  
This makefile is used by the Python package to create the extension
module (\fn{.so} files) imported and used by \mods{qtcm} objects
(as described in Section~\ref{sec:create.so}).
The makefile will, in general, be used only during \mods{qtcm}
installation, but if you wish to recompile the QTCM1 libraries
and make changes in the Python extension module,
you'll want to use/change this makefile.

\begin{description}
\item[clean] Removes old files in preparation for compiling new
	extension modules.

\item[libqtcm.a] Creates library \fn{libqtcm.a} that contains all
	QTCM1 object files in the directory \fn{src},, except
	\fn{setbypy.o}, \fn{wrapcall.o}, \fn{varptrinit.o}, and
	\fn{driver.o}.  This archive is compiled with the netCDF
	libraries.  Previous versions of \fn{libqtcm.a} are overwritten.

\item[\_qtcm\_full\_365.so] Creates the extension module
	\fn{\_qtcm\_full\_365.so}.  \mods{f2py} is run on applicable code
	in \fn{src}, and the extension module is moved to \fn{../lib}.
	Any previous versions of \fn{../lib/\_qtcm\_full\_365.so}
	are overwritten.

\item[\_qtcm\_parts\_365.so] Creates the extension module
	\fn{\_qtcm\_parts\_365.so}.  \mods{f2py} is run on applicable code
	in \fn{src}, and the extension module is moved to \fn{../lib}.
	Any previous versions of \fn{../lib/\_qtcm\_parts\_365.so}
	are overwritten.

\end{description}



	\subsection{Using \mods{f2py}}      \label{sec:using.f2py}

This section briefly describes how \mods{f2py} is used in the
makefile during the creation of the extension modules.
\htmladdnormallink{\mods{F2py}}{http://cens.ioc.ee/projects/f2py2e/} is a
program that generates shared object libraries that allow you to call
Fortran routines in Python.  \mods{F2py} comes with Python's
\htmladdnormallink{NumPy}{http://numpy.scipy.org/}
array handling package, so you do not need to install anything
extra if you have NumPy already installed.

To create the extension modules in \mods{qtcm} using
the makefile described in Section~\ref{sec:makefile.rules},
I use a method similar to the
\latexhtml{``Quick and Smart Way,''\footnote%
{http://cens.ioc.ee/projects/f2py2e/usersguide/index.html\#the-quick-and-smart-way}}%
{\htmladdnormallink{``Quick and Smart Way''}%
{http://cens.ioc.ee/projects/f2py2e/usersguide/index.html#the-quick-and-smart-way}}
described in the \mods{f2py} manual.
For the \fn{\_qtcm\_full\_365.so} extension module, the 
\mods{f2py} call is:

\begin{codeblock}
\codeblockfont{%
f2py --fcompiler=\$(FC) -c -m \_qtcm\_full\_365 driver.F90 $\backslash$ \\
\hspace*{10ex}setbypy.F90 libqtcm.a \$(NCLIB)}
\end{codeblock}

and for the \fn{\_qtcm\_parts\_365.so} extension module, the call is:

\begin{codeblock}
\codeblockfont{%
f2py --fcompiler=\$(FC) -c -m \_qtcm\_parts\_365 $\backslash$ \\
\hspace*{10ex}varptrinit.F90 wrapcall.F90 setbypy.F90 $\backslash$ \\
\hspace*{10ex}libqtcm.a \$(NCLIB)}
\end{codeblock}

For both calls, \vars{FC} and \vars{NCLIB} are the environment
variables in the makefile specifying the Fortran compiler and netCDF
libraries, respectively.  The \vars{-m} flag specifies the extension
module name (without the \fn{.so} suffix).  The \fn{.F90} files
specify the files that have modules and routines that will be
accessible at the extension module level, and the rest of the Fortran
files in QTCM1 are compiled and archived in a library \fn{libqtcm.a}.
For \mods{f2py} to work properly,
the \fn{.F90} files may define \emph{only one} module or routine.

If you add Fortran files containing new modules, and you wish those
modules to be accessible at the Python level, compile your new code
with \mods{f2py}.  If we have a file of such new code, \fn{newcode.F90},
the \mods{f2py} call to create the \fn{\_qtcm\_parts\_365.so}
extension module will become:

\begin{codeblock}
\codeblockfont{%
f2py --fcompiler=\$(FC) -c -m \_qtcm\_parts\_365 $\backslash$ \\
\hspace*{10ex}varptrinit.F90 wrapcall.F90 setbypy.F90 $\backslash$ \\
\hspace*{10ex}newcode.F90 $\backslash$ \\
\hspace*{10ex}libqtcm.a \$(NCLIB)}
\end{codeblock}

If you write new Fortran code for the compiled QTCM1 model that
will \emph{not} be accessed from the Python-level, just add the
object code filename to the variable \vars{QTCMOBJS} in the
makefile; you don't have to do anything else.  If you are adding
Fortran code to existing Fortran modules, it's even easier:  You
don't need change the makefile.  Note that for 64 bit processor
machines, you may have to use \mods{f2py} with the \cmd{-fPIC} flag;
see Section~\ref{sec:sopic} for details on how the lines above will
change.


	\subsection{Two Examples}

\emphpara{A Function:}
Let's say you have written a piece of Fortran code called
\fn{myfunction.F90} that contains one function called
\mods{myfunction}, and you want to have this function
callable from the Python level through the \class{Qtcm} 
instance method \mods{\_\_qtcm.myfunction}.  Do the following:

\begin{enumerate}
\item Move \fn{myfunction.F90} to \fn{src} in the \mods{qtcm}
	distribution directory \fn{/buildpath}.

\item Add \cmd{myfunction.o} to the end of the object file list lines
	after the target names
	\vars{\_qtcm\_full\_365.so} and
	\vars{\_qtcm\_parts\_365.so}.

\item In the
	\vars{\_qtcm\_full\_365.so} and
	\vars{\_qtcm\_parts\_365.so} target descriptions,
	add \cmd{myfunction.F90} to the 
	beginning of the list of \fn{.F90} names 
	in the \mods{f2py} lines.
\end{enumerate}


\emphpara{A Module:} 
Let's say you have written a piece of Fortran code called
\fn{mymodule.F90} that contains the Fortran module \mods{MyModule}
containing multiple routines and variables.  You want to have those
routines and variables callable from the Python level through the
\class{Qtcm} instance attribute \mods{\_\_qtcm.mymodule}.  The steps
to add \mods{MyModule} to the extension modules are exactly the
same as for a single function, with \cmd{mymodule} being
substituted in the makefile everywhere you have \cmd{myfunction}.




%---------------------------------------------------------------------
\section{Attributes and Methods in \class{Qtcm} Instances}

In this section I describe some attributes, particularly private ones,
that may be of interest to developers.
As is the convention in Python, private
attributes and methods are prepended by one or two underscores,
with two underscores being the ``more'' private attribute.
Please see the package
\latexhtml{API documentation%
		\footnote{http://www.johnny-lin.com/py\_pkgs/qtcm/doc/html-api/}}
        {\htmladdnormallink{API documentation}%
		{http://www.johnny-lin.com/py\_pkgs/qtcm/doc/html-api/}}
for details about all variables, including private variables.


	\subsection{Public \mods{num\_settings} Submodule Attributes/Methods}

\begin{itemize}
\item \vars{typecode}:  This module function returns the
	type code of the data array passed in as its argument.

\item \vars{typecodes}:  This dictionary is the same as the
	NumPy (or Numeric and \mods{numarray})
	dictionary \vars{typecodes}, except that the character
	\vars{'S'} and \vars{'c'} are added to the
	\vars{typecodes['Character']} entry, if absent.  This
	functionality is added because I found 
	\vars{typecodes['Character']} had different values in
	Mac OS X and Ubuntu GNU/Linux.
\end{itemize}


	\subsection{Private \mods{qtcm} Submodule Attributes}

This submodule of the package \mods{qtcm} is the module that defines
the \class{Qtcm} class.

\begin{itemize}
\item \vars{\_init\_prog\_dict}:  This dictionary contains
	the default values of all prognostic variables and 
	right-hand sides that can be initialized.  In the
	submodule \mods{qtcm}, it is set to
	the \vars{init\_prognostic\_dict} module variable in
	submodule \mods{defaults}.

\item \vars{\_init\_vars\_keys}:  List of all keys in
	\vars{\_init\_prog\_dict}, plus \vars{'dateofmodel'}
	and \vars{'title'}.  These names correspond to the
	field variables that are usually written out into a
	restart file.

\item \vars{\_test\_field}:  \class{Field} object instance used 
	in type tests.
\end{itemize}



	\subsection{Private \class{Qtcm} Attributes}  
					\label{sec:Qtcm.private.attrib}

\begin{itemize}
\item \vars{\_cont}:  A boolean attribute that is \vars{True}
	if the run session is a continuation run session and
	\vars{False} if not.  Set the value passed in by
	the keyword \vars{cont} when the \mods{run\_session}
	method is executed.

\item \vars{\_monlen}:  Integer array of the number of days in 
	each month, assuming a 365~day year.

\item \vars{\_\_qtcm}:  The extension module that is the
	compiled QTCM1 Fortran model for this instance.
	This attribute is unique for every instance:  The
	extension module \fn{.so} file is first copied to
	a temporary directory (given by the \vars{sodir}
	instance attribute) and then imported to the
	\class{Qtcm} instance.
	This private attribute is set on instantiation.

\item \vars{\_qtcm\_fields\_ids}:  Field ids for all default 
	field variables, set on instantiation.

\item \vars{\_runlists\_long\_names}:  Dictionary holding the
	descriptions of the standard run lists.  The keys of
	the dictionary are the names of the standard run lists.
\end{itemize}




%---------------------------------------------------------------------
\section{Creating Documentation}

The distribution of \mods{qtcm} comes with the full set of
documentation in readable form (PDF and HTML).  The documentation
consists of two kinds:  this User's Guide and the API documentation.
The User's Guide is written in \LaTeX.  The PDF version is generated
directly from \LaTeX, and the HTML version is created by
\LaTeX{2}HTML.

I use the \fn{make\_docs} shell script in \fn{doc} creates all these
documents.  Briefly, that script does the following:

\begin{itemize}
\item In the \fn{doc/latex} directory, uses \cmd{python} to
	run \fn{code\_to\_latex.py}, which generates the
	\LaTeX\ files describing the current \mods{qtcm} 
	package settings, including text in the manual which gives
	all uses of the current version number.

\item \LaTeX\ is run on the \LaTeX\ files in the \fn{doc/latex} directory.
	The PDF generated by the run is moved from \fn{doc/latex} to
	\fn{doc}.

\item \LaTeX{2}HTML is run on the \LaTeX\ files in \fn{doc/latex}.
	The HTML files generated by the run are moved to \fn{doc/html}.

\item \mods{epydoc} is run on the \mods{qtcm} package libraries.
	This is run in \fn{doc}, to make use of the \fn{epydoc}
	configuration file present there.  The syntax from the
	command line is:

\begin{codeblock}
\codeblockfont{%
epydoc -v --config epydocrc [name]}
\end{codeblock}
\vars{[name]} is either \cmd{qtcm}, if the \mods{qtcm} package is
installed in a directory listed in \vars{sys.path}, or 
\vars{[name]} is the name of the directory the \mods{qtcm} package is
located in (e.g., \fn{/usr/lib/python2.4/site-packages/qtcm}).

\end{itemize}

The \fn{make\_docs} script cannot be used without customizing it
to your system, so please \emphpara{DO NOT USE IT} if you do
not know what you are doing.  You could easily wipe out all your
documentation by mistake.





% ===== end of file =====


\chapter{Future Work}                       \label{ch:future}
% ==========================================================================
% Future
%
% By Johnny Lin
% ==========================================================================


% ------ BODY -----
%
This section describes the features and fixes I plan to work on
in this package.  The most urgent items are listed closer to the
begining of the lists.

\begin{itemize}
\item Add \code{implicit none} top setbypy.F90.

\item Check through Fortran routines that have arguments, to make sure
	f2py is properly understanding the intentions
	(i.e., in, out, inout) of the variables, since we're using the
	``quick way'' of making shared object libraries using f2py.
	The \fn{utilities.F90} file has a number of Fortran routines
	with arguments.

\item Cite:  Peterson, P. (2009) 
	F2PY: a tool for connecting Fortran and Python programs, 
	\emph{Int. J. Computational Science and Engineering,}
	Vol.\ 4, No.\ 4, pp.\ 296--305 for f2py.

\item Create a method like \mods{calc\_derived('T100')} which would
	primarily operate on a data file and provide a derived variable
	such as the temperature at 100 hPa, as given in this example.
	Figure out where to put the parameters (V1s, etc.) that are
	needed to make such a calculation.  As attributes?  Create a
	method to write the quantity out to an output file?
	Perhaps make an ability to calculate these values at heights
	at a given time each day during a run session?

\item Automate the installation using Python's
\htmladdnormallinkfoot{\mods{distutils}}{http://docs.python.org/dist/dist.html}
	utilities.

\item Describe a way of using job control (either via the operating system
	or IPython's \mods{jobctrl} module) 
	to do a quick-and-dirty parallelization of multiple
	\class{Qtcm} instance run sessions.  Or use some sort of threading
	to fire up two simulataneously running models.  Check that the
	simultaneously running models have different memory space.

\item Add capability for \fn{create\_benchmark.py} to overwrite
	existing benchmark files.

\item Make \vars{compiled\_form} set to \vars{'parts'} as the
	default instantiation.  Change documentation accordingly.

\item Currently, the \class{Qtcm} \mods{plotm} method works only on
	3-D output (time, latitude, longitude).  Some of the fields
	in the netCDF output files are 2-D.  Add the capability to
	\mods{plot\_netcdf\_output} in the \mods{plot} submodule
	to handle 2-D fields.

\item Add documentation about removing temporary files.
	Add documentation in Section~\ref{sec:model.instances}
	of details of what occurs during instantiation of 
	a \class{Qtcm} instance.

\item Add the units and long names for all field variables in the
	\mods{defaults} module.

\item Create a keyword to automatically change precipitation and
	evaporation units to mm/day (or similar).

\item Add ability to calculate and plot fields at different pressure
	levels.  Create another module like defaults that specifies
	the vertical fields and gives the equation to use to calculate
	those fields; call the module ``derivfields'' or something
	similar.

\item Throughout the \mods{qtcm} package I use the condition
	\mods{N.rank(}\dumarg{arg}\mods{)\thinspace=\thinspace0} 
	to test whether
	\dumarg{arg} is a scalar.  This works fine for \mods{numpy}
	objects, but it does not work properly for
	\mods{Numeric} and \mods{numarray} arrays.  In those
	array packages, \mods{rank('abc')} returns the value~1.
	This is not a problem, as long as everyone has \mods{numpy},
	but in order to make the package interoperable, I need to
	find a better way of testing for scalars.  The definitions
	of isscalar need to be changed in \mods{num\_settings}.

\item \mods{num\_settings} needs to be changed to truly enable me
	to test whether \mods{qtcm} works for 
	\mods{numarray} and \mods{Numeric} arrays.  The tests
	do not do this right now, because \mods{num\_settings}
	defaults to \mods{numpy}, if it exists.

\item Create makefiles for other platforms.
 
\item A few fields (e.g., \vars{u1}) have data for extra latitude bands,
	due to the use of ``ghost latitudes'' as part of the
	implementation of the numerics.  Details are found in the 
\latexhtml{%
\htmladdnormallinkfoot{QTCM1 manual}%
        {http://www.atmos.ucla.edu/$\sim$csi/qtcm\_man/v2.3/qtcm\_manv2.3.pdf}}%
{\htmladdnormallink{QTCM1 manual}%
        {http://www.atmos.ucla.edu/~csi/qtcm_man/v2.3/qtcm_manv2.3.pdf}}
\cite{Neelin/etal:2002}.

	Though adjusting to this idiosyncracy is not that difficult, 
	in the future I hope to implement a method of handing
	fields with ghost latitudes so that they have the same
	dimensions as the other gridded output variables.  In order
	to do this, I plan to write a Python method to read the
	Fortran generated binary restart file.

\item Change the \mods{set\_qtcm\_item} method so that it can 
	automatically accomodate setting Fortran real variables
	if integer values are input.

\item Currently, the \mods{get\_item\_qtcm} and 
	\mods{set\_item\_qtcm} methods will not work
	on integer and character arrays, only scalars and real arrays.
	Add that missing functionality to those methods.

\item Currently, the \mods{make\_snapshot} method duplicates the
	functionality of the pure-Fortran QTCM1 restart file mechanism.
	However, the restart file mechanism itself does not do a true
	restart.  A continuous run does not provide the same results
	as two runs over the same period, joined by the restart file.

	To see whether saving more variables would do the trick,
	I altered \mods{make\_snapshot} to store all Python level
	variables (i.e., \vars{self.\_qtcm\_fields\_ids}).  However,
	the restart failing described above still continued.  In the
	future, I hope to figure out exactly how many variables are
	needed in order to make the restart feature do a true
	restart.

\item Add a test of using the \vars{mrestart\thinspace=\thinspace1}
	restart option.  Does the \fn{qtcm.restart} file need to be
	in the current working directory or another?

\item Add a test in the unit test scripts to
	confirm that the \vars{init\_with\_instance\_state}
	attribute setting only has an effect if 
	\vars{compiled\_form\thinspace=\thinspace'parts'}.

\item Document \vars{tmppreview} keyword in \mods{plot.plot\_ncdf\_output}.

\item Confirm and document that
	for netCDF output, time is model time since dd-mm-yyyy.

\item Add to the \mods{plotm} method the ability to
	plot as text onto the figure the
	runname string and the calling line
	for the plotm method.

\item Couple with the
	\latexhtml{CliMT\footnote{http://maths.ucd.ie/$\sim$rca/climt/}}%
	{\htmladdnormallink{CliMT}{http://maths.ucd.ie/~rca/climt/}}
	climate modeling toolkit.

\item Enable Python to set \vars{arr1name}, etc., which are string
	variables at the Python level.  I haven't really thought through
	how \vars{arr1} variables work with the Python \class{Qtcm}
	instance.

\item Possible:  In the \class{Qtcm} method
	\mods{\_\_setattr\_\_}, add a test to raise an exception
	if the instance tries to set \vars{viscU}, \vars{viscT},
	or \vars{viscQ} as attributes.  Also create a method
	\code{isotropic\_visc} that will set all viscosity parameters
	non-dependent on direction.  See Section~\ref{sec:driverinit.diffs}
	for details.

\item Go through the manual and create HTML-only versions of tables
	that have table numbers (use a similar construct as in
	figure environments).

\item Go through documentation to check that
	output variable names are capitalized consistently.

\item Create way to redirect stdout.

\item Create a step method to run an arbitrary number of timesteps at
	the atmosphere level.

\end{itemize}


% ===== end of file =====





% ----- BACK MATTER OF THE DOCUMENT -----
%
\normalsize
\pagebreak
\bibliographystyle{plain}
\bibliography{/Users/jlin/work/res/bib/master}

%- Uncomment the input line below and comment out the \bibliographystyle
%  and \bibliography lines if you're running this without the master.bib 
%  BibTeX database
%% ==========================================================================
% Manual for QTCM Python Package
%
% Usage:
% - If you are running this on your own system, you will not have a copy of
%   my master.bib BibTeX database.  To run this, you'll have to comment out:
%
%      \bibliographystyle{chicago-jl}
%      \bibliography{/Users/jlin/work/res/bib/master}
%
%   and comment back in:
%
%      \input{manual.bbl}
%
%   in this file.  Then you can use pdflatex on this file to get the PDF of
%   the manual.  These 3 lines are in the back matter of the document.
%
% Revision Notes:
% - By Johnny Lin, North Park University, http://www.johnny-lin.com/
% - The chicago BibTeX style is unrecognized by latex2html, so I use
%   the plain style.
% ==========================================================================


% ------ DOCUMENT DEFINITIONS ------
%
\documentclass[12pt]{book}
\usepackage{color}
\usepackage{html}
\usepackage{graphicx}
\usepackage{textcomp}
%\usepackage{comment}    %- Unrecognized by latex2html; its use causes errors
%\usepackage{fancyvrb}   %- Unrecognized by latex2html; its use causes errors


%- Packages unrecognized by latex2html, but causes no error:
%
%\usepackage[letterpaper,margin=1in,includefoot]{geometry}
\usepackage[letterpaper,margin=1.25in]{geometry}
\usepackage{bibnames}
\usepackage{longtable}
\usepackage{multirow}


%+ Comment out explicity margin settings since use package geometry:
%\setlength{\topmargin}{0in}
%\setlength{\headheight}{0in}
%\setlength{\headsep}{0in}
%\setlength{\oddsidemargin}{0in}
%\setlength{\evensidemargin}{0in}
%\setlength{\textheight}{8.5in}
%\setlength{\textwidth}{6.5in}




% ------ COMMANDS AND LENGTHS ------
%
% --- Define colors:  Have to do this because for some reason LaTeX
%     sometimes looks for "BLUE" instead of "blue" and complains when
%     "BLUE" isn't found.
%
\definecolor{Blue}{rgb}{0,0,1}
\definecolor{BLUE}{rgb}{0,0,1}
\definecolor{green}{rgb}{0,0.6,0}
\definecolor{Green}{rgb}{0,0.6,0}
\definecolor{GREEN}{rgb}{0,0.6,0}


% --- Format code blocks.  Currently set to print out the code in just 
%     typewriter font with no box.  Will work the same for pdflatex 
%     and latex2html:
%
%     codeblock:  Environment for blocks of computer code or internet 
%       addresses.
%     codeblockfont:  Sets font for codeblocks.
%
\newenvironment{codeblock}%
	{\begin{quotation}\begin{minipage}[t]{0.9\textwidth}}%
	{\end{minipage}\end{quotation}}
	%{\begin{flushleft}}%
	%{\end{flushleft}}
\newcommand{\codeblockfont}[1]{\textcolor{blue}{\texttt{#1}}}
%     *** Version that only works for pdflatex that puts a box around 
%         the block and centers it (commented out).  Note that using
%         fancyvrb is the better way of creating such a boxed section
%         of code, but fancyvrb isn't recognized by latex2html:
%\newenvironment{codeblock}%
%	{\begin{center}\begin{tabular}{|c|} \hline \\ }%
%	{\\ \\ \hline \end{tabular}\end{center}}
%\newcommand{\codeblockfont}[1]{\parbox{0.8\textwidth}{\texttt{#1}}}


% --- Text titling/emphasis settings:
%
%     emphpara:  Emphasis for the first phrase or sentence of a 
%         paragraph.
%     booktitle:  Formats book titles.
%     tabletitle:  Title for an item block in the information table.
%     paratitle:  Title for a paragraph in an item block in the
%         information table.
%     emphdate:  Emphasize date in paragraph text.
%
%     cmd:  Commands
%     dumarg:  Dummy arguments
%     codearg:  Same as dumarg.
%     fn:  File and directory names
%     screen:  Screen display
%     vars:  Variable and attribute names
%     mods:  Module, subroutine, and method names
%     class:  Class names
%     code:  Generic code (avoid using this)
%
\newcommand{\emphpara}[1]{\textbf{#1}}
\newcommand{\booktitle}[1]{\textit{#1}}
%\newcommand{\tabletitle}[1]{\textsf{\textbf{#1}}}
\newcommand{\paratitle}[1]{\textit{#1}}
\newcommand{\emphdate}[1]{\textbf{#1}}

\newcommand{\code}[1]{\textcolor{blue}{\texttt{#1}}}
\newcommand{\cmd}[1]{\textcolor{blue}{\texttt{#1}}}
\newcommand{\dumarg}[1]{\textit{#1}}
\newcommand{\codearg}[1]{\textit{#1}}
\newcommand{\fn}[1]{\textsf{\textit{#1}}}
\newcommand{\screen}[1]{\textcolor{green}{\texttt{#1}}}
\newcommand{\vars}[1]{\textcolor{blue}{\texttt{#1}}}
\newcommand{\class}[1]{\textcolor{blue}{\texttt{#1}}}
\newcommand{\mods}[1]{\textcolor{blue}{\texttt{#1}}}


% --- Special table formatting:
%
%     tabletitlewidth:  Width for title field of an item block in the 
%         information table.
%     tablebodywidth:  Width for body field of an item block in the 
%         information table.
%     tabletabulardims:  Dimensions for the information table, used in
%         the tabular command.
%     tableitemlinespace:  Vertical spacing between item blocks in the
%         information table.
%     infotitle and infotext:  Used for two-column sub-information 
%         tables found in the body field of the information table.  
%         These are not global lengths but have values specific to the 
%         local context in which they're used.
%
\newlength{\tabletitlewidth}
\settowidth{\tabletitlewidth}{file and directory names}

\newlength{\tablebodywidth}
\setlength{\tablebodywidth}{0.9\textwidth}
\addtolength{\tablebodywidth}{-4ex}
\addtolength{\tablebodywidth}{-\tabletitlewidth}

\newcommand{\tabletabulardims}%
	{p{\tabletitlewidth}@{\hspace{4ex}}p{\tablebodywidth}}

\newcommand{\tableitemlinespace}{\baselineskip}
\newlength{\infotitle}
\newlength{\infotext}


% --- Lengths for formatting:
%
\newlength{\remainder}        % length to describe the residual of the
                              %   linewidth minus \enumlabel
\newlength{\enumlabel}        % length to describe figure sub-label width
                              %   (e.g. "(a)")


% --- TtH stuff:
%
%\def\tthdump#1{#1}


% --- LaTeX2HTML stuff:
%
%     htmlfigcaption:  Formatting for HTML replacement figure captions.
%
\newcommand{\htmlfigcaption}[1]{\parbox[c]{70ex}{\footnotesize{#1}}}


% --- Some book title abbreviations:
%
%     rute:  Booktitle for Rute User's.
%     linuxnut:  Booktitle for Linux in a Nutshell.
%     pynut:  Booktitle for Python in a Nutshell.
%
\newcommand{\rute}{\booktitle{Rute User's}}
\newcommand{\linuxnut}{\booktitle{Linux in a Nutshell}}
\newcommand{\pynut}{\booktitle{Python in a Nutshell}}


% --- Define special characters ---
%
\newcommand{\aonehat}{\ensuremath{\widehat{a_1}}}
\newcommand{\bonehat}{\ensuremath{\widehat{b_1}}}
\newcommand{\D}{\ensuremath{\mathcal{D}}}
\def\BibTeX{B\kern-.03em i\kern-.03em b\kern-.15em\TeX}




% ------ BEGINNING OF DOCUMENT TEXT ------
%
\begin{document}

    

    
% ------ TITLE AND TOC ------
%
\title{\mods{qtcm} User's Guide}
\author{Johnny Wei-Bing Lin\thanks{Physics Department, North Park University,
	3225 W.\ Foster Ave., Chicago, IL  60625, USA}}
\date{\today}
\maketitle
\tableofcontents




% ------ BODY ------
%
\chapter{Introduction}
\input{intro}

\chapter{Installation and Configuration}    \label{ch:install}
	\section{Summary and Conventions}      \label{sec:install.sum}
	\input{install_sum}
	\section{Fortran Compiler}             \label{sec:fort.compilers}
	\input{install_fort}
	\section{Required Packages}            \label{sec:py.etc.pkgs}
	\input{install_pkgs}
	\section{Compiling Extension Modules}  \label{sec:create.so}
	\input{compile_so}
	\section{Testing the Installation}     \label{sec:test.qtcm}
	\input{test_qtcm}
	\section{Model Performance}
	\input{perform}
	\section{Installing in Mac OS X}       \label{sec:install.macosx}
	\input{qtcm_in_macosx}
	\section{Installing in Ubuntu}         \label{sec:install.ubuntu}
	\input{qtcm_in_ubuntu}

\chapter{Getting Started With \mods{qtcm}}  \label{ch:getting.started}
\input{started}

\chapter{Using \mods{qtcm}}                 \label{ch:using}
\input{using}

%@@@\chapter{Combining \code{qtcm} with \code{CliMT}}
%@@@\input{climt}

\chapter{Troubleshooting}                   \label{ch:trouble}
\input{trouble}

\chapter{Developer Notes}                   \label{ch:devnotes}
\input{devnotes}

\chapter{Future Work}                       \label{ch:future}
\input{future}




% ----- BACK MATTER OF THE DOCUMENT -----
%
\normalsize
\pagebreak
\bibliographystyle{plain}
\bibliography{/Users/jlin/work/res/bib/master}

%- Uncomment the input line below and comment out the \bibliographystyle
%  and \bibliography lines if you're running this without the master.bib 
%  BibTeX database
%\input{manual.bbl}        

\appendix
\chapter{Field Settings in \mods{defaults}}  \label{app:defaults.values}
\input{defaults}




% ------ END OF DOCUMENT TEXT ------
%
\end{document}


% ===== end of file =====
        

\appendix
\chapter{Field Settings in \mods{defaults}}  \label{app:defaults.values}
% ==========================================================================
% Appendix:  Defaults from the submodule defaults
%
% By Johnny Lin
% ==========================================================================


% ------ BODY -----
%
%---------------------------------------------------------------------------
\section{Scalar Field Variables}  \label{sec:defaults.scalar}

This table lists the default settings for scalar \mods{qtcm} fields
as set by the \mods{defaults} submodule.  All fields are of class
\class{Field}.  Numerical values are rounded as per the conventions
of Python's \vars{\%g} format code.
To create a \class{Field} instance whose value is set to the
default, instantiate with the field id as the argument

\input{defaults_scalars}




%---------------------------------------------------------------------------
\section{Array Field Variables}   \label{sec:defaults.array}

This table lists the default settings for array \mods{qtcm} fields
as set by the \mods{defaults} submodule.  All fields are of class
\class{Field}.  Numerical values are rounded as per the conventions
of Python's \vars{\%g} format code.

\input{defaults_arrays}




% ===== end of file =====





% ------ END OF DOCUMENT TEXT ------
%
\end{document}


% ===== end of file =====
        

\appendix
\chapter{Field Settings in \mods{defaults}}  \label{app:defaults.values}
% ==========================================================================
% Appendix:  Defaults from the submodule defaults
%
% By Johnny Lin
% ==========================================================================


% ------ BODY -----
%
%---------------------------------------------------------------------------
\section{Scalar Field Variables}  \label{sec:defaults.scalar}

This table lists the default settings for scalar \mods{qtcm} fields
as set by the \mods{defaults} submodule.  All fields are of class
\class{Field}.  Numerical values are rounded as per the conventions
of Python's \vars{\%g} format code.
To create a \class{Field} instance whose value is set to the
default, instantiate with the field id as the argument

% This file is automatically generated by the script
% code_to_latex.py in the doc/latex directory.  It is based upon
% the values found in the qtcm.defaults submodule, and should
% not be hand-edited if you want the values to correspond to
% the values in the qtcm.defaults submodule.
        

\begin{longtable}{l|c|c|p{0.30\linewidth}}
\textbf{Field} & \textbf{Value} & \textbf{Units} & 
                                \textbf{Description} \\
\hline
\endhead
\vars{SSTdir} & ../bnddata/SST\_Reynolds &  & Where SST files are \\
\vars{SSTmode} & seasonal &  & Decide what kind of SST to use \\
\vars{VVsmin} & 4.5 & m/s & Minimum wind speed for fluxes \\
\vars{bnddir} & ../bnddata &  & Boundary data other than SST \\
\vars{dateofmodel} & 0 &  & Date of model coded as an integer as yyyymmdd \\
\vars{day0} & -1 & dy & Starting day; if $<$ 0 use day in restart \\
\vars{dt} & 1200 & s & Time step \\
\vars{eps\_c} & 0.000138889 & 1/s & 1/tau\_c NZ (5.7) \\
\vars{interval} & 1 & dy & Atmosphere-ocean coupling interval \\
\vars{it} & 1 &  & Time of day in time steps \\
\vars{landon} & 1 &  & If not 1: land = ocean with fake SST \\
\vars{lastday} & 365 & dy & Last day of integration \\
\vars{month0} & -1 & mo & Starting month; if $<$ 0 use mo in restart \\
\vars{mrestart} & 0 &  & =1: restart using qtcm.restart \\
\vars{mt0} & 1 &  & Barotropic timestep every mt0 timesteps \\
\vars{nastep} & 1 &  & Number of atmosphere time steps within one air-sea coupling interval \\
\vars{noout} & 0 & dy & No output for the first noout days \\
\vars{nooutr} & 0 & dy & No restart file for the first nooutr days \\
\vars{ntout} & -30 & dy & Monthly mean output \\
\vars{ntouti} & 0 & dy & Monthly instantaneous data output \\
\vars{ntoutr} & 0 & dy & Restart file only at end of model run \\
\vars{outdir} & ../proc/qtcm\_output &  & Where output goes to \\
\vars{runname} & runname &  & String for an output filename \\
\vars{title} & QTCM default title &  & A descriptive title \\
\vars{u0bar} & 0 &  &  \\
\vars{visc4x} & 700000 & m$^2$/s & Del 4 viscocity parameter in x \\
\vars{visc4y} & 700000 & m$^2$/s & Del 4 viscocity parameter in y \\
\vars{viscxT} & 1.2e+06 & m$^2$/s & Temperature diffusion parameter in x \\
\vars{viscxq} & 1.2e+06 & m$^2$/s & Humidity diffusion parameter in x \\
\vars{viscxu0} & 700000 & m$^2$/s & Viscocity parameter for u0 in x \\
\vars{viscxu1} & 700000 & m$^2$/s & Viscocity parameter for u1 in x \\
\vars{viscyT} & 1.2e+06 & m$^2$/s & Temperature diffusion parameter in y \\
\vars{viscyq} & 1.2e+06 & m$^2$/s & Humidity diffusion parameter in y \\
\vars{viscyu0} & 700000 & m$^2$/s & Viscocity parameter for u0 in y \\
\vars{viscyu1} & 700000 & m$^2$/s & Viscocity parameter for u1 in y \\
\vars{weml} & 0.01 & m/s & Mixed layer entrainment velocity \\
\vars{year0} & 0 & yr & Starting year; if $<$ 0 use year in restart \\
\vars{ziml} & 500 & m & Atmosphere mixed layer depth $\sim$ cloud base \\
\end{longtable}





%---------------------------------------------------------------------------
\section{Array Field Variables}   \label{sec:defaults.array}

This table lists the default settings for array \mods{qtcm} fields
as set by the \mods{defaults} submodule.  All fields are of class
\class{Field}.  Numerical values are rounded as per the conventions
of Python's \vars{\%g} format code.

% This file is automatically generated by the script
% code_to_latex.py in the doc/latex directory.  It is based upon
% the values found in the qtcm.defaults submodule, and should
% not be hand-edited if you want the values to correspond to
% the values in the qtcm.defaults submodule.
        

\begin{longtable}{l|c|c|c|c|p{0.37\linewidth}}
\textbf{Field} & \textbf{Shape} & \textbf{Max} & \textbf{Min} &
                                \textbf{Units} & \textbf{Description} \\
\hline
\endhead
\vars{Evap} & (1, 1) & 0 & 0 &  &  \\
\vars{FLW} & (1, 1) & 0 & 0 &  &  \\
\vars{FLWds} & (1, 1) & 0 & 0 &  &  \\
\vars{FLWus} & (1, 1) & 0 & 0 &  &  \\
\vars{FLWut} & (1, 1) & 0 & 0 &  &  \\
\vars{FSW} & (1, 1) & 0 & 0 &  &  \\
\vars{FSWds} & (1, 1) & 0 & 0 &  &  \\
\vars{FSWus} & (1, 1) & 0 & 0 &  &  \\
\vars{FSWut} & (1, 1) & 0 & 0 &  &  \\
\vars{FTs} & (1, 1) & 0 & 0 &  &  \\
\vars{Qc} & (1, 1) & 0 & 0 & K & Precipitation \\
\vars{S0} & (1, 1) & 0 & 0 &  &  \\
\vars{STYPE} & (1, 1) & 0 & 0 &  & Surface type; ocean or vegetation type over land \\
\vars{T1} & (1, 1) & 0 & 0 & K &  \\
\vars{Ts} & (1, 1) & 0 & 0 & K & Surface temperature \\
\vars{WD} & (1, 1) & 0 & 0 &  &  \\
\vars{WD0} & (1,) & 0 & 0 &  & Field capacity SIB2/CSU (approximately) \\
\vars{arr1} & (1, 1) & 0 & 0 &  & Auxiliary optional output array 1 \\
\vars{arr2} & (1, 1) & 0 & 0 &  & Auxiliary optional output array 2 \\
\vars{arr3} & (1, 1) & 0 & 0 &  & Auxiliary optional output array 3 \\
\vars{arr4} & (1, 1) & 0 & 0 &  & Auxiliary optional output array 4 \\
\vars{arr5} & (1, 1) & 0 & 0 &  & Auxiliary optional output array 5 \\
\vars{arr6} & (1, 1) & 0 & 0 &  & Auxiliary optional output array 6 \\
\vars{arr7} & (1, 1) & 0 & 0 &  & Auxiliary optional output array 7 \\
\vars{arr8} & (1, 1) & 0 & 0 &  & Auxiliary optional output array 8 \\
\vars{psi0} & (1, 1) & 0 & 0 &  &  \\
\vars{q1} & (1, 1) & 0 & 0 & K &  \\
\vars{rhsu0bar} & (1,) & 0 & 0 &  &  \\
\vars{rhsvort0} & (1, 1, 1) & 0 & 0 &  &  \\
\vars{taux} & (1, 1) & 0 & 0 &  &  \\
\vars{tauy} & (1, 1) & 0 & 0 &  &  \\
\vars{u0} & (1, 1) & 0 & 0 & m/s & Barotropic zonal wind \\
\vars{u1} & (1, 1) & 0 & 0 & m/s & Current time step baroclinic zonal wind \\
\vars{v0} & (1, 1) & 0 & 0 & m/s & Barotropic meridional wind \\
\vars{v1} & (1, 1) & 0 & 0 & m/s &  \\
\vars{vort0} & (1, 1) & 0 & 0 &  &  \\
\end{longtable}





% ===== end of file =====





% ------ END OF DOCUMENT TEXT ------
%
\end{document}


% ===== end of file =====
        

\appendix
\chapter{Field Settings in \mods{defaults}}  \label{app:defaults.values}
% ==========================================================================
% Appendix:  Defaults from the submodule defaults
%
% By Johnny Lin
% ==========================================================================


% ------ BODY -----
%
%---------------------------------------------------------------------------
\section{Scalar Field Variables}  \label{sec:defaults.scalar}

This table lists the default settings for scalar \mods{qtcm} fields
as set by the \mods{defaults} submodule.  All fields are of class
\class{Field}.  Numerical values are rounded as per the conventions
of Python's \vars{\%g} format code.
To create a \class{Field} instance whose value is set to the
default, instantiate with the field id as the argument

% This file is automatically generated by the script
% code_to_latex.py in the doc/latex directory.  It is based upon
% the values found in the qtcm.defaults submodule, and should
% not be hand-edited if you want the values to correspond to
% the values in the qtcm.defaults submodule.
        

\begin{longtable}{l|c|c|p{0.30\linewidth}}
\textbf{Field} & \textbf{Value} & \textbf{Units} & 
                                \textbf{Description} \\
\hline
\endhead
\vars{SSTdir} & ../bnddata/SST\_Reynolds &  & Where SST files are \\
\vars{SSTmode} & seasonal &  & Decide what kind of SST to use \\
\vars{VVsmin} & 4.5 & m/s & Minimum wind speed for fluxes \\
\vars{bnddir} & ../bnddata &  & Boundary data other than SST \\
\vars{dateofmodel} & 0 &  & Date of model coded as an integer as yyyymmdd \\
\vars{day0} & -1 & dy & Starting day; if $<$ 0 use day in restart \\
\vars{dt} & 1200 & s & Time step \\
\vars{eps\_c} & 0.000138889 & 1/s & 1/tau\_c NZ (5.7) \\
\vars{interval} & 1 & dy & Atmosphere-ocean coupling interval \\
\vars{it} & 1 &  & Time of day in time steps \\
\vars{landon} & 1 &  & If not 1: land = ocean with fake SST \\
\vars{lastday} & 365 & dy & Last day of integration \\
\vars{month0} & -1 & mo & Starting month; if $<$ 0 use mo in restart \\
\vars{mrestart} & 0 &  & =1: restart using qtcm.restart \\
\vars{mt0} & 1 &  & Barotropic timestep every mt0 timesteps \\
\vars{nastep} & 1 &  & Number of atmosphere time steps within one air-sea coupling interval \\
\vars{noout} & 0 & dy & No output for the first noout days \\
\vars{nooutr} & 0 & dy & No restart file for the first nooutr days \\
\vars{ntout} & -30 & dy & Monthly mean output \\
\vars{ntouti} & 0 & dy & Monthly instantaneous data output \\
\vars{ntoutr} & 0 & dy & Restart file only at end of model run \\
\vars{outdir} & ../proc/qtcm\_output &  & Where output goes to \\
\vars{runname} & runname &  & String for an output filename \\
\vars{title} & QTCM default title &  & A descriptive title \\
\vars{u0bar} & 0 &  &  \\
\vars{visc4x} & 700000 & m$^2$/s & Del 4 viscocity parameter in x \\
\vars{visc4y} & 700000 & m$^2$/s & Del 4 viscocity parameter in y \\
\vars{viscxT} & 1.2e+06 & m$^2$/s & Temperature diffusion parameter in x \\
\vars{viscxq} & 1.2e+06 & m$^2$/s & Humidity diffusion parameter in x \\
\vars{viscxu0} & 700000 & m$^2$/s & Viscocity parameter for u0 in x \\
\vars{viscxu1} & 700000 & m$^2$/s & Viscocity parameter for u1 in x \\
\vars{viscyT} & 1.2e+06 & m$^2$/s & Temperature diffusion parameter in y \\
\vars{viscyq} & 1.2e+06 & m$^2$/s & Humidity diffusion parameter in y \\
\vars{viscyu0} & 700000 & m$^2$/s & Viscocity parameter for u0 in y \\
\vars{viscyu1} & 700000 & m$^2$/s & Viscocity parameter for u1 in y \\
\vars{weml} & 0.01 & m/s & Mixed layer entrainment velocity \\
\vars{year0} & 0 & yr & Starting year; if $<$ 0 use year in restart \\
\vars{ziml} & 500 & m & Atmosphere mixed layer depth $\sim$ cloud base \\
\end{longtable}





%---------------------------------------------------------------------------
\section{Array Field Variables}   \label{sec:defaults.array}

This table lists the default settings for array \mods{qtcm} fields
as set by the \mods{defaults} submodule.  All fields are of class
\class{Field}.  Numerical values are rounded as per the conventions
of Python's \vars{\%g} format code.

% This file is automatically generated by the script
% code_to_latex.py in the doc/latex directory.  It is based upon
% the values found in the qtcm.defaults submodule, and should
% not be hand-edited if you want the values to correspond to
% the values in the qtcm.defaults submodule.
        

\begin{longtable}{l|c|c|c|c|p{0.37\linewidth}}
\textbf{Field} & \textbf{Shape} & \textbf{Max} & \textbf{Min} &
                                \textbf{Units} & \textbf{Description} \\
\hline
\endhead
\vars{Evap} & (1, 1) & 0 & 0 &  &  \\
\vars{FLW} & (1, 1) & 0 & 0 &  &  \\
\vars{FLWds} & (1, 1) & 0 & 0 &  &  \\
\vars{FLWus} & (1, 1) & 0 & 0 &  &  \\
\vars{FLWut} & (1, 1) & 0 & 0 &  &  \\
\vars{FSW} & (1, 1) & 0 & 0 &  &  \\
\vars{FSWds} & (1, 1) & 0 & 0 &  &  \\
\vars{FSWus} & (1, 1) & 0 & 0 &  &  \\
\vars{FSWut} & (1, 1) & 0 & 0 &  &  \\
\vars{FTs} & (1, 1) & 0 & 0 &  &  \\
\vars{Qc} & (1, 1) & 0 & 0 & K & Precipitation \\
\vars{S0} & (1, 1) & 0 & 0 &  &  \\
\vars{STYPE} & (1, 1) & 0 & 0 &  & Surface type; ocean or vegetation type over land \\
\vars{T1} & (1, 1) & 0 & 0 & K &  \\
\vars{Ts} & (1, 1) & 0 & 0 & K & Surface temperature \\
\vars{WD} & (1, 1) & 0 & 0 &  &  \\
\vars{WD0} & (1,) & 0 & 0 &  & Field capacity SIB2/CSU (approximately) \\
\vars{arr1} & (1, 1) & 0 & 0 &  & Auxiliary optional output array 1 \\
\vars{arr2} & (1, 1) & 0 & 0 &  & Auxiliary optional output array 2 \\
\vars{arr3} & (1, 1) & 0 & 0 &  & Auxiliary optional output array 3 \\
\vars{arr4} & (1, 1) & 0 & 0 &  & Auxiliary optional output array 4 \\
\vars{arr5} & (1, 1) & 0 & 0 &  & Auxiliary optional output array 5 \\
\vars{arr6} & (1, 1) & 0 & 0 &  & Auxiliary optional output array 6 \\
\vars{arr7} & (1, 1) & 0 & 0 &  & Auxiliary optional output array 7 \\
\vars{arr8} & (1, 1) & 0 & 0 &  & Auxiliary optional output array 8 \\
\vars{psi0} & (1, 1) & 0 & 0 &  &  \\
\vars{q1} & (1, 1) & 0 & 0 & K &  \\
\vars{rhsu0bar} & (1,) & 0 & 0 &  &  \\
\vars{rhsvort0} & (1, 1, 1) & 0 & 0 &  &  \\
\vars{taux} & (1, 1) & 0 & 0 &  &  \\
\vars{tauy} & (1, 1) & 0 & 0 &  &  \\
\vars{u0} & (1, 1) & 0 & 0 & m/s & Barotropic zonal wind \\
\vars{u1} & (1, 1) & 0 & 0 & m/s & Current time step baroclinic zonal wind \\
\vars{v0} & (1, 1) & 0 & 0 & m/s & Barotropic meridional wind \\
\vars{v1} & (1, 1) & 0 & 0 & m/s &  \\
\vars{vort0} & (1, 1) & 0 & 0 &  &  \\
\end{longtable}





% ===== end of file =====





% ------ END OF DOCUMENT TEXT ------
%
\end{document}


% ===== end of file =====
